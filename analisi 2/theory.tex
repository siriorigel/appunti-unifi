\documentclass[a4paper, oneside]{article}
\usepackage{graphicx}
\usepackage{amsthm}
\usepackage{amsmath}
\usepackage{amssymb}
\usepackage[a4paper,
            bindingoffset=0.2in,
            left=2cm,
            right=2cm,
            top=2cm,
            bottom=2cm,
            footskip=.25in]{geometry}
\usepackage[italian]{babel}
\usepackage{pgfplots}
\usepackage{tabularx}
\usepackage{tikz-3dplot}
\usepackage{wrapfig}
\usepackage{color}
\usepackage{multicol}
\usepackage{arydshln}
\usepackage{mathtools}
\usepackage{enumerate}
\usepackage{graphicx}
\usepackage{svg}
\usepackage{cancel}
\usepackage[d]{esvect}
\usepackage[dvipsnames]{xcolor}
\usepackage{pgfplots}
\usepackage{pifont}
\usepackage{imakeidx}
%\usepackage{animate}
%\usepackage{xfp} % utile se vuoi fare calcoli aggiuntivi
\pgfplotsset{compat=1.18}
\usetikzlibrary{tikzmark}
\newcommand{\TikzNCbar}[4][10pt]{
\tikz[overlay,remember picture]{\draw[#2] (#3) --++(0,-#1) -| (#4);}}

\graphicspath{ {images/} }

\definecolor{redish}{rgb}{255, 0, 30}
\definecolor{page}{rgb}{0.129,0.157,0.212}
\pagecolor{page}
\color{white}   
\graphicspath{ {./images/} }
\usetikzlibrary{shapes.geometric}
\usetikzlibrary{datavisualization}
\usetikzlibrary{datavisualization.formats.functions}
\pgfplotsset{width=10cm,compat=1.9}

\setlength\dashlinedash{0.2pt}
\setlength\dashlinegap{1.5pt}
\setlength\arrayrulewidth{0.3pt}

\newcommand\eqq{\stackrel{\mathclap{\normalfont\mbox{?}}}{=}}
\newcommand\bulletout  {\labelitemfont \textbullet}
\newcommand{\tab}{\hspace*{2em}}
\newcommand{\xmark}{
\tikz[scale=0.23] {
    \draw[line width=0.7,line cap=round] (0,0) to [bend left=6] (1,1);
    \draw[line width=0.7,line cap=round] (0.2,0.95) to [bend right=3] (0.8,0.05);
}}
\newcommand{\cmark}{
\tikz[scale=0.23] {
    \draw[line width=0.7,line cap=round] (0.25,0) to [bend left=10] (1,1);
    \draw[line width=0.8,line cap=round] (0,0.35) to [bend right=1] (0.23,0);
}}
 \newcommand{\hookbox}[1]{
\begin{center}
\hfill\break
\begin{tikzpicture}
\node[inner sep=0pt,outer sep=0pt,anchor=base] (A) {
\begin{minipage}{\dimexpr\linewidth-5em}
\centering
#1
\end{minipage}
};
% Draw the left bracket
\draw ([xshift=0pt]A.north west) -- ++(0, 0.5) -- ++(0.4, 0);
% Draw the right bracket
\draw ([xshift=0pt]A.south east) -- ++(0, -0.5) -- ++(-0.4, 0);
\end{tikzpicture}
\end{center}} 
\title{Teoria analisi II}
\author{Gariboldi Alessandro}
\date{ }
\makeindex



\begin{document}

\newtheoremstyle{theoremEnv}
                {}          % Space above
                {}          % Space below
                {\slshape}  % Body font
                {}          % Indent amount
                {\bfseries} % Head font
                {.}         % Punctuation after head
                {\newline}         % Space after theorem head
                {}          % Theorem head spec
\theoremstyle{theoremEnv}

\newtheorem{definition}{Definizione}[section]
\newtheorem{theorem}{Teorema}[section]
\newtheorem{lemma}{Lemma}[section]
\newtheorem{observation}{Oss.}[section]
\newtheorem{corollary}{Corollario}[theorem]
\newtheorem{example}{Esempio}[section]
\newtheorem{problem}{Problema}[section]
\newtheorem{solution}{Soluzione}[section]
\newtheorem{proposition}{Proposizione}[section]


\maketitle

\tableofcontents
\newpage
%EDO
\section{Equazioni differenziali ordinarie}
\begin{definition}[Forme normali di EDO]
    Esempi di forma normale:
    \begin{align*}
        y' = f(x, y)
        \\y'' = f(x, y, y')
    \end{align*}
    Forma normale canonica, in generale si definisce una eq. in forma normale se posso isolare la derivata n-esima e quindi ricondurmi alla forma seguente:
    \begin{align*}
        y^{(n)} = f(x, y, y', ...\ , y^{(n-1)})
    \end{align*}
\end{definition}
\begin{center}
    \fbox{L'ordine di un'eq. differenziale è il grado della derivata del grado più alto che compare nell'eq.}
\end{center}
\hfil\\
%pb di cauchy
\begin{proposition}[Prob. di Cauchy per eq. di 1° grado]
\textbf{EDO 1° ordine}
\begin{align*}
    \begin{cases}
        y' = f(x,y) \qquad x_0,y_0\text{\emph{ sono assegnati}}\\
        y(x_0) = y_0
    \end{cases}
\end{align*}
\\\textbf{EDO 2° ordine}
\begin{align*}
    \begin{cases}
        y'' = f(x,y,y') \qquad x_0,y_0,y_1\text{\emph{ sono assegnati}}\\
        y(x_0) = y_0\\
        y'(x_0) = y_1
    \end{cases}
\end{align*}
\end{proposition}
\hfill\\
%teorema di peano
\begin{theorem}[th. di Peano]
    \underbar{Se} $f(x,y)$ è definita e continua in un insieme $A$\\
    \underbar{e} $(x_0, y_0)$ è un punto interno ad $A$\\
    \underbar{allora} il Pb. di Caucy \fbox{*} ammette \textbf{almeno} una soluzione definita in un intorno di $x_0$
    \\\\La continuità di $f(x, y)$ è necessaria per l'$\exists$ di una soluzione
    \\La sola hp. di continuità \textbf{NON} garantisce che esista \textbf{solo} una soluzione
\end{theorem}
\hfill\\
%funz liptschitziane
\subsection{liptschitzianità}
\begin{definition}[Funzioni Lipschitziane]
    $g: \mathbb{I} \to \mathbb{R}$
    \\Sia $[a,b] \subset \mathbb{I}$
    \\Si dice che la funzione è Lipschitziana in $[a,b]$ 
    \\\underline{se} $\exists L\in\mathbb{R} >0$ t.c. $\forall z_1 z_2 \in [a,b]$ \fbox{$ |\frac{g(z_1) - g(z_2)}{z_1 -z_2}| \leq L$} $\quad \Longleftrightarrow  \quad \ $\fbox{$-L \leq \frac{g(z_1) - g(z_2)}{z_1 - z_2} \leq L$}
\end{definition}
%esempi grafici funz. lipschitziane
\begin{multicols}{2}
    \noindent
\begin{tikzpicture}
    \begin{axis}[
    axis lines=middle, 
    xlabel={$z$},      
    ylabel={$g(z)$},      
    xmin=-1, xmax=3,   
    ymin=-2, ymax=5,  
    xtick=\empty,
    ytick=\empty,
    width = 8cm,
    height = 7cm
]
\addplot[
    domain=-3:3,      
    samples=100,       
    smooth,           
    blue              
] {x^(1/2)}; 

\node at (250,430) {$g(z) = z^\alpha \quad \alpha\in(0,1)$};
\node at (120,200) {[};
\node at (120,170) {a};
\node at (220,200) {]};
\node at (220,170) {b};
\end{axis}

\end{tikzpicture}
\columnbreak
\tab
\begin{tikzpicture}
        \draw[->](-1, 0) -- (4, 0) node[at end, below] {$z$};
        \draw[->](0, -1) -- (0, 3) node[at end, left] {$g(z)$};
        \draw(0, 0) .. controls (1, 2) and (1.5, 1) .. (2, 1);
        \draw(2, 1) .. controls (2.5, 1.5) and (3.5, 1.2) .. (4, 1.25) node[at end, right] {$g(z)$};
        \draw(0.7, 1) -- (2, 1);
        \filldraw(0.7, 1) circle (1pt);
        \filldraw(2, 1) circle (1pt);
        \draw[|-|](0.3, 0.2) -- (0.3, -0.2) node[at end, below] {$a$};
        \draw[|-|](2.3, 0.2) -- (2.3, -0.2) node[at end, below] {$b$};
        \draw[dashed](0.7, 0) -- (0.7, 1) node[at start, below] {$z_1$};
        \draw[dashed](2, 0) -- (2, 1) node[at start, below] {$z_2$};
        \draw[<->](0.2, -0.7) -- (2.8, -0.7) node[midway, below] {$I$}; 
    \end{tikzpicture}
\end{multicols}
\hfill\\
Se in un intervallo due numeri $z_1, z_2$ calcolati in una funzione $f$ arbitrariamente vicini formano una come corda tra di loro con un coefficente angolare finito.
\begin{proof}[(servono due lemmi)]
    \hfil\\
    \underbar{\textbf{1° passo :}}
    \begin{lemma}
    Supponiamo valide le ipotesi del th. precedente
    \\\underbar{Sia} $\delta > 0$ le seguenti affermazioni sono equivalenti.
    \end{lemma}
    %derivate parziali
    \begin{definition}[derivate parziali]
    \underbar{Sia} $f(x,y) , (x_0, y_0)$ interno al dominio di $F$
    \begin{align*}
        \frac{\partial f}{\partial y}(x_0, y_0) = \lim_{x_0\to0}\frac{f(x_0, y_0 + h)}{f(x_0, y_0)}
    \end{align*}
    \end{definition}
    *punto chiave*
    \begin{lemma}    
    \underbar{Se} $f$ è continua in un insieme $A$ , $(x_0, y_0 )$ è interno ad $A$ e inoltre
    \begin{enumerate}[i)]    
    \item$\exists \frac{\partial f}{\partial y} (x,y)$
    \item$\frac{\partial f}{\partial y} (x,y)$ è continua $\forall (x,y) \in A$ Allora le hp del th. precedente sono verificate
    \end{enumerate}
    \end{lemma}
\end{proof}
%eq omogenee
\hfil\\
\begin{definition}
    Equazioni Omogenee
    \\$y' = y(x,y)$ dove $f(x,y)$ è "omogenea di grado 0"
    \\\emph{cioè} $\forall \lambda \in \mathbb{R} \forall(x,y) f(\lambda x, \lambda y) = f(x,y)$
    \\Prendiamo l'eq:
\begin{align*}
    y' = \frac{2xy}{x^2+y^2} \qquad f(x,y) \frac{2xy}{x^2+y^2}
\end{align*}
Un equazione di questo tipo può essere riscritta come:
\begin{align*}
    y' = b(y/x)
\end{align*}
\end{definition}
%th di ex globale
\begin{theorem}[di $\exists$ globale]
    Consideriamo il Pb. \fbox{*}
    $\begin{cases}
        y' = f(x,y)\\
        y(x_0) = y_0\\
    \end{cases}$
    \\\underbar{se}:
    \begin{enumerate}[i)]
        \item $f(x,y)$ e $\frac{\partial f}{\partial y}(x,y)$ sono definite e continue $\forall x \in [a,b], y \in \mathbb{R}$ dove $x_0 \in (a,b)$
        \item Esistono due numeri positivi, $h , k$ per cui risulti \fbox{$|f(x,y)| \le h + k|y|$}
    \end{enumerate}
\end{theorem}
\subsection{EDO di ordine n}
%EDO ordine n
\begin{proposition}
    EDO di ordine $n$\\
    \begin{gather*}
        \begin{cases}
          y^{(n)} = f(x, y, y', y'', ... \ ,y^{(n-1)})\\
          y(x_0) = y_0\\
          y'(x_0) = y_1\\
          \vdots\\
          y^{(n-1)}(x_0) = y_n\\
        \end{cases}
    \end{gather*}
\end{proposition}
\subsection{spazi di funzioni}
%spazi di funzionni
\begin{definition}[Spazi di funzioni]
    $I$ è un intervallo in $\mathbb{R}$\\
    si considerano \{\underbar{insieme delle funzioni} $: I \to \mathbb{R}$\}, chiamiamo questo sistema \fbox{*}\\
    prendiamo $f,g \in $ \fbox{*} e \fbox{$\alpha f(x) + \beta g(x)$} $\in $ \fbox{*} \tab con $f(x) \equiv 0$\\\\
    e si considera \{\underbar{insieme delle funzioni} $: I \to \mathbb{R}$\} come spazio vettoriale\\
    \begin{enumerate}[ ]
        \item $C^0(I) = \{ \text{insieme delle funzioni \underbar{continue}} : \to \mathbb{R}\}$
        \item $C^1(I) = \{ \text{insieme delle funzioni \underbar{continue}, derivabili in ogni $x \in I$ e la cui derivata è una funzione continua $\forall x \in I$}\}$
        \item .
        \item .
        \item .
        \item $C^{n}(I)$
    \end{enumerate}
\end{definition}

    \begin{theorem}[Soluzioni di EDO lineari]
        \hfil\\
        \begin{enumerate}[i)]
            \item L'insieme $V_0$ delle soluzioni di \fbox{2} è uno spazio vettoriale di dimensione n
            \item L'insieme delle soluzioni di \fbox{1} è
            \hookbox{\begin{gather*}
                \begin{cases}
                    y(x) +y_f(x) : y \in V_0 \text{ e } y_f \text{ è soluzione di \fbox{1}}
                \end{cases}
            \end{gather*}}
        \end{enumerate}
    \end{theorem}
    \begin{gather*}
        \\\text{\fbox{$y^{n} a_{n-1}(x) y^{(n-1)} + \dots + a_1(x) y'+ a_0(x)y = f(x)$}} \ \ \text{\fbox{1}}\\
        a_0, a_1, \dots , a_{n-1}\text{supponiamo che siano definite e continue in } I \subset \mathbb{R}\\
        y \in C^{n}(I)\\
        E(y):C^{n}(I)\to C^0(I)\\
        E(y) = y^{(n)}(x) + a_{(n-1)}(x)y^{(n-1)}(x) + \dots \ + a_0(x)y()\\
        E( \alpha y_1 + \beta y_2) = \alpha E(y_1) + \beta E(y_2)\\
        \text{\fbox{$y^{(n)}+a_{(n-1)}(x)y^{(n-1)} + ... \ + a_0(x)y(x) = 0$}} \ \ \text{\fbox{2}}\\
    \end{gather*}
\begin{proposition}[Funzioni $C^1$ e $C^2$]
    \begin{gather*} 
    C^0(I) = \{\text{ \underbar{insieme di funzioni continue}:} I \to \mathbb{R}\}\\
    C^1(I) = \{\text{ \underbar{insieme di funzioni continue e derivabili con derivata continua in}}: I \to \mathbb{R}\}\\
\end{gather*}
\end{proposition}
%th di ex globale
\begin{theorem}[$\exists !$ globale per soluzione di EDO lineari]
        \underbar{Siano} $f(x), a_i(x) \quad i = 0, ... \ , n-1$ funzioni continue in $I$\\
        \underbar{Sia} $x_0$ interno ad $I$ \\
        \underbar{Allora } $\forall$ scelta dei numeri \\
        $b_0, b_1, ... \ , b_{n-1}$ il prob. di Cauchy: \\
    \begin{gather*}
        \begin{cases}
            E(y) = f\\
            y(x_0) = b_0\\
            y'(x_0) = b_1\\
            \vdots\\
            y^{(n-1)}(x_0) = b_{n-1}\\
        \end{cases}
        \text{Ha una sola soluzione definita in tetto } I
    \end{gather*}
    (no dim)\\
\end{theorem}
%l'insieme delle soluzioni è uno spazio vettoriale di dim n
\begin{theorem}
    L'insieme delle soluzioni dell'EDO omogenea $E(y)= 0$ è uno spazio vettoriale di dimensione n.\\
    \fbox{*} è spazio vettoriale. \\
    La funzione $y(x) \eqq 0$ soddisfa l'eq $E(y) = 0$ ovviamente.\\
    \underbar{Se} $y_1$ e $y_2$ sono due soluzioni dell'omogenea, cioè\\
    $E(y_1) = 0$ e $E(y_2) = 0$\\
    e $\alpha , \beta \in \mathbb{R}$ \underbar{allora}\\
    $E(\alpha y_1 + \beta y_2) = \alpha E(y_1) + \beta E(y_2) = \alpha 0 + \beta 0 = 0$\\ 
\end{theorem}
%dim
\begin{proof}
    \hfil\\
    Dimostro il teorema quando $n=2$\\
    Def. le due funzioni $z_0(x)$ e $z_1(x)$ nel modo:\\
    \begin{multicols}{2}
        \noindent
    \begin{gather*}
       z_0 \text{è sol. del pb.}\\
       \begin{cases}
        E(y) = 0\\
        y(x_0) = 1\\
        y'(x_0) = 0\\
       \end{cases}
    \end{gather*}
    \begin{gather*}
       z_1 \text{è sol. del pb.}\\
       \begin{cases}
        E(y) = 0\\
        y(x_0) = 0\\
        y'(x_0) = 1\\
       \end{cases}
    \end{gather*}
        \columnbreak
    \end{multicols}
    
    Richiamo cosa vuol dire nel contesto degli spazi di funzioni cosa buol dire linearmente indipendenti\\
    $z_0$ e $z_1$ sono lin. indipendenti \underbar{se}\\
    $C_0 z_0(x) + c_1 z(x) = 0 \forall x \in I$\\
    dove $c_0, c_1$ sono costanti, avviene solo quando $c_0 = c_1 = 0.$\\
    \\\underbar{Siano} $c_0, c_1 \in \mathbb{R}$ : $c_0 z_0(x0 + c_1 z_1(x) \eqq 0 \forall x \in \mathbb{R})$\\
    chiamo: $z(x) $ è sol. di $z(x_0) = c_0 z_0(x_0) + c_1 z_1(x_0) = c_0 \quad \text{dove } c_0 z_0(x_0) = 1 \ \text{e } c_1 z_1(x_0) = 0$\\
    $z'(x_0) = c_0 z_0'(x_0) + c_1 z_1'(x_0) \quad \ \text{dove } c_0 z_0'(x_0)=0 \ \text{e } c_1 z_1'(x_0)= 1$\\
    \underbar{Ma} per l'ipotesi $z \eqq 0$ quindi $z(x_0) = 0$ \underbar{Ma anche} $z' \equiv 0 $ e $z'(x_0) = 0$\\
    Rimane da dimostrare che $z_0 e z_1$ generano tutto lo spazio delle soluzioni dell'omogenea.\\
    \underbar{Sia} $w(x)$ una sol. arbitraria di $E(y) = 0$.\\
    Voglio dumostrare che è probabile scegliere $c_0, c_1 \in \mathbb{R}$\\
    \underbar{t.c.} $w(x) \equiv c_0 z_0(x) + c_1 z_1(x)$\\
    Scelgo $c_0$ e $c_1$:\\

    \fbox{
    $c_0 = w(x_0)$\\
    $c_1 = w'(x_0)$\\
    }
    Sia $w$ che $c_0 z_0 + c_1 z_1$ sono soluzioni di:\\
    $
    \begin{cases}
        E(y) = 0\\
        y(x_0) = c_0\\
        y'(x_0) = c_1\\
    \end{cases}
    $
    \begin{gather*}
        c_0 z_0 (x_0) + c_1 z_1 (x_0) = c_0 \qquad \text{ dove } \ c_0 z_0 (x_0)= 1 \text{e} \ c_1 z_1 (x_0) = 0\\
        c_0 z_0' (x_0) + c_1 z_1' (x_0) = c_0 \qquad \\
    \end{gather*}
\end{proof}
%integrale generale dell'equazione completa
\begin{theorem}
    L'integrale generale dell'equazione \fbox{$E(y) = f$}\\
    si ottiene sommando l'integrale generale dell'equazione omogenea ad una soluzione particolare dell'equazione completa\\
    $E(y) = f$.
\end{theorem}
%dim
\begin{proof}
    \hfil\\
    \underbar{Sia} $y_1$ una soluzione di $E(y) = f$\\
    \underbar{e} $y_0$ una soluzione di $E(y) = 0$\\
    \underbar{Allora} $y_1 + y_0$ è soluzione di:\\
    $E(y) = f$ infatti\\
    $E(y_1 + y_0) = E(y_1) + E(y_0) = f + 0 = f$\\
    Viceversa \underbar{sia} $y_2$ una qualsiasi soluzione di $E(y) = f$\\
    \underbar{Allora}:\\
    $E(y_2) y_1 = E(y_2) - E(y_1) = f - f = 0$\\
    Quindi $y_2 - y_1 = z_0$ per una certa soluzione dell'omogenea.\\
    \underbar{Allora}: $y_2 = y_1 + z_0$\\
\end{proof}
%th per sol. eq. omogenea
\begin{theorem}
    \underbar{Sia} $\lambda \in \mathbb{R}$ ( o anche $\lambda \in \mathbb{C}$).\\
    La funzione: $y(x) = e^{\lambda x}$\\
    è soluzione dell'equazione omogenea $E(y) = 0$ $\Leftrightarrow$ $\lambda$ è radice del polinomio caratteristico\\
    cioè se $P(\lambda) = 0$\\
\end{theorem}
%dim
\begin{proof}
    \hfil\\
    Chiamo
    \begin{enumerate}[ ]
        \item $y(x) = e^{\lambda x}$
        \item $y'(x) = \lambda e^{\lambda x}$
        \item $y''(x) = \lambda^2 e^{\lambda x}$
        \item .
        \item .
        \item .
        \item $y^n(x) = \lambda^n e^{\lambda x}$
    \end{enumerate}
    \begin{gather*}
        E(y) = \lambda^n e^{\lambda x} + a_{n-1} \lambda^{n-1} e^{\lambda x} + ... + a_1\lambda e^{\lambda x } + a_0 e^{\lambda x}\\
        = (\lambda^n + a_{n-1}\lambda^{n-1}+ ... \ + a_0)e^{\lambda x}\\
        P(\lambda) e^{\lambda x} = 0 \ \Leftrightarrow P(\lambda) = 0\\
    \end{gather*}
    \begin{observation}
        \underbar{se} $\lambda_1,\lambda_2, ... , \lambda_k $ sono dumeri distinti \\ 
        \underbar{allora} $e^{\lambda x},e^{\lambda_2 x}, ... \ , e^{\lambda_k x},$ sono linearmente indipendenti.\\
    \end{observation}
\end{proof}
\end{document}
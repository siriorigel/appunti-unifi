\documentclass[a4paper, oneside]{article}
\usepackage{wrapfig}
\usepackage{graphicx}
\usepackage{amsthm}
\usepackage{amsmath}
\usepackage{amssymb}
\usepackage[a4paper,
            bindingoffset=0.2in,
            left=2cm,
            right=2cm,
            top=2cm,
            bottom=2cm,
            footskip=.25in]{geometry}
\usepackage[italian]{babel}
\usepackage{pgfplots}
\usepackage{tabularx}
\usepackage{tikz-3dplot}
\usepackage{wrapfig}
\usepackage{color}
\usepackage{multicol}
\usepackage{arydshln}
\usepackage{mathtools}
\usepackage{enumerate}
\usepackage{graphicx}
\usepackage{svg}
\usepackage{cancel}
\usepackage[d]{esvect}
\usepackage[dvipsnames]{xcolor}
\usepackage{pgfplots}
\usepackage{pifont}
\usetikzlibrary{patterns}
\makeindex
%\usepackage{animate}
%\usepackage{xfp} % utile se vuoi fare calcoli aggiuntivi
\pgfplotsset{compat=1.18}
\usetikzlibrary{tikzmark}
\newcommand{\TikzNCbar}[4][10pt]{
\tikz[overlay,remember picture]{\draw[#2] (#3) --++(0,-#1) -| (#4);}}

\graphicspath{ {images/} }

\definecolor{redish}{rgb}{255, 0, 30}
\definecolor{page}{rgb}{0.129,0.157,0.212}
\pagecolor{page}
\color{white}   
\graphicspath{ {./images/} }
\usetikzlibrary{shapes.geometric}
\usetikzlibrary{datavisualization}
\usetikzlibrary{datavisualization.formats.functions}
\pgfplotsset{width=10cm,compat=1.9}

\setlength\dashlinedash{0.2pt}
\setlength\dashlinegap{1.5pt}
\setlength\arrayrulewidth{0.3pt}

\newcommand\eqq{\stackrel{\mathclap{\normalfont\mbox{?}}}{=}}
\newcommand\bulletout  {\labelitemfont \textbullet}
\newcommand{\tab}{\hspace*{2em}}
\newcommand{\xmark}{
\tikz[scale=0.23] {
    \draw[line width=0.7,line cap=round] (0,0) to [bend left=6] (1,1);
    \draw[line width=0.7,line cap=round] (0.2,0.95) to [bend right=3] (0.8,0.05);
}}
\newcommand{\cmark}{
\tikz[scale=0.23] {
    \draw[line width=0.7,line cap=round] (0.25,0) to [bend left=10] (1,1);
    \draw[line width=0.8,line cap=round] (0,0.35) to [bend right=1] (0.23,0);
}}
% Comando:
%   \potato[opzioni]{(x,y)}{scala}
%
% Opzioni = facoltative (es. fill=red!20, draw=black, thick)
% (x,y)   = centro della patata
% scala   = fattore di scala
%
\def\potatoshape{
  (1,0) (2,1.5) (1.6,3) (0.3,2.7) (-0.4,1.2)
}
\newcommand{\potato}[3][draw=white]{
  \begin{scope}[shift={#2}, scale=#3]
    \draw[#1]
      plot [smooth cycle, tension=1]
      coordinates {\potatoshape};
  \end{scope}
}
 \newcommand{\hookbox}[1]{
\begin{center}
\hfill\break
\begin{tikzpicture}
\node[inner sep=0pt,outer sep=0pt,anchor=base] (A) {
\begin{minipage}{\dimexpr\linewidth-5em}
\centering
#1
\end{minipage}
};
% Draw the left bracket
\draw ([xshift=0pt]A.north west) -- ++(0, 0.5) -- ++(0.4, 0);
% Draw the right bracket
\draw ([xshift=0pt]A.south east) -- ++(0, -0.5) -- ++(-0.4, 0);
\end{tikzpicture}
\end{center}} 
\title{Analisi II}
\author{Gariboldi Alessandro}
\date{ }


\begin{document}

\newtheoremstyle{theoremEnv}
                {}          % Space above
                {}          % Space below
                {\slshape}  % Body font
                {}          % Indent amount
                {\bfseries} % Head font
                {.}         % Punctuation after head
                {\newline}         % Space after theorem head
                {}          % Theorem head spec
\theoremstyle{theoremEnv}

\newtheorem{definition}{Definizione}[section]
\newtheorem{theorem}{Teorema}[section]
\newtheorem{lemma}{Lemma}[section]
\newtheorem{observation}{Oss.}[section]
\newtheorem{corollary}{Corollario}[theorem]
\newtheorem{example}{Esempio}[section]
\newtheorem{problem}{Problema}[section]
\newtheorem{solution}{Soluzione}[section]
\newtheorem{proposition}{Proposizione}[section]


\maketitle
\date{16/09/25}
\section{Equazioni differenziali \underline{ordinarie}}

\begin{multicols}{2}
    \noindent
Esempi di equazioni differenziali:
\\\\$y' = 3e^y$
\\$y' = 4x$
\\$y' + y = 0$
\columnbreak\\
\hookbox{
    incognita = funzione y(x)
    \\ \emph{una sola variabile reale}
}

    Es. $y(x) = sin(x)$ è soluzione di $y'' + y = 0$

\end{multicols}

\fbox{L'ordine di un'eq. differenziale è il grado della derivata del grado più alto che compare nell'eq.}
\\\begin{align*}
    ------------------------
\end{align*}

$y' = f(x,y)$ \emph{la chiamo} \fbox{\textbf{*}}
\\\\
Una funzione di \fbox{\textbf{*}} è una funzione $y(x)$ definita per $x$
appartenente ad un qualche insieme A, e t.c. $\forall x \in A$
risulti $(x, y(x))$ appartenente al dominio $f(x,y)$, e Inoltre:
\begin{align}
    y'(x) = f(x,y(x))
\end{align}
\begin{example}

\begin{multicols}{2}
    \noindent
   \\\tab\tab\tab $y' = g(x)$
\columnbreak
\begin{tabbing}
\\\\$g$ è una funzione continua definita in un intervallo $I \subset \mathbb{R}$
\end{tabbing}
\end{multicols}

\end{example}


\begin{tabbing}
\\\\\underbar{Se} $G(x)$ indica una primitiva di $g(x)$
\\\tab\underbar{allora} $G(x) + c$ è soluzione dell'EDO \emph{eq. diff. ordinaria}
\end{tabbing}

\begin{align}
    \begin{cases}
        y' = x^2 \qquad y(x) = \frac{x^3}{3} + c\\
        y(1) = 2 \qquad y(1) = 2 \to \frac{1^3}{3} + c = 2 \qquad c = \frac{5}{2}
    \end{cases}
\end{align}


\begin{multicols}{2}
    \noindent
\begin{tikzpicture}
    \begin{axis}[
    axis lines=middle, % Puts axes in the middle of the plot
    xlabel={$x$},      % Label for x-axis
    ylabel={$y$},      % Label for y-axis
    xmin=-1, xmax=3,   % X-axis limits
    ymin=-2, ymax=5,  % Y-axis limits
    width = 8cm,
    height = 8cm
]
\addplot[
    domain=-3:3,      % Domain for the function
    samples=100,      % Number of points to plot for smoothness
    smooth,           % Makes the line smooth
    blue              % Color of the line
] {x^3/3}; % The function to plot
\addplot[
    domain=-3:3,      
    samples=100,  
    smooth,        
    red            
] {(x^3/3) + 5/3};

\filldraw [green] (axis cs:1,2) circle (2pt);
\node at (axis cs:1.2,4) {$\frac{x^3}{3} + 5/3$};
\node at (axis cs:2,2) {$\frac{x^3}{3} + c$};
 
\end{axis}

\end{tikzpicture}
\columnbreak
\begin{tabbing}
\\\\\\\\\emph{con la $c$ sposto il grafico in verticale}
\\\\\tab\emph{in questo caso la c della funzione generica in blu ha c=0}
    
\end{tabbing}
\end{multicols}
\begin{align*}
    ------------------------
\end{align*}
\begin{tabbing}
Posso riconsiderare l'eq. di prima $y' + y = 0$
\\e scrivere una soluzione generica
    
\end{tabbing}
\begin{align*}
    c_1 sin(x) + c_2 cos(x)
\end{align*}
\begin{multicols}{3}
\noindent
\begin{align*}
    \begin{cases}
        y' + y = 0\\
        y(0) = 0\\
        y'(x) = 1
    \end{cases}
\end{align*}
\columnbreak
\begin{align*}
    \begin{cases}
        c_1 sin(0) + c_2 cos(0) = 0\\
        c_1 cos(0) - c_2 sin(0) = 1
    \end{cases}
\end{align*}
\columnbreak
\begin{align*}
    \begin{cases}
        0 + c_2 = 0\\
        c_1 - 0 = 1
    \end{cases}
\end{align*}
\end{multicols}
E di conseguenza trovo i valori per i quali le costanti mi verificano il sistema
\begin{proposition}
Prob. di Cauchy per eq. di 1° grado
\\\\\textbf{EDO 1° ordine}
\begin{align*}
    \begin{cases}
        y' = f(x,y) \qquad x_0,y_0\text{\emph{ sono assegnati}}\\
        y(x_0) = y_0
    \end{cases}
\end{align*}
\\\textbf{EDO 2° ordine}
\begin{align*}
    \begin{cases}
        y'' = f(x,y,y') \qquad x_0,y_0,y_1\text{\emph{ sono assegnati}}\\
        y(x_0) = y_0\\
        y'(x_0) = y_1
    \end{cases}
\end{align*}
    
\end{proposition}
\begin{definition}
    Esempi di forma normale:
    \begin{align}
        y' = f(x, y)
        \\y'' = f(x, y, y')
    \end{align}
    Forma normale canonica, in generale si definisce una eq. in forma normale se posso isolare la derivata n-esima e quindi ricondurmi alla forma seguente:
    \begin{align*}
        y^{(n)} = f(x, y, y', ...\ , y^{(n-1)})
    \end{align*}

    Si ricorda che forme normalizzate non conducono sempre ad una sola soluzione
    \begin{example}
        \begin{align}
            F(x, y, y') \qquad
            F(x, y, y', y'')
            \\\text{Esempio numerico}
            \\(y')^2 + x^3 = 0
            \\y'^2 = -x^3
            \\y' = \pm \sqrt{-x^3} \to \text{due soluzioni}
        \end{align}
    \end{example}
\end{definition}

\begin{proposition}
    Problema di Cauchy per equazioni di 1° ordine in forma normale
    \begin{align}
        \begin{cases}
            y' = f(x.y)       \\  
            y(x_0) = y_0
        \end{cases}
        \text{e lo chiamo \fbox{*}} 
    \end{align}
\end{proposition}

\begin{theorem}
    th. di Peano
    \\\underbar{Se} $f(x,y)$ è definita e continua in un insieme $A$
    \\\underbar{e} $(x_0, y_0)$ è un punto interno ad $A$
    \\\underbar{allora} il Pb. di Caucy \fbox{*} ammette \textbf{almeno} una soluzione definita in un intorno di $x_0$
    \\\\La continuità di $f(x, y)$ è necessaria per l'$\exists$ di una soluzione
    \\La sola hp. di continuità \textbf{NON} garantisce che esista \textbf{solo} una soluzione
\end{theorem}

\begin{example}
    $
        \begin{cases}
            y' = y^{2/3}\\
            y(0) = 0
        \end{cases}
        \\f(x,y) = y^{2/3} \text{definita continua in tutto} \mathbb{R} \times \mathbb{R}
        \\\text{\underbar{sia} } y(x) \equiv 0
        \\\text{\underbar{che}} y(x) = \frac{x^3}{27} \quad y' = \frac{x^2}{9} (y(x))^{2/3}=(\frac{x^3}{27})^{2/3} = \frac{x^2}{9}
    $
    quindi abbiamo trovato che per una funzione continua potremmo trovare due soluzioni distinte, in questo caso potremmo addirittura generalizzare ad infinite soluzioni\dots\\
    \begin{align*}
    \begin{cases}
        0 \ \text{se} x \in (-\infty, a)\\
        \frac{(x-a)^3}{27} \text{se} x \in [a, +\infty)\\
    \end{cases}
\text{\fbox{$a > 0$}}    
    \end{align*}
    
    

\end{example}

\section{Funzioni Lipschitziane}
$g: \mathbb{I} \to \mathbb{R}$
\\Sia $[a,b] \subset \mathbb{I}$
\\Si dice che la funzione è Lipschitziana in $[a,b]$ 
\\\underline{se} $\exists L\in\mathbb{R} >0$ t.c. $\forall z_1 z_2 \in [a,b]$ \fbox{$ |\frac{g(z_1) - g(z_2)}{z_1 -z_2}| \leq L$} $\quad \Longleftrightarrow  \quad \ $\fbox{$-L \leq \frac{g(z_1) - g(z_2)}{z_1 - z_2} \leq L$}


\begin{multicols}{2}
    \noindent
\begin{tikzpicture}
    \begin{axis}[
    axis lines=middle, % Puts axes in the middle of the plot
    xlabel={$z$},      % Label for x-axis
    ylabel={$g(z)$},      % Label for y-axis
    xmin=-1, xmax=3,   % X-axis limits
    ymin=-2, ymax=5,  % Y-axis limits
    xtick=\empty,
    ytick=\empty,
    width = 8cm,
    height = 7cm
]
\addplot[
    domain=-3:3,      % Domain for the function
    samples=100,      % Number of points to plot for smoothness
    smooth,           % Makes the line smooth
    blue              % Color of the line
] {x^(1/2)}; % The function to plot (y = x^2)

\node at (250,430) {$g(z) = z^\alpha \quad \alpha\in(0,1)$};
\node at (120,200) {[};
\node at (120,170) {a};
\node at (220,200) {]};
\node at (220,170) {b};
\end{axis}

\end{tikzpicture}
\columnbreak
\tab
\begin{tikzpicture}
        \draw[->](-1, 0) -- (4, 0) node[at end, below] {$z$};
        \draw[->](0, -1) -- (0, 3) node[at end, left] {$g(z)$};
        \draw(0, 0) .. controls (1, 2) and (1.5, 1) .. (2, 1);
        \draw(2, 1) .. controls (2.5, 1.5) and (3.5, 1.2) .. (4, 1.25) node[at end, right] {$g(z)$};
        \draw(0.7, 1) -- (2, 1);
        \filldraw(0.7, 1) circle (1pt);
        \filldraw(2, 1) circle (1pt);
        \draw[|-|](0.3, 0.2) -- (0.3, -0.2) node[at end, below] {$a$};
        \draw[|-|](2.3, 0.2) -- (2.3, -0.2) node[at end, below] {$b$};
        \draw[dashed](0.7, 0) -- (0.7, 1) node[at start, below] {$z_1$};
        \draw[dashed](2, 0) -- (2, 1) node[at start, below] {$z_2$};
        \draw[<->](0.2, -0.7) -- (2.8, -0.7) node[midway, below] {$I$}; 
    \end{tikzpicture}
\end{multicols}
Se in un intervallo due numeri $z_1, z_2$ calcolati in una funzione $f$ arbitrariamente vicini formano una come corda tra di loro con un coefficente angolare finito.
\begin{theorem}
    Th. di $\exists!$  di Cauchy
    \\Consideriamo il pb \fbox{*}
    $
    \begin{cases}
        y' = f(x,y)\\
        y(x_0) = y_0
    \end{cases}
    $
\begin{tabbing}
    Supponiamo che $f(x,y)$ sia definita $\forall (x,y) \in I \times J$ dove $I = (x_0 - a, x_0 + a)$ e $J = (y_0 - b, y_0 + b)$
    \\Supponiamo anche che:
\end{tabbing}
    
\begin{enumerate}[i)]    
    \item $f(x,y)$ è continua in $I\times J$
    \item $\exists$ costante $L>0$ t.c. $|f(x, y_1)-f(x, y_2)|\leq L|y_1-y_2|$
    \end{enumerate}
    \underbar{Allora} $\exists \delta > 0$ e $\exists!$ funzione $y(x)$ definita in $(x_0 - \delta, x_0+ \delta)$ che risolve il pb. di Cauchy \fbox{\textbf{*}}
\end{theorem}

\begin{multicols}{2}
    \noindent
    
\begin{tikzpicture}
        \draw[->](-1, 0) -- (4, 0) node[at end, below] {$x$};
        \draw[->](0, -1) -- (0, 3) node[at end, left] {$y$};
        \draw[|-|](0.3, 0.2) -- (0.3, -0.2) node[at end, below] {};
        \draw[|-|](2.3, 0.2) -- (2.3, -0.2) node[at end, below] {};
        \draw[|-|](0.2, 0.6) -- (-0.2, 0.6) node[at end, below] {};
        \draw[|-|](0.2, 1.5) -- (-0.2, 1.5) node[at end, below] {};
        \draw[<->](0.2, -0.7) -- (2.3, -0.7) node[midway, below] {$I$}; 
        \draw[<->](-0.4, 0.6) -- (-0.4, 1.5) node[midway, left] {$J$}; 
        \draw(0.3, 0.6) -- (0.3, 1.5);
        \draw(0.3, 0.6) -- (2.3, 0.6);
        \draw(0.3, 1.5) -- (2.3, 1.5);
        \draw(2.3, 0.6) -- (2.3, 1.5);
        \filldraw [red] (1.3,1.05) circle (2pt);
        \node at(2,1.15){$(x_0,y_0)$};

    \end{tikzpicture}
    \columnbreak
    \begin{gather*}
        \exists a,b \ t.c.\\
        I=(x_0-a, x_0+a)\\
        J=(y_0-b, x_0+b)\\
    \end{gather*}

\end{multicols}

\begin{example}
    
$   \begin{cases}
        y' = y^{2/3}\\
        y(0) = 0
    \end{cases}
    \\x_0 = 0, y_0 = 0
    \\f(x,y) = y^{2/3}
$
\end{example}

\begin{proof}
    \hfil\\
    \underbar{\textbf{1° passo :}}
    \begin{lemma}
        Supponiamo valide le ipotesi del th. precedente
        \\\underbar{Sia} $\delta > 0$ le seguenti affermazioni sono equivalenti.

    \end{lemma}
\end{proof}

\begin{lemma}
    Inciso: derivata parziale 
    \\\\\underbar{Sia} $f(x,y) , (x_0, y_0)$ interno al dominio di $F$

    \begin{align*}
            \frac{\partial f}{\partial y}(x_0, y_0) = \lim_{x_0\to0}\frac{f(x_0, y_0 + h)}{f(x_0, y_0)}
    \end{align*}
\end{lemma}
    \begin{example}
    \underbar{Se} $f(x,y) = x y^3 + \sin{y}$
    \\$\frac{\partial f}{\partial y}(3,y_0) =>$derivata rispetto alla y di $3y^3 + \sin{y} = gy_0^2 + \cos{y_0}$    
    \end{example}
    
    \begin{lemma}    
    \underbar{Se} $f$ è continua in un insieme $A$ , $(x_0, y_0 )$ è interno ad $A$ e inoltre
    \begin{enumerate}[i)]    
    \item$\exists \frac{\partial f}{\partial y} (x,y)$
    \item$\frac{\partial f}{\partial y} (x,y)$ è continua $\forall (x,y) \in A$ Allora le hp del th. precedente sono verificate
    \end{enumerate}
    \end{lemma}

    \begin{example}
    se $f(x,y) = y^{\frac{2}{3}}$ e $(x_0, y_0) = (0,0)$
    \\$\frac{\partial f}{\partial y} = 2/3 y^{-1/3}$
    
    \end{example}
    
\begin{problem}
        $y' = f(x) ==> y= F(x) + c$ dove $F$ è primitiva di $f(x)$
    \begin{multicols}{2}
        \noindent
    Eq. a variabili separabili
    \\$y' = a(x) b(y)$
    \columnbreak
    \\\\ $a(x)$ continua in $\mathbb{I} \subset \mathbb{R}$
    \\ $b(y)$ continua in $\mathbb{J} \subset \mathbb{R}$
    \end{multicols}
\end{problem}

\begin{solution}
     
    \underbar{Se} $\overline{y}$ è continua t.c. $b(\overline{y}) = 0$
    \\\underbar{allora} la funzione $y(x) \equiv \overline{y}$ è soluzione
    \\\\supponiamo $b(y) \neq 0$

    $\frac{y'(x)}{b(y(x))}= a(x) \to \int\frac{y'(x)}{b(y(x))}dx = \int{a(x) dx}$

    \hfil\\\underbar{Se} pongo $y= y(x)$
    \\e poi derivo da entrambe le parti: $y'(x)dx = dy$
    \\Successivamente calcolo l'integrale da entrmbi i lati per togliere i differenziali $\int \frac{dy}{b(y)}  = \int a(x) dx$
    \\\\\underbar{Se} $B(y)$ è una primitiva di $\frac{1}{b(y)}$
    \\e $A(x)$ è una primitiva di $a(x)$
    \\si ottiene:
    \\\fbox{$B(y) = A(x) +c$}
    \\\underbar{Se} si riesce ad invertire la funzione $B$
    \\\underbar{allora}: $y(x) = B^{-1}(A(x) + c)$    
    

\end{solution}
   


\begin{gather*}
    \text{Esempio: }\\
\begin{cases}
 y' = \frac{1}{y}, \quad a(x) \equiv 1 \\[6pt]
 y(0) = 2, \quad b(y) = \frac{1}{y}\\
\end{cases}
   \\ \text{(In questo caso si ha $a(x)$ coincidente con 1 perchè è costante e coincide con uno $\forall x$)}\\
\\[10pt]
\int y \, dy = \int 1 \, dx
\\[10pt]
\frac{y^2}{2} = x + c
\\[10pt]
\fbox{$y^2 = 2x + c$}
\\[10pt]
y = \pm \sqrt{2x + c}
\end{gather*}

    
    \hfil
    \\\\La condizione $y(0) = 2$ mi spinge a scegliere $y(x) = \sqrt{2x + 2c}$
    
    
\begin{align*}
        y(0) &= \sqrt{2c} \eqq 2 \quad \text{se } c = 2 \\
        y(x) &= \sqrt{2x + 4} \\
        y' &= a\ y(1 - by) \qquad \text{con } a, b \text{ costanti } > 0 \\
        y(x) &\equiv 0 \\
        y(x) &\equiv \frac{1}{b} \\
        \int \frac{dy}{y(1 - by)} &= \int a \, dx \\
        \ln{\left|\frac{y}{1 - by}\right|} &= ax + c \\
        \left|\frac{y}{1 - by}\right| &= e^{ax} \cdot e^c = c_0\ e^{ax} \quad \text{con } c_0 > 0 \text{ costante arbitraria}
\end{align*}

\begin{lemma}
    Equazioni Omogenee
    \\$y' = y(x,y)$ dove $f(x,y)$ è "omogenea di grado 0"
    \\\emph{cioè} $\forall \lambda \in \mathbb{R} \forall(x,y) f(\lambda x, \lambda y) = f(x,y)$
    \\Prendiamo l'eq:
\begin{align}
    y' = \frac{2xy}{x^2+y^2} \qquad f(x,y) \frac{2xy}{x^2+y^2}
\end{align}
Un equazione di questo tipo può essere riscritta come:
\begin{align}
    y' = b(y/x)
\end{align}
\begin{example}
    \begin{align}
        \frac{2xy}{x^2+y^2} = \frac{2y/x}{1+(y/x)^2} \quad\text{cioè}\ b(t) = \frac{2t}{1+t^2}
    \end{align}
\end{example}

\begin{proposition}
$ y' = b(y/x)\quad \text{Ponente} \quad z(x) = \frac{y(x)}{x} $   
\\cioè $y(x) = x z(x)$ 
\\\\\underbar{Allora} $y'(x) = (x)' z(x)+x z'(x)=z+xz'$
\\L'equazione diventa: $z+xz' = b(z)$
\\cioè $z' = \frac{b(z)-z}{x}$
\end{proposition}

\begin{example}
    \begin{align}
        z' = \frac{\frac{2z}{1+z^2}-z}{x}
        z' = (\frac{z-z^3}{1+z^2})1/x
    \end{align}
\end{example}
\end{lemma}


\section{18/09/25}

\begin{gather*}
    y' = f(x,y) \qquad f(\lambda x, \lambda y) = f(x,y) = \forall \lambda \in \mathbb{R}\\
    y' = f(x,y) \to b(\frac{y}{x^3})\\
    y' = b(y/x)\\
    z \to z(x) = \frac{y(x)}{x}\\
    y(x) = x z(x)\\
    y' = 1 z + x z' = z+ xz'\\
    y' = b(y/x) \to z+xz' = b(z)\\
    z' = \frac{b(z) - z}{x} = (b(z) 1/x)
\end{gather*}

\begin{example}
    \begin{gather*}
        y' = \frac{x^3 + y^3}{xy^2}\\
        f(x,y) = \frac{x^3+y^3}{xy^2} = \frac{1+(x/y)^3}{(y/x)^2}\\
        \text{dove } b(t) = \frac{1+t^3}{t^2}\\
        z' = \frac{(\frac{1 + z^3}{z^2}-z)}{x}\\
        \text{\fbox{$z' = \frac{1}{z^2} \frac{1}{x}$}}\\\\
        \int z^2 dz = \int \frac{dx}{x}\\
        \frac{z^3}{3} = ln|x| + c\\
        z = \sqrt[3]{3ln|x| + 3c}\\
        y(x) = x z(x) = x^3 \sqrt[3]{3ln|x| + 3c}
    \end{gather*}
\end{example}

\begin{proposition}
    Esistenza locale ed esistenza globale\\\\
    Prendiamo un problema di Cauchy
    \begin{gather*}
        \begin{cases}
            y' = -2 x y^2 \qquad f(X,y) = 2xy^2 \quad \forall x \in \mathbb{R} , \forall y \in \mathbb{R}\\
            y(0) = -1 \qquad x_0 = 0, y_0 = -1
        \end{cases}
        \\f(x,y) \text{è continua } \forall x,y \in \mathbb{R}^2, \text{lo stesso vale per } \frac{\partial f}{\partial y} = -\Delta xy\\
        \int \frac{dy}{y^2} = \int -2 dx \quad -\frac{1}{y} = -x^2 + c\\
        y(0) = -1 \quad \frac{1}{0-c} = -1 = c = 1\\
        y= \frac{1}{x^2 - 1}
    \end{gather*}

    I teoremi che abbiamo visto finora dicono che esiste una soluzione in un certo intorno $x_0$, se la funzione è definita in un certo intervallo allora è \underbar{possibile} che la soluzione si trovi in quell'intervallo.
\end{proposition}

\begin{theorem}[di $\exists$ globale]
    Consideriamo il Pb. \fbox{*}
    $\begin{cases}
        y' = f(x,y)\\
        y(x_0) = y_0\\
    \end{cases}$
    \\\underbar{se}:
    \begin{enumerate}[i)]
        \item $f(x,y)$ e $\frac{\partial f}{\partial y}(x,y)$ sono definite e continue $\forall x \in [a,b], y \in \mathbb{R}$ dove $x_0 \in (a,b)$
        \item Esistono due numeri positivi, $h , k$ per cui risulti \fbox{$|f(x,y)| \le h + k|y|$}
    \end{enumerate}
\end{theorem}


\begin{proposition}
    EDO di ordine $n$\\
    \begin{gather*}
        \begin{cases}
          y^{(n)} = f(x, y, y', y'', ... \ ,y^{(n-1)})\\
          y(x_0) = y_0\\
          y'(x_0) = y_1\\
          .\\
          .\\
          .\\
          y^{(n-1)}(x_0) = y_n\\
        \end{cases}
        \overset{\text{es.}}{y'' = 3xy + 7y'}\\\\
        \text{chiamo \fbox{\tikzmarknode{A}{€}}}\\
        y_1(x) := y(x)\ \ \tikzmarknode{B}{} \\
        y_2(x) := y'(x) \\
        \begin{cases}
            y_1' = y_2\\
            y_2' = 4xy_1 + 7 y_2\\
        \end{cases}
        \text{\fbox{**}}\\
        y'' = 3xy +7y' \text{\fbox{*}}
    \end{gather*}
    \TikzNCbar[-12pt]{-latex}{A}{B}
    \underbar{Se} $y$ risolve \fbox{*} \underbar{allora} la coppia $y_1, y_2$ definite da \fbox{€} risolve \fbox{**}.\\
    Viceversa \underbar{se} $(y_1, y_2)$ risolvono \fbox{**} \underbar{allora} la funzione $y_1$ risolve \fbox{*}
\end{proposition}

\begin{proposition}
    (EDO) Equazioni differenziali ordinari lineari del 1° ordine\\
    \hookbox{
    \begin{gather*}
        y' = a(x)y = f(x)\\
    \end{gather*}
    }
    \fbox{*} $z' +a(x) (z) = 0$ omogenee\\
    \underbar{se} $z_1(x)$ e $z_2(x)$ sono soluzioni di \fbox{*}\\
    \underbar{allora} anche $\alpha z_1(x) + \beta z_2(x), \alpha , \beta \in \mathbb{R}$\\
    è soluzione di \fbox{*}\\
    infatti $(\alpha z_1(x) + \beta z_2(x))'+a(x)(\alpha z_1+\beta z_2(x)) \eqq 0$
\end{proposition}


\begin{proposition}
    \begin{gather*}
        y' + \frac{x}{y} = \sin(x) \qquad a(x) = \frac{1}{x} \quad f(x) = -\sin(x)\\
        A(x) = \int \frac{1}{x} dx = ln |x|\\
        F(x) = \int f(x) e^{A(x)} dx = \int -\sin(x)e^{ln|x|} dx = \int -|x| \sin(x) dx\\
        \text{quando } x>0 \to -\int x \sin(x) = \sin(x) - x \cos(x) \\
        \text{quando } x<0 \to -\int x \sin(x) = -\sin(x) + x \cos(x) \\
        y(x) \text{ha due soluzioni\dots}\\
        c e^{ln|x|}+e^{-ln|x|} (\sin(x)-\cos(x)) \text{quando } x>0\\
        c e^{ln|x|}+e^{-ln|x|} (-\sin(x)+\cos(x)) \text{quando } x<0\\
        y(x) = \frac{c}{|x|} + \frac{c}{x}(\sin(x)-x\cos(x)) \text{quando } x>0\\
        y(x) = \frac{c}{|x|} + \frac{c}{-x}(-\sin(x)+x\cos(x)) \text{quando } x<0\\
    \end{gather*}
L'equazione non ha senso quando $x=0$\\
\end{proposition}
\begin{definition}
    Spazi di funzioni\\
    $I$ è un intervallo in $\mathbb{R}$\\
    si considerano \{\underbar{insieme delle funzioni} $: I \to \mathbb{R}$\}, chiamiamo questo sistema \fbox{*}\\
    prendiamo $f,g \in $ \fbox{*} e \fbox{$\alpha f(x) + \beta g(x)$} $\in $ \fbox{*} \tab con $f(x) \equiv 0$\\\\
    e si considera \{\underbar{insieme delle funzioni} $: I \to \mathbb{R}$\} come spazio vettoriale\\
    \begin{enumerate}[ ]
        \item $C^0(I) = \{ \text{insieme delle funzioni \underbar{continue}} : \to \mathbb{R}\}$
        \item $C^1(I) = \{ \text{insieme delle funzioni \underbar{continue}, derivabili in ogni $x \in I$ e la cio derivata è una funzione continua $\forall x \in I$}\}$
        \item .
        \item .
        \item .
        \item $C^{n}(I)$
    \end{enumerate}

    \begin{theorem}
        \hfil\\
        \begin{enumerate}[i)]
            \item L'insieme $V_0$ delle soluzioni di \fbox{2} è uno spazio vettoriale di dimensione n
            \item L'insieme delle soluzioni di \fbox{1} è
            \hookbox{\begin{gather*}
                \begin{cases}
                    y(x) +y_f(x) : y \in V_0 \text{e} y_f \text{è soluzione di \fbox{1}}
                \end{cases}
            \end{gather*}}
        \end{enumerate}
    \end{theorem}
    \begin{gather*}
        \\\text{\fbox{$y^{n} a_{n-1}(x) y^{(n-1)} + \dots + a_1(x) y'+ a_0(x)y = f(x)$}} \ \ \text{\fbox{1}}\\
        a_0, a_1, \dots , a_{n-1}\text{supponiamo che siano definite e continue in } I \subset \mathbb{R}\\
        y \in C^{n}(I)\\
        E(y):C^{n}(I)\to C^0(I)\\
        E(y) = y^{(n)}(x) + a_{(n-1)}(x)y^{(n-1)}(x) + \dots \ + a_0(x)y()\\
        E( \alpha y_1 + \beta y_2) = \alpha E(y_1) + \beta E(y_2)\\
        \text{\fbox{$y^{(n)}+a_{(n-1)}(x)y^{(n-1)} + ... \ + a_0(x)y(x) = 0$}} \ \ \text{\fbox{2}}\\
    \end{gather*}

\end{definition}

\section{23/09/25}
abbiamo parlato delle equazioni differenziali lineari $y' +a(x) = f(X)$ in questo caso del primo ordine.\\
successivamente abbiamo introdotto quelle di ordine $n$ che appaiono in una forma del tipo\\
$y^{(n)}+ a_{n-1}(x)y^{(n-1)}+ ... \ + a_1(x)y' + a_0(x)y = f(x)$\\
dove $a_i(x), i = 0, ... \ , n-1$ sono funzioni definite e continue in un intervllo $I \subset \mathbb{R}$\\

\begin{gather*} 
    C^0(I) = \{\text{ \underbar{insieme di funzioni continue}:} I \to \mathbb{R}\}\\
    C^1(I) = \{\text{ \underbar{insieme di funzioni continue e derivabili con derivata continua in}}: I \to \mathbb{R}\}\\
\end{gather*}


\begin{gather*}
\text{Sono spazi vettoriali su} \mathbb{R}\\
E: C^n(I) \to C^0(I)\\
E (y(x) \in \text{*} ) = y^{(n)}+ a_{n-1}(x)y^{(n-1)}+ ... \ + a_1(x)y' + a_0(x)y = y(x)\\
E(\alpha y_1(x) + \beta y_2(x)) = \alpha E (y_1(x)) + \beta E (y_2(x))\\
E \text{ è lineare}\\
\end{gather*}

\begin{theorem}
        $\exists !$ globale per soluzione di EDO lineari\\
        \underbar{Siano} $f(x), a_i(x) \quad i = 0, ... \ , n-1$ funzioni continue in $I$\\
        \underbar{Sia} $x_0$ interno ad $I$ \\
        \underbar{Allora } $\forall$ scelta dei numeri \\
        $b_0, b_1, ... \ , b_{n-1}$ il prob. di Cauchy: \\
    \begin{gather*}
        \begin{cases}
            E(y) = f\\
            y(x_0) = b_0\\
            y'(x_0) = b_1\\
            \vdots\\
            y^{(n-1)}(x_0) = b_{n-1}\\
        \end{cases}
        \text{Ha una sola soluzione definita in tetto } I
    \end{gather*}
    (no dim)\\
\end{theorem}

\begin{theorem}
    L'insieme delle soluzioni dell'EDO omogenea $E(y)= 0$ è uno spazio vettoriale di dimensione n.\\
    \fbox{*} è spazio vettoriale. \\
    La funzione $y(x) \eqq 0$ soddisfa l'eq $E(y) = 0$ ovviamente.\\
    \underbar{Se} $y_1$ e $y_2$ sono due soluzioni dell'omogenea, cioè\\
    $E(y_1) = 0$ e $E(y_2) = 0$\\
    e $\alpha , \beta \in \mathbb{R}$ \underbar{allora}\\
    $E(\alpha y_1 + \beta y_2) = \alpha E(y_1) + \beta E(y_2) = \alpha 0 + \beta 0 = 0$\\ 
\end{theorem}

\begin{proof}
    \hfil\\
    Dimostro il teorema quando $n=2$\\
    Def. le due funzioni $z_0(x)$ e $z_1(x)$ nel modo:\\
    \begin{multicols}{2}
        \noindent
    \begin{gather*}
       z_0 \text{è sol. del pb.}\\
       \begin{cases}
        E(y) = 0\\
        y(x_0) = 1\\
        y'(x_0) = 0\\
       \end{cases}
    \end{gather*}
    \begin{gather*}
       z_1 \text{è sol. del pb.}\\
       \begin{cases}
        E(y) = 0\\
        y(x_0) = 0\\
        y'(x_0) = 1\\
       \end{cases}
    \end{gather*}
        \columnbreak
    \end{multicols}
    
    Richiamo cosa vuol dire nel contesto degli spazi di funzioni cosa buol dire linearmente indipendenti\\
    $z_0$ e $z_1$ sono lin. indipendenti \underbar{se}\\
    $C_0 z_0(x) + c_1 z(x) = 0 \forall x \in I$\\
    dove $c_0, c_1$ sono costanti, avviene solo quando $c_0 = c_1 = 0.$\\
    \\\underbar{Siano} $c_0, c_1 \in \mathbb{R}$ : $c_0 z_0(x0 + c_1 z_1(x) \eqq 0 \forall x \in \mathbb{R})$\\
    chiamo: $z(x) $ è sol. di $z(x_0) = c_0 z_0(x_0) + c_1 z_1(x_0) = c_0 \quad \text{dove } c_0 z_0(x_0) = 1 \ \text{e } c_1 z_1(x_0) = 0$\\
    $z'(x_0) = c_0 z_0'(x_0) + c_1 z_1'(x_0) \quad \ \text{dove } c_0 z_0'(x_0)=0 \ \text{e } c_1 z_1'(x_0)= 1$\\
    \underbar{Ma} per l'ipotesi $z \eqq 0$ quindi $z(x_0) = 0$ \underbar{Ma anche} $z' \equiv 0 $ e $z'(x_0) = 0$\\
    Rimane da dimostrare che $z_0 e z_1$ generano tutto lo spazio delle soluzioni dell'omogenea.\\
    \underbar{Sia} $w(x)$ una sol. arbitraria di $E(y) = 0$.\\
    Voglio dumostrare che è probabile scegliere $c_0, c_1 \in \mathbb{R}$\\
    \underbar{t.c.} $w(x) \equiv c_0 z_0(x) + c_1 z_1(x)$\\
    Scelgo $c_0$ e $c_1$:\\

    \fbox{
    $c_0 = w(x_0)$\\
    $c_1 = w'(x_0)$\\
    }
    Sia $w$ che $c_0 z_0 + c_1 z_1$ sono soluzioni di:\\
    $
    \begin{cases}
        E(y) = 0\\
        y(x_0) = c_0\\
        y'(x_0) = c_1\\
    \end{cases}
    $
    \begin{gather*}
        c_0 z_0 (x_0) + c_1 z_1 (x_0) = c_0 \qquad \text{ dove } \ c_0 z_0 (x_0)= 1 \text{e} \ c_1 z_1 (x_0) = 0\\
        c_0 z_0' (x_0) + c_1 z_1' (x_0) = c_0 \qquad \\
    \end{gather*}
\end{proof}
\begin{theorem}
    \hfil\\
    L'integrale generale dell'equazione \fbox{$E(y) = f$}\\
    si ottiene sommando l'integrale generale dell'equazione omogenea ad una soluzione particolare dell'equazione completa\\
    $E(y) = f$.
\end{theorem}

\begin{proof}
    \underbar{Sia} $y_1$ una soluzione di $E(y) = f$\\
    \underbar{e} $y_0$ una soluzione di $E(y) = 0$\\
    \unexpanded{Allora} $y_1 + y_0$ è soluzione di:\\
    $E(y) = f$ infatti\\
    $E(y_1 + y_0) = E(y_1) + E(y_0) = f + 0 = f$\\
    Viceversa \underbar{sia} $y_2$ una qualsiasi soluzione di $E(y) = f$\\
    \underbar{Allora}:\\
    $E(y_2) y_1 = E(y_2) - E(y_1) = f - f = 0$\\
    Quindi $y_2 - y_1 = z_0$ per una certa soluzione dell'omogenea.\\
    \underbar{Allora}: $y_2 = y_1 + z_0$\\
\end{proof}

\begin{proposition}
    Per risolvere lequazione lineare dellordine $n$ posso:\\
    \begin{enumerate}[i)]
        \item Trovare $n$ soluzioni lineari indipendenti omogenee
        \item Trovare una soluzione della equazione completa
    \end{enumerate}
    \underbar{Punto 1} $\exists$ una strategia generale solo quando $a_0,a_1, ... , \ a_{n-1}$ sono costanti.\\
    \fbox{EDO lineari a coefficienti costanti}\\
    \begin{example}
        $y'' - 3y' +2y = 0$\\
    \end{example}
    Se parto da:\\
    $y^{(n)}+ a_{n-1}y^{(n-1)}+ ... \ + a_1 y' + a_0 y = 0 \leftarrow$ associo caratteristico:\\
    $P(\lambda) \ \ $: $\ \lambda^n + a_{n-1} \lambda^{n-1} + ... \ + a_1 \lambda + a_0$\\
\end{proposition}
%th per sol. eq. omogenea
\begin{theorem}
    \underbar{Sia} $\lambda \in \mathbb{R}$ ( o anche $\lambda \in \mathbb{C}$).\\
    La funzione: $y(x) = e^{\lambda x}$\\
    è soluzione dell'equazione omogenea $E(y) = 0$ $\Leftrightarrow$ $\lambda$ è radice del polinomio caratteristico\\
    cioè se $P(\lambda) = 0$\\
\end{theorem}
%dim
\begin{proof}
    \hfil\\
    Chiamo
    \begin{enumerate}[ ]
        \item $y(x) = e^{\lambda x}$
        \item $y'(x) = \lambda e^{\lambda x}$
        \item $y''(x) = \lambda^2 e^{\lambda x}$
        \item .
        \item .
        \item .
        \item $y^n(x) = \lambda^n e^{\lambda x}$
    \end{enumerate}
    \begin{gather*}
        E(y) = \lambda^n e^{\lambda x} + a_{n-1} \lambda^{n-1} e^{\lambda x} + ... + a_1\lambda e^{\lambda x } + a_0 e^{\lambda x}\\
        = (\lambda^n + a_{n-1}\lambda^{n-1}+ ... \ + a_0)e^{\lambda x}\\
        P(\lambda) e^{\lambda x} = 0 \ \text{\underbar{Se e solo se} } P(\lambda) = 0\\
    \end{gather*}
\begin{example}
            se ho: \fbox{$y'' -3y' +2y = 0$} e considero il polinomio caratteristico associato: $\lambda^2 - 3\lambda + 2 = 0$\\
            procedendo a risolverlo per $\lambda$ ottengo: $(\lambda - 1)(\lambda - 2) = 0$
            da qui si possono vedere ad occhio delle radici semplici di $P(\lambda)$ che sono: $\lambda_1= 2$ e $\lambda_2 = 1$ \\
            di conseguenza: $e^{2x}$ e $e^x$ sono soluzioni di: $E(y) = 0$\\
\end{example}
    \begin{observation}
        \underbar{se} $\lambda_1,\lambda_2, ... , \lambda_k $ sono dumeri distinti \\ 
        \underbar{allora} $e^{\lambda x},e^{\lambda_2 x}, ... \ , e^{\lambda_k x},$ sono linearmente indipendenti.\\
    \end{observation}
\end{proof}

\begin{theorem}
    \hfil\\
    \underbar{Sia} $\lambda_0 \in \mathbb{R} $ o $\lambda_0 \in \mathbb{C}$ radice di molteplicità $s$ del polinomio $P(\lambda)$ cioè:\\
    \begin{gather*}
        P(\lambda) = (\lambda - \lambda_0)^s q(\lambda)\\
    \end{gather*}
    Con $q(\lambda_0) \neq 0$ e $P(\lambda_0) = 0$\\
    \underbar{Allora}:\\
    \begin{gather*}
        e^{\lambda_0 x}, x e^{\lambda_0 x}, x^2 e^{\lambda_0 x}, ... \ , x^{s-1} e^{\lambda_0 x}\\
    \end{gather*}
    Sono soluzioni linearmente indipendenti di $E(y) = 0$
\end{theorem}

\begin{theorem}
    \hfill\\
    \underbar{Sia} $\lambda = \alpha+i\beta$ una radice di molteplicità $s$ del polinomio caratteristico.\\
    \underbar{Allora}:\\
    \begin{gather*}
        e^{\alpha x} \cos \beta x , e^{\alpha x} \sin \beta x\\
        x e^{\alpha x} \cos \beta x ,x e^{\alpha x} \sin \beta x\\
        x^{s-1} e^{\alpha x} \cos \beta x, x^{s-1} e^{\alpha x} \sin \beta x\\
    \end{gather*}
    Sono soluzioni linearmente indipendenti \underbar{Reali} di $E(y) = 0$\\
\end{theorem}

\begin{proposition}
    Ho piu o meno risposto adesso alla domanda del punto i).\\
    Quindi sono in grado di risolvere la ED :
    \begin{enumerate}[i)]
        \item Se $f$ è un polinomio
        \item Se $f$ = polinomio $e^{\alpha x}$
        \item Se $f = P_1(x) \sin \beta x + P_2(x) \cos \beta(x)$\\
        \item Se l'equazione è a coefficienti costanti ( metodo di somiglianza). 
    \end{enumerate}
    Passiamo ora a vedere il metodo generale per trovare una soluzione della completa.\\
\end{proposition}

\begin{proposition}
    Metodo generale, anche per equazioni non a coefficienti costanti. Lo vedremo solo quando $n=2$.\\
    \underbar{Metodo della variazione delle costanti.}\\
    Come punto di partenza devo avere due soluzioni lineari indipendenti dell'omogenea e le chiamo:\\
    \begin{gather*}
        y_1(x), y_2(x)\\
    \end{gather*} 
    Cerco una soluzione della forma:\\
    \begin{gather*}
        y(x) = c_1(x) y_1(x) + c_2(x) y(x)\\
    \end{gather*}
    Come devo scegliere $c_1(x)$ e $c_2(x)$?\\
    \begin{gather*}
        y' = c_1' y_1 + c_1 y_1' + c_2' y_2 + c_2 y_2' = \text{\underbar{$c_1' y_1 + c_2' y_2'$}} + c_1 y_2 + c_2 y_2' \\\text{e pongo $c_1' y_1 + c_2' y_2' = 0$} \\
        y'' = c_1' y_1' c_2' y_2' + c_1 y_1'' + c_2 y_2''\\
        \text{riprendo} \ y'' + a_1(x)y' + a_0(x) = f(x)\\
        \text{ora sostituisco } y' = c_1' y_1' + c_2' y_2' = f \text{quindi } c_1 \text{e} \ c_2 \text{devono soddisfare il seguente sistema}\\
        \begin{cases}
            c_1' y_1 + c_2' y_2 = 0\\
            c_1' y_1' + c_2' y_2' = f\\
        \end{cases}
    \end{gather*}
\end{proposition}

\newpage
\section{30/09/25}
\textbf{INTRODUZIONE spazi METRICI e NORMATI}

\begin{definition}
    $X$ insieme, $d:X\times X \to \mathbb{R}, d$ è una METRICA su $X$ se:
    \begin{enumerate}[$i)$]
        \item $\forall x,y \in X \quad d(x,y) \geq 0 \quad$ (e questo implica che $d(x,y)=0 \Leftrightarrow (x=y)$)
        \item $\forall x,y \in X \quad d(x,y) = d(y,x)$
        \item $\forall x,y,z \in X \quad d(x,y) \leq d(x,z) + d(z,y)$
    \end{enumerate}
\end{definition}
\begin{definition}
    $(X,d)$ sp. METRICO
\end{definition}
\begin{example}
    $X=\mathbb{R} \quad d(x,y) = |x-y|$\\
\end{example}

\begin{definition}
    $V$ sp. vett., $||\cdot||:V \to \mathbb{R}$ è una norma su $V$ se:
    \begin{enumerate}[$i)$]
        \item $\forall v\in V, ||v|| \geq 0 \quad$ (e qesto implica che $||v||=0 \Leftrightarrow v=0$)
        \item $||\lambda x|| = |\lambda| ||x|| \quad \forall \lambda \in \mathbb{R}, \forall x \in V$
        \item $\forall u,v \in V \quad ||u+v|| \leq ||u||+||v||$
    \end{enumerate}
\end{definition}
\begin{definition}
    ($V,||\cdot||$) sp. normato.
\end{definition}
\begin{example}
    \begin{gather*}
        V=\mathbb{R}^N \quad ||v|| = \sqrt{\sum_{i=0}^{N}v_i^2} = \sqrt{v_1^2 + v_2^2 + \ ... \ + v_N^2} \text{ (teorema di Pitagora)}\\
        \text{con } v = (v_1, \ ... \ , v_N) \in \mathbb{R}^N
    \end{gather*}
\end{example}
\begin{definition}
    $(X, d\alpha),(X,d\beta)$ sp.metrici $d\alpha, d\beta$ sono metriche equivalenti se:\\
    \begin{gather*}
        \exists k_1,k_2 \in \mathbb{R}^+ \quad t.c. \quad k_1 d\beta(x,y) \leq d\alpha(x,y) \leq k_2 d\beta(x,y)\\
        \forall x,y \in X
    \end{gather*}
    si può anche riscrivere come:
    \begin{gather*}
        k_1 \leq \frac{d\alpha(x,y)}{d\beta(x,y)} \leq k_2
    \end{gather*}
    Nel quale si vede che le norme sono equivalenti se il rapporto è confrontabile con numeri finiti.
\end{definition}
\begin{definition}
    $(X, ||\cdot||\alpha), (V, ||\cdot|| \beta), ||\cdot|| \alpha, ||\cdot|| \beta$ sono norme equivalenti se:\\
    \begin{gather*}
        \exists c_1, c_2 \in \mathbb{R}_i
    \end{gather*}
    sp. normati
    \begin{gather*}
        \forall v \in V \quad c_1||v||_\beta \leq ||v||_\alpha \leq c_2 ||v||_\beta
    \end{gather*}
\end{definition}
\begin{theorem}
    \hfil\\
    Tutte le norme su $\mathbb{R}^N$ sono EQUIVALENTI.
\end{theorem}

    \begin{figure}[tbh]
    \centering
    \includesvg[width = 150 pt]{anal1}
    \caption{visualizzazzione delle norme su $\mathbb{R}^N$}\label{fig:testsvg}
    \end{figure}

\begin{example}
    Norme su $\mathbb{R}^N$
\begin{gather*}
    ||\text{\underbar{$x$}}||_p = \sqrt[p]{\sum_{i=1}^{N}{|x_i|}^p}\\
    ||x_1|| = \sum_{i=1}^{N} |x_i|\\
    ||x_2|| = ||\text{\underbar{$x$}}||_\epsilon \quad \text{euclidea}\\
    ||\text{\underbar{$x$}}||_\infty = \lim_{p \to \infty}(\sum{|x_i|}^p)^{\frac{1}{p}} = max_{i=1-N}|x_i|\\
    \beta_p = \{ \text{\underbar{$x$}} \in V : || \text{\underbar{$x$}}||_p \leq 1\}
\end{gather*}
\end{example}
\begin{example}
    funzioni continue derivabili $k$ volte $X = C^k ([a,b])$\\\\
    \begin{enumerate}[$I)$]
        
         \item $X = C^0([a,b])$ \\
    \begin{enumerate}[$i)$]
        \item $||f||_{C_0} = ||f||_\infty =\underset{{x \in [a,b]}}{max}|f(x)|$\\
            metrica $d_{e^0}(f,g) = d_\infty (f,g) = || f-g||_\infty$
        \item $d_{L_1}(f,g) = \int_{a}^{b} |f-g| dx$ è metrica su $C^0$ ma NON è equivalente a $d_e$
    \end{enumerate}
    \underbar{\textbf{OSS}}\\
    \begin{itemize}
        \item $X = C_b^0 ( \ (a,b) \ ) \quad ||f||_{c_b (a,b)} = \underset{x \in (a,b)}{sup} |f(x)|$
        \item $X = C^1 ( \ [a,b] \ ) || f||_{C^0} == ||f||_\infty = \underset{x \in [a,b]}{max}|f(x)|$
        \item $||f||_{C^1} = ||f||_{C^0} + ||f'||_{C^0}$
    \end{itemize}
    \item $X = C^k$\\
     $( \ [a,b] \ )_{k \geq 0} d_{C^k} (f,g) = \sum_{i=0}^{k} d_{C^0} (f^{(i)} - g^{(i)}) = \sum_{i=0}^{k} \underset{x \in [a,b]}{max}|f^{(i)} - g^{(i)}|$
\end{enumerate}
    \end{example}
\hfil\\
    Cenni di TOPOLOGIA. $(x,d)$ metrico.\\

    \begin{definition}
        Presi $x_0 \in X$ e $R>0$\\
        Un intorno circolare aperto in centro $x_0$ e raggio $R$ è\\
        \begin{gather*}
            B_R (x_0) = \{ x \in X: d(x_0, x) < R\} = \mathcal{U} (x_0,R)
        \end{gather*}
        un intorno circolare chiuso in centro $x_0$ e raggio $R$ è\\
        \begin{gather*}
            \overline{B_R} (x_0) = \{ x \in X:d(x_0, x) \leq R\}
        \end{gather*}
    \end{definition}
    \begin{definition}
        $A \subseteq X$ è un APERTO se $\forall a \in A \quad \exists R_1 \quad t.c. \quad \mathcal{U}(a, R_1) \subseteq A$
    \end{definition}
    \begin{definition}
        $D \subseteq X$ è un CHIUSO se $X\ D$ è aperto
    \end{definition}

    \underbar{\textbf{OSS}}\\
    $\bigcup$ di aperti (anche num) è aperto\\
    $\bigcap$ finita di aperti \\
    $\bigcup$  '' di chiusi '' CHIUSA\\
    $\bigcap$ di chiusi (anche num) è CHIUSA\\
    \begin{definition}
        $x_0 \in X $ è punto interno se:\\
        $\exists R \ t.c. \ \mathcal{U}(x_0, R) \subseteq X$
    \end{definition}

    \begin{theorem}
        (HEINE-BOPEL)\\
        Se $d\subseteq \mathbb{R}^N$ è un insieme compatto allora è anche chiuso e limitato.
    \end{theorem}
    \hfil\\
    SUCCESSIONI in SPAZI METRICI $(x,d)$
    \begin{definition}
        $\{x_k\} \subseteq X \quad x_k \xrightarrow{d} x $ in $X$ se \\
            $x \in X$ e:\\
            $\lim_{k \to +\infty} d(x_k, x) = 0$
    \end{definition}
    \hfil\\
    \underbar{\textbf{OSS}}\\
        se $\exists$ il limite è !\\
    \underbar{\textbf{OSS}}\\
    $d_\alpha d_\beta$ metriche equivalenti su $X$\\
    ${x_k} \subseteq X$\\
    Allora:\\
    $x_k$ converge in x rispetto alla metrica $d_\alpha$\\
    uguale se $x_k$ '' '' $d_\beta$
    \begin{definition}
        $x_k \subseteq X$ è succ di cauchy in $(X,d)$ se\\
        $\forall \epsilon > 0 \ \exists N \ t.c. \ d(x_k, x_m) < \epsilon \qquad \forall k,m > N$
    \end{definition}
    \hfil\\
    \underbar{\textbf{OSS}}\\
    $x_k$ di cauchy in $(x,d) \ \cancel{\Rightarrow} \ x_k$ converge in $(x,d)$\\
    $x_k$ di cauchy in $(x,d) \Leftarrow x_k$ convergente in $(x,d)$
    \begin{example}
    \begin{gather*}
        X = C^1 ( \ [0,1] \ )\\
        d(f,g) = d_{C^0}(f,g) = \underset{x \in [0,1]}{max} |f(x)- g(x)|\\
        x_k = \sqrt{x^2 + \frac{1}{k}} \quad \text{per } x \in [0,1]\\
        x_k: [0,1] \to \mathbb{R}, x_k \in C^1(\ [0,1] \ ) \forall k \Rightarrow \{x_k\} \subseteq C^1
    \end{gather*}     
    \hfil\\
    \underbar{\textbf{OSS}}\\
    $\{x_k\}$ di Cauchy\\
    \begin{gather*}
        d(x_k,x_m) = \underset{x \in [0,1]}{max} |x_k-x_m| = \underset{x \in [0,1]}{max} |\sqrt{x^2+\frac{1}{k}}-\sqrt{x^2+\frac{1}{m}}| \underset{sia \ m>k}{=} \text{\fbox{*}}\\
        \text{\fbox{*}} \underset{x \in [0,1]}{max} \frac{x^2 +\frac{1}{k} -(x^2+\frac{1}{m})}{\sqrt{x^2+\frac{1}{k}}+\sqrt{x^2+\frac{1}{m}}} \underset{x^2 \geq 0}{\leq} \frac{\frac{1}{k}-\frac{1}{m}}{\frac{1}{\sqrt{k}}+\frac{1}{\sqrt{m}}} = \frac{1}{\sqrt{k}}- \frac{1}{\sqrt{m}} < \epsilon
    \end{gather*}

    \hfil\\
    \underbar{\textbf{OSS}}\\
    \begin{gather*}
        x_k \ \cancel{\xrightarrow{d_{C_0}}} in \ C^1\\
        x_k \sqrt{x^2+\frac{1}{k}} \xrightarrow{d_{C^0}} \overset{\overset{\sqrt{x^2}}{||}}{|x|} = f(x)\\
        d_{C^0}(x_k,f) = \underset{x \in [0,1]}{max} \sqrt{x^2 +\frac{1}{k}}-\sqrt{x^2} = \underset{x \in [0,1]}{max} \frac{\frac{1}{k}}{\sqrt{x^2+\frac{1}{k}}+\sqrt{x^2}} \leq \frac{\frac{1}{k}}{\frac{1}{k}} = \frac{1}{\sqrt{k}} \to 0
    \end{gather*}

    \end{example}

    \newpage
    \section{1/10/25}
    \begin{example}
        \begin{gather*}
            X = C^1 ( \ [a,b] \ ) \exists \{x_k\} \text{di Cauchy} \subseteq X \ t.c. \ X_k \cancel{ \to } x_0 in X        
        \end{gather*}
    \end{example}
    \begin{theorem}
        $(X,d)$ metrico $\{x_k\} \subseteq X \ t.c. \ x_k \xrightarrow{d} x_0 \in X \Rightarrow \{x_k\}$  di Cauchy rispetto a $d$
    \end{theorem}
    \hfil\\
    \underbar{\textbf{OSS}}\\
    NON è vero il viceversa\\
    \begin{proof}
        Sia $m>n$, considero\\
        \begin{gather*}
            d(x_m,x_n) \leq d(x_m,x_0) + d(x_n,x_0)
        \end{gather*}
        Poichè $x_k \xrightarrow{d} x_0$ in $ X$ si ha che:
        \begin{gather*}
            \forall \epsilon > 0 \ \exists N \ t.c. \ \text{\underbar{se} } k > N\\
            \text{\underbar{allora} } d(x_k,x_0) < \epsilon 
        \end{gather*}
        fix. $\epsilon > 0$ sia $N \ t.c. \ d(x_k,x_0) < \epsilon \quad \forall k>N$\\
        \underbar{allora} $\forall m,n > N$\\
        \begin{gather*}
            d(x_m,x_n) \leq 2\epsilon\\
            \text{cioè } \{x_k\} \text{ è di Cauchy in} (X,d) 
        \end{gather*}
    \end{proof}


    \begin{definition}
        $(X,d)$ metrico $t.c.$ $g_n$ è successione di Cauchy è conv. è detto sp. metrico COMPLETO.
    \end{definition}

    \begin{definition}
        $(V, ||\cdot||)$ sp. vett. completo rispetto alla metrica indotta da $||\cdot||$ è detto sp. di BANACH.
    \end{definition}

    \hfil\\
    \underbar{\textbf{OSS}}\\
    $(C^1 ( \ [a,b] \ ), d_{C^0})$ non è completo\\
    $(C^0 ( \ [a,b] \ ), d_{C^1})$ non è completo
    
    \hfil\\
    \underbar{\textbf{OSS}}\\
    $(C^0 ( \ [a,b] \ ), d_{C^0})$ è completo\\
    $(C^k ( \ [a,b] \ ), d_{C^k})$ è completo

    \hfil\\
    \underbar{\textbf{OSS}}\\
    $(X,d)$ metrico $\{x_k\} \subseteq X$ $t.c.$ $x_k \xrightarrow{d} x_0 \in X \Rightarrow x_o \in \overline{Y} \ \text{chiusura}$\\
    $Y \subseteq X, Y \ \text{chiuso} \Leftrightarrow \text{contiene i limiti di tutte le succ. convergenti di } Y$

    \hfil\\
    \underbar{\textbf{OSS}}\\
    $(X.d)$ completo, $B$ chiuso $\subseteq X \Rightarrow (B,d)$ sp. metrico completo.

    \begin{definition}
        $(X, dx),(Y,dy)$ metrici $f:X \to Y$\\
        \begin{itemize}
            \item f è continua in $x_0$ \underbar{se} $\forall \{x_n\} \subseteq X: x_n \xrightarrow{\text{in } d(x,dx)} x_o$ si ha $\{f(x_n)\} \subseteq Y$ converge a $f(x_0)$ in $(Y,dy)$
            \item f è lipschitziana se $\exists L>0 \ t.c. \ dy( f(x), f(y)) \leq L \ dx(x,y) \qquad \forall x,y \in X$
            \item f si dice contrazione se lipschitziana con $L<1$
        \end{itemize}
    \end{definition}

    \underbar{\textbf{OSS}}\\
    \begin{proof}
        $[f \ lip.] \Rightarrow f \ cont.$\\
        \underbar{sia} $x_0 \in X$ considero $\{X_n\} \subseteq X \ t.c. \ x_n \to x_0 \ in \ (X,d)$ cioè $d(x_n, x_0) \xrightarrow{in \ \mathbb{R}} 0$\\
        considero $0 \leq dy ( f(x_n), f(x_o)) \leq L \underset{\to 0}{dx(x_n, x_0)} \Rightarrow dy ( f(x_n), f(x_0)) \to 0$ cioè $f(x_n) \to f(x_0)$ in $(y,dy)$\\
    \end{proof}

    \begin{theorem}
        delle CONTRAZIONI (o di BANACH-CACIPPOLI)\\
        $(X,d)$ sp. metrico completo $f:X \to X$ contrazione.\\
        \underbar{Allora: } $\exists! \overline{x} \in X $ punto fisso $t.c.$ $f(\overline{x}) = \overline{x}$\\
    \end{theorem}

    \begin{proof}
        Scelgo $x_0 \in X$ costruisco $\{x_k\} \subseteq X$\\
        \begin{enumerate}[ ]
            \item $x_1 = f(x_0)$
            \item $x_2 = f(x_1)$
            \item $x_3 = f(x_2)$
            \item .
            \item .
            \item .
            \item $x_{k+1} = f(x_k)$
        \end{enumerate}
        \begin{itemize}
            \item $\{x_k\}$ è succ. di Cauchy in $x,d$ infatti:\\
            \begin{gather*}
                d(x_{k+1}, x_k) = d(f(x_k), f(x_{k-1})) \leq  L d(x_k, x_{k-1})
            \end{gather*}
            ma: $x_{k+1} = f(x_k)$, $x_k = f(x_{k-1})$ e $x_{k-1} = f(x_{k-2})$\\
            quindi
            \begin{gather*}
                L d(f(x_{k-1}), f(x_{k-2})) \leq L^2 d(x_{k-1} x_{k-2}) \leq L^3 d(x_{k-2}, x_{k-3}) \leq L^k d(x_1, x_0) = L^k d(x_0, f(x_0))
            \end{gather*}
            considero $m>n \quad (m=n+p)$
            \begin{gather*}
                (x_m, x_n) = d(x_{n+p, x_n}) \leq d(x_{n+p}, x_{n+p-1}) +d(x_{n+p-1}, x_{n+p-2}) + \ ... \ + d(x_{n+1}, x_n)  \leq \text{\fbox{*}}\\
                \text{\fbox{*}} \leq d(x_0, f(x_0)) [L^{n+p-1}+L^{n+p-2}+ \ ... \ + L^{n+1} + L^n] = \\
                d(x_0, f(x_0)) (\frac{L^n-L^{n+p}}{1-L}) = d(x_0, f(x_0)) L^n \frac{1-L^p}{1-L} (< \frac{1}{1-L})\\
                < d(x_0, f(x_0) \ (fix.)) \frac{L^n}{1-L} \xrightarrow{lim_n \to +\infty} \text{ poichè } L \in (0,1)\\
                \Rightarrow \forall \epsilon>0 \exists N: \text{\underbar{se }} m,n>N \text{\underbar{ allora }} d(x_m, x_n) < \epsilon \text{cioè } \{x_k\} \text{di Cauchy in } (x,d)
            \end{gather*}
            \item $\exists \overline{x} \in X \ t.c. \ x:k \to \overline{x} \ in \ (X,d)$\\
            poichè $(x,d)$ p sp. metrico completo\\
            $\{x_k\}$ è di Cauchy\\
            \item $\overline{x} $ è punto fisso cioè $f(\overline{x}) = \overline{x}$
            \item infatti: f è continua $\Rightarrow$ $f$ è continua in $(x,d)$\\
            \begin{gather*}
                \text{Per l'unicità del limite} \overline{x} = f(\overline{x})\\
                    f(x_k) \to f(\overline{x}) \text{ in } (X,d) \underset{\text{per def. di} \{x_k\}}{=} x_{k+1} \text{ in } (X,d)
            \end{gather*}
            \item il punto fisso è unico infatti:
             \begin{gather*}
                \text{supp.} \exists \overline{x}, \overline{y} : f(\overline{x}) = \overline{x}, f(\overline{y}) = \overline{y} \text{ con} L \in (0,1)
             \end{gather*}
             \underline{se } $d(\overline{x}, \overline{y}) \neq 0 \\$
             \underline{allora} $< d(\overline{x}, \overline{y})$ \textbf{assurdo}\\
             $\Rightarrow d(\overline{x}, \overline{y}) = 0$ cioè $\overline{x} = \overline{y}$
        \end{itemize}
    \end{proof}

    \begin{example}
        \begin{gather*}
            x= \mathbb{R}^2 d(x,y) =
            \begin{cases}
                |x-y| \qquad \text{se $x,y,0$ sono collineari in $\mathbb{R}^2$}\\
                |x|+|y| \qquad \text{altrimenti}
            \end{cases}
        \end{gather*}
                $d$ è una metrica\\
        \begin{multicols}{2}
            \noindent
                \begin{gather*}
                    d:X \times X \to \mathbb{R} , d(x,y) \geq 0m = 0 \Leftrightarrow 
                \end{gather*}
            \columnbreak
            \begin{itemize}
                \item se $x,y,0$ sono collineari 
                \begin{gather*}
                    |x-y|=0\\
                    y=kx \quad \exists k \in \mathbb{R} \Rightarrow |x-y| = |x-kx| = |1-l| |x|\\
                    \Rightarrow \text{vale} |x-y| = 0 \Leftrightarrow |1-k| |x| = 0 \Leftrightarrow k=1 \text{ cioè} y \equiv x
                \end{gather*}
                \item se $x,y,0$ non sono collineari
                \begin{gather*}
                    d(x,y) = |x| + |y| = 0 \Leftrightarrow |x| = |y| = 0 \Leftrightarrow x = y = 0
                \end{gather*}
                \item devo dim. che $\forall x,y,z \in \mathbb{R}^2$ si ha $d(x,y) \leq d(x,z) + d(z,y)$\\
                \fbox{1} $0,x,y$ collineari\\
                    $d(x,y) = |x-y|$\\
                    sia $z \in \mathbb{R}^2$ \underbar{se} \fbox{$x,y,z,0$ collineari}\\
                    \underbar{allora} $d(x,z) +  d(z,y) = |x-z|+|z-y|$\\
                    voglio dim.
                    \begin{gather*}
                        d_\epsilon(x,y) =|x-y|\leq |x-z| + | z-y| = d_\epsilon (x,z) + d_\epsilon (z,y)\\
                        \text{vero perchè vale la dis. triang.}
                    \end{gather*}
                    \fbox{2} non collineari con $0,x,y$\\
                    $d(x,z) + d(z,y) = |x|+ |z| + |z|+ |y|$\\
                    voglio dim.
                    \begin{gather*}
                        |x-y| =\leq |x| + 2 |z| + |y| 
                    \end{gather*}
                    Uso la dis. triang. per $d_\epsilon$
            \end{itemize}
        \end{multicols}
    
         
    \end{example}

    \section{2/10/25}
    richiamo del th. di $\exists!$ di cauchy (locale)\\
    si ricorda:

\begin{tikzpicture}
        \draw[->](-1, 0) -- (4, 0) node[at end, below] {$x$};
        \draw[->](0, -1) -- (0, 3) node[at end, left] {$y$};
        \draw[|-|](0.3, 0.2) -- (0.3, -0.2) node[at end, below] {};
        \draw[|-|](2.3, 0.2) -- (2.3, -0.2) node[at end, below] {};
        \draw[|-|](0.2, 0.6) -- (-0.2, 0.6) node[at end, below] {};
        \draw[|-|](0.2, 1.5) -- (-0.2, 1.5) node[at end, below] {};
        \draw[<->](0.2, -0.7) -- (2.3, -0.7) node[midway, below] {$I$}; 
        \draw[<->](-0.4, 0.6) -- (-0.4, 1.5) node[midway, left] {$J$}; 
        \draw(0.3, 0.6) -- (0.3, 1.5);
        \draw(0.3, 0.6) -- (2.3, 0.6);
        \draw(0.3, 1.5) -- (2.3, 1.5);
        \draw(2.3, 0.6) -- (2.3, 1.5);
        \filldraw [red] (1.3,1.05) circle (2pt);
        \node at(2,1.15){$(x_0,y_0)$};

    \end{tikzpicture}
    \begin{gather*}
        \exists a,b \ t.c.\\
        I=(x_0-a, x_0+a)\\
        J=(y_0-b, x_0+b)\\
    \end{gather*}
    considero inoltre

    \begin{proof}
        (della formulazione integrale)
        \begin{gather*}
            B = \{u(x) \in C^0(I_\delta) \ t.c. \ || u-y_0||_{C^0 (I_\delta)}\leq b\}\\
        \end{gather*}

    \hfil\\
    \underbar{\textbf{OSS}}\\
    \begin{gather*}
        B =\subseteq C^0 (I_\delta) B \text{è una palla chiusa}\\
        (B,||\cdot||_{C^0}) \text{è sp. di BANACH poichè sottoinsieme completo di} (C^0(I_\delta, || \cdot ||_{C^0})) \text{che è completo}
    \end{gather*}

    \begin{gather*}
        F:B \to C^0(I_\delta) \quad F(u) = z \text{ con } z(x) = y_0 + \int_{x_0}^{x} f(t,u(t)) dt \quad \forall x \in I_\delta
    \end{gather*}
    
    \hfil\\
    \underbar{\textbf{OSS}}\\
    Immagine di $F \subseteq B$\\
    Infatti soa $u \in B$ allora $F(u) \in C^0 (I_\delta)$ inoltre:
    \begin{gather*}
        |F(u) -y_0| = |z(x) -y_0| \leq |\int_{x_0}^{x} |f(t, u(t))| dt |\leq
    \end{gather*}

    \hfil\\
    \underbar{\textbf{OSS}}\\
    $\forall x \in I_\delta , t \in [x_0, x] \subseteq I$\\
    $u(t) \in J$ poichè $u \in B $ e quindi $||u-y_0||_{C^0} \leq b$ cioè $|u(x)-y_0| \leq b \quad \forall x \in I_\delta$\\
    quindi:
    \begin{gather*}
        (t, u(t)) \in I \times J \Rightarrow |f(t, u(t)) \leq M
    \end{gather*}
    \begin{gather*}
        \leq M |x_0-x| \delta \leq b \Rightarrow \underset{x \in I_\delta}{max} |F(u) - y_0| \leq b
    \end{gather*}
    Cioè
    \begin{gather*}
        ||F(u) - y_0||_{C^0} \leq b \Rightarrow F(u) \subseteq B
    \end{gather*}

    \hfil\\
    \underbar{\textbf{OSS}}\\
    \fbox{F è una contrazione} rispetto $||\cdot||_{C^0 (I_\delta)}$\\
    \begin{gather*}
        |F(y_1) - F(y_2)| \leq | \int_{x_0}^{x} \underset{\leq L |y_1-y_2| \leq L ||y_1-y_2||_{C^0 (I_\delta)} }{| f(t_1 y_1(t)) -f(t_1y_2(t)) |}  dt | \leq L |x - x_0| ||y_1-y_2||_{C^0 I_\delta} \leq L\delta ||y_1-y_2||_{C^0 (I_\delta)}\\
        ||F(y_1) - F(y_2)||_{C^0 (I_\delta)} = \underset{x \in I_\delta}{max} \left|F(y_1)-F(y_2)\right| \leq \underset{L \delta < 1}{L \delta} || y_1-y_2||_{C^0 (I_\delta)}\\
        \Rightarrow F\text{ è una contrazione}
    \end{gather*}

    
    \hfil\\
    \underbar{\textbf{OSS}}\\
    \begin{gather*}
        \delta < min \{a, \frac{b}{M}, \frac{1}{L}\}
    \end{gather*}

    \hfil\\
    \begin{gather*}
        F:B \to B , (\underset{\text{completo}}{B, || \cdot||_{C^0 (I_\delta)}})\\
        F \text{contrazione rispetto a } ||\cdot||_{C^0 (I_\delta)} \Rightarrow \text{(teh. contraz.)}\\
        \Rightarrow \exists! t = y(x):F(y) = y \text{cioè}\\
        y(x) = y_0 + \int_{x_0}^{x} f(t, y(t)) dt \quad \forall x \in I_\delta\\
        \text{cioè } y \text{ risolve Couchy integrale}
    \end{gather*}

        \end{proof}
    
        \subsection{Funzioni in 2 variabili}

            \begin{itemize}
            \item $A \subseteq \mathbb{R}^2$ aperto se $\forall (x_0, y_0) \in A \exists R>0 t.c.$\\
            $B_r(x_0,y_0) = \{(x,y) \in \mathbb{R}^2 : ||(x,1) -(x:0, y_0) <R \} \subseteq A$\\
            \item $E \subseteq \mathbb{R}^2$\\
                $\overline{E} = $ il più piccolo chiuso $\supseteq E$ (chiusura di $E$)\\
                $\overset{\circ}{E} = $ il più grande aperto $\subseteq E$ (interno)
            \item $D$ è un dominio se $D=\overline{A}$ con aperto $D= \overset{\circ}{D} \ \cup \in D$ 
            \end{itemize}
            $f:D \to \mathbb{R}$\\
            \begin{itemize}
                \item Grafico di $f=\{(x,y,z) \in \mathbb{R}^3: \underset{z = f(x,1)}{(x,y) \in D}\}$\\
                \item linee di livello $U_t = \{(x,y) \in D \ t.c. \ f(x,y) = t \}$
                \hfil\\
                \underline{\textbf{OSS.}} $\underset{t \in \mathbb{R}}{\cup} U_t = D$
                \hfil\\
                \underline{\textbf{OSS.}} Se $f$ ammette min/max allora:\\
                $max\{t: U_t \neq 0\} = \underset{(x,y) \in D}{max} f(x,y)$
                $min\{t: U_t \neq 0\} = \underset{(x,y) \in D}{min} f(x,y)$

                \item dominio naturale
            \end{itemize}
                \begin{example}
                    Prendiamo la seguente funzione
                    $f(x,y) = ln(1-x^2-y^2)$
            \begin{enumerate}[$i)$]
                \item Determinare il $D$ naturale (il dom naturale è il dominio più grande in cui la funzione è definita)
                \item Descrivere le linee di livello
                \item Trovare min/max di $f$ in $\mathbb{R}$ $\qquad \mathbb{Q} \{ |x| \leq \frac{1}{2} , |y| \leq  \frac{1}{2}\}$
            \end{enumerate}

            \hfil\\
            Punto $i) \ D=\{(x,y) \in \mathbb{R}^2: 1-x^2-y^2 >0\} = \{(x,y) \mathbb{R}^2: x^2+y^2<1\}$\\
            \hfil\\
            Punto $ii) \ f(x,y) = t \Leftrightarrow$ \fbox{$(x,y) \in D$} $\Leftrightarrow \ln(1-x^2-y^2) = t \Leftrightarrow (x,y) \in D$\\
            $1-x^2-y^2 = e^t \quad \forall t \in \mathbb{R} \Leftrightarrow
            \begin{cases}
                (x,y) \in D\\
                \underset{\forall t \in \mathbb{R}}{x^2+y^2=1-e^t}
            \end{cases}$\\
            \hfil\\
            \underline{\textbf{OSS.}}
            \begin{itemize}
                \item  se $t>0$ cioè $e^t > 1 \qquad U_t = 0$
                \item  se $t=0 \qquad U_t = \{ \ (0,0) \ \}$
                \item se $t<0 \qquad U_t =$ circ. di centro $(0,0)$ e raggio $R=\sqrt{1-e^t}$
            \end{itemize}
            ovvero $\lim_{t \to -\infty}(U_t) $
            \end{example}
            
            \subsection{limiti di funzioni in 2 variabili}
            prendiamo una funzione $f:D \to \mathbb{R} \quad  \underset{(x_0,y_0) \in \overline{D}}{D \subseteq \mathbb{R}^2}$
            \begin{gather*}
                \lim_{(x,y) \to (x_0,y_0)} f(x,y)= L
            \end{gather*}
            Significa che:
            \begin{gather*}
                ||(x,y) -(x_0,y_0)|| \to 0 \Rightarrow |f(x,y), L| \to 0
            \end{gather*}
            Cioè
            \begin{gather*}
                \forall \{\underline{v}\} \subseteq \mathbb{R}^2 : \underset{\underline{v}_n \neq P_0}{\underline{v}_n \to }P_0 = (x_0, y_0) \Rightarrow f(\underline{v}_n) \to L\\
                f(\underline{v}_n) = f((v_n)_1, (v_n)_2) \qquad \underline{v}_n = ((v_n)_1, (v_n)_2)\\
                \forall \epsilon > 0 \exists \delta > 0 \ t.c. \ \text{se } (x,y) \in B_\delta^{x_0,y_0} / \{x,y\} \Rightarrow |f(x) - L| <\epsilon    
            \end{gather*}

            \begin{definition}
                $(x_0,y_0) \in D f$ continua in $e_0 \equiv(x_0,y_0)$ se \\
                $\forall \{\underline{v_n}\} \subseteq D: \underleftarrow{v}_n \to P_0$ si ha $f(v_1) \to f(P_0)$\\
                Cioè
                \begin{gather*}
                    ||(x,y) - (x_0,y_0) || \to 0 \Rightarrow |f(x,y) - f(x_0,y_0) | \to 0\\
                    \lim_{(x,y) \to (x_0,y_0)} f(x,y) = f(x_0,y_0)
                \end{gather*}
            \end{definition}
            \begin{theorem}
                \hfil\\
                $K $ compatto $\subseteq \mathbb{R}^2, f:K\to \mathbb{R}, f \in C^0(K)$\\
                \begin{itemize}
                    \item (Weierstrass) $f$ assume min/max in $K$
                    \item (Cantor) $f$ è definita continua in $K$ Cioè
                     \begin{gather*}
                        \forall \epsilon >0 \exists \delta >0 \ t.c. \\
                        \text{\underline{se} } x_1,x_2 \in K ||x_1,x_2||<\delta \text{ \underline{allora} } |f(x_1)-f(x_2)|< \epsilon
                     \end{gather*}
                \end{itemize}
            \end{theorem}
            \begin{theorem}
                Valori intermedi\\
                    $D$ dominio convesso e limitato  $f:D\to \mathbb{R}$\\
                    $f \in C^0(D) \Rightarrow f $ assume tutti i valori compresi tra il min e il max
            \end{theorem}

            \hfil\\
            \underline{\textbf{OSS.}}\\
            \begin{gather*}
                \exists \lim_{(x,y) \to (x_0,y_0)}f(x,y) = L \Leftrightarrow \text{Il limite coincide ($=L$) qualsiasi sia il percorso di $(x,1) \to (x_0,y_0)$}\\
            \end{gather*}
            quindi
            se $\exists$ due percorsi $(x(t),y(t)) \ t.c. \  x(t)\to x_0 \text{ e } y(t)\to y_0$\\
            $t.c. \ f(x(t),y(t)) \to $ a due valori $\neq$, \underbar{allora} $\nexists \lim_{(x,y) \to (x_0,y_0)} f(x,y)$

            \newpage
            \section{7/10/25}
            \subsection{Introduzione più approfondita allo studio di funzioni in più variabili}
    \begin{tikzpicture}
        \draw[->](-1, 0) -- (4, 0) node[at end, below] {$x$};
        \draw[->](0, -1) -- (0, 3) node[at end, left] {$y$};
        \node at(2,1.15){$(x_0,y_0)$};
    \end{tikzpicture}
    \subsection{Derivate parziali}
    Per studiare queste funzioni le derivate normali non ci bastano più e dobbiamo introdurre cioò che si chiama derivata parziale:
        \begin{gather*}
            Ap \subseteq \mathbb{R}^2, f:A \to \mathbb{R}, \underset{P_0}{(x_0,y_0)} \in A\qquad z = f(x,y)\\ 
        \end{gather*}

    \begin{figure}[tbh]
    \centering
    \includesvg[width = 150 pt]{anal2}
    \caption{Grafico di $f$ generica in due variabili}\label{fig:testsvg}
    \end{figure}


    Esempio grafico di una funzione in due variabili: $xye^{x+2y-9x^2-9y^2}$\\
    \begin{center}
    \begin{tikzpicture}
    \begin{axis}[domain=-1:1,y domain=-1:1]
        \addplot3[surf] {x*y*exp(x+2*y-9*x^2-9*y^2)};
    \end{axis}
    \end{tikzpicture}
    \end{center}

    \begin{definition}
        Derivata parziale rispetto a $x$:
        \begin{gather*}
            \lim_{h \to 0}\frac{v(x_0+h)-v(x_0)}{h}\\
            \lim_{h \to 0}\frac{v(x_0+h,y_0)-v(x_0,y_0)}{h} = \frac{\partial f}{\partial x} (x_0,y_0) = f_x(x_0,y_0) \qquad \text{\fbox{**}}\\
        \end{gather*}
        Derivata parziale rispetto a $y$:
        \begin{gather*}
            \lim_{k \to 0}\frac{u(x_0+k)-u(x_0)}{k}\\
            \lim_{k \to 0}\frac{u(x_0+k,y_0)-u(x_0,y_0)}{k} = \frac{\partial f}{\partial y} (x_0,y_0) = f_y(x_0,y_0) \qquad \text{\fbox{*}}\\
        \end{gather*}
    \end{definition}

    \begin{definition}
        Se $\exists$ \fbox{*} e \fbox{**},finite , $f$ è detta derivabile e:
        \begin{gather*}
            \exists D f(x_0,y_0) = \nabla f(x_0,y_0) = (f_x(x_0,y_0),f_y (x_0,y_0))
        \end{gather*}
        in generale $D$ indica il simbolo di derivata rispetto sia ad $x$ che $y$
    \end{definition}

    \begin{observation}
    \begin{itemize}
        \item Le funioni elementari tra loro sono derivabili in $D$
        \item Valgono le regole di derivazione e l'algebra delle derivate
    \end{itemize}
    \end{observation}

    \begin{example}
        $f(x,y) = x e^{(x^2+y^2)}$
        \begin{gather*}
            \frac{\partial f}{\partial x} (0,0) = \lim_{h \to 0} \frac{f(h,0)-f(0,0)}{h} =  \lim_{h \to 0} \frac{h e^{h^2}-0}{h} = 1\\
            \frac{\partial f}{\partial x} (0,0) = \lim_{k \to 0} \frac{f(k,0)-f(0,0)}{k} = 0\\
        \end{gather*}
        $f$ è derivabile in $(0,0)$ e $Df(0,0) = (1,0)$
    \end{example}
    Cosa succede in $(x,y) \in \mathbb{R}^2$ generico?
    \begin{observation}
        $f$ è composizione eprodotto di funzioni derivabili in $\mathbb{R}^2$ quindi posso applicare le regole di derivazione:
        \begin{gather*}
            \frac{\partial f}{\partial x} (x,y) = \frac{\partial g}{\partial x} \theta + g\frac{\partial \theta}{\partial x}=\\
            1 e^{(x^2+y^2)} + x(2x)e^{(x^2+y^2)} = e^{(x^2+y^2)}(1+2x^2) \qquad (\frac{\partial g}{\partial x} = \frac{\partial(x)}{\partial x} = 1)\\
            \frac{ \partial \theta}{\partial x} = \frac{\partial}{\partial x}(e^{x^2+y^2}) = \frac{\partial }{\partial x}(x^2+y^2) e^{x^2+y^2}=2x e^{x^2+y^2}\\\\
            \frac{\partial f}{\partial x}= \cancel{\frac{\partial g}{\partial y}}\theta + g\frac{\partial \theta}{\partial y} = x \frac{\partial}{\partial y}(e^{(x^2+y^2)}=x2ye^{x^2+y^2}) \qquad (\frac{\partial}{\partial y}(e^{(x^2+y^2)})= \frac{\partial}{\partial y}(x^2+y^2)e^{x^2+y^2}= 2y e^{x^2+y^2})
        \end{gather*}            
    \end{observation}

        \begin{example}
            $f(x,y) = \sin(x,y)$\\
            $f$ derivabile $\forall (x,y) \in \mathbb{R}^2$\\
            $Df(x,y) = (\begin{cases*}
                y\cos(x,y)\\
                x\cos(x,y)
            \end{cases*})$
            \begin{gather*}
                f_x (x,y) = y\cos(x,y)\\
                f_y (x,y) = x\cos(xy)
            \end{gather*}
                $Df \parallel (1,0) \Leftrightarrow \exists k \in \overset{k \neq 0}{\mathbb{R}} \ t.c. \\$
                $Df (x,y) = k(1,0)$
        \end{example}
        \subsection{Derivate di ordine successivo}
        le derivate successive di funzioni in due variabili crescono di numero con la derivazione: avrò due derivate prima e quattro derivate seconde (e così via).\\
        \begin{gather*}
            \frac{\partial^2}{\partial x^2} f(x,y) = \frac{\partial}{\partial x} (\frac{\partial^2}{\partial x^2} f(x,y))\\
            \frac{\partial^2}{\partial y\partial x} f(x,y) = \frac{\partial}{\partial y} (\frac{\partial^2}{\partial x^2} f(x,y))\\
            \frac{\partial^2}{\partial x \partial y} f(x,y) = \frac{\partial}{\partial x} (\frac{\partial}{\partial y} f(x,y))\\
            \frac{\partial f}{\partial y^2} f(x,y) = \frac{\partial}{\partial y} (\frac{\partial f }{\partial y}(x,y))
        \end{gather*}
        Si intro duce così la matrice \underbar{HESSIANA}
        \begin{gather*}
            \begin{bmatrix}
                \frac{\partial^2 f}{\partial x^2} (x,y) & \frac{\partial^2 f}{\partial x \partial y} (x,y)\\
                \frac{\partial^2 f}{\partial y \partial x} (x,y) & \frac{\partial^2 f}{\partial y^2} (x,y) 
            \end{bmatrix}
        \end{gather*}
        \begin{observation}
            \begin{gather*}
                Tr(D^2f) = \nabla f \text{ detto \underbar{laplaciano}}
            \end{gather*}
        \end{observation}
        \begin{theorem}[di Schwarz]
            $A_{ap} \subseteq \mathbb{R}^2, (x_0,y_0) \in A , f_A \to \mathbb{R}$ 2 volte derivabile\\
            \underbar{se} $f_{xy}, f_{yx}$ sono \underbar{continue} in $x_0,y_0$ \underbar{allora} $f_{yx}(x_0,y_0) = f_{xy}(x_0,y_0)$\\
            Da cui: \underbar{se} $f \in C^2(A)$ \underbar{allora} $D^2f$ è continua
        \end{theorem}
        \begin{observation}
            l'Hp. di continuità NON può essere rimossa\\
            Infatti $f(x,y) = \begin{cases}
                \frac{x^3y-xy^3}{x^2+y^2} \qquad \text{\underbar{se}} (x,y) \neq (0,0)\\
                0 \qquad \text{\underbar{se}} (x,y) = (0,0)\\
            \end{cases}$\\
            Ha scritto sta merda dopo aver riempito 2 lavagne di cose che non ho capito perchè si fanno\\
            $(f \in C^0(\mathbb{R}^2))$\\
            $\underset{f_{yx} (0,0)}{\exists f_{xy} (0,0)} \quad f_{xy} (0,0) \neq f_{yx}(0,0)$
        \end{observation}
        \newpage
        \section{9/10/25}
        \subsection{Differenziabilità}
        \begin{definition}
            f è differenziabile in $(x_0,y_0) \in A$ se
            \begin{itemize}
                \item è derivabile in $(x_0,y_0)$
                \item $\lim_{(h,k) \to (0,0)} \frac{f(x_0+h,y_0+k)-f(x_0,y_0) - <Df(x_0,y_0);(h,k)>}{\sqrt{h^2+k^2}}=0$ 
            \end{itemize}
            cioè $\lim_{\underline{\Sigma} \to \underline{0}} \frac{(\underline{x_0} + \underline{\Sigma}) -f(x_0)-<Df(x_0);\underline{\Sigma}>}{|\underline{\Sigma}|} = 0$\\
            cioè $f(\underset{\underline{x} = (x,y)}{\underline{x_0} + \underline{\Sigma}}) - f(\underline{x_0}) - \left\langle Df(\underline{x_0}); \underline{\Sigma}\right\rangle  \underset{per \underline{\Sigma} \to \underline{0}}{=} o(|\underline{\Sigma}|)$\\
            cioè \fbox{$f(x,y) = f(x_0,y_0)+ \left\langle D f(x_0,y_0);\begin{pmatrix} x-x_0 \\ y-y_0 \end{pmatrix}\right\rangle  + o\left(\left\lVert \begin{pmatrix} x-x_0 \\ y-y_0 \end{pmatrix} \right\rVert\right)  \quad \text{ per } (x,y) \to (x_0,y_0)$}

            \hfil\\
            Per vederlo con $x_0$ e $y_0$:\\
            \begin{gather*}
                \lim_{(x,y) \to (x_0,y_0)} \frac{f(x,y) -P(x,y)}{\sqrt{(x-x_0)^2+(y-y_0)^2}} = 0
            \end{gather*}
            Dove $P(x,y)$ è il polinomio di Taylor in due variabili e il denominatore è la norma tra le coordinate $(x_0,y_0)$ che è dove centro il mio polinomio e $(x,y)$ generici.\\
            Posso espandere $P(x,y)$ e scrivere:
            \begin{gather*}
                \lim_{(x,y) \to (x_0,y_0)} \frac{f(x,y) -f(x_0,y_0) - a(x-x_0) - b(y-y_0)}{\sqrt{(x-x_0)^2+(y-y_0)^2}} = 0
            \end{gather*}
            Dove $a$ e $b$ sono:
            \begin{gather*}
                a = \frac{\partial f}{\partial x}(x_0,y_0)\\
                b = \frac{\partial f}{\partial y}(x_0,y_0)
            \end{gather*}
            Quindi le derivate parziali di $x$ e $y$ calcolate in $(x_0,y_0)$.\\
            Se ora ci riconduciamo alla forma con $h$ e $k$ si ha:
            \begin{gather*}
                \lim_{(h,k) \to (x_0,y_0)} \frac{f(x_0 +h,y_0 +k) -f(x_0,y_0) - ah - bh}{\sqrt{h^2+k^2}} = 0
            \end{gather*}
            Da qui si può costruire il differenziale primo:
            \begin{gather*}
                f(x,y) \approx f(x_0,y_0) + \frac{\partial f}{\partial x}(x-x_0) + \frac{\partial f}{\partial y}(y-y_0)
            \end{gather*}
        \end{definition}
        \begin{observation}
            f diff. in $(x_0,y_0) \Rightarrow $ f continua in $(x_0,y_0)$\\
        \end{observation}
        \begin{proof}
            \begin{gather*}
                \lim_{(x,y) \to (x_0,y_0)} f(x,y) = \lim_{(x,y) \to (x_0,y_0)} f(x_0,y_0) - Df(x_0,y_0);\begin{pmatrix} x-x_0 \\ y-y_0 \end{pmatrix}  + o\left(\underset{\to 0}{\begin{pmatrix} x-x_0 \\ y-y_0 \end{pmatrix}} \right) = f(x_0,y_0)\\
                \left\lvert Df(x_0,y_0);\begin{pmatrix} x-x_0 \\ y-y_0 \end{pmatrix}\right\rvert  \leq \left\lvert Df(x_0,y_0)\right\rvert  \cdot  \left\lVert Df(x_0,y_0);\begin{pmatrix} x-x_0 \\ y-y_0 \end{pmatrix}\right\rVert  \\
            \end{gather*}
        \end{proof}
        \begin{observation}
            $f$ diff $\Rightarrow f$ si può approssimare (al 1° ordine) ad una funzione affine (un piano) vicino ad $(x_0,y_0)$\\
            \begin{gather*}
                \Pi_{tg}: z= f(x_0,y_0) + \left\langle Df(x_0,y_0);\begin{pmatrix} x-x_0 \\ y-y_0 \end{pmatrix}\right\rangle 
            \end{gather*}
            definisce il piano $tg$ (tangente) al grafico di $z= f(x,y) $ in $ (x_0,y_0, f(x_0,y_0))$\\
        \end{observation}
        \subsection{equazioni cartesiane e parametriche}
        sia $\textcolor{yellow}{r_{y_0}}$ la retta $tg$ di grafico della sezione trasversale $u(x) = f(x,y_0)$ in $ (x_0,y_0)$\\
        e sia $\textcolor{orange}{r_{x_0}}$ la retta $tg$ di grafico della sezione trasversale $v(y) = f(x_0,y)$ in $ (x_0,y_0)$\\  
        
        Si possono scrivere in froma sia cartesiana sia parametrica
        \begin{gather*}
            \textcolor{yellow}{r_{y_0}}
            \begin{cases}
                z = f(x_0,y_0) + f_x (x_0,y_0) (x-x_0)\\
                y = y_0
            \end{cases}
            \Leftrightarrow \begin{pmatrix}
                x-x_0\\
                y-y_0\\
                z-z_0
            \end{pmatrix} = 
            \begin{pmatrix}
                1\\
                0\\
                f_x(x_0,y_0)
            \end{pmatrix}t \quad  \forall t \in \mathbb{R}
            \\\textcolor{orange}{r_{x_0}}
            \begin{cases}
                z = f(x_0,y_0) + f_y (x_0,y_0) (y-y_0)\\
                x = x_0
            \end{cases}
            \Leftrightarrow \begin{pmatrix}
                x-x_0\\
                y-y_0\\
                z-z_0
            \end{pmatrix} = 
            \begin{pmatrix}
                0\\
                1\\
                f_y(x_0,y_0)
            \end{pmatrix}s \quad \forall s \in \mathbb{R}
        \end{gather*}   
        
        Da qui possiamo trovare il piano:
        \begin{gather*}
                \Pi : \begin{pmatrix}
                    x-x_0\\
                    y-y_0\\
                    z-z_0
                \end{pmatrix}= \begin{pmatrix}
                    1\\
                    0\\
                    f_x(x_0,y_0)
                \end{pmatrix}t + \begin{pmatrix}
                    0\\
                    1\\
                    f_y(x_0,y_0)
                \end{pmatrix}s \quad \forall t,s \in \mathbb{R} \\
                z = <Df(x_0,y_0);\begin{pmatrix}
                    x-x_0\\
                    y-y_0
                \end{pmatrix}> + \underset{\overset{||}{f(x_0,y_0)}}{z_0}
        \end{gather*}
        
        i vettori direzione sono linearmente indipendenti e generano un piano uguale al piano tangenziale a:
        \begin{gather*}
            \Pi_{P_0}: z{=}f(x,y) \text{ in } (x_0,y_0)
        \end{gather*}
        Dove $P_0=(x_0,y_0)$ ed è il punto in cui è calcolato il piano tangente
        Piano tangenziale:
        \begin{gather*}
            f(x,y) = f(x_0,y_0) + <Df(x_0,y_0); \begin{pmatrix} x-x_0 \\ y-y_0 \end{pmatrix}> + o\left( \left\lVert \begin{pmatrix} x-x_0 \\ y-y_0 \end{pmatrix} \right\rVert  \right) \qquad \text{per } (x,y) \to (x_0,y_0)
        \end{gather*}

        \begin{theorem}[del differenziale]
                \underbar{Sia} $f:A \to \mathbb{R}$\\ 
                \underbar{sia}$A_{ap} \subseteq \mathbb{R}^2 , (x_0,y_0) \in A$ \underbar{e} $f$ derivabile in $A$\\
                \underbar{se} $f_x,f_y \in C^0 (|x_0,y_0|)$\\ 
                \underbar{allora} $f$ è differenziabile in $(x_0,y_0)$
        \end{theorem}
        \begin{corollary}
            $f \in C^1(A) \Rightarrow f$ differenziabile in $A(\Rightarrow f \in C^0(A))$
        \end{corollary}
        \begin{proof}
            \begin{gather*}
                f(x_0+h,y_0+k) = \underset{u(x_0+h) -u(x_0)}{f(x_0+h,y_0+k)-f(x_0,y_0+k)}+\underset{v(y_0+k)-v(y_0)}{f(x_0+y_0+k) -f(x_0,y_0)} = h f_x(x_1,y_0+k) +k f_y(x_0,y_1)
            \end{gather*}
            Per lagrange :
            \begin{gather*}
                \exists x_1 \in (x_0,x_0+h) = u'(x_1)h = f_x(x_1,y_0+k)h\\
                \exists y_1 \in (y_0,y_0+k) = v'(y_1)k = f_y(x_0,y_1)h\\
            \end{gather*}
            considero
            \begin{gather*}
                |\frac{f(x_0+h,y_0+k)-f(x_0',y_0)-<Df(x_0,y_0);\begin{pmatrix} h \\ k \end{pmatrix}>}{\sqrt{h^2+k^2}}| =\\
                |\frac{hf_x(x_1,y_0)+kf_y(x_0,y_1)-f_x(x_0,y_0)h-f_y(x_0,y_0)k}{\sqrt{h^2+k^2}}| =\\
                |\frac{h(f_x(x_0,y_0+k)-f_x(x_0,y_0))+k(f_y(x_0,y_0)-f_y(x_0,y_0))}{\sqrt{h^2+k^2}}| \leq\\
                \frac{|h|}{\sqrt{h^2+k^2}} |f_x(x_1,y_0+k) -f_x(x_0,y_0) + \frac{|k|}{\sqrt{h^2+k^2}} |f_y (x_0,y_0) -f_y(x_0,y_0)| \leq\\
                \left\lvert f_x(x_1,y_0+k) - f_x(x_0,y_0) \right\rvert + \left\lvert f_y(x_0,y_1) - f_y(x_0,y_0) \right\rvert 
            \end{gather*}
            Passo al limite: $\lim_{(h,k) \to (0,0)}$ \underline{Allora}:
            \begin{gather*}
                x_1 \in (x_0,x_0+h) \to x_0 \\
                y_1 \in (y_0,y_0+h) \to y_0 \\
            \end{gather*}
            da cui:
            \begin{gather*}
                \lim_{h,k \to (0,0)} f_x (x_1,y_0 +k) = f_x(x_0,y_0)
                \lim_{h,k \to (0,0)} f_y (x_0,y_1) = f_y(x_0,y_0) 
            \end{gather*}
            Poichè per hp. $f_x,f_y$ sono continue in $(x_0,y_0)$\\
            si ottiene:
            \begin{gather*}
                \left| \frac{f(\underline{x_0} + \underline{\Sigma}) - <Df(\underline{x_0}); \underline{\Sigma}>}{ | \underline{\Sigma} |} \right| \leq \text{\fbox{*}} = \cancel{\left| f_x(x_0,y_0)-f_x(x_0,y_0) \right|} + \cancel{\left| f_y(x_0,y_0)-f_y(x_0,y_0) \right|}\\
                \text{\fbox{*}}
                \begin{cases}
                    \lim_{\underline{\Sigma} \to 0} \left| f_x(x_1,y_0+k)-f_x(x_0,y_0) \right| + \left| f_y(x_0,y_1)-f_y(x_0,y_0) \right|
                \end{cases} 
            \end{gather*}
        \end{proof}
        \begin{definition}
            di DIFFERENZIALE\\
            $f$ differenziabile\\
            $df = <Df;dx>$\\
            $df = <Df(x_0,y_0) ; \begin{pmatrix} x-x_0 \\ y-y_0 \end{pmatrix}>$ 
        \end{definition}
        \subsection{derivate direzionali}
    
        \begin{figure}[tbh]
        \centering
        \includesvg[width = 150 pt]{anal3}
        \caption{}\label{fig:testsvg}
        \end{figure}
    
        la linea verde è:
        \begin{gather*}
            \begin{pmatrix}
                x(t)\\
                y(t)
            \end{pmatrix} = \underline{v} t + P_0 \begin{pmatrix}
                v_1t+x_0\\
                v_2t +y_0
            \end{pmatrix}\\
            \forall t \in \mathbb{R}\\
             |\underline{v}|   = 1 \quad \underline{v} = (v_1,v_2)
        \end{gather*}
        
        \begin{gather*}
            \underline{v} \in \mathdollar ' = (\cos(\theta) , \sin(\theta)) \quad \exists \theta \in [0, 2\pi]
        \end{gather*}
        $\$'$ è dunque lo spazio di vettori unitari che vivono nel piano $x,y$ e "centrati nell'origine", in pratica tutte le direzioni possibili lungo la quale è possibile considerare una derivata direzionale.

        \begin{definition}
            Derivata direzionale di $f$ lungo la direzione $\underline{v}$ il cui modulo è $| \underline{v} | = 1$ in $P_0$, la derivata di $u(t)$ in $t=0$ cioè:
            \begin{gather*}
                \frac{\partial f}{\partial \underline{v}} (x_0,y_0) = \lim_{h \to 0} \frac{\omega(h) -\omega(0)}{h} = \\
                \lim_{h \to 0} \frac{f(\underline{v} h + P_0) - f(P_0)}{h}
            \end{gather*}
        \end{definition}
        \begin{definition}
            Def chiara derivata direzionale:
            Prendiamo una funzione che va da un insieme aperto di $\mathbb{R}^2$ ($\Omega$) 
            \begin{gather*}
                f(x,y):\Omega \subseteq \mathbb{R}^2
            \end{gather*}
            E un punto di questo insieme $\Omega$:
            \begin{gather*}
                p = (x_0,y_0) \in \Omega
            \end{gather*}
            Prendiamo ora un vettore generico e otteniamo da questo un versore dividendo per la sua norma:
            \begin{gather*}
                \underline{v}=(a,b) \qquad \text{Una norma di un vettore è: }\left\lVert \underline{v}\right\rVert = \sqrt{a^2+b^2}\\
                \underline{u} = (\frac{a}{\left\lVert \underline{v}\right\rVert },\frac{b}{\left\lVert \underline{v}\right\rVert }) \qquad \underline{u} = (a_u,b_u)
            \end{gather*}
            Si definisce quindi la derivata direzionale come:
            \begin{gather*}
                f_{\underline{u}}'(x_0,y_0) = \lim_{h \to 0} \frac{f(x_0+a_uh,y_0+b_uh)-f(x_0,y_0)}{h}
            \end{gather*}
        \end{definition}
        \begin{observation}
            \begin{gather*}
                \underline{v} = (1,0) \Rightarrow \frac{\partial f}{\partial \underline{v}} = f_x\\
                \underline{v} = (0,1) \Rightarrow \frac{\partial f}{\partial \underline{v}} = f_y\\
            \end{gather*}
        \end{observation}
        \begin{theorem}
            del Gradiente\\
            \underbar{sia} $f:A \to \mathbb{R} \quad A_{ap} \subseteq \mathbb{R}^2,\underline{x_0} \in A$\\
            \underbar{e sia} $f$ differenziabile in $\underline{x_0}$\\
            \underline{allora}:
            \begin{gather*}
                \forall \underline{ \nu} \in \mathdollar ' \ \ \ \ \exists \frac{\partial f}{\partial \underline{\nu}}(\underline{x_0})
            \end{gather*}
            e vale:
            \begin{gather*}
                \frac{\partial f}{\partial \underline{ \nu}}(\underline{x_0}) = \left\langle Df(\underline{x_0});\underline{\nu}\right\rangle 
            \end{gather*}
            Quest'ultima è detta \underbar{formula del gradiente}
        \end{theorem}
        \begin{proof}(e corollario th. gradiente)
            \begin{gather*}
                \frac{\partial f}{\partial \underline{\nu}}(\underline{x_0}) = \left\langle \left\lvert  Df(x_0);\underline{\ni} \right\rvert  \right\rangle  \underset{= \text{ sse} \underline{\nu} \parallel Df(\underline{x_0})}{\leq} \left\lvert Df(\underline{x_0})\right\rvert \cdot \underset{\underset{0}{||}}{|\underline{\nu}|}            \end{gather*}
        In particolare se verso concorde:
        \begin{gather*}
            \frac{\partial f}{\partial \underline{\nu}}(\underline{x_0}) = \left\lvert Df(\underline{x_0})\right\rvert 
        \end{gather*}
        e verso discorde
        \begin{gather*}
            \frac{\partial f}{\partial \underline{\nu}}(\underline{x_0}) = -\left\lvert Df(\underline{x_0})\right\rvert 
        \end{gather*}
        \end{proof}

        \begin{proof}
            (Formula del gradiente per $\frac{\partial f}{\partial \underline{\nu}}$)\\
            \begin{gather*}
                \frac{\partial f}{\partial \underline{\nu}} = \lim_{h \to 0} \frac{f(\underline{x_0}+h \nu)-f(\underline{x_0})}{h} = \lim_{h \to 0}\frac{F(h)-F(0)}{h} = F'(0) = \left\langle Df(\underline{x}(0)); \underline{\dot{x}}(0) \right\rangle = \left\langle Df(\underline{x_0});\underline{\nu} \right\rangle\\
                F(t) = f\overset{\overset{\underline{x}(t)}{||}}{(\underline{x_0}+t \underline{\nu})} \quad \text{ è funzione composta}\\
                \underline{x}(t) = \underline{x:0} + t \underline{\nu} \qquad \underline{\dot{x}}(t) = \underline{\nu} \qquad \underline{x}(0) = \underline{x_0}  
            \end{gather*}
        \end{proof}

        \begin{theorem}
            Direzione di massima pendenza
            \underbar{sia} $f:A \to \mathbb{R} \quad A_{ap} \leq \mathbb{R}^2,\underline{x_0} \in A$\\
            \underbar{e sia} $f$ differenziabile in $\underline{x_0}$\\
            \underline{allora}:
            \begin{itemize}
                \item $Df(\underline{x_0})$ indica la direzione (e verso) di massima crescita di $f$
                \item $Df(\underline{x_0})$ indica la direzione (e verso) di minima crescita di $f$
                \item $\forall \underline{\nu} \in \mathdollar' , \underline{\nu} \perp Df(\underline{x_0}) = 0$
            \end{itemize}
        \end{theorem}
        \subsection{funzioni composte}
        $I=(a,b) \subseteq \mathbb{R}$
        \begin{definition}
            $\underline{x}(t):I \to \mathbb{R}^3(\in \mathbb{R}^2)$ è una curva parametrica\\
            \underbar{se} $\underline{x}(t) = \begin{pmatrix} x(t) \\ y(t) \\ z(t) \end{pmatrix} \qquad x,y,z:I \to \mathbb{R}$,\\
            $x,y,z$ derivabili in $I$ $ \dot{\underline{x}}(t) = \begin{pmatrix} \dot{x} \\ \dot{y} \\ \dot{z} \end{pmatrix}, |\dot{\underbar{x}}(t)| \neq 0 \quad \forall t \in I $
        \end{definition}
        \begin{definition}
            Funzione composta\\
$            F(t) = f(\underline{x}(t)) = f(x(t),y(t))\\
            f:A \to \mathbb{R} \quad A_{ap} \subseteq \mathbb{R}^2\\
            \underline{x}(t) \text{curva piana} \qquad \underline{x}(t):I \to A$
        \end{definition}
        \begin{theorem}
            di derivazione delle funzioni composte $I=(a,b)$\\
            $\underline{x}(t):I \to A_{ap} \subseteq \mathbb{R}^2, \underline{x}$ derivabile in $\underset{\in I}{t_0},f:A \to \mathbb{R}$\\
            $f$ differenziabile in $\underline{x}(t_0) = (x_0,y_0)$\\
            \underline{Allora} $F(t) = f(\underline{x}(t)) = f(x(t),y(t))$ è derivabile in $t_o \in I$ e $F'(t_o) = \left\langle Df(\underline{x}(t_o));\underline{\dot{x}}(t_0)\right\rangle $
        \end{theorem}

        \begin{proof}
            \hfil\\
            chiamo $t = t_0 \in I$\\
            \begin{itemize}
                \item f differenziabile $\Rightarrow f(\underline{y}) = f(\underline{x}) + \left\langle Df (\underline{x}); \underline{y}- \underline{x} \right\rangle + o(\left\lvert \underline{y} - \underline{x} \right\rvert) \quad \text{per } \underline{y} \to \underline{x}$\\
                \underline{sia} $\underline{x} = \underline{x}(t) \qquad \underline{y} = \underline{x}(t+h), \quad t+h \in I$\\
                \textcolor{green}{\fbox{\textcolor{white}{$f(\underline{x}(t+h))-f(\underline{x}(t)) = \left\langle Df(\underline{x}(t));\underline{x}(t+h)-\underline{x}t \right\rangle +o(\left\lvert \underline{x} (t+h) -\underline{x}(t)\right\rvert ) \quad\text{per } h \to 0 $ }}}  \textcolor{green}{\fbox{*}}\\
                \item $\text{\fbox{$\underset{\text{\fbox{*}}}{\frac{F(t+h)-F(t)}{h}}$}} \overset{\text{\textcolor{green}{\fbox{*}}}}{=} \frac{f(\underline{x}(t+h))-f(\underline{x}(y))}{h} = \textcolor{orange}{\text{\fbox{\textcolor{white}{$\underset{\textcolor{orange}{\text{\fbox{I}}}}{\frac{\left\langle Df(\underline{x}(t));\underline{x}(t+h)\right\rangle }{h}}$}}}} + \textcolor{magenta}{\text{\fbox{$\textcolor{white}{\underset{\text{\textcolor{magenta}{\fbox{II}}}}{\frac{o(\left\lvert \underline{x}(t+h) -\underline{x}(t)\right\rvert )}{h}}}$}}}$ \\
                \item passo a $\lim_{h \to 0}$ Voglio dimostrare che $\exists \lim_{h \to 0} \text{\fbox{*}} = 0 \ , \ \lim_{h \to 0} \text{\textcolor{magenta}{\fbox{II}}} = 0 \ , \ \text{\textcolor{orange}{\fbox{I}}} = \left\langle Df(\underline{x}(t));\underline{\dot{x}}(t) \right\rangle $
            \end{itemize}
            \bulletout\textcolor{orange}{\fbox{$I$}} $\lim_{h \to 0} \left\langle Df(\underline{x}(t));\frac{\underline{x}(t+h)}{h}\right\rangle$\\
            $=\lim_{h \to 0} f_x \underset{\to \dot{x}(t)}{\text{\fbox{$(\frac{ \underline{x}(t))(x(t+h)-x(t)}{h})$}}} + f_y(\underline{x}(t))\underset{\to \dot{y}(t)}{\text{\fbox{$(y(t+h)-y(t))$}}}$\\
            $= f_x(\underline{x}(t)) \dot{x}(t)+f_y(\underline{x}(t))\dot{y}(t)$\\
            $=\left\langle Df(\underline{x}(t));\underline{\dot{x}}(t)\right\rangle $\\
            \bulletout \textcolor{magenta}{\fbox{$II$}} $\lim_{h \to 0} \left\lvert  \text{\textcolor{magenta}{\fbox{$II$}}} \right\rvert  = \lim_{h \to 0} \underset{\to 0}{\left\lvert \frac{o( (\underline{x}(t+h)-\underline{x}(t) ) )}{ \left\lvert \underline{x}(t+h)-\underline{x}(t) \right\rvert  }\right\rvert} \underbrace{\frac{\left\lvert \underline{x}(t+h)-\underline{x}(t)\right\rvert }{\left\lvert h \right\rvert }}_{\lim \exists \text{ finito per \fbox{**}}} = 0$\\
            \fbox{**}: $ \ \lim_{h \to 0} \sqrt{\underbrace{\frac{(x(t+h)-x(t))^2}{h^2}}_{\underset{\to (\dot{x}(t))^2}{(\frac{x(t+h)-x(t)}{h})^2}} + \underset{\to (\dot{y}(t))^2}{\frac{(y(t+h)-y(t))^2}{h^2}} } = \sqrt{\dot{x}^2(t)+\dot{y}^2(t)} = |\underline{\dot{x}}(t)|$\\

            \bulletout di conseguenza:$ \ \Rightarrow \exists \underset{F'(t)}{\lim_{h \to 0}\frac{F(t+h)-F(t)}{h}} = \left\langle Df(\underline{\dot{x}}(t))\right\rangle $
        \end{proof}

        \hfil\\\hfil\\\hfil\\\hfil\\\hfil\\\hfil\\
        \section{14/10/25}
        \subsection{formula di Taylor}(in più variabili)
        $A_{ap} \in \mathbb{R}^2 f:A\to \mathbb{R}$\\
        $\underline{x} \in A \quad [\underline{x}, \underline{y}] \text{segmento } \subseteq A \quad \underline{x}(t) = \underline{x} + t \underline{h} , t \in [0,1]$\\
        $ \underline{y} \in A, \ \underline{y} = \underline{x} + \underline{h} \quad \underline{h} = (h_1,h_2) \neq 0$\\\\
        \textbf{FORMULA DI TAYLOR} (I° ordine con resto di Peano)\\
        $f \in C^1(A) f(\underline{x}+\underline{h}) = f(\underline{x}) + \left\langle Df(\underline{x}), \underline{h}\right\rangle  + o(|\underline{x}|)  $\\
        \textbf{I° ordine con resto di Lagrange}\\
        $f \in C^1(A) \quad \exists\theta \in (0,1) \ t.c. \ f(\underline{x} + \underline{h}) = f(\underline{x}) + \left\langle Df(\underline{x} + \theta h); \underline{h}\right\rangle $\\
        \begin{observation}
            $\theta = \textacutedbl(\underline{x}, \underline{h})$\\
        \end{observation}

        \begin{proof}
            (resto di lagrange)\\
            \begin{gather*}
                F(t) = f(\underbar{x} + t \underline{h}) = f(x+th_1;y+th_2)\\
                F \in C^1 ((0,1)) \qquad F'(t) = \left\langle Df(\underline{x} + t\underbar{h}) ; \underline{h} \right\rangle\\
                \text{Th. di Lagrange a } F \Rightarrow \exists \theta \in (0,1): F(1) = F(0) + F'(\theta)(1-0)\\
                \text{cioè: }f(\underline{x} + \underline{h} = f(\underline{x})) + \left\langle Df(\underline{x} + \theta \underline{h}); \underline{h} \right\rangle  
            \end{gather*}
        \end{proof}
        \begin{theorem}
            FORMULA DI Taylor(II° ordine con resto di lagrange)\\
            $f \in C^2(A)$ \underbar{allora} $\exists \theta \in (0,1) (\underbar{\textbf{OSS.}} \theta = \theta (\underline{x}, \underline{h})) t.c.$\\
            $F(\underline{x}+ \underline{h}) = f(\underline{x}) + \left\langle Df(\underline{x}+ \theta \underline{})\underline{h}; \underline{h}\right\rangle $
        \end{theorem}
        \begin{proof}
            \hfil\\
            \begin{gather*}
                F(t) = f(\underline{x} + \underline{h}) \in C^2(0,1)
            \end{gather*}
            Applico Taylor a: $F$ in $t=1 \quad \exists \theta \in (0,1) \ t.c.$\\
            \begin{gather*}
                F(1) = F(0) + F'(0) 1 + \frac{1}{2} F''(\theta)1^2\\
                f(\underline{x} + \underline{h}) = f(\underline{x}) + \left\langle Df(\underline{x});\underline{h} \right\rangle + \frac{1}{2}F''(\theta)\\
                \underset{\text{\fbox{\textcolor{orange}{*}}}}{=}  f(\underline{x}) + \left\langle Df(\underline{x});\underline{h} \right\rangle + \frac{1}{2}\left\langle D^2f(\underline{x} + t \underline{h})\underline{h} ; \underline{h}\right\rangle
            \end{gather*}
            devo Calcolare $F''(t) = (F'(t))' = (\underset{g(\underline{x}(t))}{f_x(\underline{x}+ t \underline{h})h_1+f_y(\underline{x} + t\underline{h})h_2})' = \left\langle Dg(\underline{x}(t)); \dot{\underline{x}}(t) \right\rangle $\\
            \begin{gather*}
                D(f_x) = \begin{pmatrix}
                    f_{xx}\\
                    f_{xy}
                \end{pmatrix}\\
                D(f_y) = \begin{pmatrix}
                    f_{yx}\\
                    f_{yy}
                \end{pmatrix} = \begin{pmatrix}
                    f_{xy}\\
                    f_{yy}
                \end{pmatrix}\\
                Dg = \begin{pmatrix}
                    f_{xx} h_1+ f_{xy}h_2\\
                    f_{xy} h_1 + f_{yy}h_2
                \end{pmatrix}
                = D(f_x)h_1 + D(f_y)h_2
            \underset{\text{calcolato in }\underline{x} +t \underline{h}}{=} (f_{xx} h_1 + f_{xy}h_2)h_1 +(f_{xy} h_1 + f_{yy}h_2)h_2\\
            = f_{xx} h_1^2 + 2f{xy}h_1 h_2 + f_{yy}h_2^2 \\
            = \left\langle \begin{pmatrix}
                f{xx} & f_{xy}\\
                f{xy} & f{yy}
            \end{pmatrix}
            \begin{pmatrix}
                h_1\\h_2
            \end{pmatrix};\begin{pmatrix}
                h_1\\h_2
            \end{pmatrix}\right\rangle \\
            = \left\langle D^2f(\underline{x} + t \underline{h})\underline{h} ; \underline{h}\right\rangle \text{\fbox{\textcolor{orange}{*}}} \quad \forall t \in (0,1)
            \end{gather*}
        \end{proof}

        \begin{theorem}
            (II° ordine con resto di Peano)\\
            $f \in C^2(A) \quad f(\underline{x}+ \underline{h}) = f(\underline{x}) + \left\langle Df(\underline{x}); \underline{h} \right\rangle + \text{\fbox{$\frac{1}{2}\left\langle D^2f(\underline{x})\underline{h};\underline{h} \right\rangle $}} + \underline{ | \underline{h} | ^2}$ 
        \end{theorem}
        \begin{proof}
            Dalla formula con il resto di Lagrange\\
             basta provare che\dots
            \begin{gather*}
                \lim_{\underline{h} \to \underline{0}} \frac{\left\langle D^2f(\underline{x} + \theta \underline{h})\underline{h} ; \underline{h} \right\rangle - \left\langle D^2f(\underline{x})\underline{h} ; \underline{h} \right\rangle }{|h|^2} = 0\\
                \frac{\left\lvert \left\langle D^2f(\underline{x} + \theta \underline{h}) \textcolor{cyan}{\fbox{\textcolor{white}{\underline{h}}}} ; \textcolor{magenta}{\fbox{\textcolor{white}{\underline{h}}}} \right\rangle - \left\langle D^2f(\underline{x})\textcolor{cyan}{\fbox{\textcolor{white}{\underline{h}}}} ; \textcolor{magenta}{\fbox{\textcolor{white}{\underline{h}}}} \right\rangle\right\rvert  }{|h|^2} \overset{\text{\fbox{\textcolor{green}{*}}}}{=} \qquad\qquad \text{\fbox{\textcolor{green}{*}}}\left\langle  A \textcolor{cyan}{v};\textcolor{magenta}{w}\right\rangle - \left\langle B \textcolor{cyan}{v};\textcolor{magenta}{w}\right\rangle = \left\langle (A-B)\textcolor{cyan}{v};\textcolor{magenta}{w}\right\rangle\\
                \frac{\left\lvert \left\langle [D^2 f(\underline{x} + \theta \underline{h}) - D^2 f(\underline{x})] \underline{h} ; \underline{h} \right\rangle \right\rvert }{|h|^2} \underset{\textcolor{yellow}{Cauchy-Schwarz}}{\leq} \frac{\left\lvert (D^2 f(\underline{x} + \theta \underline{h}) - D^2 f(\underline{x}))  \underline{h} \right\rvert \left\lvert \underline{h}\right\rvert }{|h|^2} \underset{\text{\fbox{\textcolor{orange}{*}}}}{\leq} \frac{\overbrace{\left\lvert D^2f(\underline{x}+ \theta \underline{h} - D^2(\underline{x}))\right\rvert}^{\to 0 \text{ per } \underline{h} \to \underline{0} \quad \text{\fbox{\textcolor{red}{*}}}} \cancel{|\underline{h}^2|}}{\cancel{|\underline{h}^2|}}\\
                \text{\fbox{\textcolor{orange}{*}}}: \left\lvert A \underline{v} \right\rvert \leq \left\lvert A \right\rvert \left\lvert \underline{v} \right\rvert \qquad \text{con }\left\lvert A \right\rvert \text{norma euclidea di }\mathbb{R}^{n\times m} =\sqrt{\sum_{i}^{n}\sum_{j}^{m} a^2_{ij} }   \\
                \text{\fbox{\textcolor{red}{*}}}: \text{poichè } D^2f\in C^0 \text{ e } \underline{x} + \theta\underline{h} \to \underline{x}
            \end{gather*}
        \end{proof}
        \begin{proposition}
            \begin{gather*}
                f(\underline{x}) = \underset{\text{Piano tangente}}{f(\underline{x_0}) \left\langle Df(\underline{x_0};(\underline{x} - \underline{x_0})) \right\rangle} + \underset{\text{termine II° ordine}}{\frac{1}{2} \left\langle D^2 f(\underline{x_0})(\underline{x}- \underline{x_0}); \underline{x}- \underline{x_0}\right\rangle  + \underset{\text{Resto}}{o(| \underline{x}- \underline{x_0} |^2)}}
            \end{gather*}
            \underline{Se} $D^2f(\underline{x_0})$ è definita $> 0$ \underline{allora} il termine di II° ordine più il resto $>0$ \underline{\textbf{localmente}} $\Rightarrow \ f(x) > $ Piano tangente localmente vicino a $x_0$\\
            \underline{Se} $D^2f(\underline{x_0})$ è definita $< 0$ \underline{allora} il termine di II° ordine più il resto $<0$ \underline{\textbf{localmente}} $\Rightarrow \ f(x) < $ Piano tangente localmente vicino a $x_0$\\ 
            \underline{se} è indefinita $\exists $ direzione $\underline{x} - \underline{x_0}$ per le quali $f(x) < $ piano tangente e altre per cui $f < $ piano tangente\\
        \end{proposition}
        \begin{example}
            Prendiamo $f(x,y) = \sin(2x+y^2) \quad P_0 \equiv (\frac{\pi}{3}, 0) \quad z_0 = f(P_0) = \sin (\frac{2\pi}{3}) = \frac{\sqrt{3}}{2}$\\
            d differenziabile in $\mathbb{R}^2$\\
            $Df(x,y) = \begin{pmatrix}
                2 \cos(2x + y^2)\\
                2y \cos(2x + y^2)
            \end{pmatrix}$\\
            $Df(P_0) = \begin{pmatrix}
                -1 \\ 0
            \end{pmatrix}$\\
            $\Pi_{tg in(P_0,z_0)}: z-z_0 = \left\langle Df(P_0); \underline{x} - P_0 \right\rangle \Rightarrow z -\frac{\sqrt{3}}{2} = \left\langle \begin{pmatrix} -1 \\ 0 \end{pmatrix}; \begin{pmatrix} x- \frac{\pi}{3} \\ y \end{pmatrix} \right\rangle  \Rightarrow z = -x +\frac{\pi}{3} + \frac{\sqrt{3}}{2}$ \\ 
            \text{... foto}
        \end{example}
        \begin{proposition}
            $Det A = \lambda_1 \lambda_2 > 0 \Rightarrow \lambda_i$ concordi\\
            $Tr A = \lambda_1 + \lambda_2 < 0 \Rightarrow \lambda_i< 0$\\
            $Tr A = \lambda_1 + \lambda_2 > 0 \Rightarrow \lambda_i> 0$\\ 
        \end{proposition}
        \newpage
        \section{15/10/25}
        Introduzione allo studio di funzioni in due variabili
        \subsection{l'ottimizzazione}
        Cerchiamo il massimo di $f(x,y)$ quando $(x,y) \in A$\\
        \underbar{Se} $A$ è un insieme aperto, (quindi ogni punto è interno ad $A$) allora si parla di \underbar{massimi liberi}.\\
        \underbar{Se} $A$ NON è aperto, ad es. se $A$ è una curva o una superficie, si parla di \underbar{massimi vincolati}.\\
        \begin{example}
            Siano $P_1,P_2,P_3$ punti del piano, si vuole trovare il punto del piano che rende minima la somma delle distanze dai tre punti.
\begin{center}
                \begin{tikzpicture}
                    \draw(0, 0) -- (-1,-0.3) node[at end, below]{$P_2$};
                    \draw(0, 0) -- (1,0.5) node[at end, below]{$P_1$};
                    \draw(0, 0) -- (-1,0.5) node[at end, below]{$P_3$};
                    \filldraw [blue] (0 , 0) circle (2pt);
                    \filldraw [blue] (-1,-0.3) circle (2pt);
                    \filldraw [blue] (1,0.5) circle (2pt);
                    \filldraw [blue] (-1,0.5) circle (2pt);
                    \node at(0.28,-0.15){$(x,y)$};
                \end{tikzpicture}
                
\end{center}
            \begin{gather*}
                A = \mathbb{R}^2
                f(x,y) = \left\lVert (x,y) -P_1 \right\rVert + \left\lVert (x,y) -P_2 \right\rVert + \left\lVert (x,y) -P_3 \right\rVert 
            \end{gather*}

            \begin{gather*}
                V(x,y,z) = xyz\\
                S(x,y,z) = 2xy+2xz+2yz
            \end{gather*}
            Trovare il valore $(x,y,z)$ nel 1° ottante di $\mathbb{R}^3$ soggetti al vincolo:
            \begin{gather*}
                2xy +2yz +2xz = 1 \qquad \text{(ad esempio)}
            \end{gather*}
        \end{example}

            Mi serve definire il maxximo assoluto e relativo:
            \begin{definition}
                Max assoluto\\
                $f$ definita in $A \to \mathbb{R} \quad (x_0,y_0) \in A$ \\
                dico che $(x_0,y_0)$ è punto di max assoluto per $f$ in $A$, e che $f(x_0,y_0)$ è massimo assoluto.\\ 
            \end{definition}
            \begin{definition}
                Max relativo o locale\\
                $f$ definita in $A \to \mathbb{R} \quad (x_0,y_0) \in A$ \\
                dico che $(x_0,y_0)$ è punto di max relativo per $f$ in $A$, e che $f(x_0,y_0)$ è massimo se\underbar{text}:\\
                $\exists$ un intorno $U$ di $(x_0,y_0) \ t.c.$\\
                \begin{gather*}
                    f(x_0,y_0) \leq f(x,y) \quad \forall (x,y) \in U
                \end{gather*} 
            \end{definition}
        \hfil\\
        In generale ho il problema di:
        \begin{enumerate}[$i)$]
            \item So che esiste?
            \item Come lo trovo?
        \end{enumerate}            
        \hfil\\
        Per $i)$ posso usare il t. di Weierstrass:\\
        Se $A$ è chiuso e limitato e se f è continua in $A$ allora esiste sia il max assoluto che il min assoluto.\\
        Per usarlo so che certamente $\exists$ una palla chiusa $C$ t.c. il pinto di minimo non va fuori da $C$. QUindi il problema dell'esistenza è fondamentalmente risolto.\\
        \subsection{trovare il min/max assoluto}
        Non è altrettanto scontato il punto $ii)$, in questo caso cerchiamo delle condizioni analitiche necessarie o sufficienti.\\
        \begin{proposition}
            Condizione necessaria (Teorema di Fermat)\\
            \underbar{Se} $(x_0,y_0)$ è di max relativo; $x_0,y_0 \in _{(int)}A$ \underbar{e}\\
            f è derivabile in $(x_0,y_0)$ \underbar{allora} $\nabla f(x_0,y_0) = 0$
        \end{proposition}
        \begin{proof}
            \begin{tikzpicture}
                \draw(0, 0) -- (0,1);
                \draw(-0.3,0.3) -- (1,0.3);
                \draw[red](-0.3,0.5) -- (1,0.5);
                \node at(-0.3,0.7){$_{y_0}$};
                \node at(0.8,0.7){$_{(x_0,y_0)}$};
                \filldraw [black] (0,0.5) circle (1pt);
                \filldraw [black] (0.3,0.5) circle (1pt);
            \end{tikzpicture}
            \hfil\\
            $\varphi_1(x) = f(x,y_0)$
            $x_0$ è punto di max relativo per la funzione di 1 variabile $\varphi_1(x)$ è derivabile in $x_0$, $x_0$ è interno al dominio di $\phi_1 \Rightarrow \varphi_1'(x_0) = 0$\\
            Ma $\varphi_1'(x_0) = \frac{\partial f}{\partial x}(x_0,y_0)$\\
            Analogamente si dimostra che $\frac{\partial f}{\partial y} (x_0,y_0) = 0$
        \end{proof}
        \begin{proposition}
            Un punto in cui $\nabla f(x_0,y_0)$ è zero si chiama punto stazionario o punto critico
        \end{proposition}
        \begin{example}
            \begin{gather*}
                f(x,y) = x^2+y^2 \qquad (0,0) \text{ è critico e punto di min assoluto}
            \end{gather*}
            \begin{gather*}
                f(x,y) = x^2 = y^2 \qquad (0,0) \text{ è critico ma NON è ne di min. ne di max.}
            \end{gather*}
            Studio la restizione di f lungo l0asse $x$:\\
            \begin{gather*}
                f(x,0) = x^2
            \end{gather*}
            Studio la restizione di f lungo l0asse $x$:\\
            \begin{gather*}
                f(0,y) = -y^2
            \end{gather*}
            Questo è un esempio di punto di sella.
        \end{example}

        \begin{observation}
            un punto di min. locale o assoluto può anche essere \underbar{non critico}. QUesto può succedere se $\nexists \nabla f$ in quel punto.\\
        \end{observation}
        \begin{example}
                \begin{gather*}
                    f(x,y) = \left\lVert (x,y) \right\rVert 
                \end{gather*}
                $(0,0)$ è min assoluto, Ma $\nabla f \nexists$ in $(0,0)$\\
                Quindi $(0,0)$ \underbar{non è critico}
        \end{example}

        \begin{theorem}
            Supponiamo che $f$ sia $C^2$ in un intorno del punto critico $(x_0,y_0)$.\\
            $f(x,y) = f(x_0,y_0) + 0 + \frac{1}{2} \left\langle \underset{\text{\fbox{\textcolor{orange}{*}}}}{D^2f(x_0,y_0)}(x-x_0,y-y_0);(x-x_0,y-y_0)\right\rangle + o(\left\lVert (x-x_0,y-y_0)\right\rVert^2 )$
        \end{theorem}
        \hfil\\
        Devo studiare il segno di \fbox{\textcolor{orange}{*}}:\\
        \underbar{se} $A$ è una matrice $n\times n$ simmetrica , $h \in \mathbb{R}^n$
        \begin{gather*}
            q(h) = \left\langle Ah;h \right\rangle 
        \end{gather*}

        \begin{proposition}
            Richiamo:\\
            $q$ si dice def. positiva \underbar{se} $q(h) > 0 \quad \forall h \neq 0$\\
            $q$ si dice indefinita se $\exists h_1,h_2 \neq 0$ t.c.
            \begin{gather*}
                q(h_1),q(h_2) < 0
            \end{gather*}
        \end{proposition}

        \begin{theorem}
            (Condizione necessaria del secondo ordine).\\
            \underbar{se} $f: A \to \mathbb{R}$ è $C^2$ in un intorno di $(x_0,y_0)$ e $(x_0,y_0)$ è di minimo relativo ed è interno ad $A$ \underbar{allora} la forma quadratica con associata la matrice $h \to \left\langle D^2f(x_0,y_0);h\right\rangle$ è \underbar{\textbf{semi}definita positiva}  
        \end{theorem}
        \begin{proof}
            Sia $h \in \mathbb{R}^2, h \neq 0 \qquad h = (h_1,h_2)$
            \begin{gather*}
                \varphi(t) = f(x_0+t h_1,y_0+t h_2)\\
                t= 0 \longleftrightarrow (x_0,y_0)
            \end{gather*}
            \begin{tikzpicture}
                \draw(0,0) --(0,1);
                \draw(-0.2,0.2) -- (1,0.2);
                \draw[->](0,0.2) -- (0.8,0.8) node[at end,below]{$\varphi_1$};
            \end{tikzpicture}
            Scelgo dunque un vettore
            \begin{gather*}
                \varphi'(0) = 0 \& \varphi''(0) \geq 0\\
                (\varphi(t))' = (f(x_0+th_1,y_0+th_2))'\\
                = \frac{\partial f}{\partial x}(f(x_0+th_1,y_0+th_2))h_1\\
                = \frac{\partial f}{\partial y}(f(x_0+th_1,y_0+th_2))h_2\\
                =(\varphi(t))''= (\frac{\partial f}{\partial x}(x_0+th_1,y_0+th_2)h_1+ \frac{\partial f}{\partial y}(x_0+th_1,y_0+th_2)h_2)'\\
                = (\frac{\partial f}{\partial x}(x_0+th_1,y_0+th_2))'h_1+ (\frac{\partial f}{\partial y}(x_0+th_1,y_0+th_2))'h_2\\
                \frac{\partial^2f}{\partial x^2}(x_0+th_1,y_0+th_2)h_1^2 + \frac{\partial^2 f}{\partial y \partial x}(x_0+th_1,y_0+th_2)h_1h_2 + \frac{\partial^2 f}{\partial x\partial y}(x_0+th_1,y_0+th_2)h_2h_1+\frac{\partial^2 f}{\partial y^2} (x_0+th_1,y_0+th_2)h_2^2
            \end{gather*}
            Se la calcolo per $t=0$ diventa:
            \begin{gather*}
                \left\langle D^2f(x_0,y_0)h;h\right\rangle 
            \end{gather*}
        \end{proof}
        \begin{theorem}
            (condizione sufficiente)\\
            \underbar{Sia} $f : A \to \mathbb{R}$\\
            \underbar{sia} $(x_0,y_0)$ interno ad $A$ e \\
            \underbar{sia} $f C^2$ in un intorno di $(x_0,y_0)$.\\
            \underbar{Supponiamo} che $(x_0,y_0)$ un punto critico.\\
            \underbar{Alllora:}
            \begin{enumerate}[$i)$]
                \item \underbar{se} $D^2f(x_0,y_0)$ è definita positiva $\Rightarrow (x_0,y_0)$ è punto di min relativo
                \item \underbar{se} $D^2f(x_0,y_0)$ è definita negativa $\Rightarrow (x_0,y_0)$ è punto di max relativo
                \item \underbar{se} $D^2f(x_0,y_0)$ è indefinita $\Rightarrow (x_0,y_0)$ non è ne punto di min ne di max relativo
            \end{enumerate}
        \end{theorem}
        \begin{proof}
            punto $iii)$: segue dalla cond. necessaria
        \end{proof}

        \newpage
        \section{16/10/25}
        \textbf{(Richiamo)}
        \begin{theorem}
            (condizione sufficiente)\\
            \underbar{Sia} $f : A \to \mathbb{R}$\\
            \underbar{sia} $(x_0,y_0)$ interno ad $A$ e \\
            \underbar{sia} $f C^2$ in un intorno di $(x_0,y_0)$.\\
            \underbar{Supponiamo} che $(x_0,y_0)$ un punto critico.\\
            \underbar{Alllora:}
            \begin{enumerate}[$i)$]
                \item \underbar{se} $D^2f(x_0,y_0)$ è definita positiva $\Rightarrow (x_0,y_0)$ è punto di min relativo
                \item \underbar{se} $D^2f(x_0,y_0)$ è definita negativa $\Rightarrow (x_0,y_0)$ è punto di max relativo
                \item \underbar{se} $D^2f(x_0,y_0)$ è indefinita $\Rightarrow (x_0,y_0)$ non è ne punto di min ne di max relativo
            \end{enumerate}
        \end{theorem}
        \begin{proof}
            (punto $i/ii$)\\
            Sia $h = (h_1,h_2) \neq 0$\\
            \begin{gather*}
                f(x_0+h_1,y_0+h_2) = f(x_0,y_0) + \left\langle \nabla f(x_0,y_0); h \right\rangle  + \frac{1}{2} \left\langle D^2 f(x_0,y_0)h;h\right\rangle +o(\left\lVert h\right\rVert^2 )\\
                \frac{f(x:0+h_1,y_0+h_2)}{\left\lVert h\right\rVert^2 } =  \underset{\text{\fbox{*}}}{\frac{1}{2} \frac{\left\langle D^2 f(x_0,y_0)h;h\right\rangle}{\left\lVert h\right\rVert^2 }} +\frac{o(\left\lVert h\right\rVert^2 )}{\left\lVert h\right\rVert^2 }\\ 
            \end{gather*}
            se $q(h) = \left\langle Ah;h\right\rangle $\\
            \begin{gather*}
                q(ch) = \left\langle A(ch) ; ch\right\rangle = \left\langle cAh;ch\right\rangle = c^2\left\langle Ah;h\right\rangle   
            \end{gather*}
            Quindi lo fo per: \fbox{*}:
            \begin{gather*}
                \left\langle D^2f(x_0,y_0) \underset{\text{vett. di $\left\lVert \cdot \right\rVert$ 1}}{\frac{h}{\left\lVert h\right\rVert }}; \frac{h}{\left\lVert h\right\rVert }\right\rangle 
            \end{gather*}
            Prendiamo $v = (v_1,v_2) \in$ cerchio di centro $(0,0)$ e raggio 1. ovvero $\mathbb{S}^1$\\
            $p(v) = \left\langle D^2f(x_0,y_0)v;v\right\rangle $\\
            \begin{itemize}
                \item è una funzione continua su $\mathbb{S}^1$
                \item $p(v) > 0 \quad \forall v \in \mathbb{S}^1$
                \item $\mathbb{S}^1$ è chiuso e limitato
                \item $\exists$ minimo. se lo chiamo $m$
            \end{itemize}
            Posso dire che $m>0$. Quindi $\forall v \in \mathbb{S} \quad p(v) \geq m > 0$
            \begin{gather*}
                \frac{1}{2}\left\langle D^2f(x_0,y_0) \frac{h}{\left\lVert h\right\rVert }; \frac{h}{\left\lVert h\right\rVert }\right\rangle \geq \frac{1}{2}m
            \end{gather*}
            Per def. di $o$ piccolo:
            \begin{gather*}
                \lim_{h \to 0} \frac{o(\left\lVert h\right\rVert^2 )}{\left\lVert h\right\rVert^2 } = 0
            \end{gather*}
            \underbar{Allora} $\exists \ \delta > 0 \ t.c. \ $ \underbar{se} :\\
            \begin{gather*}
                \frac{-m}{4}< \frac{o(\left\lVert h\right\rVert^2 )}{\left\lVert h\right\rVert^2 } < \frac{m}{4}
            \end{gather*}
            Tirando le conclusioni :\\
            \underbar{se} $\left\lVert h\right\rVert < \delta$ e $h \neq 0 $ \underbar{allora}:
            \begin{gather*}
                \frac{f(x_0h_1,y_0+h_2)+f(x_0,y_0)}{\left\lVert h\right\rVert^2 } \geq \frac{1}{2}m - \frac{m}{4} = \frac{m}{4} > 0
            \end{gather*}
            Quindi $f(x_0+h_1,y_0+h_2) \geq f(x_0,y_0) + \frac{m}{4} \left\lVert h\right\rVert^2 > f(x_0,y_0)$ 
        \end{proof}
        \begin{proposition}
            Sia $A$ una matrice simmetrica $n \times n$ e $h \in \mathbb{R}^n$\\
            Sia $q(h) = \left\langle Ah;h \right\rangle$ 
        \end{proposition}
        \begin{theorem}
            \hfil\\
            \begin{enumerate}[$i)$]
                \item $q$ è definita positiva $\Leftrightarrow$ tutti gli autovalori sono $>0$
                \item $q$ è definita negativa $\Leftrightarrow$ tutti gli autovalori sono $<0$
                \item $q$ è indefinita $\Leftrightarrow \exists$ un autovalore $>0$ ed $\exists$ un autovalore $<0$
            \end{enumerate}
        \end{theorem}
        \begin{theorem}[Condizioni per forme quadratiche per matrici $3 \times 3$]
            Sia $A \in M(3 \times 3, \mathbb{K})$ con
            \begin{gather*}
                A= \begin{pmatrix}
                    a{1, 1} & a{1, 2} & a{1, 3} \
                    a{1, 2} & a{2, 2} & a{2, 3} \
                    a{1, 3} & a{2, 3} & a_{3, 3}
                \end{pmatrix}
            \end{gather*}
            Allora, chiamate $det A1 = a{1, 1}$, $A2 = \det \begin{pmatrix}
                a{1, 1} & a{1, 2} \
                a{1, 2} & a_{2, 2}
            \end{pmatrix}$ e $A_3 = \det A$ i determinanti delle matrici:
            \begin{enumerate}[$i)$]
                \item Se $\det A_1, \det A_2, \det A_3 > 0 \Longleftrightarrow $ q è definita positiva;
                \item Se $\det A_1 < 0$, $\det A_2 > 0, \det A_3 < 0 \Longleftrightarrow $ q è definita negativa;
                \item Se $\det A \neq 0$ e non valgono né $i)$ né $ii)$ allora è indefinita.
            \end{enumerate}
        \end{theorem}
        
        \subsection{Cercare e classificare i punti critici}
        \begin{example}
            \hfil\\


%\begin{center}
%\begin{animateinline}[autoplay,loop,controls]{8} % 8 fps
%  \multiframe{30}{rZ=-2+0.15}{% apre multiframe
%    % otteniamo il valore numerico e costruiamo la formula espansa
%    \pgfmathsetmacro{\zvalue}{\rZ}%
%    \edef\myfunction{%
%      x^2*y + y^2*(\zvalue) + (\zvalue)^2 - 2*x%
%    }%
%
%    \begin{tikzpicture}
%      \begin{axis}[
%        width=10cm,
%        % NON mettere \zvalue nel title: può finire scritto su .aux e causare errori
%        xlabel=$x$, ylabel=$y$, zlabel={$f(x,y,z)$},
%        domain=-2:2, domain y=-2:2,
%        samples=20, samples y=20,
%        view={45}{30},
%        z buffer=sort,
%      ]
%        % plot della superficie per il valore corrente di z
%        \addplot3[surf,shader=interp]{\myfunction};
%        % posizioniamo un nodo che mostra il valore corrente di z (non viene scritto su .aux)
%        \node[anchor=south east] at (rel axis cs:1,1) {\small $z=\pgfkeysvalueof{/pgfplots/axis cs x}$}; % placeholder (opzionale)
%      \end{axis}
%      % Mostriamo il valore numerico fuori dall'axis (più sicuro)
%      \node at (5,-1) {\small $z = \zvalue$};
%    \end{tikzpicture}
%
%  }% chiude multiframe
%\end{animateinline}
%\end{center}
            $f(x,y,z) = x^2y+y^2z+z^2-2x$
            \begin{gather*}
                \begin{cases}
                    2xy-2=0\\
                    x^2 +2yz = 0\\
                    y^2+2z = 0
                \end{cases}
                \begin{cases}
                    xy = 1\\
                    x^2 -y^3 = 0\\
                    z = -\frac{y^2}{2}
                \end{cases}
                \begin{cases}
                    x^{\frac{5}{3}} = 1 \Rightarrow  x = 1\\
                    y = x^{\frac{2}{3}} \Rightarrow y = 1\\
                    z = -\frac{y^2}{2} \Rightarrow  z= -\frac{1}{2}
                \end{cases}
                P = (1, 1, -\frac{1}{2})
            \end{gather*}
            Avrò l'hessiana che è:
            \begin{gather*}
                D^2f(x,y,z) =
                \begin{pmatrix}
                    2y & 2x & 0\\
                    2x & 2z & 2y\\
                    0 & 2y & 2
                \end{pmatrix}\\
                D^2f(1,1,-\frac{1}{2}) = \begin{pmatrix}
                    2 & 2 & 0\\
                    2 & -1 & 2\\
                    0 & 2 & 2
                \end{pmatrix}
                \det(2) = 2 > 0\\
                \det\begin{pmatrix}
                    2 & 2\\
                    2 & -1
                \end{pmatrix} = -6 < 0\\
                \det\begin{pmatrix}
                    2 & 2 & 0\\
                    2 & -1 & 2\\
                    0 & 2 & 2
                \end{pmatrix} = -20 < 0
            \end{gather*}
            Quindi $P$ è punto di sella
        \end{example}
        \begin{example}
            \hfil\\
    \begin{center}
    \begin{tikzpicture}
    \begin{axis}[domain=-2:2,y domain=-2:2, samples = 50,view={30}{30},colormap/viridis]
        \addplot3[surf] {x*y*e^(-(x^2+y^2)/(2))};
    \end{axis}
    \end{tikzpicture}
    \end{center}
            $f(x,y) = xy e^-\frac{x^2+y^2}{2}$
            \begin{itemize}
                \item trovare e classificare i punti critici in $\mathbb{R}^2$
                \item Esistono max o min assoluti in $\mathbb{R}^2$?
            \end{itemize}
            Iniziamo con risolvere il punto 1 che è da procedura\\
            \begin{gather*}
                \frac{\partial f}{\partial x} = y(1-x^2) \cancel{e^{-(\frac{x^2}+y^2{2})}}\\
                \frac{\partial f}{\partial y} = x (1-y^2) \cancel{e^{-(\frac{x^2}+y^2{2})}}
            \end{gather*}
            Soluzioni di $\frac{\partial f}{\partial x}$:
            \begin{gather*}
                y = 0 , x =0 \\
                x^2 = 1 \text{ qui ho due casi:}\\
                i) \ x = 1 \to y^2 = 1 \to y = \pm 1\\
                ii) \ x = -1 \to y^2 = -1 \to y = \pm 1
            \end{gather*}
            Foto il resto\dots      \\
            Punto $ii)$ In generale in questo caso posso pensare che un punto di min assoluto essite poichè ho un esponenziale e se fo il limite che va a infinito l'esponenziale vincerà e siccome ha come argomento il quadrato delle distanze x e y con un segno MENO farà andare il limite a 0.
            In generale si deve fare il limite all'infinito e se si trova un numero piu grande o piu piccolo dei min o max relativi, e confrontarli.
        \end{example}
        Esempi in cui $D^2 f$ non è ne def positiva, ne def. negativa, ne indefinita in qualcuno dei suoi punti critici.\\
        \begin{example}
            \hfil\\
    \begin{center}
    \begin{tikzpicture}
    \begin{axis}[domain=-2:2,y domain=-2:2, samples = 50,view={30}{30},colormap/viridis]
        \addplot3[surf] {x^4+6*x^2*y^2+y^4};
    \end{axis}
    \end{tikzpicture}
    \end{center}
            $f(x,y) = x^4 +6x^2y^2+y^4$\\
            punti critici... blablabla\dots\\
            $P = (0,0)$ quindi:\\
            \begin{gather*}
                \begin{pmatrix}
                    0 & 0\\
                    0 & 0
                \end{pmatrix}\\
                f(x,y) - \underset{\underset{0}{||}}{f(0,0)} = f(x,y)
            \end{gather*}
            Proviamo a restringere sul punto $(0,0)$ con delle rette e vedo come passano (se crescono o decrescono).\\
            Formalmente: Restringo la funzione a rette (o curve) che passano per il punto critico e studio il segno di questa funzione.
            \begin{gather*}
                f(x,mx) = x^4(1-6m^2+m^4)\\
                f(x,x) = 4x^4
            \end{gather*}
        \end{example}
        \begin{example}
            $f(x,y) = e^{x^3}(1-y^4)$ in $\mathbb{R}^2$, si devono classificare i punti critici.\\
            Ho un solo punto critico $P = (0,0)$
            \begin{gather*}
                D^2f(0,0) = \begin{pmatrix}
                    \text{qualcosa} & 0 //
                    0 & 0
                \end{pmatrix}
            \end{gather*}
            Devo usare il metodo delle restrizioni:
            \begin{gather*}
                f(x,y) - f(0,0) = e^{x^3}(1-y^4)
            \end{gather*}
            Restrizione a $x=0$:
            \begin{gather*}
                f(0,y) - f(0,0) = -y^4
            \end{gather*}
            restrizione a $y=0$
            \begin{gather*}
                f(x,0)- f(0,0) = e^{x^3}-1
            \end{gather*}
            Quindi lungo l'asse x ho sempre qualcosa di negativo e se restringo lungo l'asse y ho dei punti positivi e negativi, quindi ho un punto di SELLA.
        \end{example}
    \begin{example}
            \hfil\\
    \begin{center}
    \begin{tikzpicture}
    \begin{axis}[domain=-2:2,y domain=-2:2, samples = 50,view={30}{30}, zmin = -4,zmax = 4,colormap/viridis]
        \addplot3[surf] {(x^2+y^2-1)^2*(x+y)+2};
    \end{axis}
    \end{tikzpicture}
    \end{center}
            $f(x,y) = (x^2+y^2-1)^2(x+y)+2$
            \begin{gather*}
                \begin{cases}
                    (x^2+y^2-1)(5x^2+4xy+y^2-1) = 0\\
                    (x^2+y^2-1)(x^2+4xy+5y^2-1) = 0
                \end{cases}
            \end{gather*}
            Ho "due" soluzioni:
            \begin{gather*}
                x^2 + y^2 = 0\\
                x^2+4xy+5y^2 = 1
            \end{gather*}
            In realtà in questo caso ogni volta che avviene una condizione tutto il sistema è nullo, quindi ho una circonferenza di raggio 1 che contiene tutti punti critici.\\
            Punti critici:
            \begin{itemize}
                \item tutti i punti della circonferenza del cerchio $\mathbb{S}^1$
                \item $(+\frac{1}{\sqrt{10}}, +\frac{1}{\sqrt{10}}), (-\frac{1}{\sqrt{10}},-\frac{1}{\sqrt{10}})$
            \end{itemize}
            \begin{gather*}
            \begin{pmatrix}
            -\frac{4}{5}\frac{14}{\sqrt{10}} & -\frac{4}{5}\frac{6}{\sqrt{10}}\\
            -\frac{4}{5}\frac{6}{\sqrt{10}} & \frac{4}{5}\frac{14}{\sqrt{10}}
            \end{pmatrix}, \quad \det > 0, \quad a_{11} < 0 \;\Rightarrow\; \text{max locale}
            \\[1em]
            \begin{pmatrix}
            '' & ''\\
            '' & ''
            \end{pmatrix} \;\Rightarrow\; \text{min locale}
            \\[1em]
            f(x,y) - f(\text{punto cerchio}) = f(x,y) - 2 = (x^2 + y^2 - 1)^2 (x + y)
            \end{gather*}

        \end{example}
        \subsection{altri casi}
        $D \subset \mathbb{R}^2$ non assumo niente su $D$\\
        $f:D \to \mathbb{R} \max_{(x,y) \in D} f$
        \begin{theorem}
            Se $(x_0,y_0)$ è un punto di max/min locale per $f$ in $D$ \underbar{allora}:
            \begin{enumerate}[$i)$]
                \item o $(x_0,y_0)$ è interno e $f$ on è derivabile in $(x_0,y_0)$
                \item o $(x_0,y_0)$ è interno a $\nabla f (x_0,y_0) = 0$
                \item o $(x_0,y_0) \in \partial D$ (si trova nella frontiera)  
            \end{enumerate}
        \end{theorem}
        \begin{proposition}
            Strateia per trovare il max. assoluto fi una funzione continua su un insieme $D$ chiuso e limitato\\
            \begin{itemize}
                \item Trovo punti interni critici
                \item Trovo punti interni in cui $f$ non è derivabile
                \item Trovo i punto di max di $f$ rispetta alla frontiera
            \end{itemize}
            Confronto il valore di $f$ in tutti i punti trovati, il valore più grande corrisponde al max. assoluto
        \end{proposition}
        \begin{example}
            Trovare max assoluto di $f(x,y) = 2xy$ su $D=\{x,y:x^2+y^2\leq 4\}$\\
            \begin{itemize}
                \item punti di $f$ non derivabile $\to \nexists$
                \item punti critici: $(0,0)$
            \end{itemize}
            \begin{gather*}
                g(\theta) := f(2\cos\theta, 2 \sin\theta) = 8(\cos\theta,\sin\theta) \quad \theta \in [0,2\pi]
            \end{gather*}
            Cerco i max di questa funzione:
            \begin{itemize}
                \item punti in cui $\frac{d}{d\theta}(f) = 0$
                \item estremi dell'intervallo $g(0) = 0 \ g(2\theta) = 0$
            \end{itemize}
            \begin{gather*}
                g'(\theta) = 8(\cos^2\theta \sin^2\theta) = 8(\cos\theta\sin\theta)(\cos\theta\sin\theta)\\
                \theta = \frac{\pi}{4},\frac{3}{4}\pi,\frac{5}{4}\pi,\frac{7}{4}\pi\\
                g(\frac{\pi}{4}) = g(\frac{5}{4}\pi) = 8\frac{\sqrt{2}}{2}\frac{\sqrt{2}}{2} = 4\\
                g(-\frac{3}{4}\pi) = g(\frac{7}{4}\pi) = -4\\
                (\sqrt{2},\sqrt{2})(-\sqrt{2},-\sqrt{2})
            \end{gather*}
        \end{example}

        \newpage
        \section{21/10/25}
        Richiamo teorema 12.4.
        \begin{theorem}
            Se $(x_0,y_0)$ è un punto di max/min locale o assoluto per $f$ in $D$ \underbar{allora}:
            \begin{enumerate}[$i)$]
                \item o $(x_0,y_0)$ è interno e $f$ non è derivabile in $(x_0,y_0)$
                \item o $(x_0,y_0)$ è interno a $D$, f è derivabile e  $\nabla f (x_0,y_0) = 0$
                \item o $(x_0,y_0) \in \partial D$ (si trova nella frontiera)  
            \end{enumerate}
        \end{theorem}
        Strategia per trovare max assoluti in una funzione continua in un insieme compatto.\\
        \begin{itemize}
            \item Trovare punti interni criticiTrovare punti interni in cui $f$ non è derivabile
            \item Trovare i punti di max di $f$ ristretta alla frontiera
            \item Confrontare tutti i punti trovati e i valori di $f$ in tali punti
        \end{itemize}
        \begin{example}
            $f(x,y)$ cerchio $x^2+y^2 \leq 4$\\
            \begin{itemize}
                \item punti di non derivabilità $\emptyset$
                \item punti in cui $\nabla = 0$
                \item Studio di $\partial D$, cioè in cui $\{x^2+y^2=4\}$
            \end{itemize}

            $f$ ristretta a $\partial D \Leftrightarrow$ 2 $\cdot$ 2 cost. $\cdot$ 2 sent. ($= g(x)$)\\
            Cercare possibili max di $g(t) \quad t \in [0,2\pi]$\\
            \begin{gather*}
                g'(t) = 0 \to t = \frac{\pi}{4}, \frac{3}{4}\pi, \frac{5}{4}\pi, \frac{7}{4}\pi \qquad \text{dove nei primi 2 $g$ vale 4 e negli altri -4}\\
                g(0) = 0 \to g(2\pi) = 0
           \end{gather*}
           \begin{gather*}
            f(x,y) = x^2ye^{-(x+y)} \qquad \text{su } D=\{x \geq 0, y \geq 0, x+y \leq 4\}
           \end{gather*}
           \begin{itemize}
            \item interni in cui $f$ non è derivabile $\emptyset$
            \item Interni in cui $\nabla f = 0$
           \end{itemize}
           \begin{center}
            \begin{tikzpicture}
                \draw(0,0) -- (3,0) node[midway, below]{$\boxed{II}$};
                \draw(0,0) -- (0,2) node[midway, left]{$\boxed{I}$};
                \draw(3,0) -- (0,2) node[midway, above, right]{$\boxed{III}$};
                \filldraw[fill=black](1,0.7) circle (2pt) node[above]{$D$};
                \node at(0.7,-0.2){$2$};
                \node at(-0.2,0.3){$1$};
                \filldraw[fill=black](1,1.49) circle (2pt) node[above, right]{$(\frac{8}{3},4-\frac{8}{3})$};
                \filldraw[fill=black](3,0) circle (2pt) node[below, right]{$4$};
                \filldraw[fill=black](0,2) circle (2pt) node[above, left]{$4$};
            \end{tikzpicture}
           \end{center}
           \begin{gather*}
                \{\underset{f = 0}{x = 0, y \in [0,4]}\} \cup \{\underset{f = \frac{4}{e^3} \approx 0,199}{(2,1)}\}
           \end{gather*}
           $f$ f ristretta a \fbox{$I$} $f(0,y) \quad y \in [0,4]$
           \begin{gather*}
            f_{|_I}
           \end{gather*} 
           $f$ ristretta a \fbox{$II$} $f(x,0) \quad x \in [0,4]$
           \begin{gather*}
            f_{|_{II}}
           \end{gather*}
           $f$ ristretta a a $\underset{x+y = 4 \ y = 4-x}{\boxed{III}}$ $\underset{g(x)}{f(x,4-x)} \quad x \in [0,4]$
           \begin{gather*}
            g(x) = x^2(4-x) e^{-4} \quad g'=0 \qquad x = 0 \quad x = \frac{8}{3}
           \end{gather*}
           $g$ vale $0$ \\
           $g$ vale $\frac{256}{27}e^{-4} \approx 0,174$
           \begin{gather*}
            g(x) = x^2(4-x)e^{-4}\\
            g(0) = 0 \quad g(4) = 0
           \end{gather*}
        \end{example}
        \subsection{Ricerca di massimo vincolato}
        cercato i max/min di $x^2y e^{-(x+y)}$ sul segmento $x+y=4$ compreso tra $(0,4 )$ e $(4,0)$\\
        Ricerca di max $f(x,y)$ sul vincolo $\{g(x,y) = 0\}$\\
        tra tutti i punti che soddisfano il vincolo $g(x,y) = 0$, trovare quelli che rendono max. $f(x,y)$

        \begin{example}
            es. di prima studio sulla frontiera:\\
            Trovare il max di $f(x,y) = 2xy$ su $\underset{g(x,y)}{x^2+y^2 -4} = 0$ 
        \end{example}
        \begin{example}
            Determinare il punto della curva $x^2y -16 = 0$ la cui distanza dal punto $(0,0)$ è minima
            \begin{center}
                \begin{tikzpicture}
                    \draw[->](0,0) --(4,0);
                    \draw[->](2,0) -- (2,3) node[at end]{$\{(x,y):x^2y -16 = 0\}$};
                    \draw(0,0.2) ..controls(1.2,1).. (1.8,3);
                    \draw(4,0.2) ..controls(2.8,1).. (2.2,3);
                    \draw[->](2.9,1.1) -- (3.2,1.4) node[at end,right]{$\nabla g(x_0,y_0)$};
                    \draw[->](2.9,1.1) -- (2.6,0.8) node[at end,left,below]{$\nabla f(x_0,y_0)$}; 
                    \node at(2.3,1)[left, above]{$(x_0,y_0)$};
                \end{tikzpicture}
            \end{center}
            min $x^2+y^2$ sul vincolo $x^2y -16= 0$\\
            una possibilità: $y = \frac{16}{x^2}$
            \begin{gather*}
                f(x,\frac{16}{x^2}) \text{ cioè } x^2+\frac{16}{x^2} \quad x \in \mathbb{R}, x \neq 0
            \end{gather*}
        \end{example}
        \subsection{Metodo dei moltiplicatori di Lagrange}
        \begin{theorem}
            \underbar{Sia} $G = \{(x,y): g(x,y) = 0\}$\\
            \underbar{Sia} $(x_0,y_0)$ un punto di max locale per $f(x,y)$ ristretta al vincolo $G$.\\
            \underbar{Supponiamo} che \underbar{sia} $f$ che $g$ siano funzioni $C^1$ in un intorno di $(x_0,y_0)$ e che $\nabla g(x_0,y_0) \neq 0$\\
            \underbar{Allora} $\exists \lambda_0 \in \mathbb{R}$ \textbf{tale che} $(x_0,y_0,\lambda_0)$ è un punto critico della funzione.
        \end{theorem}
        \begin{gather*}
            \mathcal{L}(x,y,\lambda) := f(x,y) + \lambda g(x,y)
        \end{gather*}
        Cosa vuol dire che $(x_0,y_0)$ è punto critico per $\mathcal{L}$
        \begin{gather*}
            \frac{\partial \mathcal{L}}{\partial x}=0\\
            \frac{\partial \mathcal{L}}{\partial y}=0\\
            \frac{\partial \mathcal{L}}{\partial \lambda}=0\\
            \begin{cases}
                \frac{\partial f}{\partial x} = -\lambda_0 \frac{\partial g}{\partial x}(x_0,y_0)\\
                \frac{\partial f}{\partial y} = -\lambda_0 \frac{\partial g}{\partial y}(x_0,y_0)\\
                g(x_0,y_0) = 0
            \end{cases}
        \end{gather*} 
        Per trovare un punto di max o di minimo quando (x,y) è vincolato a stare sulla curva di eq. g(x,y) si costruisce una funzione a partire dalla funzione f che voglio rendere massima e dalla funzione g che descrive il vincolo introducendo una variabile $\lambda$.\\
        Il teorema mi dice proprio che il punto di max è un punto critico per questa funzione di tre variabili, quindi la prima ci dice che il p.
        \begin{proposition}
            \begin{gather*}
                \nabla f(x_0,y_0) = -\lambda_0 \nabla g(x_0,y_0)
            \end{gather*}
            $\nabla G(\overline{x}, \overline{y}) $ è perpendicolare alla linea di livello che contiene $(\overline{x}, \overline{y})$
        \end{proposition}
        \begin{definition}
            \underbar{Sia} $G=\{(x,y) : g(x,y) = 0\}$ e \underbar{sia} $(x_0,y_0) \in G$\\
            \underbar{Si dice} che $(x_0,y_0)$ è un punto di max. locale \textbf{vincolato} per $f$ sul vincolo $G$ se $\exists$ un intorno $\mathcal{U}$ di $(x_0,y_0)$ \underbar{tale che}
            \begin{gather*}
                f(x,y) \leq f(x_0,y_0) \qquad \forall (x,y) \in \mathcal{U} \cap  G
            \end{gather*} 
            \begin{center}
                \begin{tikzpicture}[replace stretch/.style args={from #1 to #2 by #3}{%
/utils/exec=\pgfmathsetmacro{\offlen}{#2-#1},
dash pattern=on #1 off \offlen pt on 10cm,
postaction={#3,dash pattern=on 0pt off #1 on \offlen pt off 10cm}}]
                    \draw(0,0)[replace stretch={from 32.5 to 92.5 by {-,draw=red}}] ..controls(1,-1) and (3,1.8).. (4,1) node[at end, right]{$\{g(x,y) = 0\} = G$};
                    \draw(2,0.4) circle (30 pt);
                    \node at(1.8,0.25)[below, right]{$(x_0,y_0)$};
                    \node at(0.6, 0.2){$\textcolor{red}{(x,y)}$};
                    \filldraw[fill = black] (2,0.43) circle(2 pt);
                \end{tikzpicture}
            \end{center}
        \end{definition}
        \begin{example}
            \begin{gather*}
                \mathcal{L} (x,y,\lambda) = x^2+y^2+\lambda(x^2y-16)\\
                \frac{\partial \mathcal{L}}{\partial x} = 0\\
                \frac{\partial \mathcal{L}}{\partial y} = 0\\
                \frac{\partial \mathcal{L}}{\partial \lambda} = 0\\
                \begin{cases}
                    2x + \lambda 2xy = 0\\
                    2y +\lambda x^2 = 0\\
                    x^2 y -16 = 0
                \end{cases}
            \end{gather*}
            Dalla prima eq.
            \begin{gather*}
                2x(1+\lambda y) = 0\\
                \cancel{x = 0} \text{ non soddisfa il vincolo}\\
                1 + \lambda y = 0
            \end{gather*}
            Dalla seconda eq.
            \begin{gather*}
                \begin{cases}
                    2y -\frac{1}{y} x^2 = 0\\
                    x^2y = 16
                \end{cases}
                \begin{cases}
                    \frac{2y^2-x^2}{y} = 0\\
                    x^2y = 16
                \end{cases}
                \begin{cases}
                    x = \pm \sqrt{2}y\\
                    2y^2y = 16
                \end{cases}
                \begin{cases}
                    x = \pm 2\sqrt{2}\\
                    y = 2
                \end{cases}\\
            (2\sqrt{2},2) \quad (-2\sqrt{2},2)    
            \end{gather*}
            Ho trovato che:
            \begin{gather*}
                \underset{G}{min f} = \underset{G \cap C}{min f}
            \end{gather*}
            Quindi il minimo sarà uno dei due candidati, e so che la distanza di $f(x,y)$ vale 12 in questi due punti, quindi la distanza è uguale in due punti,di conseguenza ho due punti di minimo (assoluto).
        \end{example}
        \begin{example}
            $f(x,y) = 2xy \qquad \{x^2+y^2 \leq 4\}$\\
            \begin{gather*}
                x^2+y^2 -4 = 0\\
                \mathcal{L}(x,y,\lambda) = 
                \begin{cases}
                    2y + 2\lambda x = 0 \quad A\\
                    2x+ 2\lambda 2x + 2\lambda y = 0 \quad B\\
                    x^2+y^2 -4 = 0
                \end{cases}
            \end{gather*}
            Sostituendo $A+B$ al posto di $A$ si ottiene\begin{gather*}
                \begin{cases}
                    2(x+y) +2\lambda(x+y) = 0 \to 2(x+y)(1+\lambda) = 0 \to y = -x \text{ o } \lambda = -1\\
                    2x+ 2\lambda 2x + 2\lambda y = 0 \\
                    x^2+y^2 = 4
                \end{cases}\\
                \begin{cases}
                    y = -x\\
                    2x+2\lambda y = 0\\
                    2x + 2\lambda y = 0\\
                    x^2+x^2 = 4 \to x = \pm \sqrt{2} \text{ e } y = \mp \sqrt{2}
                \end{cases}
                \begin{cases}
                    \lambda = -1\\
                    2x-2y = 0 \to x = y\\
                    x^2+y^2 = 4 \to x = \pm \sqrt{2} \text{ e } y = \mp \sqrt{2}
                \end{cases}
            \end{gather*}
        \end{example}
        \noindent
        La tecnica si applica anche in dimensione $> 2$
        \begin{example}
            Trovare il min o max di $f(x,y,z)$ sul vincolo $G= \{(x,y,z) : g(x,y,z) = 0\}$\\
            Esempio Trovare max. e min. di $f(x,y,z) = xy^2z^3$\\
            Sul vincolo $\{x^2+y^2+z^2=1\}$ (superficie di una sfera di raggio 1).\\
            min $f(x,y,z)$ su $G=\{\underset{g(x,y,z)}{x^2+y^2+z^2 -1 = 0} , \underset{h(x,y,z) = 0}{x+y+z -\frac{1}{2} = 0}\}$\\
            \begin{gather*}
                \mathcal{L}(x,y,z,\lambda,\mu) = f(x,y,z) + \lambda g(x,y,z) + \mu h(x,y,z)
            \end{gather*}
            Se introduco piu di un vincolo avrò piu parametri 
        \end{example}
\newpage
\section{funzioni implicite}
Prendiamo una funzione $\{F(x,y) = 0\}$ che definisce un grafico di funzione (di 1 variabile)\\
\begin{example}
    $F(x,y) = x^3-y+1$\\
    $F=0 \Leftrightarrow x^3-y+1 = 0 \Leftrightarrow y = x^3+1$
\end{example}
\begin{example}
    $x^2+y^2-1 = 0$\\
    $x^2+y^2 = 1$\\
    Globalmente questo non è un grafico di funzione, MA localmente \textbf{si}, infatti se prendiamo il grafico delimitato dalle $\times$ e dai $\circ$ otteniamo rispettivamente proprio un grafico di $y = y(x)$ e $x = x(y)$:
    \begin{center}
        \begin{tikzpicture}
            \draw[->](1,0) -- (1,2);
            \draw[->](0,1) -- (2,1);
            \draw(1,1) circle (0.8);
            \filldraw[fill=black](1,1.8) circle (1pt)node{$\times$};
            \draw(1.8,1) circle (2pt);
            \filldraw[fill=black](1,0.2) circle (1pt)node{$\times$};
            \draw(0.2,1) circle (2pt);
        \end{tikzpicture}
    \end{center}
\end{example}

\begin{theorem}[del DINI]
    $A_{ap} \subseteq \mathbb{R}^2 \ F:A\to \mathbb{R}, F \in C^1 (A)$\\
    \underbar{sia} $P_0 \equiv (x_0,y_0) \in A \ t.c. \ F(x_0,y_0) = 0 , P_0 $ Regolare (cioè $DF(x_0,y_0)\neq 0$)\\
    \underbar{Allora} $\exists U$ intorno di $x_0$ e $\exists V$ intorno di $y_0$ $ \ t.c. \ F(x,y) = 0$ definisce un grafico di funzione $y= f(x)$ oppure $x = g(y)$ in $U\times V$ intorno di $P_0$
\end{theorem}
\underbar{\textbf{In particolare:}}\\
\begin{itemize}
    \item  se $F_y(x_0,y_0) \neq 0$ allora $\exists! y = f(x):U\to V \ t.c. \ F(x,f(x)) = 0$, $f \in C^1(U)$ e $ f'(x) = \frac{F_x(x,f(x))}{F_y(x,f(x))} \quad \forall x \in \overset{\circ}{U}$
    \item se $F_x(x_0,y_0) \neq 0$ allora $\exists! x = g(y): V \to U \ t.c. \ F(g(y),y) = 0$, $g \in C^1(V)$ e $g'(y) =\frac{F_y(g(y),y)}{F_x(g(y),y)} \quad \forall y \in \overset{\circ}{V}$
\end{itemize}
$F(x,y) = x^2+y^2-1$\\
$DF = \begin{pmatrix}
    2x\\
    2y
\end{pmatrix}$\\
punti critici $\Leftrightarrow DF = \underbar{0} \Leftrightarrow P \equiv(0,0) \cancel{\in}$ linea $(F(x,y) = 0) \Rightarrow $ Tutti i punti di $F(x,y) = 0$ sono regolari.\\
\begin{observation}
    Geometricamente cosa significa che una curva non è grafico di funzione?
    \begin{center}
        \begin{tikzpicture}
            \draw[->](0,0) -- (2,0);
            \draw[red](1,-1) -- (1,1);
        \end{tikzpicture}
        \begin{tikzpicture}
            \draw[->](0,0) -- (2,0);
            \draw[red](1,-0.7) ..controls(0.6,0).. (1,0.7);
        \end{tikzpicture}
    \end{center}
    Questi sono esepi che NON sono grafici $y=y(x)$\\
    Ma sono grafici di $x=x(y)$
    \begin{center}
        \begin{tikzpicture}
            \draw[->](1,0) -- (1,2);
            \draw[->](0,1) -- (2,1);
            \draw[cyan](0,0) -- (2,2);
            \draw[cyan](0,2) -- (2,0);
            \filldraw[fill=black](1,1) circle(2pt)node[right]{$P_0$};
        \end{tikzpicture}
    \end{center}
    Questo è un esempio dovel'intorno di $P_0$ ovvero $\mathcal{U}_{P_0}$ NON può essere grafico di funzione
\end{observation}

\begin{observation}
    Se $\underbar{r}(t)$ è curva piana regolare $t \in I \quad \forall t_0 \in \mathring{I}$ il sostegno è localmente grafico
\end{observation}
\begin{observation}
    $F \in C^2$ ( e vale teorema Dini) $\Rightarrow f \in C^2$\\
    Da cui $f(x) = \underset{\textcolor{cyan}{\underset{y_0}{||}}}{f(x_0)} = f(x_0) + \underset{\textcolor{yellow}{= -\frac{F_x(x_0,y_0)}{F_y(x_0,y_0)}}}{f'(x_0)}(x-x_0) + \textcolor{orange}{\boxed{\textcolor{white}{\frac{f''(x_0)}{2}}}}(x-x_0)^2 + o((x-x_0)^2)$
    per $x \in I_{x_0}$
    \begin{gather*}
        f''(x) = \left(-\frac{F_x(x,f(x))}{f_y(x,f(x))}\right)' = -\frac{(F_x(x,f(x)))'F_y(x,f(x))-F_x(x,f(x))(F_y(x,f(x)))'}{\left(F_y(x,f(x))\right)^2 }\\
        = -\frac{1}{(F_y)^2}[(F_{xx} + F_{xy}f')F_y-F_x(F_{\textcolor{yellow}{xy}}+F_{yy}f')]\vert_{x,f(x)} 
    \end{gather*}
\end{observation}
\begin{example}
    \begin{center}
    \begin{tikzpicture}
    \begin{axis}[domain=-2:2,y domain=-2:2, samples = 25,view={30}{30}, zmin = -4,zmax = 4,colormap/viridis]
        \addplot3[surf] {x^2-x^4-y^2};
    \end{axis}
    \end{tikzpicture}
    \end{center}
    $x^2-x^4-y^2 = 0$\\
    $F(x,y) = x^2 x^4 -y^2$\\
    \begin{gather*}
        DF = (2xx-4x^3-2y) = 2(x(1-2x^2)-y)
    \end{gather*}
    Punti critici $\Leftrightarrow y = 0 \land x(1-2x^2) = 0$ $\quad \Leftrightarrow \quad P_0\equiv(0,0) \ P_1\equiv(\frac{1}{\sqrt{2}},0) \ P_2\equiv(-\frac{1}{\sqrt{2}},0)$
    \begin{gather*}
        \textcolor{cyan}{F(P_0)= 0 \quad F(P_1) = \frac{1}{2}-\frac{1}{4} \neq 0 \quad F(P_2)= \frac{1}{2}-\frac{1}{4} \neq 0}
    \end{gather*}
    Trovare adesso la retta tangente a $F=0$ in $P \equiv (-\frac{1}{\sqrt{2}}, \frac{\sqrt{3}}{4})$\\
    \begin{gather*}
        r_{tg}: y=f'(\frac-{1}{2})(x-(\frac{-1}{2})) + \frac{\sqrt{3}}{4}
    \end{gather*}
    Dove $f(x)$ è l'uncia funzione definita in $I_{-\frac{1}{2}}$ data dal teorema DINI.
    \begin{gather*}
        \Rightarrow f'(-\frac{1}{2}) = -\frac{F_x(-\frac{1}{2},\frac{\sqrt{3}}{4})}{F_y(-\frac{1}{2},\frac{\sqrt{3}}{4})}
    \end{gather*}

\end{example}
\begin{proof}
    (teor, FMS, Giusti) $P_0$ regolare supp. $F_y(x_0,y_0) \neq 0 , F_y(x_0,y_0) > 0$\\
    \fbox{I} $\exists U$ intorno di $x_0$ e  $V = [y_0-\delta,y_0+\delta]$ intorno di $y_0$   
    \begin{gather*}
        F_y(x_0,y_0) > 0, F_y \in C^0 \Rightarrow \exists R \text{rettangolo chiuso } = W \times V \ t.c. \ W \text{int. di } x_0, V \text{int. di } y_0\\
        F_y (x_0,y_0) > 0 \quad \forall (x,y) \in R
    \end{gather*}
    Quindi $\forall x \in W $ fissato $F(x,y)$(come funzione di $y$) $\nearrow$ streattmente\\
    in part. $F(x_y,y) \ \nearrow $ strettamente , con $F(x_0,y_0) > 0$\\
    \begin{gather*}
        \Rightarrow F(x_0,y_0-\delta) < 0 \text{ e } F(x_0,y_0+\delta)
    \end{gather*} 
    Considero $F(x_0,y_0-\delta)$ funzione continue per la permanenza del segno $F(x,y_0-\delta) < 0 \quad \forall x \in U$\\
    \ \ \ \ \ $F(x_0,y_0+\delta)$ $\exists U $ intorno di $x_0, U \subseteq W$ e $F(x,y_0+\delta) > 0\quad  \forall x \in U$\\
    \fbox{II}$\exists! f:U \to V \ t.c. \ F(x,f(x)) = 0 \quad \forall x \in U$\\
    osserviamo $\forall x \in U \quad y \to F(x,y) \nearrow $ strett.\\
    con\begin{gather*}
        F(x,y_0-\delta) <0 \quad \forall x \in U \\
        F(x,y_0+\delta) >0 \quad \forall x \in U \\
        F \in C^0
    \end{gather*}
    $\Rightarrow$ (Th. esist. zeri + stretta monotonia) $\forall x \in U \quad \exists! y \in V:F(x,y) = 0$ cioè $y=f(x)$\\
    \fbox{III} $f\in C^0(U)$\\
    fix. $x_1 \in U$ considero $x \in U$
    \begin{gather*}
        G(t) = F'\left( \underset{\xi_t }{(1-t)x_1 +tx} ; \underset{\eta_t }{(1-t)f(x_1) + tf(x) }\right) 
    \end{gather*}
    $G \in C^1$
    \begin{gather*}
        G(0) = F(x_1,f(x_1) = 0)\\
        G(1) = F(x,f(x) = 0)\\
    \end{gather*}
    Uso Th. ROlle $\Rightarrow \exists \tau \in (0,1) \ t.c. \ G'(\tau) = 0$\\
    \begin{gather*}
        0 = G'(\tau) = F_x(\xi_\tau, \eta_\tau)(x-x_1) + F_y(\xi_\tau,\eta_\tau)(f(x)-f(x_1))\\
        (f(x)-f(x_1)) (\underset{\xi_\tau}{F_y(1-\tau)x_1 + \tau x}; \underset{\eta_\tau}{(1-\tau)f(x_1)+\tau f(x)} ) = -(x-x_1)(F_x(\xi_\tau,\eta_tau))\\
        f(x) -f(x_1) = -(x-x_1)\frac{F_x(\xi, \eta)}{F_y(\xi, \eta)}\\
        \left\lvert f(x)-f(x_1) \right\rvert \leq \left\lvert x-x_1 \right\rvert \underset{(x,y) \in \mathbb{R}}{max}\left\lvert \frac{F_x(x,y)}{F_y(x,y)}\right\rvert  \leq \left\lvert x-x_1 \right\rvert \frac{\underset{(x,y) \in \mathbb{R}}{max} \left\lvert F_x(x,y) \right\rvert }{\underset{(x,y) \in \mathbb{R}}{min} \left\lvert F_y(x,y) \right\rvert } \neq 0
    \end{gather*}
    Poichè $F_y \in C^0, F_y \neq 0 $ in $\mathbb{R}$\\
    Si evidenzia che $\left\lvert x-x_1\right\rvert $ è un infinitesimo per $x\to x_1$ e il resto è limitato
    \begin{gather*}
        \Rightarrow \lim_{x \to x_1} f(x) = f(x_1) \text{ cioè } f \in C^0(U)
    \end{gather*}
    \fbox{IV} $\exists f'(x) \quad \forall x \in U$ e $f'(x) = -\frac{F_x(x,f(x))}{F_y(x,f(x))} \quad \forall x \in U$\\
    \begin{gather*}
        f'(x_1) = \lim_{x \to x_1} \frac{f(x)-f(x_1)}{x-x_1} \underset{\text{per }}{\boxed{**}} = \lim_{x \to x_1} -\frac{F_x(\xi\tau,\eta_\tau)}{F_y(\xi_\tau,\eta_\tau)} \underset{\text{oss. \fbox{*}}}{=} -\frac{F_x(x_1,fx_1)}{F_y(x_1,f(x_1))}
    \end{gather*}
    \begin{observation}
        \fbox{*} $F_x,F_y \in C^0(U \times V)$\\
        per $x \to x_1: \xi_\tau = (1-\tau) x_1+\tau x \to x_1$\\
        $\eta_\tau = (1-\tau)f(x_1) + \tau \underset{\text{per \fbox{III} }\to f(x_1)}{f(x)} \to f(x_1)$
    \end{observation}


\end{proof}
\begin{observation}
    Vale il th. anche se $\begin{matrix}
        F_y(x_0,y_0) \neq 0 \text{ e solo } F_y \in C^0\\
        F_x(x_0,y_0) \neq 0 \text{ e solo } F_x \in C^0
    \end{matrix}$
\end{observation}
\begin{theorem}
    Dini in 3 variabili\\
    $F(x,y,z) \in C^1(A) \quad A_{ap} \subseteq \mathbb{R}^3$\\
    $P_0 \equiv (x_0,y_0,z_0) \in A$ Regolare e $t.c. \ F(x_0,y_0,z_0) = 0$\\
    supponiamo $F_z(x_0,y_0,z_0) \neq 0$ \underbar{allora}  $\exists \mathcal{U}_{P_0} \subseteq A$\\
    $t.c. \ \{F(x,y,z) = 0\} \cap \mathcal{U}_{P_0}$ è grafico $z=f(x,y)$\\
    inoltre $f\in C^1$ e:\\
    $\frac{\partial f}{\partial x} f(x,y) = -\frac{F_x(x,y,f(x,y))}{F_z(x,y,f(x,y))}$ localmente vicino a $(x_0,y_0)$\\
    $\frac{\partial f}{\partial y} f(x,y) = -\frac{F_y(x,y,f(x,y))}{F_z(x,y,f(x,y))}$
\end{theorem}
\newpage
\section{23/10/25}
\subsection{Curve parametriche}
\begin{definition}
    \begin{gather*}
        \underbar{r}(t) : I \to \mathbb{R}^3 \qquad I = [a,b]\\
        \underbar{r}(t) = \begin{pmatrix}
            x(t)\\ y(t) \\ z(t)
        \end{pmatrix}
        \qquad x,y,z : I \to \mathbb{R}\\
        \gamma \text{ sostegno di } \underbar{r} \quad \gamma = \underbar{r}(I)
    \end{gather*}
\end{definition}
\noindent
Con sostegno si intende proprio il disegno della traiettoria del punto.
\begin{example}
    \begin{gather*}
        \underbar{r}(t) = (1-t)P_0 + \underset{t \in [0,1]}{t} P_1\\
        \underbar{x}(s) = P_0 + tg(s)(P_1-P_0) \quad s \in [0,\frac{\pi}{4}]\\
        \underline{\xi}(\mu) = P_1 + \mu^2(P_0-P_1) \quad \mu \in [0,1]
    \end{gather*}
\end{example}
\begin{definition}
    L'orientazione è il verso di percorrenza del sostegno
\end{definition}
\begin{definition}
    $\underbar{r}(t)$ semplice \underbar{se} non ha autointersezioni cioè $\underbar{r}(t)$ è iniettiva
\end{definition}
\begin{definition}
    $\underbar{r}$ semplice è chiusa \underbar{se} $\underbar{r}(a) = \underbar{r}(b)$
\end{definition}
\begin{example}
    circonferenza (nel piano)
    \begin{gather*}
        \underbar{x}(\theta) = 
        \begin{pmatrix}
            R \cos(\theta) + x_0\\
            R \sin(\theta) + xy0
        \end{pmatrix}\\
        c \equiv (x_0,y_0) \qquad R>0
    \end{gather*}
    \begin{center}
        \begin{tikzpicture}
            \draw[->](0,0) -- (3,0);
        \end{tikzpicture}
    \end{center}
    $\underbar{x}(\theta)$ è semplice e chiusa
\end{example}
\begin{definition}
    $\underbar{r}(t)$ è regolare \underbar{se}:
    \begin{itemize}
        \item è semplice
        \item $x,y,z$ sono derivabili
        \item $\left\lvert \underline{\dot{r}}(t) \right\rvert \neq \underline{0} \quad \forall t \in \mathring{I}$
    \end{itemize}
    L'ultima condizione vuol dire che:
    \begin{gather*}
        \dot{x}^2(t) + \dot{y}^2(t) + \dot{z}^2(t) \neq 0 \quad \forall t \in (a,b)
    \end{gather*}
    \begin{gather*}
        \underline{\dot{r}}(t) = \begin{pmatrix}
            \dot{x}(t)\\
            \dot{y}(t)\\
            \dot{z}(t)
        \end{pmatrix}\quad t \in \mathring{I}
    \end{gather*}
\end{definition}
\begin{definition}
    $\underline{r}(t)$ è Regolare  a tratti \underbar{se}:
    \begin{gather*}
        \underline{r}(t) = \begin{cases}
            \underline{r_1}(t) \text{ se } t \in [a,t_1]\\
            \underline{r_2}(t) \text{ se } t \in [t_1,t_2]\\
            .\\
            .\\
            .\\
            \underline{r_k}(t) \text{ se } t \in [t_k,b]
        \end{cases}
    \end{gather*} 
    Con $\underline{r}_i$ curve regolari, e $i = 1..k$
\end{definition}
\begin{example}
    Prendiamo uno strofoide:
    \begin{gather*}
        \underline{\mathcal{S}}(t) = \begin{pmatrix}
            t^3-t\\
            t^2-1
        \end{pmatrix}\quad t \in \mathbb{R}
    \end{gather*}
    \tiny
    \begin{center}
        \begin{tikzpicture}[x=1cm,y=1cm]
            \draw[->](-2,0) -- (2,0);
            \draw[->](0,-2) -- (0,2);
            \draw[red, variable=\t, samples=50, domain=-1.5:0] 
                plot({\t^3-\t}, {-(\t^2-1) -2});
            \draw[red, variable=\t, samples=50, domain=0:1.5] 
                plot({\t^3-\t}, {\t^2-1});
            \draw[->](0,0) -- (0.3,-0.3) node[at end, below, right]{$\vv{V}(-1)$};
            \draw[->](0,0) -- (0.3,0.3) node[at end, right, above]{$\vv{V}(1)$};
        \end{tikzpicture}
    \end{center}
    \normalsize
    $\mathcal{S}$ ha un autointersezione quindi non è semplice.\\
    Si può vedere che una curva è semplice verificando che due immagini hanno una sola soluzione, quindi:
    \begin{gather*}
        \underline{\mathcal{S}}(t) = \underline{\mathcal{S}}(s) \Leftrightarrow t = s
    \end{gather*}
    E si può verificare in questo modo:
    \begin{gather*}
        \begin{pmatrix}
            t^3-t\\
            t^2-1
        \end{pmatrix} =
        \begin{pmatrix}
            s^3-s\\
            s^2-1
        \end{pmatrix} \Leftrightarrow
        \begin{cases}
            t^4-t = s^3-s\\
            t^2\cancel{-1} = s^2 \cancel{-1}
        \end{cases}
        \Leftrightarrow
        \begin{cases}
            t(t^2-1) = s (s^2-1)\\
            |t| = |s|
        \end{cases}
        \Leftrightarrow t^2= s^2=1 \lor t = s
    \end{gather*}
    Vediamo che in questo caso abbiamo quattro possibilità:
    \begin{gather*}
        t = 1 \land s = -1\\
        t=-1 \land s = 1\\
        (t=s=1 \land t=s=-1)
    \end{gather*}
    Vediamo che in questo caso quindi la condizione iniziale non è soddisfatta infatti $\exists t \neq s \ t.c. \ \underline{\mathcal{S}}(t)=\underline{\mathcal{S}}(s)$ cioè $\underline{\mathcal{S}}$ non iniettiva $\Rightarrow$ non semplice\\
    Prendiamo ora la sua derivata e vediamo i tratti in cui è definita oltre che semplice, anche regolare:
    \begin{gather*}
        \underline{\dot{\mathcal{S}}}(t) =
        \begin{pmatrix}
            3t^2-1\\
            2t
        \end{pmatrix} \quad t \in \mathbb{R} 
    \end{gather*} 
    Possiamo verificare che $\underline{\mathcal{S}}$ è regolare a tratti, infatti è definita regolare nei tratti:
    \begin{gather*}
        \underline{\mathcal{S}}(t) = 
        \begin{cases}
            t \in [1,1]\\
            t \in [-\infty,-1]\\
            t \in [1,+\infty]
        \end{cases}\\
        \text{$\underline{\mathcal{S}}(I)$ sostegno di $\mathcal{S}$}
    \end{gather*}
    Con $I$ l'intervallo su cui è definito $t$
\end{example}
\begin{observation}
    Siccome una curva $\underline{r}(t)$ è regolare in un punto $t_o$ se $\underline{\dot{r}}(t_0) \neq 0$ allora possiamo dire che la Regolarità dipende anche dalla parametrizzazione, due parametrizzazioni infatti possono descrivere lo stesso sostegno ma una può essere regolare e l'altra no, facendo un esempio:
    \begin{gather*}
        r_1(t) = \left(\cos(t) \ , \ \sin(t)\right)  \longrightarrow r_1'(t) = \left(-\sin(t) \ , \ \cos(t)\right)  \neq 0 \forall t \qquad \text{regolare } \cmark\\
        r_2(t) = (\cos(t^3) \ , \ \sin(t^3)) \longrightarrow r_2'(t) = \left( -3t^2\sin(t^3) \ , \ 3t\cos(t^3) \right) \Rightarrow r_2'(0,0) = 0 \qquad \text{ non regolare } \xmark 
    \end{gather*}
\end{observation}
\begin{example}
    Prendiamo l'esempio di un cardioide $\mathcal{C}$:
    \begin{center}
        \begin{tikzpicture}
            \draw[->](0,-1.5) -- (0,1.5);
            \draw[->](-1.5,0) -- (2,0);
            \draw[domain=0:540,scale=1.5,samples=500] plot (\x:{cos(\x/3)^3});
        \end{tikzpicture}
    \end{center}
\begin{gather*}
        \underline{\mathcal{C}}(\theta) = 
        \begin{pmatrix}
            (1 + \cos(\theta)) \cos(\theta)\\
            (1 + \cos(\theta)) \sin(\theta)
        \end{pmatrix}\quad \theta \in [0,\pi]\\
        \underline{\dot{\mathcal{C}}} (\theta) = 
        \begin{pmatrix}
            -\sin(\theta)\cos(\theta) - \sin(\theta)(1 + \cos(\theta))\\
            -\sin^2(\theta) + \cos((\theta)(1 + \cos(\theta)))
        \end{pmatrix}\\
        \underline{\mathcal{\dot{C}}} = 0 \Leftrightarrow 
        \begin{cases}
            - \sin(\theta) (1+2\cos(\theta)) = 0\\
            \cos(\theta) + \cos(2 \theta) = 0
        \end{cases} \text{ se } \theta = \pi \Rightarrow \underline{\dot{\mathcal{C}}}(\theta) = 0\\
        \Longrightarrow  \underline{\mathcal{C}} \text{ non regolare in } \theta = \pi
        \underline{\mathcal{C}}(\theta) =
        \begin{pmatrix}
            (1+\cos(\theta)) \cos(\theta)\\
            (1+\cos(\theta)) \sin(\theta)
        \end{pmatrix}\quad \theta \in [-\pi,\pi]
    \end{gather*}
    È parametrica regolare (equivalente), poichè $\underline{\dot{\mathcal{C}}} \neq 0 \quad \forall \theta \in (-\pi, \pi)$
\end{example}
\begin{definition}
    Presa una curva $\underbar{r}(t):I \to \mathbb{R}^3$ regolare\\
    \underbar{sia} $t_0 \in (a,b) \quad \text{con }(a,b) = I$\\
    e \underbar{sia} $P_0 = \underbar{r}(t_0)$ \\
    si definisce la retta tangente al sostegno di $\underbar{r}(I)$ in $P_0$ come:
    \begin{gather*}
        r_{tg}: x(s) = s \underline{\dot{r}}(t_0) + P_0 \qquad s \in \mathbb{R}
    \end{gather*}
    Dove $s$ è il parametro che generalizza la retta nel punto $P_0$ formandola lungo lo span impostato dall'inclinazione $\underline{\dot{r}}(t_0)$ e traslata di $P_0$ nel piano $\mathbb{R}^3$.
\end{definition}
\begin{definition}
    Presa una curva $\underbar{r}(t):I \to \mathbb{R}^3$ regolare\\
    \underbar{sia} $(a,b) = I$\\
    definiamo il versore tangente a una curva parametrica (ad un tempo $t$ generico) come:
    \begin{gather*}
        \vv{\tau}(t) = \frac{\underline{\dot{r}}(t)}{||\underline{\dot{r}}(t)||} \qquad t \in (a,b)
    \end{gather*}
\end{definition}
\hfil\\
\textbf{Ricordiamo che:}\\
$\underline{r}(t)$ è la traiettoria\\
$\underline{\dot{r}}(t)$ è la velocità\\
$\underline{\ddot{r}}(t)$ è l'accelerazione

\begin{definition}
    Due parametriche:
    \begin{gather*}
        \underline{\varphi}:[a,b] \to \mathbb{R}^3\\ 
        \underline{\psi}: [\alpha,\beta] \to \mathbb{R}^3 
    \end{gather*}
    sono equivalenti se:
    \begin{gather*}
        \exists g : I \to J \quad g \in C^1 \ t.c. \ I = [a,b] \quad J = [\alpha,\beta]\\
        g'(t) \neq 0 \quad \forall t \in \mathring{I} \ t.c. \ \underline{\varphi}(t) = \underline{\psi}(g(t)) \quad \forall t \in I
    \end{gather*}
    \begin{observation}
        $g$ è invertibile
    \end{observation}
    \begin{observation}
        Se $\underline{\varphi}, \underline{\psi}$ sono equivalenti \underbar{allora}:\\
        $\vv{\tau}_{\underline{\varphi}} = \vv{\tau}_{\underline{\psi}}$ nei punti corrispondenti\\
        oppure $\vv{\tau}_\varphi = - \vv{\tau}_{\underline{\psi}}$
    \end{observation}
\end{definition}
\subsection{lunghezza di una curva}
\begin{center}
    \begin{tikzpicture}
        \draw(0,0) ..controls(1,-1) and(2,1).. (3,0);
        \node at(0,0)[left]{$P_0 \equiv A$};
        \node at(3,0)[right]{$P_k \equiv B$};
    \end{tikzpicture}
\end{center}
Si chiama poligonale inscritta:
\begin{gather*}
    P_i \equiv P(t_i) = \underline{r}(t_i)\\
    \mathcal{L}(\underline{r}) \geq \mathcal{L}(\wp)
\end{gather*}
\begin{observation}
    \begin{gather*}
        P_i \in \gamma \qquad i = 0...k
    \end{gather*}
\end{observation}
\begin{definition}
    $\underline{r}(t)$ è rettificante se:
    \begin{gather*}
        sup\{\mathcal{L}(\wp) \wp \text{poligonali inscritte}\} = L < +\infty
    \end{gather*}
    in tal caso $\mathcal{L}(\underline{r}) = L$

\end{definition}

\begin{theorem}
    (di rettificabilità) \\
    Presa una curva parametrica $\varphi: [a,b] \to \mathbb{R}^3$ quindi $\underline{\varphi} = \begin{pmatrix}
        x(t)\\ y(t)\\ z(t)
    \end{pmatrix}$\\
    \underbar{Sia} $\underline{\varphi} \in C^1((a,b))$\\
    \underbar{e} Regolare a tratti\\
    \tab \underbar{\textbf{allora}} $ \underline{\varphi}$ è rettificabile e $\mathcal{L}(\varphi) = \int_{a}^{b} ||\underline{\varphi' }(t)|| dt = \int_{a}^{b} \sqrt{\dot{x}^2(t)+\dot{y}^2(t)+\dot{z}^2(t)} dt$
\end{theorem}
\begin{proposition}
    Prese due curve parametriche $\underline{\varphi}, \underline{\psi}$\\ 
    \underbar{se} sono parametriche equivalenti \\
    \tab$\Rightarrow \mathcal{L}(\varphi) = \mathcal{L}(\underline{\psi})$\\\\
    cioè $ \int_{a}^{b} | \underline{\varphi}'(t) | dt = \int_{\alpha}^{\beta} | \underline{\psi}' (s) | ds := \mathcal{L}(| \gamma |)$ quindi dipende solo dal sostegno
\end{proposition}
\subsection{ascissa curvilinea o parametro d'arco}
\begin{definition}
    Presa una curva parametrica $\underline{r}(t):[a,b] \to \mathbb{R}^3$\\
    \underbar{sia} $\underline{r}(t)$ Regolare\\
    \underbar{sia} $t_0 \in [a,b]$ \tiny solitamente si considera il punto iniziale quindi $t_0 = a$\normalsize \\
    Viene definita l'ascissa curvilinea $S(t)$ nel seguente modo:\\ $\forall t \in [a,b] \quad S(t) = \int_{t_0}^{t}  \left\lVert \underline{\dot{r}}(\ell)\right\rVert  d\ell$ \tiny si cambia variabile da $t$ a $\ell$ perchè la variabile di integrazione e gli estremi di integrazione non possono essere uguali\normalsize\\
    cioè la lunghezza dell'arto della curva $\gamma$ tra $\underline{r}(t_0)$, che posso chiamare $P_0$, e $\underline{r}(t)$\\
    con $S(t)$ ascissa curvilinea (di $\underline{r}$) centrata in $t_0$ \\
    \end{definition}
    Considero ora $t_0 = a$:
    \begin{itemize}
        \item $S:[a,b] \to [0,L] \qquad L = \mathcal{L}(\gamma)$
        \item $S(t) = \int_{a}^{b} | \underline{\dot{r}} (\tau)| d\tau \qquad s(a) = 0 \ s(b) = 0$
        \item $s \nearrow , \dot{s}(t) = | \underline{\dot{r}} |> 0$ con $s \in C^1 $
    \end{itemize}
    l'ultimo punto ci conferma che è cambio di parametro ammissibile
\begin{gather*}
    \exists t(s)\\
    s= s(t) \text{ cambio di parametro}
    \boxed{\underline{\varphi} = \underline{r}(t(s))}
\end{gather*}
parametrizzazione rispetto all'ascissa curvilinea
\begin{gather*}
    \underline{\dot{\varphi}}(s) = \underline{\dot{r}}(t(s)) \dot{t}(s) \underset{\boxed{*}}{=} \underline{\dot{r}}(t(s)) / |\underline{\dot{r}}(t(s))|
\end{gather*}
\fbox{*}: ma $t(s)$ è inversa di $s(t)$ quindi:
\begin{gather*}
    \dot{t}(s) = \frac{1}{\dot{s}(t(s))} = \frac{1}{| \underline{\dot{r}} t(s) |}
\end{gather*}

\newpage
\section{28/10/25}
\subsection{curve in forma polare}
\begin{center}
    \begin{tikzpicture}
        \draw[->](0,0) -- (0,3);
        \draw[->](0,0) -- (3,0);
        \draw[->,cyan](0,0) -- (2,1) node[at end, right]{$P(\theta)$};
    \end{tikzpicture}
\end{center}

Un esempio è una circonferenza:
\begin{gather*}
    \rho(\theta) = R>0
\end{gather*}
Equazioni in forma polare $\rho = rho(\theta) \quad$ con $\rho(\theta) > 0 \qquad $ con $\theta \in I$\\
\begin{gather*}
    \underline{r}(theta) = \begin{pmatrix}
        \rho(\theta) \quad \cos(\theta)\\
        \rho(\theta) \quad \sin(\theta)
    \end{pmatrix}\theta \in I
\end{gather*}
\begin{example}
    $\rho(\theta) = A\theta \quad A>0, \theta \geq 0$
    \begin{center}
        \begin{tikzpicture}
            \draw[->](0,-1) -- (0,3);
            \draw[->](-2,0) -- (2,0);
        \end{tikzpicture}
    \end{center}
\end{example}
\subsection{integrali di linea di prima specie}
\begin{gather*}
    \int_{a}^{b} g(t) dt
\end{gather*}
\begin{center}
    \begin{tikzpicture}
        \draw[->](-1,0) -- (3,0);
        \draw[red](0,1) ..controls(0.5,2) and(1.5,-1).. (2,1);
        \draw[dashed](0,1) -- (0,0) node[at end,below]{$a$};
        \draw[dashed](2,1) -- (2,0) node[at end,below]{$b$};
    \end{tikzpicture}
\end{center}
\begin{gather*}
     \left\lVert \underline{\dot{r}}(t) \right\rVert  =\left\lVert \sqrt{\dot{x}^2(t) + \dot{y}^2(t)} dt\right\rVert 
\end{gather*}
L'area del sottografico di $f(x,y)$ lungo la curva $\underline{r}(t)$ è proprio
\begin{gather*}
    \int_{a}^{b} f(x(t),y(t)) \sqrt{\dot{x}^2(t) + \dot{y}^2(t)} dt = \int_{\gamma} f(x,y) ds 
\end{gather*}
\begin{theorem}
    $\underline{x}(t):[a,b] \to \mathbb{R}^2$ parametrizzazioni equivalenti regolari (a tratti di $\gamma$)\\
    $\underline{r}(u):[\alpha,\beta] \to \mathbb{R}^2$\\
    $f:a \to \mathbb{R} \quad A \subseteq \mathbb{R}^2, \gamma \subseteq A$\\
    \underline{Allora} $\int_{a}^{b} f(\underline{x}(t))\left\lVert \underline{\dot{x}} (t)\right\rVert dt = \int_{\alpha}^{\beta} f(\underline{r}(u)) \left\lVert \underline{\dot{r}}(u) \right\rVert du$\\
    e quindi è ben definito $\int_{\gamma} f ds = \int_{a}^{b} f(\underline{x}(t)) \left\lVert \underline{\dot{x}}(t)\right\rVert dt$\\
    Quindi non dipende dalla parametrizzazione scelta. 
\end{theorem}
\begin{observation}
    \begin{gather*}
        f \equiv 1 \quad \int_{\gamma} f ds = \int_{a}^{b} \left\lVert \dot{x}(t) \right\rVert dt = \mathcal{L}(\gamma)
    \end{gather*}
\end{observation}
\begin{proof}
    \hfil\\
    sia $g:[a,b] \to [\alpha,\beta]$ cambio di paraetro $t \mapsto u=g(t)$\\
    supponiamo $g'\underset{\textcolor{yellow}{<}}{>}0$ (viene mantenuta l'orientazione)\\
    $\underline{x}(t) = \underline{r}(g(t))$
    Specifichiamo la seguente uguaglianza che viene da questo termine
    \begin{gather*}
        \left\lVert \underline{\dot{x}}(t)\right\rVert = \left\lVert \underline{\dot{r}}(g(t)) g'(t) \right\rVert = \left\lVert \underline{\dot{r}}(g(t)) \right\rVert \left\lVert g'(t) \right\rVert = g'(t) \left\lVert \underline{\dot{r}}(g(t)) \right\rVert \qquad g'>0    
    \end{gather*}
    Ora si prosegue con i conti:
    \begin{gather*}
        \int_{\alpha}^{\beta} f(\underline{r}(u)) \left\lVert \underline{\dot{r}}(u) \right\rVert du = \\
        \text{sostituzione: } u=g(t) \quad du = g'(t) dt\\
        = \int_{\underset{\textcolor{yellow}{b}}{a}}^{\overset{\textcolor{yellow}{a}}{b}} f(\underline{r}(g(t))) \left\lVert \underline{\dot{r}}(g(t)) \right\rVert g'(t) dt = \int_{a}^{b} f(\underline{x}(t)) \left\lVert \underline{\dot{x}}(t) \right\rVert dt  
    \end{gather*}
\end{proof}
\subsection{applicazioni} 
$\gamma$ sostegno di $\underline{x}(t):[a,b] \to \mathbb{R}^3$\\
\bulletout \underbar{Massa di un filo con densità}:
\begin{gather*}
    \rho(x,y,z) \qquad M_{\gamma} = \int_{\gamma} \rho ds = \int_{a}^{b} \rho(x(t),y(t),z(t)) \left\lVert \underline{\dot{x}}(t)\right\rVert dt\\
    \left( \rho(t) \qquad M_\gamma = \int_{a}^{b} \rho(t) \left\lVert \underline{\dot{x}}(t) \right\rVert dt  \right)  
\end{gather*} 
\bulletout \underbar{Centro di massa di un filo materiale}:
\begin{gather*}
    G \equiv (x_G,y_G,z_G)\\
    x_G  = \frac{1}{m_\gamma} \int_{\gamma} \rho x ds = \frac{1}{m_\gamma} \int_{a}^{b} \rho(\underline{x}(t)) x(t) \left\lVert \underline{\dot{x}}(t) \right\rVert dt \\
    y_G ,z_G
\end{gather*}
\begin{observation}
    se $\varphi \equiv 1, M_\varphi = \mathcal{L}_\gamma$\\
    Centro di massa $\equiv$ baricentro geometrico.
\end{observation}
\begin{example}
    $\underline{x} = \begin{pmatrix}
        t^2\\
        t^3+1
    \end{pmatrix} t \in[0,2]$\\
    Sia $\gamma$ il sostegno:
    \begin{enumerate}[a)]
        \item provare che $\underline{x}$ è parametrizzazione regolare
        \item Calcolare $\mathcal{L}(\gamma) = L$
        \item calcolare il baricentro geometrico
        \item calcolare massa di un filo materiale di profilo $\gamma$ e densità
        \begin{itemize}
            \item $\varphi = \sqrt{1 + \frac{9}{4}x}$
            \item $\rho(t) = t$
        \end{itemize}
    \end{enumerate}
    \begin{gather*}
        \underline{\dot{x}}(t) = \begin{pmatrix}
            2t\\
            3t^2
        \end{pmatrix}t \in (0,2) \qquad \left\lVert \underline{\dot{x}}(t) \right\rVert = \sqrt{4t^2 + 9t^4} = t\sqrt{4+9t^2} \qquad \forall t \in (0,2)\\
        B \equiv (x_B,y_B)\\
        L(x_B) = \int_{\gamma} x ds = \int_{0}^{2} t^2 t \sqrt{4+9t^2} dt \overset{P.P.}{=}\\
        t^2 \frac{1}{27}(4+9t^2)^\frac{3}{2} |_0^2 - \int_{0}^{2} \frac{2t}{27}(4+9t^2)^\frac{3}{2} dt = \\
        \frac{t^2(4+9t^2)^2}{27}|_0^2 -\frac{2}{27} (4+9t^2)^\frac{5}{2} |_0^2\\\\
        L(y_B) = \int_{\gamma} = \int_{0}^{2} (t^3+1)t \sqrt{4+9t^2} dt = \\
        \int_{0}^{2} t^3t\sqrt{4+9t^2} dt = \int_{0}^{2} t \sqrt{4+9t^2} dt\\
        '' \ \text{per es}\\
        M_\gamma = \int_{\gamma} \rho ds = \int_{0}^{2} \rho(x(t)) , y(t) \left\lVert \dot{x}(t) \right\rVert dt\\
        \Rightarrow \int_{0}^{2} \sqrt{1 + \frac{9}{4}t^2} t \sqrt{4+9t^2} dt = \int_{0}^{2} \frac{1}{2} t(4+9t^2) dt  
    \end{gather*}
\end{example}
\underline{esercizio del teorema dini su foto cell}\\

\newpage
\section{29/10/25}
\subsection{forme differenziali}
sono un modo matematico per esprimere i campi vettoriali
\begin{definition}[Campo vettoriale]
    \underline{sia} $A \subseteq \mathbb{R}^3$ un campo vettoriale su $A$\\
    è una funzione $F:A \to \mathbb{R}^3$\\
    Se $\overset{\in A}{(x,y,z)} \overset{F}{\to} \underset{(F_1(x,y,z),F_2(x,y,z),F_3(x,y,z))}{F(x,y,z)} \in \mathbb{R}^3$\\
    le componenti si possono scrivere anche con la forma con i versori: $F_1(x,y,z)\underline{i},F_2(x,y,z)\underline{j},F_3(x,y,z)\underline{k}$
\end{definition}
\begin{example}
    massa puntiforme $m$ che si trova nel punto $P_0=(x_0,y_0,z_0)$, il campo sarà:
    \begin{gather*}
        F(x,y,z) = -c \ m \ \frac{x-x_0,y-y_0,z-z_0}{((x-x_0)^2+(y-y_0)^2+(z-z_0)^2)^{\frac{3}{2}}} = -c \ m \ \frac{(P-P_0)}{\left\lVert P-P_0 \right\rVert^\frac{3}{2} }
    \end{gather*}
    \begin{center}
        \begin{tikzpicture}
            \draw[dashed](0,0) -- (3,1);
        \end{tikzpicture}
    \end{center}
    Qui $c>0$,$P=(x,y,z)$
\end{example}
\begin{example}
    campo di velocità di un corpo rigido che ruota intorno all'asse $z$ con velocità ancolare $\Omega$
    \begin{gather*}
        v(x,y,z) = -\Omega_y \underline{i} + \Omega_x \underline{j} + o\underline{k}
    \end{gather*}
    \begin{center}
        \begin{tikzpicture}
            \draw(0,0) circle (10pt);
            \draw(0,0) circle (20pt);
        \end{tikzpicture}
    \end{center}
\end{example}
\begin{example}
    \underline{Se} $f:A \to \mathbb{R}$ differenziabile in ogni punto di $A$\\
    \underbar{allora} $\nabla f:A\to \mathbb{R}^3$
    \begin{gather*}
        \nabla f(x,y,z) := \left( \frac{\partial f}{\partial x},\frac{\partial f}{\partial y},\frac{\partial f}{\partial z} \right)
    \end{gather*}
    \textbf{Esempio:} se $f(x,y,z) = x^2,y^2,z^2$
    \begin{gather*}
        \nabla f=(2x,2y,2z);
    \end{gather*} 
\end{example}
\begin{definition}[forma differenziale]
    \underline{Sia} $V$ uno spazio vettoriale,\\
    si considera l'insieme: $\text{applicazioni lineari} :V \to \mathbb{R}$, per definizione questo si indica il duale di $V$ come $V^*$\\
    Consideriamo ora $L \in (\mathbb{R}^{3})^*$
    \begin{gather*}
        a_1=L(e_1)\\
        a_2=L(e_2)\\
        a_3=L(e_3)\\
    \end{gather*}
    Preso $v \in \mathbb{R}^3$ si ha che $L(v) = a_1v_1+a_2v_2+a_3v_3$\\
    definisco una base di $(\mathbb{R}^3)^*$
    \begin{gather*}
        dx_1 : \mathbb{R}^3 \to \mathbb{R} \qquad (dx_1)(\overset{\in \mathbb{R}^3}{v}) = v_1\\
        dx_2 : \mathbb{R}^3 \to \mathbb{R} \qquad (dx_2)(\overset{\in \mathbb{R}^3}{v}) = v_2\\
        dx_3 : \mathbb{R}^3 \to \mathbb{R} \qquad (dx_3)(\overset{\in \mathbb{R}^3}{v}) = v_3
    \end{gather*}
    Così posso scrivere $L$ come:
    \begin{gather*}
        L = a_1 dx_1 + a_2 dx_2 + a_3 dx_3
    \end{gather*}
    Si ricorda a questo punto che $L(v)$può essere riscritto come:
    \begin{gather*}
        L(v) = a_1 dx_1(v) + a_2 dx_2(v) + a_3 dx_3(v) = (a_1 dx_1 + a_2 dx_2 + a_3 dx_3)(v)
    \end{gather*}
    \underline{Sia} $A \subseteq \mathbb{R}^3$, una forma differenziale $\omega$ su $A$, è un'applicazione che ad ogni elemento di $A$ associa un elemento di $(\mathbb{R}^3)^*$ 
    \begin{gather*}
        (x,y,z) \in A \longrightarrow L (x,y,z) \in (\mathbb{R}^3)^*
    \end{gather*}
    Gli elementi costanti dipendono però dall'elemento $(x,y,z)$ di partenza:
    \begin{gather*}
        L(x,y,z) = a_1(x,y,z) dx_1 + a_2(x,y,z) dx_2 + a_2(x,y,z) dx_3
    \end{gather*}
    A questa forma differenziale posso associare il campo vettoriale ($F$):
    \begin{gather*}
        F(x,y,z) = (a_1(x,y,z),a_2(x,y,z),a_3(x,y,z))
    \end{gather*}
\end{definition}
\begin{proposition}
    \underbar{se} $f:A \to \mathbb{R}$ è una funzione differenziabile \underbar{allora} chiamo differenziale di $f$: $df$\\
    \begin{gather*}
        df(x,y,z)  = \frac{\partial f}{\partial x}(x,y,z) dx_1 + \frac{\partial f}{\partial y}(x,y,z) dx_2 + \frac{\partial f}{\partial z}(x,y,z) dx_3
    \end{gather*}
    Questo puo essere visto come il gradiente della forma differenziale.
\end{proposition}
Ci si chiede ora quali forme differenziali siano esatte 
\begin{definition}
    \underline{Sia} $\omega:A \to (\mathbb{R}^3)^*$ una forma differenziale $\omega$ si dice \textbf{esatta} \underline{se}\\
    $f:A \to \mathbb{R}$ differenziabile $t.c. \ \omega = df$\\
    cioè (se $\omega= a_1(x,y,z)dx_1+a_2(x,y,z)dx_2+a_3(x,y,z)dx_3$) \underbar{se} $\exists f$ $t.c.$
    \begin{gather*}
        a_1=\frac{\partial f}{\partial x_1} \qquad a_2=\frac{\partial f}{\partial x_2} \qquad a_3=\frac{\partial f}{\partial x_3}
    \end{gather*}
    Tale $f$ si chiama \textbf{primitiva}
\end{definition}
\begin{observation}
    \underbar{se} $f$ è primitiva di $\omega$ anche $f+c$ lo è
\end{observation}
\begin{definition}[Campo conservativo]
    \underbar{Sia} $F:A \to \mathbb{R}^3$ un campo vettoriale $F=(F_1,F_2,F_3)$\\
    $F$ si dice \textbf{conservativo} \underbar{se} $\exists f:A \to \mathbb{R}$ differenziabile $t.c.$ $F=\nabla f$\\
    In più la funzione $f$ si chiama potenziale di $F$.
\end{definition}
\begin{example}
\begin{gather*}
    F(x,y,z) = -c \ m \ \frac{(x-x_0,y-y_0,z-z_0)}{((x-x_0)^2+(y-y_0)^2+(z-z_0)^2)^{\frac{3}{2}}}
\end{gather*}
è conservativo e un suo potenziale è:
\begin{gather*}
    f(x,y,z) = c \ m / ((x-x_0)^2+(y-y_0)^2+(z-z_0)^2)^{\frac{3}{2}}
\end{gather*}
Infatti ad esempio:
\begin{gather*}
    \frac{\partial f}{\partial x} = \frac{\partial }{\partial x} c \ m \ ((x-x_0)^2+(y-y_0)^2+(z-z_0)^2)^{-\frac{1}{2}}=\\
    c \ m (-\frac{1}{\cancel{2}})("")^{-\frac{3}{2}} \cancel{2} (x-x_0)
\end{gather*}
\end{example}
\hfil\\
\textbf{Ogni campo è conservativo?}\\
\textbf{Ogni forma differenziale è esatta?}\\
Ovviamente no\dots
\begin{example}
    siamo in $\mathbb{R}^2$ e  $\omega(x,y) = 3x^2 \ dx -xy \ dy$\\
    non è esatta. infatti se lo fosse e $\omega = df$
    \begin{gather*}
            \frac{\partial f}{\partial x}(x,y) = 3x^2\\
            \frac{\partial f}{\partial y}(x,y) = -xy\\
            \frac{\partial^2 f}{\partial y \partial x} = 0\\
                \frac{\partial^2 f}{\partial x \partial y} = -y\\
            f(x,y) = x^3+c(y)\\
            \frac{\partial f}{\partial y} = 0 + c'(y)
    \end{gather*}
\end{example}
\begin{proposition}
    Lavoro di un campo vettoriale lungo una curva orientata (o integrale curvilineo di II specie)
    \begin{gather*}
        F:A \to \mathbb{R}^3\\
        \gamma \text{ curva orientata } \subset A
    \end{gather*}
    \underbar{sia} $\gamma$ regolare a tratti, $F$(campo vettoriale) sia continua(il che vuol dire che entrambe le sue componenti sono continue)\\
    $\varphi:[a,b] \to A$ una parametrizzazione regolare di $\gamma$ concorde con l'orientamento di $\gamma$\\
    \begin{gather*}
        \varphi(a) = P_0 , \varphi(b) = P_1, \tau(t) = \frac{\dot{\varphi(t)}}{\left\lVert \dot{\varphi(t)} \right\rVert }
    \end{gather*}
\end{proposition}
\begin{definition}
    Si definisce lavoro di $F$ su $\gamma$ come:
    \begin{gather*}
        \int_{a}^{b} \left\langle F(\varphi(t));\dot{\varphi}(t) \right\rangle \left\lVert \dot{\varphi}(t) \right\rVert dt
    \end{gather*}
    Può essere scritto anche come:
    \begin{gather*}
        \int_{\gamma} \left\langle F;\tau \right\rangle ds 
    \end{gather*}
    Dove ogni punto di $\gamma \quad \tau$ è un versore tangente a $\gamma$ concorde con l'orientazione di $\gamma$ 
\end{definition}
\hfil\\
Infatti:
\begin{gather*}
    \int_{\gamma} \left\langle F ; \tau\right\rangle ds = \int_{a}^{b} \left\langle F(\varphi(t)); \tau(\varphi(t)) \right\rangle \left\lVert \dot{\varphi(t)} \right\rVert dt\\
    \text{con } \tau(\varphi(t)) = \int_{a}^{b} \left\langle F(\varphi(t)); \frac{\dot{\varphi(t)}}{\cancel{\left\lVert \dot{\varphi(t)} \right\rVert}} \right\rangle \cancel{\left\lVert \dot{\varphi(t)} \right\rVert} dt
\end{gather*}
\textbf{Dipende dall'orientazione su $\gamma$}:
\begin{gather*}
    \int_{\gamma^-} \left\langle F; \tau \right\rangle ds = \int_{\gamma^+} \left\langle F; \tau\right\rangle ds 
\end{gather*}
\begin{example}
    $F(xy,y,0)$ con $\qquad \varphi_1 = \cos(t),\sin(t),0 \quad t \in [0,2\pi]$\\
    contenuta nel piano $\{ z = 0 \}$ di $\mathbb{R}^3$
    \begin{gather*}
        \int_{0}^{2 \pi} \left\langle (\cos(t),\sin(t),0);(-\sin(t),\cos(t),0) \right\rangle dt =\\
        \int_{0}^{2\pi} -\sin(t) \cos(t) +\sin(t) \cos(t) dt = \left[ \frac{-\sin^3(t)}{3} + \frac{\sin^2(t)}{2} \right]_0^{2\pi} = 0 
    \end{gather*}
    Prendo ora un'altra curva: $\varphi_2(t)= (2,0,0) + (\cos(t,\sin(t),0)) = (2+\cos(t),\sin(t),0) \qquad t \in [-\pi,\pi]$
    \begin{gather*}
        w_2 = \int_{-\pi}^{+\pi} \left\langle ((2+\cos(t))\sin(t),\sin(t),0);(-\sin(t),\cos(t),0) \right\rangle dt =\\
        \int_{-\pi}^{\pi} -2\sin^2(t) + \text{l'integrale di prima} =\\
        \left[ -2(t+\frac{1}{4}\sin(2t)) - \frac{\sin^3(t)}{3} + \frac{\sin^2(t)}{2} \right]_{-\pi}^{\pi} = -2\pi  
    \end{gather*}
\end{example}


\newpage
\section{30/10/25}
\begin{proposition}
    \underbar{sia} $\omega : A \to (\mathbb{R}^3)^* \quad A \subset \mathbb{R}^3$ una forma differenziale continua\\
    \underbar{se} $\omega = a_1 dx_1 + a_2 dx_2 + a_3 dx_3$ con $dx$ funzioni $A \to \mathbb{R}$\\\\
    \underbar{sia} $\gamma$ unca curva orientata con supporto $\subset A$ regolare a tratti e \underbar{sia} $\varphi(t):[a,b] \to A$ una sua parametrizzazione concorde cn l'orientazione di $\gamma$\\\\
    \begin{center}
        \begin{tikzpicture}
            \draw(0,0) circle (60pt);
            \draw(-1.5,0) ..controls(0,1) and(1,-1).. (2,0);
            \draw[->](0,0) -- (1,1);
        \end{tikzpicture}
        c'è poi un punto x e il versore a quel punto.
    \end{center}
    Definisco:
    \begin{gather*}
        \int_{\gamma} \omega: \int_{\gamma} \omega(x) [\tau(x)] ds
    \end{gather*}
    ed è proprio l'applicazione lineare da $\mathbb{R}^3 \to \mathbb{R}$
\end{proposition} con componente a $x$ applicata al variare dela tangente $\tau(x)$ alla curva $\gamma$ nel suo punto $x$.
\begin{gather*}
    = \int_{a}^{b} \underbrace{\omega(\varphi(t)) \left[ \frac{\dot{\varphi}(t) }{\left\lVert \dot{\varphi}(t)  \right\rVert }\right]}_{\frac{\omega(\varphi(t)) \dot{\varphi}(t)}{ \cancel{\left\lVert \dot{\varphi}(t) \right\rVert} }} \cancel{\left\lVert \dot{\varphi}(t) \right\rVert } \\
    \int_{a}^{b} (a_1(\varphi(t)) dx_1 + a_2(\varphi(t)) dx_2 + a_3(\varphi(t)) dx_3)\left[ \dot{\varphi_1(t)} \right] dt\\
    \boxed{\int_{a}^{b} a_1(\varphi(t)) \dot{\varphi_1(t)} + a_2(\varphi(t)) \dot{\varphi_2(t)} + a_3(\varphi(t)) \dot{\varphi_3(t)} dt}
\end{gather*}
Non dipende dalla parametrizzazione.Dipende dal verso:
\begin{gather*}
    \int_{\gamma^+} \omega = -\int_{\gamma^-} \omega
\end{gather*}
\begin{example}
    $\omega(x,y) = ydx -xy dy$\\
    $\gamma : (\cos(t), \sin(t)) \quad t \in [0,\pi]$ percorsa in senso orario
    \begin{center}
        \begin{tikzpicture}
            \draw[->](0,0) circle (20pt);
            \draw[->](20pt,0) -- (20pt,0); 
            \fill[fill = page](-1,0) rectangle (1,-2);
            \draw[->](0,0) -- (0,2);
            \draw[->](-1,0) -- (2,0);
        \end{tikzpicture}
    \end{center}
    \begin{gather*}
        \int_{\gamma} \omega = \int_{0}^{\pi} \sin(t) (-\sin(t)) + (-\sin(t)\cos(t)) \cos(t) \ dt = \int_{0}^{\pi} \sin^2(t) + \sin(t) \cos^2(t) \ dt = \left[ \frac{t}{2} - \frac{1}{4} \sin(2t) -\frac{\cos^3(t)}{3} \right]_0^\pi = \frac{\pi}{2}  + \frac{2}{3}
    \end{gather*}
\end{example}

\begin{theorem}[integrazione delle forme differneziali esatte]
    \underbar{sia} $\omega$ una forma differenziale \textbf{esatta} e \textbf{continua} definita nell'aperto di $A$ di $\mathbb{R}^3$\\
    \underbar{sia} $\gamma$ una curva \textbf{regolare} a tratti con sostegno contenuto in $A$ di estremi $P_0$ e $P_1$, orientata nel verso che va da $P_0$ a $P_1$\\
    \underbar{sia} $f$ una primitiva di $\omega$\\
    \underbar{Allora}:
    \begin{gather*}
        \int_{\gamma} \omega = f(P_1) - f(P_0) 
    \end{gather*}
\end{theorem}
    \begin{center}
        \begin{tikzpicture}
            \draw(0,0) circle (60pt);
            \draw(-1.5,0) ..controls(0,1) and(1,-1).. (2,0);
            \draw[->](0,0) -- (1,1);
        \end{tikzpicture}
    \end{center}
    \begin{proof}
        In gradiente: $\omega$ è esatta e $f$ è una sua primitiva. Questo vuol dire che:
        \begin{gather*}
            a_1(x_1,x_2,x_3) = \frac{\partial f}{\partial x_1}(x_1,x_2,x_3)\\
            a_2(x_1,x_2,x_3) = \frac{\partial f}{\partial x_2}(x_1,x_2,x_3)\\
            a_3(x_1,x_2,x_3) = \frac{\partial f}{\partial x_3}(x_1,x_2,x_3)\\
        \end{gather*}
        \underbar{Sia $\varphi:[a,b] \to A$} una parametrizzazione di $\gamma$ concorde con l'orientazione, in particolare:
        \begin{gather*}
            \varphi(a) = P_0 \qquad \varphi(b) = P_1\\
            \int_{\gamma} \omega = \int_{a}^{b} \underbrace{\left[ \frac{\partial f}{\partial x_1}\dot{\varphi_1}(t) + \frac{\partial f}{\partial x_2}\dot{\varphi_2}(t) + \frac{\partial f}{\partial x_3}\dot{\varphi_3}(t)\right]}_{\frac{d}{dt} f(\varphi(t))}  \ dt\\
            \int_{a}^{b} \frac{d}{dt} f(\varphi(t)) \ dt \underset{th.f.c.i.}{=} f(\varphi(t)) |_a^b = f(\varphi(b)) - f(\varphi(a)) = f(P_1) - f(P_2)
        \end{gather*}
    \end{proof}
    Il lavoro di questo campo lungo una curva è uguale al potenziale calcolato nel punto finale meno il potenziale calcolato nel punto iniziale
\begin{theorem}[caratterizzazione delle forme esatte]
    \underbar{sia} $A \subset \mathbb{R}^3$ un \textbf{aperto connesso}, $\omega$ una forma differenziale continua definita su $A$ e $\gamma,\gamma_1,\gamma_2$ curve regolari a tratti con con sostegno contenuto in $A$. le tre proprietà seguenti sono equivalenti:
    \begin{enumerate}
        \item $\omega$ è esatta
        \item per ogni curva chiusa $\gamma$ contenuta in $A$ risulta $\int_{\gamma} \omega = 0$
        \item Se $\gamma_1$ e $\gamma_2$ hanno gli stessi estremi e lo stesso verso di percorrenza si ha $\int_{\gamma_1} \omega = \int_{\gamma_2} \omega$
    \end{enumerate} 
\end{theorem}
\begin{proof}
    \bulletout $1 \rightarrow 2$\\
    \begin{center}
        \begin{tikzpicture}
            \draw(0,0) circle (1);
            \filldraw[fill=black](1,0) circle(2pt) node[right]{$P_0 = P_1$};
            \node at(0,1)[above]{$\gamma$};
        \end{tikzpicture}
    \end{center}
    dal teorema precedente
    \begin{gather*}
        \int_{\gamma} \omega = f(P_1) - f(P_0) = 0
    \end{gather*}
    \bulletout $2 \rightarrow 3$\\
    \begin{center}
        
    \end{center}
    \begin{gather*}
        \gamma_1 \cup \gamma_2^- 
    \end{gather*}
    \begin{center}
        
    \end{center}
    \begin{gather*}
        0 = \int_{\gamma_1 \cup \gamma_2} \omega = \int_{\gamma_1} \omega + \int_{\gamma_2^-} \omega = \int_{\gamma_1} \omega -\int_{\gamma_2} \omega\\
        \Rightarrow \int_{\gamma_1} \omega = \int_{\gamma_2} \omega 
    \end{gather*}
    \bulletout $3 \rightarrow 1$\\
    %\begin{figure}[tbh]
    %\centering
    %\includesvg[width = 150 pt]{anal2}
    %\caption{Grafico di $f$ generica in due variabili}\label{fig:testsvg}
    %\end{figure}
    \begin{gather*}
        f(x_1,x_2,x_3) = \int_{\gamma} \omega \\
        f(x_1+h,x_2,x_3) = \int_{\gamma \cup \varphi} \omega
    \end{gather*}
    \begin{gather*}
        \frac{f(x_1+h,x_2,x_3)}{h} = \int_{\gamma \cup \varphi} \omega - \int_{\gamma} \omega\\
        = \int_{\gamma} \omega + \int_{\varphi} - \int_{\gamma} \omega = \frac{1}{h} \int_{0}^{h} \left( a_1(x_1+t,x_2,x_3) 1 + a_2(") 0 + a_3(") 0 \right)dt
        = \frac{1}{h} \int_{0}^{h} a_1(x_1+t,x_2,x_3) dt \quad \text{se chiamato } P(h) = \int_{0}^{h} a_1(x_1+t,x_2,x_3) \quad P(0) = 0\\
        \lim_{h \to 0} \frac{f(x_1+h,x_2,x_1) - f(x_1,x_2,x_3)'}{h} = \lim_{h \to 0} \frac{P(h) - P(0)}{h} = P'(0) =
    \end{gather*}
    Il t.f.c.i. garantisce che $P'(0) \exists$ e che coincide con l'integranda calcolata per $t=0$. cioè con $a_1(x_1,x_2,x_3)$\\\\
    Ho quindi dimostrato che:
    \begin{gather*}
        \frac{\partial f}{\partial x_1} (x_1,x_2,x_3) = a_1(x_1,x_2,x_3)
    \end{gather*}
\end{proof}
\begin{example}
    $\omega = d(x^2y+\frac{y^3}{3}) = 2xd \ dx + (x^2+y^2) \ dy$
    \begin{gather*}
        A = \mathbb{R}^2 \qquad P_0 = (0,0) \qquad P=(x,y) \qquad \gamma:(tx,ty) \ t \in [0,1]
    \end{gather*}
    \begin{center}
        \begin{tikzpicture}
            \draw(-1,0) -- (2,0);
            \draw(0,0) -- (0,2);
            \draw[->](0,0) -- (1,1) node[right]{$P=(x,y)$};
            \filldraw[fill = black](0.5,0.5) circle (2pt) node[below]{$\gamma$};
            \node at(0.5,0.5)[above]{$(tx,ty)$};
        \end{tikzpicture}
    \end{center}
    \begin{gather*}
        \int_{\gamma} \omega = \int_{0}^{1} (2txty) x +  ((tx)^2 +(ty)^2)y \ dt\\
        = \int_{0}^{1} 2t^2 x^2y + t^2 (x^2+y^2)y \ dt = \frac{2}{3}t^3x^2y + \frac{t^3}{3}(x^2+y^2)y|_0^1\\
        = \frac{2}{3}x^2y + \frac{1}{3}(x^2+y^2)y = x^2y + \frac{y^3}{3}
    \end{gather*}
\end{example}
\begin{proposition}[Forme chiuse e forme esatte]
    \underbar{sia} $\omega$ una forma differenziale $C^1$ in $A \subseteq \mathbb{R}^n$
    \begin{gather*}
        \omega = a_1(x) dx_1 + \ ... \ + a_n(x) dx_n
    \end{gather*}
    DOve $x= (x_1,x_2, \ ... , x_n)$\\
    Si dice che $\omega$ è \textbf{chiusa} \underbar{se}
    \begin{gather*}
        \forall i,j = 1, \ ... \ ,n \quad i \neq j \qquad \text{risulta}\\
        \frac{\partial a_i(x)}{ax_j} = \frac{\partial a_j(x)}{\partial ax_i} \qquad \forall x \in A
    \end{gather*}
    Cosa diventa se n=2\\
    
\end{proposition}
\begin{theorem}
    \underbar{sia} $A \subset \mathbb{R}^3$ e $\omega$ una forma differenziale $C^1$ in $A$ \\
    \underbar{se} $\omega$ è esatta \underbar{allora} $\omega$ è chiusa\\
    (vale anche in $\mathbb{R}^3 \ \forall n \geq 2$)
\end{theorem}
\begin{proof}
    Se $\omega$ è esatta, $\omega = df$, cioè $a_1 = \frac{\partial f}{\partial x_1} \quad a_2 = \frac{\partial f}{\partial x_2} \quad a_3 = \frac{\partial f}{\partial x_3}   $\\
    con $f$ potenziale di $\omega$.\\
    Scegliete $i,j = 1,2,3$ con $i \neq j$\\
    Poichè $\frac{\partial f}{\partial x_i}$ e $\frac{\partial f}{\partial x_j}$ sono $C^1$ sono derivabili e le loro derivate sono continue\\
    In particolare esistono e sono continue:
    \begin{gather*}
        \frac{\partial^2 f}{\partial x_i\partial x_j} \quad \text{e} \quad \frac{\partial^2 f}{\partial x_j\partial x_i} 
    \end{gather*}
    La continuità e il teorema di Schwarz garantiscono che sono uguali allora:
    \begin{gather*}
        \frac{\partial}{\partial x_j} a_i = \frac{\partial}{\partial x_j} (\frac{\partial f}{\partial x_i}) = \frac{\partial}{\partial x_1}(\frac{\partial f}{\partial x_j}) = \frac{\partial}{\partial x_1} a_j
    \end{gather*}
\end{proof}
\noindent
\textbf{RICHIAMO} $F=(F_1,F_2,F_3) \in C^1(A)$\\
definisco:
\begin{gather*}
    rot F = \nabla_x F =\\
    \det\begin{pmatrix}
        i & j & k\\
        \frac{\partial}{\partial x_1} & \frac{\partial}{\partial x_2} & \frac{\partial}{\partial x_3}\\
        F_1 & F_2 & F_3
    \end{pmatrix}\\
    = \underline{i} (\frac{\partial F_3}{\partial x_2} + \frac{\partial F_2}{\partial x_3}) + \underline{j} (\frac{\partial F_1}{\partial x_3} + \frac{\partial F_3}{\partial x_1}) + \underline{k} (\frac{\partial F_2}{\partial x_1} + \frac{\partial F_1}{\partial x_2})
\end{gather*}
\begin{theorem}
    \underbar{sia} $F \in C^1(A)$\\
    \underbar{se} $F$ è conservativo \underbar{allora}\\
    \begin{gather*}
        \text{rot } F \equiv 0
    \end{gather*}
    ($F$ è irrotazionale)
\end{theorem}
\noindent
\textbf{Domanda}: \ Se $\omega$ è una forma chiusa $\omega$ è esatta?\\
Se $F$ è un campo irrotazionale $F$ è conservativo?\\
La risposta breve è: in generale no se il dominio è semplicemente connesso si
\begin{example}
    \begin{gather*}
        \omega = \frac{-y}{x^2+y^2} dx + \frac{x}{x^2 + y^2} dy
    \end{gather*}
    Definita e $C^1$ in $\mathbb{R}^2 \backslash \{(0,0)\}$\\
    $\omega$ è chiusa infatti:
    \begin{gather*}
        \frac{\partial}{\partial y} (\frac{-y}{x^2+y^2}) = \frac{-1(x^2+y^2) -(-y)^2y}{(x^2+y^2)} = \frac{y^2-x^2}{(x^2+y^2)^2}\\
        \frac{\partial}{\partial x}(\frac{x}{x^2+y^2}) = \ ... \ = //
    \end{gather*}
    È esatta  in $\mathbb{R}^2 \backslash \{(0,0)\}$?\\
    Cioè $\exists$ $f$ definita in tutto $\mathbb{R}^2 \backslash \{(0,0)\}$ $t.c. \ \omega = df$ ? \textbf{NO}\\
    Se scelgo come curva chiusa la circonferenza:
    \begin{gather*}
        (\cos(t),\sin(t)) , t \in[0,2\pi] = S^1\\
        \int_{0}^{2a} \left( \frac{-\sin(t)}{\sin^2(t) + \cos^2(t)},(-\sin(t)) \right) + \left( \frac{\cos(t)}{\sin^2(t) + \cos^2(t)} \right) \cos(t) \ dt\\
        = \int_{0}^{2\pi} 1 dt = 2\pi\\
        \int_{S_1} \omega = 2\pi \neq 0  
    \end{gather*}
    \textbf{Domanda}: $\omega$ è esatta nel pianto $\{(x,y) \in \mathbb{R}^2: x>0\}$? \ \textbf{SI} es: $\arctan(\frac{y}{x})$:
    \begin{gather*}
        \frac{\partial}{\partial x} ( ") = \frac{1}{1+(\frac{y}{x})^2}(\frac{-y}{x^2}) = \frac{-y}{x^2+y^2} = a_1 \cmark\\
        \frac{\partial}{\partial x} ( ") = \frac{1}{1+(\frac{y}{x})^2}(\frac{1}{x}) = \frac{x}{x^2+y^2} = a_2 \cmark
    \end{gather*}
\end{example}
\begin{theorem}
    \underbar{se} $\omega$ è una fomra differenziale \textbf{chiusa} definita e $C^1$ in un insieme semplicemente connesso $A$\\
    \underbar{allora} $\exists$ una primitiva di $\omega$ definita in $A$.
\end{theorem}
\begin{observation}
    Un insieme si dice semplicemente connesso se data una curva contenuta in quellinsieme posso deformarla in modo continuo fino a ridurla a un punto.
\end{observation}


\newpage
\section{4/11/25}
Ricordiamo dalle scorse volte:\\
Se $\omega$ è esatta $\Rightarrow \omega$ è chiusa\\
Se $F$ è conservativo $\Rightarrow$ rot$F=0$  \\
Ci siamo poi chiesti vale il viceversa? $\overset{?}{\Leftarrow}$\\
In generale NO. Se $\omega$ è definita su un certo insieme $A$ ed è chiusa allora non è detto cje ima funzione $f$ definita su tutto $A$ talce che $\omega = df$\\
\begin{example}
    Se si cambia il dominio è possibile riuscirci:
    \begin{center}
        \begin{tikzpicture}
            \draw(-1,0) -- (1,0);
            \draw(0,-1) -- (0,1);
            \node at(1,1){$\omega$};
            \filldraw[fill=black](0,0) circle (2pt);
        \end{tikzpicture}
    \end{center}
    \begin{gather*}
        \nexists f \text{ definita in tutto } \mathbb{R}^2\backslash\{ (0,0) \}: \omega = df
        \exists f \text{ definita in } \{ (x,y): x>0 \}: \omega = df
    \end{gather*}
    Se $A$ è semplicemente connesso quindi la risposta è si.
\end{example}
\begin{definition}[essere semplicemente connesso]
    Un aperto $A$ si dice semplicemente conensso \underbar{se} è connesso e inoltre ogni curva chiusa è internamente contenuta in $A$ può essere ridotta mediante una deformazione continua in un unico punto senza mai uscire da $A$
\end{definition}
\tiny questa definizione è abbastanza intuitiva ma non rigorosa, quella rigorosa si vedrà più avanti.\normalsize\\
\begin{example}[in $\mathbb{R}^2$]
\begin{multicols}{2}
    \noindent
        Sono semplicemente connessi:
    \begin{itemize}
        \item cerchi
        \item ellissi
        \item poligonali
        \item semipiani
        \item il piano intero
        \item il piano privato di una semiretta
    \end{itemize}
    \columnbreak
    \hfil\\
    Non sono semplicemente connessi:
    \begin{itemize}
        \item il piano o un cerchio ( o un ellisse o un poligono) privato di un punto interno: una corona circolare.
        \item un insieme che presenta un buco.
    \end{itemize}
\end{multicols}
\end{example}
\begin{example}[in $\mathbb{R}^3$]
    \begin{multicols}{2}
        noindent
        Sono semplicemente connessi:
    \begin{itemize}
        \item Sfere
        \item ellissoidi
        \item poliedri convessi
        \item una corona sferica
        \item un semispazio
        \item lo spazio privato di un numero finito di punti
    \end{itemize}
    \columnbreak
    \hfil\\
    Non sono semplicemente connessi:
    \begin{itemize}
        \item il toro (ciambella)
        \item la sfera privata di un diametro
        \item lo spazio priato di una retta
    \end{itemize}
    \end{multicols}
\end{example}
Tutti gli insiemi connessi sono semplicemente connessi
Tutti gli insieme stellati sono semplicemente connessi
\begin{definition}[insieme stellato]
    Un insieme $A$ si dice stellato\\
    \underbar{se} $\exists \ P_0 \in A \ t.c. \ \forall P \in A$: \tab tutto il segmento ci estremi $P$ e $P_0$ è contenuto in $A$
\end{definition}
\hfil\\
Per introdurre la seconda def. di semplicemente connesso serve introdurre il concetto di Omotopia tra curve:
\begin{definition}[Omotopia tra curve]
    Siano $\gamma_1$ e $\gamma_2$ curve contenute in un aperto connesso $A \subseteq \mathbb{R}^2$ o $A \subseteq \mathbb{R}^3$ e supponiamo che:\\
    $\varphi_1[a,b] \to A$ \tab sia una cruva parametrica di $\gamma_1$\\
    $\varphi_2[a,b] \to A$ \tab sia una cruva parametrica di $\gamma_2$\\
    tali che: \tab $\varphi_1(a) = \varphi_2(a) \ $  $ \ \varphi_1(b) = \varphi_2(b)$\\
    $\gamma_1$ e $\gamma_2$ si dicono \textbf{omotope} in $A$ \underbar{se} esiste una funzione continua: $\phi(t, \lambda) \quad t \in [a,b] \quad \lambda \in [0,1]$\\
    tale che:
    \begin{enumerate}
        \item $\phi(t, 0) = \varphi_1(t) \quad \phi(t,1) = \varphi_2(t) \quad \forall t \in [a,b]$
        \item $\phi(a, \lambda) = P_a \quad \phi(b , \lambda) = P_b \quad \forall \lambda \in [0,1]$
    \end{enumerate}
    E infine che $\forall \lambda \in [0,1]$ la curva $\phi_\lambda$ di equazione $\phi = \phi(t,\lambda)$ sia contenuta in $A$.
    \begin{center}
        \begin{tikzpicture}
            \draw(0,0) ..controls(1,1.5).. (2,1) node[at start, left]{$P_a$} node[midway, above]{$\gamma_1$};
            \draw(0,0) ..controls(1,-0.5).. (2,1) node[at end, right]{$P_b$} node[midway, below]{$\gamma_2$};
            \draw[dashed](0,0) ..controls(1,1).. (2,1);
            \draw[dashed](0,0) ..controls(1,0).. (2,1);
            \node at(1,2){$\lambda = 0$};
            \node at(1,-1.6){$\lambda = 1$};
            \draw(1,0) circle (2.5);
            \node at(-2,0){$A$};
        \end{tikzpicture}
    \end{center}
\end{definition}
\begin{proposition}
    Se $\gamma_1$ e $\gamma_2$ sono chiuse $\varphi_1(a) = \varphi_1(b)$,$\varphi_2(a) = \varphi_2(b)$, esse si dicono omotope in $A$ se vale la definizione precedente con la condizione 2) sostituita dalla seguente:
    \begin{gather*}
        2)' \quad \varphi(a,\lambda) = \varphi(b,\lambda) \quad \forall \lambda \in [0,1]
    \end{gather*}
\end{proposition}
\begin{definition}[\textbf{rigorosa} di essere semplicemente connesso]
    Un aperto $A$ di $\mathbb{R}^2$ o di $\mathbb{R}^3$ si dice semplicemente connesso se è connesso e due curve qualsiasi contenute in $A$ e aventi gli stessi estremi sono omotrope.\\
    La def. può essere data in termini di curve chiuse: ogni curva chiusa contenuta in $A$ è omotopa a una curva costante (cioè che si riduce ad un solo punto).
\end{definition}
\begin{theorem}
    \underbar{Sia} $\omega= a_1 \ (x,y) \ dx + a_2 \ (x,y) \ dy$ una forma differenziale $C^1$\\
    \underbar{e} chiusa in un insieme $A \subseteq \mathbb{R}^2$ semplicemente connesso\\
    \underbar{Allora} $\omega$ è esatta.\\
    Vale un enunciato analogo in $\mathbb{R}^3$
    \begin{center}
        \begin{tikzpicture}
            \draw(0,0) ..controls(1,1.5).. (2,1) node[at start, left]{$P_a$} node[midway, above]{$\gamma_1$};
            \draw(0,0) ..controls(1,-0.5).. (2,1) node[at end, right]{$P_b$} node[midway, below]{$\gamma_2$};
            \draw(1,0) circle (2.5);
            \node at(-2,0){$A$};
        \end{tikzpicture}
    \end{center}
\end{theorem}
\begin{proof}
    Siano $\gamma_1$ e $\gamma_2$ due curve in $A$ cge gabbi gku stessu estremi. Poich $A$ è semplicemente connesso $\gamma_1$ e $\gamma_2$ sono omotope.\\
    $\exists \gamma(t, \lambda) :[a,b]\times[0,1] \to A$ che soddisfa le proprietà enunciate nella def. di omotopia.\\
    Definisco $\forall \lambda \in [0,1]$
    \begin{gather*}
        I(\lambda) = \int_{\text{curva }\gamma_\lambda} \omega = \int_{a}^{b} a_1 \left( x(t,\lambda) , y (t,\lambda) \right) \frac{\partial x}{\partial t}(t,\lambda) + a_2 \left( x(t,\lambda) , y (t,\lambda) \right) \frac{\partial y}{\partial t}(t,\lambda) \ dt
    \end{gather*}
    Suppongo che $\varphi(t,\lambda)$ \underbar{sia} $C^1\left( [a,b] \times [0,1] \right)$ e che le derivate secode miste $\frac{\partial^2 x}{\partial \lambda \partial t}, \frac{\partial^2 y}{\partial \lambda \partial t}$ \underbar{siano} continue.\\
    Ora calcolo la derivata rispetto a $\lambda$ i calcoli del passaggio precedente:
    \begin{gather*}
        \frac{d}{d\lambda}\left(I(\lambda) \right) = \frac{d}{d\lambda}\left(\int_{\text{\tiny curva \normalsize}\gamma_\lambda} \omega\right)  = \int_{a}^{b} \frac{d}{d\lambda}\left(a_1 \left( x(t,\lambda) , y (t,\lambda) \right) \frac{\partial x}{\partial t}(t,\lambda) + a_2 \left( x(t,\lambda) , y (t,\lambda) \right) \frac{\partial y}{\partial t}(t,\lambda)\right)  \ dt
    \end{gather*}
    Voglio dimostrare che l'ipotesi $\omega$ chiusa implica che:
    \begin{gather*}
        \int_{a}^{b} \frac{d}{d \lambda}\left( \ \ \  \right)dt\\
        = \int_{a}^{b} \frac{d}{dt}\left( a_1(x(t, \lambda),y(t, \lambda))\frac{\partial x}{\partial \lambda}(t, \lambda) + a_2(x(t, \lambda),y(t, \lambda))\frac{\partial y}{\partial \lambda}(t, \lambda)\right)  \ dt \\
        = \left[a_1(x(t, \lambda),y(t, \lambda))\frac{\partial x}{\partial \lambda}(t, \lambda) + a_2(x(t, \lambda),y(t, \lambda))\frac{\partial y}{\partial \lambda}(t, \lambda)\right]_{t=a}^{t=b} = 0 
    \end{gather*}
    \begin{gather*}
        \frac{d}{d\lambda}\left[ a_1(x(t,\lambda),y(t,\lambda)) \frac{\partial x}{\partial t}(t, \lambda) + a_2(x(t,\lambda),y(t,\lambda)) \frac{\partial y}{\partial t}(t, \lambda) \right] \qquad \boxed{I} \\
        \frac{d}{d t}\left[a_1(x(t, \lambda),y(t, \lambda))\frac{\partial x}{\partial \lambda}(t, \lambda) + a_2(x(t, \lambda),y(t, \lambda))\frac{\partial y}{\partial \lambda}(t, \lambda)\right] \qquad \boxed{II}\\
    \end{gather*}
    Voglio far vedere che le loro derivate sono uguali:
    \begin{gather*}
        \frac{d}{d\lambda} \boxed{I} = \left[ \frac{\partial a_1}{\partial x}( '' ) \frac{\partial x}{\partial \lambda}( '' )+ \frac{\partial a_1}{\partial y}( '' )\frac{\partial y}{\partial \lambda}\right]\frac{\partial x}{\partial t}( '' ) + a_1( '' ) \frac{\partial^2 x}{\partial \lambda \partial t} + \left[ \frac{\partial a_2}{\partial x}( '' )\frac{\partial x}{\partial \lambda} + \frac{\partial a_2}{\partial y}( '' )\frac{\partial y}{\partial \lambda} \right]\frac{\partial y}{\partial t} + a_2( '' )\frac{\partial^2 y}{\partial \lambda \partial t}\\
        \frac{d}{d \lambda} \boxed{II} = \left[ \frac{\partial a_1}{\partial x}( '' ) \frac{\partial x}{\partial t}( '' )+ \frac{\partial a_1}{\partial y}( '' )\frac{\partial y}{\partial t}\right]\frac{\partial x}{\partial \lambda}( '' ) + a_1( '' ) \frac{\partial^2 x}{\partial t \partial \lambda} + \left[ \frac{\partial a_2}{\partial x}( '' )\frac{\partial x}{\partial t} + \frac{\partial a_2}{\partial y}( '' )\frac{\partial y}{\partial t} \right]\frac{\partial y}{\partial \lambda} + a_2( '' )\frac{\partial^2 y}{\partial t \partial \lambda}
    \end{gather*}
    So che è chiusa cioè che $\forall(x,y) \in A \frac{\partial^2 a_1 (x,y)}{\partial y}(x,y) = \frac{\partial^2 a_2}{\partial x} (x,y)$
\end{proof}

\newpage
\section{5/11/25}
Vari esempi sui campi:
\begin{example}
    Consideriamo il campo $F=\begin{pmatrix}
        -y\\
        x
    \end{pmatrix}$\\
    È conservativo?\\
    Calcolare il suo integrale lungo l'ellisse: $\begin{cases}
        x = a \cos(t)\\
        y = b \sin(t)
    \end{cases}t \in [0,2\pi]$
    Questo è un campo di vettori ortogonali:
    \begin{center}
        \begin{tikzpicture}
            \draw[->](0,-1.5) -- (0,1.5);
            \draw[->](-1.5,0) -- (1.5,0);
            \draw[->](1,0) -- (0.5,0.5);
            \draw[->](0,1) -- (-0.5,0.5);
            \draw[->](0,-1) -- (0.5,-0.5);
            \draw[->](-1,0) -- (-0.5,-0.5);
        \end{tikzpicture}
    \end{center}
    Per prima cosa per veder ese è conservativo bisogna verificare che sia irrotazionale.
    \begin{gather*}
        \frac{\partial x}{\partial x} \quad \frac{\partial (-y)}{\partial y}
    \end{gather*}
    Per verificare che sia irrotazionale bisogna verificare che:
    \begin{gather*}
        \left( \frac{\partial F_2}{\partial x} = \frac{\partial F_1}{\partial y} \right) \quad \text{se } F=(F_1,F_2) 
    \end{gather*}
    Il che vorrebbe dire che il rotore di $F$ sarebbe:
    \begin{gather*}
        rotF(x,y) = \text{det}\begin{pmatrix}
            i & j & k\\
            \frac{\partial}{\partial x} & \frac{\partial}{\partial y} & \frac{\partial}{\partial z}\\
            F_1 & F_2 & \varnothing
        \end{pmatrix}= i \cancel{\left(\frac{\partial F_2}{\partial z}\right) } + j \cancel{\left(\frac{\partial F_1}{\partial z}\right)} + k \left(\frac{\partial F_2}{\partial x} - \frac{\partial F_2}{\partial y} \right)
    \end{gather*}
    Svolgendo\dots
    \begin{gather*}
        F_1(x,y) = -y\\
        F_2(x,y) = x\\
        \frac{\partial F_2}{\partial x} = -1\\
        \frac{\partial F_1}{\partial y} = -1
    \end{gather*}
    Un inciso importante:
        \begin{center}
            \begin{tikzpicture}
                \draw[->](0,-1) -- (0,1);
                \draw[->](-1,0) -- (1,0);
                \draw(0,0) ellipse (0.8 and 0.5); 
                \filldraw[fill = black](0,0.5) circle (2pt) node[above, right]{$(0,b)$};
                \filldraw[fill = black](0,-0.5) circle (2pt) node[above, right]{$(0,-b)$};
                \filldraw[fill = black](0.8,0) circle (2pt) node[above, right]{$(0,a)$};
                \filldraw[fill = black](-0.8,0) circle (2pt) node[above, left]{$(0,-a)$};
            \end{tikzpicture}
        \end{center}
        Le equazioni parametriche di questa ellisse sono:
        \begin{gather*}
            \begin{cases}
                x = a \cos(t)\\
                y = b \sin(t)
            \end{cases} \quad t \in [0,2\pi]
        \end{gather*}
        L'equazione cartesiana è:
        \begin{gather*}
            \frac{x^2}{a^2} + \frac{y^2}{b^2} = 1
        \end{gather*}
        \hfil\\
        Calcoliamo ora il lavoro:
        \begin{gather*}
            \int_{0}^{2\pi} \left\langle  (F_1,F_2);(\dot{x},\dot{y}) \right\rangle \ dt\\
            = \int_{0}^{2\pi} \left\langle \left(-y(t), x(t)\right);\left(\dot{x}(t), \dot{y}(t)\right)   \right\rangle \ dt\\
            = \int_{0}^{2\pi} \left\langle \left(-b \sin(t) , a \cos(t)\right);\left(-a \sin(t), b \cos(t)\right)   \right\rangle \ dt\\
            = +ab \ \int_{0}^{2\pi} \left(\sin^2(t) + \cos^2(t)\right) = +2\pi ab     
        \end{gather*}
\end{example}
\begin{example}
    Consideriamo il campo $F=(\frac{x}{x^2+y^2} , \frac{y}{x^2+y^2})$ in $\mathbb{R}^2 \backslash \{(0,0)\}$
    \begin{gather*}
        \left\lVert F \right\rVert = \frac{\left\lVert (x,y) \right\rVert }{x^2+y^2} =  \frac{\left\lVert x,y \right\rVert }{\left\lVert (x,y) \right\rVert^2 } = \frac{1}{\left\lVert (x,y) \right\rVert }
    \end{gather*}
    I punti richiesti sono:
    \begin{enumerate}[$a)$]
        \item è conservativo?
        \item calcolare il lavoro lungo l'ellisse : $3x^2 - xy + 10y^2 = 1$ posta in centro amtiorario
        \item Calcolare il lavoro lungo l'arco di parabola $\gamma$ di equazione $y=1+x^2 \quad x \in [0,2]$
    \end{enumerate}
    \textbf{punto $a$}\\
    Comincio col calcolare il lavoro lungo un qualsiasi arco di una qualsiasi circonferenza centrata in $(0,0)$.\\
    Mi chiedo quindi se è irrotazionale:
    \begin{gather*}
        \frac{\partial F_2}{\partial x} = \frac{0 (x^2+y^2) - y \ 2x}{(x^2+y^2)^2} = \frac{2 y x}{(x^2+y^2)^2}\\
        \frac{\partial F_1}{\partial y} = \frac{0 (x^2+y^2) - y \ 2x}{(x^2+y^2)^2} = \frac{2 y x}{(x^2+y^2)^2}\\
    \end{gather*}
    Quindi è irrotazionale.
    \begin{gather*}
        f(x,y) = \int (F_1,F_2) (x,y) \ dt    
    \end{gather*}
    Cammino che collega un punto fisso $(x_0,y_0)$ ad un punto $(x,y)$ (nel verso da $(x_0,y_0)$ a $(x,y)$)
Se il dominio è stellato rispetto a $(0,0)$
\begin{center}
    \begin{tikzpicture}
        \draw(0,0) -- (2,-0.5) node[at end, below]{$(x,y)$} node[at start, below]{$(0,0)$};
    \end{tikzpicture}
\end{center}
\begin{gather*}
    t \in [0,1] \to (tx,ty)
\end{gather*}
Ma il mio insieme è stellato?
\begin{gather*}
    f(x,y) = \int_{0}^{1} \left\langle \left(F_1(tx,ty), F_2(tx,ty)\right) , (x,y)  \right\rangle \ dt 
\end{gather*}
Qindi $\mathbb{R}^2 \backslash \{(0,0)\}$ \textbf{non} è stellato\\
Qualunque centro $(x_0,y_0)$ vuoi consideriate c'è sempre un punto $(x,y)$ con la propietà che il segmenteo che unsice $(x,y)$ a $(x_0,y_0)$ \textbf{non} è tutto contenuto in $\mathbb{R}^2 \backslash \{(0,0)\}$\\
Cerco quindi $f(x,y)$:
\begin{gather*}
    \frac{\partial f}{\partial x} = F_1 \quad \frac{\partial f}{\partial y} = F_2\\
    \text{cioè } \boxed{\frac{\partial f}{\partial x} = \frac{x}{x^2+y^2}} , \frac{\partial f}{\partial y} = \frac{y}{x^2+y^2}\\
    \int \frac{x}{x^2+c^2} \ dx \overset{\tiny \text{rispetto a x}\normalsize}{\to} \frac{1}{2} \ln(x^2+c^2) + d\\
    \boxed{f(x,y) = \frac{1}{2} \ln(x^2+y^2) + d(y)}\\
    \frac{\partial}{\partial y} = \frac{y}{x^2 + y^2} + d'(y)\\
\end{gather*}
Se $d'(y) = 0$ f è primitiva del campo. Basta che scelga $f(y)$ costante o addirittura $d\equiv 0$.
\begin{gather*}
    f(x,y) = \frac{1}{2} \ln(x^2+y^2)
\end{gather*}
\textbf{punto $b$}\\
è zero
\textbf{punto $c$}\\
\begin{center}
    \begin{tikzpicture}
        \draw[->](-1,0) -- (2.5,0);
        \draw[->](0,-1) -- (0,2);
        \draw(0,0.3) ..controls(0.5,0.4).. (1,1) node[at end, right]{$y = 1+x^2$};
        \filldraw[fill = black](0,0.3) circle (2pt) node[left]{$(0,1)$};
        \filldraw[fill = black](0,0) circle (2pt) node[left]{$0$};
        \filldraw[fill = black](1,0) circle (2pt) node[left]{$2$};
        \filldraw[fill = black](1,1) circle (2pt) node[left]{$(2,5)$};
    \end{tikzpicture}
\end{center}
\begin{gather*}
    x \in [0,2] \to (x, 1+x^2)\\
    \int_{\gamma} F(x,y) = f(2,5) - f(0,1) = \frac{1}{2} \ln(4+25) -\frac{1}{2} \ln(1) 
\end{gather*}
\end{example}
\begin{example}
    Consideriamo il campo $F=(xy - \sin(z) , \frac{1}{2} x^2 - \frac{e^y}{z} , \frac{e^y}{z^2} - x \cos(z))$ definito in $\mathbb{R}^3 \backslash \{z = 0\}$\\
    \bulletout Controllare se è conservativo e se lo è trovare il potenziale\\
    \begin{gather*}
        \frac{\partial F_1}{\partial y} = \frac{\partial F_2}{\partial x} = x\\
        \frac{\partial F_1}{\partial z} = \frac{\partial F_3}{\partial x} = -\cos(z)\\
        \frac{\partial F_2}{\partial z} = \frac{\partial F_3}{\partial y} = \frac{e^y}{z^2}
    \end{gather*}
    \cmark è irrotazionale\\
    \hfil\\
    Cerco un potenziale in $\{z= 0\}$:
    \begin{center}
        \begin{tikzpicture}
            \draw(0,0) -- (2,0.5) node[at end, below]{$(x,y,z)$} node[at start, below]{$(0,0,1)$};
            \draw(0,-1) -- (2,-1) node[at end, right]{$z=0$};
        \end{tikzpicture}
    \end{center}
    \begin{gather*}
        (tx,ty,t(z-1)) \quad t \in [0,1]
    \end{gather*}
    \begin{gather*}
        f \\
        \frac{\partial f}{\partial x} = xy - \sin(z)\\
        \frac{\partial f}{\partial y} = \frac{1}{2}x^2 - \frac{e^y}{z}\\
        \frac{\partial f}{\partial z} = \frac{e^y}{z^2} -x\cos(z)
    \end{gather*}
    \begin{gather*}
        f = \frac{x^2}{2}y -(\sin(z)) x + c (y,z)\\
        \frac{\partial }{\partial y}\left( '' \right) = \frac{x^2}{2} + \frac{\partial c}{\partial y} \overset{?}{=} \frac{x^2}{2} -\frac{e^y}{2} \\
        \boxed{f = \frac{x^2}{2}y - (\sin(z))x - \frac{e^y}{2} + \cancel{d(z)}} \text{ se } \frac{\partial c}{\partial y} = -\frac{e^y}{z} \Rightarrow c = -\frac{e^y}{z} + d(z)\\
        \frac{\partial}{\partial z}\left( '' \right) = 0 -\underset{\cmark}{(\cos(z))x} + \underset{\cmark}{\frac{e^y}{z^2}} + d'(z) \overset{?}{=}  
    \end{gather*}
    Deve essere $d'(z) = 0$ ad esempio $d\equiv 0 $
\end{example}
\begin{example}
    Sia $f: [0,+\infty) \to \mathbb{R}$ una funzione, Determinare $f$ in modo che la forma diff:
    \begin{gather*}
        \omega = -\frac{x}{\sqrt{x^2+y^2}} f(\sqrt{x^2+y^2}) dx + \frac{y}{\sqrt{x^2+y^2}} dy
    \end{gather*}
    sia chiusa in $\mathbb{R}^2 \backslash \{(0,0)\}$ . Trovare anche una primitiva di $\omega$.
    \hfil\\
    È chiusa se
    \begin{gather*}
        \frac{\partial}{\partial y} \left( -\frac{x}{\sqrt{x^2+y^2}} f(\sqrt{x^2+y^2}) \right) = \underbrace{\frac{\partial}{\partial x} \left(\frac{y}{\sqrt{x^2+y^2}}\right)}_{-xy(x^2+y^2)^{-3/2}}\\
        \frac{\partial}{\partial y}\left(-x(x^2+y^2)^{\frac{-1}{2}}f(\sqrt{x^2+y^2})   \right)\\
        = -x \left[ -\frac{1}{\cancel{2}} \right]
    \end{gather*}

\end{example}


\newpage
\section{6/11/25}
\begin{definition}[EDO famiglia di curve]
    \begin{gather*}
        F_c(x,y) = 0
    \end{gather*}
    famiglia ad 1 parametro ($c \in \mathbb{R}$) di curve.
    Le ipotesi che devono essere soddisfatte sono
    \begin{itemize}
        \item $F_c \in C^1(A)$ Aperto $\subseteq \mathbb{R}^2$, derivabile rispetto c.\\
        \item $(x,y)$ regolare, $\frac{\partial F_c}{\partial y} \neq 0$
    \end{itemize}
    La seconda ipotesi vuol dire che una curva parametrica la possiamo considerare come grafico in $f(x)$ in un punto in cui è regolare.\\
    Quindi cerco $y = y(x) \ t.c. \ \boxed{F_c(x,y(x)) = 0} \quad \forall x \in I$
    Derivando si ottiene:
    \begin{gather*}
        \forall x \in I \qquad \frac{d}{dx}(F_c (x,y(x))) = 0\\
        \text{cioè } \boxed{\frac{\partial F_c(x,y(x))}{\partial x} + y'(x \frac{\partial F_c(x,y(x))}{\partial y}) = 0} \quad \forall c
    \end{gather*}
    Cerco di ricavare $c$ e trovo l'equazione:
    \begin{gather*}
        f(x,y,y') = 0
    \end{gather*}
\end{definition}
\begin{example}
    parabola con $v \equiv (0,0)$
    \begin{gather*}
        y = cx^2\\
        y-cx^2 = 0 \quad{*}\\
        F_c(x,y) = y-cx^2 \\
        y' -2cx = 0
    \end{gather*}
    dalla eq. \fbox{*} ricavo $c = \frac{y}{x^2} \text{ se } s \neq 0$, e posso sostituire:
    \begin{gather*}
        y' -2\frac{y}{x^2} x = 0\\
        y' - 2 \frac{y}{x} = 0\\
        xy' -2y = 0; y(0) = 0 , y'(0) = 0 
    \end{gather*}
\end{example}
Ricerca di traiettorie ortogonali a $F(x,y) = 0 \Leftrightarrow \frac{\partial F}{\partial x} + y' \frac{\partial F}{\partial y} \Leftrightarrow f(x,y,y') = 0$
E traiettoreie ortogonali sono curve che intersecano in un uncio punto ogni curva della famiglia e in tale punto sono $\perp$ (cioè i rispettivi vettori tangenti sono $\perp$)
\begin{itemize}
    \item Passa per $\begin{pmatrix} x \\ y(x) \end{pmatrix}$ cioè $\exists t \in J \ t.c. \ \begin{pmatrix} t \\ u(t) \end{pmatrix} = \begin{pmatrix} x \\ y(x) \end{pmatrix}$
    \item in $\begin{pmatrix} x \\ y(x) \end{pmatrix}$ i vettori sono $\perp$
\end{itemize}
Cioè:
\begin{gather*}
    \begin{pmatrix}
        1\\
        y'(x)
    \end{pmatrix} \perp \begin{pmatrix}
        1\\
        u(t)
    \end{pmatrix}_{t = x}
\end{gather*}
Cioè $1+y'(x) u'(x) = 0$\\
cioe $u'(x) = -\frac{1}{y'(x)} \qquad y' = \frac{1}{u'}$\\
Troo l'edo delle traiettorie ortogonali:
\begin{gather*}
    f(x,u(x), \frac{-1}{u'(x)}) = 0
\end{gather*}
Cerco dunque le traiettorie ortogonali alla famiglia di parabole localmente grafici ($x,u(x)$) deve soddisfare $f(x,u,\frac{-1}{u}) = 0$\\
cioè:
\begin{gather*}
    \frac{-1}{u'} -2 \frac{u}{x} = 0\\
    2u u' = -x
\end{gather*}
Passo a risolverla, per prima cosa la metto in forma normale:
\begin{gather*}
    (u^2) = \frac{-x^2}{2} +k \quad k \in \mathbb{R}\\ 
    u^2 + \frac{x^2}{2} = k
\end{gather*}
\begin{example}
    Trovare le curve piane regolari (localmente grafico  $y = y(x)$)\\
    $t.c.$ $\forall (x,y) \in \gamma$ la distanza $((x,y) ,Q) = $ distanza$(Q;u(0))$\\
    Dove $Q$ = intersezione tra la retta tangente a $\gamma$ in $(x,y)$ e l'asse $y$
    \hfill\\
    abbiamo il sostegno della curva:
    \begin{gather*}
        \gamma \underline{x}(t) = \begin{pmatrix}
            x(t) \\
            y(t)
        \end{pmatrix}
    \end{gather*}
    Questa è regolare quindi è localmente grafico di funzione, quindi supponiamo $y=y(x)$, quindi il vettore tangente a $\gamma$ in $(x,y(x))$ è $\begin{pmatrix} 1 \\ y'(x) \end{pmatrix}$\\
    Questo perchè ho parametrizzato secondo la funzione nel punto come grafico. quindi $x(t) = t$ e $y(t) = y(t)$. che derivati mi danno il vett. tangente.
    Ora calcolaimo la retta tangente a $\gamma$ in $(x,y(x)) : \begin{pmatrix} x(t) \\ y(t) \end{pmatrix} = \begin{pmatrix} 1 \\ y(x) \end{pmatrix} \tau + \begin{pmatrix} x \\ y(x) \end{pmatrix}_{\tau \in \mathbb{R}}$\\
    Di conseguenza posso calcolare la distanza come:
    \begin{gather*}
        Y - y = y'(x) (X-x) \Rightarrow Q \equiv (0,y - xy'(x))\\
        dist^2 (Q,(x,y)) = (x-0)^2 + (\cancel{y} - (\cancel{y} -xy'(x)))^2 = x^2 + x^2y'^2(x)\\
        dist^2 (Q,(1,0)) = (1-0)^2 + (y-xy'-0)^2 = (x-xy')^2 +1 \leftrightarrow x^2 + \cancel{x^2y^2} = 1 + y^2 -2xyy' + \cancel{y^2x^2} \quad \forall x \in J\\
        \boxed{2xyy' = 1-x^2+y^2}
    \end{gather*}
    Siamo in un caso in cui l'equazione diff. non è ne lineari ne a variabili separabili, il modo migliore (come abbiamo visto nei casi di eulero e bernulli) è cercare una sostituzione:
    \begin{gather*}
        2yy' = (y^2)' \quad \text{ sostituz.}\\
        u(x) = y^2(x)\\
        \boxed{xu' = \-x^2 + u} \quad \text{ è edo lineare in $u$}\\
        u' + Au + B = 0
    \end{gather*}
    Ci chiediamo se è calcolabile per $x=0$ che è dove da prolemi.\\
    Infatti per $x=0$ si ha $u(0) = -1$ cioè $y^2(0) = -1$ il che è impossibile quindi imponiamo la condizione \fbox{$x \neq 0$}\\
    Procedo a risolvere:
    \begin{gather*}
        u' - \frac{u}{x} + \frac{x^2-1}{x} = 0\\
        u(x) = x^2 + cx -1\\
        y^2(x) = -x^2 +cx -1\\
        x^2 -cx +y^2 +1 = 0 \quad c \in \mathbb{R}
    \end{gather*}
\end{example}
Facciamo un altro esempio 
\begin{example}
    trovare le linee di massima pendenza della funzione $u(x,y) = x(x^2-3y^2)$\\
    una linea di massima pendenza è il cammino lungo la funzione in cui il gradiente è sempre massimo:
    \begin{gather*}
        \underline{x}(t)_{t \in I} \text{ linea di max. pend. } \Leftrightarrow \underline{\dot{x}} \parallel Du(x(t,y(t)))
    \end{gather*}
    Mi restringo a studiare le linee che sono localmente grafici $y=y(x)$.\\
    \begin{gather*}
        \underline{x}(t) = \begin{pmatrix}
             t \\
             y(t)
        \end{pmatrix}_{t \in I} \qquad \qquad \underline{\dot{x}}(t) \begin{pmatrix}
            1\\
            \dot{y}(t)
        \end{pmatrix}_{t \in I}\\
        \text{cerco } y = y(t) \ t.c. \ \begin{pmatrix}
            1\\
            \dot{y}(t) 
        \end{pmatrix}\parallel \begin{pmatrix}
            3t^2 - 3y^2(t)\\
            -6t y (t)
        \end{pmatrix}\\
        Du = \begin{pmatrix}
            x^2-3y^2 + 2x^2\\
            -6xy
        \end{pmatrix} = \begin{pmatrix}
            3x^2 - 3y^2\\
            -6xy
        \end{pmatrix} \quad \Leftrightarrow \quad \begin{cases}
            1 = k(3t^2 - 3y^2(t))\\
            \dot{y}(t) = k (-6 t y(t))
        \end{cases}\\
        \exists k \neq 0 \Rightarrow \dot{y} = \frac{-6ty}{3(t^2-y^2)}\\
        k = \frac{1}{3(t^2-y^2)}\\
        \dot{y} = \frac{-2ty}{t^2-y^2} \qquad [z = \frac{y}{t}]
    \end{gather*}
\end{example}
\newpage
\section{18/11/25}
\subsection{Integrali per funzioni a più variabili}
Quando si parla di questi integrali si parla di più concetti.\\
Ad esempio si può parlare di area in tre o più dimensioni, in intuizione geometrica sarebbe il volume sotteso dal grafico. Anche se non sembra non sono concetti chiari poichè devo definire cosa intendo per area e per volume e fare una th di integrazione si intende definire volumi di insiemi e aree di insiemi, l'area non è un concetto universale (se si parla di casi più particolare), ad es. si prende un quadrato di $x \leq 1$ e $y\leq 1$ ma si prendono solo le x t.c siano razionali e y t.c siano irrazionali, e questo caso è difficile da definire un'area.\\
I casi più  completi sono le integrazioni di Lebeqe che non tratteremo in questo corso ci concentreremo invece sull'integrazione di cauchuy-rimann e quella di jordan.\\
Prendiamo per prima cosa il dominio:
\tiny
\begin{center}
    \begin{tikzpicture}
        %asse x
        \draw[->] (0,0) -- (5,0) node[right] {\normalsize$x$\tiny};
        %asse y
        \draw[->] (0,0) -- (0,5) node[above] {\normalsize$y$\tiny};
        \draw(-0.2,0.5) -- (0.2,0.5) node[left, at start]{$y_0 = c$};
        \draw(-0.2,1) -- (0.2,1) node[left, at start]{$y_1$};
        \draw(-0.2,1.5) -- (0.2,1.5) node[left, at start]{$y_2$}; 
        \draw(-0.2,2) -- (0.2,2) node[left, at start]{$y_3$};
        \draw(-0.2,2.5) -- (0.2,2.5) node[left, at start]{$y_4$};
        \draw(-0.2,3) -- (0.2,3) node[left, at start]{$y_5 = d$};  
        \draw(0.5,-0.2) -- (0.5,0.2) node[below, at start]{$x_0 = a$};
        \draw(1,-0.2) -- (1,0.2) node[below, at start]{$x_1$};
        \draw(1.5,-0.2) -- (1.5,0.2) node[below, at start]{$x_2$};
        \draw(2,-0.2) -- (2,0.2) node[below, at start]{$x_3$};
        \draw(2.5,-0.2) -- (2.5,0.2) node[below, at start]{$x_4$};
        \draw(3,-0.2) -- (3,0.2) node[below, at start]{$x_5 = b$};
        %griglia
        \draw (0.5,0.5) -- (0.5,3);
        \draw (1,0.5) -- (1,3);
        \draw (1.5,0.5) -- (1.5,3);
        \draw (2,0.5) -- (2,3);
        \draw (2.5,0.5) -- (2.5,3);
        \draw (3,0.5) -- (3,3);
        \draw (0.5,0.5) -- (3,0.5);
        \draw (0.5,1) -- (3,1);
        \draw (0.5,1.5) -- (3,1.5);
        \draw (0.5,2) -- (3,2);
        \draw (0.5,2.5) -- (3,2.5);
        \draw (0.5,3) -- (3,3);
    \end{tikzpicture}
    \begin{tikzpicture}
        %asse x
        \draw[->] (0,0) -- (4,0) node[right] {\normalsize$x$\tiny};
        %asse y
        \draw[->] (0,0) -- (0,3) node[above] {\normalsize$y$\tiny};
        \draw(-0.2,2) -- (0.2,2) node[left, at start]{$y_k$};
        \draw(-0.2,1.5) -- (0.2,1.5) node[left, at start]{$y_{k-1}$};
        \draw(2, -0.2) -- (2,0.2) node[below, at start]{$x_h$};
        \draw(3, -0.2) -- (3,0.2) node[below, at start]{$x_{h-1}$};
        %rettangolo
        \draw (2,1.5) -- (2,2) -- (3,2) -- (3,1.5) -- (2,1.5);
        \node at (2.5,1.75) {$I_{hk}$};
        \node at (2.5, 0.5) {$\frac{b-a}{n}$};
        \node at (0.8, 1.75) {$\frac{d-c}{n}$};
    \end{tikzpicture}
\end{center} 
\normalsize
Prendo $n$ parametro, $n\in \mathbb{N}$\\
divido $[a,b]$ in $n$ parti uguali.\\
divido $[c,d]$ in $n$ parti uguali.\\
Ottengo così una griglia di $n^2$ rettangoli.\\
Chiamo $I_{hk}$ il rettangolo in posizione $h,k$\\
Area di $I_{hk} = |I_{hk}| = \frac{b-a}{n}\frac{d-c}{n}$ 
Definisco somma di Cauchy-Riemann come:
\begin{definition}
    \begin{gather*}
        S_n = \sum_{h,k = 1}^{n} f(P_{hk})|I_{hk}| \\
    \end{gather*}
\end{definition}
\tiny
\begin{center}
    \begin{tikzpicture}
        %assi xyz
        \draw[->](0,0) -- (5,0) node[right] {\normalsize$y$\tiny};
        \draw[->](0,0) -- (0,5) node[above] {\normalsize$z$\tiny};
        \draw[->](0,0) -- (-3,-3) node[below right] {\normalsize$x$\tiny};
        %parallelepipedo 1
        \draw[red](-1,-1) -- (1,-1) -- (2,0);
        \draw[red](-1,-1) -- (-1, 2);
        \draw[red](1,-1) -- (1,2);
        \draw[red](2,0) -- (2,3);
        \draw[red](0,0) -- (0,3);
        \draw[red](0,0) -- (-1,-1);
        \draw[red](-1,2) -- (1,2) -- (2,3) -- (0,3) -- (-1,2);
        %parallelepipedo 2
        \draw[blue](2,0) -- (4,0) -- (3,-1) -- (1,-1) -- (2,0);
        \draw[blue](2,0) -- (2,2.5);
        \draw[blue](4,0) -- (4,2.5);
        \draw[blue](3,-1) -- (3,1.5);
        \draw[blue](2,2.5) -- (4,2.5) -- (3,1.5) -- (1,1.5) -- (2,2.5);
        %parallelepipedo 3
        \draw[green](-2,-2) -- (0,-2) -- (1,-1) -- (-1,-1) -- (-2,-2);
        \draw[green](-2,-2) -- (-2,0.7);
        \draw[green](0,-2) -- (0,0.7);
        \draw[green](1,-1) -- (1,1.7);
        \draw[green](-2,0.7) -- (0,0.7) -- (1,1.7) -- (-1,1.7) -- (-2,0.7);
        %parallelepipedo 4
        \draw[cyan](1,-1) -- (3,-1) -- (2,-2) -- (0,-2);
        \draw[cyan](1,-1) -- (1, 1);
        \draw[cyan](3,-1) -- (3, 1);
        \draw[cyan](2,-2) -- (2,0);
        \draw[cyan](1,1) -- (3,1) -- (2,0) -- (0,0) -- (1,1);
        %P11
        \filldraw[black] (-0.8,-1.5) circle (2pt) node[above right] {$P_{11}$};
        \draw[dashed] (-0.8,-1.5) -- (-0.8,1.5);
        \filldraw[black](-0.8,1.5) circle (2pt) node[above right] {$f(P_{11})$};
        %P22
        \filldraw[black] (2.5,-0.5) circle (2pt) node[above right] {$P_{22}$};
        \draw[dashed] (2.5,-0.5) -- (2.5,2);
        \filldraw[black](2.5,2) circle (2pt) node[above right] {$f(P_{22})$};
        %P21
        \filldraw[black] (0.5,-0.4) circle (2pt) node[above right] {$P_{21}$};
        \draw[dashed] (0.5,-0.4) -- (0.5,2.5);
        \filldraw[black](0.5,2.5) circle (2pt) node[above right] {$f(P_{21})$};
        %P12
        \filldraw[black] (1.8,-1.6) circle (2pt) node[above right] {$P_{12}$};
        \draw[dashed] (1.8,-1.6) -- (1.8,0.4);
        \filldraw[black](1.8,0.4) circle (2pt) node[above right] {$f(P_{12})$};
    \end{tikzpicture}
    \hfill\\
    Questo è un esempio di somma di Cauchy-Riemann con $n=2$. In cui viene approssimata una funzione $f(x,y)$ con dei parallelepipedi.
\end{center}
\normalsize
\begin{definition}
    Diremo che la funzione $f:[a.b] \times [c,d]$, 
    limitata, è integrabile sul rettangolo $R=[a,b] \times 
    [c,d]$ se esiste finito il limite di  $\lim_{n \to \infty}s_n$ e 
    \underbar{se} tale limite \textbf{non} dipende da come si sono scelti 
    i punti di $P_{hk}$ nei rispettivi rettangoli ad ogni passo della costruzione. 
\end{definition}
\begin{example}
    $f[0,1]\times[0,1] \to \mathbb{R}$ definita $f(x,y) =   \begin{cases}
        1 & \text{se } x \in \mathbb{Q} \\
        0 & \text{se } x \in \mathbb{R}\backslash\mathbb{Q}
    \end{cases} $\\
    $f$ NON è integrabile $\qquad S_n = \sum 1 |I_{hk}| = 1$\\
\end{example}
\begin{theorem}
    \underbar{se} $f:[a,b] \times [c,d]$ è contuinua
    \underbar{allora} è integrabile.
\end{theorem}
(no dim)
\begin{center}
    \begin{tikzpicture}
        %assi x y e z
            \draw[->](0,0) -- (8,0) node[right] {$y$};
            \draw[->](0,0) -- (0,5) node[above] {$z$};
            \draw[->](0,0) -- (-3.5,-3.5) node[below right] {$x$};
            %parallelepipedo
            \draw[very thick](1,-1) -- (-1,-3) -- (0,-3) -- (2,-1) -- (1,-1);
            \draw[very thick](1,-1) -- (1,2);
            \draw[very thick](-1,-3) -- (-1,0);
            \draw[very thick](0,-3) -- (0,0);
            \draw[very thick](2,-1) -- (2,2);
            \draw[very thick](1,2) -- (-1,0) -- (0,0) -- (2,2) --(1,2);
            %Parappellepipedo inclinato
            \draw[very thick](2,-1) -- (6,-1) -- (4,-3) -- (0,-3);
            \draw[very thick](6,-1) -- (6,1.5);
            \draw[very thick](4,-3) -- (4,-0.7);
            \draw[very thick](2,2) ..controls(3.5,2.1).. (6,1.5) node[midway, above]{$z=f(x,y)$};
            \draw[very thick](6,1.5) ..controls(5,0.8).. (4,-0.7);
            \draw[very thick](4,-0.7) ..controls(3.5,-1)and(1,0).. (0,0);
            %punti e proiezioni
            \draw(-3.2,-3) -- (-2.8,-3) node[at start, left]{$b$};
            \draw(-1.2,-1) -- (-0.8,-1) node[at start, left]{$a$};
            \draw[dashed](-2.8,-3) -- (-1,-3);
            \draw[dashed](-0.8,-1) -- (1,-1);
            \draw(1.8,-0.2) -- (2.2,0.2) node[above, at end]{$c$};
            \draw(6.8,-0.2) -- (7.2,0.2) node[above, at end]{$d$};
            \draw[dashed](1,-1) -- (1.8,-0.2);
            \draw[dashed](6,-1) -- (6.8,-0.2);
    \end{tikzpicture}
\end{center}
\begin{gather*}
    \left( \int_{[a,b] \times [c,d]} f(x,y) \ dx\ dy \right) = \int_{c}^{d} \left( \int_{a}^{b} f(x,y) \ dx\right) \ dy  
\end{gather*}
\begin{example}
    \begin{gather*}
        \int_{[0,1] \times [0,2]} x e^{xy} \ dx \ dy\\
        \int_{0}^{1} \left( \int_{0}^{2} f(x,y) \ dx \right) \ dy \Rightarrow e^{xy} |_{y=0}^{y=2}  
    \end{gather*}
\end{example}
\begin{theorem}
    \underbar{Se} $f:[a,b] \times [c,d] \to \mathbb{R}$ è continua\\
    \underbar{allora} 
    \begin{gather*}
        \left( \int_{[a,b] \times [c,d]} f(x,y) \ dx \ dy\right) = \int_{c}^{d} \left( \int_a^b f(x,y) \ dx \right) \ dy = \int_{a}^{b} \left(\int_{c}^{d} f(x,y) \ dy\right) \ dx       
    \end{gather*}
\end{theorem}
\begin{lemma}
    \underbar{Se} $f(x,y)$ è continua in $[a,b] \times [c,d]$\\
    La funzione $\varphi(x) = \int_{c}^{d} f(x,y) \ dy $ è continua in $[a,b]$.\\
    Analogamente $\psi(y) = \int_{a}^{b} f(x,y) \ dx$ \tab\tab \tiny (no dim)\normalsize\\
\end{lemma}
Ho diviso l'intervallo in varie parti:
\begin{center}
    \tiny
    \begin{tikzpicture}
        \draw[->](0,0) -- (4,0) node[right] {\normalsize$x$\tiny};
        \draw(0.5,-0.2) -- (0.5,0.2) node[below, at start]{$a$};
        \draw(1,-0.2) -- (1,0.2);
        \draw(1.5,-0.2) -- (1.5,0.2) node[below, at start]{$x_{h-1}$};
        \draw(2,-0.2) -- (2,0.2)  node[below, at start]{$x_h$};
        \draw(2.5,-0.2) -- (2.5,0.2);
        \draw(3,-0.2) -- (3,0.2) node[below, at start]{$b$};
        \draw[red](1.5,0) -- (2,0) node[above, midway]{$\textcolor{white}{x_h^*}$};
    \end{tikzpicture}
    \normalsize
\end{center}
\begin{gather*}
    \int_{a}^{b} \ dx = \sum_{h=1}^{n} \int_{x_{h-1}}^{x_h} \varphi(x) \ dx \overset{\text{\tiny th. della m int. \normalsize}}{=} \sum_{h=1}^{n} \varphi(x_h^*) (x_h - x_{h-1}) = \left(\frac{b-a}{n}\right) \sum_{h=1}^{n} \varphi(x_h^*)\\
\end{gather*}
Nell'ultimo passaggio praticamente si è considerato che ogni intervallo è uguale a $\frac{b-a}{n}$ e questo termine è moltiplicato per ogni termine della sommatoria, siccome non dipende dall'indice si porta   in evidenza.
\begin{gather*}
    = \left(\frac{b-a}{n}\right) \sum_{h=1}^{n} \int_{c}^{d} f(x_h^* , y) \ dy = \left(\frac{b-a}{n}\right) \sum_{k=1}^{n} \int_{y{k-1}}^{y_k} f(x_h^*,y) \ dy\\
\end{gather*}
Ora siamo passati dalla funzione $\varphi(x^*_h)$ a $f(x^*_h,y)$, lo possiamo fare perchè $\varphi(x) = \int_{c}^{d} f(x,y) \ dy$ in generale ero solo partito dall'intervallo $[a,b]$ ma posso farlo con entrambi, così facendo posso riscrivere l'integrale con la sommatoria e riapplicare il th. della media integrale per finire i passaggi.
\begin{gather*}
    = \left(\frac{b-a}{n}\right) \sum_{h=1}^{n} \sum_{k=1}^{n} f(x_h^*,y_k^*) \underset{\left(\frac{d-c}{n}\right)}{\left(y_k - y_{k-1}\right) } \\
    = \sum_{h,k = 1}^{n} f(x_h^*,y_k^*) \underbrace{\left(\frac{b-a}{n}\right)\left(\frac{d-c}{n}\right)}_{\left\lvert I_{hk} \right\rvert }\\
    = \sum_{h,k=1}^n f(x_h^*,y_k^*) |I_{hk}| 
    \end{gather*}
Si ricorda che abbiamo supposto che la funzione sia continua, e quindi integrabile. Arriviamo ad una particolare somma di rimann che dipende dai punti che non possiamo scegliere arbitrariamente ma ci sono stati forniti dal th. della media integrale.\\
\subsection{Integrali su domini non rettangolari}
\begin{definition}
    \underbar{sia} $f:\Omega \to \mathbb{R}$ , con $\Omega \subseteq \mathbb{R}^2$ limitato\\
    \underbar{sia} $R = [a,b] \times [c,d]$ un rettangolo contenente $\Omega$ \\
    e \underbar{sia} $\tilde{f}:R \to \mathbb{R}$ definita: $\tilde{f}(x,y) = \begin{cases}
        f(x,y) & \text{se } (x,y) \in \Omega \\
        0 & \text{se } (x,y) \in R \backslash \Omega
    \end{cases}$\\
    \underbar{Se} $\tilde{f}$ è integrabile su $R$ diremo che $f$ è integrabile su $\Omega$ e:
    \begin{gather*}
        \int_{R} \tilde{f}(x,y) \ dx \ dy = \int_{\Omega} f(x,y) \ dx \ dy
    \end{gather*}
\end{definition}
\begin{definition}[insiemi semplici e regolari]
    $E \subseteq \mathbb{R}^2$ si duce y-semplice se è del tipo:
    \begin{gather*}
        E \{ (x,y):x \in [a,b] \quad g_1(x) \leq y \leq g_2(x) \ \}
    \end{gather*} 
    con $g_1,g_2:[a,b] \to \mathbb{R}$ continue\\
    Inoltre si dice \underbar{regolare} se è unione di un numero finito di insiemi semplici.
\end{definition}
\begin{center}
    \tiny
    \begin{tikzpicture}
        \draw[->](0,0) -- (3,0) node[right] {\normalsize$x$\tiny};
        \draw[->](0,0) -- (0,3) node[above] {\normalsize$y$\tiny};
        \draw(-0.2,1) -- (0.2,1) node[left, at start]{$1$};
        \draw(-0.2,2) -- (0.2,2) node[left, at start]{$2$};
        \draw(1,-0.2) -- (1,0.2) node[below, at start]{$1$};
        \draw(2,-0.2) -- (2,0.2) node[below, at start]{$2$};
        \draw[dashed](1,1) -- (1,2) -- (2,2);
        \draw(1,1) -- (2,1) -- (2,2);
        \draw(1,1) -- (2,2);
        \node at (1.7,1.3) {$E$};
    \end{tikzpicture}
    \hfill\\
    Questo grafico rappresenta un caso in cui una funzione è sia x-semplice sia y-semplice.
    \normalsize
\end{center}
Nello specifico il grafico descrive:
\begin{gather*}
    E = \{ (x,y) 1 \leq x \leq 2 \quad 1\leq y \leq x \}
\end{gather*}

\newpage
\section{19/11/25}
\begin{definition}
    Un insieme $E \subseteq \mathbb{R}^2$ \\
    y-semplice è del tipo:
    \begin{gather*}
        E = \{ (x,y): x \in [a,b], \quad g_1(x) \leq y \leq g_2(x) \}
    \end{gather*}
    Dove $g_1,g_2:[a,b] \to \mathbb{R}$ sono continue
\end{definition}
*Un nome spesso usato per y-semplice è "normale rispetto alle x"\\
\begin{center}
    \begin{tikzpicture}
        \draw[->](0,0) -- (4,0) node[right] {$x$};
        \draw[->](0,0) -- (0,3) node[above] {$y$};
        \draw(1,-0.2) -- (1,0.2) node[below, at start]{$a$};
        \draw(3,-0.2) -- (3,0.2) node[below, at start]{$b$};
        \draw(1,1) -- (1,2);
        \draw(3,0.5) -- (3,2.5);
        \draw(1,2) .. controls (1.5,3) .. (3,2.5) node[midway, above]{$y=g_2(x)$};
        \draw(1,1) .. controls (1.5,1.5) .. (3,0.5) node[midway, below]{$y=g_1(x)$};
        \draw[dashed](1.2,0.2) -- (1.2,3.5);
        \draw[dashed](2,0.2) -- (2,3.5);
        \draw[dashed](2.8,0.2) -- (2.8,3.5);
        \draw[red](1,0) -- (3,0);
    \end{tikzpicture}
    \begin{tikzpicture}
        \draw[->](0,0) -- (4,0) node[right] {$x$};
        \draw[->](0,0) -- (0,3) node[above] {$y$};
        \draw(1.3,-0.2) -- (1.3,0.2) node[below, at start]{$a$};
        \draw(3,-0.2) -- (3,0.2) node[below, at start]{$b$};
        \draw(3,1) .. controls (2.5,2) .. (3,3);
        \draw(3,1) .. controls (1,2) .. (3,3);
        \draw[dashed](2,0.2) -- (2,3.5);
        \draw[dashed](2.8,0.2) -- (2.8,3.5);
        \draw[red](1.3,0) -- (3,0);
    \end{tikzpicture}
    \begin{tikzpicture}
        \draw[->](0,0) -- (4,0) node[right] {$x$};
        \draw[->](0,0) -- (0,3) node[above] {$y$};
        \draw(1,-0.2) -- (1,0.2) node[below, at start]{$a$};
        \draw(3.5,-0.2) -- (3.5,0.2) node[below, at start]{$b$};
        \draw(1,1) rectangle (2,2);
        \draw(2.5,1) rectangle (3.5,2);
        \draw[dashed](1.2,0.2) -- (1.2,3.5);
        \draw[dashed](2.8,0.2) -- (2.8,3.5);
        \draw[red](1,0) -- (2,0);
        \draw[red](2.5,0) -- (3.5,0);
    \end{tikzpicture}
\end{center}
Perchè una funzione sia y-semplice deve poter essere proiettata sull'asse delle x in modo che formi un solo segmento, inoltre qualsiasi retta lo intersechi verticalmente intersecandolo al massimo in due punti.\\
Perchè una funzione sia x-semplice deve poter essere proiettata sull'asse delle y in modo che formi un solo segmento, inoltre qualsiasi retta lo intersechi orizzontalmente intersecandolo al massimo in due punti.\\
\begin{example}
    
\end{example}
\begin{theorem}
    \underbar{Se} $f:\Omega \to \mathbb{R}$ è continua in $\Omega$ e\\
    \underbar{se} $\Omega$ è regolare\\
    \underbar{allora} $f$ è integrabile
    \begin{center}
        \begin{tikzpicture}
            \draw(0,0) rectangle (3,3);
            \draw(0.5,1) -- (0.5,2);
            \draw(2.5,0.5) -- (2.5,2.5);
            \draw(0.5,2) .. controls (1.5,3) .. (2.5,2.5);
            \draw(0.5,1) .. controls (1.5,0.5) .. (2.5,0.5);
            \node at(1,2){$f$};
            \node at(1.5,1.5){$\Omega$};
        \end{tikzpicture}
    \end{center}
\end{theorem}
\subsection{Misura di un insieme del piano}
\begin{center}
    \begin{tikzpicture}
        \draw[->](0,0) -- (0,4) node[above] {$y$};
        \draw[->](0,0) -- (4,0) node[right] {$x$};
        \draw(1,-0.2) -- (1,0.2) node[below, at start]{$a$};
        \draw(3,-0.2) -- (3,0.2) node[below, at start]{$b$};
        \draw(1,1) -- (3,1);
        \draw[dashed](1,0) -- (1,1);
        \draw[dashed](3,0) -- (3,1);
    \end{tikzpicture}
\end{center}
\hfill\\\hfill\\
\begin{definition}
    Sia $\Omega \subseteq \mathbb{R}^2$ limitato\\
    $\Omega$ si dice misuraile se la funzione costante 1 in $\Omega$ è integrabile.\\
    In tal caso chiameremo misura (o area) di $\Omega$ e indicheremo con il simbolo $|\Omega|$ il numero:
    \begin{gather*}
        |\Omega| = \int_{\Omega} 1 \ dx \ dy
    \end{gather*} 
\end{definition}
\begin{observation}
    \hfill\\
    \begin{enumerate}[$i)$]
    \item\underbar{Se} $\Omega$ è regolare\\
    \underbar{allora} è misurabile (perchè la funzione è continua in $\Omega$ e quindi integrabile per il th. precedente).
    \item\underbar{Se} $\Omega=\{(x,y): x \in[0,1] \cap \mathbb{Q}, y \in [0,1]\}$ 
    \end{enumerate}
    \begin{center}
        \begin{tikzpicture}
            \draw[->](1,1) -- (1,4) node[above] {$y$};
            \draw[->](1,1) -- (4,1) node[right] {$x$};
            \draw(1,0.8) -- (1,1.2) node[below, at start]{$0$};
            \draw(3,0.8) -- (3,1.2) node[below, at start]{$1$};
            \draw(0.8,1) -- (1.2,1) node[left, at start]{$0$};
            \draw(0.8,3) -- (1.2,3) node[left, at start]{$1$};
            \draw(1,1) rectangle (3,3);
            \draw[dashed](1.2, 1) -- (1.2,3);
            \draw[dashed](1.4, 1) -- (1.4,3);
            \draw[dashed](1.6, 1) -- (1.6,3);
            \draw[dashed](1.8, 1) -- (1.8,3);
            \draw[dashed](2, 1) -- (2,3);
            \draw[dashed](2.2, 1) -- (2.2,3);
            \draw[dashed](2.4, 1) -- (2.4,3);
            \draw[dashed](2.6, 1) -- (2.6,3);
            \draw[dashed](2.8, 1) -- (2.8,3);
            \draw[dashed](3, 1) -- (3,3);
        \end{tikzpicture}
    \end{center}
        $\Omega$ NON è misurabile

\end{observation}
\hfil\\
Quando Un insieme è misurabile e ha misura uguale a zero?\\
\begin{proposition}
    Un insieme limitato $\Omega \subseteq \mathbb{R}^2$ è misurabile e ha misura zero\\
    \underbar{se e solo se} vale la seguente propirietà: \\
    Detto $R$ un rettangolo contenente $\Omega$ considerata la suddivisione di $R$ in $n^2$ rettangoli uguali (come nella def. di integrale) e detta
    $A_n$ la somma delle aree dei rettangoli che hanno intersezione non vuota con $\Omega$ si ha che $A_n \to 0 $ per $n \to \infty$ 
\end{proposition}
\begin{center}
    \begin{tikzpicture}
        \draw[->](0,0) -- (0,4) node[above] {$y$};
        \draw[->](0,0) -- (4,0) node[right] {$x$};
        \draw(1,1) rectangle (3.8,3.8);
        \draw[dashed](1,0) -- (1,1) node[at start, below]{$a$};
        \draw[dashed](1.4,0) -- (1.4,1);
        \draw[dashed](1.8,0) -- (1.8,1);
        \draw[dashed](2.2,0) -- (2.2,1);
        \draw[dashed](2.6,0) -- (2.6,1);
        \draw[dashed](3,0) -- (3,1);
        \draw[dashed](3.4,0) -- (3.4,1);
        \draw[dashed](3.8,0) -- (3.8,1) node[at start, below]{$b$};
        \draw[dashed](0,1) -- (1,1) node[left, at start]{$c$};
        \draw[dashed](0,1.4) -- (1,1.4);
        \draw[dashed](0,1.8) -- (1,1.8);
        \draw[dashed](0,2.2) -- (1,2.2);
        \draw[dashed](0,2.6) -- (1,2.6);
        \draw[dashed](0,3) -- (1,3);
        \draw[dashed](0,3.4) -- (1,3.4);
        \draw[dashed](0,3.8) -- (1,3.8) node [left, at start]{$d$};
        \draw(1.4,1) -- (1.4,3.8);
        \draw(1.8,1) -- (1.8,3.8);    
        \draw(2.2,1) -- (2.2,3.8);
        \draw(2.6,1) -- (2.6,3.8);
        \draw(3,1) -- (3,3.8);
        \draw(3.4,1) -- (3.4,3.8);
        \draw(1,1.4) -- (3.8,1.4);
        \draw(1,1.8) -- (3.8,1.8);    
        \draw(1,2.2) -- (3.8,2.2);
        \draw(1,2.6) -- (3.8,2.6);
        \draw(1,3) -- (3.8,3);
        \draw(1,3.4) -- (3.8,3.4);
        \draw(1,3.8) -- (3.8,3.8);
        \filldraw[fill=white](1,1.4) rectangle (1.4,1.8);
        \filldraw[fill=white](1.4,1.4) rectangle (1.8,2.2);
        \filldraw[fill=white](1.8,2.2) rectangle (2.2,2.6);
        \filldraw[fill=white](2.6,1.8) rectangle (3,2.2);
        \filldraw[fill=white](2.2,2.2) rectangle (2.6,2.6);
        \filldraw[fill=white](3,1.4) rectangle (3.4,1.8);
        \filldraw[fill=white](3.4,1.4) rectangle (3.8,1.8);
    \end{tikzpicture}\\
    Rappresentazione di una curva sul dominio.
\end{center}
$g(x)$ continua in $[a,b]$\\
\fbox{Il grafico di $g$ ha misura nulla}
\begin{proposition}
    L'unione di un numero finito di insiemi di misura nulla ha misura nulla.
\end{proposition}
\begin{corollary}
    Il bodro di un insieme semplice e il bordo di un insieme regolare hanno misura nulla.
\end{corollary}
\begin{theorem}
    \underbar{Sia} $\Omega \to \mathbb{R}$ un dominio regoalre e $f:\Omega \to \mathbb{R}$ una funzione limitata continua a eccezione di un insieme di misura nulla di punti di discontinuità.\\
    \underbar{Allora} è integrabile
\end{theorem}
\begin{example}
    $f:[0,1]\times[0,1] \to \mathbb{R}$
    \begin{gather*}
        f(x,y)\begin{cases}
            1 \quad \text{se} \quad y\leq x\\
            -1 \quad \text{se} \quad y<x\\
        \end{cases}
    \end{gather*}
\end{example}
\begin{theorem}
    \underbar{se} $f:[a,b] \times [c,d] \to \mathbb{R}$ è limitatta e continua salvo un insieme di misrua nulla di punti di discontinuità\\
    \underbar{allora} valgono ancora le formule di riduzione cioè:
    \begin{gather*}
        \int_{[a,b]\times[c,d]} f \ dx \ dy = \int_{a}^{b} \left( \int_{c}^{d} f(x,y) \ dy \right) \ dx = \int_{c}^{d} \left( \int_{a}^{b} f(x,y) \ dx \right) \ dy  
    \end{gather*}
\end{theorem}

\begin{proposition}
        \underbar{sia} $\Omega \subseteq \mathbb{R}^2$ limitato e misurabile, $f$ e $g$ funzioni integrabili in $\Omega$, $c \in \mathbb{R}$\\
    \begin{enumerate}[$a)$]
        \item l'integrale è lineare:
        \begin{gather*}
            \int_{\Omega} (f+g) \ dx \ dy = \int_{\Omega} f \ dx \ dy + \int_{\Omega} g \ dx \ dy\\
            \int_{\Omega} c \ f \ dx \ dy = c \int_{\Omega} f \ dx \ dy
        \end{gather*}
        \item \underbar{se} $f\geq 0 \Rightarrow \int_{\Omega} f \ dx \ dy \geq 0$\\
        \underbar{se} $f \geq g \Rightarrow \int_{\Omega} f \ dx \ dy \ \geq \int_{\Omega} g \ dx \ dy$\\
        In particolare dato che $-|f| \leq f < |f|$
        \begin{gather*}
            \Rightarrow \ -\int_{\Omega} |f| \ dx \ dy \leq \int_{\Omega} f \ dx \ dy \leq \int_{\Omega} |f| \ dx \ dy \Rightarrow \left| \int_{\Omega} f \ dx \ dy \right| \leq \int_{\Omega} |f| \ dx \ dy
        \end{gather*}
        \item \underbar{se} $\Omega' \subseteq \Omega$ è misurabile \\
        \underbar{allora} $f$ è integrabile in $\Omega'$, \underbar{se} inoltre $f \geq 0$ \\
        \underbar{allora} $\int_{\Omega'} f \ dx \ dy \ \leq \int_{\Omega} f \ dx \ dy$
        \item\underbar{se} $\Omega_1 \Omega_2$ sono domini regolari, $\Omega_1 \cap \Omega_2$ ha misura nulla e f è integrabile in $\Omega_1 \cup \Omega_2$\\
        \underbar{allora} 
        \begin{gather*}
            \int_{\Omega_1 \cup \Omega_2} f \ dx \ dy = \int_{\Omega_1} f \ dx \ dy + \int_{\Omega_2} f \ dx \ dy
        \end{gather*}
        \item se $|\Omega| = 0 \Rightarrow \int_{\Omega} f \ dx \ dy = 0$
    \end{enumerate}
\end{proposition}
\begin{theorem}[formula di riduzione nel caso di insiemi semplici]
    \underbar{Sia} $f:\Omega \to \mathbb{R}$ continua e sia $\Omega$ un dominio x-semplice, cioè:
    \begin{gather*}
        \Omega = \{ (x,y) \mathbb{R}^2: y \in[c,d] h_1(y) \leq x \leq h_2(y)\}
    \end{gather*}
    con $h_1,h_2:[c,d] \to \mathbb{R}$ continue.\\
    \underbar{Allora}:
    \begin{gather*}
        \int_{\Omega} f(x,y) \ dx \ dy = \int_{c}^{d} \left( \int_{h_1(y)}^{h_2(y)} f(x,y) \ dx \right) \ dy
    \end{gather*}
\end{theorem}
\begin{example}
    calcolare $\int_{\Omega} xy \ dx \ dy$
    \begin{gather*}
        \Omega = \{(x,y): y \int [0,1], y^2\leq x \leq \sqrt{y} \}\\
        \int_{\Omega} xy \ dx \ dy = \int_{0}^{1} \left( \int_{y}^{\sqrt{y}} xy \ dx \right) \ dy = \int_{0}^{1} \left( \frac{x^2 y}{2} |_{x=y^2}^{x=\sqrt{y}} \right) \ dy = \int_{0}^{1} \left( \frac{y^2}{2} - \frac{y^5}{2} \right) \ dy = \left[ \frac{y^3}{6} - \frac{y^6}{12} \right]^1_0 = \left( \frac{1}{6}- \frac{1}{12} \right) - 0  \\ 
    \end{gather*}
\end{example}
\begin{example}
    \begin{center}
        \begin{gather*}
            \int_{\Omega} e^x \ dx \ dy
        \end{gather*}
        \begin{tikzpicture}
            \draw[->](0,0) -- (4,0) node[right] {$x$};
            \draw[->](0,0) -- (0,4) node[above] {$y$};
            \draw(0,0) -- (3,3) node[midway, above,left]{$y=x$};
            \draw(3,0) -- (3,3);
            \node at(3,0) [below]{$1$};
            \node at(0,3) [left]{$1$};
        \end{tikzpicture}
    \end{center}
    \hfil\\
    y-semplice  $= \{(x,y): x \in[0,1] \quad 0 \leq y = x\}$
    \begin{gather*}
        \int_{0}^{1} \left( \int_{0}^{x} e^{x^2}\ dy \right) \ dx = \frac{1}{2} \int_{0}^{1} 2x e^{x^2} \ dx = \frac{1}{2} e^{x^2} ]_0^1 = \frac{1}{2} (e-1) 
    \end{gather*}
    x-semplice $= \{(x,y): y \in[0,1] \quad y \leq x \leq 1\}$
    \begin{gather*}
        \int_{0}^{1} \left( e^{x^2} \ dx \right) \ dy 
    \end{gather*}
\end{example}
\begin{example}
    Calcolare $\int_{\Omega} x(1-y) \ dx \ dy$ con $\Omega$ che è l'area tra unarco di raggio 1 e una retta $y=x$:
    \begin{center}
        \begin{tikzpicture}
            \draw[->](0,0) -- (4,0) node[right] {$x$};
            \draw[->](0,0) -- (0,4) node[above] {$y$};
            \draw(0,0) -- (3,3) node[midway, above,left]{$y=x$};
            \draw (3,0) arc (0:90:3);
            \draw[dashed](2.1,0) -- (2.1,2.1) node[at start, below]{$\frac{1}{\sqrt{2}}$};
        \end{tikzpicture}
    \end{center}
\end{example}
$\Omega$ è x-semplice $= \{ (x,y): y \in [0,\frac{1}{\sqrt{2}}] \quad y \leq x \leq \sqrt{1-y^2} \}$
\begin{gather*}
    \int_{0}^{\frac{1}{\sqrt{2}}} \left( \int_{y}^{\sqrt{1-y^2}} x(1-y) \ dx \right) \ dy
\end{gather*}
y-semplice $= \{(x,y): x \in [0,1] \quad  0 \leq y \leq h(x) \}$
\begin{gather*}
    h(x)=
    \begin{cases}
        x \quad \text{se} x \in [0,\frac{1}{\sqrt{2}}]\\
        \sqrt{1-x^2} \quad \text{se} x \in [\frac{1}{\sqrt{2}},1]
    \end{cases}\\
    \int_{0}^{1} \left( \int_{0}^{h(x)} x(1-y) \ dy \right) \ dx = \int_{0}^{\frac{1}{\sqrt{2}}} \left( x \left( 1-y \right) \right) \ dx + \int_{\frac{1}{\sqrt{2}}}^{1} \left( x \left( 1-y \right)  \right) \ dx\\
\end{gather*}

\newpage
\section{20/11/25}
Dalla lez. precedente si richiama un teorema:
\begin{theorem}
    \underbar{Sia} $\Omega \to \mathbb{R}$ un dominio regoalre e $f:\Omega \to \mathbb{R}$ una funzione limitata continua a eccezione di un insieme di misura nulla di punti di discontinuità.\\
    \underbar{Allora} è integrabile
\end{theorem}
Tra quelli visti fino ad esso è il più generale
\begin{example}
    ci permette ad esempio di dire che se ho una funzione in un dominio regolare, questa è integrabile perchè se la vado a mettere in un rettangolo.
    \begin{center}
        \begin{tikzpicture}
            \draw(-1.5,-1.5) rectangle (1.5,1.5);
            \draw[red](0,0) circle (0.5);
            \draw[red](0,0) circle (1);
        \end{tikzpicture}
    \end{center}
    Da qui si vede che la ha misura 0.
\end{example}
Altro richiamo dalla lezione scorsa:
\begin{theorem}[formula di riduzione nel caso di insiemi semplici]
    \underbar{Sia} $f:\Omega \to \mathbb{R}$ continua e sia $\Omega$ un dominio x-semplice, cioè:
    \begin{gather*}
        \Omega = \{ (x,y) \mathbb{R}^2: y \in[c,d] h_1(y) \leq x \leq h_2(y)\}
    \end{gather*}
    con $h_1,h_2:[c,d] \to \mathbb{R}$ continue.\\
    \underbar{Allora}:
    \begin{gather*}
        \int_{\Omega} f(x,y) \ dx \ dy = \int_{c}^{d} \left( \int_{h_1(y)}^{h_2(y)} f(x,y) \ dx \right) \ dy 
    \end{gather*}
\textbf{Vale} una formula analoga per domini y-semplici
\end{theorem}
\begin{proof}
Ora procediamo a dimostrarla
    \begin{center}
        \begin{tikzpicture}
            \draw[->](0,0) -- (0,3) node[at end, left]{$y$};
            \draw[->](0,0) -- (4,0) node[at end, below]{$x$};
            \draw(-0.2,2) -- (0.2,2) node[at start, left]{$d$};
            \draw(-0.2,1) -- (0.2,1) node[at start, left]{$c$};
            \draw(1,-0.2) -- (1,0.2) node[at start, below]{$a$};
            \draw(3,-0.2) -- (3,0.2) node[at start, below]{$b$};
            \draw[thin, gray](1,1) rectangle (3,2) node[at end, right]{$R$};
            \draw[ultra thick](1.5,2) -- (2.5,2);
            \draw[ultra thick](1.3,1) -- (2,1);
            \tiny
            \draw[ultra thick](1.5,2) ..controls(1.7,1.3).. (1.3,1) node[midway, left]{$x=h_1(y)$};
            \draw[ultra thick](2.5,2) ..controls(2.7,1.7)and(1.7,1.3).. (2,1) node[midway, right]{$x=h_2(y)$};
            \normalsize
            \node at(1.95,1.7){$\Omega$};
            \draw(0,1.2) -- (3.5,1.2) node[at start, left]{$y$};
            \tiny
            \draw[dashed](1.55,1.2) -- (1.55,0) node[at end, below]{$h_1(y)$};
            \draw[dashed](1.95,1.2) -- (1.95,0);
            \node at(2.2,-0.2){$h_2(y)$};
            \normalsize
            \filldraw[black](0,1.2) circle (1.3pt);
            \filldraw[black](1.55,1.2) circle (1.3pt);
            \filldraw[black](1.95,1.2) circle (1.3pt);
        \end{tikzpicture}
    \end{center}
    definisco 
    \begin{gather*}
        \tilde{f} : R \to \mathbb{R} \text{ come } \tilde{f}(x,y) = \begin{cases}
            f(x,y) \quad \text{se } (x,y) \in \Omega\\
            0 \quad \text{se } (x,y) \in R \backslash \Omega
        \end{cases}\\
        \int_{\Omega} f \ dx \ dy = \int_{R} \tilde{f} \ dx \ dy = \int_{c}^{d} \left( \int_{a}^{b} \tilde{f} (x,y) \ dx \right) \ dy \qquad \boxed{*} 
    \end{gather*}
    grafico di $\tilde{f}(x,y)$ come funzione di $x$ con $y$ fissato
\begin{minipage}{0.48\textwidth}
            \begin{tikzpicture}
                \draw[->](0,0) -- (0,3) node[at end, left]{$z$};
                \draw[->](0,0) -- (3,0) node[at end, below]{$x$};
                \tiny
                \draw(1,-0.2) -- (1,0.2) node[at start, below]{$h_1(y)$};
                \draw(2,-0.2) -- (2,0.2) node[at start, below]{$h_2(y)$};
                \normalsize
                \draw(0.5,-0.2) -- (0.5,0.2) node[at start, below]{$a$};
                \draw(2.5,-0.2) -- (2.5,0.2) node[at start, below]{$b$};
                \draw[dashed](1,0) -- (1,1.5);
                \draw[dashed](2,0) -- (2,1.5);
                \draw(1,1.5) ..controls(1.2,1.7)and(1.4,1.3).. (2,1.5) node[midway, above]{$z=f(x,y)$};
            \end{tikzpicture}
\end{minipage}
    \begin{gather*}
        \int_{a}^{b} \tilde{f} \ dx = \left( \int_{a}^{h_1(y)} + \int_{h_1(y)}^{h_2(y)} + \int_{h_2(y)}^{b} \right) \tilde{f} (x,y) \ dx\\
        = 0 + \int_{h_1(y)}^{h_2(y)} f(x,y) dx + 0 = \int_{h_1(y)}^{h_2(y)} f(x,y) \ dx 
    \end{gather*}
    Da adesso se torniamo al passaggio \fbox{*} 
    \begin{gather*}
        = \int_{c}^{d} \left( \int_{h_1(y)}^{h_2(y)} f(x,y) \ dx \right) \ dy 
    \end{gather*}
\end{proof}
\begin{example}
    Ho un semicerchio di raggio 1 e centro $(2,0)$
\end{example}

\begin{wrapfigure}{l}{0.45\textwidth}
    \centering
            \begin{tikzpicture}
                \draw[->](0,-1) -- (0,2) node[at end, left]{$y$};
                \draw[->](0,0) -- (3,0) node[at end, below]{$x$};
                \draw(2.5,0) arc (0:180:1);
                \node at(0.5,0)[below]{$1$};
                \node at(2.5,0)[below]{$3$};
                \draw(0,-0.8)--(0.5,0);
                \draw(0,-0.8)--(2.5,0);
                \node at(0,-0.8)[left]{$-1$};
            \end{tikzpicture}
\end{wrapfigure}
    \hfil\\
    Prendo il dominio: $(x-2)^2+y^2=1$ e lo scrivo come x-semplice e y-semplice:
    \begin{gather*}
        D_1 = \{ (x,y): \quad x \in [0,1] \quad -1+\frac{x}{3} \leq y \leq -1 +x\}\\
        D_2 = \{ (x,y): \quad x \in[1,3] \quad -1+\frac{x}{3} < y \leq \sqrt{1 - (x-2)^2} \}
    \end{gather*}
    Procedo ora con l'integrale:
    \begin{gather*}
        \text{si ricorda: } \int_{a}^{b} \left( \int_{g_1(x)}^{g_2(x)} f(x,y) \ dy \right) \ dx = \int_{a}^{b} dx \int_{g_1(x)}^{g_2(x)} f(x,y) \ dy \\
    \end{gather*}
    Calcolo vero e proprio:
    \begin{gather*}
        \int_{\Omega} xy \ dx \ dy = \int_{D_1 \cup D_2}  = \int_{D_1} + \int_{D_2} = \int_{0}^{1} dx + \int_{-1 + \frac{x}{3}}^{-1+x} xy \ dx \ dy + \int_{1}^{3} dx \int_{-1+\frac{x}{3}}^{\sqrt{1+(x-2)^2}} xy \ dx \ dy\\
        = \int_{0}^{1} dx \left[ \frac{xy^2}{2} \right]^{y=-1+x}_{y=-1+\frac{x}{3}} + \frac{1}{3} dx \left[ \frac{xy^2}{2} \right]^{y= \sqrt{1-(x-2)^2}}_{y=-1+\frac{x}{3}}  
    \end{gather*}
\subsection{Cambiamenti di variabili}
\begin{gather*}
    \int_{\Omega} f(x,y) dx \ dy
\end{gather*}
\begin{example}
    \begin{gather*}
        \Omega = \left[ B(0,2) \backslash B(0,1) \right] \\
        \int_{\Omega} x^2 \ dx \ dy
    \end{gather*}
\end{example}
\begin{wrapfigure}{r}{0.45\textwidth}
            \begin{tikzpicture}
        \draw[->](0,0) -- (0,3.5) node[at end, left]{$\theta$};
        \draw[->](0,0) -- (3,0) node[at end, below]{$\rho$};
        \draw(1,0) rectangle (2,3) node[at end, right]{$\Omega'$};
        \fill[pattern=north east lines](1,0) rectangle (2,3);
        \draw(-0.2,-0.2) -- (0.2,0.2) node[at start, below]{$0$};
        \draw(-0.2, 3) -- (0.2, 3) node[at start, left]{$2\pi$};
        \node at(1,0)[below]{$1$};
        \node at(2,0)[below]{$2$};
    \end{tikzpicture}
\end{wrapfigure}
Cambio in coordinate polari:
\begin{gather*}
    (x,y) \to (\rho, \theta) = \begin{cases}
        x = \rho \cos(\theta)\\
        y = \rho \sin(\theta)
    \end{cases}\\
    \rho \geq 0 \qquad \theta \in [0,2\pi]
\end{gather*}
\begin{gather*}
    \int_{\Omega '} \rho^2 \cos^2(\theta) \ dx \ dy \overset{?}{\to} (d\rho \ d\theta)
\end{gather*}
\hfill\\
\subsection{cambiamento di differenziale}
\begin{center}
    \begin{tikzpicture}
        \draw[->](0,0) -- (0,3.5) node[at end, left]{$y$};
        \draw[->](0,0) -- (3,0) node[at end, below]{$x$};
        \potato{(1,0)}{1};
        \node at(3.5,1.5){$\Omega$};
        \tiny
        \draw(1.3,1.5) ..controls(1.6,1.7).. (2,1.8) node[midway, above]{1};
        \draw(1.5,1) ..controls(1.8,1.2).. (2.2,1.3) node[midway, below]{3};
        \draw(1.3,1.5) ..controls(1.45,1.3).. (1.5,1) node[midway, left]{2};
        \draw(2,1.8) ..controls(2.15,1.5).. (2.2,1.3) node[midway, right]{4};
        \normalsize
        \draw[rotate around={29:(0,0)}, gray](1.7,0.1) rectangle (2.7,0.7);
    \end{tikzpicture}
    \tab$\quad$
    \begin{tikzpicture}
        \draw(1.3,1) rectangle (2.3,1.5);  
        \draw[dashed](1.3,0) -- (1.3,1) node[at start, below]{$U$};
        \draw[dashed](2.3,0) -- (2.3,1) node[at start, below]{$U+dU$};
        \draw[dashed](0,1) -- (1.3,1) node[at start, left]{$V$};
        \draw[dashed](0,1.5) -- (1.3,1.5) node[at start, left]{$V+dV$};
        \potato{(1,0)}{1};
        \node at(3.5,1.5){$\Omega'$};
        \draw[->](-2.5,1.5) -- (-3.5,1.5) node[above, midway]{\tiny{(trasf. non lineare)}};
        \draw[->](0,0) -- (0,3.5) node[at end, left]{$V$};
        \draw[->](0,0) -- (4,0) node[at end, below]{$U$};
    \end{tikzpicture}
\end{center}
I vari lati distorti del parallelepipedo sono descritti dalle seguenti equazioni:
\begin{align*}
    &1: g(U,V+dV), h(U,V+dV)\\
    &2: g(U,V), h(U,V)\\
    &3: g(U+dU,V), h(U+dU,V)\\
    &4: g(U+dU,V+dV), h(U+dU,v+dV)
\end{align*}
Fondamentalmente approssimeremo l'area del parallelepipedo irregolare ad una regolare per calcolarla...
\begin{gather*}
    \begin{cases}
        x = g(U,V)\\
        y = h(U,V)
    \end{cases}
    T:\underset{\subseteq \text{piano } (U,V)}{\Omega'} \to \underset{\subseteq \text{piano } (x,y)}{\Omega}
\end{gather*}
Supponiamo che:
\begin{itemize}
    \item $T$ sia biunivoca $\Omega ' \to \Omega$
    \item le funzioni $g$ e $h$ siano $C^1 (\Omega')$
    \item la matrice Jacobiana di $T$ ha det. $\neq 0$ in ogni $(U,V) \in \Omega'$
\end{itemize}
Con lo Jacobiano che è:
\begin{gather*}
    \mathcal{J} (U,V) = \begin{pmatrix}
        \frac{\partial g}{\partial U} & \frac{\partial g}{\partial V}\\
        \frac{\partial h}{\partial U} & \frac{\partial h}{\partial V}
    \end{pmatrix}
\end{gather*}
\begin{theorem}[formula di cambiamento di variabili]
    \underbar{se} $\Omega$ è regolare\\
    \underbar{se} $T$ soddisfa le ipotesi scritte adesso\\
    \underbar{allora}:
    \begin{gather*}
        \int_{\Omega} f(x,y) \ dx \ dy = \int_{\Omega'} f(g(U,V), h(U,V)) \left\lvert det \mathcal{J}_T (U,V) \right\rvert \ du \ dv 
    \end{gather*}
\end{theorem}

\begin{gather*}
    x = \rho \cos(\theta) = g(\rho , \theta)\\
    y = \rho \sin(\theta) = h(\rho , \theta)
\end{gather*}
La mia Jacobiana diventa:
\begin{gather*}
    \mathcal{J} = \begin{pmatrix}
        \cos(\theta) & -\rho \sin(\theta)\\
        \sin(\theta) & \rho \cos(\theta)
    \end{pmatrix}
    det(\mathcal{J}) = \rho \cos^2(\theta) + \rho \sin^2(\theta) = \rho\\
    \int_{\Omega} x^2 \ dx \ dy = \int_{\Omega'} \rho^2 \cos^2(\theta) \ d\rho = \int_{0}^{2\pi} \ d\theta (\cos^2(\theta) (4-\frac{1}{4}))\\
    = \frac{15}{4} \int_{0}^{2\pi} \frac{1+\cos(2\theta)}{2} \ d\theta = \frac{15}{4} (0 + \pi) = \frac{15\pi}{4}
\end{gather*}
\begin{gather*}
    \left(g(U,V+dV)-g(U,V), h(U,V+dV)-h(U,V)\right) \\
    \approx \left( \frac{\partial g}{\partial V} dV, \frac{\partial h}{\partial V} dV \right) \left( \frac{\partial g}{\partial V} dU, \frac{\partial h}{\partial V} dU \right)\\
    = \left( \frac{\partial g}{\partial V}, \frac{\partial h}{\partial V} \right) dV\\
    \mathcal{J} (U,V) = \left\lvert \begin{pmatrix}
        \frac{\partial g}{\partial U} & \frac{\partial g}{\partial V}\\
        \frac{\partial h}{\partial U} & \frac{\partial h}{\partial V}
    \end{pmatrix}\right\rvert dU \ dV
\end{gather*}
\begin{example}
    Prendiamo l'integrale:
    \begin{gather*}
        \int_{\Omega} \frac{1}{xy} \ dx \ dy
    \end{gather*}
    Sul dominio $\Omega$:
    \begin{center}
        \begin{tikzpicture}
            \draw[->](0,0) -- (0,4) node[at end, left]{$y$};
            \draw[->](0,0) -- (4,0) node[at end, below]{$x$};
            \draw(0,0) -- (2.6,1.5) node[at end, above]{$y = x$};
            \draw(0,0) -- (2,2) node[at end, above]{$y = 2x$};
            \draw(2,-0.5) -- (-0.5,2) node[at start, below]{$x+y=1$};
            \draw(3,-0.5) -- (-0.5,3) node[at end, left]{$x+y=3$};
            \node at(1.2,0.9){$\Omega$};
        \end{tikzpicture}
    \end{center}
    \begin{gather*}
        \begin{cases}
            U = x+y\\
            V = \frac{y}{x}
        \end{cases}
        \begin{cases}
            \Omega' = \{ (U,V): 1 \leq U \leq 3 \ , \ 1 \leq V \leq 2\}
        \end{cases}\\
        \begin{cases}
            x= \frac{U}{1+V}\\
            y = \frac{UV}{1+U}
        \end{cases}\rightarrow
        \quad \mathcal{J}_T= \begin{pmatrix}
            \frac{1}{1+V} & -\frac{U}{(1+V)^2}\\
            \frac{V}{1+V} & \frac{U}{(1+v)^2}
        \end{pmatrix}\left\lvert det \mathcal{J} \right\rvert = \frac{\left\lvert U\right\rvert }{(1+v)^2}\\
        \int{\Omega'} \frac{1}{\left(\frac{U}{1+V}\right)\left(\frac{UV}{1+V}\right) \frac{\left\lvert U \right\rvert }{(1+V)^2} } \ dU \ dV = \int_{1}^{3} dU \int_{1}^{2} \frac{(1+V)^2}{U^2 V} \ dV =  \dots \dots \ln(2) \ln(3)\\
    \end{gather*}
\end{example}
\begin{example}
    \begin{gather*}
        \int_{\Omega} \sqrt{x^2 + y^2} \ dx \ dy \qquad \Omega = \{ (x,y): \quad (x-1)^2+y^2 \leq 1 \}
    \end{gather*}
    Cambio di variabile che semplifica il dominio:
    \begin{gather*}
        x-1 = \rho \cos(\theta)\\
        y = \rho \sin(\theta)\\
        det \mathcal{J} = \rho \qquad \Omega' = \{ 0 \leq \rho \leq 1 \quad , \quad \theta \int [0,2\pi] \}\\
        \int_{\Omega'} \rho \sqrt{1+\rho \cos(\theta)} \ d\rho \ d\theta = \int_{\Omega'} \sqrt{1+\rho^2 + 2\rho \cos(\theta)} \ d\rho \ d\theta
    \end{gather*}
    Cambio di variabile che mi semplifica la funzione, scelgo coordinate polari cnetrate in $(0,0)$ quindi avrò:
    \begin{gather*}
        \begin{cases}
            x = \rho \cos(\theta) \\
            y = \rho \sin(\theta) 
    \end{cases}\\
    (\rho \cos(\theta))^2 + \rho^2 \sin^2(\theta) = 1\\
    \rho^2 \cos^2(\theta) + 1 -2\rho \cos(\theta) + \rho^2 \sin^2(\theta) = 1\\
    \rho^2 - 2 \rho \cos(\theta) = 0\\
    \rho(\rho -2\cos(\theta)) = 0\\
    \Omega' = \{ (\rho, \theta): \theta \in [-\frac{\pi}{2}, \frac{\pi}{2}] \quad , \quad 0 \leq \rho \leq 2 \cos(\theta) \}\\
    \int_{\Omega} \sqrt{ x^2 + y^2} \ dx \ dy = \int_{\Omega''} \sqrt{\rho^2 \cos^2(\theta) + \rho^2 \sin^2(\theta)} \rho \ d\rho \ d\theta\\
    \int_{\Omega''} \rho^2 \ d\rho \ d\theta \qquad \text{se } (x,y) \in \Omega \text{ allora } 0 \leq \rho \leq 2\\
    \int_{-\frac{\pi}{2}}^{\frac{\pi}{2}} d\theta \int_{0}^{2 \cos(\theta)} \rho^2 \ d\rho = \int_{-\frac{\pi}{2}}^{\frac{\pi}{2}} \frac{(2\cos(\theta))^3}{3} \ d\theta = 8 \int_{-\frac{\pi}{2}}^{\frac{\pi}{2}} \dots
    \end{gather*}
    \end{example}
    \begin{example}
        \begin{gather*}
            \int_{B(0,R)} e^{-(x^2+y^2)} \ dx \ dy\\
            = \int_{0}^{2\pi} d\theta \underset{\left[-\frac{e^{-\rho^2}}{2}\right]_o^R }{\int_{0}^{R} e^{-\rho^2} \ d\rho}\\
            -\int_{0}^{2\pi} \left( \frac{e^{-R^2}}{2}-1 \right) = 2\pi ( 1-\frac{e^{-R^2}}{2})
            -\int_{0}^{2\pi} \left( \frac{e^{-R^2}}{2}-1 \right) = 2\pi ( 1-\frac{e^{-R^2}}{2})
        \end{gather*}
    \end{example}
    \subsection{integrali doppi generalizzati}
    \begin{itemize}
        \item o il dominio è illimitato
        \item o la funzione è illimitata
    \end{itemize}
    \begin{example}
        \begin{gather*}
            \int_{\mathbb{R}^2} e^{-x^2+y^2} \ dx \ dy = \lim_{R \to +\infty} \int_{B(0,R)} e^{-x^2+y^2} \ dx \ dy = \lim_{R \to +\infty} \pi (1-e^{R^2}) = \pi 
        \end{gather*}
        Posso pensare $\mathbb{R}^2$ come:
        \begin{gather*}
            \int_{-\infty}^{+\infty} dx \int_{-\infty}^{+\infty} e^{-x^2+y^2} \ dy = \int_{-\infty}^{+\infty} dx \int_{-\infty}^{+\infty} e^{-x^2} e^{-y^2} \ dy \\
            \int_{-\infty}^{+\infty} e^{-x^2} dx \int_{-\infty}^{+\infty} e^{-y^2} \ dy = \pi 
        \end{gather*}
        Questo lo so che perchè so che il quadrato di questo integrale è $\pi$ quindi cos' posso conoscere l'integrale della gaussiana che è $\sqrt{\pi}$
    \end{example}
    \begin{example}
        \begin{gather*}
            \int_{0}^{1} \frac{1}{x^\alpha} \ dx \qquad \int_{1}^{+\infty} \frac{1}{x^\alpha} \ dx 
        \end{gather*}
        Quindi ik primo integrale è finito se $\alpha < 1$ metre il secondo è finito se $\alpha > 1$
        \begin{gather*}
            \int_{B(0,1)} \frac{1}{(\sqrt{x^2+y^2})^\alpha} \ dx \ dy \qquad \int_{R^2\backslash(B(0,1))} \frac{1}{(\sqrt{x^2+y^2})^\alpha }\ dx \ dy\\
            \underset{2\pi}{\int_{0}^{2\pi}} \int_{0}^{1} \frac{d\rho}{\rho^{\alpha-1}}
        \end{gather*}
        Quest'ultimo è finito se $\alpha < 2$ ed è infinito se $\alpha \geq 2$
    \end{example}



\section{25/11/25}
\subsection{integrali tripli}
\begin{center}
  \begin{tikzpicture}
    %assi x y e z
    \draw[->](0,0) --(5,0) node[anchor=north east]{$x$};
    \draw[->](0,0) --(0,5) node[anchor=north west]{$y$};
    \draw[->](0,0) --(-2,-2) node[anchor=south]{$z$};
    %proiezione su xz
    \draw[thin](0,-0.7) -- (-0.7,-1.4) -- (1.1,-1.4) -- (1.8,-0.7) -- cycle;
    \draw[dashed](0,-0.7) -- (-0.7,-0.7) node[at end, left]{$b$};
    \draw[dashed](-0.7,-1.4) -- (-1.4,-1.4) node[at end, left]{$a$};
    %parallelepipedo al centro
    \draw[ultra thick](-0.3, 0.3) -- (0.4,1) -- (1.8,1) -- (1.1,0.3) -- cycle;
    \draw[ultra thick](-0.3,0.3) -- (-0.3, 2.8);
    \draw[ultra thick](0.4,1) -- (0.4,3.5);
    \draw[ultra thick](1.8,1) -- (1.8,3.5);
    \draw[ultra thick](1.1,0.3) -- (1.1,2.8);
    \draw[ultra thick](-0.3,2.8) -- (0.4,3.5) -- (1.8,3.5) -- (1.1,2.8) -- cycle;
    %proiezione sul piano yz
    \draw[thin](-1.4,0.3) -- (-1.4,2.8) -- (-0.7,3.5) -- (-0.7,1) -- cycle;
    \draw[dashed](-0.7,1) -- (0, 1.7) node[at end, right]{$e$};
    \draw[dashed](-0.7,3.5) -- (0, 4.2) node[at end, right]{$f$};
    %proiezione sul piano zy
    \draw[thin](0.7,1.7) -- (2.1,1.7) -- (2.1, 4.2) -- (0.7, 4.2) -- cycle;
    \draw[dashed](0.7,1.7) -- (0.7, 0) node[at end, below]{$c$};
    \draw[dashed](2.1,1.7) -- (2.1, 0) node[at end, below]{$d$};
    %intersezione
    \draw[fill=red, opacity= 0.3](-0.3, 1.55) -- (0.4,2.25) -- (1.8,2.25) -- (1.1,1.55) -- cycle;
    \draw[->](4,2.05) ..controls(3,2.5).. (1,2.05) node[at start, right]{$\begin{matrix}
      \text{intersezione }\underbrace{P \cap \{z = z_0\}}_{P(z_0)}\\
      \{x = x_0 , y = y_0 , z = z_0\}
    \end{matrix}$};
  \end{tikzpicture}
\end{center}

\begin{proposition}[integrazione per fili]
  \begin{gather*}
    P=\{ (x,y,z): \quad x\in[a,b] \ , \ y \in[c,d] \ , \ z \in[e,f] \}\\
    = [a,b] \times [c,d] \times [e,f]\\
    \int_P f \ dx \ dy \ dz = \int_ {[a,b] \times [c,d]} \left( \int_e^f f(x,y,z) \ dz \right) dx \ dy\\
  \end{gather*}
\end{proposition}
\begin{proposition}
  Formula di integrazione "per fili " per un dominio più generale:\\
  \underbar{Supponiamo} che $\Omega$ si possa rappresentare in questa forma:
  \begin{gather*}
    \Omega = \left\{ (x,y,z) : (x,y) \in D \ , \ g_1(x,y) \leq z \leq g_2(x,y) \right\} 
  \end{gather*}
  dove $D$ è un dominio\\
  \underbar{Allora} se $f:\Omega \to \mathbb{R}$ è continua l'integrale si può calcolare mediante:
  \begin{gather*}
    \int_\Omega f(x,y,z) \ dx \ dy \ dz = \int_D \left( \int_{g_1(x,y)}^{g_2(x,y)} f(x,y,z) \ dz \right) dx \ dy
  \end{gather*}
\end{proposition}
\begin{example}
  Una sfera nello spazio il cui dominio è un cerchio di raggio 1 e centrato nell'origine.\\
  Il dominio è un cilindro per ottenere una sfera devo prendere come limitazioni superiori e inferiori le funzioni che mi danno due semisfere.\\
  \begin{center}
    \begin{tikzpicture}
      %assi xyz
        \draw[->](0,0) --(5,0) node[anchor=north east]{$x$};
        \draw[->](0,0) --(0,5) node[anchor=north west]{$y$};
        \draw[->](0,0) --(-2,-2) node[anchor=south]{$z$};
        %cerchio
        \draw(1.5,-1) ellipse (0.8 and 0.5);
        \node at(1.5,-1){$D$};
        %sfera con semisfere colorate diversamente
        \tdplotsetmaincoords{60}{120}
        \begin{axis}[
          hide axis,
          view={25}{35},
          width=5cm,
          height=5cm,
          zmin=-1.1,zmax=1.1,
          xmin=-0.5,xmax=2.5,
          ymin=-2.5,ymax=0.5,
          axis equal image,
        ]
        % semisfera superiore (y domain 0:90)
        \addplot3[
          surf,
          shader=flat,
          samples=50,
          samples y=20,
          domain=0:360,
          y domain=0:90,
          opacity=0.9,
          draw=black!30,
          colormap={upper}{color(0cm)=(red!50) color(1cm)=(red!30)}
        ]
        ({1 + sin(y)*cos(x)},{-1 + sin(y)*sin(x)},{cos(y)});
        % semisfera inferiore (y domain 90:180)
        \addplot3[
          surf,
          shader=flat,
          samples=50,
          samples y=20,
          domain=0:360,
          y domain=90:180,
          opacity=0.9,
          draw=black!30,
          colormap={lower}{color(0cm)=(blue!50) color(1cm)=(blue!30)}
        ]
        ({1 + sin(y)*cos(x)},{-1 + sin(y)*sin(x)},{cos(y)});
        \end{axis}
        \node at(1.5,1.5,0){$\Omega$};
        \node at(3,0.8,0){$z = g_1(x,y)$};
        \node at(3,2.5,0){$z = g_2(x,y)$};
    \end{tikzpicture}
  \end{center}
  \begin{gather*}
    \Omega = \text{sfera di centro (0,0,0) e raggio 1}\\
    D = \left\{ (x,y): x^2 + y^2 \leq 1 \right\} \\
    \Omega = \left\{ (x,y,z) : (x,y) \in D \ , \ -\sqrt{1-x^2-y^2} \leq z \leq \sqrt{1-x^2-y^2} \right\} 
  \end{gather*}
  Ora si passa a fare l'integrale:
  \begin{gather*}
    \int_\Omega x^2 z \ dx \ dy \ dz = \int_D \left( \int_{-\sqrt{1-x^2-y^2}}^{\sqrt{1-x^2-y^2}} x^2 z \ dz \right) dx \ dy\\
    = \int x^2 \left( 1-x^2-y^2 \right) \ dx \ dy = \int_0^{2\pi} d\theta \int_0^1 \rho (\rho\cos\theta)^2 (1-\rho^2) \ d\rho = \dots = \frac{\pi}{24}\\
  \end{gather*}
\end{example}
\begin{example}
  Prendiamo ora il paraboloide $z= x^2+y^2$\\
  E prendiamo $\Omega = \text{la regione compresa tra il paraboloide } z= x^2 + y^2$ e il piano $z=3-2y$
\end{example}

  \begin{wrapfigure}{r}{6cm}
    \begin{center}
      \begin{tikzpicture}
        \begin{axis}[
          axis lines = center,
          xlabel = $x$,
          ylabel = $y$,
          zlabel = $z$,
          domain=-1.2:1.2,
          y domain=-1.2:1.2,
          samples=25,
          samples y=25,
          colormap/cool,
          zmin=-0.5, zmax=4,
          view={10}{30},
          ]
          % paraboloide z = x^2 + y^2
          \addplot3 [
            surf,
            shader=flat,
            opacity=0.8,
          ]
          {x^2 + y^2};
          % piano z = 3 - 2y
          \addplot3[
            surf,
            shader=flat,
            opacity=0.6,
            draw=gray,
            colormap/jet,
          ]
          {3 - 2*y};
        \end{axis}
      \end{tikzpicture}
    \end{center}
  \end{wrapfigure}
\begin{gather*}
  \left\lvert \Omega \right\rvert = \int_\Omega 1 \ dx \ dy \ dz\\
  \Omega = \left\{ (x,y,z) : (x,y) \in D \ , \quad  x^2 + y^2\leq z \leq 3-2y \right\} \\
\end{gather*}
Vista da davanti viene:
\begin{center}
  \begin{tikzpicture}[scale=0.5]
    %assi x y
    \draw[->](0,0) --(5,0) node[anchor=north east]{$x$};
    \draw[->](0,0) --(0,5) node[anchor=north west]{$y$};
    %parabola
    \draw[domain=-2:2, smooth, variable=\x, blue] plot ({\x}, {(\x*\x)}) node[at end, above]{$y= x^2$};
    %piano
    \draw[domain=-1:1, smooth, variable=\x, red] plot ({\x}, {3-2*\x}) node[at end, below]{$y= 3-2x$};
    \filldraw[fill = black](0,3) circle (2pt) node[left]{$3$};
  \end{tikzpicture}
\end{center}

Per determinare $D$ cerco gli $(x,y)$ tali che la $z$ deò piano e la $z$ del paraboloide sono uguali:
\begin{gather*}
  \begin{cases}
    z=3-2y\\
    z = x^2 + y^2
  \end{cases}
  x^2+y^2 = 3-2y\\
  x^2 + y^2 + 2y = 3\\
  x^2 + (y+1)^2 = 4
\end{gather*}
Passando ora a calcolare l'integrale:
\begin{gather*}
  \int_D \ dx \ dy \int_{x^2+y^2}^{3-2y} 1 \ dz = \\
  \rightarrow \text{passaggio in polari}\begin{cases}
    x = \rho \cos(\theta)\\
    y +1  = \rho \sin(\theta)\\
  \end{cases}\\
  \int_D \ dx \ dy \ 4-x^2-(y+1)^2 = \int_0^{2\pi} d\theta \int_0^2 \rho \left( 4 - \rho^2 \right) \ d\rho = 8\pi \\
  D \to \{(\rho , \theta) \quad 0 \leq 2 \ , \ \theta \in [0,2\pi] \}
\end{gather*}
%fine esempio
\subsection{curve nello spazio}
Prendiamo una curva di questo tipo:
\begin{center}
  \begin{tikzpicture}
    %assi x y e z
    \draw[->](0,0) --(5,0) node[anchor=north east]{$x$};
    \draw[->](0,0) --(0,5) node[anchor=north west]{$y$};
    \draw[->](0,0) --(-2,-2) node[anchor=south]{$z$};
    %curva
    \draw(1,0.5) .. controls (2,2) and (0,3) .. (0.5,4);
    \draw(-1,0.5) .. controls (-2,2) and (0,3) .. (-0.5,4);
    %sezioni ellittiche
    \draw(0,0.6) ellipse (1.05 and 0.3);
    \draw(0,2) ellipse (1.15 and 0.3);
    \draw(0,4) ellipse (0.5 and 0.2);
    %piano intermedio
    \draw[fill=red, opacity=0.3](-2.5,2.5) -- (1.5,2.5) -- (2,2.8) -- (-2,2.8) -- cycle;
  \end{tikzpicture}
\end{center}
Qui l'integrale per fili è poco pratico se non addirittura impossibile.\\
Si usa quindi il metodo della cosiddetta integrazione "per fette" o "per strati".\\
\begin{gather*}
  \Omega = \left\{ (x,y,z) : \quad h_1 \leq z \leq h_2 \quad (x,y)\in\Omega (z) \right\} 
\end{gather*}
Dove $\forall z \in[h_1, h_2] , \Omega(z)$ è un dominio regolare del piano.\\
\begin{proposition}[integrazione per strati]
  \begin{gather*}
    \int_\Omega f(x,y,z) \ dx \ dy \ dz = \int_{h_1}^{h_2} \left( \int_{\Omega(z)} f(x,y,z) \ dx \ dy \right) dz
  \end{gather*}
\end{proposition}
\begin{example}
  volume di un solido di rotazione:
  \begin{center}
    \begin{tikzpicture}
      %assi x,y,z
      \draw[->](0,0) --(5,0) node[anchor=north east]{$x$};
      \draw[->](0,0) --(0,5) node[anchor=north west]{$y$};
    \draw[->](0,0) --(-2,-2) node[anchor=south]{$z$};
      %curva
      \draw(1,0.5) .. controls (1.5,2) and(0.2,2.8).. (0,2.8) node[midway, right]{$y=h(z)$};
      \draw(-1,0.5) .. controls (-1.5,2) and(-0.2,2.8).. (0,2.8);
      %sezioni ellittiche
      \draw(0,0.6) ellipse (1.05 and 0.3) node[right]{$h_1$};
      \draw(0,2) ellipse (0.9 and 0.3) node[right]{$h_2$};
    \end{tikzpicture}
  \end{center}
  Ruotando la curva attorno all'asse $y$ ottengo un solido di rotazione.\\
  \begin{gather*}
    \Omega (z) = \text{cerchio } x^2+y^2 \leq h^2(z)
  \end{gather*}
  Per calcolare il volume:
  \begin{gather*}
    \text{Volume}(\Omega) = \left\lvert \Omega \right\rvert = \int_\Omega 1 \ dx \ dy \ dz \\
    = \int_{h_1}^{h_2} \left( \underbrace{\int_{x^2+y^2 \leq h^2(z)}}_{\text{area del cercio }=\pi h^2(z)} \right) \ dz = \pi \int_{h_1}^{h_2} h^2(z) \ dz  \\
  \end{gather*}
  Quindi sfera di raggio $R$ e centrata nell'origine:
  \begin{gather*}
    \text{volume sfera} = \pi \int_{-R}^{R} (R^2 - z^2) \ dz = \pi\left[ R^2 z - \frac{z^3}{3} \right]_{-R}^{R}  = \frac{4}{3} \pi R^3
  \end{gather*}
\end{example} 




\section{26/11/25}
\subsection{cambiamenti di variabili per integrali tripli}
La situazione è analoga al caso del piano e agli integrali doppi.
\begin{center}
  \begin{tikzpicture}
    % due grafici affiancati usando scope con traslazione orizzontale
    \begin{scope}[xshift=0cm]
      % primo sistema (x,y,z)
      \draw[->] (0,0,0) -- (5,0,0) node[right] {$x$};
      \draw[->] (0,0,0) -- (0,5,0) node[above] {$z$};
      \draw[->] (0,0,0) -- (0,0,5) node[above] {$y$};
      \potato{(2,2,2)}{0.5};
      \filldraw (2.5,3,2) circle (1.5pt) node[anchor=south east] {$P$};
      \node at (2,4,2) {$D$};
    \end{scope}
    \draw[->](7,2.5,2) -- (6,2.5,2) node[above] {$T$};
    \begin{scope}[xshift=7.5cm]
      % secondo sistema (u,v,w)
      \draw[->] (0,0,0) -- (5,0,0) node[right] {$v$};
      \draw[->] (0,0,0) -- (0,5,0) node[above] {$w$};
      \draw[->] (0,0,0) -- (0,0,5) node[above] {$u$};
      \potato{(2,2,2)}{0.5};
      \filldraw (2.5,3,2) circle (1.5pt) node[anchor=south east] {$P$};
      \node at (2,4,2) {$D'$};
    \end{scope}
  \end{tikzpicture}
\end{center}
\begin{gather*}
  x = a(u,v,w) \\
  y = b(u,v,w) \\
  z = c(u,v,w) \\
\end{gather*}
\begin{itemize}
  \item Si suppone che $a,c,b$ siano $C^1$
  \item La trasformazione da $D \to D'$ è biunivoca
  \item La matrice Jacobiana della trasformazione abbia $det \ \neq 0$ in ogni punto di $D'$ 
\end{itemize}
\underbar{Se} $f$ è continua in $D$ \underbar{allora}:
\begin{gather*}
  \int_D f(x,y,z) \ dxdydz = \int_{D'} f(a(u,v,w),b(u,v,w),c(u,v,w)) \cdot \left| det(Jac) \right| \ dudvdw \\
\end{gather*}
\newpage
\begin{definition}[Coordinate cilindriche]
  \begin{gather*}
    P = (x,y,z) \Leftrightarrow (\rho,\theta,z) \\
    \begin{cases}
      z = z\\
      x = \rho\cos(\theta) \\
      y = \rho\sin(\theta) \\
    \end{cases}
    \left| det \mathcal{J} \right| = \rho 
  \end{gather*}
\end{definition}

    \begin{center}
\tdplotsetmaincoords{70}{110}
    \begin{tikzpicture}[tdplot_main_coords, scale=1.5]
        \draw[->] (0,0) -- (3,0) node[right] {$x$};
        \draw[->] (0,0) -- (0,3) node[above] {$y$};
        \draw[->] (0,0) -- (0,0,3) node[above] {$z$};
        \draw[dashed] (0,0) circle (2);
        \draw[thick, cyan] (0,0) -- (30:2) node[midway, above right] {$\rho$};
        \draw[thick, cyan] (30:2) -- ({2*cos(30)},{2*sin(30)},2) node[midway, right] {$z$};
        \draw[thick, cyan] (0,0,2) -- ({2*cos(30)},{2*sin(30)},2);
        \draw[->] (0.2,0) arc (0:30:0.5) node[midway, right] {$\theta$};
        \filldraw[fill=black] ({2*cos(30)},{2*sin(30)},2) circle (1.5pt) node[anchor=south east] {$P'$};
        \filldraw[fill=black] ({2*cos(30)},{2*sin(30)},0) circle (1.5pt) node[anchor=south east] {$P$};
      \end{tikzpicture}
    \end{center}
\hfil\\
\begin{example}
  \begin{gather*}
    \int_D (x^2+y^2) \ dx \ dy \ dz \\
  \end{gather*}
  D è l'insieme del limitato da $z=0, z=1$ dalle superfici cilindriche:
  \begin{gather*}
    x^2+y^2 = 1 \quad \text{e} \quad x^2+y^2 = 4
  \end{gather*}
  Dai piani $x=0$ e $y=0$\\
  \textbf{Come lo scrivo in coordinate cilindriche?}\\
  $z$ si può scrivere come $1 \leq \rho \leq 4$ per le altre coordinate:
  \begin{gather*}
    0 \leq  x \leq y\\
    0 \leq \rho\cos(\theta) \leq \rho\sin(\theta) \\
  \end{gather*}
  Quindi il dominio in coordinate cilindriche diventa:
  \begin{gather*}
    D' = \left\{ (\rho, \theta ,z) : z \in[0,1] , \quad \rho \in [1,2] , \quad \theta \in \left[\frac{\pi}{4}, \frac{\pi}{2}\right] \right\} \\
  \end{gather*}
  Quindi procedendo con l'integrale:
  \begin{gather*}
    \int_D x^2+y^2 \ dx \ dy \ dz = \int_{D'} \underbrace{(\rho\cos(\theta))^2 + (\rho\sin(\theta))^2}_{f \text{ è espressa in termini di } \rho , \theta , z} \cdot \underbrace{\rho}_{det \mathcal{J}} \ d\rho \ d\theta \ dz \\
    = \int_{D'} \rho^3 \ d\rho \ d\theta \ dz = \int_{\rho \in[1,2] , \theta \in [\frac{\pi}{4}, \frac{\pi}{2}]} \left(\int_0^1 \rho^3 dz\right)  \ d\rho \ d\theta \\
    = \int_1^2 \int_{\frac{\pi}{4}}^{\frac{\pi}{2}} \rho^3 \ d\theta \ d\rho = \int_1^2 \left( \rho^3 \cdot \left( \frac{\pi}{2} - \frac{\pi}{4} \right) \right) d\rho = \frac{\pi}{4} \int_1^2 \rho^3 \ d\rho = \frac{\pi}{4} \cdot \left[ \frac{\rho^4}{4} \right]_1^2 = \frac{15\pi}{16} \\
  \end{gather*}
\end{example}
\begin{example}
  Sia $D$ l'insieme di punti sopra il paraboloide $z=x^2 + y^2$ e dentro la sfera i centro $(0,0,0)$ e raggio $\sqrt{6}$. In termini di disuguaglianze.
  \begin{gather*}
    D = \left\{ (x,y,z): z \geq x^2+y^2, \quad x^2+y^2+z^2 \leq 6 \right\} 
  \end{gather*}
  Calcolare volume $D = \int_D 1 \ dx \ dy \ dz$\\
  Esprimo $D$ in coordinate cilindriche:
  \begin{gather*}
    D' = \left\{ (\rho, \theta , z) : z \geq \rho^2\cos^2(\theta) + \rho^2 \sin^2(\theta) , \quad \rho^2\cos^2(\theta) + \rho^2 \sin^2(\theta) + z^2 \leq 6 \right\} \\
    D' = \left\{ (\rho, \theta , z) : z \geq \rho^2 , \quad \rho^2 + z^2 \leq 6 \right\} \\
    D' = \left\{ (\rho, \theta , z) : (\rho, z) \in E, 0 \leq \theta \leq 2\pi \right\} \\
  \end{gather*}
  Passando all'integrale:
  \begin{gather*}
    \int_D 1 \ dx \ dy \ dz = \int_{D'} 1 \cdot \rho \ d\rho \ d\theta \ dz = \int_E \left( \int_0^{2\pi} \rho \ d\theta \right) d\rho \ dz \\
    = 2\pi \int_E \rho \ d\rho \ dz\\
    E= \left\{ (\rho,z) : \rho \in [0,\sqrt{2}] , \quad \rho^2 \leq z \leq \sqrt{6-\rho^2} \right\} \\
    = \int_0^{\sqrt{2}} d\rho \ \int_{\rho^2}^{\sqrt{6-\rho^2}} \rho \ dz = 2\pi \int_0^{\sqrt{2}} \rho \left( \sqrt{6-\rho^2} - \rho^2 \right) d\rho \\
    = 2\pi \left( \int_0^{\sqrt{2}} \rho \sqrt{6-\rho^2} \ d\rho - \int_0^{\sqrt{2}} \rho^3 \ d\rho \right) \\
    = 2\pi \left( \left[ -\frac{1}{3} (6-\rho^2)^{\frac{3}{2}} \right]_0^{\sqrt{2}} - \left[ \frac{\rho^4}{4} \right]_0^{\sqrt{2}} \right) = 2\pi \left( -\frac{1}{3} (6-2)^{\frac{3}{2}} + \frac{1}{3} 6^{\frac{3}{2}} - \frac{(\sqrt{2})^4}{4} \right) \\
    = 2\pi \left( -\frac{8}{3} + \frac{6\sqrt{6}}{3} - 1 \right) = \frac{2\pi}{3} (6\sqrt{6} - 11) \\
  \end{gather*}
  \end{example}
\subsection{coordinate sferiche}
A differenza delle coordinate cilindriche, si utilizzano due angoli invece di uno e due due coordinate per la distanza.
\begin{center}
  \begin{tikzpicture}
    \draw[->] (0,0,0) -- (5,0,0) node[right] {$x$};
    \draw[->] (0,0,0) -- (0,5,0) node[above] {$z$};
    \draw[->] (0,0,0) -- (0,0,5) node[above] {$y$};
    \draw[red, ->](0,0,0) -- (2,2,1) node[midway, above] {$\rho$};
    \draw(0,0,0) -- (2,0,1);
    \draw(2,0,1) -- (2,2,1);
    \draw(0,2,0) -- (2,2,1) ;
    \filldraw[fill=black](2,2,1) circle (2pt) node[right] {$P$};
    \draw[dashed](2,0,0) -- (2,0,1);
    \draw[dashed](2,0,1) -- (0,0,1);
    \draw[->, cyan](0.5,0,0.3) arc (0:45:0.5) node[midway, above right] {$\varphi$};
    \draw[->, cyan](0,0,0.3) arc (-130:-60:0.7) node[midway, below] {$\theta$};
    \filldraw[fill=black](2,0,1) circle (2pt) node[below] {$P'$};
  \end{tikzpicture}
\end{center}
Dove $\rho$ è la distanza di $P$ da $(0,0,0)$, e $P$ è definito da $\rho, \theta , \varphi$
\begin{gather*}
  \rho \geq 0\\
  \varphi \in [0,\pi] \\
  \theta \in [0,2\pi) \\
  \begin{cases}
    z = \rho\cos(\varphi) \\
    x = \rho\sin(\varphi)\cos(\theta) \\
    y = \rho\sin(\varphi)\sin(\theta) \\
  \end{cases}
  \left\lvert det \mathcal{J} \right\rvert = \rho^2 \sin(\varphi)
\end{gather*}
Prendiamo ora $\rho = cost$ e $\varphi = cost$ e ho quindi due sfere centrate in $(0,0,0)$ e ho quindi coni circolari con asse di simmetria l'asse $z$.
\begin{center}
  \hfill\\\hfill\\\hfill\\\hfill\\\hfill\\
  \begin{tikzpicture}
    \tdplotsetmaincoords{70}{110}
    \begin{tikzpicture}[tdplot_main_coords, scale=2]
      % Axes
      \draw[->] (0,0,0) -- (3,0,0) node[right] {$x$};
      \draw[->] (0,0,0) -- (0,3,0) node[above] {$y$};

      % Parameters
      \pgfmathsetmacro{\Rsphere}{1.5}
      \pgfmathsetmacro{\zbase}{1} % height where cone meets the sphere (base of the displayed cone)
      \pgfmathsetmacro{\rbase}{sqrt(\Rsphere*\Rsphere - \zbase*\zbase)} % base radius at z = \zbase

      % Sphere (shaded)
      \shade[ball color=blue!20, opacity=0.5] (0,0,0) circle (\Rsphere);
      \draw[dashed] (0,0,0) circle (\Rsphere);

      % Cone surface: approximate by triangular fan from apex to base circle
      \foreach \a in {0,10,...,350} {
        \pgfmathsetmacro{\an}{\a+10}
        \fill[red!25,opacity=0.7,draw=red!70,very thin]
          (0,0,0) -- ({\rbase*cos(\a)},{\rbase*sin(\a)},{\zbase}) -- ({\rbase*cos(\an)},{\rbase*sin(\an)},{\zbase}) -- cycle;
      }

      % Base circle / intersection curve
      \draw[thick,green!60!black] plot [domain=0:360,samples=200]
        ({\rbase*cos(\x)},{\rbase*sin(\x)},{\zbase});

      % A couple of cone edges for clarity
      \draw[thick,cyan!80] (0,0,0) -- ({\rbase},0,{\zbase});
      \draw[thick,cyan!80] (0,0,0) -- ({-0.6*\rbase},{0.8*\rbase},{\zbase});
      \draw[->] (0,0,0) -- (0,0,2) node[above] {$z$};
    \end{tikzpicture}
  \end{tikzpicture}
\end{center}
  Se invece tengo $\theta$ costante ho dei semipiani contenenti l'asse $z$.
  \begin{example}
    Sia $H$ la sfera $\left\{ x^2+y^2+z^2 \leq R^2, \ z \geq 0 \right\} $\\
    Supponiamo che la diensita di massa $d(x,y,z) = (2R-\rho)$\\
    Dove $\rho$ è la distanza da $(0,0,0)$\\
    Calcolare la massa di $H$:
    \begin{gather*}
      M = \int_H d(x,y,z) \ dx \ dy \ dz \\
      H' = \begin{cases}
        \rho
      \end{cases}
    \end{gather*}
    $\Omega$ sottoinsieme di $\mathbb{R}^3$\\
    Volume di $(\Omega) = \left\lvert \Omega \right\rvert = \int_\Omega 1 \ dx \ dy \ dz $\\
    massa totale di un corpo che occupa la regione $\Omega$ e che ha densità di massa $d(x,y,z)$ è:
    \begin{gather*}
      M = \int_\Omega d(x,y,z) \ dx \ dy \ dz
    \end{gather*}
    Devo calcolare il baricentro:
    \begin{gather*}
      \text{coordinata x:} \ = \int_\Omega x d(x,y,z) \ dx \ dy \ dz\\
      \text{coordinata y:} \ = \int_\Omega y d(x,y,z) \ dx \ dy \ dz\\
      \text{coordinata z:} \ = \int_\Omega z d(x,y,z) \ dx \ dy \ dz\\
    \end{gather*}
    Diviso per la massa totale.\\
    \hfil\\
    Calcoliamo ora il momento di inerzia facendo:
    \begin{gather*}
      \int_\Omega \delta^2(x,y,z) d(x,y,z) \ dx \ dy \ dz \\
    \end{gather*}
    Dove $\delta(x,y,z)$ è la distanza di un punto $(x,y,z)$ dall'asse di rotazione, ovvero l'asse rispetto a cui si calcola im momento di inerzia.\\
    Ad esempio momento di inrezia rispetto all'asse $z$:
    \begin{gather*}
      \int_\Omega (x^2+y^2) d(x,y,z) \ dx \ dy \ dz \\
    \end{gather*}
    Massa di $H$:
    \begin{gather*}
      M = \int_H (2R-\rho) \ dx \ dy \ dz = \int_{H'} (2R-\rho) \cdot \rho^2 \sin(\varphi) \ d\rho \ d\varphi \ d\theta \\
      = \int_0^{2\pi} d\theta \ \int_0^{\frac{\pi}{2}} \sin(\varphi) \ d\varphi \ \int_0^R (2R-\rho) \rho^2 \ d\rho \\
      = 2\pi \cdot 1 \cdot \left[ 2R \frac{\rho^3}{3} - \frac{\rho^4}{4} \right]_0^R = 2\pi \left( \frac{2R^4}{3} - \frac{R^4}{4} \right) = \frac{5\pi R^4}{6} \\
    \end{gather*}
  \end{example}

  \newpage
  \section{27/11/25}
  
In queste lezioni ci concentreremo sulle applicazioni dei teoremi: th. della divergenda e th. di Stokes. Prima nel piano e poi nello spazio.\\
Per il caso de piano vedremo che saranno casi particolari delle formule di Gauss-Green.\\
In seguito si vedranno le successioni di funzioni e serie di funzioni.\\
Se abbiamo $f:[a,b] \to \mathbb{R}, \ C^1$\\
Se si valuta l'integrale:
\begin{gather*}
\int_a^b \frac{df}{dx}(x) dx = f(b) - f(a)
\end{gather*}
Questo è il teorema fondamentale del calcolo integrale.\\
Consideriamo ora un insieme in $\mathbb{R}$:
\begin{center}
  \begin{tikzpicture}
    \draw(0,0) -- (2,0);
    \draw[ultra thick](0.5,0) -- (1.5,0) node[at end, above]{$b$};
    \node at(0.5,0)[above]{$a$};
  \end{tikzpicture}
\end{center}

E questa formula appunto lega l'integrale di questo intervallo a "qualcosa" che dipende \underbar{solamente} dai valori degli estremi degli intervalli.\\
Se abbiamo una certa forma differenziale $\omega$ esatta, e $\omega = \frac{\partial f}{\partial x} (x,y) dx + \frac{\partial f}{\partial y} (x,y) dy$ è una curva orientata allora:
\begin{gather*}
  \int_\gamma \frac{\partial f}{\partial x} (x,y) dx  + \frac{\partial f}{\partial y} (x,y) dy = f(P_2) - f(P_1)
\end{gather*}
Dove $P_2$ è l'estremo finale della curva e $P_1$ l'iniziale.\\
\begin{center}
  \begin{tikzpicture}
    \draw(0,0) .. controls (1,2) and (3,0) .. (4,2);
    \node at(0,0)[below]{$P_1$};
    \node at(4,2)[below , right]{$P_2$};
  \end{tikzpicture}
\end{center}


\underbar{Se} $D \subseteq \mathbb{R}^2$ che sodisfa certe $h_1$ e $\partial D$ è la curva frontiera orientata in un certo modo\\
\begin{gather*}
  \int_D \begin{pmatrix}
    \text{Espress. che}\\
    \text{coinvolge } f(x,y)
  \end{pmatrix}
  \ dx \ dy = \int_{\partial D} f
\end{gather*}
\begin{wrapfigure}{l}{5cm}
  \begin{tikzpicture}
    \potato{(0,0)}{1};
    \node at(0.7,1.5){$D$};
  \end{tikzpicture}
\end{wrapfigure}
\hfil\\
$D$ insieme limitato di $\mathbb{R}^2$ supponiamo che $\partial D$ sia una curva semplice chiusa regolare a tratti.\\
\hfill\\
Chiameremo orientazione positiva di $\partial D$ quella antioraria.\\
Si noti che percorrendo $\partial D$ in nel verso positivo l'insieme $D$ mi rimane sempre sulla sinistra.\\
\hfill\\\hfil\\\hfil\\
\begin{lemma}
  \underbar{Sia} $F$ un campo vettoriale piano, di componenti:
  \begin{gather*}
    F(x,y) = \left( P(x,y) \ , \ Q(x,y) \right)
  \end{gather*}
  \underbar{Supponiamo} che $F \in C^1 (\overline{D})$ 
  \underbar{Sia} $D \subseteq \mathbb{R}^2$\\ 
  \underbar{Se} $D$ è y-semplice e $\partial D$ è una curva, chiusa, regolare a tratti \underbar{allora}:
  \begin{gather*}
    \int_D \frac{\partial P}{\partial y}(x,y) \ dx \ dy = -\int_{\partial^+ D} P(x,y) \ dx \qquad (1)
  \end{gather*}
  \underbar{se} $D$ è x-semplice e $\partial D$ è una curva, chiusa, regolare a tratti \underbar{allora}:
  \begin{gather*}
    \int_D \frac{\partial Q}{\partial x}(x,y) \ dx \ dy = \int_{\partial^+ D} Q(x,y) \ dy \qquad (2)
  \end{gather*}
\end{lemma}
\begin{proof}
  punto $(1)$\\
  \begin{gather*}
    D=\left\{ (x,y): x \in[a,b] \ \varphi_1(x) \leq y \leq \varphi_2(x) \right\} 
  \end{gather*}
  Con $\varphi_1, \varphi_2 \quad C^1$ a tratti.\\
  \begin{center}
    \begin{tikzpicture}
      %assi xy
      \draw[->] (-1,0) -- (5,0) node[right] {$x$};
      \draw[->] (0,-1) -- (0,5) node[above] {$y$};
      %funzione sopra
      \draw(1,1.5) .. controls (2,3) and (3,4) .. (4,3.5) node[right] {$\varphi_2(x)$} node[midway, above]{$\boxed{3}^+$};
      %funzione sotto
      \draw(1,0.5) .. controls (2,1) and (3,0.5) .. (4,1) node[right] {$\varphi_1(x)$} node[midway, below]{$\boxed{1}^+$};
      %linee verticali
      \draw[dashed](1,0) -- (1,0.5) node[at start, below]{$a$};
      \draw[dashed](4,0) -- (4,1) node[at start, below]{$b$};
      %lati
      \draw[thick](1,1.5) -- (1,0.5) node[midway, left]{$\boxed{4}^+$};
      \draw[thick](4,3.5) -- (4,1) node[midway, right]{$\boxed{2}^+$};
    \end{tikzpicture}
    Fig. di riferimento
  \end{center}
\begin{gather*}
  \int_D \frac{\partial P(x,y)}{\partial y} \ dx \ dy = \int_a^b dx \int_{\varphi_1(x)}^{\varphi_2(x)} \frac{\partial P(x,y)}{\partial y} \ dy =\\
  = \int_a^b \left[ P(x,y) \right]_{y=\varphi_1(x)}^{y=\varphi_2(x)} \ dx = \int_a^b \left[ P(x,\varphi_2(x)) - P(x,\varphi_1(x)) \right] \ dx
\end{gather*}
\fbox{Richiamo cosa è esattamente $\int_\gamma g(x,y) \ dx $ dove $\gamma:[c,d] \to \mathbb{R}^2$\\
è una curva orientata $\int_c^d g\left( \gamma_1(t) , \gamma_2(t) \right) \dot{\gamma}_1(t) \ dt $}\\
Ora considero i vari integrali di linea sui 4 tratti della curva (segnati nel disegno):
\begin{gather*}
  \int_{\partial^+ P} P \ dx = \int_{\boxed{1}^+} P \ dx + \int_{\boxed{2}^+} P \ dx + \int_{\boxed{3}^+} P \ dx + \int_{\boxed{4}^+} P \ dx\\
  \int_{\boxed{2}^+} P \ dx = \int_{\boxed{4}^+} P \ dx = 0
\end{gather*}
Perchè in quei tratti la variazione di $x$ è nulla:\\
\begin{gather*}
  t \to (b,t) \qquad t \in [\varphi_1(b), \varphi_2(b)] \\
  \int_{\varphi_1(t)}^{\varphi_2(t)} P(b,t) \cdot 0 \ dt = 0
\end{gather*}
%\underbar{Se} $\omega = a(x,y) \ dx + \cancel{b (x,y) \ dy}$\\
%\begin{gather*}
%  \int_\gamma \omega = \int_c^d a(\gamma_1(t)) \dot{\gamma}_2(t) \ \dot{\gamma}_1(t) + \cancel{b} dt
%\end{gather*}
Calcoliamo ora gli integrali sui tratti rimanenti:
\begin{gather*}
  \int_{\boxed{1}^+} P(x,y) \ dx \\
  \text{tenendo a mente: } t \to (t, \varphi_1(t)) \quad t \in [a,b]\\
  = \int_a^b P(t,\varphi_1(t)) \cdot 1 \ dt = \int_a^b P(t,\varphi_1(t)) \ dt \\
  \int_{\boxed{3}^+} P(x,y) \ dx \\
  \text{tenendo a mente: } t \to (t, \varphi_2(t)) \quad t \in [a,b]\\  
  = -\int_a^b P(t,\varphi_2(t)) \ dt
\end{gather*}
Quindi abbiamo:
\begin{gather*}
  \int_a^b P\left( x, \varphi_2(x) \right) \ dx - \int_a^b P\left( x, \varphi_1(x) \right) \ dx\\
  -\int_{\boxed{3}^+} P \ dx = \int_{\boxed{1}^+} P \ dx
\end{gather*}
\end{proof}

\begin{lemma}
  \underbar{Sia} $F$ un campo vettoriale piano, di componenti:
  \begin{gather*}
    F(x,y) = \left( P(x,y) \ , \ Q(x,y) \right)
  \end{gather*}
  \underbar{Supponiamo} che $F \in C^1 (\overline{D})$ 
  \underbar{Sia} $D \subseteq \mathbb{R}^2$\\ 
  \underbar{Se} $\partial D$ Sia una curva semlice , chiusa, regolare a tratti.\\
  \underbar{Se} $D$ è sia x-semplice che y-semplice \underbar{allora}:
  \begin{gather*}
    \int_D \left( \frac{\partial Q}{\partial x} - \frac{\partial P}{\partial y} \right) \ dx \ dy = \int_{\partial^+ D} P \ dx + Q \ dy
  \end{gather*}  
  Quest'ultima è la stessa cosa che dire $\int_{\partial^+ D} \left\langle F,T \right\rangle \ ds $
\end{lemma}
\hfill\\
Diciamo che un insieme $D\subseteq \mathbb{R}^2$ è s-decomponibile\\
\underbar{se} esso è decomponibile in un numero finito di sottodomini $D_1, D_2, \dots , D_k$ semplici rispetto ad entrambi gli assi.
\hfill\\\hfill\\\hfill\\\hfill\\\hfill\\\hfill\\\hfill\\
\begin{example}
  La corona circolare non è ne x-semplice ne y-semplice ma se si divide in quattro parti ogni parte è sia x-semplice sia y-semplice:
  \begin{center}
    \begin{tikzpicture}[scale = 0.5]
      \draw(0,0) circle (1);
      \draw(0,0) circle (2);
      \draw[red](0,-2) -- (0,2);
      \draw[red](-2,0) -- (2,0);
      \fill[page](0,0) circle (0.9);
    \end{tikzpicture}
  \end{center}
\end{example}
\hfill\\
\underbar{Se} $D$ è s-decomponibile come orientazione positiva su $\partial D$ si intende quella data alle singole componenti di $\partial D$ in modo che percorrendole si lasci il dominio sempre alla propria sinistra.\\
\begin{theorem}
  \underbar{Se} $D$ è s-decomponibile \underbar{e} $F \in C^1 (\overline{D})$\\
  \underbar{allora} vale la formula:
  \begin{gather*}
    \int_D \left( \frac{\partial Q}{\partial x} - \frac{\partial P}{\partial y} \right) \ dx \ dy = \int_{\partial^+ D} P \ dx + Q \ dy
  \end{gather*}  
  (già vista nel lemma prec.)
\end{theorem}
\begin{proof}
\hfill\\
  \begin{center}
    \begin{tikzpicture}
      \draw[red, thick](0,0) ellipse (2 and 1.3);
      \draw[red, thick](-0.4,0.1) ellipse (0.5 and 0.7);
      \draw[red, thick](0.9,-0.5) ellipse (0.35 and 0.2);
      \draw[red, thick](-0.25, 1.3) -- (-0.35, 0.8);
      \draw[red, thick](-0.45, -0.6) -- (-0.55, -1.25);
      \draw[red, thick](1.25, -0.45) -- (1.9, -0.35);
      \draw[red, thick](0.05, -0.24) -- (0.56, -0.46);
      \draw[red, thick](-0.9,-0.05) -- (-2,0.15);
      \draw[red, thick](0.9,-0.3) -- (1.3, 1);
      \draw[red, thick](0.9, -0.7) -- (0.9, -1.15);
      %\draw[->](-0.2, 1.3) -- (-0.3, 0.8);
      %\draw[->](-0.4, 0.8) -- (-0.3, 1.3);
      %\draw[->](0.1, -0.2) -- (0.55, -0.4);
      %\draw[->](0.5, -0.5) -- (0.05, -0.3);
      %\draw[->](1.3, -0.5) -- (1.9, -0.4);
      %\draw[->](1.9, -0.3) -- (1.2, -0.4);
      %\draw[->](-0.9, 0) -- (-2, 0.2);
      %\draw[->](-2, 0.1) -- (-0.9, -0.1);
      %\draw[->](-0.5,-0.6) -- (-0.6, -1.2);
      %\draw[->](-0.5,-1.2) -- (-0.4, -0.6);
      \node[scale=2] at (0.5,0.3) {$\circlearrowleft $};
      \node[scale=1.5] at (1.5,0.2) {$\circlearrowleft $};
      \node[scale=1] at (1.4,-0.7) {$\circlearrowleft $};
      \node[scale=1.5] at (0.3,-0.8) {$\circlearrowleft $};
      \node[scale=1.5] at (-1.2, 0.6) {$\circlearrowleft $};
      \node[scale=1.5] at (-1.2,-0.6) {$\circlearrowleft $};
      \node[scale=1.5] at (-0.4,0.1) {$\circlearrowright $};
      \node[scale=0.8] at (0.9,-0.5) {$\circlearrowright $};
      \node [red, rotate = -26] at (1,1.1){$<$};
      \node [red, rotate = 22] at (-1,1.1){$<$};
      \node [red, rotate = 26] at (1,-1.1){$>$};
      \node [red, rotate = -22] at (-1,-1.1){$>$};
      \node [red, rotate = 50] at (-0.1,-0.45){$<$};
      \node [red, rotate = -110] at (-0.85,0.4){$<$};
      \node [scale = 0.8, red, rotate = 20] at (0.75,-0.32){$<$};
      \node [scale = 0.8, red, rotate = -130] at (1.19,-0.6){$<$};
    \end{tikzpicture}
  \end{center}
  \begin{gather*}
    \int_D \left( \frac{\partial Q}{\partial x} - \frac{\partial P}{\partial y} \right) \ dx \ dy = \sum_{i=1}^{k} \int_{D_i} \left( \frac{\partial Q}{\partial x} - \frac{\partial P}{\partial y} \right) \ dx \ dy = \sum_{i=1}^{k} \int_{\partial^+ D_i} P \ dx + Q \ dy = \int_{\partial^+ D} P \ dx + Q \ dy
  \end{gather*}
\end{proof}

\begin{example}
  Usare le formule di Gauss-Green per calcolare:
  \begin{gather*}
    \int_\gamma (x-y^3) \ dx + (y^3 + x^3) \ dy
  \end{gather*}
  Dove $\gamma$ è la frontiera di $D = \{ (x,y): x^2+y^2 \leq 1 \ , \ x \geq 0 \ , \ y \geq 0 \}$ orientata positivamente.\\
  \begin{gather*}
    \int_{\partial^+ D} (x-y^3) \ dx + (y^3 + x^3) \ dy = \int_D \frac{\partial}{\partial x} (y^3 + x^3) - \frac{\partial}{\partial y} (x - y^3) \ dx \ dy\\
    = \int_D (3x^2 + 3y^2) \ dx \ dy = 3 \int_0^\frac{\pi}{2} d\theta \int_0^1 \rho \cdot \rho^2 \ d\rho = \frac{3\pi}{8}
  \end{gather*}
\end{example}

\begin{theorem}[di Stokes nel piano]
  Riprendiamo la definizione di rotore dato un campo vettoriale:
  \begin{gather*}
    F = (F_1(x,y), F_2(x,y), 0)\\
    rot F = \begin{pmatrix}
      i & j & k\\
      \frac{\partial}{\partial x} & \frac{\partial}{\partial y} & \frac{\partial}{\partial z}\\
      F_1 & F_2 & 0
    \end{pmatrix} = k \left( \frac{\partial F_2}{\partial x} - \frac{\partial F_1}{\partial y} \right) 
  \end{gather*}
  th. vero e proprio:\\
  \underbar{Se} $D$ è s-decomponibile e $F \in C^1(\overline{D})$ \underbar{allora}:
  \begin{gather*}
    \int_D (rotF) \cdot k \ dx \ dy = \int_{\partial^+ D} \left\langle F,T \right\rangle ds 
  \end{gather*}
  Quando fo il prodotto scalare con $k$ si ricorda che $k = \begin{pmatrix}0\\ 0 \\ 1\end{pmatrix}$
  \begin{gather*}
    F = (F_1(x,y,z), F_2(x,y,z), F_3(x,y,z))\\
    rot F = ( \ )i + ( \ )j + ( \ )k\\
    \int_D rot \cdot k \ dx \ dy = \int_{\partial^+ D} \left\langle F \cdot T \right\rangle \ ds 
  \end{gather*}
\end{theorem}

\begin{center}
  \begin{tikzpicture}
    %assi xyz
    \draw[->](0,0,0) -- (3,0,0) node[at end, below]{$x$};
    \draw[->](0,0,0) -- (0,2,0) node[at end, left]{$z$};
    \draw[->](0,0,0) -- (0,0,3) node[at end, above]{$y$};
    \draw[dashed](1,0,2) ellipse (0.7 and 0.5);
    \draw[->](1,0,2) -- (1,1,2) node[at end, right]{$N = R$};
    \node at(1,0,3.4){$D$};
  \end{tikzpicture}
\end{center}
\begin{definition}[Divergenza]
  \underbar{Se} 
  \begin{gather*}
    F = \left( F_1(x,y), F_2(x,y) \right) 
  \end{gather*}
  Chiamo $div F = \frac{\partial F_1}{\partial x} + \frac{\partial F_2}{\partial y}$\\
  Se $F= (F_1, F_2, F_3)$ chiamo $div F = \frac{\partial F_1}{\partial x} + \frac{\partial F_2}{\partial y} + \frac{\partial F_3}{\partial z}$\\
  $div F$ è una funzione da $\mathbb{R}^2(o \mathbb{R}^3)$ in $\mathbb{R}$.
\end{definition}
\hfill\\
\begin{theorem}[della divergenza nel piano]
  \underbar{Sia} $D$ s-decomponibile \underbar{e} $F = \left(F_1(x,y) , F_2(x,y)\right) \in C^1(\overline{D})$:\\
  \underbar{allora}
  \begin{gather*}
    \int_D div F \ dx \ dy = \int_{\partial^+ D} \left\langle F, N \right\rangle \ ds \qquad \tiny{\text{il prof non ha il de col + ma forse c'è}}
  \end{gather*}
  dove ogni punto di $\partial D \ N$ indica il versore normale a $\partial D$ orientato in modo da puntare all'esterno.
\end{theorem}
\begin{gather*}
    \int_D div F = \int_D \frac{\partial F_1}{\partial x} + \frac{\partial F_2}{\partial y} \ dx \ dy \qquad (*)
  \end{gather*}
  Definisco $P$ e $Q$ in modo che:
  \begin{gather*}
    \frac{\partial Q}{\partial x} - \frac{\partial P}{\partial y} = \frac{\partial F_1}{\partial x} + \frac{\partial F_2}{\partial y}
  \end{gather*}
  cioè definisco $Q = F_1$ e $P = F_2$\\
  Ritornando al punto (*):
  \begin{gather*}
    \int \frac{\partial Q}{\partial x} - \frac{\partial P}{\partial y} \ dx \ dy = \int_{\partial^+ D} P \ dx + Q \ dy = \int_{\partial^+ D} -F_2 \ dx + F_1 \ dy
  \end{gather*}
  \underbar{se} $\left( \gamma_1(t), \gamma_2(t) \right) : [a,b]$ è una parametrizzazione coerente con l'orientazione di $\partial D$:
  \begin{gather*}
    \int_a^b -F_2 \left( \gamma_1(t), \gamma_2 (t) \right) \dot{\gamma}_1(t) + F_1\left( \gamma_1(t), \gamma_2(t) \right) \dot{\gamma}_2(t) \ dt \\
    = \int_a^b \left\langle \left( F_1\left( \gamma_1(t), \gamma_2(t) \right) F_2\left( \gamma_1(t), \gamma_2(t) \right) ; \left( \dot{\gamma}_2(t) , -\dot{\gamma}_1(t) \right)    \right)  \right\rangle \ dt\\
    = \int_a^b \left\langle \left( F_1\left( \gamma_1(t), \gamma_2(t) \right), F_1\left( \gamma_1(t), \gamma_2(t) \right)  \right) ; \underbrace{\frac{\left(\dot{\gamma}_1(t) , -\dot{\gamma}_2(t)\right) }{\sqrt{\dot{\gamma}_1^2 + \dot{\gamma}_2^2}}}_{N}  \right\rangle  \left( \sqrt{\dot{\gamma}_1^2 + \dot{\gamma}_2^2} \right)\\
    = \int_\gamma \left\langle F ; N \right\rangle  \ ds
  \end{gather*}
  \begin{itemize}
    \item Come usare le formule di G-G per calcolare l'area di $D$
    \item Se si sceglie $Q = 0$ e $P = y$
    \begin{gather*}
      \int_D 0-1 \ dx \ dy = \int_{\partial^+ D} y \ dx
    \end{gather*}
  \end{itemize}
    o anche area:
    \begin{gather*}
      D = -\int_{\partial^+ D} y \ dx
    \end{gather*}
    In maniere analoghe:
    \begin{gather*}
      \text{area }D = \int_{\partial^+ D} x \ dy\\
      \text{area }D = \frac{1}{2}\int_{\partial^+ D} x \ dy - y \ dx
    \end{gather*}


    %RIVEDERE






\section{2/12/25}
Si introducoo le superfici regolari parametriche.\\
Sono di fatto come le curve ma ad un parametro in più, avendo due parametri si può descrivere una superficie a due dimensioni nello spazio.\\
Come per le curve, distinguiamo una superficie parametrica regolare da una ordinaria.\\
\begin{definition}
  Una superficie parametrica è una funzione:
  \begin{gather*}
    \underline{\varphi} : D \subseteq \mathbb{R}^2 \to \mathbb{R}^3\\
    (u,v) \to \underline{\varphi}(u,v) = \begin{pmatrix}
      x(u,v)\\
      y(u,v)\\
      z(u,v)
    \end{pmatrix}
  \end{gather*}
  dove $x,y,z$ sono funzioni reali di due variabili reali (i due parametri che sto considerando).\\
  La superficie è regolare \underline{se}:
  \begin{itemize}
    \item $\underline{\varphi}$ è di classe $C^1$ in un aperto $A$ contenente $D$
    \item $\underline{\varphi}$ è iniettiva in $AV$
    \item la matrice jacobiana ha rango 2 in ogni punto del dominio.
  \end{itemize} 
  La matrice jacobiana si ricorda essere:
  \begin{gather*}
    \mathcal{J} = \begin{pmatrix}
      \frac{\partial x}{\partial u} & \frac{\partial x}{\partial v}\\
      \frac{\partial y}{\partial u} & \frac{\partial y}{\partial v}\\
      \frac{\partial z}{\partial u} & \frac{\partial z}{\partial v}
    \end{pmatrix} = (r_u \ | \ r_v) = \mathcal{J}_r(u,v)
  \end{gather*}
  Si ricorda che dire che $\mathcal{J}$ ha rango 2 implica che i due vettori "colonna" $r_u, r_v$ sono linearmente indipendenti.
\end{definition}
\noindent Facciamo alcuni esempi di Superfici di rotazione:

    \begin{example}
      Elicoide
      \begin{center}
        \begin{tikzpicture}
          %assi xyz
          \draw[->] (-4,2,0) -- (4,2,0) node[right] {$x$};
          \draw(0,2,0) -- (0,4,0);
          \draw[->] (0,2,-3) -- (0,2,5) node[above] {$y$};
          %elica
          \def\R{1} %raggio
          \def\k{0.5} %passo
          \def\a{6} %estremo inferiore
          \def\b{12} %estremo superiore
          \def\n{6} %numero di giri
          \def\h{\n * 2 * pi * \k} %altezza totale
          %superficie elicoidale
          \def\eps{0.2} %spessore superficie
          \def\Rpeps{\R + \eps}
          \def\Rmeps{\R - \eps}
          \tdplotsetmaincoords{70}{110}
          \begin{scope}[tdplot_main_coords]
            \foreach \v in {0, 0.1, ..., 1} {
              \draw[gray!50, domain=\a:\b,smooth,variable=\u, orange] plot 
              ({(\Rmeps + \v * 2 * \eps) * cos(\u r)}, 
               {(\Rmeps + \v * 2 * \eps) * sin(\u r)}, 
               {\k * \u});
            }
          \end{scope}
          \draw[->] (0,4,0) -- (0,6,0) node[above] {$z$};
          \filldraw[fill=black] (0,3.8,0) circle (2pt) node[left]{$\begin{pmatrix}
            0\\
            0\\
            z
          \end{pmatrix}$};
          \draw[black](0,3.8,0) -- (1.3,4,1) node[at end, right]{$P$};
          \filldraw[fill=black] (1.3,4,1) circle (2pt);
        \end{tikzpicture}
      \end{center}
      Si parte dall'equazione dell'elica cilindrica:
      \begin{gather*}
        \xi(u) = \begin{pmatrix}
          R \cos(u)\\
          R \sin(u)\\
          k u
        \end{pmatrix} \quad u \in \mathbb{R} \text{ \tiny{oppure} } u \in [a,b]
      \end{gather*}
      Si ottiene l'elicoide: $\underline{\varphi}_\xi$ collegando ogni punto di $\xi$ con l'asse $z$ tramite un segmento $\parallel$ al piano $xy$:
      \begin{gather*}
        \underline{\varphi}_\xi (u,v) = \begin{pmatrix}
          R u \cos(u)\\
          R v \sin(u)\\
          k u
        \end{pmatrix} \quad u \in I, v \in [0,1]
      \end{gather*}
      Considero quindi ad esempio il segmento da $P_z$ a $\begin{pmatrix}
        0\\
        0\\
        z \underline{P}_\xi
      \end{pmatrix}$ 
      cioè da $\begin{pmatrix}
        R \ \cos(u)\\
        R \ \sin(u)\\
        k \ u
      \end{pmatrix}$ a $\begin{pmatrix}
        0\\
        0\\
        k \ u
      \end{pmatrix} \quad \forall u \in I$
      \begin{gather*}
        \underset{\tiny \text{da } \underline{P}_\xi \text{ a } \begin{pmatrix}
          0\\
          0\\
          z \underline{P}_\xi
        \end{pmatrix}}{\text{segmento}} = \begin{pmatrix}
          v(R  \cos(u))\\
          v(R  \sin(u))\\
          \underset{= ku}{(1-v) ku + v \ ku}
        \end{pmatrix} \quad v \in [0,1]
      \end{gather*}
      Quando una uperficie è regolare?\\
      Guardo la matrice jacobiana:
        $\underline{\varphi}_\xi$ è superficie regolare in $D = [a,b] \times [0,1]$ e $A = (a,b) \times (0,1) \quad I = [a,b]$\\
        \underline{Se} ho le seguenti proprietà:\\
        (nel caso specifico dell'elica)
        \begin{itemize}
          \item $\underline{\varphi}_\xi \in C^1(A)$
          \item $\underline{\varphi}_\xi$ è iniettiva in $AV$
          \item Se la jacobiana ha rango 2 quindi:\\
          $\left(\underline{\varphi}_\xi\right)_u = \begin{pmatrix}
            -Rv \ \sin(u)\\
            Rv \ \cos(u)\\
            k
          \end{pmatrix} $ e $\left(\underline{\varphi}_\xi\right)_v = \begin{pmatrix}
            R \ \cos(u)\\
            R \ \sin(u)\\
            0
          \end{pmatrix} \quad \forall u \in (a,b) \ \forall v \in (0,1)$ 
        \end{itemize}
    \end{example}
    \begin{example}
      Superficie di rotazione generale da una cuva piana $\gamma$ ($\subseteq$ piano considerato) che ruota attorno ad un asse (asse coordinato).\\
      \begin{center}
        \tdplotsetmaincoords{70}{110}
        \begin{tikzpicture}[tdplot_main_coords, scale=1.5]
          %assi xyz
          \draw[->] (-4,0,0) -- (4,0,0) node[right] {$x$};
          \draw[->] (0,-4,0) -- (0,4,0) node[above] {$y$};
          \draw[->] (0,0,-1) -- (0,0,5) node[above] {$z$};
          %superficie di rotazione
          \draw[dashed, gray](0,0,1) circle (1);
          \draw[dashed, green](0,0,2) circle (1.5);
          \draw[dashed, gray](0,0,3) circle (1.5);
          \node at (0,1.5,2) {\textcolor{green}{parallelo}};
          \draw[green](-1,0,1) .. controls (-1.5,0,1.5) and (-0.5,0,2) ..  (-1.5,0,3) node[at end, right] {meridiano};
          %punto sulla curva e sull'asse
          \draw[dashed, orange] (0,0,2) -- (1.5,0.15,2);
          %curva piana
          \draw[red](1,0,1) .. controls (1.5,0,1.5) and (0.5,0,2) ..  (1.5,0,3) node[at end, right] {$\gamma$};
          \filldraw[black] (0,0,2) circle (0.8pt) node[above right] {$c$};
          \filldraw[black] (1.5,0.15,2) circle (0.8pt) node[above right] {$P$};
        \end{tikzpicture}
      \end{center}
      $\gamma \subseteq$ piano $yz$ (cioè $x=0$) che ruota attorno all'asse $z$.\\
      Parametrizzo la curva:
      \begin{gather*}
        \underline{\gamma}(t) = \begin{pmatrix}
          0\\
          y(t)\\
          z(t)
        \end{pmatrix} \quad t \in I
      \end{gather*}
      Con $y(t) \neq 0 \quad \forall t \in I$ e $y,z \in C^1(\mathring{I})$\\
      Prendo un punto $P$ sulla curva che descrive la circonferenza di centro sull'asse $z$ e raggio $y_P$, inoltre chiamo $c$ il punto sull'asse $z$ che connette il centro della circonferenza con il punto $P$:
      \begin{gather*}
        \begin{pmatrix}
          y_P \sin(\theta)\\
          y_P \cos(\theta)\\
          z_P
        \end{pmatrix} \quad \theta \in [0,2\pi]
      \end{gather*}
      Osservo che $y_P = y(t)$ e $z_P = z(t)$ quindi la parametrizzazione della superficie di rotazione è:
      \begin{gather*}
        \underline{\varphi}(t,\theta) = \begin{pmatrix}
          y(t) \ \sin(\theta)\\
          y(t) \ \cos(\theta)\\
          z(t)
        \end{pmatrix} \quad t \in I \quad \theta \in [0,2\pi]\\
        \text{per ex} \underline{\varphi}_\gamma \text{ è superfic. parametrica regolare (partendo da $\gamma$ curva regolare)}
      \end{gather*}
    \end{example}
  \subsection{caso assi scambiati}
  Possiamo avere lo stsso fenomeno solo con un'altro asse di rotazione ovviamente, per esempio prendiamo l'asse y:\\
  E prendiamo una curva $\gamma$ nel piano $yz$ che ruota appunto attorno all'asse $y$.\\
  $P(t) = (0,y(t), z(t))$ compie una circonferenza nel piano $\parallel zx$\\
  Quindi avremo il centro $C \equiv (0, y_P, 0)$ ed un raggio $R = |z_P|$:
  \begin{gather*}
    \underline{\varphi}_{\underline{\gamma}} (t, \theta) = \begin{pmatrix}
      z(t) \sin(\theta)\\
      y(t)\\
      z(t) \cos(\theta)
    \end{pmatrix} \quad \theta \in [0,2\pi], \ t \in I
  \end{gather*}
\begin{observation}
La regolarità dipende: sia dalla geometria che dalla parametrizzazione usata.
\end{observation}

\begin{definition}
  Le curve $t \to \varphi_\gamma (t , \ \theta \text{\tiny{fix.}})$ sono meridiani\\
  Le curve $\theta \to \varphi_\gamma (t \text{\tiny{fix.} }, \  \theta)$ sono paralleli.
\end{definition}
\begin{proof}
\hfill\\
Proviamo che la superficie di rotazione è regolare:\\
Bisogna provare:
\begin{itemize}
  \item che sia iniettiva
  \item
  \[
  \begin{aligned}
    \bigl(\underline{\varphi}_{\underline{\gamma}}\bigr)_t
      &= (\dot y\sin\theta,\ \dot y\cos\theta,\ \dot z),\\
    \bigl(\underline{\varphi}_{\underline{\gamma}}\bigr)_\theta
      &= (y\cos\theta,\ -y\sin\theta,\ 0),\\
    \bigl(\underline{\varphi}_{\underline{\gamma}}\bigr)_t\times\bigl(\underline{\varphi}_{\underline{\gamma}}\bigr)_\theta
      &= (y\dot z\sin\theta,\ -y\dot z\cos\theta,\ -y\dot y).
  \end{aligned}
  \]
\end{itemize}
\end{proof}

\newpage
\section{3/12/25}
In questa lezione vedremo come trovare il piano tangente ad una superficie data la sua parametrizzazione, e come calcolare l'area di una superficie(gli integrali di superficie)\\
\begin{center}
    \textbf{Superficie generica:}
\end{center}
\begin{center}
\begin{tikzpicture}
    \draw[->](0,0) -- (3,0) node[right]{$U$};
    \draw[->](0,0) -- (0,3) node[above]{$V$};
    \potato{(1,1)}{1};
    \draw (0.5,0.5) node{$D$};
  \end{tikzpicture}
  \begin{tikzpicture}
    \draw[->](-1,2) -- (-2.5,2) node[midway, above]{$\varphi$};
      \begin{axis}[
         view={60}{30},
         colormap/viridis,
         samples=50,
         samples y=15,
         domain=0:360,
         y domain=0:1,
         axis equal image
         ]
         \addplot3[surf, shader=interp]
         ({x/180 + 0.3*y*sin(x/20)},
        {y + 0.2*sin(x/35)},
        {0.8*sin(x/25)*cos(y*pi) 
         + 0.6*cos(x/40)*sin(y*pi)
         + 0.5*sin(x/15 + 2*y*pi)*cos(y*pi)
         + 0.4*sin(2*x/45)*cos(3*y*pi)
         + 0.35*cos(x/50)*sin(2*y*pi)
         + 0.25*sin(x/30 - y*pi)*cos(4*y*pi)
         + 0.2*cos(x/60)*sin(5*y*pi)
         + 0.15*sin(3*x/40)*cos(2*y*pi)
         + 0.6*exp(-(x-180)^2/8000)*exp(-20*(y-0.5)^2)
         + 0.45*exp(-(x-90)^2/5000)*exp(-15*(y-0.7)^2)});
         \end{axis}
  \end{tikzpicture}
\end{center}
Consideriamo quindi $\varphi$ che porta un dominio $D$ da $\mathbb{R}^2$ a $\mathbb{R}^3$:
\begin{gather*}
  \varphi : \underset{\subseteq \mathbb{R}^2 }{D} \to \mathbb{R}^3\\
\end{gather*}
E deve soddisfare la proprietà:
\begin{gather*}
  S = \varphi(D)
\end{gather*}
Ovvero l'immagine di D tramite $\varphi$ è la superficie $S$.\\
\hfill\\
Posso ora considerare $(U_0, V_0) \in \mathring{D}$ ovvero un punto interno al dominio D.\\
E il suo punto immagine sulla superficie tramite $\gamma(t)$ che è la curva che si comporta in modo che $V_0$ sia costante e $U_0$ varia con un parametro ($t$).\\
Inoltre considero $\widetilde{\varphi}(t) = \varphi(\gamma(t))$ che è proprio $\widetilde{\gamma} = \varphi(U_0 + t, V_0)$. Prendendo $U$ e $V$ generici ho:
\begin{gather*}
  \varphi(U,V) = \left( X(U,V), Y(U,V), Z(U,V) \right)\\
  \frac{\partial \varphi}{\partial U} = \left( \frac{\partial X}{\partial U}(U,V), \frac{\partial Y}{\partial U}(U,V), \frac{\partial Z}{\partial U}(U,V) \right) = \varphi_U (U,V)\\ 
\end{gather*}
Considero adesso $\dot{\widetilde{\gamma}}(t)|_{t=0}$, che è proprio:
\begin{gather*}
  \frac{d}{dt} \varphi(U_0 + t, V_0) |_{t=0} = \frac{\partial}{\partial U}\varphi(U_0,V_0)
\end{gather*}
Inoltre siccome $\widetilde{\gamma}(t) = \varphi(U_0 + t, V_0)$ se riconsidero la sua derivata ho:
\begin{gather*}
  \dot{\widetilde{\gamma}}(t) = \frac{d}{dt} \widetilde{\gamma}(t) = \frac{d}{dt} \varphi(U_0 + t, V_0) = \frac{\partial \varphi}{\partial U}(U_0 + t, V_0) \underset{= 1}{\boxed{\frac{d}{dt}(U_0 + t)}} + \frac{\partial \varphi}{\partial V}(U_0 + t, V_0) \underset{= 0}{\cancel{\frac{d}{dt}(V_0)}} = \frac{\partial \varphi}{\partial U} (U_0 + t,V_0)
\end{gather*}
\hfill\\
Prendiamo ora un'altro caso in cui $\gamma(t)$ è una qualsiasi curva che passa per il mio punto $(U_0, V_0)$, quindi adesso cos'è $\widetilde{\gamma}$?
\begin{gather*}
  \widetilde{\gamma}(t) = \varphi  \gamma(t) = \varphi(U(t), V(t))\\
  \dot{\widetilde{\gamma}}(0) = \frac{d}{dt} \varphi(U(t), V(t)) |_{t=0} = \frac{\partial \varphi}{\partial U}(U_0, V_0) \cdot \frac{d}{dt}U(0) + \frac{\partial \varphi}{\partial V}(U_0, V_0) \cdot \frac{d}{dt}V(0)
\end{gather*}
Per domodità useremo la notazione: $\frac{\partial \varphi}{\partial U} \leftrightarrow \varphi_U$\\
Prendiamo ora un vettore normale al piano tangente alla superficie in $\varphi(U_0, V_0)$, per farlo farò il prodotto vettoriale (o prodotto wedge $\land$)
\begin{gather*}
  (\varphi_U \land \varphi_U) (U_0, V_0)
\end{gather*}
Questo è un vettore normale al piano tangente a $S$ in $\varphi(U_0, V_0)$.\\
Se ora lo divido per la sua norma ottengo il versore normale:
\begin{gather*}
  N = \frac{\varphi_U \land \varphi_V}{\left\lVert \varphi_U \land \varphi_V \right\rVert }
\end{gather*}
Se rappresento questa operazione con una matrice ho:
\begin{gather*}
  \begin{matrix}
    i & j & k\\
    X_U & Y_U & Z_U\\
    X_V & Y_V & Z_V
  \end{matrix}\\
  \begin{matrix}
    i \underbrace{det\begin{pmatrix}
      Y_U & Z_U\\
      Y_V & Z_V
    \end{pmatrix}}_{\frac{\partial (y,z)}{\partial (U,V)}}   - j \underbrace{det\begin{pmatrix}
      X_U & Z_U\\
      X_V & Z_V
    \end{pmatrix}}_{\frac{\partial (x,z)}{\partial (U,V)}} + k \underbrace{det\begin{pmatrix}
      X_U & Y_U\\
      X_V & Y_V
    \end{pmatrix}}_{\frac{\partial (x,y)}{\partial (U,V)}}
  \end{matrix}
\end{gather*}
Se ora considero $(x,y,z) \in$ \textbf{piano tang.} a $S$ in $\varphi(U_0, V_0)$, posso scrivere l'equazione del piano tangente:
\begin{gather*}
  \left\langle (x,y,z) - \varphi(U_0, V_0) , (\varphi_U \land \varphi_V)(U_0, V_0)\right\rangle = 0\\
  \left\langle \left(x-x(U_0,V_0), y-y(U_0,V_0), z-z(U_0,V_0),\right) , (\varphi_U \land \varphi_V)(U_0, V_0)  \right\rangle 
\end{gather*} 
Questo considerando la scrittura precedente può anche essere espresso come:
\begin{gather*}
  det \begin{pmatrix}
    x - x(U_0,V_0) & y - y(U_0,V_0) & z - z(U_0,V_0)\\
    X_U(U_0,V_0) & Y_U(U_0,V_0) & Z_U(U_0,V_0)\\
    X_V(U_0,V_0) & Y_V(U_0,V_0) & Z_V(U_0,V_0)
  \end{pmatrix} = 0
\end{gather*}

Facciamo ora alcuni esempi:
\begin{example}
  La sfera, prendiamone una di centro $(0,0,0)$ e raggio $R>0$:
  \begin{gather*}
    D= \overset{\psi \in}{[0,\pi]} \times \overset{\theta \in}{[0,2\pi]}
  \end{gather*}
  \begin{center}
\tdplotsetmaincoords{70}{110}
    \begin{tikzpicture}[tdplot_main_coords]
      \draw[->](0,0,0) -- (0,0,3) node[above]{$z$};
      \draw[->](0,0,0) -- (3,0,0) node[right]{$x$};
      \draw[->](0,0,0) -- (0,3,0) node[above]{$y$};
      \draw[dashed, gray](0,0,0) circle (2);
      \draw[red](0,0,0) -- (2,2,2) node[at end, right]{$P$};
      \draw[dashed](2,2,0.35) -- (2,2,2);
      \node at (0,0,0.3)[above, right] {$\varphi$};
      \node at(0,0,0) [below]{$\theta$};
    \end{tikzpicture}
  \end{center}
  \begin{gather*}
    (\psi, \theta) \in D \to \varphi(\psi, \theta) :\left(
      R \sin(\psi) \cos(\theta),
      R \sin(\psi) \sin(\theta),
      R \cos(\psi)
    \right)
  \end{gather*}
  Ricerco il piano tangente in un punto generico $(\theta_\text{cost}, \psi_\text{cost})$, quindi calcolo:
  \begin{gather*}
    \varphi_\psi = \left(
      R \cos(\psi) \cos(\theta),
      R \cos(\psi) \sin(\theta),
      -R \sin(\psi)
    \right)\\
    \varphi_\theta = \left(
      -R \sin(\psi) \sin(\theta),
      R \sin(\psi) \cos(\theta),
      0
    \right)\\
    \varphi_\psi \land \varphi_\theta = \left( \underbrace{R^2\sin(\psi)\cos(\theta), R^2 \sin(\psi)\sin(\theta), R^2\cos(\psi)}_{R\varphi(\psi, \theta)} \right) = R \sin(\psi) \varphi(\psi, \theta)
  \end{gather*}
  Calcolo ora il versore normale:
  \begin{gather*}
    N = \frac{\varphi_\psi \land \varphi_\theta}{\left\lVert \varphi_\psi \land \varphi_\theta \right\rVert} = \frac{\cancel{R} \cancel{\sin(\psi)} \varphi(\psi, \theta) }{\cancel{R} \cancel{\sin(\psi)} \underbrace{\left\lVert \varphi(\psi, \theta) \right\rVert}_{= R} } = \frac{\varphi(\psi, \theta)}{R}\\
    \left\lVert \varphi_\psi \land \varphi_\theta \right\rVert = R^2 \sin(\psi) 
  \end{gather*}
\end{example}

\begin{example}
  Prendiamo ora in esempio un piano generico:
  \begin{gather*}
    \varphi : D \to \mathbb{R}^3\\
    \varphi(x,y) = (x,y, f(x,y))\\
    \varphi_x (x,y) = (1,0,f_x(x,y))\\
    \varphi_y (x,y) = (0,1,f_y(x,y))
  \end{gather*}
  Calcoliamo ora il vettore normale:
  \begin{gather*}
    \varphi_x \land \varphi_y = det \begin{vmatrix}
      i & j & k\\
      1 & 0 & f_x\\
      0 & 1 & f_y
    \end{vmatrix} = (-f_x, -f_y, 1)\\
    \left\lVert \varphi_x \land \varphi_y \right\rVert = \sqrt{(f_x)^2 + (f_y)^2 + 1}\\
    N = \frac{(-f_x, -f_y, 1)}{\sqrt{1 + \left\lVert \nabla f \right\rVert^2 }}
  \end{gather*}
  Troviamo ora l'equazione del piano tangente al grafico in $(x_0,y_0, f(x_0,y_0))$\\
\begin{gather*}
    \left\langle \left( x-x_0, y-y_0, z - f(x_0,y_0) \right), \left( -f_x(x_0,y_0), -f_y(x_0,y_0), 1 \right)  \right\rangle = 0\\
    (x-x_0) f_x(x_0,y_0) + (y-y_0) f_y(x_0,y_0) + f(x_0,y_0) = z
\end{gather*}\end{example}
\subsection{integrali di superficie}

Per integrali di superficie si intende calcolare l'area superficiale di una data funzione.\\
Per fare questo in generale si considera la trasformazione non lineare $T$ che porta dal piano del dominio $D$ alla superficie ovvero la funzione, dopodichè si considera il piano tangente ad una superficie infinitesima e si calcola quel'area, si fa questo per tutta la superficie e si somma il tutto. Vediamo nel dettaglio cosa significa.\\
\hfil\\
Consideriamo la trasformazione $T$
\begin{center}
    \begin{tikzpicture}
        \draw[->](0,0) -- (0,3.5) node[at end, left]{$y$};
        \draw[->](0,0) -- (3,0) node[at end, below]{$x$};
        \potato{(1,0)}{1};
        \node at(3.5,1.5){$\quad \varphi(U,V)$};
        \tiny
        \draw(1.3,1.5) ..controls(1.6,1.7).. (2,1.8) node[at end, above]{4};
        \draw(1.5,1) ..controls(1.8,1.2).. (2.2,1.3) node[at start, below]{1};
        \draw(1.3,1.5) ..controls(1.45,1.3).. (1.5,1) node[at start, left]{2};
        \draw(2,1.8) ..controls(2.15,1.5).. (2.2,1.3) node[at end, right]{3};
        \draw[->, orange](1.5,1) -- (2.2,1.3);
        \draw[<-, orange, opacity = 0.4](1.3,1.5) -- (1.5,1);
        \normalsize
    \end{tikzpicture}
    \tab$\quad$
    \begin{tikzpicture}
        \draw(1.3,1) rectangle (2.3,1.5);  
        \draw[dashed](1.3,0) -- (1.3,1) node[at start, below]{$U_0$};
        \draw[dashed](2.3,0) -- (2.3,1) node[at start, below]{$U_0+dU$};
        \draw[dashed](0,1) -- (1.3,1) node[at start, left]{$V_0$};
        \draw[dashed](0,1.5) -- (1.3,1.5) node[at start, left]{$V_0+dV$};
        \potato{(1,0)}{1};
        \node at(3.5,1.5){$D$};
        \draw[->](-2.5,1.5) -- (-3.5,1.5) node[above, midway]{$T$};
        \draw[->](0,0) -- (0,3.5) node[at end, left]{$V$};
        \draw[->](0,0) -- (4,0) node[at end, below]{$U$};
    \end{tikzpicture}
\end{center}
I vari vertici del segmento infinitesimo della funzione sono:
\begin{align*}
    &1: \varphi(U_0,V_0)\\
    &2: \varphi(U_0,V_0+dV)\\
    &3: \varphi(U_0+dU,V_0)\\
    &4: \varphi(U_0+dU,V_0+dV)
\end{align*}
Se ora consideriamo da ognuno di questi punti i vertici che partono da essi e collegano direttamente i vertici adiacenti otteniamo un parallelogramma infinitesimo, per esempio quello in arancione intenso è:
\begin{gather*}
  \varphi(U_0 + dU,V_0) - \varphi(U_0,V_0) \approx \varphi_U (U_0,V_0) dU + o(dU)\\
\end{gather*}
Così facendo stiamo considerando i segmenti rettiche partono dal punto $(U_0,V_0)$ in questo modo:
  \begin{multicols}{2}
    \noindent
    \begin{center}
     \begin{tikzpicture}[scale = 2.5]
     \draw[rotate around={29:(0,0,0)}](1.3,1.5) ..controls(1.6,1.7).. (2,1.8);
     \draw[rotate around={29:(0,0,0)}](1.5,1) ..controls(1.8,1.2).. (2.2,1.3);
     \draw[rotate around={29:(0,0,0)}](1.3,1.5) ..controls(1.45,1.3).. (1.5,1);
     \draw[rotate around={29:(0,0,0)}](2,1.8) ..controls(2.15,1.5).. (2.2,1.3);
     \draw[->, red, very thick, rotate around={29:(0,0,0)}](1.3, 1.5) -- (1.5,1.2) node[at end, left, below]{$\varphi_U (U_0,V_0)dU$};
     \draw[->, red, very thick, rotate around={29:(0,0,0)}](1.3, 1.5) -- (2,2) node[at end, left]{$\varphi_V (U_0,V_0)dV$};
     \draw[dashed, red, rotate around={29:(0,0,0)}](1.5,1) -- (1.5,1.2);
     \draw[dashed, red, rotate around={29:(0,0,0)}](2,1.8) -- (2,2);
     \draw[red, rotate around={29:(0,0,0)}](1.5,1.2) -- (2.2,1.7);
     \draw[red, rotate around={29:(0,0,0)}](2,2) -- (2.2,1.7);
     \draw[dashed, red, rotate around={29:(0,0,0)}](2.2,1.3) -- (2.2 ,1.7);
    \end{tikzpicture}
    \end{center}
    \columnbreak
    \hfill\\
    \scriptsize Questa rappresentazione \textbf{non} è coerente con l'esempio precedente per via di prospettiva ma è utile per visualizzare i vettori considerati. \normalsize\\
  \end{multicols}
Una volta che abbiamo questi due vettori ne consideriamo l'area del parallelogramma che formano, che è data dal modulo del loro prodotto vettoriale:
\begin{gather*}
  \left\lVert \varphi_U (U_0, V_0) \land \varphi_V (U_0 , V_0) \right\rVert dU dV 
\end{gather*}
Che è proprio l'area infinitesima della superficie considerata.\\
Se ne consideriamo quindi la somma (e quindi l'integrale) otteniamo:
\begin{gather*}
  Area(S) = \iint_D \left\lVert \varphi_U \land \varphi_V \right\rVert dUdV
\end{gather*}

\newpage
\section{4/12/25}
Presa una superficie $S$ una superficie descritta da una parametrizzazione:
\begin{gather*}
  \varphi : D \subseteq \mathbb{R}^2 \to \mathbb{R}^3\\
  \varphi(u,v) = (x(u,v), y(u,v), z(u,v))\\
  \varphi_u = \frac{\partial \varphi}{\partial u} = (x_u, y_u, z_u)\\
  \varphi_v = \frac{\partial \varphi}{\partial v} = (x_v, y_v, z_v)
\end{gather*}
E considerando un punto nella superficie considerato il piano tangente alla superficie in quel punto si identifica il vettore normale al piano tangente come:
\begin{gather*}
  N = \frac{\varphi_u \land \varphi_v}{\left\lVert \varphi_u \land \varphi_v \right\rVert}
\end{gather*}
E abbiamo anche definito l'area della superficie come:
\begin{gather*}
  Area(S) = \iint_D \left\lVert \varphi_u \land \varphi_v \right\rVert dudv
\end{gather*}
\subsection{Integrali di una funzione su una superficie}
Presa una funzione $f(x,y,z)$ sia una funzione definita in un sottoinsieme di $\mathbb{R}^3$ che contiene $S$ integrale (superficiale) , di $f$ su $S$.
\begin{gather*}
  \int_S f d \sigma
\end{gather*}
Questa funzione va ha come dominio la superficie $S$ e come codominio i numeri reali.\\
Per calcolare questo integrale si usa la parametrizzazione della superficie:
\begin{gather*}
  \int_S f d \sigma = \iint_D f(\varphi(u,v)) \left\lVert \varphi_u \land \varphi_v \right\rVert dudv\\
  = \iint_D f(x(u,v), y(u,v), z(u,v)) \left\lVert \varphi_u \land \varphi_v \right\rVert dudv
\end{gather*}

\begin{example}
  Paraboloide grafico di :
  \begin{gather*}
    f(x,y) = \frac{x^2+y^2}{2}
  \end{gather*}
  dove $(x,y) \in \{x^2+y^2 \leq 8\}$, e VOglio calcolare l'area di questa superficie.
  \begin{gather*}
    Area(S) = \iint_D \left\lVert \varphi_x \land \varphi_y \right\rVert dxdy
  \end{gather*}
  Con $\left\lVert \varphi_x \land \varphi_y \right\rVert = \sqrt{1+ \left\lVert \nabla f(x,y) \right\rVert^2 }$\\
  Procedendo con i calcoli:
  \begin{gather*}
    \int_{x^2+y^2 \leq 8} \sqrt{1 + \left\lVert \nabla f(x,y) \right\rVert^2 } dxdy
  \end{gather*}
  Calcoliamo il dragiente:
  \begin{gather*}
    \frac{\partial f}{\partial x} = x\\
    \frac{\partial f}{\partial y} = y\\
  \end{gather*}
  Quindi l'integrale diventa:
  \begin{gather*}
    \int_{x^2+y^2 \leq 8} \sqrt{1 + x^2 + y^2} dxdy\\
    \text{Passo in coordinate polari: } x = \rho \cos(\theta), \ y = \rho \sin(\theta)\\
    dxdy = \rho d\rho d\theta\\
    \rho \in [0, \sqrt{8}], \ \theta \in [0, 2\pi]\\
    \int_0^{2\pi} \int_0^{\sqrt{8}} \rho \sqrt{1 + \rho^2} d\rho d\theta\\
    = 2\pi \int_0^{\sqrt{8}} \rho \sqrt{1 + \rho^2} d\rho\\
    = 2\pi \left[ \frac{(1+\rho^2)^{3/2}}{3} \right]_0^{\sqrt{8}}\\
    = \frac{2\pi}{3} \left( 9^{3/2} - 1 \right) = \frac{2\pi}{3} (27 - 1) = \frac{52\pi}{3}
  \end{gather*}
\end{example}

\subsection{Facciamo alcuni esempi fisici}
\begin{example}
  Se $S$ superficie e $\delta$ indica la densità superficiale di massa allora
  \begin{gather*}
    M = \int_S \delta d \sigma
  \end{gather*}
  Il centro di massa della superficie ha coordinate:
  \begin{gather*}
    x_b = \frac{1}{M} \int_S x \ \delta(x,y,z) \ d \sigma\\
    y_b = \frac{1}{M} \int_S y \ \delta(x,y,z) \ d \sigma\\
    z_b = \frac{1}{M} \int_S z \ \delta(x,y,z) \ d \sigma
  \end{gather*}
  E il baricentro geometrico è:
  \begin{gather*}
    x_G = \frac{1}{Area(S)} \int_S x \ d \sigma\\
    y_G = \frac{1}{Area(S)} \int_S y \ d \sigma\\
    z_G = \frac{1}{Area(S)} \int_S z \ d \sigma
  \end{gather*}
  Che è quando la densità è costante.
\end{example}
\begin{example}
  Prendo $S =$ calotta superiore della superficie sferica di centro $(0,0,0)$ e raggio $R$ quindi:\\
  (uso le coordinate sferiche per parametrizzare la superficie)
  \begin{gather*}
    D = \left\{ (\psi, \theta) \in [0,2\pi] \times [0, \frac{\pi}{2}]\right\} \\
    \varphi(\psi, \theta) = \begin{pmatrix}
        x =R \sin(\psi) \cos(\theta)\\
        y =R \sin(\psi) \sin(\theta)\\
        z =R \cos(\psi)
    \end{pmatrix}\\
    \left\lVert \varphi_\psi \land \varphi_\theta \right\rVert = R^2 \sin(\psi)
  \end{gather*}
  Per la massa $S$ con la densità data di $\delta = x^2+y^2$
  \begin{gather*}
    \int_S \ d \sigma = \iint_D \underbrace{\delta(x(\psi, \theta), y(\psi, \theta), z(\psi, \theta))}_{R^2 \sin^2(\psi)} \underbrace{\left\lVert \varphi_\psi \land \varphi_\theta \right\rVert}_{R^2 \sin(\psi)} d\psi d\theta\\
    = \int_0^{2\pi} \int_0^{\pi/2} R^2 \sin^2(\psi) \cdot R^2 \sin(\psi) d\psi d\theta\\
    = 2\pi R^4 \int_0^{\pi/2} \sin^3(\psi) d\psi\\
    = 2\pi R^4 \left[ -\cos(\psi) + \frac{\cos^3(\psi)}{3} \right]_0^{\pi/2} = 2\pi R^4 \left( 1 - \frac{1}{3} \right) = \frac{4\pi R^4}{3}
  \end{gather*}
  Calcoliamone ora il centro di massa $x_b=y_b=0$ per simmetria di $S$
  \begin{gather*}
    z_b = \frac{1}{M} \int_S z \ \delta(x,y,z) d \sigma\\
    = \frac{1}{M} \iint_D R \cos(\psi) \cdot R^2 \sin^2(\psi) \cdot R^2 \sin(\psi) d\psi d\theta\\
    = \frac{2\pi R^5}{M} \int_0^{\pi/2} \cos(\psi) \sin^3(\psi) d\psi\\
    = \frac{2\pi R^5}{M} \left[ -\frac{\sin^4(\psi)}{4} \right]_0^{\pi/2} = \frac{2\pi R^5}{M} \cdot \frac{1}{4} = \frac{\pi R^5}{2M}\\
    = \frac{\pi R^5}{2 \cdot \frac{4\pi R^4}{3}} = \frac{3R}{8}
  \end{gather*}
\end{example}

\begin{example}
  
  Superficie di rotazione ottenuta intorno all'asse $z$ da una curva contenuta nel piano $yz$, (cioè $x=0$) in cui $x>0$:
  \begin{center}
      \begin{tikzpicture}
        %assi x,y,z
        \draw[->](0,0) --(5,0) node[anchor=north east]{$x$};
        \draw[->](0,0) --(0,5) node[anchor=north west]{$z$};
      \draw[->](0,0) --(-2,-2) node[anchor=south]{$y$};
        %curva
        \draw(1,0.5) .. controls (1.5,2) and(0.2,2.8).. (0,2.8) node[midway, right]{$(y(t), z(t))$};
        \draw(-1,0.5) .. controls (-1.5,2) and(-0.2,2.8).. (0,2.8);
        %sezioni ellittiche
        \draw(0,0.58) ellipse (1.029 and 0.3) node[right]{$h_1$};
        \draw(0,2) ellipse (0.88 and 0.3) node[right]{$h_2$};
      \end{tikzpicture}
    \end{center}
    $\gamma \subseteq$ nel piano $(y,z)$ e la sua parametrizzazione è
    \begin{gather*}
      \gamma(t) : [a,b] \to (y(t), z(t))\\
      S: \varphi (t, \theta) = (y(t) \cos(\theta), y(t) \sin(\theta), z(t))\\
      D = \{ (t, \theta): t \in [a,b] , \theta \in [0 , 2\pi] \}
    \end{gather*} 
\end{example}

\begin{example}
  Prendiamo un cono descritto nel seguente modo:
  \begin{center}
    \tdplotsetmaincoords{70}{110}
    \begin{tikzpicture}[tdplot_main_coords]
      %assi xyz
      \draw[->](0,0,0) -- (5,0,0) node[right]{$x$};
      \draw[->](0,0,0) -- (0,5,0) node[above]{$y$};
      \draw[->](0,0,0) -- (0,0,5) node[above]{$z$};
      %cono
      \draw[dashed, gray](0,0,3) circle (2);
      \draw[orange](0,0,0) -- (2,0,3);
      \draw[orange](0,0,0) -- (0,2,3);
      \draw[orange](0,0,0) -- (-2,0,3);
      \draw[orange](0,0,0) -- (0,-2,3);
    \end{tikzpicture}
  \end{center}
  \begin{gather*}
    \left\lVert \varphi_t \land \varphi_\theta \right\rVert = y(t) \sqrt{(\dot{y}(t))^2 + (\dot{z}(t))^2}\\
    \text{area}(S) \int_D \left\lVert \varphi_t \land \varphi_\theta \right\rVert dt d\theta\\
    \varphi(t) : [0,1] \to (t,t)\\
    \left\lVert \varphi_t \land \varphi_\theta \right\rVert = \underset{y(t) \ (\frac{dy}{dt})^2+(\frac{dz}{dt})^2}{t \sqrt{1+1}} = t \sqrt{2}\\
    \text{area}(S) = \int_0^{2\pi} \int_0^1 t \sqrt{2} dt d\theta = 2\pi \cdot \frac{\sqrt{2}}{2} = \pi \sqrt{2}
  \end{gather*}
\end{example}

\subsection{Perchè cambiando parametrizzazione di una superficie non cambiano i risultati}
Prendo una mappa $\phi : T \to D$\\
Ho ipotesi du $\phi$ quelle che ho supposto quando ho parlato di cambiamento di variabili negli integrali doppi:
\begin{itemize}
  \item $\phi$ è biunivoca
  \item $\phi$ è di classe $C^1$
  \item Il det Jacobiano di $\phi \neq 0$ in ogni punto di $T$
\end{itemize}
\begin{gather*}
  \phi : \mathbb{R}^2 \to \mathbb{R}^2\\
  (s,t) \overset{\phi}{\to} (u, v)\\
  \psi (s,t) = (u(s,t), v(s,t))
\end{gather*}
Definisco ora $\psi := \varphi_0 \phi$
\begin{gather*}
  \psi (s,t) = \varphi (u(s,t), v(s,t)) = (x(u(s,t), v(s,t)), y(u(s,t), v(s,t)), z(u(s,t), v(s,t)))\\
  \psi_s (s,t) = \frac{\partial}{\partial s} \psi(s,t) = \varphi_u (u(s,t), v(s,t)) \cdot \frac{\partial u}{\partial s} (s,t) + \varphi_v (u(s,t), v(s,t)) \cdot \frac{\partial v}{\partial s}(s,t)\\
  \psi_t (s,t) = \frac{\partial}{\partial t} \psi(s,t) = \varphi_u (u(s,t), v(s,t)) \cdot \frac{\partial u}{\partial t} (s,t) + \varphi_v (u(s,t), v(s,t)) \cdot \frac{\partial v}{\partial t}(s,t)\\
\end{gather*}
Che relazione c'è tra $\psi_s, \psi_t$ e $\varphi_u, \varphi_v$?\\
\hfill\\
Andiamo ora a farne il prodotto vettoriale:
\begin{gather*}
  \psi_s (s,t) \land \psi_t (s,t) = \\
  \left( \underline{\varphi_u} u_s + \underline{\varphi_v} v_s \right) \land \left( \underline{\varphi_u} u_t + \underline{\varphi_v} v_t \right)\\
  = u_s u_t ( \varphi_u \land \varphi_u ) + u_s v_t ( \varphi_u \land \varphi_v ) + v_s u_t ( \varphi_v \land \varphi_u ) + v_s v_t ( \varphi_v \land \varphi_v )\\
  = (u_s v_t - v_s u_t) ( \varphi_u \land \varphi_v )
  %\begin{vmatrix}
  %  i & j & k\\
  %  \psi_{u_1} & \psi_{u_2} & \psi_{u_3}\\
  %  \psi_{v_1} & \psi_{v_2} & \psi_{v_3}
  %\end{vmatrix} \cdot \begin{vmatrix}
  %  u_s & u_t\\
  %  v_s & v_t
  %\end{vmatrix}\\ = (\varphi_u \land \varphi_v)(u(s,t), v(s,t))
   \cdot det \begin{pmatrix}
    u_s & u_t\\
    v_s & v_t
  \end{pmatrix}
\end{gather*}
Una notazione comoda è:
\begin{gather*}
  \frac{\partial (u,v)}{\partial (s,t)} = det \begin{pmatrix}
    u_s & u_t\\
    v_s & v_t
  \end{pmatrix}
\end{gather*}

\subsection{gli integrali non dipendono dalla parametrizzazione}
\begin{gather*}
  \int_S f d\sigma \overset{\text{usando } \varphi}{=} \int_D f(x(u,v), y(u,v), z(u,v)) \left\lVert \varphi_u \land \varphi_v \right\rVert dudv\\
  \overset{\text{cambio var. ass. a } \phi}{=} \int_T f(x(u(s,t), v(s,t)), y(u(s,t), v(s,t)), z(u(s,t), v(s,t))) \cdot \left\lVert \varphi_u \land \varphi_v \right\rVert \cdot \left| \frac{\partial (u,v)}{\partial (s,t)} \right| dsdt\\
  \text{usando la param.} \psi \begin{cases}
    = \int_T f( \ "" \ ) \cdot \left\lVert (\psi_s \land \psi_t) (s,t) \right\rVert \ ds \ dt\\
    = \int_S f \ d\sigma
  \end{cases}
\end{gather*}

\subsection{Orientazione e bordo di una superficie}
Generalemtne per questi concetti si utilizza la geometria differenziale. Per quello che ci interessa useremo delle definizioni intuitive.\\
\hfill\\
Sia  $S$ una superficie e supponiamo che si possa 
scegliere il verso normale in ogni punto di $S$ in modo 
che partendo da un punto $P_0 \in S$ e seguendo una qualsiasi 
curva continua chiusa (che dunque vi torni in $P_0$) 
contenuta in $S$ il vettore normale vari con continuità e ritorni alla posizione iniziale. 
In tal caso diremo che $S$ è orientabile.\\
\hfill\\
Un esempio di superficie orientabile è una sfera.\\
Un esempio di superficie non orientabile è il nastro di Moebius
\begin{center}
  \begin{tikzpicture}
    %nastro di Moebius
\begin{axis}[
    hide axis,
    view={40}{40}
]
\addplot3 [
    surf, shader=faceted interp,
    point meta=x,
    colormap/greenyellow,
    samples=40,
    samples y=5,
    z buffer=sort,
    domain=0:360,
    y domain=-0.5:0.5
] (
    {(1+0.5*y*cos(x/2)))*cos(x)},
    {(1+0.5*y*cos(x/2)))*sin(x)},
    {0.5*y*sin(x/2)});
\addplot3 [
    samples=50,
    domain=-145:180, % The domain needs to be adjusted manually, depending on the camera angle, unfortunately
    samples y=0,
    thick
] (
    {cos(x)},
    {sin(x)},
    {0});
\end{axis}
  \end{tikzpicture}
\end{center}
La parametrizzazione del nastro di Moebius è:
\begin{gather*}
  \varphi(u,v) = \left( \left(1 + \frac{v}{2} \cos\left(\frac{u}{2}\right)\right) \cos(u), \left(1 + \frac{v}{2} \cos\left(\frac{u}{2}\right)\right) \sin(u), \frac{v}{2} \sin\left(\frac{u}{2}\right) \right)\\
  u \in [0, 2\pi], \ v \in [-1, 1]
\end{gather*}
\hfill\\
Un'altra definizione importante è quella di bordo di una superficie.\\
Prendiamo in esempio una semisfera e consideriamone la sua superficie $S$.\\
Il suo bordo è il cerchio che si ottiene tagliando la sfera (che formerei da due delle mie semisfere) a metà.\\
Se ad esempio invece di una semisfera perfetta si ha una parte di una sfera il bordo sarà la circonferenza che seziona la sfera.\\
\hfill\\
Più in generale il bordo di una superficie $S$ è l'insieme dei punti di $S$ che 
possono essere raggiunti da una curva contenuta in $S$ che non può essere 
estesa oltre $S$.\\
\hfill\\
Un esempio di superficie senza bordo è la sfera.\\
Un esempio di superficie con bordo è il paraboloide limitato da un piano orizzontale.
\hfill\\ 
\hfill\\ 
Passiamo ora al concetto di orientazione della superficie del bordo. teniamo a mente l'esempio della semisfera per praticità\\
Fissiamo una orientazione della superficie (orientabile [presa per hp.]) $S$.\\
Si determina due "lati" di $S$, chiamando lato positivo quello verso cui punta il versore normale $N$ e lato negativo quello opposto.\\
Diciamo che $\partial S$ è orientato positivamente rispetto a $S$ se, percorrendo $\partial S$ orientandosi sul lato positivo di $S$ si lasciano i punti di $S$ sulla sinistra.\\
Ci sono figure che non hanno bordo:
\begin{itemize}
  \item sfera $\partial = \emptyset$
  \item toro $\partial = \emptyset$
  \item Paraboloide non limitato $\partial = \emptyset$
\end{itemize}
Tutte le superfici chiuse non hanno bordo ma non tutte le superfici senza bordo sono chiuse (es. paraboloide non limitato).


\newpage
\section{9/12/25}
\begin{definition}[Per th. di Stokes]
  Si para di flusso di un campo lungo una superficie.\\
  \underbar{Se} io ho un campo $F$ definito in un intorno di una superficie orientabile e\\
  \underbar{se} $N$ è una scelta del vettore normale a $S$.\\
  Definisco flusso di $F$ normale a $S$ come:
  \begin{gather*}
    S = \int_S \langle F, N \rangle d \sigma
  \end{gather*}
  Appurato questo:\\
  \hfill\\
  \underbar{se} $\varphi : D \to \mathbb{R}^3$ è una parametrizzazione di $S$ quell'integrale, \\
  \underbar{se} $\varphi(u,v) = (x(u,v), y(u,v), z(u,v))$, diventa:
  \begin{gather*}
    \int_D \underbrace{\left\langle F(x(u,v), y(u,v), z(u,v)) , \frac{\varphi_u \land \varphi_v}{\cancel{\left\lVert \varphi_u \land \varphi_v \right\rVert}}\right\rangle }_{\left\langle F,N \right\rangle } \cdot \cancel{\left\lVert \varphi_u \land \varphi_v \right\rVert} dudv\\
    = \int_D \left\langle F(x(u,v), y(u,v), z(u,v)) , \varphi_u \land \varphi_v \right\rangle dudv
  \end{gather*}
\end{definition}

\begin{example}
  Prendiamo la funzione $z = x^2 + y^2$
  e il dominio:
  \begin{gather*}
    D = \{ (x,y) : \quad |x| \leq 1 \ , \ |y| \leq 1 \}
  \end{gather*}

    \begin{center}
      \begin{tikzpicture}
        %paraboloide x^2+y^2
        \begin{axis}[
            view={40}{40}
        ]
        \addplot3 [
            surf, shader=faceted interp,
            point meta=x,
            colormap/greenyellow,
            samples=40,
            samples y=40,
            z buffer=sort,
            domain=-1:1,
            y domain=-1:1
        ] (
            {x},
            {y},
            {x^2 + y^2});
        \end{axis}
      \end{tikzpicture}
    \end{center}

    Ho $S: (x,y) \in D \to (x,y,x^2+y^2)$\\
    e oriento $S$ scegliendo la normale che punta verso l'alto:\\
    In questo esempio ho $f(x,y) = x^2+ y^2$ quindi:
    \begin{gather*}
      N = \frac{\left(\frac{-\partial f}{\partial x} , \frac{-\partial f}{\partial y} , 1\right) }{\left\lVert \sqrt{1+ \left\lVert \nabla f\right\rVert^2 } \right\rVert} \left\lVert \varphi_x \land \varphi_y \right\rVert = \sqrt{1 + \left\lVert \nabla f\right\rVert^2 }\\
      N = \frac{\left(\frac{-\partial f}{\partial x} , \frac{-\partial f}{\partial y} , 1\right) }{\left\lVert \sqrt{1+ 4x^2 + 4y^2} \right\rVert} \left\lVert \varphi_x \land \varphi_y \right\rVert = \sqrt{1 + 4x^2 + 4y^2 }\\
    \end{gather*}
    Questo vale perchè $S$ è grafico di una funzione $f(x,y)$\\
    Ora scelgo il campo $F(x,y,z) = z, 0, x^2$ \tab \tiny (il campo è dato dall'esercizio è indipendente dalla funzione) \normalsize\\
    Quindi il flusso di $F$ attraverso $S$ è:
    \begin{gather*}
      S = \int_S \left\langle F(x,y,\underbrace{z(x,y)}_{x^2+y^2}) , N \right\rangle \ d\sigma \\
      = \int_D \left\langle \left(x^2+y^2, 0, x^2\right) , \frac{(-2x, -2y, 1)}{\sqrt{1 + 4x^2 + 4y^2}}  \right\rangle \sqrt{1 + 4x^2 + 4y^2} \ dx dy\\
      = \int_D \left( -(x^2+y^2)2x + x^2 \right) \ dxdy\\
      = \int_{-1}^1 dx \int_{-1}^1 \left( -2x^3 - 2xy^2 + x^2 \right) \ dy\\
      = \int_{-1}^1 \left( -2x^3 y - 2x \frac{y^3}{3} + x^2 y \right)_{y=-1}^{y=1} dx\\
      = \int_{-1}^1 \left( -4x^3 + \frac{4}{3} x + 2x^2 \right) dx\\
      = \frac{4}{3}
    \end{gather*}
\end{example}

\begin{theorem}[di Stokes]
  \underbar{Sia} $S$ una superficie orientabile di $\mathbb{R}^3$, \underbar{avente} un campo normale unitario $N$ \underbar{il cui} bordo consiste in un numero finito di curve chiuse con orientazione ereditata dall'orientazione di $S$.\\
  \underbar{Se} $F$ è un campo vettoriale definito su un insieme aperto contenente $S$ e $C^2$, e $T$ è 
  un versore tangente a $\delta^+ S$, orientato in modo concorde al verso positivo di $\delta^+ S$\\
  \underbar{allora}:
  \begin{gather*}
    \int_S \left\langle rot \ F, N \right\rangle d\sigma = \int_{\partial^+ S} \left\langle F, T \right\rangle ds  
  \end{gather*}
\end{theorem}

\begin{proof} \hfill\\
   Questa dim. la vedremo solo in un caso particolare, quando $S$ è il grafico di una funzione $z = f(x,y)$ con $(x,y) \in D$.\\
   Supponiamo come nell'esempio precedente che $N$ punti verso l'alto.\\
   \begin{center}
      \tdplotsetmaincoords{70}{110}
      \begin{tikzpicture}[tdplot_main_coords]
      %assi xyz
      \draw[->](0,0,0) -- (5,0,0) node[right]{$x$};
      \draw[->](0,0,0) -- (0,5,0) node[above]{$y$};
      \draw[->](0,0,0) -- (0,0,5) node[above]{$z$};
      %dominio
      \draw[ultra thick](1,1,0) -- (3,1,0) -- (3,2,0) -- (1,2,0) -- cycle;
      \node at(2,1.5,0){$D$};
      \draw[->] (1.2,0.8,0) -- (2.8,0.8,0);
      \draw[->] (3.2,1.2,0) -- (3.2,1.8,0);
      \draw[->] (2.8,2.2,0) -- (1.2,2.2,0);
      \draw[->] (0.8,1.8,0) -- (0.8,1.2,0);
      %superficie
      \begin{scope}[scale=2.5,  shift={(0,0.65,-0.5)}, rotate = 40]
        \draw[ultra thick](1.3,1.5,2) ..controls(1.6,1.7,2).. (2,1.8,2);
        \draw[<-](1.3,1.6,2) ..controls(1.6,1.8,2).. (2,1.9,2);

        \draw[ultra thick](1.5,1,2) ..controls(1.8,1.2,2).. (2.2,1.3,2);
        \draw[->](1.7,1,2) ..controls(2,1.2,2).. (2.4,1.3,2);
        
        \draw[ultra thick](1.3,1.5,2) ..controls(1.45,1.3,2).. (1.5,1,2);
        \draw[->](1.1,1.5,2) ..controls(1.25,1.3,2).. (1.3,1,2);

        \draw[ultra thick](2,1.8,2) ..controls(2.15,1.5,2).. (2.2,1.3,2);
        \draw[<-](2.2,1.9,2) ..controls(2.35,1.6,2).. (2.4,1.4,2);
      \end{scope}
      %punto
      \filldraw[white](3,1.5) circle (1.5pt) node[below]{$(x(t),y(t))$};
      \filldraw[black](3,1.5) circle (1pt);
      \draw[dashed](3,1.5,0) -- (3,1.5,3.05);
      \filldraw[white](3,1.5,3.05) circle (1.5pt) node[left]{$(x(t),y(t),f(x(t),y(t)))$};
      \filldraw[black](3,1.5,3.05) circle (1pt);
      \draw[->, very thick](2.2,1.9,3.2) -- (2.2,1.9,4.5)node[at end, right]{$N$};
      \node at(1.2,3.5,3.2){$z = f(x,y)$};
  \end{tikzpicture}
   \end{center}
   A $\delta D$ corrisponde $\delta S$, e all'orientazione positiva di $\delta D$ corrisponde l'orientazione positiva di $\delta S$\\
   \begin{gather*}
    \int_S \left\langle rot \ F , N \right\rangle d\sigma = \int_D \left\langle \underbrace{\left( \frac{\partial F_3}{\partial y} - \frac{-\partial F_2}{\partial z} , \frac{\partial F_1}{\partial z} \frac{\partial F_3}{\partial x} , \frac{\partial F_2}{\partial x} - \frac{\partial F_1}{\partial y} \right)}_{rot(F_1, F_2, F_3)} , \left( \frac{-\partial f}{\partial x}, \frac{-\partial f}{\partial y} , 1 \right) \right\rangle  \ dxdy\\
    = \int_D \left[ -\left(\frac{\partial F_3}{\partial y}(x,y,f(x,y))\right) \frac{\partial f}{\partial x}  - \left(\frac{\partial F_1}{\partial z}( \ "" \ )- \frac{\partial F_3}{\partial x} ( \ "" \ )\right)\frac{\partial f}{\partial y} + \left(\frac{\partial F_2}{\partial x}( \ "" \ )-\frac{\partial F_1}{\partial y}\right)  \right] \ dxdy 
   \end{gather*}
   Abbiamo fatto la parte sinistra ora facciamo la destra e verificheremo che sono uguali:
   \begin{gather*}
    \int_{\partial^+ S} \left\langle F, T \right\rangle ds
   \end{gather*}
   Supponiamo che $\partial^+ D$ consista di una sola curva chiusa.\\
   \underbar{Sia} $\gamma(t) = (x(t), y(t)) \quad t \in [a,b]$ una parametrizzazione di $\partial^+ D$  coerente con l'orientazione positiva.\\
   \underbar{Allora} una parametrizzazione di $\partial^+ S$ è:
   \begin{gather*}
    \widetilde{\gamma}(t) = (x(t), y(t), f(x(t), y(t))) \quad t \in [a,b]\\
    \int_a^b F(x(t), y(t), f(x(t), y(t))) \cdot \widetilde{\gamma}'(t) dt
   \end{gather*}
   Calcoliamo $\dot{\widetilde{\gamma}}$:
   \begin{gather*}
    \dot{\widetilde{\gamma}} = \frac{d}{dt} \widetilde{\gamma} (t) = \left( \dot{x}(t), \dot{y}(t), \frac{\partial f}{\partial x} (x(t), y(t)) \cdot \dot{x}(t) + \frac{\partial f}{\partial y} (x(t), y(t)) \cdot \dot{y}(t) \right)
   \end{gather*}
   Quindi l'integrale di linea diventa:
   \begin{gather*}
    \int_a^b F(x(t), y(t), f(x(t), y(t))) \cdot \left( \dot{x}(t), \dot{y}(t), \frac{\partial f}{\partial x} (x(t), y(t)) \cdot \dot{x}(t) + \frac{\partial f}{\partial y} (x(t), y(t)) \cdot \dot{y}(t) \right) dt\\
    = \int_a^b \left[ F_1( \ "" \ ) \dot{x}(t) + F_2( \ "" \ ) \dot{y}(t) + F_3( \ "" \ ) \left( \frac{\partial f}{\partial x} (x(t), y(t)) \cdot \dot{x}(t) + \frac{\partial f}{\partial y} (x(t), y(t)) \cdot \dot{y}(t) \right) \right] dt\\
   \end{gather*}
   Voglo raccogliere $\dot{x}$ e $\dot{y}$
   \begin{gather*}
      = \int_a^b \left( \left[ F_1( \ "" \ ) + F_3( \ "" \ )\frac{\partial f}{\partial x}(x,y) \right] \dot{x}(t) + \left[ F_2( \ "" \ ) + F_3( \ "" \ )\frac{\partial f}{\partial y}(x,y) \right] \dot{y}(t) \right) dt\\
      = \int_{\partial^+ D} \left( \left[ F_1(x,y,f(x,y)) + F_3(x,y,f(x,y))\frac{\partial f}{\partial x}(x,y) \right] dx + \left[ F_2(x,y,f(x,y)) + F_3(x,y,f(x,y))\frac{\partial f}{\partial y}(x,y) \right] dy \right)
   \end{gather*}
   Sono passato da un integrale su $t$ ad uno su $\partial^+ D$, a questo posso applicare le formule di Gauss-Green.\\
   Le formule di Gauss-Green dicono che tale integrale è:
   \begin{gather*}
    \int_D \frac{\partial}{\partial x}\left[ F_2(x,y,f(x,y)) + F_3(x,y,f(x,y)) \frac{\partial f}{\partial y}(x,y) \right]- \frac{\partial}{\partial y} \left[ F_1(x,y,f(x,y)) + F_3(x,y,f(x,y))\frac{\partial f}{\partial x} (x,y) \right] \ dxdy  
   \end{gather*}
   Devo derivare rispetto a $x$ e $y$:
   \begin{gather*}
    = \int_D \left[ \frac{\partial F_2}{\partial x} ( x,y,f(x,y) ) + \frac{\partial F_2}{\partial z} ( \ "" \ ) \frac{\partial f}{\partial x}(x,y) + \frac{\partial F_3}{\partial x}\frac{\partial f}{\partial y} + \frac{\partial F_3}{\partial z} ( \ "" \ ) \frac{\partial f}{\partial x}\frac{\partial f}{\partial y}(") + F_3( \ "" \ ) \frac{\partial^2 f}{\partial x \partial y} \right] - \\
    \left[ \frac{\partial F_1}{\partial y} ( \ "" \ ) + \frac{\partial F_1}{\partial z} ( \ "" \ ) \frac{\partial f}{\partial y} + \frac{\partial F_3}{\partial y} ( \ "" \ ) \frac{\partial f}{\partial x} + \frac{\partial F_3}{\partial z}\frac{\partial f}{\partial x}\frac{\partial f}{\partial y} + F_3( \ "" \ ) \frac{\partial^2 f}{\partial y \partial x} \right] \ dxdy
  \end{gather*}
  Dopo aver tolto i termini che si cancellano ottengo:
  \begin{gather*}
    \int_D \frac{\partial f}{\partial x} (x,y) \left[ \frac{\partial F_2}{\partial z} - \frac{\partial F_3}{\partial y} \right] + \frac{\partial f}{\partial y}\left[ \frac{\partial F_3}{\partial x} -\frac{\partial F_1}{\partial z}\right] + \left[ \frac{\partial F_2}{\partial x}( " )- \frac{\partial F_1}{\partial y} \right] \ dxdy   
  \end{gather*}
\end{proof}


\begin{example}
  Calcolare $\int_{\gamma^+} \left\langle F, T\right\rangle \ ds$ dove $F(x,y,z) = (-y^3, x^3, -z^3)$ e 
  $\gamma^+$ è la curva di sezione del cilindro $x^2+y^2=1$ e del piano $z = 3-2x-2y$\\
  Voglio utilizzare il Teorema di Stokes.\\
  $\gamma^+$ è il bordo della superficie $(x,y) \in \{ x^2+y^2 \leq 1 \} \to (x,y,3-2x-2y)$\\
  Scelgo su $S$ l'orientazione in modo che $\gamma^+$ sia il bordo di $S$ orientato positivamente.
  Questo succede se $N$ punta verso l'alto. Quindi:
  \begin{gather*}
    rot \ F = (0,0, 3(x^2+y^2))\\
    N = \frac{(-2,-2,1)}{\left\lVert (-2,-2,1) \right\rVert }
  \end{gather*}
  Quindi l'integrale iniziale diventa:
  \begin{gather*}
    = \int_S \left\langle rot F , N\right\rangle d\sigma\\
    = \int_{\{x^2+y^2 \leq 1\}} 3(x^2+y^2) \ dxdy = \int_0^{2\pi} d\theta \int_0^1 3r^2 \cdot r \ dr\\
    = 2\pi \left[ \frac{3r^4}{4} \right]_0^1 = \frac{3\pi}{2}
  \end{gather*}
\end{example}

%Da mettere recupero del 10/12/25

\newpage
\section{11/12/25}
In quest alezione parleremo di successioni di funzioni.\\
Per prima cosa consideriamo uno spazio metrico particolare, lo spazio delle funzioni reali continue su un intervallo:
\begin{gather*}
  \mathcal{F}_i = \{ g: I \to \mathbb{R} \} \quad I \subseteq \mathbb{R}\\
  \{f_k (x)\} \subseteq \mathcal{F}_i
\end{gather*} 

\begin{definition}[Convergenza puntuale e uniforme]
  Una successione converge puntualmente a $f \in \mathcal{F}_i$ se:
  \begin{gather*}
    f_k \to f \quad \text{in} \quad (\mathcal{F}_i ; | \cdot |) \quad \text{cioè} \quad \lim_{k \to \infty} |f_k(x) - f(x)| = 0 \quad \forall x \in I\\
  \end{gather*}
  Una successione converge uniformemente a $f \in \mathcal{F}_i$ se:
  \begin{gather*}
    f_k \rightrightarrows f \quad (\mathcal{F}_i ; \left\lVert \cdot \right\rVert ) \quad \text{cioè} \quad \lim_{k \to \infty} \underset{x \in I}{sup}\left\lvert f_k(x) - f(x) \right\rvert = 0
  \end{gather*}
\end{definition}

\begin{observation}
  La convergenza uniforme implica la convergenza puntuale.\\
  \begin{gather*}
    f_k \rightrightarrows  f \quad \Rightarrow f_k \overset{I}{\to} f
  \end{gather*}
  Non è vero il viceversa
\end{observation}

\begin{example}
  \begin{multicols}{2}
    \noindent
    \begin{center}
      \begin{tikzpicture}
        \draw[->](0,0) -- (4,0) node[at end, below]{$x$};
        \draw[->](0,0) -- (0,2) node[at end, left]{$y$};
        \draw(0,1.5) -- (2,1.5) node[at end, above]{$y = f_k(x)$};
        \draw[dashed](2,0) -- (2,1.5) node[at start, below]{$\frac{1}{k}$};
        \filldraw[white](2,1.5) circle (2pt);
        \draw(0,0.8) -- (3,0.8) node[at end, below]{$y = f(x)$};
        \draw[dashed](3,0) -- (3,0.8) node[at start, below]{$1$};
        \filldraw[white](3,0.8) circle (2pt);
        \node at(0,1.5)[left]{$a$};
        \node at(0,0.8)[left]{$b$};
        \draw(2.5,-0.1) -- (2.5,0.1) node[at start, below]{$x$};
      \end{tikzpicture}
    \end{center}
    \columnbreak
    \hfill\\\hfill\\
    \begin{gather*}
    f_k (x) = \begin{cases}
      a \quad \text{se } 0 < x \leq \frac{1}{k}\\
      b \quad \text{se } x = 0 \lor \frac{1}{k} < x \leq 1
    \end{cases}\\
    a \neq b \qquad I = [0,1]\\
    f_k (x) \to f(x) \equiv b \quad \forall x \in I\\
  \end{gather*}
  \end{multicols}
  infatti se $x=0$ \cmark \\
  se $x \neq 0$ esiste $\overline{k}$ tale che $x > \frac{1}{\overline{k}}$, quindi per ogni $k > \overline{k}$ vale $f_k(x) = b$ \cmark \\
  \textbf{Ma non converge uniformemente}:
  \begin{gather*}
    f_k \cancel{\rightrightarrows} f
  \end{gather*}
  Infatti:
  \begin{gather*}
    \underset{x \in [0,1]}{sup} \left\lvert f_k(x) - f(x) \right\rvert = \left\lvert b-a \right\rvert\\
    \lim_{k \to \infty} \underset{x \in [0,1]}{sup} \left\lvert f_k(x) - f(x) \right\rvert = \lim_{k\to \infty} \left\lvert b-a \right\rvert  = |b-a| \neq 0  
  \end{gather*}
\end{example}

\begin{theorem}[continuità del limite]
  Preso un intervallo $I = [a,b]$\\
  \underbar{se} $\{f_k\} \in C^0 (I)$ e $f_k \in I$\\
  \underbar{allora} $f \in C^0 (I)$
\end{theorem}
\begin{proof}
  fissiamo $x_0 \in [a,b]$\\
  fisso inoltre $\varepsilon > 0$\\
  Per valutare la continuità devo valutare:
  \begin{gather*}
    \left\lvert f(x) - f(x_0) \right\rvert \leq \left\lvert f(x) - f_{k_0}(x_0)\right\rvert + \left\lvert f_{k_0}(x) - f_{k_0}(x_0) \right\rvert + \left\lvert f_{k_0}(x_0) - f(x_0) \right\rvert   \\
    f_k \rightrightarrows f \quad \Rightarrow \exists \nu : \forall k_0 > \nu \quad \left\lvert f_k(x) - f(x) \right\rvert < \varepsilon \quad \forall x \in I\\
  \end{gather*}
  Ora sappiamo che:
  \begin{gather*}
    \left\lvert f(x) - f_{k_0}(x_0)\right\rvert < \varepsilon \\
    \left\lvert f_{k_0}(x_0) - f(x_0) \right\rvert < \varepsilon \\
  \end{gather*}
  Devo stimare:
  \begin{gather*}
    \left\lvert f_{k_0}(x) - f_{k_0}(x_0) \right\rvert
  \end{gather*}
  \begin{observation}
      $f_{k_0} \in C^0 (I) \quad \Rightarrow \exists \delta > 0 :$ \underbar{ se } $x \in I \text{ con } |x-x_0| < \delta$\\
      \underbar{allora} $| f_{k_0}(x) - f_{k_0}(x_0) | < \varepsilon$
  \end{observation}
  Quindi:
  \begin{gather*}
    |f_{k_0}(x) - f_{k_0}(x_0)| < \varepsilon
  \end{gather*}
  se $x \in I $ con $|x-x_0| < \delta$\\
\end{proof}

\begin{theorem}[criterio di cauchy per la convergenza uniforme]
  \begin{gather*}
    f_k \rightrightarrows f \in I \Leftrightarrow \forall \varepsilon > 0 \quad \exists \nu \in \mathbb{N} : \forall h,k > \nu \quad \left\lvert f_k(x) - f_h(x) \right\rvert < \varepsilon \quad \forall x \in I
  \end{gather*}
\end{theorem}

  \begin{observation}
    \begin{gather*}
      f_k(x) = x^k \cancel{\rightrightarrows} f \in [0,1]\\
      f(x) = \begin{cases}
        0 \quad x \in [0,1)\\
        1 \quad x = 1
      \end{cases}\\
    \end{gather*}
    Non è preservata la continuità quindi \textbf{non} c'è convergenza uniforme
  \end{observation}
  Questo ci dice che questo teorema si usa per dimostrare che non c'è convergenza uniforme.

\begin{theorem}[Passaggio sotto $\int$]
  \begin{gather*}
    I = [a,b] \quad \{ f_k \} \subseteq C^0 (I), f_k \rightrightarrows f \in I\\
    \Rightarrow \lim_{k \to \infty} \int_a^b f_k(x) \ dx = \int_a^b \lim_{k \to \infty} f_k(x) \ dx
  \end{gather*}
\end{theorem}

\begin{proof}
  $f \in C^0 (I)$ per th. continuità del limite $\Rightarrow \ \exists \int_a^b f(x) \ dx$ \\
\end{proof}
Si usa adesso la lineraità dell'integrale e porto dentro il valore assoluto maggiorandolo.
\begin{gather*}
  \left\lvert \int_a^b f_k(x) \ dx - \int_a^b f(x) \ dx \right\rvert  \leq \int_a^b \underbrace{\left\lvert f_k(x) - f(x) \right\rvert}_{\leq \underset{x \in [a,b]}{max} \left\lvert f_k(x) - f(x) \right\rvert = \left\lVert f_k - f \right\rVert_\infty  } \ dx \leq (b-a) \cdot \left\lVert f_k - f \right\rVert_\infty \overbrace{\to 0 }_{\text{poichè } f_k \rightrightarrows f \in [a,b]}\\
\end{gather*}
\begin{observation}
  non vale il th. se la convergenza è puntuale
  \begin{gather*}
    \text{se } \cancel{\rightrightarrows }\\
    \text{ma solo } \to 
  \end{gather*}
\end{observation}

\begin{example}
  \begin{gather*}
    f_k(x) = k x e^{-kx^2} \quad \forall x \in [0,1]\\
    f_k(x) \to f(x) \equiv 0 \quad \forall x \in [0, 1)\\
    f_k \rightrightarrows f \in [0,1]\\
  \end{gather*}
  Infatti $\underset{x \in [0,1]}{max} \left\lvert f_k(x) - \cancel{f(x)} \right\rvert  \overset{\text{da studio di funz.}}{ = } f_k (\frac{1}{\sqrt{2k}}) = \frac{k}{\sqrt{2k}} e^{-\frac{k}{2k}} \overset{f_k \text{assume max in } x_k = \frac{1}{\sqrt{2k}}}{\cancel{\to}} 0$
  \begin{gather*}
    \int_0^1 f_k(x) \ dx = \int_0^1 kx e^{-kx^2} \ dx = \frac{-1}{2} e^{-kx^2} \Big|_0^1 = \frac{1}{2} [1 - e^{-k}] \to \frac{1}{2}\\
    \int_0^1 f(x) \ dx = 0
  \end{gather*}
\end{example}

\begin{theorem}[Passaggio al limite sotto ]
  Preso un intervallo $I = [a,b]$\\
  \underbar{Sia} $\{f_k\} \in C^0 (I)$\\
  \underbar{Se} $\exists x_0 \in I $ t.c. $ f_k(x_0) \to \ell \in \mathbb{R}$\\
  \underbar{e} $f'_k \rightrightarrows g \in I$\\
  \underbar{allora}
  \begin{gather*}
    \exists f \in C^1 (I) : f_k \rightrightarrows f \in I\\
    \text{e } f'_k \to f' \quad \in I\\
  \end{gather*}
\end{theorem}

\begin{observation}
  $f_k$ derivabile, $f_k \overset{I}{\rightrightarrows} f$ \textbf{non} implica che $f$ sia derivabile.\\
\end{observation}

\begin{example}
  \begin{gather*}
    f_k (x) = \sqrt{x^2 + \frac{1}{k}} \quad x \in \mathbb{R} \quad x \in [-1,1]\\
    \{ f_k \} \subseteq C^1(I) , f_k \rightrightarrows f(x) = |x| \cancel{\in} C^1 \quad I = [-1,1]\\
  \end{gather*}
\end{example}

\begin{example}
  \begin{gather*}
    f_k(x) = \frac{1}{k} \sin (kx) \quad x \in [a,b] \subseteq \mathbb{R}\\
    f_k(x) \to f(x) \equiv 0\\
    f_k \rightrightarrows f \in \mathbb{R}\\
    f'_k(x) = \cos(kx) \cancel{\to} g \quad \text{non converge puntualmente in } \mathbb{R}\\
    f'_k (x) \cancel{\to} 0 = f'(x) \quad \text{perchè non converge}
  \end{gather*}
\end{example}

\begin{theorem}[Del DINI della convergenza monotona]
  Preso un intervallo $I = [a,b]$\\
  \underbar{Sia} $\{f_k\} \subseteq C^0 (I)$ t.c. monotona rispetto a $k$ (cioè $f_{k+1} \overset{o <}{>} f_k(x) \quad \forall x \in I$)\\
  \underbar{e} $f_k(x) \to f(x) \in C^0 (I)$\\
  \underbar{allora} $f_k \rightrightarrows f \in I$
\end{theorem}

\begin{example}
  \begin{gather*}
    f_k (x) = x^k \quad x \in [0,\frac{1}{2}]\\
  \end{gather*}

  \begin{observation}
    \begin{gather*}
      f_k(x) \to f(x) \equiv 0 \quad \forall x \in I\\
    \end{gather*}
  \end{observation}
  Continuando con l'esercizio:
  \begin{gather*}
    f_{k+1}(x) - f_k(x) = x^{k+1} - x^k = \underbrace{x^k}_{\geq 0} \underbrace{(x-1)}_{< 0} \leq 0 \quad \text{cioè } f_{k+1}(x) \leq f_k(x) \quad \forall x \in I\\
    \Rightarrow f_k \rightrightarrows f \in I
  \end{gather*}
\end{example}

\hfill\\\hfill\\
\textbf{Strategie} per studiare la convergenza di $\{f_k\}$:\\
Quindi $f_k \rightrightarrows f$ in $I$:
\begin{itemize}
  \item usare la definizione calcolando $\underset{x \in [a,b]}{max} o \underset{x \in (a,b)}{sup}$ di $g_k(x) = |f_k(x) - f(x)|$
  \item Usare la convergenza monotona
\end{itemize}
Più utile può essere vedere quando NON c'è convergenza uniforme:\\
Quindi $f_k \cancel{\rightrightarrows} f$ in $I$:
\begin{itemize}
  \item $\{f_k\} \subseteq C^0, f \cancel{\in} C^0$ perchè contraddirebbe il th. della continuità del limite
  \item Usare la definizione (cerco quindi max e min di $g_k(x) = |f_k(x) - f(x)|$ e vedo che il limite per $k \to \infty$ non è 0)
  \item cerco  $\{x_k\} \subseteq I$ t.c. $|f_k(x_k) - f(x_k)| \cancel{\to } 0$
\end{itemize}

\begin{example}
  $f_n(x) = \frac{nx}{1+n^2x^2}$\\
  \begin{gather*}
    f_n(x) \to f(x) \equiv 0 \quad \forall x \in [-1,1]\\
    \begin{cases}
      \text{se } x = 0 \quad f_n(0) = 0 \\
      \text{se } x \neq 0 \quad f_n(x) = \frac{nx}{1+n^2x^2} = \frac{\frac{1}{nx}}{\underbrace{1+\frac{1}{n^2x^2}}_{\to 1}} \to 0
    \end{cases}
  \end{gather*}
  Guardo adesso se c'è convergenza uniforme $f_n \overset{?}{\rightrightarrows} f \in [-1,1]$
  \begin{gather*}
    \underset{x \in (-1,1)}{sup} |\underbrace{f_k(x) - f(x)}_{g_k(x)}| = \underset{x \in [-1,1]}{max} \left\lvert \frac{kx}{1+k^2x^2} \right\rvert\\
    g_k(x) = \frac{kx}{1+k^2x^2} \quad x \in [0,1] \quad \text{(funz. pari)}\\
    g'_k(x) = \frac{k(1+k^2x^2) - kx(2k^2x)}{(1+k^2x^2)^2} = \frac{k - k^3x^2}{(1+k^2x^2)^2} = 0 \quad \Rightarrow x = \frac{1}{k}\\
    g_k \text{ cresce in } \left[0, \frac{1}{k}\right] \text{ e decresce in } \left[\frac{1}{k}, 1\right]\\
  \end{gather*}
  Guardo adesso $f_n \overset{?}{\rightrightarrows} f \in \mathcal{J} \subseteq [-1,1]$\\
  In più so che $k$ è grande quindi nell'intervallo $[0,1]$ $\frac{1}{k}$ è vicino a $0$, prendo quindi un sottointervallo $[0,a]$ vicino a $0$ che contiene $\frac{1}{k}$
  \begin{gather*}
    \mathcal{J} = [a,1] \quad a > 0\\
    \mathcal{J} = [a,1] \cup [-1,-a] \quad a>0\\
  \end{gather*}
  considero:
  \begin{gather*}
    \underset{x \in [a,1]}{max} \left\lvert f_k(x) - f(x) \right\rvert = \underset{x \in [a,1]}{max} \left\lvert \frac{kx}{1+k^2x^2} \right\rvert = \overbrace{\frac{ka}{1+k^2a^2}}_{\to 0} = g_k(a) \quad a \neq 0\\ 
  \end{gather*}
  Potevo in alternativa osservare che:
  \begin{gather*}
    f_k \rightrightarrows f \in \begin{cases}
      \mathcal{J}_1 = [a,1] \text{ poichè } \{f_k\} \searrow \\
      \mathcal{J}_2 = [-1,-a] \text{ poichè } \{f_k\} \nearrow
    \end{cases}\\
    f_{k+1} - f_k
  \end{gather*}
\end{example}

(altri esercizi / esempi su foto)
    

\newpage
\section{15/12/25} 
\subsection{recupero della lezione 10/12/25}

\begin{theorem}[Teorema di Stokes (richiamo)]
  \underbar{se} $S$ è una superficie regolare orientabile\\
  $\partial^+ S$ indica il suo bordo orientato in modo concorde alla scelta alla sceta del versore normale $N$\\
  e $F$ è un campo vettoriale $C^1$ definito in un intorno di $S$\\
  \underbar{allora} si ha che:
  \begin{gather*}
    \int_s \left\langle rot F , N \right\rangle  d\sigma = \int_{\partial^+ S} \left\langle F , T \right\rangle ds
  \end{gather*}
  Con $T$ versore tangente a $\partial S$ concorde cn l'orientazione
\end{theorem}
\begin{theorem}
  [della divergenza (richiamo)]
  \underbar{Sia} $D \subseteq \mathbb{R}^3$ un aperto limitato la cui frontiera è una superficie regolare (o regolare a tratti) chiusa orientabile.\\
  \underbar{Sia} in ogni punto di $\partial D$ sia il versore normale $N$ orientato in modo da puntare verso l'esterno di $D$.\\
  \underbar{Sia} $F$ un campo definito e $C^1$ in un intorno di $D$.\\
  \underbar{Allora}:
  \begin{gather*}
    \int_D  \text{div}F \ dxdydz = \int_{\partial D} \left\langle F , N \right\rangle d\sigma
  \end{gather*} 
\end{theorem}
\hfill\\
In generale si ha che:\\
Per Stokes quindi si ha che un integrale di superficie è uguale alll'integrale del bordo di quella superficie (una curva)\\
Per il th. della divergenza l'integrale di volume è uguale ad un integrale del bordo di quel volume (una superficie).\\
Questo si rivede un po' sul th. fondamentale del calcolo dove si ha che l'integrale di una funzione su un intervallo è uguale alla differenza dei valori della primitiva agli estremi dell'intervallo (che sono i bordi dell'intervallo stesso).\\
\hfil\\
\textbf{Esempi} a cui si applica il th. della divergenza:
\begin{itemize}
  \item sfere
  \item semisfere
  \item cubi (regolare a pezzi che sarebbero le varie facice del cubo)
  \item cilindri (si intende un cilindro solido, pieno)
  \item la regione compresa tra due sfere concentriche
\end{itemize}
Non si applica a:
\begin{itemize}
  \item la sfera privata di un punto (non è regolare a pezzi)
\end{itemize}
\hfill\\
\begin{example}
  Prendiamo come dominio un cilindro di raggio 1 e altezza 1:
  \begin{gather*}
    D = \{(x,y,z) \in \mathbb{R}^3: \ x^2 + y^2 \leq 1 \ , \ 0 \leq z \leq 1 \}
  \end{gather*}
  \begin{center}
    %cilindro
    \begin{tikzpicture}[scale=2]
      %assi
      \draw[->] (-1.5,0) -- (1.5,0) node[right] {$x$};
      \draw[->] (0,0) -- (0,2.5) node[above] {$z$};
      \draw[->] (0,0) -- (-1,-1) node[above left] {$y$};
      % Disegna la base inferiore
      \draw[fill=blue!20, opacity=0.5] (0,0) ellipse (1cm and 0.3cm);
      % Disegna il corpo del cilindro
      \draw[fill=blue!20, opacity=0.5] (-1,0) -- (-1,2) arc (180:360:1cm and 0.3cm) -- (1,0) -- cycle;
      % Disegna la base superiore
      \draw[fill=blue!20, opacity=0.5] (0,2) ellipse (1cm and 0.3cm);
      % Disegna le linee di contorno
      \draw (-1,0) -- (-1,2);
      \draw (1,0) -- (1,2);
      \draw[dashed] (0,0) ellipse (1cm and 0.3cm);
    \end{tikzpicture}
  \end{center}
  Calcolare l'integrale di superficie:
  \begin{gather*}
    F = \left( 1 - (x^2+y^2)^3 , 2(1-(x^2+y^2)^3) , x^2z^2 \right)\\
    \int_D \text{div}F \ dxdydz \qquad \boxed{*}
  \end{gather*}
  Calcoliamo div$F$:
  \begin{gather*}
    \frac{\partial }{\partial x} \left( 1- (x^2+y^2)^3 \right) + \frac{\partial }{\partial y} \left( 2(1-(x^2+y^2)^3) \right) + \frac{\partial }{\partial z} \left( x^2z^2 \right) = \\
    -6x(x^2+y^2)^2 - 12y(x^2+y^2)^2 + 2xz \\
    \end{gather*}
    Per calcolare l'integrale usiamo coordinate cilindriche:
    \begin{gather*}
      D \to D' = \{(\rho,\theta,z): \ 0 \leq \rho \leq 1 \ , \ 0 \leq \theta \leq 2\pi \ , \ 0 \leq z \leq 1 \} \\
      \int_0^{2\pi} d\theta \int_0^1 \rho \ d\rho \int_0^1 \left( -6\rho \cos\theta \rho^4 - 12 \rho \sin\theta \rho^4 + 2 (\rho\cos(\theta))^2 z \right) dz = \ \dots \ = \frac{\pi}{4}\\
    \end{gather*}
    Calcoliamo ora l'integrale \fbox{*} usando il th. della divergenza:
    \begin{gather*}
      \int_{\partial D} \left\langle F , N \right\rangle d\sigma = \int_\text{base} + \int_\text{coperchio} + \int_\text{sup. laterale} \\
    \end{gather*}
    Partiamo dalla base:
    \begin{gather*}
      \int_\text{base} \left\langle F , N \right\rangle d\sigma = \int_\text{base} \left\langle F(x,y,0) , (0,0,-1) \right\rangle d\sigma = \int_\text{base} \left\langle F((1-(x^2+y^2)^3),(2(1-(x^2+y^2)^3)),0) , (0,0,-1) \right\rangle d\sigma\\
    \end{gather*}
    Di fatto questo è il grafico di $z \equiv 0$ con $(x,y) \in \{x^2+y^2 \leq 1\}$\\
    ad es. se\\
    \begin{gather*}
      z= f(x,y) \ (x,y)\in \mathbb{R}\\
      N = \frac{-\frac{\partial f}{\partial x}, \ -\frac{\partial f}{\partial y}, \ 1}{\sqrt{1 + |\nabla f|^2}}\\
      \left\lVert  \right\rVert = \sqrt{1 + |\nabla f|^2}
    \end{gather*}
    Questa è una formula generale per il calcolo del versore normale ad un grafico.\\
    Guardiamo adesso cosa accade con il coperchio, e sarà definito da $(x,y) \in B(0,1)$ con b che è la base del cilindro e $z=1$:
    \begin{gather*}
      \int_\text{coperchio} \left\langle F , N \right\rangle d\sigma = \int_\text{coperchio} \left\langle F(x,y,1) , (0,0,1) \right\rangle d\sigma = \int_\text{coperchio} \underbrace{\left\langle F((1-(x^2+y^2)^3),(2(1-(x^2+y^2)^3)),x^2) , (0,0,1) \right\rangle}_{x^2} d\sigma\\
      \int_\text{coperchio} x^2 \ d\sigma = \int_{(x,y): x^2+y^2 \leq 1} x^2 \sqrt{1+0} \ dxdy 
    \end{gather*}
    Passiamo di nuovo in coordinate polari:
    \begin{gather*}
      = \int_0^{2\pi} \cos^2(\theta) d\theta \int_0^1 \rho^3  \ d\rho = \frac{\pi}{4}
    \end{gather*}
    Adesso tocca alla superficie laterale:
    \begin{gather*}
      \int_\text{superficie laterale} \left\langle F , N \right\rangle d\sigma = \int_\text{superficie laterale} \underbrace{\left\langle (0,0,x^2z) , (N_x, N_y, 0) \right\rangle}_{0} d\sigma = 0
    \end{gather*}
    Non so quanto sono $N_y$ e $N_x$ ma so che in ogni caso il prodotto scalare sarà 0.
\end{example}
\hfil\\
Il th. della divergenza espresso con le varie componenti afferma che:
\begin{gather*}
  \int_D \frac{\partial F_1}{\partial x} + \frac{\partial F_2}{\partial y} + \frac{\partial F_3}{\partial z} \ dxdydz = \int_{\partial D} F_1 N_1 + F_2 N_2 + F_3 N_3 \ d\sigma
\end{gather*}
Con $N_1,N_2,N_3$ componenti del versore normale esterna.
\begin{proposition}
  \underbar{se} $f$ è una funzione (scalare) def e $C^1$ in un intorno di $D$\\
  \underbar{allora}:
  \begin{gather*}
    \int_D \frac{\partial f}{\partial x} \ dxdydz = \int_{\partial D} f N_1 \ d\sigma \\
    \int_D \frac{\partial f}{\partial y} \ dxdydz = \int_{\partial D} f N_2 \ d\sigma \\
    \int_D \frac{\partial f}{\partial z} \ dxdydz = \int_{\partial D} f N_3 \ d\sigma
  \end{gather*}
\end{proposition}
\hfill\\
Come si può aggirare la presenza di un punto interno a $D$ in cui il campo non è definito o non è $C^1$?\\
Prendiamo come esempio il campo $F$:
\begin{gather*}
  F(x,y,z) = \frac{(x,y,z)}{\left\lVert (x,y,z) \right\rVert^3}
\end{gather*}
\begin{center}
  \begin{tikzpicture}
    \draw(0,0) circle (1);
    \filldraw[fill = black] (0.3,-0.5) circle (2pt) node[below]{$(0,0,0)$};
    \draw[->] (0.7,0.7) -- (1.2,1.2) node[right]{$N$};
    \node at(1.1,0) [above]{$D$};
  \end{tikzpicture}
\end{center}
Calcoliamo:
\begin{gather*}
  \int_{\partial D} \left\langle F , N \right\rangle d\sigma
\end{gather*}
La prima componente sarà:
\begin{gather*}
  \text{div}F = 0\\
  \frac{\partial}{\partial x} \frac{x}{(x^2+y^2+z^2)^{\frac{3}{2}}} = \frac{1(x^2+y^2+z^2)^{\frac{3}{2}} - x\frac{3}{\cancel{2}} (x^2+y^2+z^2)^{\frac{1}{2}} \cancel{2}x}{(x^2+y^2+z^2)^3} = \frac{(x^2+y^2+z^2)^\frac{3}{2} - 3x^2 (x^2+y^2+z^2)^\frac{1}{2} }{(x^2+y^2+z^2)^3}\\
\end{gather*}
La divergenza è la somma delle tre derivate parziali, in ogni caso le componenti per $y$ e $z$ vengono in maniera analoga:
\begin{gather*}
  \frac{\partial}{\partial y} (\frac{y}{"})\\
  \frac{\partial}{\partial z} (\frac{z}{"})\\
  \text{div}F = \frac{3(x^2+y^2+z^2)^{\frac{3}{2}} - 3(x^2+y^2+z^2)(x^2+y^2+z^2)^{\frac{1}{2}}}{(x^2+y^2+z^2)^3} = 0
\end{gather*}
Adesso si fa un passaggio furbo, si considera una sfera di raggio $\varepsilon$ centrata nell'origine (punto in cui $F$ non è definito) e si definisce:
\begin{center}
  \begin{tikzpicture}
    \draw(0,0) circle (1);
    \filldraw[fill = black] (0.3,-0.5) circle (2pt);
    \draw(0.3,-0.5) circle (0.4);
    \node at(0.4,-0.8)[below right]{$B((0,0,0), \varepsilon) = B_\varepsilon$};
    \draw[->] (0.7,0.7) -- (1.2,1.2) node[right]{$N$};
    \node at(1.1,0) [above]{$D$};
  \end{tikzpicture}
\end{center}
Quindi ora posso dire:
\begin{gather*}
  \underbrace{\int_{D \backslash B_\varepsilon} \text{div}F \ dxdydz}_{0} = \int_{\partial (D \backslash B_\varepsilon)} \left\langle F , N \right\rangle d\sigma
\end{gather*}
Quel campo viene $0$ dai calcoli precedenti dove abbiamo visto che veniva un campo a divergenza nulla.\\
Continuando:
\begin{gather*}
  0 = \int_{\partial (D \backslash B_\varepsilon)} \left\langle F , N \right\rangle d\sigma\\
  = \int_{\partial D} \left\langle F , N \right\rangle d\sigma + \int_{\partial B_\varepsilon} \left\langle F , N \right\rangle d\sigma
\end{gather*}
Ora consideriamo che:
\begin{gather*}
  \int_{\partial B_\varepsilon} \left\langle F , N \right\rangle d\sigma = \int_{\partial B_\varepsilon} \left\langle \frac{(x,y,z)}{\left\lVert (x,y,z) \right\rVert^3} , \frac{-(x,y,z)}{\left\lVert (x,y,z) \right\rVert } \right\rangle d\sigma = \int_{\partial B_\varepsilon} \frac{-\left\lVert (x,y,z) \right\rVert^2}{\left\lVert (x,y,z) \right\rVert^4} d\sigma = \int_{\partial B_\varepsilon} \frac{-1}{\underbrace{\left\lVert (x,y,z) \right\rVert^2}_{\varepsilon^2}} d\sigma\\
  = \frac{-1}{\varepsilon^2} \int_{\partial B_\varepsilon} d\sigma = \frac{-1}{\varepsilon^2} \underbrace{4 \pi \varepsilon^2}_{\text{area di} \partial B_\varepsilon} = -4\pi
\end{gather*}
Abbiamo quindi dimostrato:
\begin{gather*}
  \int_{\partial D} \left\langle F , N \right\rangle d\sigma = 4\pi
\end{gather*}



\newpage
\section{16/12/25}
\subsection{Serie di funzioni}
\begin{definition}
  \begin{gather*}
    \{f_k\} \subseteq \mathcal{F}_I \quad S_n(x) = \sum_{k=0}^n f_k(x) \in \mathcal{F}_I\\
    \sum_0^{+\infty} f_k(x) \text{ serie } = \lim_{n \to +\infty} S_n(x) \quad (\mathcal{F}_I; \text{norma}) \\
  \end{gather*}
\end{definition}

\begin{definition}
  $\sum_0^{+\infty} f_k$ converge puntualmente a $S(x)$ \underbar{se} $S_n(x) \to S(x)$ puntualmente in $I$\\
  $\sum_0^{+\infty} f_k$ converge assolutamente a $S(x)$ \underbar{se} $\sum_0^n |f_k(x)| \to S(x)$ puntualmente\\
  $\sum_0^{+\infty} f_k$ converge uniformemente a $S(x)$ \underbar{se} $S_n(x) \to S(x)$ uniformemente in $I$\\rie
  $\sum_0^{+\infty} f_k$ converge totalmente a $S(x)$ \underbar{se} $\exists \{M_n\} \subseteq \mathbb{R} \ t.c. \ |f_n(x)| \leq M_n \forall x \in I$ e $\sum_0^{+\infty} M_n$ converge\\
  quest'ultima vuol dire anche che $\sum_{n=0}^{+\infty} \underset{x \in I}{sup} |f_n(x)|$ converge.
\end{definition}

\begin{theorem}
  $\sum f_k$ converge totalmente $ \overset{\in I}{\implies}$ converge uniformemente in $I$ $\overset{\in I}{\implies}$ converge puntualmente in $I$.
\end{theorem}

\begin{theorem}
  \begin{itemize}
    \hfil\\
    \item continuità della somma.
    \begin{gather*}
      \{f_k\} \subseteq C^0(I) \ , \ \sum f_k(x) \rightrightarrows S(x) \in I \implies S(x) \in C^0(I)
    \end{gather*}
    \item Integrazione per Serie
    \begin{gather*}
      I = [a,b] \{f_k\} \subseteq C^0(I) \ , \ \sum f_k(x) \overset{I}{\rightrightarrows} S(x) \in I \implies \int_a^b S(x) dx = \sum_0^{+\infty} \int_a^b f_k(x) dx
    \end{gather*}
    \item Derivazione per Serie
    \begin{gather*}
      I = [a,b] \{f_k\} \subseteq C^1(I) \ , \ \sum f_k(x) \overset{I}{\rightrightarrows} S(x) \in I \ , \ \sum f_k'(x) \rightrightarrows G(x) \in I \implies S(x) \in C^1(I) \ , \ S'(x) = \sum f_k'(x) \quad \forall x \in I
    \end{gather*}
  \end{itemize}
\end{theorem}

\begin{example}
  \begin{gather*}
    \sum_0^{+\infty} 2^n \sin(\frac{x}{3^n})
  \end{gather*}

  \begin{enumerate}[a)]
    \item studiare la vonc. puntuale e totale
    \item detta $S(x)$ la somma della serie, calcolare $\lim_{x \to 0} \frac{S(x)}{x}$
  \end{enumerate}
  Punto a)
  \begin{gather*}
    f_n(x) = 2^n \sin(\frac{x}{3^n}) = \frac{2^n}{3^n} x \frac{\sin(\frac{x}{3^n})}{\frac{x}{3^n}}    
  \end{gather*}
  Per x fissato ha segno (definitivamente) costante (quindi posso applicare il th del confronto)
  \begin{gather*}
    \sum f_n \text{ha los tesso carattere di } \sum (\frac{2}{3})^n \text{ che converge} \\
    \implies \sum f_n(x) \text{ converge puntualmente } \forall x \in \mathbb{R}
  \end{gather*}
  Voglio studiare la convergenza totale 
  \begin{gather*}
    \underset{x \in I}{sup} |f_n(x)| = \underset{x \in I}{sup} 2^n |\sin(\frac{x}{3^n})| \leq 2^n
  \end{gather*}
  Così non fuonziona dobbiamo trovare una magiorazione meno grossolana
  \begin{gather*}
    \text{considero } I = [a,b] \text{ invece che come prima } I = \mathbb{R}\\
    \underbrace{\leq}_{|\sin(t)| \leq |t|} 2^n \underset{x \in I}{sup} \frac{|x|}{3^n} ? (\frac{2}{3})^n \underset{x \in I}{sup} |x| \\
    = (\frac{2}{3})^n \quad max\{|a|, |b|\} := M_n \\
    \sum M_n \text{ converge } \implies \sum f_n \text{ converge totalmente in } [a,b] \ \forall a,b \in \mathbb{R}
  \end{gather*}
  Per provare che non c'è convergenza totale cerco $\{x_n\} \subseteq \mathbb{R}$ $t.c.$ $\sum f_n(x_n)$ non converge\\
  \textbf{es.} $\{x_n\} = \{3^n\} \subseteq\mathbb{R} \quad f_n(3^n) = 2^n$ con $\sum 2^n$ diverge 
  \hfil\\
  Punto b)
  \begin{observation}
    \begin{gather*}
      \lim_{x \to 0} S(x) \overset{?}{=} S(0) = \sum_{n=0}^{+\infty} 2^n \sin(\frac{0}{3^n}) = 0 \\
    \end{gather*}
    $S(x) \in C^0 \quad (\{x=0\})$ poichè $\sum f_n$ converge totalmente in $[-a,a]$ e quindi $\sum f_n$ converge unoformemente, il th. di continuità del limite garantisce $S(x) \in C^0 ([-a,a])$
  \end{observation}
  \begin{gather*}
    \lim_{x \to 0} \frac{S(x)}{x} \overset{0/0}{=} \lim_{x \to 0} \frac{S(x) - S(0)}{x - 0} \overset{se \exists}{=} S'(0) = \sum_0^{+\infty} f_k'(0) = \sum_0^{+\infty} \frac{2^n}{3^n} \cancel{\cos(\frac{0}{3^n})} = 1 \\ 
  \end{gather*}
  l'ultimo passaggio abbiamo usato il th. di derivazione per Serie\\
  Da questo:\\
  $\sum f_k$ converge puntualmente a $S(x)$ \cmark\\
  $\sum f_k' \rightrightarrows G \in I$ \cmark\\
  $\implies S'(x) = \sum f_k'(x)$
  \begin{gather*}
    |f_k'(x)| = \left\lvert \frac{2^n}{3^n} \cos\left(\frac{x}{3^n}\right) \right\rvert \leq \left(\frac{2}{3}\right)^n \implies \underset{x \in [-a,a]}{sup} |f_k'(x)| \leq \left(\frac{2}{3}\right)^n = M_n \\
    \implies \sum f_k' \text{ converge totalmente in } [-a,a] \quad \text{quindi converge uniformemente, inoltre:} \sum M_n \text{conv}\\
    = \sum_{0}^{+\infty} \left(\frac{2}{3}\right)^n = \frac{1}{1 - \frac{2}{3}} = 3
  \end{gather*}
\end{example}

\subsection{serie di potenze}
\begin{definition}
  $\{a_k\} \subseteq \mathbb{R}$ serie di potenze di centro $x_0 \in \mathbb{R} \qquad \sum_{n=0}^{+\infty} a_n (x - x_0)^n$
\end{definition}
\begin{observation}
  posso suppore $x_0=0$ $y=x-x_0$ considerando $\sum a_n y^n$
\end{observation}
\begin{observation}
  $\sum a_n x^n$ converge in $x=0$ \small sempre (poichè $S_n(0) = a_0$)\normalsize
\end{observation}
\begin{theorem}
  $\exists \xi \neq 0 \ t.c. \ \sum a_n x^n $ converge in $\xi$\\
  $\implies$ la serie converge totalmente in $[a,b] \quad \forall [a,b] \subseteq (-|\xi|, |\xi|)$\\
  e quindi conerge puntualmente in $(-|\xi|, |\xi|)$
\end{theorem}
\begin{observation}
  $(-|\xi|, |\xi|) \subseteq$ L'insieme di convergenza
\end{observation}
\begin{proof}
  \hfil\\
  sia $\eta \in (-|\xi|, |\xi|)$\\
  $I = [-\eta, \eta] \in I$\\
  \begin{gather*}
    \left\lvert a_n x^n \right\rvert = \underbrace{\left\lvert a_n \xi^n \right\rvert} \left\lvert \frac{x}{\xi} \right\rvert^n := M_n \text{ e oss. che } \sum M_n \text{conv. } \left(\left\lvert \frac{\eta}{\xi}\right\rvert  < 1 \right) \\
    \exists M \ t.c. \ \leq M \quad \forall n \in \mathbb{N}
    \implies \sum a_n x^n \text{ converge totalmente } \in I = [-\eta, \eta] \\
  \end{gather*}
\end{proof}

\begin{observation}
  $\left(-|\xi|, |\xi|\right) $L'insieme di convergenza di una serie di potenze
\end{observation}

\begin{definition}
  \begin{gather*}
    \boxed{\rho := sup \{\xi \in \mathbb{R} : \sum a_n  x^n \text{ conv. in } \xi \}}
  \end{gather*}
  Detto raggio di convergenza della serie di potenze
  \begin{gather*}
    \rho = \begin{cases}
      +\infty \leftrightarrow \sum a_n x^n \text{conv. } \forall x \in \mathbb{R}\\
      = R \in \mathbb{R} \\
      = 0 \leftrightarrow \sum a_n x^n \text{conv. solo per } x=0
    \end{cases}
  \end{gather*}
\end{definition}
\begin{theorem}
  $\rho = R \leftrightarrow $\\
  \begin{gather*}
    \sum a_n x^n \text{converge se } |x| < R \\
    \sum a_n x^n \text{non converge se } |x| > R
  \end{gather*} 
  Con $R := $ raggio di convergenza
\end{theorem}
\begin{observation}
  per $|x| = R$ non si sa nulla in generale
\end{observation}
\begin{proof}
  [\textbf{parte 1} $\Rightarrow$]
  $\exists \xi \in \mathbb{R}: |x| < |\xi| \leq R$ e $\sum a_n \xi^n $ converge.\\
  Dal th. prec si ha convergenza in $(-|\xi|, |\xi|)$ quindi anche in x (poichè converge in $-|x|$ e  $|x|)$\\
  \underbar{sia} $\xi: \xi > R$
  suppongo $\sum a_n \xi^n $ converge $\implies$ dal th.prec. $\sum a_n \xi^n$ converge anche in $x$ t.c. $R < x \leq \xi$ è assurdo perchè $R = sup \{\dots\}$\\

  [\textbf{parte 2} $\Leftarrow$]
  $\sum a_n x^n $ conv per $|x| < \rho$ o non conv per $|x| > \rho$ (vogliamo dimostrare $\rho = R$)\\
  \begin{itemize}
    \item se $x \in (-\rho, \rho)$ dal th. prec. si ha $x \leq sup \{\dots\} = R \quad \forall x \in (-\rho,\rho) \implies \rho \leq R$
    \item se fosse $\rho < R \quad \exists \rho < \xi < R$ con $\sum a_n \xi^n$ converge dal th. prec.
    Ma per hp. $\sum x_n \xi^n$ non converge per $|x| > \rho$ quindi $\rho = R$
  \end{itemize}
  \begin{observation}
    $(-R,R) \subseteq$ l'insieme di convergenza ed è un intervallo
  \end{observation}
\end{proof}

\begin{theorem}
  [di Cauchy-Hadamard]
  \begin{gather*}
    \sum_{0}^{+\infty} a_n x^n \quad \ell := \overline{\lim_{n \to +\infty}} \sqrt[n]{|a_n|} \\
  \end{gather*}
  \underbar{allora:}
  \begin{itemize}
    \item se $\ell = 0 \implies R = +\infty$
    \item se $\ell = +\infty \implies R = 0$
    \item $\ell \in R, \ell \neq 0 \implies R = \frac{1}{\ell}$
  \end{itemize}
\end{theorem}
\begin{proof}
  \hfil\\
  \textbf{casi possibili:} $\exists \lim_{n \to \infty} \sqrt[n]{|a_n|} = \ell \begin{cases}
    0\\
    +\infty\\
    \ell \in \mathbb{R}, \ell \neq 0
  \end{cases}$\\
  \textbf{caso 1:} $\ell = 0$\\
  \bulletout \underbar{sia} $x \neq 0$ $\lim \sqrt[k]{|a_k| |x|^k} = |x| \cdot \ell = \begin{cases}
    0 \text{ se } \ell = 0\\
    +\infty \text{ se } \ell = +\infty\\
    |x| \cdot \ell \text{ se } \ell \in \mathbb{R}, \ell \neq 0 
  \end{cases}$
  \textbf{caso 2:}  $\ell = +\infty$\\
  \bulletout \underbar{se} $\ell = 0$ il criterio della radice per la serie numerica $\sum |a_n| |x|^n$ garantisce la convergenza.\\
  \bulletout \underbar{se} $\ell = +\infty$ il criterio della radice per la serie numerica $\sum |a_n| |x|^n$ garantisce la divergenza.\\
  \bulletout \textbf{caso 3:} $\ell \in \mathbb{R}, \ell \neq 0$\\
  il criterio della radice fa si che la serie converga se $\ell |x| < 1$ e non converga se $|x| > 1$\\
  Dal th. 1.4 segue $R = \frac{1}{\ell}$
\end{proof}

\begin{example}
  \begin{gather*}
    \sum_{1}^{+\infty} 2^n x^n
  \end{gather*}
\end{example}

Un esempio su foto.


\section{17/12/25}
\subsection{limiti superiori}
si indicano con $\overline{\lim} a_n = max\{ \lim a_{nk} \}$
\begin{example}
  \begin{gather*}
    \sum_{n=0}^{+\infty} 2^n x^{2n} = 1 +2x^2 + 4x^4 + 8x^6 + \dots
  \end{gather*}
  Serie di potenze $\leftrightarrow \sum_{k=0}^{+\infty} a_k x^k$
\end{example}

\textbf{I)} \\
$y = x^2$ considero $\sum 2^k y^k$ è serie potenza $a_k = 2^k$\\
 questa serie converge per la $y = (-\frac{1}{2}, \frac{1}{2})$ ne dedco quello per $x$.
 \begin{gather*}
    \frac{-1}{2} < x^2 < \frac{1}{2} \implies -\frac{1}{\sqrt{2}} < x < \frac{1}{\sqrt{2}}
 \end{gather*} 
\hfill\\
\textbf{II)}
\begin{gather*}
  \sum a_k x^k = a_0 + a_1 x + a_2 x^2 + a_3 x^3 + a_4 x^4 + a_5 x^5 + a_6 x^6 + \dots \\
\end{gather*}

\hfill\\
\begin{gather*}
  a_k = \begin{cases}
    2^{n} \quad k = 2n \text{ quindi k pari}\\
    0 \quad k = 2n + 1 \text{ quindi k dispari}
  \end{cases} \\
  \underbrace{\overline{\lim}}_{\overset{||}{\limsup_{n \to \infty}}} \sqrt[k]{|a_k|} = \lim_{n\to \infty} \sqrt[2n]{2^n} = \sqrt{2} \\
\end{gather*}
\begin{gather*}
  \sqrt[k]{|a_k|} = \begin{cases}
    0\\
    (2^n)^{\frac{1}{2n}} \quad k = 2n
  \end{cases} = \begin{cases}
    0 \quad k \text{ dispari}\\
    \sqrt{2} \quad k \text{ pari}
  \end{cases}
\end{gather*}

\begin{example}
  \begin{gather*}
    \sum_{0}^{+\infty} 2^n x^{n^2} = 1 + 2x + 4x^4 + 8x^9 + 16x^{16} + \dots
  \end{gather*}
  è serei di potenze $\sum a_k x^k$
  \begin{gather*}
    a_k = \begin{cases}
      2^n \quad k = n^2\\
      0 \quad k \neq n^2 \text{ quindi se non è un quadrato perfetto}
    \end{cases} \\
    \sqrt[k]{|a_k|} = \begin{cases}
      0 \quad k \neq n^2\\
      (2^n)^{\frac{1}{n^2}} = 2^{\frac{1}{n}} \quad k = n^2
    \end{cases} \\
    \overline{\lim} \sqrt[k]{|a_k|} = \lim_{n \to \infty} \sqrt[n^2]{2^n} = \lim_{n \to \infty} 2^{\frac{1}{n}} = 1 \\
    R = 1
  \end{gather*}
  Quindi ho il raggio di convergenza $= 1$ e converge nell'aperto $I_\text{conv} = (-1,1)$ di conseguenza ho $I_\text{non conv} = (-\infty, -1] \cup [1, +\infty)$
\end{example}


\begin{theorem}
  [derivazione serie di potenze]
    $\sum_{0}^{+\infty} a_n x^n \quad R = raggio di convergenza$ \\
    \underbar{allora}: la serie derivata $\sum_{1}^{+\infty} k a_k x^{k-1}$ ha ancora raggio di convergenza $R$.
\end{theorem}
\begin{observation}
  $x^{k-1} = \frac{x^2}{x}$
\end{observation}

\begin{theorem}
  [di derivazione e integrazione per serie di potenze]
  $\sum_{n=0}^{+\infty} a_n x^n$ con $R \neq 0$ raggio di convergenza, e $S(x) = \sum a_n x^n$ per $x \in I$\\
  \underbar{allora}: $S'(x) = \sum_{1}^{+\infty} n a_n x^{n-1} \quad \forall x:|x| < R$ cioè in $\mathring{I}$\\
  e $\int_0^x S(t) dt = \sum_{0}^{+\infty} \frac{a_n}{n+1} x^{n+1} \quad \forall x: |x| < R$
\end{theorem}
\begin{proof}
  \hfil\\
  segue dal theorema della serie derivate e dalla convergenza uniforme della serie di potenze in $[a,b] \subseteq (-R,R)$ + valgono i teoremi di derivazione e integrazione per serie di funzioni applicati a:
  \begin{gather*}
    S(x) = \sum_{0}^{+\infty} a_n x^n \\
    G(x) = \sum_{1}^{+\infty} n a_n x^{n-1} \\
  \end{gather*}
\end{proof}

\begin{example}
  \begin{gather*}
    \sum_{n=0}^{+\infty} n 2^n x^n \quad R = \frac{1}{2} \\
    I_\text{conv} = (-\frac{1}{2}, \frac{1}{2}) \\
  \end{gather*}
  \begin{observation}
    $x(2^n x^n)' = n 2^n x^n$ quindi $\sum n 2^n x^n = \sum x (2^n x^n)' = \boxed{x \sum_{n=0}^{+\infty} (2^n x^n)'} = x \cdot S'(x)$
  \end{observation}
  dove $S(x) = \sum_{n=0}^{+\infty} 2^n x^n = \frac{1}{1-2x}$ (serie geometrica nota) e quindi:
  \begin{gather*}
    = x \cdot S'(x) = x \cdot \left(\frac{1}{1-2x}\right)' = x \cdot \frac{2}{(1-2x)^2} = \frac{2x}{(1-2x)^2} \\
  \end{gather*}
\end{example}
\begin{theorem}
  [di Abel]
  Presa: $\sum a_n x^n$ con $ R>0 $ e $\sum a_n R^n$ converge\\
  \underbar{allora}: $\sum a_n x^n $ converge uniformemente in $[0,R]$ e la somma è $C^0$ per $x \to R$
\end{theorem}
\begin{problem}
  $\sum a_n x^n + \sum b_n x^n \overset{?}{=} \sum (a_n + b_n) x^n$ 
\end{problem}
\begin{solution}
  \hfil\\
  Vale quando:\\
  $a_n x^n $ converge $\leftrightarrow$ converge $|x|<R_1$\\
  $b_n x^n $ converge $\leftrightarrow$ converge $|x|<R_2$\\
  e definisco $R := min \{ R_1, R_2 \}$
\end{solution}
\begin{proof}
  vediamo di dimostrare questo fatto.\\
  \begin{gather*}
    f(x) = \sum a_n x^n \quad g(x)\\
    g(x) = \sum b_n x^n \\
    S_N = \sum_{n=0}^{N} (a_n + b_n) x^n = \sum_{n=0}^{N} a_n x^n + \sum_{n=0}^{N} b_n x^n \\
  \end{gather*}
  Vogliamo dimostrare che $\lim S_N = f(x) - g(x)$
  \begin{gather*}
    \left\lvert S_N(x) - (f(x)- g(x)) \right\rvert = \left\lvert \sum_{0}^{N} a_n x^n - f(x) + \sum_{0}^{N} b_n x^n - g(x) \right\rvert \\
    \leq \underbrace{\left\lvert \sum_{0}^{N} a_n x^n - f(x) \right\rvert}_{\to 0 \text{ se } |x| < R} + \underbrace{\left\lvert \sum_{0}^{N} b_n x^n - g(x) \right\rvert}_{\to 0 \text{ se } |x| < R} \\
  \end{gather*}
\end{proof}

\begin{theorem}
  $\sum a_n x^n$ con $R\neq 0$ raggio di convergenza e $f(x) = \sum a_n x^n$ per $x \in I = (-R;R)$\\
  \underbar{allora}: $f \in C^\infty (\mathring{I}) \quad \forall m \in \mathbb{N}$ si ha:\\
  \begin{gather*}
    f^{(m)} (x) = \sum_{k=m}^{+\infty} k(k-1)(k-2) \dots (k - m + 1) a_k x^{k-m} \text{ in } \mathring{I}\\
  \end{gather*}
  e in particolare:
  \begin{gather*}
    a_k = \frac{f^{(k)}(0)}{k!} \quad \text{cioè} \quad f(x) = \sum_{k=0}^{+\infty} \frac{f^{(k)}(0)}{k!} x^k 
  \end{gather*}
\end{theorem}
\begin{proof}
  Si applica il teorema di derivazione per serie m-volte:\\
  Calcolo:
  \begin{gather*}
    f^{(m)}(0) = \sum_{k=m}^{+\infty} k(k-1)(k-2) \dots (k - m + 1) a_k \ x^{k-m} \Bigg\rvert_{x=0} = m(m-1) \dots 2 \cdot 1 \cdot a_m = m! \ a_m \\
  \end{gather*}
\end{proof}
\begin{definition}
  Si chiama serie di Taylor di $f \in C^\infty$ centrata in $x_0$:
  \begin{gather*}
    \sum_{k=0}^{+\infty} \frac{f^{(k)}(x_0)}{k!} (x-x_0)^k = S_{T,x_0}
  \end{gather*} 
\end{definition}
\begin{observation}
  sia $f(x) = \sum a_n (x-x_0)^n \quad x \in (a,b)$\\
  allora $\forall I_{x_0} \subseteq (a,b) , |x-x_0| < \rho$ si ha:
  \begin{itemize}
    \item $f$ è $\infty$-derivabile per $|x-x_0| < \rho$
    \item $a_n = \frac{f^{(n)}(x_0)}{n!}$ se $f(x)$
  \end{itemize}
  Quindi $f$ è sviuppabile in serie di Taylor se $f(x)$ stessa è uguale alla sua serie di Taylor
\end{observation}
\begin{observation}
  se $f$ è $\infty$-derivabile e $\sum \frac{f^{(k)}(0)}{k!}$ convergente $\cancel{\implies} f(x) = \sum \frac{f^{(k)}(0)}{k!} x^k$\\
  \begin{example}
    \begin{gather*}
      f(x) = \begin{cases}
        e^{-\frac{1}{x^2}} \quad \text{se } x \neq 0 \qquad f \infty-\text{deriv.} in \mathbb{R} 
        0 \quad \text{se x = 0} \qquad f(0) = f'(0) = ... = f^{(k)}(0) = 0
      \end{cases}
    \end{gather*}
    ma $f(x) \neq \sum \frac{f^{(k)}(0)}{k!} x^k = 0$ poichè $f \cancel{\equiv} 0$
  \end{example}
\end{observation}
\begin{theorem}
  [criterio di sviluppabilità in serie di Taylor]
  Presa $f$ $\infty$-derivabile in $(a,b)$ ed $\exists M,L > 0: |f^{(k)}(0)| \leq M L^k \quad \forall x \in (a,b)$\\
  \underbar{allora}: $\forall x_0 \in (a,b) f$ è Taylo-sviluppabile di centro $x_0 \in (a,b)$\\
  cioè $f(x) = \sum \frac{f^{(k)}(x_0)}{k!} (x-x_0)^k \quad \forall x \in (a,b)$
\end{theorem}
\begin{example}
  Sviluppare in un intorno di $x_0 = 0$ e calcolare $g^{17}(0)$ della funzione:
  \begin{gather*}
    g(x) := \ln(1+x)
  \end{gather*}
  \begin{gather*}
    g'(x) = \frac{1}{1+x} = \sum_{n=0}^{+\infty} (-x)^n\\
    g(x) = \int_0^x g'(t) dt = \int_0^x \sum_{n=0}^{+\infty} (-t)^n dt \overset{\text{th. di integ. per serie pot.}}{=} \sum_{n=0}^{+\infty} \int_0^x (-t)^n dt = \sum_{n=0}^{+\infty} \frac{(-1)^n}{n+1} t^{n+1} \big\rvert_0^x = \sum_{n=0}^{+\infty} \frac{(-1)^n}{n+1} x^{n+1} \qquad n+1 = k
  \end{gather*}
  Andiamo ora a calcolare la derivata 17-esima in 0:
  \begin{gather*}
    g^{17}(0) = 17! \cdot a_{17} = 17! \cdot \frac{(-1)^{18}}{17} = 16!
  \end{gather*}
\end{example}

\begin{example}
  Calcolare il valore di $\pi$ con un errore massimale di $\varepsilon$ assegnato.\\
  \begin{gather*}
    \frac{\pi}{4} = \arctan(1)\\
  \end{gather*}
  Sappiamo che:
  \begin{gather*}
    \arctan(x) = \int_0^1 \frac{1}{1+t^2} \ dt = \int_0^1 \sum_{n=0}^{+\infty} (-t^2)^n dt = \overset{\text{th. di int.}}{=}  \sum_{n=0}^{+\infty} \frac{(-1)^n}{2n+1} t^{n+1} \big\rvert_0^1 \underset{|x| \leq 1}{ = } \sum_{n=0}^{+\infty} \frac{(-1)^n}{2n+1} x^{2n+1}\\
  \end{gather*}
  Ho quindi che $\arctan(1) = \sum_{n=0}^{+\infty} \frac{(-1)^n}{2n+1} = \sum_{n=0}^{N} \frac{(-1)^n}{2x+1} + \underbrace{R_N}_{< \varepsilon}$\\
  Con $|R_N| = \left\lvert \sum_{n=N+1}^{+\infty} \frac{(-1)^n}{2n+1} \right\rvert = \left\lvert \frac{(-1)^{n+1}}{2(n+1)+1} + \frac{(-1)^{n+2}}{2(n+2)+1} + \frac{(-1)^{n+3}}{2(n+3)+1} \right\rvert \\
  \leq \left\lvert \frac{(-1)^{N+1}}{2(N+1)+1} \right\rvert = \frac{1}{2(N+1)+1} < \varepsilon$\\
\end{example}
Questa è piccola a piacere proprio perchè la serie $\frac{1}{2n+1}$ è decrescente.

\textbf{Curiosità}:
\begin{gather*}
  \pi = 48 \arctan(\frac{1}{18}) + 32 \arctan(\frac{1}{57}) - 20 \arctan(\frac{1}{239})
\end{gather*}

\begin{example}
  Data questa serie:
  \begin{gather*}
    \sum_{n=0}^{+\infty} (2^n - (-1)^n3^n) x^{n^2}
  \end{gather*}
  Determinare $I_\text{conv}$ e $\lim_{x \to 0} \frac{S(x)}{x}$\\
  Si vede intanto che è serie di potenze $\sum a_k x^k$
  \begin{gather*}
    a_k = \begin{cases}
      2^n - (-1)^n 3^n \quad k = n^2\\
      0 \quad k \neq n^2
    \end{cases} \\
    \sqrt[k]{|a_k|} = \begin{cases}
      0 \quad k \neq n^2\\
      (2^n - (-1)^n 3^n)^{\frac{1}{n^2}} \quad k = n^2
    \end{cases} \\
    \overline{\lim} \sqrt[k]{|a_k|} = \lim_{n \to \infty} \left\lvert 2^n - (-1)^n 3^n\right\rvert ^{\frac{1}{n^2}} = \lim_{n \to +\infty} \overbrace{3^{\frac{1}{n}}}^{\to 1} \underbrace{\left\lvert (\frac{2}{3})^n - (-1)^n \right\rvert}_{(\frac{1}{2})^{\frac{1}{n^2}} < \ |\ "\ | \ < 2^{\frac{1}{n^2}}}^{\frac{1}{n^2}} = 1 \\
    R = 1
  \end{gather*}
  Quindi $(-1,1) \subseteq I_\text{conv}$ poichè per $x=1$ non converge $x=-1$ neanche.\\
  Passo ora al secondo punto dell'esercizio cioè calcolare il limite $ \lim_{x \to 0} \frac{S(x)}{x}$:
  \begin{gather*}
    S(x) = 0 + (2 - (-3)) x^{1} + (4 - 9) x^{4} + R_N(x)\\
    \lim_{x \to 0} \frac{S(x)}{x} = \lim_{x \to 0} \frac{5x + (-5)x^4 + o(x^2)}{x} = \lim_{x \to 0} 5 + (-5)x^3 + \frac{o(x^2)}{x} = 5 + 0 + 0 = 5 \\
  \end{gather*}
\end{example}

\begin{example}
  Data la serie di potenze generica:
  \begin{gather*}
    S(x) = \sum_{n=0}^{+\infty} a_n x^n
  \end{gather*} 
  Ed è dato che il raggio di convergenza è $R=1$.\\
  \bulletout Sapendo che $\lim_{x \to 1^-} S(x) = L \in \mathbb{R}$ si può dire che $\sum a_n$ converge?\\
  Sospetto che la risposta sia no (poichè dal th. di Abel vale il contrario ma non viceversa) e cerco un controesempio:
  \begin{gather*}
    S(x) = \frac{1}{1+x} = \sum_{n=0}^{+\infty} (-1)^n x^n
  \end{gather*}
  Il limite per $x \to 1^-$ è:
  \begin{gather*}
    \lim_{x \to 1^-} S(x) = \lim_{x \to 1^-} \frac{1}{1+x} = \frac{1}{2} \in \mathbb{R}
  \end{gather*}
  Ma la serie numerica:
  \begin{gather*}
    \sum_{n=0}^{+\infty} (-1)^n
  \end{gather*}
  non converge.\\
  Quindi la risposta è no.
  \bulletout Mi chiedo ora:\\
  $\sum |a_n| < +\infty \overset{?}{\implies} \exists$ finiti $\lim_{x \to 1^-} S(x) , \lim_{x \to 1^-} S(x)$\\
  Qua mi aspetto che la risposta sia si ma devo dimostrarlo:
  \begin{gather*}
    \sum |a_n| \text{conv. } \implies \sum a_n x^n I_\text{conv} = [0,1] \\
  \end{gather*}
  Inoltre converge totalmente in $[-1,1]$ che implica che converge uniformemente in $[-1,1]$\\
  Quindi vale il theorema di continuità della somma il che imlica che:
  \begin{gather*}
    S(x) \in C^0 ([-1,1])
  \end{gather*}
  \bulletout Adesso ancora mi chiedo $\sum a_n$ div. con $a_n \geq 0$ si può dire che $\lim_{x \to 1^-} S(x) = +\infty$\\
  Di nuovo mi aspetto che sia vero ma devo dimostrarlo:\\
  fisso $M>0$\\
  \begin{gather*}
    S(x) = \sum a_n x^n \geq \sum_{0}^{k} a_n x^n \quad \forall k \in \mathbb{N}\\
    \sum a_n \text{div. } \implies \exists k_m: \sum_{0}^{k_m} a_n > M \\
    \text{sia } \delta > 0: 1-\delta < x < 1 , (1-\delta)^{k_n} > \frac{1}{2}
  \end{gather*}
  Di consegneuza:
  \begin{gather*}
    \sum_{0}^{k} a_n \underbrace{x^n}_{> (1-\delta)^k} > (1-\delta)^{k_n} \sum_{0}^{k_n} a_n > \frac{1}{2} 2M = M
  \end{gather*}
\end{example}

\end{document}
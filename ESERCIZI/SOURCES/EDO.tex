\documentclass{article}
\usepackage{graphicx} % Required for inserting images
\usepackage{amsmath}
\usepackage{amssymb}

\title{Analisi II - Equazioni differenziali}
\author{Marco Delton\thanks{esercizi dei prof. \textit{Gabriele Bianchi}, \textit{Chiara Bianchini} e \textit{Luca Bisconti}}}
\date{A.A. 2025/26}

\begin{document}

\maketitle

\section{EDO del 1° ordine}
Trovare l'\textbf{integrale generale} delle seguenti equazioni differenziali
\begin{enumerate}
    \item $y'=e^{\frac{y}{x}}+\frac{y}{x}$
    \item $y'=\frac{(x-y)y}{x^2}$
    \item $\begin{cases}
        y'=-\frac{x+y}{x}\\
        y(1)=0
    \end{cases}$
    \item $\begin{cases}
        y'=-\frac{x+y}{x}\\
        y(1)=-\frac{1}{2}
    \end{cases}$
    \item $\begin{cases}
        y'=\frac{x^2+y^2}{xy}\\
        y(4)=0
    \end{cases}$
    \item $\begin{cases}
        y'=\frac{x^2+y^2}{xy}\\
        y(1)=1
    \end{cases}$
    \item $y'=\frac{y}{x-2\sqrt{xy}}$
\end{enumerate}

\newpage

\section{EDO lineari del 2° ordine}
Trovare l'\textbf{integrale generale} delle seguenti equazioni differenziali del 2° ordine lineari e a coefficienti costanti\\
\begin{enumerate}
    \item \begin{enumerate}
        \item $y''-4y=4e^{-x}+x^2+2$
        \item $y''-4y=3e^{2x}$
        \item $y''-4y=3e^{2x}-e^{-2x}\\$
    \end{enumerate}
    \item \begin{enumerate}
        \item $y''-4y'+5y=2x^2+3e^{-x}$
        \item $y''-4y'+5y=e^{2x}\cos(x)$
        \fbox{\tiny{\textbf{NOTA: } $e^{2x}\cos(x)=\operatorname{Re}(e^{\lambda_0x})$ con $\lambda_0=2+i$}}
        \item $y''-4y'+5y=e^{2x}\sin(x)$
        \fbox{\tiny{\textbf{NOTA: } $e^x\sin(x)=\operatorname{Im}(e^{\lambda_0x})$}}$\\$
    \end{enumerate}
    \item \begin{enumerate}
        \item $4y''-4y'+y=x$
        \item $4y''-4y'+y=3x^2+2x-1$
        \item $4y''-4y'+y=e^{\frac{x}{2}}$
        \item $4y''-4y'+y=3xe^{\frac{x}{2}}\\$
    \end{enumerate}
    \item \begin{enumerate}
        \item $y''+y=e^x\cos{(2x)}$
        \item $y''+y=xe^x\cos{(2x)}$
        \item $y''+y=-e^{-x}\cos{x}\\$
    \end{enumerate}
    \item \begin{enumerate}
        \item $y''-3y'+2y=ax^2+bx+c$
        \item $y''-3y'+2y=e^{3x}$
        \item $y''-3y'+2y=e^x\\$
    \end{enumerate}
    \item \begin{enumerate}
        \item $y''-y'+y=\pi e^{-\frac{x}{2}}$
        \item $y''-y'+y=\pi x e^{-\frac{x}{2}}$
        \item $y''-y'+y=-\cos{\left(\frac{\sqrt{3}}{2}x\right)}\\$
    \end{enumerate}
    \item \begin{enumerate}
        \item $y''-2y'+y=ax^2+bx+c$
        \item $y''-2y'+y=3e^x\\$
    \end{enumerate}
    
    \item Risolvere
    \[y''-5y+6y=x\]
    con il metodo di \textbf{variazione delle costanti} e per \textbf{similarità}\\
    \fbox{\tiny{Cerco $y_p(x)=A(x)e^{2x}+B(x)e^{3x}$ $\Rightarrow$ Cerco $A(x)$ e $B(x)$ t.c. $\begin{cases}
        A'e^{2x}+B'e^{3x}=0\\
        2A'e^{2x}+3B'e^{3x}=x
    \end{cases}$}}
    
    \item Risolvere
    \[y''-y=\sin(x)\]
    con il metodo di \textbf{variazione delle costanti} e per \textbf{similarità}\\
    \fbox{\tiny{Cerco $y_p(x)=A(x)e^{-x}+B(x)e^x$ $\Rightarrow$ Cerco $A(x)$ e $B(x)$ t.c. $\begin{cases}
        A'e^{-x}+B'e^x=0\\
        -A'e^{2x}+B'e^x=\sin{x}
    \end{cases}$}}$\\$

    \item Ricondursi all'EDO lineare e risolverla, sfruttando un opportuno cambio di variabili
    \begin{enumerate}
        \item $\begin{cases}
            x^2y''-xy'-3y=0\\
            y(1)=1\\
            y'(1)=0
        \end{cases}$
        \item $x^2y''-xy'+y=x+\frac{4}{x}$
        \item Determinare le soluzioni di
        \[x^2y''+xy'-4y=\frac{1}{x^3}-x\]
        tali che 
        \[\lim_{x\to \infty}\frac{y(x)}{x}=\frac{1}{3}\]
    \end{enumerate}
\end{enumerate}

\subsection{Foglio n.2}
\begin{itemize}
    \item $y''-5y'+6y = x-1+e^{2x}$
    \item $y''-4y'+8y=e^{2x}\cos(2x)$
\end{itemize}

\newpage

\section{EDO qualitative}
\begin{enumerate}
    \item Studiare l'andamento di $y(x) \quad \forall a \in \mathbb{R}$
    \[\begin{cases}
        y'=4y(1-y)\\
        y(0)=a
    \end{cases}\]

    \item Dato il seguente problema di Cauchy:
    \[\begin{cases}
        y'=x\left(1+\frac{1}{y}\right)\\
        y(0)=a \quad \text{con} \quad a \ne 0
    \end{cases}\]
    \begin{enumerate}
        \item Provare che $y(x)$ è definita $\forall x \in \mathbb{R}$
        \item Studiare l'andamento di $y(x) \ \forall a \in \mathbb{R}$ con $a \ne 0$
        \item Studiare la natura del punto critico $x=0$
        \item Studiare eventuali asintoti orizzontali e obliqui di $y(x)$
    \end{enumerate}

    \item Dato il seguente problema di Cauchy
    \[\begin{cases}
        4x^2y'+2\sin{(y)}\cos{(y)}=1\\
        y(1)=\pi
    \end{cases}\]
    \begin{enumerate}
        \item Provare che $y(x)$ è prolungabile a $+ \infty$
        \item Determinare se $\displaystyle \exists \lim_{x\to \infty}y(x)$
        \item Trovare la soluzione esplicita
    \end{enumerate}

    \item Studiare l'andamento di $y(x)$ $\forall a \in \mathbb{R}$
    \[\begin{cases}
        y'=\sin{\left(y^2\right)}\\
        y(0)=a \in \mathbb{R}
    \end{cases}\]

    \item Dato il seguente problema di Cauchy: 
    \[\begin{cases}
        y'=e^{y^2}\\
        y(0)=0
    \end{cases}\]
    \begin{enumerate}
        \item Provare che $x=0$ è un punto di flesso
        \item Studiare qualitativamente la soluzione massimale $y(x)$ (eventuali simmetrie, massimi e minimi locali, etc.)
    \end{enumerate}

    \item Dato il seguente problema di Cauchy, studiare qualitativamente la soluzione massimale $y(x) \quad \forall a \in \mathbb{R}$ (crescenza e descrescenza, concavità, punti critici, informazioni sul dominio, etc.):
    \[\begin{cases}
        y'=\frac{\tan{(y)}}{1+y^2}\\
        y(0)=a \in \mathbb{R}
    \end{cases}\]
\end{enumerate}

\end{document}

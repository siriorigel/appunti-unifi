\documentclass{article}
\usepackage{amsmath}
\usepackage{amssymb}

%COMANDI NUOVI
\newcommand{\R}{\mathbb{R}}

\title{Analisi II - Successioni di funzioni}
\author{Marco Delton\thanks{esercizi dei prof. \textit{Chiara Bianchini} e \textit{Luca Bisconti}}}
\date{A.A. 2025/26}

\begin{document}
\maketitle

\section{Foglio n.1}
Analizzare la convergenza puntuale e uniforme della seguenti successioni di funzioni: 
\begin{enumerate}
    \item $f_n(x) = \left[\sin(x)\right]^{\frac{1}{n}}$
    \item $f_n(x) = e^x - \left(1+\frac{x}{n}\right)^n$
    \item $f_n(x) = \frac{x}{nx-1}$
    \item $f_n(x) = \frac{1}{nx+1}$
    \item $f_n(x) = e^{-nx} \ \sin(nx)$
    \item $f_n(x) = \frac{1}{1+n^2(x^2-1)^2}$
    \item $f_n(x) = \frac{nx+1}{2+n}$
    \item $f_n(x) = \frac{n+n^{x^3}}{(n+x^2)^{2x^2}}$
    \item $f_n(x) = \left[1+(x-n)^2\right]^2 \ \left[\cos^2\left(\frac{1}{1+(x-n)^2}\right)-1\right]$
    \item $\displaystyle f_n(x) = \int_{n^x}^{n^{2x}} \frac{1}{t} \ \arctan\left(\frac{1}{t}\right) \ dt$
    \item $f_n(x) = nx \ e^{-nx^2}$
    \item $f_n(x) = \sqrt{x^2 + \frac{1}{n}}$
    \item $f_n(x) = \frac{1}{n} \ \sin(nx)$
    \item $f_n(x) = \frac{n^2 \ \sin\left(\frac{x}{n}\right)}{\sqrt{1+n^2x^2}}$
    \item $f_n(x) = \sin\left[\frac{x}{2+\sin^n(x)}\right]$
    \item $f_n(x) = n \ \left(e^{\frac{x}{n}}-1\right)$
    \item $f_n(x) = \frac{nx \ \arctan(nx)}{\sqrt{n^2+x^2}}$
    \item $f_n(x) = x^n \ e^{-(x+1)^n}$
    \item $f_n(x) = \left(x^{2n}+1\right)^\frac{1}{n}$
\end{enumerate}

\newpage

\section{Foglio n.2}
Analizzare la convergenza puntuale e uniforme della seguenti successioni di funzioni: 
\begin{enumerate}
    \item Stabilire per quali $\alpha\in\R$, $\alpha<2$, si ha che $f_n(x)$ converge uniformemente in $I=[-1,1]$ con:
    \[f_n(x) = \frac{n^\alpha \ x}{1+n^2x^2}\]

    \item $\displaystyle f_n(x) = \begin{cases}
        1 & \text{se} \ n\leq x < n+1 \\
        0 & \text{altrimenti}
    \end{cases}$

    \item $f_n(x) = \frac{n \ (x-1)}{x^n}$ \\
    Provare che:
    \begin{itemize}
        \item $f_n(x)$ converge puntualmente in $[1,+\infty)$. 
        \item $f_n(x)$ non converge uniformemente in $[1,+\infty)$.
        \item $f_n(x)$ non converge uniformemente $[1,2]$.
        \item $f_n(x)$ converge uniformemente in $[2,+\infty)$.
    \end{itemize}

    \item $f_n(x) = n \ \sin\left(\frac{x}{n}\right)$ \\
    Studiare la convergenza in $\R$ e in $[a,b]$

    \item Sia $\{a_n\}\subseteq\R$, $\{a_n\}\to 0$:
    \begin{itemize}
        \item $f_n(x) = a_n^2 \ x^n$ su $x\in[-1,1]$
        \item $f_n(x) = e^{-(x-a_n)^2}$ su $x\in\R$ e $x\in[a,b]$
        \item $f_n(x) = e^{-\left(x-\frac{1}{a_n}\right)^2}$ su $x\in\R$ e $x\in[a,b]$
    \end{itemize}

    \item $f_n(x) = 1+n \ \sin\left[\frac{x \ \arctan(x) \ \ln(x^2+1)}{n^2}\right]$ su $I=[0,+\infty)$
    \item $f_n(x) = n \ \int_{n}^{n+1} \sin(\frac{x}{t}) \ dt$
    \item $f_n(x) = e^{-\left(x-\frac{1}{n}\right)^2} \ \cos\left[e^{\left(x-\frac{1}{n}\right)^2}\right]$ in $\R$ e in $[a,b]$
    \item $\displaystyle f_n(x) = \begin{cases}
        0 & x\in\left[0,\frac{1}{2}\right] \\
        n \ \left(x-\frac{1}{2}\right) & x\in\left(\frac{1}{2}; \ \frac{1}{2} + \frac{1}{n}\right) \\
        1 & x\in\left[\frac{1}{2} + \frac{1}{n}; \ 1\right]
    \end{cases}$
\end{enumerate}

\newpage

\section{Foglio n.3}
\begin{enumerate}
    \item Studiare la convergenza puntuale e uniforme di:
    \[f_n(x) = \frac{x \ \cos(nx)}{n}\]
    con $x\in\R$

    \item Si consideri $f_n: \ [0,1]\to\R$, dove $f_n(x) = x^n \ (1-x^n)$, $n\geq 1$:
    \begin{itemize}
        \item Determinare l'insieme di convergenza puntuale $L \subseteq [0,1]$, e la funzione limite
        $f=f(x)$, $f: \ L\subseteq[0,1]\to\R$
        \item Studiare la convergenza uniforme di $f_n$ e $f$ sia su $L$ sugli intervalli
        $[0,l]$ con $0<l<1$
    \end{itemize}

    \item Studiare la convergenza puntuale e uniforme della successione di funzioni: 
    \[f_n(x) = n^2x^2 \ (1-x)^n\]
    su $x\in[0,1]$

    \item Studiare la convergenza puntuale e uniforme della successione di funzioni:
    \[f_n(x) = e^{-nx} \ \arctan(nx)\]
    su $x\in\R$

    \item Studiare la convergenza puntuale e uniforme della successione di funzioni:
    \[f_n(x) = \begin{cases}
        \frac{2nx}{n-1} & x\in\left[0,1-\frac{1}{n}\right] \\
        3-\frac{n^2}{4}\left(\frac{1-n}{n}+x\right)^2 & x\in\left(1-\frac{1}{n},1\right]
    \end{cases}\]

    \item Studiare la convergenza puntuale e uniforme della successione di funzioni: 
    \[f_n(x) = \frac{n \ x^\frac{1}{3}}{2+n^2x^2}\]
    su $x\in[1,+\infty)$

    \item Studiare la convergenza puntuale e uniforme della seguente successione di funzioni
    al variare del parametro $p\in\R$:
    \[f_n(x) = n^p \ x \ e^{-nx}\]
    su $x\in[0,1]$

    \item Studiare la convergenza puntuale e uniforme della seguente successione di funzioni:
    \[f_n(x) = \sqrt{\cos^2(x) + \frac{1}{n^4}}\]
    su $x\in\R$
\end{enumerate}
\end{document}
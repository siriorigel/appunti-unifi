\documentclass{article}
\usepackage{amsmath}
\usepackage{amssymb}

\title{Analisi II - Spazi Metrici e Normati}
\author{Marco Delton\thanks{esercizi dei prof. \textit{Gabriele Bianchi} e \textit{Chiara Bianchini}}}
\date{November 2025}

\begin{document}

\maketitle

\begin{enumerate}
    \item Sia $(X,d)$ spazio metrico. Sia $d:X\times X \to \mathbb{R} $ t.c. $\forall x,y \in X$
    \[\delta(x,y)=\frac{d(x,y)}{1+d(x,y)}\]
    Provare che $(X,d)$ è spazio metrico.\\

    \item Sia $X=\mathbb{R}^2$. $\forall x,y \in X$ definiamo: \\
    \[d(x,y)=\begin{cases}
        |x-y| &\textsf{se x,y sono collineari in $\mathbb{R}^2$}\\
        |x|+|y| &\textsf{altrimenti}
    \end{cases}\]
        \begin{itemize}
            \item Provare che $d$ è una metrica su $\mathbb{R}^2$
            \item Descrivere graficamente le palle di $D$
        \end{itemize}
        
    \item Sia $d:\mathbb{R} \times \mathbb{R} \to [0,+\infty)$ t.c. $d(x,y)=\arctan(|x-y|)$
    $\forall x,y \in \mathbb{R}$.
        \begin{itemize}
            \item Provare che $(\mathbb{R},d)$ è spazio metrico
            \item Stabilire se è completo
        \end{itemize}

    \item Sia $X=C^0([0,1])$. Sia $|| \cdot ||_{L^2}=(\int_0^1|f(x)|^2 dx)^\frac{1}{2}$\\
    Provare che $|| \cdot ||_{L^2}$ è una norma su $X$\\
    \fbox{\tiny\textbf{Suggerimento:} Usare la disuguaglianza di Cauchy-Schwartz}\\

    \item Sia $(\mathbb{R}, d \varepsilon)$ spazio metrico. $f:\mathbb{R} \to \mathbb{R}$, $f \in C^0(\mathbb{R})$.\\
    Provare o confutare tramite controesempi le seguenti affermazioni: 
    \begin{itemize}
        \item $f(A)$ aperto $\implies$ $A$ aperto
        \item $A$ aperto $\implies$ $f(A)$ aperto
        \item $fA)$ chiuso $\implies$ $A$ chiuso
        \item $A$ chiuso $\implies$ $f(A)$ chiuso
    \end{itemize}

    \newpage

    \item Sia $X=\left(1+\frac{1}{n}\right)^n$. 
    \begin{itemize}
        \item Provare che $\{x_n\}$ è di Cauchy in $(\mathbb{Q},d\varepsilon)$
        \item Dedurre che $(\mathbb{Q},d\varepsilon)$ non è uno spazio metrico completo
    \end{itemize}

    \item Sia \[f_n(x)=\begin{cases}
        0 & \textsf{se }x \in [0,\frac{1}{2}]\\
        n\left(x-\frac{1}{2}\right) & \textsf{se } x \in \left(\frac{1}{2},\frac{1}{2}+\frac{1}{n}\right)\\
        1 & \textsf{se } x \in \left[\frac{1}{2}+\frac{1}{n},1\right]
    \end{cases}\]
    \begin{itemize}
        \item Provare che $\{f_n\}$ è di Cauchy in $\left(C^0([0,1]),d_{L^1}\right)$ \\con 
        $d_{L^1}=\int_0^1|f(x)-g(x)|dx$
        \item Dedurre che $\left(C^0([0,1]),d_{L^1}\right)$ non è completo
    \end{itemize}

    \item Sia $X=(0,1)$. Provare che $d(x,y)=\left|\frac{1}{x}-\frac{1}{y}\right|$ è una metrica su $X$\\

    \item Sia $X=\mathbb{R}$. Quali delle seguenti funzioni sono metriche su $X$?
    \begin{itemize}
        \item $d(x,y)=\begin{cases}
            x^3-y^3 & \textsf{se } x\geq y\\
            y-x & \textsf{se } x<y\end{cases}$

        \item $d(x,y)=\operatorname{min}\{|x-y|,1\}$

        \item $d(x,y)=|x-y|+|x^3+y^3|$

        \item $d(x,y)=\sqrt{|x-y|}$
        
    \end{itemize}
\end{enumerate}

\end{document}

\documentclass{article}
\usepackage{graphicx}
\graphicspath{{./IMMAGINI/}}
\usepackage{amsmath}
\usepackage{amssymb}

%comandi nuovi
\newcommand{\R}{\mathbb{R}}
\newcommand{\vF}{\vec{F}}
\newcommand{\Si}{\Sigma}
\newcommand{\vr}{\vec{r}}
\newcommand{\de}{\partial}

\title{Analisi II - Integrali di superficie}
\author{Marco Delton\thanks{esercizi della prof.ssa\textit{Chiara Bianchini}}}
\date{A.A. 2025/26}

\begin{document}
\maketitle

\section{Foglio n.1}

\begin{enumerate}
    %ES. 1
    \item Calcolare l'area della superficie:
    \[\vec{r}(\mu,\nu) = \begin{pmatrix}
        \mu \ \cos{(\nu)} \\ \mu \ \sin{(\nu)} \\ \mu^2
    \end{pmatrix}_{(\mu,\nu)\in[0,1]\times[0,\pi)}\\\]

    %ES. 2
    \item Calcolare il flusso del campo:
    \[\vF (x,y,z) = \begin{pmatrix}
        xy \\ xy \\ z
    \end{pmatrix} \ \text{attraverso} \ \Sigma = \left\{(x,y,z)\in\R^3: \ \begin{matrix}
        z=1-x^2-y^2 \\ z \geq 0
    \end{matrix}\right\}\\\]

    %ES. 3
    \item Calcolare l'area della superficie elicoidale che si avvolge di un giro. $\\$
    
    %ES. 4
    \item Calcolare: 
    \[\iint_\Sigma \frac{z+y^2}{\sqrt{1+4\left(x^2+y^2\right)}} \ d\sigma\]
    dove $\Sigma$ è parte della superficie $z=x^2-y^2$ che si proietta in:
    \[T=\left\{\begin{matrix}
        x^2+y^2 \geq 1 \\ x^2+4y^2 \leq 4
    \end{matrix}\right\}\\\]

    %ES. 5
    \item Calcolare l'area della porzione di cilindro:
    \[y^2+z^2=a^2\]
    interna al cilindro:
    \[x^2+y^2=a^2\\\]

    %ES. 6
    \item Una massa $M$ è distribuita uniformemente sul tronco di cono:
    \[\begin{cases}
        z^2=x^2+y^2 \\ 1 \leq z \leq 2
    \end{cases}\]
    \begin{enumerate}
        \item Trovare la densità superficiale $\rho$. 
        \item Calcolare il momento di inerzia rispetto all'asse $z$. $\\$
    \end{enumerate}

    %ES. 7
    \item Una carica elettrica è distribuita uniformemente sulla superficie sferica: 
    \[x^2+y^2+z^2=R^2\]
    Calcolare il potenziale elettrostatico da essa generato fuori e dentro la sfera, 
    in un punto $P(0,0,z)$ sull'asse $z$. \\
    \textbf{Suggerimento:} Il potenziale elettrostatico in $P$ è:
    \[U(0,0,z) = -k \ \iint_\Sigma \frac{d\sigma}{r}\]
    con: 
    \begin{itemize}
        \item $k$ è costante
        \item $\Sigma$ è una superficie sferica
        \item $\rho$ è la densità di carica uniforme
        \item $r$ è la distanza di un punto da $P\\$ 
    \end{itemize}

    %ES. 8
    \item Determinare il momento di inerzia di una lamina omogenea di densità costante $\delta$
    e superficie $\Sigma$ rispetto alla retta intersezione dei piani $y=1$ e $z=0$. 
    \[\Sigma = \left\{(x,y,z)\in\R^3: \ \begin{matrix}
        x^2+y^2=1 \\ 0 \leq z \leq 2
    \end{matrix}\right\}\]
    $\\$

    %ES. 9
    \item Calcolare il flusso del campo:
    \[\vF (x,y,z)= \begin{pmatrix}
        1\\ 0\\ 1
    \end{pmatrix}\]
    attraverso $\Si$ descritta dalle seguenti relazioni:
    \[\vr (\mu,\nu) = \begin{pmatrix}
        \mu^2 \\ \sqrt{2} \ \mu\nu \\ \nu^2
    \end{pmatrix} \ \text{con} \ (\mu,\nu)\in \mathbb{T}\]
    \[\mathbb{T} = \left\{\begin{matrix}
        1 \leq \mu^2+\nu^2 < 2 \\ \mu < \nu
    \end{matrix}\right\}\\\]


    %ES. 10
    \item Calcolare il lavoro del campo:
    \[\vF = \begin{pmatrix}
        y+z \\ z+x \\ x-y
    \end{pmatrix}\]
    lungo la circonferenza $\gamma$ intersezione tra la superficie sferica $x^2+y^2+z^2=1$ e il
    piano $z=y$.\\
    \textbf{Suggerimento:} Sia
    \[\gamma_t = \begin{pmatrix}
        \cos{(t)} \\ 
        \frac{\sin{(t)}}{\sqrt{2}}\\
        \frac{\sin{(t)}}{\sqrt{2}}
    \end{pmatrix}\]
    $\gamma$ è $\partial\Sigma$ dove $\Sigma$ è l'intersezione tra la sfera e il piano.\\
    Per il teorema di Stokes posso dire: 
    \[\mathcal{L}_{\mathbb{E},\gamma} = \int_\gamma \vec{F} \cdot ds = \int_{\partial^+\Sigma} \vec{F}\cdot\vec{t} \ ds\\\]

    %ES. 11
    \item Calcolare il momento di inerzia della superficie $\Si$ attorno all'asse $z$, dove $\Si$ è data 
    dall'intersezione tra il cilindro $x^2+y^2=1$ e i piani $z=0$ e $z=1+x$, comprese le basi superiore 
    e inferiore.$\\$

    %ES. 12
    \item Calcolare il flusso del campo:
    \[\vF = \begin{pmatrix}
        z\\ y\\ x
    \end{pmatrix}\]
    attraverso la sfera:
    \[x^2+y^2+z^2=1\\\]

    %ES. 13
    \item La temperatura di una sfera metallica è proporzionale al quadrato della distanza dal centro
    del corpo. \\
    Trovare il flusso della quantità di calore che attraversa una sfera $S$ di raggio $a$, centrata nel
    centro della sfera metallica.$\\$

    %ES. 14
    \item Calcolare il lavoro del campo:
    \[\vF = \begin{pmatrix}
        -z\\ x\\ y
    \end{pmatrix}\]
    lungo la curva $\gamma$ data dall'intersezione del piano $z=y$ con il paraboloide $z=x^2+y^2$
\end{enumerate}

\newpage

\section{Foglio n.2}

\begin{enumerate}
    %ES. 1
    \item Calcolare $A(\de\mathbb{E})$, dove $\mathbb{E}$ è l'intersezione tra due palle di raggio 2
    i cui centri sono posti a distanza 3. $\\$

    %ES. 2
    \item Sia $\gamma: [1,2]\to\R^3$ t.c.:
    \[\vec{\gamma}(t) = \begin{pmatrix}
        t\\ t^2\\ t
    \end{pmatrix}\]
    di sostegno $\Gamma$. \\
    Sia $\Si$ l'insieme dei segmenti che congiungono i punti di $\Gamma$ con le loro proiezioni
    sul piano $z=0$.
    \begin{enumerate}
        \item Provare che $\Si$ è una superficie regolare.
        \item Calcolare $A(\Si)$.\\
    \end{enumerate}

    %ES. 3
    \item Calcolare:
    \[\iint_\Si g(x,y,z) \ d\sigma \ \text{su} \ \Si = \left\{\begin{matrix}
        z=x^2+y^2\\ x^2+y^2 \leq 1
    \end{matrix}\right\}\]
    con:
    \[g(x,y,z) = \frac{y}{\sqrt{1+4z}}\]$\\$

    %ES. 4
    \item Sia:
    \[\mathbb{E} = \left\{(x,y,z)\in\R^3: \begin{matrix}
        x^2+y^2 \leq 1 \\ 0 \leq 2z \leq 4+x+y
    \end{matrix}\right\}\]
    e sia $\Si\equiv\de\mathbb{E}$. \\
    Calcolare: 
    \[\iint_\Si xyz \ d\sigma\]\\

    %ES. 5
    \item Sia $\mathbb{E}$ l'intersezione di un cono circolare retto pieno di apertura $\frac{\pi}{3}$
    e di una palla centrata nel vertice del cono e avente ragio $\frac{3}{2}$. \\
    Calcolare l'area di $\de\mathbb{E}$.$\\$

    %ES. 6
    \item Sia: 
    \[\vec{\gamma}: \ \begin{cases}
        x = \rho (t) \\
        z = z(t)
    \end{cases}\]
    con $t\in\mathbb{I}$, una curva nel piano $y=0$. \\
    Calcolare l'area della superficie ottenuta ruotando $\gamma$ attorno all'asse $z$.$\\$

    %ES. 7
    \item Svolgere l'esercizio (6) nel caso in cui la curva è grafico di una funzione nella forma
    $x = f(z)\\$

    %ES. 8
    \item Calcolare l'area della superficie di un toro. $\\$
    
    %ES. 9
    \item Sia $\mathbb{F}$ la porzione di superficie sferica $x^2+y^2+z^2=1$ interna al cilindro: 
    \[\left(x-\frac{1}{2}\right)^2 + \ y^2 = \frac{1}{4}\]
    Calcolare l'area di $\mathbb{F}\\$

    %ES. 10
    \item Calcolare l'area della superficie $\Si: \ z=y$ che si proietta sull'insieme $\mathbb{T}$, dove:
    %immagine
    \begin{center}
        \includegraphics[height=4.5cm,width=8cm]{Screenshot 2025-12-06 alle 12.23.18.jpeg}
    \end{center}

    \[\vec{\gamma} (t) = \begin{pmatrix}
        \cos(t) \\ \sin(t)
    \end{pmatrix}_{t\in \left[0,\frac{\pi}{2}\right]} \quad \text{e} \quad \vec{\Gamma} (t) = \begin{pmatrix}
        t-\sin(t) \\ 1+\cos(t)
    \end{pmatrix}_{t\in [-\pi,\pi]}\]$\\$

    %ES. 11
    \item Calcolare l'area del paraboloide tronco descritto dalle equazioni:
    \[z=x^2+y^2\]
    \[0 \leq z \leq 1\]$\\$

    %ES. 12
    \item Sia $\mathbb{P}$ un paraboloide pieno di equazione $z \geq x^2+y^2$.\\
    Sia $\mathbb{B}$ una palla di equazione $x^2+y^2+z^2=1$.\\
    Sia $\mathbb{E} = \mathbb{P} \cap \mathbb{B}$.\\
    Sia $\Si \equiv \de\mathbb{E}$.\\
    Calcolare il baricentro di $\Si\\$

    %ES. 13
    \item Sia $\Si$ una superficie in $\R^3$ t.c.:
    \[\Si = \left\{(x,y,z)\in\R^3: \ 2x=y^2\right\}\]
    Calcolare l'area della porzione di $\Si$ contenuta all'interno dell'insieme $\mathbb{G}$:
    \[\mathbb{G} = \left\{(x,y,z)\in\R^3: \ 0 \leq z \leq 4-y^2\right\}\]

    %ES. 14
    \item Calcolare l'area di $\de\mathbb{E}$ dove:
    \[\mathbb{E} = \left\{(x,y,z)\in\R^3: \ \begin{matrix}
        4x^2+y^2+z^2 \geq y+1 \\ 0 \leq y \leq 2-2\sqrt{4x^2+z^2}
    \end{matrix}\right\}\]$\\$

    %ES. 15
    \item Sia:
    \[\vF (x,y,z) = \begin{pmatrix}
        x\\ -(2x+y)\\ z
    \end{pmatrix}\]
    il vettore densità di flusso di un fluido. \\
    Sia:
    \[\mathbb{S} = \begin{cases}
        x^2+y^2+z^2=1 \\ z\geq 0
    \end{cases}\]
    Calcolare la massa di fluido passante attraverso $\mathbb{S}$ nell'unità di tempo, nel verso di
    \underline{n} normale esterna. \\

    %ES. 16
    \item Un guscio sferico omogeneo è tagliato da un cono circolare retto il cui vertice è il centro
    della sfera, e l'angolo al vertice del cono è $\alpha$, con $0<\alpha<\pi$. \\
    Determinare, in funzione di $(R,\alpha)$, il centro di massa del guscio che si trova dentro il cono
\end{enumerate}
\end{document}
\documentclass{article}
\usepackage{amsmath}
\usepackage{amssymb}

%comandi utili
\newcommand\R{\mathbb{R}}
\newcommand\vu{\underline{\nu}}
\newcommand\D{\mathbb{D}}
\newcommand\se{\text{se}}
\newcommand\de{\partial}

\title{Analisi II - Derivabilità, Differenziabilità, Piano tangente}
\author{Marco Delton\thanks{esercizi dei prof. \textit{Gabriele Bianchi}, \textit{Chiara Bianchini} e \textit{Luca Bisconti}}}
\date{A.A. 2025/26}

\begin{document}
\maketitle

\section{Foglio n. 1}
\begin{enumerate}
    %1
    \item Determinare per quale $\alpha\in\R$ il piano tangente al grafico di:
    \[z=\sin\left(\alpha x+y^2\right)\]
    nel punto $(0,\sqrt{\pi},0)$ è parallelo alla retta $x=y=2z$ (in $\R^3$). \\
    Esistono valori di $\alpha\in\R$ per cui è perpendicolare?

    %2
    \item Data:
    \[f(x,y) = e^{2x-y} + \sqrt{3+x^2+3y^2}\]
    \begin{enumerate}
        \item Verificare che $f$ è differenziabile in $(1,2)$
        \item Scrivere l'equazione del piano tangente al grafico di $f$ in $(1,2,5)$
        \item Calcolare $\nabla_{\vu}f(1,2)$ dove $\vu$ è il versore della retta $y=\sqrt{3} \ x$ orientato
        nel verso delle $x$ decrescenti
    \end{enumerate}

    %3
    \item Data:
    \[f(x,y) = x \ e^y -2\ln(x) +3y\]
    Stabilire se $f$ si trova al di sopra o al di sotto del piano tangente in un intorno
    del punto $(1,0,1)$

    %4
    \item Sia $z=f(r)$ con $r=\sqrt{x^2+y^2}$, sia $g:\R^2\to\R$ differenziabile e t.c. 
    $g(x,y)=f(r)$. \\
    Scrivere $f'(r)$ in termini del seguente dominio $\D$:
    \[\D = \begin{cases}
        x=r \ \cos(\theta) \\
        y=r \ \sin(\theta)
    \end{cases}\]
    \fbox{\tiny\textbf{Esempio:} $g(x,y) = x^2+y^2 = r^2 = f(r)$}

    %5
    \item Data:
    \[f(x,y) = x^y-2y+2x\]
    Determinare per quale direzione $\vu = \begin{pmatrix}
        \cos(\alpha) \\ \sin(\alpha)
    \end{pmatrix}$ si ha $\nabla_{\vu}f(1,1)=2$.\\
    Determinare qual è la direzione lungo la quale $f$ cresce maggiormente in un intorno di $(1,1)$

    %6
    \item Data 
    \[u(x,t) = \frac{1}{\sqrt{t}} \ e^{-\frac{x^2}{4t}}\] 
    Verificare che soddisfa l'equazione $u_t-u_{xx}=0 \ \forall t>0, \ \forall x\in\R$\\
    \fbox{\tiny\textbf{Curiosità:} Questa equazione è detta \textit{equazione del calore}}

    %7
    \item Data:
    \[f(x,y) = \sqrt[3]{x^2(y-1)}+1\]
    Provare che non è differenziabile in $P_0\equiv(0,1)$.\\
    Inoltre, $\forall\vu\in\R^2$ versore, calcolare $\nabla_{\vu}f(0,1)$
\end{enumerate}

\newpage

\section{Foglio n.2}
Stabilire se le seguenti funzioni sono differenziabili nel loro dominio
\begin{enumerate}
    \item $f(x,y) = \begin{cases}
        \frac{\sqrt{\sin(x^2y^2)}}{|xy|} & \se \ xy\ne 0 \\
        1 & \se \ xy=0 
    \end{cases}$

    \item $f(x,y) = \begin{cases}
        x^2y^2\sin\left(\frac{1}{x^2y^2}\right) & \se \ xy \ne 0 \\
        0 & \se \ xy=0
    \end{cases}$

    \item $f(x,y) = \begin{cases}
        \left(\frac{x^2y}{x^4+y^2}\right) & \se \ (x,y) \ne (0,0) \\
        0 & \se \ (x,y)=(0,0)
    \end{cases}$\\
    Verificare che $\exists\frac{\de}{\de\vu}(0,0) \ \forall\vu$ direzione, ma $f$ 
    non è differenziabile in $(0,0)$

    \item $f(x,y) = \begin{cases}
        \frac{x^2y}{x^2+y^2} & \se \ (x,y) \ne (0,0) \\
        0 & \se \ (x,y)=(0,0)
    \end{cases}$

    \item $f(x,y) = \begin{cases}
        \frac{xy(x^2-y^2)}{x^2+y^2} & \se \ (x,y) \ne (0,0) \\
        0 & \se \ (x,y)=(0,0)
    \end{cases}$

    \item $f(x,y) = \begin{cases}
        y^2\arctan\left(\frac{x}{y}\right) & \se \ y\ne 0 \\
        0 & \se \ y=0
    \end{cases}$

    \item Data:
    \[f(x,y) = \frac{1}{xy}\]
    Si assuma che per $xy \ne 0$ è differenziabile. \\
    Trovare il piano tangente a $z=f(x,y)$ in $(1,1,1)$

    \item Data:
    \[f(x,y) = x^2y^2\ln(x^2+y^2) \ \quad\se \ (x,y) \ne (0,0)\]
    E' possibile estendere $f$ con continuità in (0,0)? \\
    La funzione ottenutà è differenziabile in (0,0)?

    \item Data:
    \[f(x,y) = \begin{cases}
        \frac{xy \ \sin(x)}{x^2+\arcsin^2(y)} & \se \ (x,y) \ne (0,0) \\
        0 & \se \ (x,y)=(0,0)
    \end{cases}\]
    \begin{enumerate}
        \item Calcolare in $(0,0)$ le derivate parziali e la derivata direzionale $\frac{\de f}{\de\vu}$ dove
        $\vu=\left(\frac{1}{\sqrt{2}},\frac{1}{\sqrt{2}}\right)$
        \item Dire se $f$ è differenziabile in (0,0).
    \end{enumerate}
\end{enumerate}

\newpage

\section{Foglio n.3}
\begin{enumerate}
    \item Studiare la differenziabilità in $(0,0)$ della funzione:
    \[f(x,y) = |x| \ \ln(1+y)\]

    \item Studiare la differenziabilità in ogni $(x,y)\in\R^2$ della funzione:
    \[f(x,y) = \begin{cases}
        \frac{xy^2}{y^2+|x|} & \se \ (x,y)\ne (0,0) \\
        0 & \se \ (x,y)=(0,0)
    \end{cases}\]

    \item Studiare la differenziabilità in $(0,0)$ della funzione:
    \[f(x,y) = \sqrt{|xy|}\]

    \item Calcolare le derivate direzionali di:
    \[f(x,y) = \begin{cases}
        \frac{x^2y}{x^2+y^2} & \se \ (x,y) \ne (0,0) \\
        0 & \se \ (x,y)=(0,0)
    \end{cases}\]
    nel punto $(0,0)$ lungo una generica direzione $\vu = (h,k)$, con $h^2+k^2=1$. \\
    Studiare la continuità di $f(x,y)$ nell'origine degli assi

    \item Calcolare le derivate direzionali di:
    \[f(x,y) = \begin{cases}
        \frac{x}{y^2} \ (x^2+y^2) & \se \ y\ne 0 \\
        0 & \se \ y=0
    \end{cases}\]
    nel punto $(0,0)$ lungo una generica direzione $\vu = (h,k)$, con $k\ne 0$ e $h^2+k^2=1$

    \item Studiare la differenziabilità della funzione:
    \[f(x,y) = \left[x^2(y-1)\right]^{\frac{1}{3}}+1\]
    nel punto $(0,1)$. \\
    Calcolare, se esistono, le derivate direzionali $\frac{\de f}{\de\vu}(0,1)$, dove $\vu$ è 
    un versore ($\Vert\vu\Vert=1$)

    \item Studiare la differenziabilità in $(0,0,0)$ della funzione:
    \[f(x,y,z) = \begin{cases}
        |x| \ \sqrt{x^2+y^2+z^2} \ \sin\left(\frac{1}{\sqrt{x^2+y^2+z^2}}\right) & \se \ (x,y,z)\ne (0,0,0) \\
        0 & \se \ (x,y,z)=(0,0,0)
    \end{cases}\]
\end{enumerate}
\end{document}
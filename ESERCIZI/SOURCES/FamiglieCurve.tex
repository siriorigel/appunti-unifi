\documentclass{article}
\usepackage{graphicx} % Required for inserting images
\usepackage{amsmath}
\usepackage{amssymb}

\title{Analisi II - Famiglie di curve}
\author{Marco Delton\thanks{esercizi dei prof. \textit{Gabriele Bianchi} e \textit{Chiara Bianchini}}}
\date{a.a. 2025/26}

\begin{document}

\maketitle

\begin{enumerate}
    \item Trovare le curve piane regolari tali che: sia $(x,y)$ un punto qualsiasi della curva, 
    la distanza tra $(x,y)$ e il punto in cui la retta tangente alla curva in $(x,y)$ incontra l'asse $y$ è uguale alla distanza tra quest'ultimo punto e il punto $(1,0)$\\

    \item Condsiderare le parabole che hanno un dato asse e un dato fuoco. Determinare le traiettorie ortogonali di tali parabole\\

    \item Trovare le curve piane regolari che escono dall'origine e hanno questa proprietà: la lunghezza dell'arco tra l'origine e un punto qualsiasi della curva è il doppio della distanza tra l'origine e il punto medesimo \\

    \item Trovare le curve piane regolari tali che: per ogni punto $P(x,y)$ della curva, il quadrato della lunghezza dell'arco compreso tra $P$ e il punto in cui la curva incontra l'asse $y$ è pari alla metà dell'ordinata di $P$\\

    \item Trovare le traiettorie ortogonali alle circonferenze di centro fissato\\

    \item Dati un punto $P$ e una retta $r$, passante per $P$, considerare le circonferenze che passano da $P$ e hanno centro su $r$. Determinare le traiettorie ortogonali a tali circonferenze\\

    \item Trovare le linee di massima pendenza della funzione:
    \[u(x,y)=x(x^2-3y^2)\]
\end{enumerate}

\end{document}

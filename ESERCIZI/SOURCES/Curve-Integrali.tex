\documentclass{article}
\usepackage{graphicx} % Required for inserting images
\usepackage{amsmath}
\usepackage{amssymb}

\title{Analisi II - Curve e Integrali curvilinei}
\author{Marco Delton\thanks{esercizi della prof.ssa \textit{Chiara Bianchini}}}
\date{A.A. 2025/26}

\begin{document}

\maketitle

\section{Curve parametriche}
\begin{enumerate}
    %ES.1 scheda
    \item Un punto si muove nello spazio secondo la legge
    \[\vec{r}(t)=\begin{pmatrix}
        \cos(t)\\
        \sin(t)\\
        t
    \end{pmatrix}_{t\in [0,6\pi]}\]
    \begin{enumerate}
        \item Calcolare la velocità e l'accelerazione
        \item Verificare che i vettori velocità e accelerazione centripeta sono perpendicolari fra loro, e che l'accelerazione è rivolta "internamente". 
        \item Calcolare lo spazio percorso
    \end{enumerate}$\\$

    %ES. 2 scheda
    \item Data
    \[\vec{r}(t)=\begin{pmatrix}
        2\sin(t)\\
        -3\cos(t)\\
        4t
    \end{pmatrix}_{t \in \mathbb{R}}\]
    \begin{enumerate}
        \item Verificare che $\vec{r}$ definisce una curva regolare
        \item Verificare che $\vec{r}$ passa dal punto $P(0,-3,0)$
        \item Calcolare la retta tangente a $\vec{r}$ in $P\\$
    \end{enumerate}
    
    %ES. 3 scheda
    \item Calcolare la lunghezza:
    \begin{enumerate}
        \item dell'\textbf{astroide} di equazione
        \[\vec{r}(t)=\begin{pmatrix}
            \cos^3(t)\\
            \sin^3(t)
        \end{pmatrix}_{t\in [0,2\pi]}\\\]

        \item del \textbf{cardioide} di equazione polare
        \[\rho (\theta)=1+\cos{\theta} \quad \theta \in [0,2\pi]\\\]

        \item dell'\textbf{elica conica} di equazione
        \[\vec{r}(t)=\begin{pmatrix}
            t \ \cos(t)\\
            t \ \sin(t)\\
            t
        \end{pmatrix}_{t\in [0,6\pi]}\\\]
    \end{enumerate}

    %ES. 4 scheda
    \item Trovare i valori di $\alpha \in \mathbb{R}$ per cui
    \[\vec{\gamma}(t)=\begin{cases}
        t^2\underline{i}+\alpha t\underline{j} & \text{se } t \leq 0\\
        t\underline{j}+t^2\underline{k} & \text{se } t>0
    \end{cases}\]
    è regolare su $\mathbb{R}\\$

    %ES. 5 scheda
    \item Scrivere una parametrizzazione della linea $\gamma$ appartenente alla superficie $z=\sqrt{2y^2-x}$ che si proietta nel piano $xy$ nella linea di equazione
    \[x=y^2\]
    Dire se tale curva è regolare su $\mathbb{R}\\$

    %ES. 6 scheda
    \item Calcolare la lungezza della curva intersezione delle superfici
    \[y=x^2e\]
    \[3z=2xy\]
    dal punto $(0,0,0)$ al punto $\left(2,4,\frac{16}{3}\right)\\$

    %ES. 7 scheda
    \item Trovare l'equazione parametrica della curva ottenuta come intersezioni tra
    \begin{enumerate}
        \item il cilindro $x^2+y^2=1$
        \item il piano $y+z=2$
    \end{enumerate}
    Calcolare la retta tangente alla curva nel punto $(-1,0,2)$\\
    
    %ES. 8 scheda
    \item Dimostrare che la curva di equazione
    \[\vec{x}(t)=\begin{pmatrix}
        t \ \cos(t)\\
        t \ \sin(t)\\
        t
    \end{pmatrix}\]
    giace sul cono di equazione $z^2=x^2+y^2\\$

    %ES. 9 scheda
    \item Data la curva di equazione
    \[\vec{x}(t)=\begin{pmatrix}
        \sin(t)\\
        \cos(t)\\
        \sin^2(t)
    \end{pmatrix}\]
    \begin{enumerate}
        \item Dimostrare che è l'intersezione delle superfici $z=x^2$ e $x^2+y^2=1$
        \item Dimostrare che è regolare
        \item Trovare la retta tangente a $\vec{x}(t)$ la cui direzione è parallela a $\vec{v}=\left(1,-1,\sqrt{2}\right)$. Specificarne l'equazione e il punto di tangenza $\\$
    \end{enumerate}

    %ES. 10 scheda
    \item Determinare l'angolo di intersezione tra l'origine e le curve:
    \[\vec{r_1}(t)=\begin{pmatrix}
        t\\
        t^2\\
        t^3
    \end{pmatrix}\]
    \[\vec{r_2}(t)=\begin{pmatrix}
        \sin(t)\\
        \sin(2t)\\
        t
    \end{pmatrix}\\\]

    %ES. 11 scheda
    \item Provare che se il vettore posizione $\vec{r}(t)$ è sempre perpendicolare al vettore tangente $\dot{\vec{r}}(t)$, allora la curva giace su una sfera centrata in $O\\$

    %ES. 12 scheda
    \item Riparametrizzare le seguenti curve rispetto all'ascissa curvilinea misurata a partire dal punto in cui $t=0$, nella direzione delle $t$ crescenti:
    \begin{enumerate}
        \item $\vec{r}(t)=\begin{pmatrix}
            e^t\sin(t)\\
            e^t\cos(t)
        \end{pmatrix}$
        
        \item $\vec{\gamma}(t)=\begin{pmatrix}
            1+2t\\
            3+t\\
            -5t
        \end{pmatrix}$

        \item $\vec{x}(t)=\begin{pmatrix}
            3\sin(t)\\
            4t\\
            3\cos(t)
        \end{pmatrix}$
    \end{enumerate}
\end{enumerate}

\newpage

\section{Integrali curvilinei}
\begin{enumerate}
    %ES. 1 scheda
    \item Calcolare l'area della superficie parallela all'asse $z$ compresa tra il piano $z=0$ e il grafico della funzione $f(x,y)=xy$ che interseca il piano $z=0$ lungo l'arco di parabola
    \[\vec{r}(t)=\begin{pmatrix}
        t\\
        t^2
    \end{pmatrix}_{t\in[0,1]\\}\]

    %ES. 2 scheda
    \item Calcolare il momento di inerzia rispetto all'asse $z$ di un filo di densità lineare di massa $\delta = cost.$, avente la forma di un'elica 
    \[\vec{x}(t)=\begin{pmatrix}
        R \ \cos(t)\\
        R \ \sin(t)\\
        t
    \end{pmatrix}_{t\in [0,4\pi]\\}\]

    %ES. 3 scheda
    \item Calcolare il baricentro di un filo omogeneo avente la forma del cicloide di equazione
    \[\vec{\gamma}(t)=\begin{pmatrix}
        \alpha\left[t-\sin(t)\right]\\
        \alpha \left[1-\cos(t)\right]
    \end{pmatrix}_{t\in [0,2\pi]} \quad \text{con } \alpha >0\\\]

    %ES. 4 scheda
    \item Calcolare la massa di un filo con densità lineare di massa $\delta (x,y,z)=x$ avente la forma dell'elica
    \[\vec{r}(t)=\begin{pmatrix}
        4t\\
        -3\cos(t)\\
        3\sin(t)
    \end{pmatrix}_{t\in [0,\pi]}\\\]

    %ES. 5 scheda
    \item Calcolare
    \[_\gamma\int f(x,y,z)dS \quad \text{con }f(x,y,z)=\sqrt{2y^2+z^2}\]
    e $\gamma$ è la circonferenza intersezione tra $x^2+y^2+z^2=4$ e $x=y\\$

    %ES. 6 scheda
    \item Calcolare
    \[_\gamma\int \left(x^2+y^2\right)^2dS\]
    dove $\gamma$ è l'equazione polare $\rho (\theta)=e^{2\theta}$ con $\theta \in (-\infty,0]\\$

    %ES. 7 scheda
    \item Calcolare
    \[_C\int 2xdS\]
    dove $C$ è composta dall'arco di parabola $y=x^2$ da $(0,0)$ a $(1,1)$ e dal segmento verticale da $(1,1)$ a $(1,2)\\$

    %ES. 8 scheda
    \item Trovare il lavoro compiuto dalle forze del campo
    \[\vec{F}(x,y)=\begin{pmatrix}
        x^2\\
        -xy
    \end{pmatrix}\]
    per spostare una particella lungo il primo quarto di circonferenza unitaria $\\$

    %ES. 9 scheda
    \item Trovare il lavoro compiuto dal campo di forze 
    \[\vec{F}(x,y)=\begin{pmatrix}
        x\sin(y)\\
        y
    \end{pmatrix}\]
    su una particella che viene spostata lungo la parabola $y=x^2$ da $(-1,1)$ a $(2,4)\\$

    %ES. 10 scheda
    \item Un uomo del peso di 80 kg trasporta una latta del peso di 10 kg lungo una scala elicoidale che circonda un silos cilindrico di 6 m di diametro. Il silos è alto 4,5 m, e occorrono esattamente 3 giri per arrivare alla sommità. \\
    Calcolare il lavoro compiuto dall'uomo contro la forza di gravità.\\

    %ES. 11 scheda
    \item Calcolare
    \[_C\int 2x\sin{y}\textbf{dx}+\left(x^2\cos{y}-3y^2\right)\textbf{dy}\]
    dove $C$ è il segmento che unisce i punti $(-1,0)$ e $(5,1)\\$

    %ES. 12 scheda
    \item Calcolare il lavoro fatto dal campo di forze $\vec{F}$ per spostare un oggetto lungo il segmento $\overline{PQ}$
    \begin{enumerate}
        \item $\vec{F}(x,y)=\begin{pmatrix}
            x^2y^3\\
            x^3y^2
        \end{pmatrix} \quad \text{con } P(0,0),\quad Q(2,1)$
        \item $\vec{F}(x,y)=\begin{pmatrix}
            \frac{y^2}{x^2}\\
            -\frac{2y}{x}
        \end{pmatrix} \quad \text{con } P(1,1),\quad Q(4,-2)$
    \end{enumerate}
\end{enumerate}

\end{document}

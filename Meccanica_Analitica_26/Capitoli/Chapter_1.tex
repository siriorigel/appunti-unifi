\section{Vettori in $\mathbb{R}^3$}
Se vogliamo descrivere un punto nello spazio, è necessario definire cos'è lo spazio:

Prendiamo una terna destrorsa di assi cartesiani $\Rightarrow$ Descrivo $\mathbb{R}^3$.

\begin{center}
    \tdplotsetmaincoords{70}{110}

    \begin{tikzpicture}[tdplot_main_coords, scale=3]

        % Assi cartesiani
        \draw[->, thick] (0,0,0) -- (1.5,0,0) node[anchor=north east]{$x$};
        \draw[->, thick] (0,0,0) -- (0,1.5,0) node[anchor=north west]{$y$};
        \draw[->, thick] (0,0,0) -- (0,0,1.5) node[anchor=south]{$z$};

        % Coordinate del punto A
        \def\xa{1.2}
        \def\ya{0.8}
        \def\za{1.0}

        % Punto A
        \filldraw[ocra] (\xa,\ya,\za) circle (0.8pt)
            node[anchor=south west] {$A(\textcolor{red}{\xa},\textcolor{blue}{\ya},\textcolor{green!60!black}{\za})$};

        % Vettore r = OA
        \draw[->, very thick, purple] (0,0,0) -- (\xa,\ya,\za)
            node[midway, above left] {$\vec r$};

        % Proiezione su piano xy
        \draw[dashed] (\xa,\ya,\za) -- (\xa,\ya,0);
        \draw[dashed] (\xa,\ya,0) -- (\xa,0,0);
        \draw[dashed] (\xa,\ya,0) -- (0,\ya,0);

        % Componenti cartesiane colorate
        \draw[very thick, red] (0,0,0) -- (\xa,0,0)
            node[midway, below] {$\textcolor{red}{x_A}$};

        \draw[very thick, blue] (\xa,0,0) -- (\xa,\ya,0)
            node[midway, right] {$\textcolor{blue}{y_A}$};

        \draw[very thick, green!60!black] (\xa,\ya,0) -- (\xa,\ya,\za)
            node[midway, right] {$\textcolor{green!60!black}{z_A}$};

    \end{tikzpicture}
\end{center}

Considero un vettore $\vec r = (O-A)$, esso non è altro che una terna di valori ordinati $\vec r = (x_A, y_A, z_A)$.

Definisco il \textbf{modulo}, o lunghezza, di un vettore come: $\left | \vec v\right |=\sqrt{x^2 + y^2 +z^2} $

Se $\alpha \in \mathbb{R}$, posso definire $\vec u = \alpha \vec v$, e sarà un vettore che giace sulla stessa retta (ha la stessa direzione) di $\vec v$, verso come $sgn (\alpha)$ e modulo $\left | \vec u\right |=\left |\alpha \right |\left |\vec v\right |$.

Farà anche comodo definire un \textbf{Versore}, cioè un vettore con modulo unitario, che indicherò con $\hat u$.

\begin{example}
    Se volessi individuare $\hat v // \vec v$ è sufficiente fare: $\hat v = \frac{\vec v}{|\vec v|}$
\end{example}

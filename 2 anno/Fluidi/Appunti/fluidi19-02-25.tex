\documentclass[a4paper, oneside]{article}
\usepackage{graphicx}
\usepackage{amsthm}
\usepackage{amsmath}
\usepackage{amssymb}
\usepackage[a4paper,
            bindingoffset=0.2in,
            left=2cm,
            right=2cm,
            top=2cm,
            bottom=2cm,
            footskip=.25in]{geometry}
\usepackage[italian]{babel}
\usepackage{pgfplots}
\usepackage{tabularx}
\usepackage{tikz}
\usepackage{wrapfig}
\usepackage{color}
\usepackage[d]{esvect}
\usepackage{chemfig}
\usepackage{mhchem}
\definecolor{page}{rgb}{0.129,0.157,0.212}
\pagecolor{page}
\color{white}
\graphicspath{ {./images/} }
\usetikzlibrary{shapes.geometric}
\usetikzlibrary{datavisualization}
\usetikzlibrary{datavisualization.formats.functions}
\usetikzlibrary{patterns}
\pgfplotsset{width=10cm,compat=1.18}

\title{Appunti di Fluidodinamica (Landi)}
\author{Tommaso Miliani}
\date{19-12-25}

\begin{document}
\newtheoremstyle{theoremEnv}
                {}          % Space above
                {}          % Space below
                {\slshape}  % Body font
                {}          % Indent amount
                {\bfseries} % Head font
                {.}         % Punctuation after head
                {\newline}  % Space after theorem head
                {}          % Theorem head spec
\theoremstyle{theoremEnv}

\newtheorem{definition}{Definizione}[section]
\newtheorem{theorem}{Teorema}[section]
\newtheorem{lemma}{Proposizione}[section]
\newtheorem{observation}{Osservazione}[section]
\newtheorem{corollary}{Corollario}[theorem]
\newtheorem{example}{Esempio}[section]
\newtheorem{remark}{Enunciato}[section]

\maketitle

\section{Fluidi reali}
\begin{wrapfigure}{r}{0.4\textwidth}
    \centering
    \caption{}
    \begin{tikzpicture}
        \draw[thick](0, 2) -- (3, 2);
        \draw[thick](0, 0) -- (3, 0);
        \draw[->](3, 2.5) -- (4, 2.5) node[at end, above] {$\vv{F}$};
        \draw[->](1.5, 1.5) -- (3, 1.5);
        \draw[->](1.5, 1) -- (2.5, 1);
        \draw[->](1.5, 0.5) -- (2, 0.5);
    \end{tikzpicture}    
\end{wrapfigure}
Presi due piani fermi e posto un fluido tra questi due 
piani (distanziati di $D$), si comincia ad imprimere una forza $\vv{F}$
ad uno dei due piani. Se il fluido fosse ideale, starebbe fermo
e non succederebbe assolutamente niente. Dato che il fluido non è 
ideale, esso si mette in movimento e comincia ad avere una certa
velocità, che dipende dalla distanza dalla parete in movimento.
QUesto effetto è dato dall'attrito della parete sul fluido.
\begin{gather*}
    F \propto \eta\frac{S\Delta u}{D}
\end{gather*}
Dove $\eta$ è il \textbf{coefficiente di attrito dinamico}
dove $\Delta u$ è la differenza di velocità tra il fluido sulla parete che
si muove e il fluido sulla parete che non si muove. Si può anche scrivere la 
pressione del fluido come
\begin{gather*}
    \frac{F}{S} = \eta \frac{\Delta u}{D}
\end{gather*}
Immaginando di avere invece due elementi fluidi uno sopra l'altro, uno 
dei due fluidi accelera l'altro. Si può dire che la pressione esercitata
dal fluido sarà sostanzialmente
\begin{gather*}
    \frac{\vv{F} }{S} = \eta \frac{\Delta u_x}{D} \hat{x} 
\end{gather*}
Si può dunque dire che il contributo elementare della forza è dato da
\begin{gather*}
    d\vv{F}  = \eta \frac{du_x}{dz} \hat{x} \ d\sigma  = \eta (\hat{z} \cdot  \vv{\nabla}  )\vv{u} \ d\sigma 
\end{gather*}
Il fluido reale è dunque un fluido nel quale non si trascurano le forze di interazione del fluido. 

\section{Calcoli sul fluido reale (DA RIVEDERE)}
Il gradiente di un vettore prende il nome di tensore 
\begin{gather*}
    \vv{\vv{\Pi} } = \begin{pmatrix}
        \Pi_{x x} & \Pi_{x y} & \Pi_{x z} \\
        \Pi_{y x} & \Pi_{y y} & \Pi_{y z} \\
        \Pi_{z x} & \Pi_{z y} & \Pi_{z z} \\
    \end{pmatrix} 
\end{gather*}
Questo tensore mi garantisce che il suo prodotto scalare con un 
versore qualsiasi della terna cartesiana destrorsa mi da un vettore
le cui componenti sono 
\begin{gather*}
    \vv{\vv{\Pi} } \hat{z} = \Pi_{x z} \hat{x} + \Pi_{y z} \hat{y} + \Pi_{z z} \hat{z}     
\end{gather*}
Ossia che ha tutte e tre le componenti.  Supponendo di avere un versore $\hat{n}$ in un 
qualunque sistema di riferimento e si considera il prodotto scalare
\begin{gather*}
    p\vv{\vv{\Pi} } \hat{n} = \begin{pmatrix}
        p & 0 & 0 \\
        0 & p & 0 \\
        0 & 0 & p
    \end{pmatrix} \begin{pmatrix} \hat{n_x} \\
    \hat{n_y} \\
    \hat{n_z}    \end{pmatrix}  = pn_x \hat{x} + pn_y \hat{y} + pn_z\hat{z}    
\end{gather*} 
Ossia la forza parallela al vettore $\hat{n}$ utilizzando il tensore.  Si utilizza questo sistema poiché in questo
modo si può considerare una forzaz che agisce su di una superficie in modo tale che
la risultante delle forze sulla superficie è 
\begin{gather*}
    \int_{\Sigma (V)}^{} p\hat{n} \ d\sigma = \int_{}^{}  \vv{\vv{\Pi} } \hat{n} \ d\sigma =  
\end{gather*}
Si utilizza la stessa strategia usata per calcolare la forma differenziale per la pressione. Utilizzando il 
solito sistema di riferimento, si considera un cubo di fluido e si iniziano a determinare le forze
che agiscono sul fluido sui vari piani
\begin{gather*}
    \left.d\vv{F}\right|_{z + dz} = \left.\vv{\vv{\Pi} } \hat{z}\right|_{z + dz} dx dy = (\Pi_{x z} \hat{x} + \Pi_{y z} \hat{y} + \Pi_{z z} \hat{z} ) dx dy\\
    \left.d\vv{F}\right|_{z} = \left.-\vv{\vv{\Pi} } \hat{z}   \right| _{z} dx dy = - (\Pi_{x z} \hat{x} + \Pi_{y z} \hat{y} + \Pi_{z z} \hat{z} ) dx dy
\end{gather*}
Il contributo delle forze di superficie lungo $y$ rispetto alle superfici
ortogonali all'asse $z$:
\begin{gather*}
    dF_y = \frac{\left.\vv{\vv{\Pi} }_{y z}\right|_{z - dz} dx dy - \left.\Pi_{y z}\right|_{z} }{dz}dx dy dz = -\frac{\partial \Pi_{yz}}{\partial z}d V
\end{gather*}
Si ha lo stesso per le altre due componenti, inziando con le $x$:
\begin{gather*}
    d\vv{F}_{x + dx}  = \vv{\vv{\Pi} } \hat{x} dy dz =  ( \Pi_{x x} \hat{x} + \Pi_{x y} \hat{y} + \Pi_{x z} \hat{z}  ) dy dz \\
    d\vv{F}_{x} = -\vv{\vv{\Pi} } \hat{x} dy dz =  -( \Pi_{x x} \hat{x} + \Pi_{x y} \hat{y} + \Pi_{x z} \hat{z}  ) dy dz 
\end{gather*}
Si somma dunque i contributi e si ottiene
\begin{gather*}
    dF_y = \frac{d\vv{F}_{x + dx} - d\vv{F}_{x}}{dx}dx dy dz = \frac{\partial }{\partial x} \Pi_{ y x} dV \\
    dF_x = \frac{\partial }{\partial x} \vv{\vv{\Pi_{x x}} }  d V \\
    dF_z = \frac{\partial }{\partial x} \vv{\vv{\Pi} }_{z x} d V  
\end{gather*}
La divergenza di un tensore mi da un campo vettoriale
\begin{gather*}
    \vv{\nabla}\cdot \vv{\vv{\Pi} } = \sum_i \frac{\partial }{\partial x_i}\Pi_{j, i}   
\end{gather*}
L'integrale sulla superdicie diventa
\begin{gather*}
    \oint_{\Sigma(V)} \left(-p\hat{n}  + \vv{\vv{\Pi} \hat{n}  } \right) \ d\sigma = \int dV\left(- \vv{\nabla}p + \vv{\nabla} \vv{\vv{\Pi} }   \right)
\end{gather*}
Adesso posso esplicare
\begin{gather*}
    \rho \left(\frac{\partial \vv{u} }{\partial t} + (\vv{u} \cdot \vv{\nabla}  ) \vv{u}  \right) = -\vv{\nabla} p + \rho \vv{g} + \vv{\nabla}\cdot \vv{\vv{\Pi} }    
\end{gather*}
Adesso possiamo determinare che forma ha il tensore, per le componenti qualsiasi
\begin{gather*}
    \Pi_{i, j} = \eta\left(\frac{\partial u_i}{\partial x_j} + \frac{\partial u_j}{\partial x_i} - \frac{2}{3}\sum_{k = 1}^{3} \frac{\partial u_k}{\partial x_k} \delta_{i, k} \right) + \xi \sum_k \frac{\partial u_k}{\partial x_k} \delta_{i, x} 
\end{gather*}
$\eta$ prende il nome di \textbf{viscosità di scorrimento} e $\xi$ è invece la \textbf{bulk viscosity }, o \textbf{viscosità di volume}. Si può definire un oggetto 
che è il gradiente di un campo vettoriale, che è un tensore:
\begin{gather*}
    \vv{\nabla} \vv{u} = \frac{\partial u_i}{\partial x_j}   
\end{gather*}
Dunque
\begin{gather*}
    \vv{\vv{\Pi} } = \eta \left(\vv{\nabla}\vv{u} + (\vv{\nabla} \vv{u} )^{T}  - \frac{2}{3}\mathbf{II} \vv{\nabla} \cdot \vv{u}   \right) + \xi \mathbf{II} \vv{\nabla}\cdot \vv{u}   
\end{gather*}   
Per un fluido incomprimibile si ottiene
\begin{gather*}
    \vv{\vv{\Pi} } = \eta \left(\frac{\partial u_i}{\partial x_j} + \frac{\partial u_j}{\partial x_i}  \right)  \ \Longrightarrow \ \vv{\nabla} \cdot \vv{\vv{\Pi} } = \eta \sum_j \frac{\partial }{\partial x_j} \left(\frac{\partial u_i}{\partial x_j} + \frac{\partial u_j}{\partial x_i}  \right) = \eta \sum_i \frac{\partial^{2} u_i}{\partial x_j}  + \eta \sum_j \frac{\partial}{\partial x_j}    \frac{\partial u_j}{\partial x_i} 
\end{gather*}
Posso dunque invertire le derivate 
\begin{gather*}
    \eta \sum \frac{\partial u_i}{\partial x_i} + \eta \frac{\partial }{\partial x_i} \sum_j \frac{\partial u_j}{\partial x_j} = \eta \sum_i \frac{\partial^{2}}{\partial x_i^{2}} u_i + \eta \frac{\partial \vv{\nabla} \cdot \vv{u}  }{\partial x_i}   
\end{gather*}
Dove
\begin{gather*}
    \vv{\nabla}\cdot (\vv{\nabla}\cdot \vv{u}  ) = \nabla^{2}\vv{u} 
\end{gather*}
E' l'operatore Laplaciano. Posso esprimere per un fluido reale incomprimibile l'equazione
di stato come
\begin{gather*}
    \rho \left(\frac{\partial \vv{u} }{\partial t} +  (\vv{u} \cdot \vv{\nabla}  ) \vv{u} \right) - \vv{\nabla}p + \rho \vv{g} + \eta \nabla^{2} \vv{u}    
\end{gather*}
che prende il nome di \textbf{equazione di Navier-Stokes}. Se si assume densità
costante, si può dividere tutto per la pressione e ricavare 
\begin{align}
    \nu = \frac{\eta}{\rho}
\end{align}
Ossia il coefficiente di \textbf{viscosità cinematica}, la quale ha come dimensioni
superficie su unità di tempo (ossia il coefficiente di diffusione). 

\section{Tempi caratteristici}
In un fluido in caduta, da una certa quota $l$, si può determinare il tempo caratteristico di caduta come
\begin{gather*}
    \tau_g = \sqrt{\frac{l}{2g}} 
\end{gather*}
Dunque a livello di energia
\begin{gather*}
    \frac{u^{2}}{2} - gl \ \Longrightarrow \ u = \sqrt{2gl} 
\end{gather*}
Si può stimare il tempo caratteristico dell'ultimo termine facendo un analisi
dimensionale dell'equazione:
\begin{gather*}
    \frac{u}{\tau_\nu} \approx \nu \frac{u}{l^{2}} \ \Longrightarrow \ \tau_\nu = \frac{l^{2}}{\nu}
\end{gather*}
Questo mi permette di dire che il tempo di diffusione viscosa è lungo quanto meno è viscoso il fluido.
Si può dire, in generale, che se
\begin{gather*}
    \tau_g << \tau_\nu 
\end{gather*}
Allora l'approssimazione ideale è corretta, altrimenti siamo nel caso non ideale. Possiamo 
scrivere un rapporto tra questi due per ottenere:
\begin{gather*}
    \frac{\tau_\nu}{\tau_g} >> 1 \quad \text{ideale}
\end{gather*}
Se si considera un tempo caratteristico
\begin{gather*}
    \tau_c = \frac{l}{u} \quad \tau_\nu = \frac{l^{2}}{\nu}
\end{gather*} 
Si ha che siamo in un regime \textbf{turbolento} quando
\begin{gather*}
    \frac{\tau_\nu}{\tau_c} >> 1
\end{gather*}
Il rapporto
\begin{gather*}
    \frac{\tau_\nu}{\tau_c} = \frac{lu}{\nu} = R_\nu
\end{gather*}
In un tubo, $l$ è la larghezza del tubo stesso mentre, nel caso di un'ostacolo,
$l$ rappresenta la larghezza dell'ostacolo.

\section{Applicazione (Legge di Poiselle)}
Preso un tubo a sezione cilindrica di raggio $r_0$ in cui passa
un fluido in condizioni stazionarie. Se fosse ideale, la velocità del fluido
sarebbe costante ma, dato che si vuole che esca l'acqua dal tubo, ci deve essere un gradiente di pressione
in modo che spinga il fluido veros una regione a pressione minore. Si può dunque scrivere
l'equazione (trascurando $g$ in quanto il tubo è orizzontale).
\begin{gather*}
    (\vv{u} \cdot \vv{\nabla}  )\vv{u} = -\frac{\vv{\nabla}p }{\rho_0} + \nu \vv{\nabla}^{2} \vv{u}  
\end{gather*}
Si può esprimere la velocità del fluido in funzione della distanza dal centro:
\begin{gather*}
    \vv{u} (r) \hat{z} \ \Longrightarrow \ -\vv{\nabla}\frac{p}{\rho_0} + \nu \vv{\nabla}^{2} \vv{u}     
\end{gather*}
Il primo termine è zero poiché $\vv{u}$ è funzione di $r$ e dunque il gradiente
è diretto lungo $r$ mentre $u$ è diretta lungo $z$. Nell'ipotsi che $p$
vari linearmente nella lunghezza,
\begin{gather*}
    p(z) = p_0 + (p_L - p_0) \frac{z}{L}
\end{gather*} 
Dunque
\begin{gather*}
    \frac{1}{p_0} \frac{\partial p}{\partial t} = \frac{1}{\rho_0} \frac{p_L - p_0}{L} = \nu \frac{1}{r} \frac{\partial }{\partial r}\left(r \frac{\partial u_z}{\partial r} \right) 
\end{gather*}
Posso risolvere per $u_z$ l'equazione e le derivate parziali diventano delle derivate
totali in quanto $u$ dipende solo da $z$. Si ottiene per intergreaizone diretta:
\begin{gather*}
    r \frac{du}{dr} = \int_{0}^{r} \frac{p_L - p_0}{L \rho_0 \nu } r'\ dr
\end{gather*}
Dunque si ottiene
\begin{gather*}
    -\frac{\Delta p}{L\eta} \frac{r^{2}}{2} \qquad \eta = \rho_0 \nu
\end{gather*}
E allora
\begin{gather*}
    \frac{du}{dr} = - \frac{\Delta p}{L\eta} \frac{r^{2}}{2}
\end{gather*}
Posso integrare e ottenere
\begin{gather*}
    u(r) = - \frac{\Delta p}{L\eta} \frac{r^{2}}{4} + C
\end{gather*}
Si impone allora che $u(r_0) = 0$. 
\begin{gather*}
    u(r_0) = - \frac{\Delta p}{L\eta} \frac{r_0^{2}}{4} + C = 0 \ \Longrightarrow \ C = \frac{\Delta p}{L\eta} \frac{r_0^{2}}{4}
\end{gather*}
E dunque si ottiene
\begin{gather*}
    u(r)  =\frac{\Delta p}{L\eta} r_0^{2} \left(1 - \frac{r^{2}}{r_0^{2}}\right)
\end{gather*}
Ossia la velocità del fluido è maggiore al centro del tubo proprio a
causa degli attriti con le pareti del tubo. Si può dunque calcolare la
portata (di volume) del fluido:
\begin{gather*}
    Q_V = 2\pi \int_{0}^{r_0} r u(r) \ dr = 2\pi \frac{\Delta pr_0^{2}}{4L\eta} \int_{0}^{r_0} r(1 - \frac{r^{2}}{r_0^{2}})  dr  = \frac{\pi}{8} \frac{\Delta pr_0^{4}}{L\eta}
\end{gather*}
che prende il nome di Legge di Poiselle.


\end{document}
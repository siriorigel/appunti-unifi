\documentclass[a4paper, oneside]{article}
\usepackage{graphicx}
\usepackage{amsthm}
\usepackage{amsmath}
\usepackage{amssymb}
\usepackage[a4paper,
            bindingoffset=0.2in,
            left=2cm,
            right=2cm,
            top=2cm,
            bottom=2cm,
            footskip=.25in]{geometry}
\usepackage[italian]{babel}
\usepackage{pgfplots}
\usepackage{tabularx}
\usepackage{tikz}
\usepackage{wrapfig}
\usepackage{color}
\usepackage[d]{esvect}
\usepackage{chemfig}
\usepackage{mhchem}
\definecolor{page}{rgb}{0.129,0.157,0.212}
\pagecolor{page}
\color{white}
\graphicspath{ {./images/} }
\usetikzlibrary{shapes.geometric}
\usetikzlibrary{datavisualization}
\usetikzlibrary{datavisualization.formats.functions}
\usetikzlibrary{patterns}
\pgfplotsset{width=10cm,compat=1.18}

\title{}
\author{Tommaso Miliani}
\date{31-10-25}

\begin{document}
\newtheoremstyle{theoremEnv}
                {}          % Space above
                {}          % Space below
                {\slshape}  % Body font
                {}          % Indent amount
                {\bfseries} % Head font
                {.}         % Punctuation after head
                {\newline}  % Space after theorem head
                {}          % Theorem head spec
\theoremstyle{theoremEnv}

\newtheorem{definition}{Definizione}[section]
\newtheorem{theorem}{Teorema}[section]
\newtheorem{lemma}{Proposizione}[section]
\newtheorem{observation}{Osservazione}[section]
\newtheorem{corollary}{Corollario}[theorem]
\newtheorem{example}{Esempio}[section]
\newtheorem{remark}{Enunciato}[section]

\maketitle

\section{Copiare da pipas}


\section{Dimostrazione cp}
\begin{gather*}
    \delta Q = \mathcal{C}_p dT
\end{gather*}
Adesso posso dire che il sistema 1 scambierà un calore
pari a
\begin{gather*}
    Q_1 = \int_{T_1}^{T_e} \delta Q  = \int_{T_1}^{T_e}\mathcal{C}_1 dT 
\end{gather*}
incrementando piano piano la temperatura, sono sicuro che
l'integrale sia sicuramente maggiore di zero e dunque il sistema 
prende del calore (anche se non so necessariamente calcolare
l'integrale). Posso quindi determinare la variazione di
Entropia del sistema 1 attraverso la seguente:
\begin{gather*}
    \Delta S_1 = \int_{T_1}^{T_e} \frac{\delta Q}{T} = \int_{T_1}^{T_e} \frac{\mathcal{C}_1}{T} \ dT  > 0
\end{gather*}
Invece per il calore $Q_2$ si ha che
\begin{gather*}
    Q_2 = \int_{T_2}^{T_e} \mathcal{C}_2 \ d T = - \int_{T_e}^{T_2}\mathcal{C}_2 \ dT  
\end{gather*}
Dato che il numero dentro l'integrale è positivo, il calore scambiato dal sistema 2
è dunque negativa (poiché cede calore all'ambiente, ossia il sistema 1). 
La sua variazione di entropia sarà dunque data da
\begin{gather*}
    \Delta S_2 = \int_{T_2}^{T_e} \frac{\delta Q}{T} = \int_{T_2}^{T_e} \frac{\mathcal{C}_2}{T}\ dT = - \int_{T_e}^{T_2}\frac{\mathcal{C}_2}{T} \ dT < 0  
\end{gather*}
Questo risultato non viola nulla in quanto i due sistemi stanno interagendo termicamente. Siccome
il sistema complessivo è racchiuso tra pareti adiabatiche, il sistema complessivo non scambia
dunque calore con l'esterno e non subisce né fa lavoro. Quindi la variazione 
dell'energia interna del sistema complessivo sarà data esattamente da
\begin{gather*}
    \Delta U = \Delta U_1 + \Delta U_2 = 0
\end{gather*}
Dato che non viene compiuto alcun lavoro (e non è subito), la variazione dell'energia
interna sarà data esclusivamente dal calore:
\begin{gather*}
    \Delta U = Q_1 + Q_2 = 0 \ \Longrightarrow \ Q_1 = -Q_2
\end{gather*}
Deve dunque risultare che la temperatura del termostato 
\begin{gather*}
    T < T_e \ \Longrightarrow \ \frac{1}{T} > \frac{1}{T_e} \ \Longrightarrow \ \frac{\mathcal{C}_1}{T} > \frac{\mathcal{C}_1}{T_e}
\end{gather*}
E quindi si ottiene la variazione di entropia del sistema 1:
\begin{gather*}
    \int_{T_1}^{T_e} \frac{\mathcal{C}_1}{T} \ dT > \int_{T_1}^{T_e} \frac{\mathcal{C}_1}{T_e} \ dT  \\
    \int_{T_1}^{T_e} \frac{\mathcal{C}_1}{T_e} \ dT = \frac{1}{T_e} \int_{T_1}^{T_e} \mathcal{C}_1 \ dT = \frac{Q_1}{T_e} \ \Longrightarrow \ \Delta S_1 > \frac{Q_1}{T_e}  
\end{gather*}
Per il secondo sistema invece devo considerare che il sistema sta perdendo calore,
dunque il termostato successivo è più freddo del precedente e si ha un flusso in uscita di 
calore. 
\begin{gather*}
    T > T_e \ \Longrightarrow \ \frac{1}{T} < \frac{1}{T_e} \ \Longrightarrow \ \delta Q < 0  
\end{gather*}
Posso esprimere dunque secondo gli integrali
\begin{gather*}
    \int_{T_2}^{T_e} \frac{\mathcal{C}_2}{T} \ dT > \int_{T_2}^{T_e} \frac{\mathcal{C}_2}{T} \ dT \ \Longrightarrow \ \Delta S_2 > \frac{Q_2}{T_e}  
\end{gather*}
Dunque le entropie si relazionano
\begin{gather*}
    \Delta S_1 > \frac{Q_1}{T_2} > -\Delta S_2 \\
    S_1 > -\Delta S_2 \ \Longrightarrow \ \Delta S_1 + \Delta S_2 > 0
\end{gather*}
Dunque la variazione di entropia è sempre maggiore di zero 
per qualsiasi sistema isolato (in questo caso è il sistema 1 + il sistema 2).
E' quindi possibile avere una variazione di entropia che sia minore
di zero, ma se e solo se il sistema non è isolato e sta scambiando del calore
con un ambiente esterno (come nel caso del sistema 2). 

\section{Variazione dell'energia}
Supponendo che il sistema fluido faccia solamente lavoro tramite
le forze di pressione, posso ricavare l'energia interna come 
\begin{gather*}
    dU = \left(\frac{\partial U}{\partial T} \right)_V \ dT + \left(\frac{\partial U}{\partial V} \right)_T \ dV = \delta Q - p\ dV
\end{gather*}
Posso dunque dividere tutto per la temperatura e ottenere
\begin{gather*}
    \frac{\delta Q}{T} = \frac{1}{T} +  \left(\frac{U}{T}\right)_V \ dT + \frac{1}{T} \left(\left(\frac{\partial U}{\partial V}\right)_T + p\right)\ dV
\end{gather*}
Facendo l'ipotesi che la quantità di calore che è scambiata sia scambiata in maniera reversibile, 
posso determinare il differenziale dell'entropia all'espressione 
dell'energia interna. (con $S = S(V, T)$).
\begin{gather*}
    dS = \left(\frac{\partial S}{\partial T} \right)_V \ dT + \left(\frac{\partial S}{\partial V} \right)_T \ dV
\end{gather*}
Immaginando di prendere la derivata rispetto al volume e devo derivare in maniera
mista la funzione:
\begin{gather*}
    \frac{\partial }{\partial V} \frac{\partial S}{\partial T} = \frac{\partial }{\partial T} \frac{\partial S}{\partial V}    \\
    \frac{\partial }{\partial V} \left(\frac{1}{T} \left(\frac{\partial U}{\partial T} \right)_V\right) = \frac{\partial }{\partial T} \left(\frac{1}{T}\left(\left(\frac{\partial U}{\partial V} \right)_T + p\right)\right) \\
    \frac{1}{T} \frac{\partial^{2 } U}{\partial V \partial T} = \frac{1}{T} \left(\left(\frac{\partial U}{\partial V} \right)_T + p\right) + \left(\frac{\partial ^{2} U}{\partial T \partial V} \right) + \left(\frac{\partial p}{\partial T} \right)_V   
\end{gather*}
Ottenendo allora 
\begin{gather*}
    \frac{1}{T}\left(\left(\frac{\partial U}{\partial V} \right)_T + p\right) = \left(\frac{\partial p}{\partial T} \right)_V
\end{gather*}
Si esprime allora l'energia interna di un sistema fluido qualunque e come questa si relazione 
in funzione delle variabili termodinamiche:
\begin{gather*}
    \left(\frac{\partial U}{\partial V} \right)_T = T\left(\frac{\partial p}{\partial T} \right)_V - p
\end{gather*}
Dunque ottengo
\begin{gather*}
    T\ dS = \delta Q = dU + p \ dV \\
    T \ dS = \mathcal{C}_V \ dT + \left(\frac{\partial U}{\partial V} \right)_T \ dV + p\  dV
\end{gather*}
E dunque si ha l'espressione dell'entropia di un qualsiasi sistema fluido 
come funzione di stato. 
\begin{gather*}
    T \ dS = \mathcal{C}_V \ dT + T\left(\frac{\partial p}{\partial T} \right)_V \ dV
\end{gather*}
Si può ottenere una analoga semplicemente utilizzando l'equazione dell'energia interna:
\begin{gather*}
    T \ dS = \mathcal{C}_p \ dT - T\left(\frac{\partial V}{\partial T} \right)_p \ dp
\end{gather*}
Inoltre si ottengono le quattro equazioni termodinamiche di MAxWell:
\begin{gather*}
    \left(\frac{\partial S}{\partial T} \right)_V = \frac{\mathcal{C}_V}{T} \\
    \left(\frac{\partial S}{\partial V} \right)_T = \left(\frac{\partial p}{\partial T} \right)_V \\
    \left(\frac{\partial S}{\partial T} \right)_p = \frac{\mathcal{C}_p}{T} \\
    \left(\frac{\partial S}{\partial p} \right)_T = - \left(\frac{\partial V}{\partial T} \right)
\end{gather*}
Per un gas perfetto si ha che
\begin{gather*}
    dS = \frac{\mathcal{C}_V}{T} \ dT + \left(\frac{\partial p}{\partial T} \right)_V \ dV \\
    dS = \mathcal{C}_V \frac{dT}{T} + nR \frac{dV}{V}
\end{gather*}
Dunque si ha l'espressione dell'entropia per un gas perfetto 
a volume costante:
\begin{gather*}
    S(T, V) = \mathcal{C}_V \ln\frac{T}{T_0} + nR\ln \frac{V}{V_0}\\
    S(T; V) = \mathcal{C}_V (\ln \frac{T}{T_0} + \frac{nR}{\mathcal{C}_V}\ln\frac{V}{V_0})
\end{gather*}
Ossia
\begin{gather*}
    S(T, V) = \mathcal{C}_V \left(\ln\frac{T}{T_0} + \ln \left(\frac{V}{V_0}\right)^{\gamma -1}\right) \ \Longrightarrow \ S(V, T) = \mathcal{C}_V \ln\left(\frac{TV^{\gamma - 1}}{TV_0^{\gamma -1}}\right)
\end{gather*}

\end{document}
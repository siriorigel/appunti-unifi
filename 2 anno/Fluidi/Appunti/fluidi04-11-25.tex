\documentclass[a4paper, oneside]{article}
\usepackage{graphicx}
\usepackage{amsthm}
\usepackage{amsmath}
\usepackage{amssymb}
\usepackage[a4paper,
            bindingoffset=0.2in,
            left=2cm,
            right=2cm,
            top=2cm,
            bottom=2cm,
            footskip=.25in]{geometry}
\usepackage[italian]{babel}
\usepackage{pgfplots}
\usepackage{tabularx}
\usepackage{tikz}
\usepackage{wrapfig}
\usepackage{color}
\usepackage[d]{esvect}
\usepackage{chemfig}
\usepackage{mhchem}
\definecolor{page}{rgb}{0.129,0.157,0.212}
\pagecolor{page}
\color{white}
\graphicspath{ {./images/} }
\usetikzlibrary{shapes.geometric}
\usetikzlibrary{datavisualization}
\usetikzlibrary{datavisualization.formats.functions}
\usetikzlibrary{patterns}
\pgfplotsset{width=10cm,compat=1.18}

\title{}
\author{Tommaso Miliani}
\date{04-11-24}

\begin{document}
\newtheoremstyle{theoremEnv}
                {}          % Space above
                {}          % Space below
                {\slshape}  % Body font
                {}          % Indent amount
                {\bfseries} % Head font
                {.}         % Punctuation after head
                {\newline}  % Space after theorem head
                {}          % Theorem head spec
\theoremstyle{theoremEnv}

\newtheorem{definition}{Definizione}[section]
\newtheorem{theorem}{Teorema}[section]
\newtheorem{lemma}{Proposizione}[section]
\newtheorem{observation}{Osservazione}[section]
\newtheorem{corollary}{Corollario}[theorem]
\newtheorem{example}{Esempio}[section]
\newtheorem{remark}{Enunciato}[section]

\maketitle

\section{Entropia nelle trasformazioni liquido gas}
\subsection{Come la pressione e la temperatura sono relazionate in una trasformazione dal liquido al gas}
\begin{wrapfigure}{r}{0.4\textwidth}
    \centering
    \caption{}
    \begin{tikzpicture}
        \draw[->](0, 0) -- (4, 0);
        \draw[->](0, 0) -- (0, 4);
        \draw(0.5, 3) -- (1, 1.5) node[at start, above] {$T$} -- (3, 1.5) -- (3.5, 0.5) ;
        \draw[dashed](0, 1.5) -- (1, 1.5) node[at start, left] {$p_s$};
        \draw[dashed](1, 0) -- (1, 1.5) node[at start, below] {$V_l$};
        \draw[dashed](3, 0) -- (3, 1.5) node[at start, below] {$V_g$}; 
    \end{tikzpicture}    
\end{wrapfigure}
Durante il passaggio di stato la pressione rimane 
costante e la temperatura pure durante un passaggio di stato,
tuttavia, fissata una temperatura, trovo una pressione che
rimane costante e viceversa. Si possono relazionare queste grandezze
mediante la TDS. Si vuole trovare, fissata la temperatura, la pressione
della trasformazione che dipende, data la TDS, solamente dal volume
\begin{gather*}
    T \ dS = T\left(\frac{\partial p}{\partial T} \right)_V dV \ \Longrightarrow \ \left(\frac{\partial p_s}{\partial T} \right)_V = \frac{d p}{dT}
\end{gather*}
Ossia la pressione dipende interamente dalla temperatura:
\begin{gather*}
    T \ dS = T \frac{dp_s}{dT}  \ dV
\end{gather*}
Integrando da entrambe le parti si ha che
\begin{gather*}
    Q = T \frac{\partial p_s}{\partial T}(V_g - V_l) 
\end{gather*}
Ossia la quantità di calore scambiato dipende dalla differenza di volume 
tra lo stato gassoso e quello liquido e dal calore latente di vaporizzazione:
\begin{gather*}
    Q = M l_V = T \frac{dp_s}{ dT}(V_g - V_l) 
\end{gather*}
E dunque si ottiene l'equazione di Clapayron:
\begin{align}
    \frac{\partial p_s}{\partial T} = \frac{Ml_V}{T(V_g - V_l)} 
\end{align}
Questa equazione, se si trascura il volume del liquido rispetto a quello del gas,
(in generale è vero), allora l'equazione può essere semplificata notevolmente:
\begin{align}
    \frac{\partial p_s}{\partial T} = \frac{Ml_V}{TV_g} 
\end{align}
In prima approssimazione un fluido gassoso è un gas perfetto: introducendo questa
approssimazione all'equazione si ottiene allora, per una mole di gas:
\begin{gather*}
    p_s V_g = RT \ \Longrightarrow \  V_g = \frac{RT}{p_s}
\end{gather*}
Allora 
\begin{align}
    \frac{\partial p_s}{\partial T} = \frac{Ml_V}{RT^{2}}p_s  
\end{align}
Posso anche esprimere la temperatura di ebollizione di un fluido in funzione della
pressione de fluido stesso
\begin{align}
    \frac{\partial T_e}{\partial p} = \frac{RT_e^{2}}{ pMl_V}  
\end{align}
Nel caso reale con l'atmosfera,  con l'utilizzo della legge fondamentale dei gas di stato:
\begin{gather*}
    dp = -\rho g \ dz \ \Longrightarrow \ \frac{\partial T_e}{\partial p} = - \rho g \frac{RT_e^{2}}{pMl_V}
\end{gather*}
Dove $M$ è la densità molare del fluido considerato e $\rho$ la densità dell'aria.

\subsection{Trasformazioni da solido a gas}
\begin{wrapfigure}{r}{0.4\textwidth}
    \centering
    \caption{}
    \begin{tikzpicture}
        \draw[->](0, 0) -- (4, 0);
        \draw[->](0, 0) -- (0, 4);
        \draw(0.5, 3) -- (1, 1.5) node[at start, above] {$T$} -- (3, 1.5) -- (3.5, 0.5) ;
        \draw[dashed](0, 1.5) -- (1, 1.5) node[at start, left] {$p_s$};
        \draw[dashed](1, 0) -- (1, 1.5) node[at start, below] {$V_s$};
        \draw[dashed](3, 0) -- (3, 1.5) node[at start, below] {$V_l$}; 
    \end{tikzpicture}    
\end{wrapfigure}
Nelle trasformazioni da solido a gas si ha la situazione analoga:
per cui posso esprimere (sfruttando le equazioni trovate prima):
\begin{gather*}
    \frac{\partial p}{\partial T} =  \frac{Ml_F}{T(V_l - V_s)} 
\end{gather*}
Generalmente le sostanze quando fondono aumentano il loro volume, 
tuttavia, per alcune sostanze il volume del liquido è minore del volume
del solido. Per l'acqua, per esempio, $V_l - V_s < 0$. Quando aumenta
la pressione, la temperatura di fusione diminuisce.  Per questa categoria di fluidi 
però si osserva il fenomeno macroscopico di galleggiamento della parte solida
sulla parte liquida. Nel caso dell'acqua, normalmente tutto il ghiaccio
è inferiore alla temperatura di fusione, ma l'aumento di pressione all'interno 
del ghiacciaio fa sì che lo strato a contatto con il terreno sia liquido. 

\section*{Fisica statistica}
\section*{Introduzione}
\section{Potenziali termodinamici e principi variazionali}
\subsection{Spiegazione della correzione di Maxwell nell'equazione di Van Der Walls}
L'equazione di Van Der Walls mi permette di esprimere il potenziale 
in funzione del volume e della pressione. La coesistenza nek grafico 
corrisponde al fatto che si sceglie una pressione $p_{l, g}$, ossia la pressione 
nella quale il liquido ed il gas sono in equilibrio in modo tale che la
vera isoterma è quella prevista di Van Der Walls ma risulta che le due 
aree $A_1$ e $A_2$ sono le stesse. Questa è una conseguenza del principio variazionale. 
Il volume nel quale il gas è tutto gas è $V_g$ mentre il volume nel quale il 
gas è tutto liquido è $V_l$ e nel punto in cui la curva isoterma incrocia il punto
di contatto delle due aree lo si chiama $V^{\star}$. 
\begin{gather*}
    F = U - TS \ \Longrightarrow \ dF = -S \ dT - p\ dV
\end{gather*}
Nel caso in cui si è a temperatura costante si avrà 
\begin{gather*}
    dF = -p \ dV
\end{gather*}
Ossia posso ricavare $F$ come funzione del volume intergando 
sulla curva a volume costante
\begin{gather*}
    F(T, V) = - \int_{T}^{} p \ dV
\end{gather*}
Dato che la funzione $p$ è dato dalla funzione Van Der Walls, posso 
trovare la funzione
\begin{gather*}
    F = - \frac{8}{3}\tilde{T} \ln \left(\tilde{V} - \frac{1}{3}\right) - \frac{3}{\tilde{V}} + F_0(\tilde{T})
\end{gather*}

\begin{wrapfigure}{r}{0.4\textwidth}
    \centering
    \caption{}
    \begin{tikzpicture}
        \draw[->](0, 0) -- (4, 0) node[at end, below] {$V$};
        \draw[->](0, 0) -- (0, 4) node[at end, left] {$-F$};
        \draw(0.5, 1) .. controls (0.7, 1.8) .. (1.5, 1.5);
        \draw(1.5, 1.5) .. controls (1.8, 1.8) and (2, 2.5) .. (2.5, 2.5);
        \draw[dashed](0.7, 1.2) -- (0.7, 0) node[at end, below] {$V_l$};
        \draw[dashed](1.5, 1.5) -- (1.5, 0) node[at end, below] {$V^{\star}$};
        \draw[dashed](2.2, 2.2) -- (2.2, 0) node[at end, below] {$V_g$};
    \end{tikzpicture}    
\end{wrapfigure}
Dove $\tilde{V}$ è il volume rapportato al punto critico (per cui è adimensionale)
mentre il termine $F_0$ è la costante per cui si integra. Si può dunque ragionare
sulla funzione senza necessariamente risolvere l'integrale ed ottenere questa funzione. 
Credendo a Van Der Walls, questo ci predice che $F$ è minore di quella prevista
da Vam Der Wallss, ma il sistema, dato che vuole minimizzare la sua energia interna,
allora eseguirà una transizione di fase. $F$ varia dunque in funzione del volume 
in modo lineare in quanto è una interpolazione tra l'energia del liquido e quello del gas.
Allora variando ciascuna delle due fasi posso minimizzare la $F$. 



\end{document}
\documentclass[a4paper, oneside]{article}
\usepackage{graphicx}
\usepackage{amsthm}
\usepackage{amsmath}
\usepackage{amssymb}
\usepackage[a4paper,
            bindingoffset=0.2in,
            left=2cm,
            right=2cm,
            top=2cm,
            bottom=2cm,
            footskip=.25in]{geometry}
\usepackage[italian]{babel}
\usepackage{pgfplots}
\usepackage{tabularx}
\usepackage{tikz}
\usepackage{wrapfig}
\usepackage{color}
\usepackage[d]{esvect}
\definecolor{page}{rgb}{0.129,0.157,0.212}
\pagecolor{page}
\color{white}
\graphicspath{ {./images/} }
\usetikzlibrary{shapes.geometric}
\usetikzlibrary{datavisualization}
\usetikzlibrary{datavisualization.formats.functions}
\usetikzlibrary{patterns}
\pgfplotsset{width=10cm,compat=1.9}

\title{Termodinamica}
\author{Tommaso Miliani}
\date{23-10-25}

\begin{document}
\newtheoremstyle{theoremEnv}
                {}          % Space above
                {}          % Space below
                {\slshape}  % Body font
                {}          % Indent amount
                {\bfseries} % Head font
                {.}         % Punctuation after head
                {\newline}         % Space after theorem head
                {}          % Theorem head spec
\theoremstyle{theoremEnv}

\newtheorem{definition}{Definizione}[section]
\newtheorem{theorem}{Teorema}[section]
\newtheorem{lemma}{Proposizione}[section]
\newtheorem{observation}{Osservazione}[section]
\newtheorem{corollary}{Corollario}[theorem]
\newtheorem{example}{Esempio}[section]

\maketitle

\section{Moti convettivi}
\begin{wrapfigure}{r}{0.4\textwidth}
    \centering
    \caption{La convezione}
    \begin{tikzpicture}
        \draw[very thick](0, 0) rectangle (4, 4);
        \filldraw[cyan, opacity = 0.3](0, 0) rectangle (4, 4);
    \end{tikzpicture}    
\end{wrapfigure}
Se si ha una regione calda del fluido e una regione meno calda del fluido,
ci si possa immaginare che (come nel caso dell'atmosfera) ci sia una sorgente
di calore la base del fluido, allora i moti convettivi del fluido saranno maggiormente
accentuati. Questo è un modo per trasportare energia all'interno di un fluido: quello che succede
quando si ha un fluido stratificato e un termostato alla base del fluido, si creano
delle colonne convettive che salgono e delle colonne convettive che scendono.
La convezione è il metodo più efficiente per il trasporto di energia all'interno del fluido. 
Dall'elettromagnetismo ogni fluido emette della radiazione elettromagnetica in base
alla temperatura: più è caldo il fluido è maggiore sarà l'intensità delle onde
elettromagnetiche irraggiate: dunque il fluido può raffreddarsi tramite irraggiamento
(come avviene nelle stelle).
Se si ha un fluido a contatto con una fonte di calore, si ha, dai dati sperimentali,
che la quantità di calore passata al fluido dipende dalla differenza di temperatura
tra la sorgente di calore ed il fluido:
\begin{align}
    \frac{\delta Q}{dt} \propto T_p - T_f
\end{align}
Dove $T_p$ è la temperatura 
A dispetto dei termosifoni, gli split dei condizionatori sono 
molto più efficienti in quanto generano dei moti di convezione nel fluido e dunque
permettono un passaggio di calore molto più rapido rispetto al termosifone. 

\section{Scambiare calore tra due fluidi separati: la conduzione}
\begin{wrapfigure}{r}{0.4\textwidth}
    \centering
    \caption{}
    \begin{tikzpicture}
        \draw[->](-2, 0) -- (2, 0) node[at end, below] {$x$};
        \draw(-2, 2) -- (-2, -2);
        \draw(0, -2) -- (0, 2);
        \draw[<->](-1.9, -2) -- (-0.1, -2) node[midway, below] {$l$};
        \node at(-2.5, 0.5) {$T_1$};
        \node at(0.5, 0.5) {$T_2$};
    \end{tikzpicture}    
\end{wrapfigure}
Come avviene il trasferimento di calore tra due fluidi attraverso una parete?
Immaginando di avere una parete spessa $l$ e poniamo a contatto con questa parete
due fluidi a temperature diverse; allora ci sarà passaggio di calore: il \textbf{flusso di calore} che
dipende dalla seguente relazione
\begin{gather*}
    q = \frac{\delta Q}{dS \ dt}
\end{gather*}
Ossia la quantità di energia che passa attraverso una data superficie per un
certo quantitativo di tempo. Questo flusso è positivo se transita dalla
parte sinistra a quella destra (in quanto è concorde con la direzione scelta dell'asse
delle ascisse). Posso allora utilizzare la seguente relazione per porlo in
funzione della temperature:
\begin{gather*}
    q = -k\frac{T_2 - T_1}{l}
\end{gather*}
Ossia il flusso di calore dipende dalla differenza di temperatura tra i due fluidi moltiplicato
per una costante $k$ che prende il nome di \textbf{conducibilità termica}, che è propria del
materiale di cui è composta la parete. Il flusso di calore è anche inversamente proporzionale allo spessore
della parete stessa. Alcune conducibilità termiche per alcuni materiali (la cui unità
di misura è di $W \ m ^{-1} \ K^{-1} $):
\begin{gather*}
    \begin{tabular}[10]{c | c}
        Materiale & Conducibilità\\
        \hline
        argento & 400 \\
        \hline
        vetro & 0.8 \\
        \hline
        sughero in 
        fibra di vetro & 0.04 
    \end{tabular}
\end{gather*}
Prendendo per buona la relazione, possiamo assumere ora che la temperatura sia funzione
della posizione: adesso anche la legge del flusso di calore acquisisce una dipendenza spaziale: devo allora considerare
il limite della funzione per $l \to 0$:
\begin{gather*}
    q = -k \frac{T(x + l) - T(x)}{l}
\end{gather*}
Adesso, passando al limite:
\begin{gather*}
    q(x) = -k\lim_{l \to 0} \frac{T(x + l) - T(x)}{l}
\end{gather*}
E dunque si ottiene la \textbf{legge di Fourier}:
\begin{align}
    q(x) = -k\frac{dT(x)}{dx}
\end{align}
In questo caso non è un equilibrio termodinamico globale bensì un equilibrio termodinamico locale.
Adesso si vuole capire come avviene il passaggio di energia tra un fluido 
e l'altra e ci si aspetta che ci siano delle situazioni in cui la temperatura vari anche
in base al tempo poiché nel tempo la temperatura deve necessariamente cambiare se i due sistemi
termodinamici di temperatura $T_1$ e $T_2$ non sono termostati.  
Allora la relazione che abbiamo considerato vale solamente in condizioni stazionarie:
se si fissa la temperatura in un punto stazionario allora possiamo escludere la dipendenza
dal tempo e quella relazione diventa una derivata parziale rispetto allo spazio:
\begin{gather*}
     q(x, t) = -k\frac{\partial T(x)}{\partial x}
\end{gather*}
Si deve ora ricavare la funzione $q(x, t)$ in modo tale che sia dipendente
sia dallo spazio che dal tempo. 


\begin{wrapfigure}{r}{0.4\textwidth}
    \centering
    \caption{}
    \begin{tikzpicture}
        \draw[->](0, 0) -- (4, 0) node[at end, below] {$x$};
        \draw(0, -2) -- (0, 2);
        \draw(0.5, -2) -- (0.5, 2);
        \draw(0, -1) -- (0.5, -1);
        \draw(0, -1.5) -- (0.5, -1.5);
        \node at(0.75, -1.25) {$dS$};
        \node at(0, -2.2) {$x$};
        \node at(0.5, -2.2) {$dx$};
    \end{tikzpicture}    
\end{wrapfigure}
Per determinarla ci mettiamo nella situazione nella quale
lo spessore della parete è infinitesima e dunque la superficie
della parete sarà $dS$ ed un volume $dV = dS \cdot dx$. Posso avere,
in questa configurazione, solamente un flusso di calore lungo l'asse $x$
e mi chiedo quale è il calore che passa da una parte all'altra in un intervallo
infinitesimo $dT$.
\begin{gather*}
    \delta Q = - q_{TOT} dS \ dt 
\end{gather*}
Dove $q_{TOT}$ è il flusso di calore di tutta la parete; 
adesso posso utilizzare la capacità termica (a volume costante in quanto è quella definita) come
\begin{gather*}
    \delta Q = \mathcal{C}_p \ dT \ \Longrightarrow \ \mathcal{C}_p \ dT = - q_{TOT} dS \ dt
\end{gather*}
Se dividessi tutto per l'intervallo di tempo $dt$ otterrei che 
il \textbf{campo di temperatura} è dato da:
\begin{gather*}
    -q_{TOT}dS = \mathcal{C}_p \frac{\partial T}{\partial t} 
\end{gather*}
Ossia
\begin{gather*}
    -q_{TOT}dS = c_P \rho \ dV \frac{\partial T}{\partial t} = \rho c_p \frac{\partial T}{\partial t} dx \ dS 
\end{gather*}
Si deve adesso determinare come è definito il flusso totale di calore come il flusso della parete di destra
meno il contributo che viene dalla parete sinistra:
\begin{gather*}
    -q_{TOT} = - \left(q( x + dx) - q(x)\right) = -\left(q(x) + \frac{\partial q}{\partial x} dx - q(x) \right) = - \frac{\partial q}{\partial x}dx 
\end{gather*}
Questo mi porta ad ottenere la seguente formulazione:
\begin{gather*}
    - \frac{\partial q}{\partial x}dx \ dS = \rho c_p \frac{\partial T}{\partial t}dx \ dS  
\end{gather*}
Siamo dunque arrivati alla relazione che mi permette di esprimere la derivata parziale
rispetto alla posizione del flusso di calore come
\begin{gather*}
    -\frac{\partial q}{\partial x} = \rho c_p \frac{\partial T}{\partial t}  
\end{gather*}
Allora, con la legged i Fourier, posso esprimere il campo di temperatura in relazione
al flusso di calore e dunque posso sostituire ed ottenere la relazione 
per la derivata del flusso di calore rispetto ad $x$:
\begin{align}
    c_p \rho \frac{\partial T}{\partial x} = k \frac{\partial^{2} T}{\partial x^{2} }  \ \Longrightarrow \ \frac{\partial T(x, t)}{\partial t} = D_T \frac{\partial ^{2} T(x, t) }{\partial x^{2} }  
\end{align}
Dove $D_T$ è il \textbf{coefficiente di diffusione termica} ed ha la seguente espressione
\begin{align}
    D_T = -\frac{k}{c_p \rho}
\end{align}
Siamo arrivati a determinare l'evoluzione del campo di temperatura rispetto alla sua dipendenza
temporale e spaziale: l'equazione considerata prende il nome di \textbf{equazione di diffusione}
che permette di determinare lo scambio delle'energia termica tra due sistemi termodinamici. 

\subsection{Situazione stazionaria}
La soluzione stazionaria deve soddisfare la seguente
\begin{gather*}
    D_T \frac{\partial ^{2}T(x, t) }{\partial x^{2} } = 0 
\end{gather*}
Questo tipo di funzioni sono delle funzioni lineari: la variazione
della temperatura in funzione della posizione è dunque una formulazione lineare; e,
dato che siamo in condizioni stazionarie, allora non si ha più la dipendenza
dal tempo:
\begin{gather*}
    \frac{d^{2} T}{d x^{2} } = 0
\end{gather*}
E dunque si ottiene la funzione della temperatura in funzione della
posizione:
\begin{gather*}
    T( x) = T_1  + \frac{T_2 - T_1}{l}x
\end{gather*}
In generale se esiste una funzione che ha una dipendenza
lineare da una variabile e derivo due volte, allora se avessi
una soluzione del tipo $f(t)$, allora anche la soluzione
del tipo $f(-t)$ è soluzione (anche se si hanno condizioni diverse)
della medesima equazioni. Esistono dunque sempre delle condizioni
per cui esistono queste soluzioni: in questo caso si parla di \textbf{fenomeni 
reversibili}. Queste equazioni non sono invarianti rispetto all'inversion 
temporale. Se invece considerassi l'evoluzione del campo di temperatura rispetto
ad una direzione allora queste trasformazioni sono irreversibili poiché è invariante rispetto alla direzione
in cui è avvenuta.

\clearpage
\subsection{Soluzione del campo di temperatura}
\begin{wrapfigure}{r}{0.6\textwidth}
    \centering
    \caption{Funzione campo di temperatura per l'acciaio}
    \begin{tikzpicture}[scale=0.65]
            \begin{axis}[grid=major, view={60}{15}, xlabel = $x$, ylabel = $t$]
            \addplot3[
            surf,
            shader=interp,
            samples=50,
            domain = -4:4,
            y domain=-2:-0.05,
            ] {(4 * pi * 58 * y)^(-1 / 2) * exp(-(x^2) / (4* 58 * y))};
            \addplot3[
            surf,
            shader=interp,
            samples=50,
            domain = -4:4,
            y domain=0.05:2,
            ] {(4 * pi * 58 * y)^(-1 / 2) * exp(-(x^2) / (4* 58 * y))};
            \end{axis}
    \end{tikzpicture}    
\end{wrapfigure}
Una soluzione per l'equazione del campo di temperatura è
\begin{gather*}
    T(x, t) = \left(4\pi D_T t\right)^{-\frac{1}{2}} \exp\left(-\frac{x^{2} }{4D_T t}\right) 
\end{gather*}
In particolare questa funzione è una Gaussiana per cui una possibile soluzione
per il campo di calore è una Gaussiana (solamente per quanto riguarda lo spazio) in quanto
il parametro di larghezza della funzione cresce nel tempo come
\begin{gather*}
    \sigma \propto \sqrt{t} 
\end{gather*}
Ossia è una Gaussiana che si spancia sempre di più: per $t \to \infty$
si ha uns situazione come in figura.
Se il tempo è 0, allora questa funzione è considerabile come un limite
di una successione di funzioni che tendono a zero. Questo 
porta a dire che 
\begin{gather*}
    T(x, 0) = \delta (x)
\end{gather*}
e dunque la gaussiana si stringerebbe inifinitamente.
Il flusso del calore tende in modo che il profilo della gaussiana si uniformi
ed il picco corrisponde 


\end{document}
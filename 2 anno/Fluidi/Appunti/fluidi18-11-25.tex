\documentclass[a4paper, oneside]{article}
\usepackage{graphicx}
\usepackage{amsthm}
\usepackage{amsmath}
\usepackage{amssymb}
\usepackage[a4paper,
            bindingoffset=0.2in,
            left=2cm,
            right=2cm,
            top=2cm,
            bottom=2cm,
            footskip=.25in]{geometry}
\usepackage[italian]{babel}
\usepackage{pgfplots}
\usepackage{tabularx}
\usepackage{tikz}
\usepackage{wrapfig}
\usepackage{color}
\usepackage[d]{esvect}
\usepackage{chemfig}
\usepackage{mhchem}
\definecolor{page}{rgb}{0.129,0.157,0.212}
\pagecolor{page}
\color{white}
\graphicspath{ {./images/} }
\usetikzlibrary{shapes.geometric}
\usetikzlibrary{datavisualization}
\usetikzlibrary{datavisualization.formats.functions}
\usetikzlibrary{patterns}
\pgfplotsset{width=10cm,compat=1.18}

\title{Appunti di Fisica statistica}
\author{Tommaso Miliani}
\date{18-11-25}

\begin{document}
\newtheoremstyle{theoremEnv}
                {}          % Space above
                {}          % Space below
                {\slshape}  % Body font
                {}          % Indent amount
                {\bfseries} % Head font
                {.}         % Punctuation after head
                {\newline}  % Space after theorem head
                {}          % Theorem head spec
\theoremstyle{theoremEnv}

\newtheorem{definition}{Definizione}[section]
\newtheorem{theorem}{Teorema}[section]
\newtheorem{lemma}{Proposizione}[section]
\newtheorem{observation}{Osservazione}[section]
\newtheorem{corollary}{Corollario}[theorem]
\newtheorem{example}{Esempio}[section]
\newtheorem{remark}{Enunciato}[section]

\maketitle

\section{Distribuzione della probabilità di urto}
Oltre a sapere il numero tipico di urti in un sistema chiuso, ci
si può anche chiedere quale sia la distribuzione di probabilità degli urti in un
sistema. Ci si aspetta dunque che il valore atteso degli urti sia dato proprio da
\begin{align}
    \left< n \right>  = \frac{t}{\tau}
\end{align}
Questa probabilità discreta, posso indicare quale è la probabilità che in questo tempo $t$
ci siano delle collisioni. Si prende l'intervallo di tempo e lo si divide in tanti
intervalli piccoli. Se li prendessi sufficientemente piccoli posso dire che
in ogni intervallo avvenga un solo urto; in ognuno di essi posso definire la
probabilità di collisione come $\frac{t}{N\tau}$ e la probabilità che non 
avvenga dunque $1 - \frac{t}{N\tau}$. Tuttavia, devo considerare anche che
negli altri intervalli non avvenga alcun urto:
\begin{align}
    \left(\frac{t}{\tau N}\right)^{n} \left(1 - \frac{t}{N\tau}\right)^{N - n}
\end{align}
Ottengo allora il numero medio di urti come
\begin{align}
    \begin{pmatrix} n \\
    N \end{pmatrix} = \frac{N!}{n!(N - n)!} 
\end{align}
Facendo tendere $N \to  +\infty $, posso ottenere la probabilità di urto di 
un singolo intervallo $p_n$ si ottiene secondo la distribuzione binomiale:
\begin{align}
    p_n = \frac{N!}{n!(N - n)!} \left(\frac{t}{\tau N}\right)^{n}\left(1 - \frac{t}{\tau N}\right)^{N - n}
\end{align}
I cui limiti: 
\begin{gather*}
    \lim_{N \to \infty } \frac{N!}{n!(N - n)!} \approx \frac{N^{n}}{n!} \\
    \lim_{N \to \infty } \left(1 - \frac{t}{\tau N}\right)^{N - n} \approx \left(1 - \frac{t}{\tau N}\right)^{N} 
\end{gather*}
Allora
\begin{gather*}
    p_n = \lim_{N \to \infty }  \frac{N^{n}}{n!} \left(\frac{t}{\tau N}\right)^{n}\left(1 - \frac{t}{\tau N}\right)^{N} = \frac{1}{n!}\left< n! \right>^{n}e^{-\left< n \right> } 
\end{gather*}
Ossia una distribuzione di Poisson. Questa distribuzione governa tutti i
processi discreti in fisica nei quali non ci si aspetta alcun tipo di correlazione 
tra i singoli eventi stocastici. Allora posso ottenere la probabilità totale come
\begin{gather*}
    \sum_{n = 0}^{\infty } p_n = e^{-\frac{t}{\tau}} \sum_{n = 0}^{\infty } \frac{1}{n!}\left(\frac{t}{\tau}\right)^{n} = e^{-\frac{t}{\tau}}  e^{\frac{t}{\tau}} = 1
\end{gather*} 
Da questa posso anche stimare il tempo medio tra gli urti: infatti tutti gli urti
non avvengono nello stesso istante di tempo.
\begin{align}
    f(t)\ dt =e^{-\frac{t}{\tau}}\frac{dt}{\tau} \ \Longrightarrow \ f(t) = \frac{1}{\tau}e^{-\frac{t}{\tau}}
\end{align}
Espressa come la probabilità che non ci siano urti fino a $t$ moltiplicata
per la probabilità che ci siano urti da $t$ fino a $t + dt$. 
Ci si aspetta che il tempo medio di urto deve essere
\begin{align}
    \left< t \right> &= \int_{0}^{\infty } dt \frac{t}{\tau}e^{-\frac{t}{\tau}} = \tau \int_{0}^{\infty }xe^{-x}\ dx = \tau   \\
    \left< t^{2} \right> &= \int_{0}^{\infty } dt \frac{t^{2}}{\tau}e^{-\frac{t}{\tau}} = \tau \int_{0}^{\infty }xe^{-x}\ dx = 2\tau^{2}   \\
\end{align}
Si può calcolare adesso la varianza del tempo di urti coem:
\begin{align}
    \sigma_t^{2} = \left< t^{2} \right>  - \left< t \right>^{2} = 2\tau^{2} - \tau^{2} = \tau^{2} = \left< t \right>^{2}  
\end{align}
Si dà per buono che anche la distribuzione delle distanze degli urti
sarà la solita distribuzione esponenziale che prende il nome di
\begin{align}
    g(\lambda) = \frac{1}{l}e^{-\frac{\lambda}{l}}
\end{align}
Più è lungo l'intervallo di urto e minore è la probabilità di
ottenere un urto, dove $\lambda$ è la distanza dell'urto e $l$ è 
la distanza media degli urti. Si verifica (così come per $\tau$) con
\begin{align}
    \left< \lambda \right> &= \int_{0}^{\infty } g(\lambda)\lambda \ d\lambda = l  \\
    \left< \lambda^{2} \right> &= \int_{0}^{\infty } g(\lambda)\lambda^{2} \ d\lambda = 2l^{2}  
\end{align}
Si ottiene anche la varianza come
\begin{align}
    \sigma_\lambda^{2} = \left< \lambda^{2} \right> - \left< \lambda \right>^{2} = l^{2}  
\end{align} 

\section{Dinamica caotica}
Con queste considerazioni si può definire il \textbf{cammino aleatorio} o \textbf{random walk}, 
ossia quando la direzione di una particella ha una direzione qualsiasi sia dopo che prima
dell'urto, abbia una distribuzione uniforme per gli angoli pre e
post urto (la velocità dopo l'urto 
non è dunque correlata né con la velocità prima dell'urto né con la velocità della
particella con cui impatta). Sto dunque assumendo una ipotesi molto sbagliata
che però mi semplifica notevolmente il modello stocastico. Non solo non è corretto dire 
che l'urto "scorrela" le velocità, ma è proprio affermare il contrario in quanto nella meccanica
classica l'urto correla la velocità delle particelle. 
\begin{wrapfigure}{r}{0.4\textwidth}
    \centering
    \caption{Relazione tra coefficiente di impatto e angolo dopo l'impatto}
    \begin{tikzpicture}[scale=1.2]
        \draw(0, 0) circle (0.5);
        \draw(0.75, -0.7) circle (0.5);
        \draw[<-](1.25, -0.7) -- (2, -0.7);
        \draw[dashed](0, 0) -- (2, 0) node[midway, above] {$b$};
        \draw[dashed](0, 0) -- (1.5, -1.4);
        \filldraw(0, 0) circle(1pt);
        \filldraw(0.75, -0.7) circle (1pt);
        \draw(0, 0) -- (0.6, -2);
        \draw(1.2, 0) arc (0:-35:1.5) node[midway, right] {$\frac{\theta}{2}$};
    \end{tikzpicture}    
\end{wrapfigure}
Si può dimostrare che in realtà questa ipotesi sia corretta: 
prese due sfere omogenee di massa uguale che si urtano in maniera perfettamente 
elastica, la loro velocità finale dopo l'impatto sia legato al coefficiente di 
impatto e l'angolo della velocità finale dipende dall'angolo iniziale di impatto
(ossia l'angolo tra le velocità dei dischi). Il parametro di impatto
è dato da
\begin{gather*}
    b = 2R \sin \frac{\theta}{2} \ \Longrightarrow \ \frac{b}{2} = R\sin\frac{\theta}{2}
\end{gather*}
Dato che si può conoscere l'incertezza $\Delta b_1$, posso determinare
l'incertezza sull'angolo $\theta$ applicando la propagazione
degli errori:
\begin{gather*}
    \frac{b_1 + \Delta b_1}{2} = R\sin\left(\frac{\theta_1 + \Delta \theta_1}{2}\right)
\end{gather*}
Approssimando l'argomento del seno (poiché si presuppone che sia molto piccolo),
si può utilizzare lo sviluppo di Taylor e dunque ottenere:
\begin{gather*}
    \Delta \theta_1 \approx \frac{\Delta b_1}{R\cos\left(\frac{\theta_1}{2}\right)} = \frac{\Delta b_1}{R}
\end{gather*}
Il successivo urto sarà distante di una certa lunghezza, dunque se avessi una certa incertezza sull'angolo di 
uscita, per il secondo urto si potrebbe dire che
\begin{gather*}
    \Delta b_2 \approx l\Delta \theta_1 \ \Longrightarrow \ \Delta \theta_2 \approx \frac{\Delta b_2}{R} = \frac{l}{R}\Delta \theta_1
\end{gather*}
Allora per il terzo urto
\begin{gather*}
    \Delta \theta_3 \approx \frac{l}{R}\Delta \theta_2 = \left(\frac{l}{R}\right)^{2}\Delta \theta_1
\end{gather*}
Dunque quello che accade è che per $n$ collisioni, l'incertezza sull'angolo 
dopo l'impatto è
\begin{gather*}
    \Delta \theta_n = \left(\frac{l}{R}\right)^{n - 1}\Delta \theta_1
\end{gather*}
Questa incertezza cresce in maniera esponenziale in base al numero di urti;
si può ottenere anche il numero di collisioni che servono far sì che questa incertezza abbia ordine 
$\pi$ (ossia che non si sappia più la direzione dell particella).
\begin{gather*}
    \pi = \left(\frac{l}{R}\right)^{n - 1}10^{-20} \qquad \left\{\begin{array}{l}
        l \approx 5 \cdot  10^{-8}\ m \\
        R \approx 10^{-10} \ m
    \end{array}\right. \ \Longrightarrow \ n \approx 9
\end{gather*}
Se si conoscesse con precisione infinita il parametro iniziale, allora sarà possibile
determinare in maniera infinita il comportamento del sistema,
se invece non si conoscesse con precisione infinita il parametro iniziale,
allora dopo un tempo molto piccolo questa incertezza è tale che non si conosce più il
comportamento del suddetto sistema. Questa tipologia di studio di sistemi 
prende il nome di \textbf{dinamica caotica}.

\section{Random Walk}
\begin{wrapfigure}{r}{0.4\textwidth}
    \centering
    \caption{}
    \begin{tikzpicture}
        \draw[->](0, 0) -- (1, 1) node[at end, above] {$\vv{\lambda_1}$ };
        \draw[->](1, 1) -- (1.2, 0.4) node[at end, below] {$\vv{\lambda_2}$};
        \draw[->](1.2, 0.4) -- (3, 1.2) node[at end, above] {$\vv{\lambda_3}$};
    \end{tikzpicture}    
\end{wrapfigure}
E' possibile determinare quanta strada compie una singola particella
che si muove in modo caotico all'interno di un gas? L'ipotesi che sia 
un cammino aleatorio e che la direzione della velocità dopo l'urto sia
uniforme determinata, dunque il vettore posizione della particella è dato da
\begin{gather*}
    \vv{r} = \sum_{i = 1}^{n}\vv{\lambda_i}   
\end{gather*}
Allora il vettore posizione 
\begin{gather*}
    \vv{r} = \left| \vv{r}  \right|  = \vv{r} \cdot  \vv{r} \sum_{i = 1}^{n} \vv{\lambda_i} \cdot \sum_{j = 1}^{n}\vv{\lambda_j} = \sum_{i = 1}^{n}\sum_{j = 1}^{n}\lambda_i\lambda_j\cos\theta_{i, j} = \\
    \sum_{i = 1}^{n}\left(\lambda_i^{2} + \lambda_i \sum_{j\neq i }^{n}\lambda_j \cos\theta_{i, j} \right)               
\end{gather*}
Dato che
\begin{gather*}
    \sum_{j\neq i }^{n}\lambda_j \cos\theta_{i, j} = N\frac{1}{N}\sum_{j\neq i }^{n} \lambda_i \cos\theta_{i, j} 
\end{gather*}
Allora essendo
\begin{gather*}
    \frac{1}{N}\sum_{j\neq i }^{n} \lambda_i \cos\theta_{i, j} = \left< \lambda \cos\theta \right> = l \cos\theta 
\end{gather*}
Si ottiene il valore atteso della distribuzione del random walk dove $l$ è il cammino medio,
si può ottenere 
\begin{gather*}
    \left< \cos\theta \right> = \int_{-\pi}^{\pi}f(\theta)\cos\theta \ d\theta = \frac{1}{2\pi}\int_{-\pi}^{\pi}\cos\theta \ d\theta  = 0
\end{gather*}
Il primo pezzo della sommatoria è invece
\begin{gather*}
    r^{2} = \sum_{i = 1}^{n}\lambda_i^{2} = N\frac{1}{N}\sum_{i = 1}^{n}\lambda_i^{2} \ \Longrightarrow \ N\left< \lambda^{2} \right> 
\end{gather*}
Quando $N >> 1$. Allora si trova che, per un certo istante $t$, quando sono avvenuti 
$N$ urti, si ottiene che
\begin{gather*}
    r^{2}(t) = N\left< \lambda^{2} \right> = 2Nl^{2} 
\end{gather*}
Allora si ha che
\begin{gather*}
    Nl = \left< v \right>t \ \Longrightarrow \ r^{2}(t) = 2lvt 
\end{gather*}
Se la mia particella si muove ogni volta girando a caso, alla fine la direzione non
la posso conoscere: infatti sommando tutti i vettori posizione possibili devono essere zero
in quanto può andare in tutte le direzioni allo stesso modo. Tuttavia,
in modulo, posso vedere di quanto si è allontanata dall'urto:
\begin{align}
    r(t) = \sqrt{r^{2}(t)} \propto t^{\frac{1}{2}} 
\end{align}
Questo è l'esempio di un processo microscopico che macroscopicamente si
osserva con l'equazione di diffusione dei gas. Per esplorare una regione
di circa un metro, per una particella dell'aria, ci vorrebbe circa 
5 ore!! Questo vuol dire che la distribuzione di calore (o di odori) non avviene
per conduzione ma attraverso moti convettivi. 
Nelle ipotesi di equilibrio i valori di pressione e temperatura
dipendono dal valore atteso della velocità al quadrato $\left< v^{2} \right>$. 
Quello che conta è legare la temperatura al valore atteso della temperatura, ossia
l'energia cinetica del sistema totale fratto il numero di particelle. GLi urti non cambiano
questo risultato in quanto gli urti sono elastici e dunque non dissipano nessuna energia cinetica.
All'equilibrio dunque metterci o meno gli urti non cambia i risultati ottenuti e la
distribuzione della velocità è invariante dagli urti. 

\section{Moto Browniano}
Prendendo della particelle in sospensione all'interno di un liquido, 
queste particelle si muovono all'interno del liquido secondo il random walk.
Il primo ad accorgersene è stato Robert Brown nel 1828 che osservava al microscopio
le particelle di polline nell'acqua ma solo Einstein è riuscito a dare una spiegazione a questo fenomeno:
la sua spiegazione è stata che il polline e l'acqua sono all'equilibrio termodinamico tra 
di loro (chiamando con lettere maiuscole le particelle del moto browniano):
\begin{gather*}
    \frac{1}{2}m\left< v^{2} \right> = \frac{1}{2}M\left< V^{2} \right> = \frac{3}{2}R_B T 
\end{gather*}




\end{document}
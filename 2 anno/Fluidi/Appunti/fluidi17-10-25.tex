\documentclass[a4paper, oneside]{article}
\usepackage{graphicx}
\usepackage{amsthm}
\usepackage{amsmath}
\usepackage{amssymb}
\usepackage[a4paper,
            bindingoffset=0.2in,
            left=2cm,
            right=2cm,
            top=2cm,
            bottom=2cm,
            footskip=.25in]{geometry}
\usepackage[italian]{babel}
\usepackage{pgfplots}
\usepackage{tabularx}
\usepackage{tikz}
\usepackage{wrapfig}
\usepackage{color}
\usepackage[d]{esvect}
\definecolor{page}{rgb}{0.129,0.157,0.212}
\pagecolor{page}
\color{white}
\graphicspath{ {./images/} }
\usetikzlibrary{shapes.geometric}
\usetikzlibrary{datavisualization}
\usetikzlibrary{datavisualization.formats.functions}
\usetikzlibrary{patterns}
\pgfplotsset{width=10cm,compat=1.9}

\title{Appunti Termodinamici}
\author{Tommaso Miliani}
\date{17-10-25}

\begin{document}
\newtheoremstyle{theoremEnv}
                {}          % Space above
                {}          % Space below
                {\slshape}  % Body font
                {}          % Indent amount
                {\bfseries} % Head font
                {.}         % Punctuation after head
                {\newline}         % Space after theorem head
                {}          % Theorem head spec
\theoremstyle{theoremEnv}

\newtheorem{definition}{Definizione}[section]
\newtheorem{theorem}{Teorema}[section]
\newtheorem{lemma}{Proposizione}[section]
\newtheorem{observation}{Osservazione}[section]
\newtheorem{corollary}{Corollario}[theorem]
\newtheorem{example}{Esempio}[section]

\maketitle

\section{Continuo altra volta}
La funzione $g(T)$ dipende da come è fatto il gas anche se in generale
la funzione dipende solamente dalla classe a cui appartiene il gas.
\begin{itemize}
    \item Gas \textbf{Monoatomici}: sono gas che presentano solamente molecole
    monoatomiche;
    \item Gas \textbf{Diatomici}: sono gas con molecole diatomiche (anche fatte
    con atomi diversi come il $CO$).
    \item Gas \textbf{Poliatomici}: sono gas con più di una molecola (ma non lineare)
    per esempio il gas $CO_2$ è considerato diatomico in quanto è lineare.
\end{itemize}
Il valore della costante non dipende dalla temperatura ma solo dalla classe
di appartenenza del gas:
\begin{gather*}
    \begingroup
    \setlength{\tabcolsep}{10pt} % Default value: 6pt
\renewcommand{\arraystretch}{1.5} % Default value: 1
    \begin{tabular}[10]{c | c | c | c | c}
        & $\frac{C_V}{R}$ & $\frac{C_P}{R}$ & $\gamma = \frac{C_P}{C_V}$ & \\
        \hline
        Monoatomici & $\frac{3}{2}$ & $\frac{5}{2}$ & $\frac{5}{3}$ & \\
        \hline
        Diatomici & $\frac{5}{2}$ & $\frac{7}{2}$ & $\frac{7}{5}$ & \\
        \hline
        Poliatomici & 3 & 4 & $\frac{4}{3}$ &
    \end{tabular}
    \endgroup
\end{gather*}
Dove $\gamma$ è il rapporto tra le due capacità termiche. Gli esperimenti
ci dicono che quando la temperatura si abbassa notevolmente vicino a zero kelvin,
la capacità termica dei gas inizia a dipendere dalla temperatura fino a raggiungere zero. 
Avendo la dipendenza della capacità termica dalla temperatura, posso ricavare 
l'energia interna attraverso la capacità termica. Considerandoci
nel caso impossibile di $T = 0$ Kelvin, possiamo, per un gas perfetto
l'equazione:
\begin{gather*}
    \frac{dU}{dT} = 0
\end{gather*}
INtegrando da entrambe le parti posso ottenere che
\begin{gather*}
    \int_{0}^{T} \frac{dU}{dT} \ dT = \int_{0}^{T}\mathcal{C}_V \ dT \ \Longrightarrow \ U(T) - U(0) = \mathcal{C}_V T
\end{gather*}
Si è trovato allora che l'energia interna di un sistema a volume costatante è proprio
\begin{align}
    U(T) = \mathcal{C}_V T
\end{align}
Siccome si sa che 
\begin{gather*}
    \mathcal{C}_P - \mathcal{C}_V = nR
\end{gather*}
Dato che $\gamma \mathcal{C}_V = \mathcal{C}_P$, posso ottenere,
sostituendolo nell'espressione di prima e poi nell'energia interna:
\begin{align}
    U(T) = T\frac{nR}{\gamma - 1} = \frac{pV}{\gamma - 1}
\end{align}

\section{Determinare le isoterme adiabatiche di un gas perfetto}
Si possono determinare ora le isoterme adiabatiche di un gas perfetto,
in quanto non c'è scambio di energia termica con l'esterno. Adesso
posso utilizzare il primo principio specializzato in tanti aspetti:
\begin{gather*}
    dU = - \delta L \ \Longrightarrow \ \mathcal{C}_V dT = - pdV
\end{gather*}
Allora posso ottenere l'equazione per l'isoterma attraverso la seguente:
\begin{gather*}
    \mathcal{C}_V \ dT = -\frac{nRT\ dV}{V} \ \Longrightarrow \ \frac{dT}{T} = -\frac{nR}{\mathcal{C}_V}\frac{dV}{V}
\end{gather*}
Ai fini della notazione sappiamo che $T, n, \mathcal{C}_V$ sono costanti. Per 
semplificare ancora di più l'equazione, posso esprimere $\mathcal{C}_V$ in funzione
del rapporto $\gamma$. Adesso possiamo avere una sostituzione comoda che
mi permette di annullare la dipendenza da $nR$:
\begin{gather*}
    \frac{dT}{T} = -(\gamma - 1)\frac{dV}{V}
\end{gather*}
SI può integrare da un certo stato di riferimento indicato
con le coordinate termodinamiche $O \equiv T_0, p_0, V_0$ fino ad uno stato
qualunque che vogliamo considerare $S \equiv (T, p, V)$. Dato che la pressione
non esiste nell'equazione (è fissata), allora posso semplicemente integrare
\begin{gather*}
    \int_{T_0}^{T} \frac{dT}{T} = -(\gamma - 1)\int_{V_0}^{V} \frac{dV}{V}
\end{gather*}
Chiaramente si sta considerando una trasformazione quasi statica, allora
posso integrare questa espressione e considerare queste considerazioni come valide,
risolvendo ora l'integrale:
\begin{gather*}
    \ln \frac{T}{T_0} = -(\gamma - 1)\ln\frac{V}{V_0}
\end{gather*}
Adesso posso fare l'esponenziale per entrambi i membri e dunque posso 
ottenere il rapporto tra le pressioni in funzione del volume:
\begin{gather*}
    \frac{T}{T_0} = \left(\frac{V}{V_0}\right)^{-(\gamma -1)} 
\end{gather*}
Ossia
\begin{gather*}
    TV^{\gamma -1} = T_0V_0^{\gamma -1}  
\end{gather*}
Si è ottenuta allora la prima legge che lega le trasformazioni
isoterme adiabatiche alla legge dei gas come
\begin{align}
    TV^{\gamma -1} = \text{const} 
\end{align}
Vale anche l'analoga
\begin{wrapfigure}{r}{0.4\textwidth}
    \centering
    \caption{}
    \begin{tikzpicture}
        \draw[->](0, 0) -- (4, 0) node[at end, below] {$V$};
        \draw[->](0, 0) -- (0, 3)  node[at end, left] {$p$};
        \filldraw(1, 3) circle (1pt);
        \draw(1, 3) .. controls (1.5, 1.2) and (2.25, 0.8) .. (3, 0.5);
        \draw[dashed](1, 3) .. controls (1.5, 2) and (2, 1.6) .. (3, 1.5);
        \draw[dashed](3, 1.5) -- (3, 0) node[at end, below] {$V_f$};
        \draw[dashed](1, 0) -- (1, 3) node[at start, below] {$V_i$};
    \end{tikzpicture}    
\end{wrapfigure}
\begin{align}
    T^{\gamma}p^{1 - \gamma} = \text{const}  
\end{align}
E' chiaro che se volessi altre coordinate potrei semplicemente 
esprimerle attraverso la legge di stato dei gas perfetti per ottener 
l'espressione di altre coordinate termodinamiche; per esempio
possiamo ottener
\begin{align}
    pV^{\gamma} = \text{const} 
\end{align}
Questa curva non è proprio né un iperbole né un paraboloide. La curva
isoterme sono meno ripide delle adiabatiche. Ne segue che il lavoro
compiuto per fare una trasformazione isoterma è sempre maggiore
della corrispondente trasformazione adiabatica dallo stesso
stato iniziale allo stesso volume finale. Si dimostra considerando il lavoro di una trasformazione
adiabatica come
\begin{gather*}
    L = \int_{V_0}^{V} p \ dV 
\end{gather*}
Dato che
\begin{gather*}
    L = \frac{p_0V_0^{\gamma} }{V^{\gamma} } \ \Longrightarrow \ L = \int_{V_0}^{V} \frac{p_0V_0^{\gamma} }{V^{\gamma} } \ dV = p_0 \int_{V_0}^{V} \left(\frac{V_0}{V}\right)^{\gamma} \ dV 
\end{gather*}
Allora si ha
\begin{gather*}
    L = p_0V_0\int_{1}^{\frac{V}{V_0}} x^{-\gamma} \ dx = \frac{p_0 V_0}{1 - \gamma} \left(\left(\frac{V}{V_0}\right)^{1 - \gamma}  - 1\right) 
\end{gather*}
Dove $x = \frac{V}{V_0}$. Dato che $\gamma > 1$, si ha che il lavoro è esattamente
\begin{gather*}
    L = \frac{nRT_0}{\gamma - 1} \left(1 - \left(\frac{V}{V_0}\right)^{1 - \gamma}\right) 
\end{gather*}
Ottenendo allora una nuova equazione per l'isoterma per un gas perfetto:
\begin{align}
    pV^{\gamma} = p_0V_0^{\gamma} = nRT_0V_0^{\gamma -1}   
\end{align}
Con $\gamma = 1$ si ottiene esattamente l'equazione di stato per i gas reali. Nel
calcolo del lavoro non si è utilizzato il rapporto tra le capacità termiche (e non
si può mettere $\gamma = 1$ anche se posso considerare il suo limite)
\begin{gather*}
    \lim_{\gamma \to 1} \frac{nRT_0}{\gamma - 1} \left(1 - \left(\frac{V}{V_0}\right)^{1 - \gamma}\right) 
\end{gather*}
Posso allora ottenere il lavoro come
\begin{gather*}
    L = nRT_0 \ln\left(\frac{V_0}{V}\right)
\end{gather*}
Possiamo definire una classe di trasformazioni con un esponente che
$k$ che vogliamo in modo tale che la pressione ed il volume
sono legate tra di loro secondo una costante. 

\begin{wrapfigure}{r}{0.4\textwidth}
    \centering
    \caption{Visualizzazione delle curve politropiche}
    \begin{tikzpicture}
        \draw[->](0, 0) -- (3, 0) node[at end, below] {$V$};
        \draw[->](0, 0) -- (0, 3) node[at end, left] {$p$};
        \filldraw(1, 2) circle (1pt);
        \draw[green](0.5, 2) -- (2.5, 2) node[at end, right] {$k = 0$};
        \draw[cyan](1, 0.5) -- (1, 2.5) node[at start, below] {$k = \infty $};
        \draw(0.75, 2.5) .. controls (1.2, 1.4) and ( 2, 1) .. (2.5, 0.8 ) node[at end, right] {$k = 1$};
    \end{tikzpicture}    
\end{wrapfigure}
Per ogni valore del parametro $k$ posso definire classi di trasformazioni che
prendono il nome di \textbf{politropiche}:
\begin{itemize}
    \item $k < 1$: curve meno ripide;
    \item $k > 1$: curve più ripide;
    \item $k = \gamma$: equazioni della famiglia delle adiabatiche.
    \item $k = 1$: sono isoterme.
    \item $k = 0$: E' una trasformazione a pressione costante;
    \item $k \to \infty $: E' una trasformazione a volume costante. 
\end{itemize}
Queste funzioni sono un'eccellente approssimazioni delle trasformazioni termodinamiche reali. Posso
determinare la quantità infinitesima di lavoro durante una di queste trasformazioni politropiche è
definito come
\begin{gather*}
    \delta L = p \ dV = p_0V_0^{k}V^{-k} \ dV = p_0 V_0^{k} d\frac{V^{1 - k} }{1 - k}   = \\
    d\left(\frac{p_0V_0^{k} V^{-k} V}{1 - k}\right) = d\left(\frac{pV}{1 - k}\right) = d\left(\frac{nRT}{1 - k}\right)
\end{gather*}
Allora posso ottenere che il lavoro è proprio
\begin{gather*}
    \delta L = \frac{nR}{1 - k} \ dT
\end{gather*}
Allora per ogni trasformazione politropica quasi statica per un
gas perfetto con $k\neq 1$ il lavoro è esprimibile come
\begin{align}
    L_k = \frac{nR}{1 - k}(T - T_0) = \frac{nR}{1 - k}\Delta T
\end{align}

\subsection{Utilizzo del primo principio nelle politropiche}
Dato il primo principio 
\begin{gather*}
    (\delta Q)_k = dU + + \delta L = \mathcal{C}_V dT + \frac{nR}{1 - k}dT
\end{gather*}
E quindi si ha l'espressione per la capacità termica della 
politropica per un gas perfetto nelle condizioni di stato  quando
$k = 0$:
\begin{gather*}
    \mathcal{C}_k = \mathcal{C}_V + \frac{nR}{1 - k}
\end{gather*}
Dove
\begin{gather*}
    \mathcal{C}_V = \frac{nR}{\gamma - 1}
\end{gather*}
In una isoterma ($k = 1$), possiamo ottenere l'espressione
della pressione come $p = \text{const}\rho$ e dunque nelle
politropiche:
\begin{gather*}
    p = \text{const}\rho^{k} 
\end{gather*}


\section{Misurazione della capacità termica di un gas perfetto}
\begin{wrapfigure}{r}{0.4\textwidth}
    \centering
    \caption{L'esperimento di Ruchardt}
    \begin{tikzpicture}
        \draw(0, 0) -- (4, 0) -- (4, 2) -- (2.5, 2) -- (2. 5, 3);
        \draw(0, 0) -- (0, 2) -- (1.5, 2) -- (1.5, 3);
        \draw(2, 2.5) circle (0.5) node[anchor = center] {$m$};
    \end{tikzpicture}    
\end{wrapfigure}
L'esperimento per determinare la capacità termica di un gas perfetto 
è stato per la prima volta fatto dal fisico tedesco Ruchardt. L'esperimento
prevede una bottiglia (per cui con un collo molto sottile\footnote{Il collo nel disegno
non è sottile per semplicità di rappresentazione}). Si inserisce ora una palla nel collo di bottiglia
e all'interno del recipiente il gas per cui si vuole determinare la capacità termica 
(ovviamente nelle condizioni di gas perfetto).
La pallina all'interno del collo è in grado di scorrere
senza attrito, dunque la pressione del gas sulla
pallina è uguale a quella atmosferica $p = p_0$ ma,
dato che c'è anche la massa $m$ a premere sul gas
\begin{gather*}
    p = p_0 + \frac{mg}{S}
\end{gather*}
Dove $S$ è l'area del cilindro del collo della bottiglia poiché (oltre ad un motivo geometrico)
le forze che agiscono sulla sfera c'è anche un vincolo fisico: se la risultante
della forze non fosse zero, allora il corpo ruoterebbe su sé stesso generando
energia cinetica senza che gliene sia fornita alcuna.
Se la pallina si muove di una certa $y$, molto piccola, rispetto al sistema di riferimento
allora il volume del gas varia in relazione allo spostamento per la sezione del collo di bottiglia:
\begin{gather*}
    dV = Sy
\end{gather*}
Adesso posso esprimere questa variazione di volume secondo la legge
del gas perfetto nelle trasformazioni politropiche:
\begin{gather*}
    pV^{\gamma} = \text{const} 
\end{gather*}
Posso allora dire che
\begin{gather*}
    V^{\gamma}dp + \gamma V^{\gamma -1} pdV = 0 \\
    dp + \gamma \frac{V^{\gamma -1 } }{V^{\gamma} } pdV = 0  
\end{gather*}
Ossia
\begin{gather*}
    dp = -\frac{\gamma p}{V}dV = -\frac{\gamma pS }{V}y
\end{gather*}
Dato che è presente una pressione sulla pallina, allora ci deve anche
essere una forza associata alla pressione a cui è sottoposta la pallina:
\begin{gather*}
    F = Sdp = -\gamma \frac{pS^{2} }{V}y
\end{gather*}
Allora, posso impostare la prima cardinale e
ottenere l'espressione di un o oscillatore armonico
\begin{gather*}
    ma = -\frac{\gamma pS^{2} }{V}y
\end{gather*}
Allora posso dire che la pulsazione del moto è
\begin{gather*}
    \omega = \sqrt{\frac{\gamma p S^{2} }{mV}} 
\end{gather*}
E dunque posso ottenere il periodo di oscillazione
\begin{gather*}
    \tau = \frac{2\pi}{\omega}
\end{gather*}
Risolvendo ed elevando al quadrato posso dire che
\begin{align}
    \gamma = \frac{4\pi^{2}mV }{\tau^{2}S(p_0S + mg) }
\end{align}

\end{document}
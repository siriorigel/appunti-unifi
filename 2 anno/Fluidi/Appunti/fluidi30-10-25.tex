\documentclass[a4paper, oneside]{article}
\usepackage{graphicx}
\usepackage{amsthm}
\usepackage{amsmath}
\usepackage{amssymb}
\usepackage[a4paper,
            bindingoffset=0.2in,
            left=2cm,
            right=2cm,
            top=2cm,
            bottom=2cm,
            footskip=.25in]{geometry}
\usepackage[italian]{babel}
\usepackage{pgfplots}
\usepackage{tabularx}
\usepackage{tikz}
\usepackage{wrapfig}
\usepackage{color}
\usepackage[d]{esvect}
\definecolor{page}{rgb}{0.129,0.157,0.212}
\pagecolor{page}
\color{white}
\graphicspath{ {./images/} }
\usetikzlibrary{shapes.geometric}
\usetikzlibrary{datavisualization}
\usetikzlibrary{datavisualization.formats.functions}
\usetikzlibrary{patterns}
\pgfplotsset{width=10cm,compat=1.9}

\title{Termodinamica}
\author{Tommaso Miliani}
\date{30-10-25}

\begin{document}
\newtheoremstyle{theoremEnv}
                {}          % Space above
                {}          % Space below
                {\slshape}  % Body font
                {}          % Indent amount
                {\bfseries} % Head font
                {.}         % Punctuation after head
                {\newline}         % Space after theorem head
                {}          % Theorem head spec
\theoremstyle{theoremEnv}

\newtheorem{definition}{Definizione}[section]
\newtheorem{theorem}{Teorema}[section]
\newtheorem{lemma}{Proposizione}[section]
\newtheorem{observation}{Osservazione}[section]
\newtheorem{corollary}{Corollario}[theorem]
\newtheorem{example}{Esempio}[section]

\maketitle

\section{Il teorema di Clausius}
Dato che
\begin{gather*}
    \eta \leq \eta_R
\end{gather*}
Allora posso dire che
\begin{gather*}
    1 + \frac{Q_2}{Q_1} \leq 1  -\frac{T_2}{T_1}
\end{gather*}
Dunque si ottiene che
\begin{gather*}
    \frac{Q_1}{T_1} + \frac{Q_2}{T_2} \leq 0
\end{gather*}
Questa disuguaglianza è uguale a zero se e solo se la
macchina è reversibile. Durante delle trasformazioni cicliche, se si
scambia calore con dei termostati si ha che
\begin{gather*}
    \sum \frac{Q_1}{T_1} \leq 0
\end{gather*} 
\begin{wrapfigure}{r}{0.4\textwidth}
    \centering
    \caption{Sistema in grado di dimostrare il teorema di Clausius.}
    \begin{tikzpicture}
        \draw(0, 0) circle (0.5) node[midway] {$M$};
        \draw(0.25, -0.42) -- (1, -0.75) node[at end, below] {$L$};
        \draw(-1, 1) rectangle (-0.5, 1.5) node[midway] {$T_1$};
        \draw(0, 1) rectangle (0.5, 1.5) node[midway] {$T_2$};
        \filldraw(1, 1.25) circle (1pt);
        \filldraw(1.5, 1.25) circle (1pt);
        \draw(2, 1) rectangle (2.5, 1.5) node[midway] {$T_n$};
        \draw(-0.75, 1.5) -- (-0.75, 2);
        \draw(-0.75, 2.5) circle (0.5) node[anchor = center] {$R$};
        \draw(-1.25, 2.5) -- (-1.75, 2.5) node[at end, above] {$L$};
        \draw(-0.75, 3) -- (0, 4) node[midway, left] {$Q_{1,0}$};
        \draw(-0.5, 4) rectangle (2, 5) node[midway] {$T_0$};
        \filldraw(0.25, 2.5) circle (1pt);
        \filldraw(1, 2.5) circle (1pt);
        \filldraw(1.75, 2.5) circle (1pt);
        \draw(-0.75, 1) -- (-0.25, 0.42) node[midway, left] {$Q_1$};
    \end{tikzpicture}    
\end{wrapfigure}
Che fa zero solamente nel caso reversibile (dimostrato da Clausius). Per
dimostrarlo possiamo innanzitutto determinare il sistema termodinamico: questo sistema $M$ 
che compie solamente dei cicli. Questo sistema interagisce meccanicamente 
con l'ambiente compiendo del lavoro; inoltre scambia con $n$ termostati
una certa quantità di calore $Q_1, \dots, Q_n$. Posso apporre ai termostati
una macchina reversibile che compie del lavoro in modo tale che il calore che 
prende la macchina $R$ è lo stesso calore che deve fornire 
la macchina $M$ al termostato $T_1$ e viceversa. In questo modo 
il bilancio energetico del termostato $1$ risulta nullo. Si 
applicano altre $R_{n - 1}$ macchine termiche in modo tale che
il calore totale scambiato con il termostato $T_1$ e tutti gli altri termostati
siano zero. Dato che tutte le macchine sono reversibili, allora sono anche
delle macchine di Carnot, dunque si può imporre 
\begin{gather*}
    \frac{Q_{i, 0}}{T_0} - \frac{Q_i}{T_i}
\end{gather*}
Dunque il calore scambiato con il termostato $T_0$ è proprio
\begin{gather*}
    Q_{i, 0} = T_0 \frac{Q_i}{T_i}
\end{gather*}

\begin{wrapfigure}{r}{0.4\textwidth}
    \centering
    \caption{Schematizzazione del sistema del teorema di Clausius}
    \begin{tikzpicture}
        \draw(0, 0) ellipse (1 and 0.5) node[midway] {$M + \sum R$};
        \draw(0, 0.5) -- (0, 1.5);
        \draw(-1, 1.5) rectangle (1, 2.5) node[midway] {$T_0$}; ù
        \draw(1, 0) -- (2, 0) node[at end, right] {$L_{tot}$};
    \end{tikzpicture}    
\end{wrapfigure}
Il Sistema, complessivamente, fa un certo lavoro che sarà la somma
del lavoro della macchina $M$ e delle altre macchine reversibili. Il sistema
complessivo è un sistema che scambia calore con un unico termostato $T_0$. QUesto 
vuol dire che è un sistema ciclico che fa (o subisce) una certa quantità 
di lavoro
\begin{gather*}
    L_{tot} = L +\sum L_i
\end{gather*}
Ma scambia calore solamente con $T_0$. Possiamo allora dire che
\begin{gather*}
    L_{tot} = Q_{tot} = \sum Q_{i, 0} \leq 0
\end{gather*}
Si sostituisce con il calore ottenuto prima
\begin{gather*}
    T_0 \sum \frac{Q_i}{T_i} \leq 0 \ \Longrightarrow \ \sum \frac{Q_i}{T_i} \leq 0
\end{gather*}
Adesso, nell'ipotesi in cui la macchina $M$ sia reversibile, per non 
violare il secondo principio della Termodinamica, allora 
quella sommatoria deve necessariamente essere uguale a zero per evitare che 
l'unica processo che si otterrebbe da questo sistema scambiando calore con un solo 
con un termostato sia lavoro positivo.  
Posso riscrivere la tesi del teorema di Clausius come 
\begin{gather*}
    Q_i = \sum_{k = 1}^{m_i} \delta Q_k^{(i)}  
\end{gather*}
E dunque
\begin{gather*}
    \sum_{i = 1}^{n} \frac{Q_i}{T_i} = \sum_{i = 1}^{n} \frac{1}{T_i} \sum_{k = 1}^{m_i} \delta Q_k^{(i)}  = \sum_{i = 1}^{n} \sum_{k = 1}^{m_i} \frac{\delta Q_k^{(i)} }{T_i} 
\end{gather*}
Posso allora riscrivere la seconda sommatoria come un integrale e dunque
\begin{gather*}
    \sum_{i = 1}^{n} \int_{i} \frac{\delta Q^{(i)} }{T_i} = \oint \frac{\delta Q}{T} \leq 0
\end{gather*}
Questo è possibile pensando che il ciclo, invece che scambiare calore
con un numero finito di termostati, scambia quantità infinitesime di calore 
con una distribuzione continua di sorgenti:
\begin{gather*}
    \lim_{n \to \infty } \sum_{i = 1}^{n}\frac{\delta Q_i}{T_i}  = \oint \frac{\delta Q}{T}
\end{gather*}
Questa notazione è dunque una generalizzazione del teorema di Clausius 
per cui si implica che se è valido per un caso finito allora è possibile che sia
valido anche per il caso in cui il numero di termostati sia infinito. Se
dunque le trasformazioni sono solamente reversibile allora 
\begin{gather*}
    \oint \left(\frac{\delta Q}{T}\right)_{REV} = 0
\end{gather*}

\section{Definizione di Entropia}
\begin{wrapfigure}{r}{0.4\textwidth}
    \centering
    \caption{}
    \begin{tikzpicture}
        \draw(0, 0) circle (2);
        \filldraw(-1, -1.72) circle (1pt) node[anchor = north] {$A$};
        \filldraw(1, 1.72) circle (1pt) node[anchor = south] {$B$};
    \end{tikzpicture}    
\end{wrapfigure}
Nel caso in cui il ciclo sia reversibile, è possibile prendere due stati qualunque
attraversati da questa trasformazione e dire che l'integrale di linea
è possibile scomporlo in due pezzi
\begin{gather*}
    \int_{A}^{B} \left(\frac{\delta Q}{T}\right)_{REV} + \int_{B}^{A} \left(\frac{\delta Q}{\delta T}\right)_{REV} = 0
\end{gather*}
Potrei voler invertire gli estremi dell'integrale quando voglio percorrere il cammino al contrario,
e in questo caso è possibile farlo in quanto siamo nell'ipotesi di
trasformazione reversibile (non è vero però nel caso in cui le
trasformazioni non siano reversibili). 
\begin{gather*}
     \int_{A}^{B} \left(\frac{\delta Q}{T}\right)_{REV}  = - \int_{B}^{A} \left(\frac{\delta Q}{T}\right)_{REV} 
\end{gather*}
Dato che posso trovare una funzione di stato tale per cui è possibile
esprimere quell'integrale, si può definire
\begin{gather*}
     \int_{A}^{B} \left(\frac{\delta Q}{T}\right)_{REV} = f(A, B) \ \Longrightarrow \ f(A, A) = 0
\end{gather*}
Posso esprimere questo come differenza tra i valori di una certa funzione
di stato $S$ tra istante iniziale e finale:
\begin{align}
        S(B) - S(A) =  \int_{A}^{B} \left(\frac{\delta Q}{T}\right)_{REV} 
\end{align}
Ossia l'\textbf{Entropia}. Si definisce dunque l'entropia di un certo stato
termodinamico $P$ rispetto ad un certo stato di equilibrio $O$ come
\begin{gather*}
    S_P = \int_{O}^{P} \left(\frac{\delta Q}{T}\right)_{REV} 
\end{gather*}
L'entropia è una quantità estensiva ed additiva: l'entropia di due
sistemi risulta essere la somma della loro entropia. L'entropia non 
è una funzione della trasformazione: infatti dire che l'entropia durante 
una trasformazione varia non ha senso poiché, a meno che la
trasformazione non sia quasi statica, non si riesce a determinare quello 
che accade all'entropia durante una trasformazione. L'entropia 
diventa allora una funzione di stato, dunque, differenziando, posso ottenere
\begin{gather*}
    dS = \frac{\delta Q_{REV}}{T}
\end{gather*}
Se si scambia dunque del calore in modo reversibile con un termostato 
a temperatura $T$, il differenziale dell'entropia diventa un differenziale esatto
anche se il differenziale del calore non è esatto. Tutte le trasformazioni reversibili
che si possono descrivere sono anche quasi statiche per cui posso 
determinare l'entropia per tutte le trasformazioni reversibili. Tuttavia
questa ipotesi non è molto forte in quanto è vero che non sono in grado 
di trovare delle trasformazioni reversibili che non siano quasi statiche,
ma questo non mi assicura che esistano. 

\section{Il criterio per determinare se una trasformazione è reversibile o meno}
\begin{wrapfigure}{r}{0.4\textwidth}
    \centering
    \caption{}
    \begin{tikzpicture}
        \filldraw(0, 0) circle (1pt) node[anchor = north] {$A$};
        \filldraw(2, 2) circle (1pt) node[anchor = south] {$B$};
        \draw[dashed] (0, 0) .. controls (0.6, 2) and (1.5, 2.2) .. (2, 2) node[midway, above] {$1$};
        \draw(0, 0) .. controls (0.6, 0.2) and (1.6, 0.8) .. (2, 2) node[midway, below] {$2$};
    \end{tikzpicture}    
\end{wrapfigure}
La definizione di entropia mi permette di determinare se una
trasformazione sia reversibile oppure no (è un procedimento analitico).
Immagino che tra due stati $A$ e $B$ ci sia uno spazio degli stati
(che è continuo come lo spazio $\mathbb{R}^{2}$ ) e ipotizzo che
esista almeno una trasformazione reversibile che connetta i due punti
(anche se non posso a priori dire se è quasi statica oppure no). 
Le due trasformazioni $1$ e $2$ costituiscono un ciclo (con la trasformazione
2 che è reversibile):
\begin{gather*}
    AB^{(1)} + BA^{(2)} = \text{ciclo}  
\end{gather*}
E dunque posso dire che
\begin{gather*}
    \int_{A, 1}^{B} \frac{\delta Q}{T} + \int_{B, 2}^{A} \left(\frac{\delta Q}{T}\right)_{REV} \leq 0
\end{gather*}
Dato che la trasformazione $AB^{(2)}$ è una trasformazione reversibile allora 
\begin{gather*}
    \int_{A, 1}^{B}\frac{\delta Q}{T} - \int_{A, 2}^{B} \left(\frac{\delta Q}{T}\right)_{REV} \leq 0
\end{gather*} 
E dunque l'entropia:
\begin{gather*}
    S(B) - S(A) \geq \int_{A, 2}^{B} \frac{\delta  Q}{T}
\end{gather*}
Se il segno dell'entropia non fosse quello, allora si sarebbe commesso uno
dei seguenti tre errori:
\begin{enumerate}
    \item Ho sbagliato i conti
    \item Esiste una trasformazione che in realtà non esiste (una trasformazione
    adiabatica che connette due stati alla stessa temperatura).
    \item Ho violato il secondo principio della termodinamica per come è stata definita
    l'entropia. 
\end{enumerate}
Si ottiene allora l'algoritmo per determinare se una trasformazione è reversibile:
\begin{gather*}
    \Delta S > \int_{A, 2}^{B} \frac{\delta Q}{T} \ \Longrightarrow \ \textbf{Irreversibile} \\
    \Delta S = \int_{A, 2}^{B} \frac{\delta Q}{T} \ \Longrightarrow \ \textbf{Reversibile}
\end{gather*}
C'è un caso particolare in cui quell'integrale è molto facile: ossia se non 
c'è scambio di calore tra il sistema e l'ambiente e dunque per un
sistema termicamente isolato $\Delta S \geq 0$. L'entropia vincola dunque l'evoluzione
futura del sistema: se non scambia calore con l'ambiente esterno, allora nel futuro la sua entropia
è destinata a crescere. L'entropia, che è una quantità fisica misurabile, 
mi permette di determinare se lo stato di un sistema
si trova più avanti nel tempo oppure più indietro nel tempo. E' per questo 
che si dice che, dato che è un sistema chiuso, l'entropia dell'universo 
è destinata ad aumentare. 

\end{document}
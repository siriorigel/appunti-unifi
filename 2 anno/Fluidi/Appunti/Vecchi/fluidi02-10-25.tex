\documentclass[a4paper, oneside]{article}
\usepackage{graphicx}
\usepackage{amsthm}
\usepackage{amsmath}
\usepackage{amssymb}
\usepackage[a4paper,
            bindingoffset=0.2in,
            left=2cm,
            right=2cm,
            top=2cm,
            bottom=2cm,
            footskip=.25in]{geometry}
\usepackage[italian]{babel}
\usepackage{pgfplots}
\usepackage{tabularx}
\usepackage{tikz}
\usepackage{wrapfig}
\usepackage{color}
\usepackage[d]{esvect}
\definecolor{page}{rgb}{0.129,0.157,0.212}
\pagecolor{page}
\color{white}
\graphicspath{ {./images/} }
\usetikzlibrary{shapes.geometric}
\usetikzlibrary{datavisualization}
\usetikzlibrary{datavisualization.formats.functions}
\usetikzlibrary{patterns}
\pgfplotsset{width=10cm,compat=1.9}

\title{Appunti di fluidi}
\author{Tommaso Miliani}
\date{02-10-25}

\begin{document}
\newtheoremstyle{theoremEnv}
                {}          % Space above
                {}          % Space below
                {\slshape}  % Body font
                {}          % Indent amount
                {\bfseries} % Head font
                {.}         % Punctuation after head
                {\newline}         % Space after theorem head
                {}          % Theorem head spec
\theoremstyle{theoremEnv}

\newtheorem{definition}{Definizione}[section]
\newtheorem{theorem}{Teorema}[section]
\newtheorem{lemma}{Proposizione}[section]
\newtheorem{observation}{Osservazione}[section]
\newtheorem{corollary}{Corollario}[theorem]
\newtheorem{example}{Esempio}[section]

\maketitle

\section{Come è fatto un termometro}       
\begin{wrapfigure}{r}{0.4\textwidth}
    \centering
    \caption{}
    \begin{tikzpicture}
        \draw[->](0, 0) -- (4, 0) node[at end, below] {$x$};
        \draw[->](0, 0) -- (0, 4) node[at end, left] {$y$};
        \draw(1, 1) .. controls (1.4, 0.9) and (1.9, 2.2) .. (2, 3) node[at end, above] {$h(x, y) = T_1$};
        \draw(0, 2) -- (4, 2) node[at end, right] {$Y = Y_0$};
        \draw[dashed](1.78, 0) -- (1.78, 2) node[at start, below] {$x_1$};
        \draw(2, 1) .. controls (2.4, 0.9) and (2.9, 2.2) .. (3, 3) node[at end, right] {$h(x, y) = T_2$};
        \draw[dashed](2.78, 0) -- (2.78, 2) node[at start, below] {$x_2$};
    \end{tikzpicture}    
\end{wrapfigure}
Nei primi termometri si utilizzava il  mercurio in quanto
a temperatura ambiente è liquido ed ha un coefficiente di dilatazione
molto grande e soprattutto molto più lineare rispetto ad altri fluidi
come l'acqua. L'idea di un termometro è in generale quella di trovare la temperatura
di un dato sistema date come variabili $x, y$, che corrispondono rispettivamente
al volume e alla pressione. Posso allora decidere (dato che è comodo) di trovare una funzione
lineare prototipo del tipo:
\begin{gather*}
    T(x) = ax   + b
\end{gather*}
Adesso devo trovare un modo per poter fissare queste costanti: se io so che
ci sono delle situazioni che si verificano sempre alla stessa temperatura, allora
sono in grado di determinarle. Esistono quindi dei punti chiamati \textbf{punti fissi
termodinamici} : questi punti esistono e corrispondono proprio ai cambi di stato. Quando un materiale
passa da uno stato all'altro si verificano sempre alla stessa temperatura purché sia fissata almeno una
delle due variabili da cui ricavo la temperatura. Storicamente per determinare le scale dei
termometri si utilizzano i punti termodinamici fissi dell'acqua come pressione fissata a $p = 1 \ atm$.
\begin{gather*}
    T_1 = ax_1 + b \\
    T_2 = ax_2 + b
\end{gather*}
Posso allora ricavare le costanti come
\begin{gather*}
    a = \frac{T_2 - T_1}{x_2 - x_1} \\
    b = T_1 - x_1\frac{T_2 - T_1}{x_2 - x_1}
\end{gather*}
Tutti questi procedimenti sono validi se e solo se si rimane
ad un certo $Y$ assegnato. Questa è solo una definizione per una
curva che modellizzi la curva isoterma (è solamente una approssimazione
e non una legge). Utilizzando ora i punti fissi dell'acqua si ha
\begin{gather*}
    t(x) = 100\text{°}\frac{x - x_1}{x_2 - x_1}
\end{gather*}
La variabile $x$ è la temperatura dell'oggetto che si vuole
misurare mentre $x_1$ e $x_2$ sono la temperatura della scala iniziale e finale.

\subsection{Le varie scale}
La scala Fahrenheit è una scala che utilizza altri punti fissi rispetto all'acqua
utilizzando una miscela di acqua ghiaccio e ammoniaca. Si può ricavare una relazione tra
la temperatura Fahrenheit e la temperatura Celsius:
\begin{align}
    t_F[^{\circ}F ] = \frac{9}{5}t[^{\circ}C ] + 32
\end{align}
Quando misuro una temperatura con due termometri che utilizzano fluidi diversi
ma la stessa scala termometrica questi leggeranno sempre misure diverse
per tutte le temperature intermedie tra i due punti fissi in quanto la funzione
che ho utilizzato per approssimare la temperatura non è congruente con la curva di
temperatura effettiva. Un'altro problema per i termometri è la lenta evaporazione del 
fluido utilizzato che rende piano piano il termometro sempre meno preciso.


\section{I Gas}
\subsection{La storia della teoria dei gas e la definizione di gas perfetto}
Quando si fornisce tanta energia ad un sistema questo diventa un gas: si
è scoperto che i gas obbediscono a delle leggi molto semplici. Robert Boyle
ha scoperto, misurando pressioni e volume, che variando quelle grandezze il loro prodotto risultava
essere costante 
\begin{gather*}
    pV =  \text{const}
\end{gather*}
Successivamente il fisico Mariotte ha scoperto che tale costante è 
la temperatura. Molti anni dopo il fisico Gay-Lussac si è domandato come
possa dipendere una variabile dall'altra e ha trovato 
una dipendenza della pressione dalla temperatura, e lo stesso per il volume,  secondo le seguenti relazioni:
\begin{gather*}
    p = p_0 (1 + \beta T) \ (V\text{ const}) \\
    V = V_0 (1 + \gamma T) \ (p \text{ const})
\end{gather*}
Allora si vede che entro gli errori queste due costanti sono molto vicine tra
di loro e non dipendono dalla sostanza dalla quale si è fatto le misurazioni. 
Queste leggi valgono se e solo se si hanno piccole variazioni di temperatura 
e dunque si ottiene che
\begin{gather*}
    \beta \approx \gamma = \frac{1}{273}\ ^{\circ}C^{-1}  
\end{gather*}
Poco dopo il chimico Avogadro scopre che 
\begin{gather*}
    p = \text{const} \quad T = \text{const} \ \Longrightarrow \  V \propto n
\end{gather*}
Ossia il volume è direttamente proporzionale al numero di moli di un gas se e solo se il gas
è mantenuto a pressione e temperatura costanti.
Nonostante queste leggi siano storiche, queste sono tanto più valide tanto più un gas è
rarefatto: posso allora definire un limite secondo il quale
\begin{gather*}
    \lim_{p \to 0} \ \Longrightarrow \  \text{gas perfetto} 
\end{gather*}
Il gas si dice \textbf{perfetto} quando la pressione tende a zero e dunque le leggi scritte
per la pressione ed il volume valgono ed il valore dei coefficienti è esattamente:
\begin{gather*}
    \beta = \gamma = \frac{1}{273.15 \ ^\circ C }
\end{gather*}
Posso allora determinare la temperatura in funzione della pressione
quando il volume è costante:
\begin{gather*}
    T = \frac{273.15}{p_0}p - 273.15 \ [^{\circ}C ]
\end{gather*}
In questo modo posso rendere un gas sempre più rarefatto e misurarne la temperatura:
così posso estrapolare la curva di temperatura con volume costante 
in funzione della variazione della pressione. Se volessimo
una scala più comoda potremmo definire una nuova scala di temperatura
chiamata \textbf{scala Kelvin} (grado Kelvin non è giusto!) :
\begin{align}
    T = 273.15 + T(^{\circ}C ) \ [K]
\end{align}
\begin{wrapfigure}{r}{0.4\textwidth}
    \centering
    \caption{Il punto triplo}
    \begin{tikzpicture}
        \draw[->](0, 0) -- (4, 0) node[at end, below] {$T$};
        \draw[->](0, 0) -- (0, 4) node[at end, left] {$p$};
        \draw(0, 0) .. controls (0.5, 0.2)  and (0.8, 0.6).. (1, 1);
        \draw(1, 1) .. controls (2, 1.2) and (3, 1.7) .. (4, 3);
        \draw(1, 1) -- (0.5, 3.5);
        \filldraw[red](1, 1) circle (1pt);
    \end{tikzpicture}    
\end{wrapfigure}
Allora la temperatura assoluta definita a volume costante
utilizzando la scala Kelvin sarà:
\begin{gather*}
    T = \frac{p}{p_0}T_0
\end{gather*}
Esistono, oltre ai due punti fissi di fusione e ebollizione, esiste anche un unico
valore in certe condizioni di pressione, volume e temperatura nel 
quale le tre fasi coesistono simultaneamente che prende il nome di \textbf{punto triplo}.
Le condizioni del punto triplo dell'acqua sono:
\begin{gather*}
    T_{\text{triplo}} = 0.01\  ^{\circ}C = 273.16 \ K \\
    p_{\text{triplo}} = 611 \ Pa \approx 10^{-3} \ atm \\
\end{gather*}
Dato che il gas prefetto non esiste, io posso determinare la temperatura di un oggetto
con un termometro che contiene un gas perfetto è data da:
\begin{gather*}
    T = \frac{p}{p_3}T_3
\end{gather*}
Dato che il gas perfetto non esiste, io prendo una successione di pressioni
che convergono verso zero, di conseguenza la temperatura sarà sempre diversa:
la $p_3$ è la pressione che misura il termomemtro che misura quando lo metto in contatto
con l'acqua alla temperatura del punto triplo e cambierà sempre di poco. In questo modo posso ottenere una procedura
univoca per determinare la temperatura degli oggetti che tutti possono ripetere. 


\end{document}
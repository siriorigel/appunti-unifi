\documentclass[a4paper, oneside]{article}
\usepackage{graphicx}
\usepackage{amsthm}
\usepackage{amsmath}
\usepackage{amssymb}
\usepackage[a4paper,
            bindingoffset=0.2in,
            left=2cm,
            right=2cm,
            top=2cm,
            bottom=2cm,
            footskip=.25in]{geometry}
\usepackage[italian]{babel}
\usepackage{pgfplots}
\usepackage{tabularx}
\usepackage{tikz}
\usepackage{wrapfig}
\usepackage{color}
\definecolor{page}{rgb}{0.129,0.157,0.212}
\pagecolor{page}
\color{white}
\graphicspath{ {./images/} }
\usetikzlibrary{shapes.geometric}
\usetikzlibrary{datavisualization}
\usetikzlibrary{datavisualization.formats.functions}
\pgfplotsset{width=10cm,compat=1.9}

\title{Appunti fluidi}
\author{Tommaso Miliani}
\date{18-09-25}

\begin{document}
\newtheoremstyle{theoremEnv}
                {}          % Space above
                {}          % Space below
                {\slshape}  % Body font
                {}          % Indent amount
                {\bfseries} % Head font
                {.}         % Punctuation after head
                {\newline}         % Space after theorem head
                {}          % Theorem head spec
\theoremstyle{theoremEnv}

\newtheorem{definition}{Definizione}[section]
\newtheorem{theorem}{Teorema}[section]
\newtheorem{lemma}{Proposizione}[section]
\newtheorem{observation}{Osservazione}[section]
\newtheorem{corollary}{Corollario}[theorem]
\newtheorem{example}{Esempio}[section]

\maketitle

\section{Prototipo di funzioni che descrivono il potenziale di interazione}
\subsection{Funzioni di Leonard - Jones}
\begin{wrapfigure}{r}{0.4\textwidth}
    \centering
    \caption{La funzione di potenziale}
    \begin{tikzpicture}
        \draw[->] (-1, 0) -- (4, 0) node[at end, below] {$\frac{r}{r_0}$};
        \draw[->](0, -1) -- (0, 3) node[at end, left] {$\frac{V}{\epsilon}$};
        \draw(0.75, 3) ..controls (1, -2.5) and (1.35, -0.25) .. (2, -0.05);
        \draw(2, -0.05) -- (3.5, -0.05);
        \draw[dashed](3.5, -0.05) -- (4, -0.05);
        \filldraw(1, 0) circle (1pt) node[] {$\sigma$};
        \draw(1.2, 0) -- (1.2, -0.65) node[at end, below] {$r_0$};
    \end{tikzpicture}    
\end{wrapfigure}
Il potenziale di interazione è dato dalla seguente relazione
\begin{align}
    V(r) = \epsilon \left(\left(\frac{r_0}{r}\right)^{12} - 2\left(\frac{r_0}{r}\right)^{6} \right)
\end{align}
Il modello funziona molto
bene per i gas nobili in quanto sono atomi neutri che non interagiscono con nessun'altro atomo.
Anche per lo studio di casi più complessi il modello da seguire è essenzialmente questo entro certe approssimazioni;
spesso si riscrive in modo leggermente diversa questa formulazione, ossia nella seguente
forma 
\begin{align}
    V(r) = 4\epsilon \left(\left(\frac{\sigma}{r}\right)^{12} - \left(\frac{\sigma}{r}\right)^{6} \right)
\end{align}
Dove si ha che $r_0 = 2^{1/6} \cdot  \sigma \approx 1.12 \sigma$.
Sigma è il valore oltre al quale il potenziale diventa negativo, ossia il punto in cui
si azzera l'energia potenziale; il valore di $\epsilon$ per i gas nobili è dato da
\begin{gather*}
    \epsilon = 10^{-3} \sim 10^{-2} eV   
\end{gather*}
mentre il range di $\sigma$ (sempre per i gas nobili) è 
\begin{gather*}
    \sigma = 0.2 \sim 0.4 \ nm
\end{gather*}
L'interazione alla sesta è data dai termini di Van Der Walls che dipendono
dalle interazioni tra le nubi elettroniche; è possibile ricavare che il contributo attrattivo sia
di potenza $6$ mentre il contributo repulsivo è di potenza $12$ in quanto 
è computazionalmente comoda. Questo modello puramente meccanico ci permette di descrivere le interazioni
atomiche a livello approssimativo poiché non si può usare le leggi della meccanica classica.

\section{Studio del minimo dell'energia totale}
\subsection{Configurazione a minore energia}
Nello studio di sistemi conservativi si possono imparare un sacco di cose
dallo studio della funzione dell'energia potenziale senza andare a derivare;
per esempio in uno oscillatore armonico nel punto più
basso dell'energia potenziale il sistema sta fermo, se però si dà un po' di energia
al sistema, allora il sistema avrà una regione all'interno della quale si può muovere.
Dall'espressione dell'energia totale 
\begin{gather*}
    E = \frac{1}{2}m\sum_{i = 1}^{n}|\vec{v}_i|^{2} + U(\vec{r}_1, \dots, \vec{r}_n  )     
\end{gather*}
si può trovare il minimo dell'energia. Il livello minimo più intuitivo si ha quando tutte le particelle sono ferme,
tuttavia esisterà anche un altro punto di minimo che è più complicato da ricavare.
Riesprimendo il potenziale come
\begin{gather*}
    U = \frac{1}{2}\sum_{i = 1}^{n}\sum_{j = 1}^{n}V(|\vec{r}_i - \vec{r}_j|  )_{i \neq j}  
\end{gather*}
Trascurando l'effetto della differenza tra due e tre dimensioni e, al posto di utilizzare metodi analitici, cerchiamo di trovare il minimo attraverso
metodi fisici e matematici. Lo scopo è allora capire come si dispongono gli atomi nello spazio
per poter determinare il minimo dell'interazione tra di essi. \\
Possiamo innanzitutto ricondurci ad un caso molto semplice con soli due atomi $N = 2$.
L'energia potenziale è allora minima quando $r = r_0$, che si ottiene centrando il
sistema di riferimento nella particella 1 e ponendo la seconda a distanza $r_0$.\\
Se avessi ora tre particelle, avrei tre contributi energetici per il potenziale dato da
\begin{gather*}
    U = V(r_{1, 2}) + V(r_{1, 3}) + V(r_{2, 3})
\end{gather*}
\begin{wrapfigure}{r}{0.4\textwidth}
    \centering
    \caption{La configurazione a minore energia: il reticolo regolare}
    \begin{tikzpicture}
        \filldraw(0, 1) circle (0pt);
        \filldraw(0, 0) circle (2pt);
        \filldraw(1, 0) circle (2pt);
        \filldraw(-1, 0) circle (2pt);
        \filldraw(0.5, 0.75) circle (2pt);
        \filldraw(-0.5, 0.75) circle (2pt);
        \filldraw(0.5, -0.75) circle (2pt);
        \filldraw(-0.5, -0.75) circle (2pt);
    \end{tikzpicture}    
\end{wrapfigure}
Se la distanza tra tutte le particelle è esattamente $r_0$, allora il mio sistema
avrà energia potenziale minima. Da $N > 4$ le cose cambiano radicalmente
in quanto, essendo vincolati ad un piano, non c'è nessun modo per far sì
che le distanze tra tutte le particelle sia uguale. Devo disporre quindi tutte le particelle in modo
tale che stiano secondo uno schema simmetrico e regolare in quanto le particelle da una parte della configurazione
esercitano una certa forza mentre quelle che stanno dall'altra esercitano una forza
uguale e contraria per cui il contributo netto sulle singole
particelle è zero. Con questa disposizione regolare allora avremmo che la funzione presenta sicuramente un punto
stazionario anche se, secondo l'analisi, si avranno altri punti stazionari. Dobbiamo fare in modo che questa configurazione regolare si 
contribuisca con la minore energia possibile e quindi posso ripetere il pattern del triangolo
equilatero per poter ottenere la minore energia potenziale. \\
Si può dimostrare che il reticolo regolare è la configurazione con minore energia potenziale
purché il numero di particelle considerate sia molto grande rispetto ad $1$ anche se il minimo
vero è diverso poiché stiamo ignorando il contributo potenziale con
la parete. Possiamo ottenere tutte le posizioni delle particelle in un reticolo
come se fosse un sottospazio vettoriale con generatori 
\begin{gather*}
    \vec{a} = r_0 \hat{i} \\
    \vec{b} = \frac{r_0}{2}\left(-\hat{i} + \sqrt{3}\hat{j}   \right)\\
    \ \Longrightarrow \ \vec{r}_{n, m} = n\vec{a} + m\vec{b} \quad  n, m \in \mathbb{Z}      
\end{gather*}
Si può allora ottenere che per determinare il minimo per ogni particella si considerano le particelle
all'interno dell'esagono più vicino poiché abbiamo dimostrato che, in valore assoluto, hanno contributo maggiore
e dunque si ottiene che il minimo è proprio:
\begin{align}
    E_{mn} \approx \frac{1}{2}\sum_{i = 1}^{N} \sum_{j \neq i}^{6}V(r_0) = -3N\epsilon  
\end{align}
Il \textbf{cristallo} è la disposizione di minima energia possibile in quanto
la fisica quantistica fa prediligere le formazioni cristalline come
configurazione di energia minore possibile.

\subsection{Validità del modello entro un certo $\Delta E$ rispetto al minimo}
Dobbiamo ora cercare di convincerci che questo modello non 
funzioni solo per un modello con poche particelle ma anche per configurazioni abbastanza simili:
\begin{gather*}
    E = E_{min} + \Delta E
\end{gather*}
Posso applicare allora un po' di energia cinetica al nostro
modello attraverso il calore per poter ottenere un innalzamento dell'energia totale
(utilizzando l'approssimazione per la quale l'energia che applichiamo al sistema
è uniformemente condivisa tra tutti gli atomi). 
Possiamo prendere l'$i$-esimo atomo con energia pari a
\begin{gather*}
    \frac{1}{2}m|\vec{v}_i|^{2} \approx \frac{\Delta E}{N} \qquad \Delta E << E_{min} \qquad  \frac{1}{2}m|\vec{v}_i|^{2} << \epsilon    
\end{gather*}
Se sviluppassi con Taylor al secondo ordine la funzione di potenziale rispetto a $r$
si ottiene che
\begin{gather*}
    V(r_{i, j}) \approx -\epsilon + \frac{1}{2}V''(r_0)(r_j - r_0)^{2}  \qquad V''(r_0) = 72\frac{\epsilon}{r_0^{2} }
\end{gather*}
Tutte le coppie di atomi si comportano allora come oscillatori armonici e quindi si può approssimare
il modello come un network di oscillatori armonici trascurando le molle più lontane
e considerando solo quelle dei 6 atomi più vicini. Macroscopicamente si comporta come un solido in quanto per poterlo deformare
bisogna applicare forze molto grandi. Se si continua a far crescere l'energia, ad un certo
punto dobbiamo aggiungere termini successivi allo sviluppo di Taylor, il quale diventa più grande,
e il modello necessita dell'introduzione dei termini successivi. Da questo
si ha che più un sistema è caldo (e quindi più calore forniamo ad un sistema) 
e più questo si dilata. 

\subsection{Gli effetti di $\Delta E$ molto grandi}
\begin{wrapfigure}{r}{0.4\textwidth}
    \centering
    \caption{La buca di potenziale}
    \begin{tikzpicture}
        \draw[->] (-1, 0) -- (4, 0) node[at end, below] {$\frac{r}{r_0}$};
        \draw[->](0, -1) -- (0, 3) node[at end, left] {$\frac{V}{\epsilon}$};
        \draw(0.75, 3) ..controls (1, -2.5) and (1.35, -0.25) .. (2, -0.05);
        \draw(2, -0.05) -- (3.5, -0.05);
        \draw[dashed](3.5, -0.05) -- (4, -0.05);
        \draw[<->](1.3, 0) -- (1.3, -0.65);
        \draw[dashed](-0.5, 1) -- (3, 1);
    \end{tikzpicture}    
\end{wrapfigure}Aumentando sempre di più $\Delta E$ il modello avrà un asintoto verticale e
quindi se applico una energia maggiore della differenza tra il minimo e zero
allora gli atomi non saranno più legati tra di loro e quindi il sistema inizierà
a comportarsi come un gas e ogni particella inizierà a muoversi
indipendentemente dalle altre.
Come si vede dal grafico, se aumenta l'energia, i valori possibili di energia
per $r_0$ tende ad essere uno solo, ossia nella parte del grafico dove l'energia
potenziale è repulsiva. \\
Il sistema si comporterà come un solido cristallino se e solo se l'energia delle
particelle è minore di $\epsilon$ e non supera la buca di potenziale; se superassi allora questa
buca di potenziale si comporterà come un gas. Ovviamente esiste anche la situazione di mezzo
per cui il corpo si comporta come un liquido.
A livello macroscopico la differenza tra un liquido ed un solido
non è la comprimibilità ma lo scorrimento delle molecole l'una sulle altre.



\end{document}
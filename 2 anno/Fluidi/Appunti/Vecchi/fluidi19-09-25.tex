\documentclass[a4paper, oneside]{article}
\usepackage{graphicx}
\usepackage{amsthm}
\usepackage{amsmath}
\usepackage{amssymb}
\usepackage[a4paper,
            bindingoffset=0.2in,
            left=2cm,
            right=2cm,
            top=2cm,
            bottom=2cm,
            footskip=.25in]{geometry}
\usepackage[italian]{babel}
\usepackage{pgfplots}
\usepackage{tabularx}
\usepackage{tikz}
\usepackage{wrapfig}
\usepackage{color}
\definecolor{page}{rgb}{0.129,0.157,0.212}
\pagecolor{page}
\color{white}
\graphicspath{ {./images/} }
\usetikzlibrary{shapes.geometric}
\usetikzlibrary{datavisualization}
\usetikzlibrary{datavisualization.formats.functions}
\pgfplotsset{width=10cm,compat=1.9}

\title{Appunti fluidi}
\author{Tommaso Miliani}
\date{19-09-25}

\begin{document}
\newtheoremstyle{theoremEnv}
                {}          % Space above
                {}          % Space below
                {\slshape}  % Body font
                {}          % Indent amount
                {\bfseries} % Head font
                {.}         % Punctuation after head
                {\newline}         % Space after theorem head
                {}          % Theorem head spec
\theoremstyle{theoremEnv}

\newtheorem{definition}{Definizione}[section]
\newtheorem{theorem}{Teorema}[section]
\newtheorem{lemma}{Proposizione}[section]
\newtheorem{observation}{Osservazione}[section]
\newtheorem{corollary}{Corollario}[theorem]
\newtheorem{example}{Esempio}[section]

\maketitle

\section{Le proprietà microscopiche dello stato liquido}
\begin{wrapfigure}{r}{0.4\textwidth}
    \centering
    \caption{}
    \begin{tikzpicture}
        \filldraw(0, 1) circle (0pt);
        \filldraw(0, 0) circle (2pt);
        \filldraw(1, 0) circle (2pt);
        \filldraw(-1, 0) circle (2pt);
        \filldraw(0.5, 0.75) circle (2pt);
        \filldraw(-0.5, 0.75) circle (2pt);
        \filldraw(0.5, -0.75) circle (2pt);
        \filldraw(-0.5, -0.75) circle (2pt);
    \end{tikzpicture}    
\end{wrapfigure}
Considerando un esagono di atomi dal reticolo cristallino e riprendendo allora la definizione che si è data al reticolo,
ci poniamo dunque nelle ipotesi di avere ancora un solido.
Concentrandosi sull'atomo centrale, possiamo allora chiedersi come si può muovere questo assumendo che gli altri
atomi siano in una posizione fissata. Questo diventa il problema del moto di un punto
materiale in due dimensioni: posso esprimere l'energia potenziale del sistema
intorno all'atomo considerato come
\begin{gather*}
    U(x, y) = \sum_{i = 1}^{n} V(|\vec{r} - \vec{r_i}|  ) 
\end{gather*}
In questo problema si ha una funzione di due variabili che dipende parametricamente
anche dalle altre dodici variabili (che posso allora fissare come voglio dato che
sono parametri e non variabili). L'energia meccanica sarà allora data dal contributo anche
cinetico
\begin{gather*}
    E = \frac{1}{2}m(v_x^{2}  + v_y^{2} ) + U(x, y)
\end{gather*}
Quale è la regione di piano accessibile al moto del nostro punto centrale? La regione accessibile
dall'atomo centrale è quella in cui l'energia potenziale sia minore o uguale all'energia meccanica totale. L'energia
cinetica in questo modello semplificato può essere applicata solamente all'atomo centrale: questo moto è descritto allora
da questa regione di movimento specifica. Che succede allora se gli altri sei atomi siano fermi ed in un
reticolo a bassa energia? La particella non si muove molto poiché è ingabbiata dagli altri atomi
anche nel caso in cui si fornisca una energia molto grande di $10\epsilon$.

\begin{wrapfigure}{r}{0.4\textwidth}
    \centering
    \caption{Situazione fisica}
    \begin{tikzpicture}
        \draw[->](-1.5, -2) -> (-1.5, 2);
        \draw[->](-2, -1.5) -> (2, -1.5);
        \draw (0,0.5) circle (0.25);
        \draw(1, 0) ..controls(0.6, -0.4).. (0.5, -0.8);
        \draw(0.5, -0.8) ..controls(0, -0.6).. (-0.5, -0.8);
        \draw(-1, 0) ..controls(-0.8, -0.2).. (-0.5, -0.8);
        \draw(-1, 0) ..controls(-0.8, 0.4).. (-1, 0.8);
        \draw(1, 0.8) ..controls(0.8, 0.5).. (1, 0);
        \draw(-0.3, 2) ..controls(-0.3, 1.2).. (-1, 0.8);
        \draw(0.3, 2) ..controls(0.3, 1.2).. (1, 0.8);
    \end{tikzpicture}    
\end{wrapfigure}
Se si considera la situazione più realistica secondo la quale tutti gli atomi
ricevono energia cinetica: in questo caso gli atomi tenderanno tutti ad agitarsi e dunque il raggio $r_0$
di interazione tenderà a crescere e il moto della particella risulterà maggiore rispetto
alla situazione ideale anche con solo $\frac{1}{2}\epsilon$. La particella ha quindi molto più margine
di movimento in quanto gli atomi in media sono più distanti tra di loro e ciascuno si muove di più rispetto
agli altri. Tra tutte le fluttuazioni possibili se due degli atomi del reticolo
intorno si muovono in direzioni opposte, allora il margine di movimento della particella ha 
di punto in bianco una regione di movimento maggiore e si crea un canale di uscita; 
anche se la situazione più probabile è quella che la particella centrale si scambia di posizione con una del reticolo intorno.
Se si aspetta abbastanza ogni particella potrebbe percorrere tutto il reticolo (e questa è proprio la situazione
che accade nel liquido!) senza dispendio di energia anche se, in un dato istante, si volesse osservare cosa accade,
ogni particella apparirà circondata da altre 6 particelle. 

\subsection {Riassunto delle proprietà delle fasi e ipotesi di validità}
Per ogni fase noi consideriamo una energia minima in cui si ha il
reticolo perfetto di un materiale e aggiungiamo un certo $\Delta E$: 
\begin{align*}
    &\text{Solido} &\frac{\Delta E}{N} &\llless \epsilon \\
    &\text{Liquido} &\frac{\Delta E}{N} &\approx \epsilon \\
    &\text{Gas} &\frac{\Delta E}{N} &\gtrapprox 3\epsilon 
\end{align*}
Tutte queste ipotesi valgono solo quando la distribuzione degli
atomi è omogenea e la distribuzione di energia applicata al sistema sia
uniforme a tutti gli atomi. Gli stati della materia dunque sono omogenei
in quanto sono interazioni a corto raggio mentre interazioni a lungo raggio
come la forza di gravità non è omogenea. Questo modello vale
solo ed esclusivamente nel caso in cui gli atomi sono approssimabili a piccole sferette
anche se a livello macroscopico potremmo anche assumere gli atomi come 
puntiformi in quanto in fisica, gli eventi che avvengono ad una data scala sono
influenzati solo dalle scale immediatamente vicine ed ignorano (salvo varie
eccezioni) ciò che accade alle scale più lontane. Le dimensioni contano in quanto è
possibile ricavare le leggi per livelli superiori partendo dalle leggi del 
piccolo però dalle leggi macroscopiche non si può ricavare le leggi microscopiche. 

\subsection{Determinare gli effetti probabilistici nel sistema}
Considerando la forza applicata ad ogni oggetto
\begin{gather*}
    \vec{F_i} = m\vec{a_i} \qquad i = 1, \dots, N \qquad N \ggg 1  
\end{gather*}
Per esempio in un bicchiere d'acqua ci sono $8.4 \cdot 10^{24}$ atomi all'interno
del bicchiere e quindi questo problema è perfettamente determinato attraverso le leggi che
abbiamo definito. L'unico problema è che abbiamo $10^{25}$ punti materiali nel nostro
sistema e computazionalmente è molto scomodo. Supponiamo che per ogni punto materiale utilizziamo
$1$Byte di informazioni, ossia $10^{13}TB$ solo per tenere in memoria le posizioni dei punti materiali.
Le prime persone che hanno provato a calcolare le posizioni attraverso un calcolatore è
stato negli anni 50 con un set di dati di $N =32$ e $N = 64$ solo nel moto unidimensionale
per cercare di studiare la dinamica molecolare. \\
Con un supercomputer si potrebbe ora simulare sistemi dell'ordine di $10^{6}$, se il nostro sistema stesse in un cubo, allora segue
\begin{gather*}
    N \propto V = L^{3} \ \Longrightarrow \ L \propto N^{1/3}  
\end{gather*}
Allora il numero di atomi che sta sul bordo è proporzionale a $L^{2}$ e quindi 
$N_{bordo} \propto L^{2} \propto N^{2/3}$ e quindi
\begin{gather*}
    \frac{N_{bordo}}{N} \propto \frac{N^{2/3} }{N} = N^{1/3} 
\end{gather*}  
Più il sistema è piccolo e più gli effetti del brodo contano, anche
con soli $10^{6}$ atomi, avrò un errore di circa $10^{-2}$ nello studio
della dinamica molecolare. Se noi facciamo la dinamica molecolare in un piano
con punti materiali allora dovremmo rivedere le nostre relazioni utilizzando un 
software per poter simulare sistemi molecolari.  



\end{document}
\documentclass[a4paper, oneside]{article}
\usepackage{graphicx}
\usepackage{amsthm}
\usepackage{amsmath}
\usepackage{amssymb}
\usepackage[a4paper,
            bindingoffset=0.2in,
            left=2cm,
            right=2cm,
            top=2cm,
            bottom=2cm,
            footskip=.25in]{geometry}
\usepackage[italian]{babel}
\usepackage{pgfplots}
\usepackage{tabularx}
\usepackage{tikz}
\usepackage{wrapfig}
\usepackage{color}
\usepackage[d]{esvect}
\definecolor{page}{rgb}{0.129,0.157,0.212}
\pagecolor{page}
\color{white}
\graphicspath{ {./images/} }
\usetikzlibrary{shapes.geometric}
\usetikzlibrary{datavisualization}
\usetikzlibrary{datavisualization.formats.functions}
\usetikzlibrary{patterns}
\pgfplotsset{width=10cm,compat=1.9}

\title{Appunti di Fluidi}
\author{Tommaso Miliani}
\date{25-09-25}

\begin{document}
\newtheoremstyle{theoremEnv}
                {}          % Space above
                {}          % Space below
                {\slshape}  % Body font
                {}          % Indent amount
                {\bfseries} % Head font
                {.}         % Punctuation after head
                {\newline}         % Space after theorem head
                {}          % Theorem head spec
\theoremstyle{theoremEnv}

\newtheorem{definition}{Definizione}[section]
\newtheorem{theorem}{Teorema}[section]
\newtheorem{lemma}{Proposizione}[section]
\newtheorem{observation}{Osservazione}[section]
\newtheorem{corollary}{Corollario}[theorem]
\newtheorem{example}{Esempio}[section]

\maketitle

\section{Isolivelli}
\begin{wrapfigure}{r}{0.35\textwidth}
    \centering
    \caption{}
    \begin{tikzpicture}
        \draw[->](-1, 0) -- (3, 0);
        \draw[->](0, -1) -- (0, 3);
        \draw(0.5, 1.5) .. controls (1.5, 2.2) and (2.2, 1.7) .. (2.5, 1);
        \draw(0.5, 1.5) .. controls (0.5, 1)  and  (1.5, 1.5) .. (2, 0.75);
        \draw(2, 0.75) .. controls (2, 0.2) .. (2.5, 1);
        \draw[dashed](1.5, 0) -- (1.5, 1.8) node[at start, below] {$x_0$};
        \draw[dashed](0, 1.8) -- (1.5, 1.8) node[at start, left] {$y_0$};
        \draw[thick, red, ->] (1.5, 1.8) -- (1.5, 2.5) node[at end, right] {$\vv{\nabla}p$ };
    \end{tikzpicture}    
\end{wrapfigure}
Posso vedere che il rotore della pressione è il piccolo vettorino
sulla superficie della curva chiusa per cui posso scrivere
l'equazione per la mia superficie come
\begin{align}
    -\vv{\nabla}p + \rho \vv{g} = 0 
\end{align}
Ossia l'equazione fondamentale dell'idrostatica. Ogni vettorino
$\vv{\nabla}p$ è ortogonale alla superficie del fluido. Si ottengono allora
gli \textbf{isolivelli} per $p$ sulla superficie del fluido: ossia la pressione
del fluido è costante su tutta la sua superficie.  
Dato che la densità non si conosce a priori, devo derivarla dalla pressione:
posso prendere un gas e, comprimendolo, se riduce il suo volume allora deve necessariamente
aumentare la densità.

\section{Legge di Stevino}
\begin{wrapfigure}{r}{0.4\textwidth}
    \centering
    \caption{}
    \begin{tikzpicture}
        \draw[->](0, 0) -- (4, 0) node[at end, below] {$x$};
        \draw[->](0, 1) -- (0, -3) node[at end, left] {$z$};
        \draw[->](-0.2, -0) -- (-0.2, -1) node[at end, left] {$\vv{g}$ };
        \draw(-0.1, -1.75) -- (0.1, -1.75) node[at start, left] {$z$};
        \draw(1, -1) ..  controls (1.5, -1) and (2.2, -1.5) .. (2.5, -2);
        \draw(1, - 1) .. controls (0.5, -1.5) and (1, -2) .. (1.5, -2.5);
        \draw(1.5, -2.5) .. controls (2, -2.7) .. (2.5, -2);
        \filldraw(2, -1.75) circle (1pt);
    \end{tikzpicture}    
\end{wrapfigure}
Il caso più semplice della risoluzione dell'equazione è con $\vv{g}$ costante e nel caso
in cui il fluido sia \textbf{incomprimibile} ($\rho = \rho_0$). Tutte le derivate parziali
della pressione rispetto ai vari assi sono allora zero
tranne che per l'asse $z$ (perché l'unica forza è proprio
lungo la direzione di $z$, ossia la forza gravitazionale), per cui l'equazione fondamentale 
dell'idrostatica ci darà
\begin{gather*}
    \frac{\partial p}{\partial z} = \rho g 
\end{gather*}
A questo punto la funzione varia solo per $z$. La derivata parziale diventa una derivata
totale. 
\begin{gather*}
    \int_{p_0}^{p}\frac{dp}{dz}dz = \int_{z_0}^{z}\rho g \ dz 
\end{gather*}
Si ha quindi la \textbf{legge di Stevino} per cui la differenza di pressione
è legata unicamente alla profondità all'interno del fluido. 
\begin{align}
    p - p_0 = \rho g (z - z_0)
\end{align}
In un fluido incomprimibile la pressione di un fluido dipende interamente
dalla sua altezza. La pressione spinge in tutte le direzioni del contenitore
e più ci si addentra dentro al fluido e più diventa grande la pressione: questo
è dovuto proprio alla legge di Stevino.

\subsection{Stima per un caso specifico}
Quanta pressione esercita un metro d'acqua:
data la densità $ \rho_a \approx 1 \ g /cm^{3}$ e preso $g \approx 10$,
in modo tale che la pressione dell'acqua è $10^{4} N /m^{2}$. In 10
metri si avrebbe invece $10^{5} N / m^{2}$. L'esperimento di
Pascal consiste nel prendere un piccolo tubo di una sezione molto piccola
tale per cui per $3$ metri di tubo si ha un litro d'acqua (per esempio) 
per esercire una grossa pressione nel contenitore sottostante.

\subsection{Contenitore con liquido}
\begin{wrapfigure}{r}{0.4\textwidth}
    \centering
    \caption{}
    \begin{tikzpicture}
        \draw(0, 1) -- (0, 0) -- (3, 0) -- (3, 1);
        \draw(0, 0.8) -- (1.2, 0.8);
        \draw(1.8, 0.8) -- (3, 0.8);
        \draw(1.2, 0.5) -- (1.2, 1.5) -- (1.8, 1.5) -- (1.8, 0.5);
    \end{tikzpicture}    
\end{wrapfigure}
Bernoulli in questo esperimento ha osservato che la pressione che agisce sul 
fluido all'interfaccia con l'aria deve bilanciarsi con la pressione dell'aria.
Per la legge di Stevino posso osservare che
\begin{gather*}
    p = p_{\perp} g h
\end{gather*}
Questa pressione dovrà coincidere con la pressione atmosferica in quanto i due fluidi
devono essere in equilibrio sull'interfaccia. L'interfaccia tra liquido
ed aria non si sposta in quanto le pressioni che intercorrono
tra le due facce sono in equilibrio. Bernoulli con questo esperimento
è riuscito a calcolare la pressione dell'aria di $10^{5}Pa$ al livello 
del mare (e coincide con la sua comprimibilità). 
Posso definire allora 
\begin{gather*}
    10^{5} \ N/ m = 1 \ Bar \approx 1 \ Atm  (1, 03 \cdot  10^{5} \ Pa ) \approx 1 \ kg / cm^{2} 
\end{gather*}
Se si ipotizzasse che dentro al contenitore ci sia acqua, allora
la legge di Stevino diventerebbe
\begin{gather*}
    p - p_0 = \rho_{acqua} g h
\end{gather*}

\begin{wrapfigure}{r}{0.4\textwidth}
    \centering
    \caption{}
    \begin{tikzpicture}
        \draw(0, 0) -- (0, 3) -- (1, 3);
        \draw(1, 3) .. controls (1.2, 1) .. (2, 0);
        \draw(0,0) -- (4, 0);
        \draw(1, 2.5) -- (4, 2.5);
        \draw[->](1.2, 2) -- (0.5, 2) node[midway, above] {$\vv{F}_d$ };
        \draw[->](1.2, 1.8) -- (0.2, 1.8) node[midway, below] {$\vv{F_a}$ };
    \end{tikzpicture}    
\end{wrapfigure}
Quando si è alla profondità dell'ordine di un chilometro
la comprimibilità dell'acqua è da tenere in considerazione
e la densità dell'acqua aumenta a causa della pressione del liquido sovrastante.
Le dighe infatti non sono dei muri dritti
Il pezzo sotto della diga deve resistere sia alla pressione
del cemento sovrastante che alla pressione dell'acqua 
che cresce con la profondità.
La pressione totale a cui deve resistere la diga con una certa quota
di acqua rispetto alla base della diga è
l'integrale della pressione
\begin{gather*}
    R(z) = \int_{0}^{z} \rho g z' \ dz' = \frac{\rho}{2}g z^{2} 
\end{gather*}

\subsection{Fluido immerso in un fluido}
\begin{wrapfigure}{r}{0.4\textwidth}
    \centering
    \caption{}
    \begin{tikzpicture}
        \draw(0, 3) -- (0, 0) -- (3, 0) -- (3, 3);
        \draw[->](2, 1.4) -- (2, 0.5) node[at end, right] {$M_F \vv{g}$ };
        \draw[->](2, 1.6) -- (2, 2.5) node[near end, right] {$-M_F \vv{g}$ };
        \draw(1, 1.5) .. controls (1.5, 2.5) and (1.7, 2) .. (1.75, 1.5);
        \draw(1, 1.5) .. controls (0.8, 1) and (1.25, 0.3) .. (1.75, 1.5);
        \draw(0, 2.5) -- (3, 2.5);
        \filldraw(1, 2) circle (0pt) node[anchor = north] {$F$};
        \filldraw(1.5, 2) circle (0pt) node[anchor = north] {$S$};
    \end{tikzpicture}    
\end{wrapfigure}
Nel caso in cui si immerga un fluido all'interno di un fluido,
se il fluido immerso è lo stesso fluido contenitore $F$, allora la risultante è
diretta lungo la stessa direzione della gravità ma con verso
opposto per avere l'equilibrio
\begin{gather*}
    \rho_F g V = R_{\Sigma}
\end{gather*}
La risultante nel caso in cui il fluido $S$ contenuto nel
fluido $F$ sia diverso
sarà allora data dall'integrale rispetto al volume
\begin{gather*}
    R_{\Sigma} = - \int_{V} \rho \ d V \vv{g} = -M_F \vv{g}  
\end{gather*}

\end{document}
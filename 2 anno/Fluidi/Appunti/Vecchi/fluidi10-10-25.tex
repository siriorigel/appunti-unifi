\documentclass[a4paper, oneside]{article}
\usepackage{graphicx}
\usepackage{amsthm}
\usepackage{amsmath}
\usepackage{amssymb}
\usepackage[a4paper,
            bindingoffset=0.2in,
            left=2cm,
            right=2cm,
            top=2cm,
            bottom=2cm,
            footskip=.25in]{geometry}
\usepackage[italian]{babel}
\usepackage{pgfplots}
\usepackage{tabularx}
\usepackage{tikz}
\usepackage{wrapfig}
\usepackage{color}
\usepackage[d]{esvect}
\definecolor{page}{rgb}{0.129,0.157,0.212}
\pagecolor{page}
\color{white}
\graphicspath{ {./images/} }
\usetikzlibrary{shapes.geometric}
\usetikzlibrary{datavisualization}
\usetikzlibrary{datavisualization.formats.functions}
\usetikzlibrary{patterns}
\pgfplotsset{width=10cm,compat=1.9}

\title{appunti fluidi}
\author{Tommaso Miliani}
\date{10-10-25}

\begin{document}
\newtheoremstyle{theoremEnv}
                {}          % Space above
                {}          % Space below
                {\slshape}  % Body font
                {}          % Indent amount
                {\bfseries} % Head font
                {.}         % Punctuation after head
                {\newline}         % Space after theorem head
                {}          % Theorem head spec
\theoremstyle{theoremEnv}

\newtheorem{definition}{Definizione}[section]
\newtheorem{theorem}{Teorema}[section]
\newtheorem{lemma}{Proposizione}[section]
\newtheorem{observation}{Osservazione}[section]
\newtheorem{corollary}{Corollario}[theorem]
\newtheorem{example}{Esempio}[section]

\maketitle

\section{Espansione libera adiabatica di un gas quasi statica}
\begin{wrapfigure}{r}{0.4\textwidth}
    \centering
    \caption{Il pistone}
    \begin{tikzpicture}
        \draw[very thick](0, 0) rectangle (2, 2);
        \filldraw[cyan, opacity = 0.3] (0, 0) rectangle (2, 2);
        \draw[very thick] (2, 2) -- (3.5, 2 );
        \draw[very thick](2, 0) -- (3.5, 0);
        \draw[->, thick] (3, 1) -- (2, 1) node[midway, above] {$\vv{F}$ };
    \end{tikzpicture}    
\end{wrapfigure}
Per far si che l'espansione libera adiabatica sia quasi statica, io
non posso rimuovere la barriera immediatamente ma devo applicare una forza
sulla parete che sia in grado di equilibrarla esattamente: se la diminuissi
istante per istante, allora il gas si espanderebbe lentamente. In questo
modo riesco ad ottenere una espansione libera adiabatica di un gas in 
maniera quasi statica. Si potrebbe ottenere la \textbf{compressione} adiabatica
del sistema in maniera quasi statica se invece facessi crescere istante per istante
la forza necessaria. Il contenitore in figura prende il nome di \textbf{pistone adiabatico}. 
Si ottengono allora le seguenti situazioni:
\begin{itemize}
    \item $\Delta V > 0 \ \Longrightarrow \ \Delta T < 0$;
    \item $\Delta V < 0 \ \Longrightarrow \ \Delta T > 0$; 
    \item $\Delta V \neq 0 \ \Longrightarrow \ \Delta T \neq 0$. 
\end{itemize}
Non esistono solamente trasformazioni adiabatiche. 

\section{Trasformazioni non adiabatiche}
\subsection{Il termostato}
\begin{wrapfigure}{r}{0.4\textwidth}
    \centering
    \caption{}
    \begin{tikzpicture}[scale=0.85]
        \draw(-2, -1) rectangle (0, 3) node[midway] {$T$};
        \draw(0, 0) rectangle (2, 2);
        \filldraw[cyan, opacity = 0.3] (0, 0) rectangle (2, 2);
        \draw[very thick] (0, 2) -- (3.5, 2);
        \draw[very thick](0, 0) -- (3.5, 0);
        \draw[very thick](2, 0) -- (2, 2);
        \draw[->, thick] (3, 1) -- (2, 1) node[midway, above] {$\vv{F}$ };
    \end{tikzpicture}    
\end{wrapfigure}
Per studiare trasformazioni non adiabatiche si introduce il concetto di
\textbf{termostato}: ossia un sistema termodinamico tale che la sua
temperatura possa sempre rimanere costante qualsiasi cosa gli venga fatta. Ovviamente
il termostato perfetto non esiste anche se è possibile definire termostati rispetto
ad altri sistemi se e solo se la sua temperatura del termostato varia molto meno rispetto al sistema
a cui è paragonato. Si può creare un termostato con un sistema di massa molto 
grande rispetto all'altro sistema con il quale lo stiamo facendo interagire.
Se $M$ è la massa del termostato e $m$ è la massa del sistema si deve avere
\begin{align}
    M >> m
\end{align}
La parete con cui il pistone è a contatto con il termostato è una parete diatermica:
per costruzione allora $\Delta T = 0$, ossia la compressione o espansione
quasi statica avviene sempre a temperatura costante. Se io attaccassi un sistema
ad un termostato anche se la trasformazione non avviene in maniera quasi statica,
la temperatura iniziale e quella finale coincideranno poiché,
data la natura del termostato, la temperatura rimane la stessa (anche se
durante la trasformazione può cambiare). Queste tipologie di trasformazioni prendono
il nome di \textbf{isoterme}. 

Se volessi fare delle trasformazioni a volume costante con aumento di temperatura 
in maniera quasi statica, potrei introdurre una successione di termostati con temperatura
sempre crescente in modo tale da raggiungere l'equilibrio istante per istante. Se non facessi tutti  i passaggi
intermedi, allora non sarebbe una trasformazione quasi statica e dunque non potrei dire che
la pressione cresce in maniera proporzionale rispetto alla temperatura.

\subsection{Il lavoro della forze di pressione di un fluido}
\begin{wrapfigure}{r}{0.4\textwidth}
    \centering
    \caption{Il lavoro delle forze di pressione}
    \begin{tikzpicture}
        \draw[very thick](0, 0) rectangle (2, 2);
        \draw[very thick](2, 2) -- (4, 2);
        \draw[very thick](2, 0) -- (4, 0);
        \draw[<->](2, -0.2) -- (2.4, -0.2) node[midway, below] {$dl$};
        \filldraw[cyan, opacity = 0.3] (0, 0) rectangle(2, 2);
        \draw[dashed](2.4, 0) -- (2.4, 2);
    \end{tikzpicture}    
\end{wrapfigure}
L'unica forza che può imprimere un fluido che è compresso o che si espande è
la forza che deriva dalla pressione del gas. Posso quantificare il lavoro 
(in maniera meccanica) della pressione su di un gas considerando un pistone
con all'interno un fluido. Dato che è un pistone, allora il gas è in grado di essere
compresso oppure è in grado di espandersi a seconda della forza che vi 
è sulla parete mobile del pistone. Se si considerasse il caso in cui il gas si espande di un certo 
$dl$, allora, dato $\vv{dl} = dl \hat{i}$:
\begin{gather*}
    V \to V + dV
\end{gather*}  
Le forze di pressione agiscono sulla parete e dunque compiono un lavoro
infinitesimo che è dato da:
\begin{gather*}
    \delta L = \vv{F} \cdot  \vv{dr}  
\end{gather*}
Possiamo esprimere allora la forza dovuta dalla pressione come
\begin{gather*}
    \vv{F} = pA \hat{i}  
\end{gather*}
Dove $A$ è la superficie di contatto della parete mobile del pistone 
lungo il versore $\hat{i}$. Si ottiene allora l'espressione del lavoro infinitesimo
come :
\begin{gather*}
    \delta L = pA \hat{i} \cdot  dl \hat{i} \ \Longrightarrow \ pAdl 
\end{gather*} 
Ricordando allora che $Adl$ è esattamente il volume infinitesimo di cui si è espanso il corpo:
\begin{align}
    \delta L = pdV
\end{align}
Il lavoro è positivo se il fluido si espande ed è negativo se 
invece è compresso. Questa espressione del lavoro vale per qualunque
trasformazione infinitesima di un fluido quando varia di un certo  $dV$
ben definito. 

\begin{wrapfigure}{r}{0.4\textwidth}
    \centering
    \caption{L'espansione libera di un fluido}
    \begin{tikzpicture}
        \draw(0, 0) ellipse (2 and 1) node[midway] {$V$};
        \draw[dashed](0, 0) ellipse (2.2 and 1.1);
        \draw[->](2.5, -1.2) -- (1.8, -0.8) node[at start, right] {$V + dV$};
    \end{tikzpicture}    
\end{wrapfigure}
Dato un fluido racchiuso in una forma arbitraria, posso applicargli una
trasformazione in maniera tale che il fluido possa variare il suo volume.
Il nuovo volume occupato dall'oggetto è allora $V + dV$ Se io
considerassi una parte del fluido, potrei approssimare la curva arbitraria
come se ci fossero tanti pistoncini che si stanno espandendo dalla curva
normale a quella tratteggiata. La sezione del pistone non importa; posso
allora pensare che ognuno di questi pistoncini si espandano con $dl_1, dl_2\dots$
tutti diversi tra di loro. Posso allora considerare un solo pistoncino per il quale
il lavoro compiuto dalla pressione dal gas per l'espansione infinitesima:
\begin{gather*}
    \delta L_i = p  \ dA_i \ dl_i
\end{gather*}
Allora il lavoro totale dei pistoncini diventerà la sommatoria
dei lavori infinitesimi
\begin{gather*}
    \delta L = \sum p \ dA_i \ dl_i 
\end{gather*}
Dato che $p$ è costante per tutta la superficie del fluido, e che
$dV_i = dA_i \ dl_i$, si ha il lavoro totale come:
\begin{gather*}
    \delta L = p\sum dV_i
\end{gather*}

\begin{wrapfigure}{r}{0.4\textwidth}
    \centering
    \caption{Il lavoro lungo una curva generica}
    \begin{tikzpicture}
        \draw[->](0, 0) -- (3, 0) node[at end, below] {$V$};
        \draw[->](0, 0) -- (0, 3) node[at end, left] {$p$};
        \draw(0.5, 1) .. controls (1, 1.4) and (1.5, 1.4) .. (2.5, 1);
        \filldraw(0.5, 1) circle (1pt) node[anchor = south] {I};
        \filldraw(2.5, 1) circle (1pt) node[anchor = south] {F};
        \draw[dashed](0.5, 0) -- (0.5, 1);
        \draw[dashed](2.5, 0) -- (2.5, 1);
        \filldraw[cyan, opacity = 0.3] (2.5, 1) -- (2.5, 0) -- (0.5, 0) -- (0.5, 1) .. controls (1, 1.4) and (1.5, 1.4) .. (2.5, 1);
    \end{tikzpicture}    
\end{wrapfigure}
Se il volume del fluido variasse di una quantità finita di volume, allora
non saprei dire il lavoro della forza di pressione, in quanto la trasformazione non
sarebbe quasi statica. Se fosse quasi statica, per ogni stato di equilibrio,
dovrei avere fissata la pressione del fluido. Dunque il lavoro della pressione in una 
trasformazione quasi statica è dato da:
\begin{align}
    L = \int_{V_i}^{V_f}p \ d V
\end{align}  
dove $p$ è esattamente una funzione del volume  $p(V)$ (che non so). Non
posso però considerare $\delta L$ come $dL$, ossia non posso dire che
il lavoro infinitesimo sia il differenziale esatto in quanto
le funzioni $p(V)$ possono essere infinite e dunque il lavoro non è
una funzione di stato. Durante le espansioni libere, il lavoro delle forze di
pressione è nullo in quanto la parete è rimossa istantaneamente e sull'ambiente 
esterno non si è spostato nulla. 


\subsection{Trasformazioni quasi statiche a pressione costante}
Dato che la pressione è costante, allora è valida la relazione 
del lavoro (anche perché è quasi statica) e dunque il lavoro 
è proprio l'integrale
\begin{align}
    L = p\int_{V_i}^{V_f}
\end{align}

\subsection{Trasformazioni quasi statiche isoterme}
Dato che è isoterma, allora $T$ è costante: non so per un sistema
qualunque quanto può fare il lavoro delle forze di pressione, anche
se il sistema è quasi statico. Tuttavia, nel caso in cui la funzione di stato
allora posso ricavarmi una relazione per poter esprimere l'equazione della curva
isoterma: allora in quel caso posso determinare il lavoro delle forze di pressione.
Dato che, per un gas perfetto, $pV = nRT$, allora
\begin{align}
    L = \int_{V_i}^{V_f} \frac{nRT}{V} \ dV \ \Longrightarrow \ L = nRT \cdot  \ln\frac{V_f}{V_i}
\end{align}

\subsection{Lavoro del sistema e dell'ambiente}
\begin{wrapfigure}{r}{0.4\textwidth}
    \centering
    \caption{Il lavoro del gas su di un corpo di massa $m$}
    \begin{tikzpicture}
        \draw[very thick](0, 0) --  (2, 0) -- (2, 4);
        \draw[very thick](0, 0) -- (0, 4);
        \draw[very thick](0, 2) -- (2, 2);
        \draw(0.5, 2) rectangle (1.5, 3) node[midway] {$m$};
        \filldraw[cyan, opacity = 0.3](0, 0) rectangle (2, 2);
        \draw[|-|](-0.2, 2) -- (-0.2, 3.2) node[midway, left] {$h$};
        \draw[dashed](0, 3.2) -- (2, 3.2);
        \draw[dashed](0.5, 3.2) rectangle (1.5, 4.2) node[midway] {$m$};
    \end{tikzpicture}    
\end{wrapfigure}
Posso dare alcune definizioni tecniche:
\begin{itemize}
    \item $\delta L > 0$: il sistema compie del lavoro sull'ambiente esterno
    \item $\delta L < 0$: l'ambiente esterno compie del lavoro sul sistema.
\end{itemize}
Il lavoro è il modo meccanico per far sì che ambiente e 
sistema si scambino energia: questo vuol dire che è possibile anche
immagazzinare energia sottoforma di energia potenziale.  Nell'esempio semplice
di un peso su di un pistone, posso dire che l'energia potenziale dell'ambiente
è variata secondo
\begin{gather*}
    \Delta U(A) = mgh
\end{gather*}
Posso esprimere il lavoro che ha compiuto la pressione
come
\begin{gather*}
    L(S) = p\Delta V \ \Longrightarrow \ L = \frac{mg}{B} hB \ \Longrightarrow \ L(S) = mgh
\end{gather*}
Dove $B$ è la superficie del pistone, $F$ è la forza che scaturisce la pressione
del gas e $h$ la variazione di quota dovuta alla pressione del gas.


\section{Lavoro del sistema durante una trasformazione adiabatica}
\begin{wrapfigure}{r}{0.4\textwidth}
    \centering
    \caption{}
    \begin{tikzpicture}
        \draw[very thick](0, 0) rectangle (5, 2);
        \draw[very thick](1, 0) -- (1, 2);
        \draw[very thick](2, 0) -- (2, 2);
        \filldraw[cyan, opacity = 0.3](0, 0) rectangle (1, 2);
        \draw(2, 1) -- (5.5, 1);
    \end{tikzpicture}    
\end{wrapfigure}
Per arrivare a definire il lavoro del sistema in una trasformazione adiabatica
si può partire da considerare una situazione semplice in cui il sistema è
in espansione. Dato che non esiste un unico modo per poter passare da uno
stato iniziale I ad uno stato finale F, allora posso utilizzare una
scatola, con pareti adiabatiche, che mi permetta di ottenere 3 trasformazioni
per riempire tutta la scatola. Considerando un gas perfetto e molto rarefatto
all'interno della scatola, questa presenta una prima parete, così come la seconda, presentano delle aperture che
possono essere controllate dall'esterno senza interferire sullo stato del sistema.
La seconda parete, è controllata dall'esterno tramite un'asta, in modo tale che possa muoversi
dalla sua posizione iniziale fino a tutta la lunghezza della scatola. 


\begin{wrapfigure}{r}{0.4\textwidth}
    \centering
    \caption{La schematizzazione delle trasformazioni}
    \begin{tikzpicture}
        \draw[->](0, 0) -- (4, 0) node[at end, below] {$V$};
        \draw[->](0, 0) -- (0, 4) node[at end, left] {$p$};
        \draw[dashed](1, 0) -- (1, 3.5) node[at start, below] {$V_0$};
        \draw[dashed](2, 0) -- (2, 3) node[at start, below] {$V_1$};
        \draw[dashed](3, 0) -- (3, 1.5) node[at start, below] {$V_2$};
        \draw[dashed](4, 0) -- (4, 1.3) node[at start, below] {$V_3$};
        \filldraw(1, 3.5) circle (1pt);
        \filldraw(2, 3) circle (1pt);
        \filldraw(3, 1.5) circle (1pt);
        \filldraw(4, 1.3) circle (1pt);
        \draw[thick, ->](1, 3.5) .. controls (1.3, 3.2) and (1.7, 3) .. (2, 3) node[at end, right] {$T_0 = T_1$};
        \draw[dashed](1, 2.5) .. controls (2, 1.7) and (3, 1.45) .. (4, 1.3) node[at end, right] {$T_2 = T_3$};
        \draw[thick, ->](2, 3)  -- (3, 1.5);
        \draw[thick, ->](3, 1.5) .. controls (3.4, 1.4) and (3.7, 1.35) .. (4, 1.3);
    \end{tikzpicture}    
\end{wrapfigure}
Posso riassumere tutte le trasformazioni nel grafico volume-pressione,
di seguito sono elencate in ordine le trasformazioni: 
\begin{itemize}
    \item \textbf{I trasformazione}: la prima trasformazione è una espansione libera
    adiabatica (la cui trasformazione sta sull'isoterma che comprende le due temperature
    inziali e finiali):
    \begin{gather*}
        V_1 > V_0 \quad T_0 = T_1 \ \Longrightarrow \ L_{0, 1} = 0
    \end{gather*}
    \item \textbf{II trasformazione}: La seconda trasformazione consiste nell'espansione del 
    volume del gas in maniera quasi statica, controllando il movimento della seconda parete. 
    Questa trasformazione, essendo quasi statica, è una curva sul piano che non si sa disegnare,
    anche se si sa che questa curva deve essere tale che la temperatura dello stato in cui si atterra,
    deve essere una temperatura diversa da quella che si aveva all'inizio.
    \begin{gather*}
        V_2 > V_1 \quad T_2 < T_1 \ \Longrightarrow \ L_{1, 2} = \int_{V_1}^{V_2}p \ dV
    \end{gather*}
    \item \textbf{III trasformazione}: A questo punto, apro la seconda parete, così inzia unn'altra
    espansione libera del gas, per cui
    \begin{gather*}
        V_3 > V_2 \quad T_3 = T_2 \ \Longrightarrow \ L_{2, 3} = 0
    \end{gather*}
\end{itemize}
Dato che questa è solamente una trasformazione di tutte le infinite trasformazioni possibili, 
è possibile ricavare sempre altre tre trasformazioni per cui si va dallo stesso stato iniziale allo stesso stato finale.
Qualunque sia l'insieme delle trasformazioni applicate al sistema,
il lavoro complessivo sarà sempre dato dal solo contributo
dell'espansione del volume del gas in maniera quasi statica dovuto al
controllo del movimento della seconda parete. 


\end{document}
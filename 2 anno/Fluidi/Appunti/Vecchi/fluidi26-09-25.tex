\documentclass[a4paper, oneside]{article}
\usepackage{graphicx}
\usepackage{amsthm}
\usepackage{amsmath}
\usepackage{amssymb}
\usepackage[a4paper,
            bindingoffset=0.2in,
            left=2cm,
            right=2cm,
            top=2cm,
            bottom=2cm,
            footskip=.25in]{geometry}
\usepackage[italian]{babel}
\usepackage{pgfplots}
\usepackage{tabularx}
\usepackage{tikz}
\usepackage{wrapfig}
\usepackage{color}
\usepackage[d]{esvect}
\definecolor{page}{rgb}{0.129,0.157,0.212}
\pagecolor{page}
\color{white}
\graphicspath{ {./images/} }
\usetikzlibrary{shapes.geometric}
\usetikzlibrary{datavisualization}
\usetikzlibrary{datavisualization.formats.functions}
\usetikzlibrary{patterns}
\pgfplotsset{width=10cm,compat=1.9}

\title{Appunti di Fluidi}
\author{Tommaso Miliani}
\date{26-09-25}

\begin{document}
\newtheoremstyle{theoremEnv}
                {}          % Space above
                {}          % Space below
                {\slshape}  % Body font
                {}          % Indent amount
                {\bfseries} % Head font
                {.}         % Punctuation after head
                {\newline}         % Space after theorem head
                {}          % Theorem head spec
\theoremstyle{theoremEnv}

\newtheorem{definition}{Definizione}[section]
\newtheorem{theorem}{Teorema}[section]
\newtheorem{lemma}{Proposizione}[section]
\newtheorem{observation}{Osservazione}[section]
\newtheorem{corollary}{Corollario}[theorem]
\newtheorem{example}{Esempio}[section]

\maketitle

\section{Esempi di esercizi dei fluidi}
In linea generale per la risoluzione dei problemi in
statica dei fluidi è trovare una relazione tra le pressione
e la densità. Quando la pressione è funzione della densità si parla di \textbf{Legge barotropica}:
\begin{align}
     p = p(\rho) 
\end{align}
Questo permette di ottenere una relazione tipo
\begin{gather*}
    \vv{\nabla} p = \rho \vv{g}  
\end{gather*}
Si può anche assumere una certa legge di proporzionalità
sulla potenza della densità:
\begin{align}
    p \propto C \rho^{\gamma} \ \Longrightarrow \  \frac{p}{p_0} = \left(\frac{\rho}{\rho_0}\right)^{\gamma} 
\end{align}
Questa legge prende il nome di \textbf{Legge politropica}. Ci sono dei motivi fisici
che ci portano a determinare questa relazione: in quanto il valore $\gamma$ è dato
dal rapporto dei calori specifici (che si tratterà più avanti). Possiamo considerare
il seguente esempio per determinare perché questa relazione è valida:
\begin{gather*}
    \frac{\partial p}{\partial x} = 0 \qquad \frac{\partial p}{\partial y} = 0 \qquad \frac{\partial p}{\partial z} = -\rho g   
\end{gather*}
Allora posso dire anche in questo caso che la derivata parziale rispetto a $z$ è esattamente
la derivata della pressione sulla costante $z$ poiché rispetto agli altri assi
è nulla la pressione. Allora posso considerare il caso semplice in cui non si abbia la potenza
delle $\rho$ con la legge di Stevino:
\begin{gather*}
    p = p_0 \left(\frac{\rho}{\rho_0}\right) \ \Longrightarrow \ \rho = \rho_0 \frac{p}{p_0}
\end{gather*}
Nell'espressione della derivata:
\begin{gather*}
    \frac{dp}{dz} = -\rho_0 \frac{p}{p_0}g \ \Longrightarrow \ \int_{p_0}^{p}\frac{dp'}{p'} = - \int_{0}^{z} \frac{\rho_0g}{p_0}dz'
\end{gather*}
Gli apici mi servono per differenziare i differenziali rispetto agli estremi di
integrazione, quindi ho separato le variabili. Posso quindi risolvere l'equazione
e ottenere:
\begin{gather*}
    \ln \frac{p}{p_0} = -\frac{\rho_0g}{p_0}z \ \Longrightarrow \ \boxed{p(z) = p_0 \exp\left(-\frac{\rho_0g}{p_0}z\right)}
\end{gather*}
Ho ottenuto una relazione esponenziale per la pressione rispetto alla pressione,
se chiamassi allora
\begin{gather*}
    H = \frac{p_0}{\rho_0g}
\end{gather*}
Allora posso dire che la legge politropica si esprime come
\begin{align}
     p = p_0 \exp\left(-\frac{z}{H}\right)
\end{align}
In questo modo ho trovato una relazione esponenziale della variazione
della pressione a seconda della quota $z$: il fatto che sia una espressione
esponenziale (e non lineare) è coerente a livello fisico poiché se si avesse avuto una espressione
lineare, questa ad un certo punto sarebbe diventata negativa, il che non avrebbe avuto alcun senso.
Per la Terra posso ora stimare il fattore $H$ e dire quando la pressione diminuisce
per un fattore $\frac{1}{e}$, ossia dopo circa $8.5 km$, la pressione si dimezza invece
dopo $\approx 6 \ km$ rispetto al livello del mare (assumendo che $g$ non cambi). 

\subsection{Anticipazione legge dei gas perfetti}
La legge dei Gas perfetti ci dice che
\begin{gather*}
    pV = nRT
\end{gather*}
Posso determinare con il numero di Avogadro il numero di molecole
all'interno del gas:
\begin{gather*}
    pV = \frac{NRT}{N_A} \qquad \text{con} \qquad N = nN_A
\end{gather*}
Posso anche esprimere la relazione tramite la costante di Boltzmann ($k_B = \frac{R}{N_A}$)
\begin{gather*}
    pV = Nk_BT
\end{gather*}
Per un volume infinitesimo $dV$ ci sarà un numero infinitesimo di particelle
contenute all'interno di questo volumetto di gas. Se io lo moltiplicassi
per la massa media delle particelle $\overline{m}$:
\begin{gather*}
    pdV = \overline{m}dN \frac{k_BT}{\overline{m} } 
\end{gather*} 
Si ha allora che il termine $\overline{m}dN$ è esattamente la massa infinitesima,
allora posso riesprimere la relazione dei gas perfetti per un certo volume infinitesimo
di gas come
\begin{gather*}
    p = \frac{dm}{dV} \frac{k_BT}{\overline{m} } \ \Longrightarrow \  p = \frac{k_BT}{\overline{m} } \rho
\end{gather*} 
Allora, assumendo che la temperatura sia la stessa da qui a $8.5km$, allora posso esprimere il rapporto
tra la pressione e la densità come e ottenere allora una approssimazione (non molto corretta) 
per determinare la pressione dell'atmosfera a $8.5km$ secondo questo modello invece che
secondo la relazione (più precisa) che si era trovata prima.
\begin{gather*}
    \frac{p}{\rho} = \frac{k_BT}{\overline{m} }
\end{gather*}
Essendo questa una grossa approssimazione, in futuro sarà rivista e trattata in modo
più completo nella parte di termodinamica e nella parte di fisica statistica. 

\section{Le forze apparenti nei fluidi}
\begin{wrapfigure}{r}{0.5\textwidth}
    \centering
    \caption{Forze apparenti nei fluidi}
    \begin{tikzpicture}[domain=0:2]
        \draw[->](-1, 0) -- (3, 0) node[at end, below] {$x$};
        \draw[->](0, -1) -- (0, 3) node[at end, left] {$z$};
        \draw[color=cyan]  plot (\x, {(\x)^2 / 2 });
        \draw(1.3, 0.9) rectangle (1.5, 1.1);
        \draw[->](1.4, 1) -- (2.2, 1) node[at end, right] {$\vv{F_{cf}}$ };
        \draw[->](1.4, 1) -- (0.8, 1.6) node[at end, above] {$\vv{\nabla}p dV$ };
        \draw[->](1.4, 1) -- (1.4, 0.25) node[at end, right] {$\vv{g} \rho dV$};
    \end{tikzpicture}    
\end{wrapfigure}
Abbandonando le ipotesi di barotropica e politropica, supponiamo di avere un fluido
incomprimibile e che sia in rotazione rispetto all'asse $z$, posso esprimere il
vettore velocità angolare e l'accelerazione di gravità come:
\begin{align*}
    \vv{\omega} &= \omega \hat{z} \\
    \vv{g} &= -g \hat{z}    
\end{align*}
La quota $z_0$ è l'altezza del liquido nell'asse $z_0$. Preso un piccolo elemento
di fluido piccolo a piacere, questo sentirà l'effetto di due forze: un effetto da
parte della forza di gravità e anche la forza centrifuga (se osservato in un SdR inerziale).
Ponendosi in coordinate cilindriche, posso definire $\vv{r}$ la distanza dall'asse di rotazione
e quindi esprimere la forza centrifuga come:
\begin{gather*}
    \vv{F}_{cf} = \rho_0 dV \omega^{2}r \hat{r}   
\end{gather*} 
Ossia diretta lungo la direzione radiale. La somma delle due forze può essere
bilanciata solamente da una forza $\vv{\nabla} pdV$. Non ci resta
altro che scrivere le relazioni:
\begin{gather*}
    \vv{\nabla}p  =\rho \vv{g}  
\end{gather*}
Posso esprimere il gradiente della forza da coordinate cartesiane
a coordinate cilindriche come
\begin{gather*}
    \vv{\nabla}F = \frac{\partial F}{\partial x}\hat{x} + \frac{\partial F}{\partial y}\hat{y} + \frac{\partial F}{\partial z} \hat{z}     \\
    \ \Longrightarrow \ \vv{\nabla}F = \frac{\partial F}{\partial r}\hat{r} + \frac{1}{r}\frac{\partial F}{\partial \theta}\hat{\theta} + \frac{\partial f}{\partial z}\hat{z}       
\end{gather*}
Posso allora esprimere la derivata parziale rispetto alla pressione come
\begin{gather*}
    \left\{\begin{array}{l}
        \frac{\partial p}{\partial r} = \rho_0 \omega^{2}r \\
        \frac{\partial p}{\partial \theta} = 0 \\
        \frac{\partial p}{\partial z} = -\rho_0 g    
    \end{array}\right.
\end{gather*}
La pressione allora dipenderà solamente dalla distanza rispetto all'asse di rotazione
e dalla quota $z$ mentre non dipende dall'angolo di rotazione. Posso assumere
allora che la pressione possa essere divisa in una funzione solo di $r$ e una solo di $z$:
\begin{gather*}
    p(r, z)  - p_0 = f_1(r) - f_2(z)  = \rho_0 \frac{\omega^{2} }{p}r^{2} - \rho_0gz + c_1 + c_2 
\end{gather*}
Posso allora esprimere le derivate parziali rispetto alle due funzioni, che derivano proprio da
dalla derivata parziale della pressione rispetto a $r$ e $z$.
\begin{gather*}
    \left\{\begin{array}{l}
        \frac{\partial f_1}{\partial r} = \rho_0 \omega^{2} r  \ \Longrightarrow \ f_1 = \rho_0 \frac{\omega^{2} }{2}r^{2} + c_1 \\
        \frac{\partial f_2}{\partial z} = -\rho_0 g \ \ \Longrightarrow \ f_2 = -\rho_0 gz + c_2   
    \end{array}\right.
\end{gather*}
Per comodità posso escludere le costanti (dipende dal fatto che viene considerata la pressione $p_0$
nell'espressione della pressione in funzione del raggio e della quota): infatti quando $r = 0$
posso considerare le costanti come zero. infatti potrei esprimere la funzione
della pressione come
\begin{align}
    \boxed{p(r, z) = \rho_0 \frac{\omega^{2} }{2}r^{2} - \rho_0 gz + p_0 }
\end{align}
Se volessi trovare il valore di $p_0$, allora troverei la condizione
\begin{gather*}
    \rho_0 \frac{\omega^{2} }{2}r^{2} = \rho_0 gz 
\end{gather*}
E dunque il valore di $z$ per cui si ha $p_0$ sono tutti i punti con
\begin{gather*}
    z = \frac{\omega^{2} }{2}\frac{r^{2} }{g} 
\end{gather*}
Tutte le isosuperfici sono date da una parabola nel grafico (ossia tutte le curve
lungo le quali la pressione è la stessa). Quando un liquido ruota la superficie di un 
liquido tende a formare un paraboloide proprio per questo motivo analitico. Il pelo del liquido
allora (ossia la superficie di contatto tra il fluido ed un altro fluido) tende a formare
un vortice quando ruota. 

\section{Autogravitazione dei fluidi nel caso di una sfera di fluido}
\begin{wrapfigure}{r}{0.4\textwidth}
    \centering
    \caption{La sfera di fluido}
    \begin{tikzpicture}
        \draw(0, 0) circle (2);
        \draw(0, 0) -- (0, 2) node[midway, left] {$r$};
        \filldraw[red](-0.1, 2) rectangle (0.1, 2.2) node[anchor = west] {$dV$};
        \draw[->, red, thick](0, 2) -- (0, 1) node[near end, right] {$\vv{F_g}$};
        \draw(0, 0) -- (-2, 2) node[midway, below] {$R$};
        \draw(0, 0) circle (2.85);
        \draw[->, red, thick](0, 2.2) -- (0, 3.2) node[at end, right] {$\vv{\nabla}p$ };
    \end{tikzpicture}    
\end{wrapfigure}
Fino ad ora si era considerato equilibri tra forza di pressione e forza esterna (come la gravità)
generate da corpi esterni. E' l'equilibrio tra le pressioni e la forza di gravità che
mantiene intatti i corpi celesti. La forza esercitata su unità di massa è data da:
\begin{gather*}
    \frac{\vv{F} }{m} = - \frac{GM(r)}{r^{2} } \rho dV \hat{r}  
\end{gather*}
Sto cercando di determinare la forza che agisce su di una parte del fluido $dV$ da parte della
gravità del fluido interno ad una distanza $r$ dal centro. La parte di fluido di raggio $r$ agisce su di una certa porzione
del fluido $dV$ ad una certa distanza $r$ in \textbf{autogravità}, posso esprimere la massa 
a distanza $r$ dall'origine della sfera
\begin{gather*}
    M(r) = \int_{0}^{r} 4\pi r^{2}\rho (r) \ dr
\end{gather*}
Posso cambiare sistema di riferimento ed utilizzare le coordinate sferiche ed esprimere il gradiente
della pressione in funzione del versore $\hat{r}$ come
\begin{gather*}
    \vv{\nabla}p = \rho \vv{g}(r) = -\frac{GM(r)}{r^{2} }\hat{r}   
\end{gather*} 
Il gradiente di una forza in questo caso sarà dato dalla seguente relazione 
(devo trasformarlo dal sistema di riferimento cartesiano a quello sferico) utilizzando
l'angolo $\theta$ rispetto alla verticale e l'angolo $\phi$ rispetto ad uno dei due assi perpendicolari
a quello verticale ed il raggio $r$ come la distanza dal centro della sfera si ha:
\begin{gather*}
    \vv{\nabla}F = \frac{\partial F}{\partial r}\hat{r} + \frac{1}{r}\frac{\partial F}{\partial \theta} \hat{\theta} + \frac{1}{r\sin\theta}\frac{\partial F}{\partial \phi} \hat{\phi}       
\end{gather*}
Posso allora derivare parzialmente la pressione (tuttavia considero solamente rispetto al raggio
in quanto, essendo a simmetria sferica, le derivate rispetto a $\phi$ e $\theta$ sono zero):
\begin{gather*}
    \frac{\partial p}{\partial r} = -G \rho \frac{M(r)}{r^{2} } = -\frac{4}{3}\pi G\rho_0^{2}r 
\end{gather*}
Posso integrare da entrambe le parti con separazione di variabili e dunque  ottenere l'integrale:
\begin{gather*}
    \int_{p_0}^{p} dp = -\int_{R_{\oplus}}^{r} \frac{4\pi}{3}G\rho_0^{2}r' \ dr \ \Longrightarrow \ -\frac{4\pi}{3}\rho_0^{2}\frac{G}{2} (r^{2} - R^{2}_{\oplus}  ) 
\end{gather*}
Dove $R_{\oplus}$ è il raggio della sfera. Posso allora esprimere la pressione
con la seguente funzione:
\begin{gather*}
    p = p_0 + \frac{4}{3}\pi\rho_0^{2}\frac{G}{2}(R_{\oplus}^{2} -r^{2}  ) = p_0 + \frac{4}{3}\pi\frac{G}{2}R_{\oplus}^{2}\rho_0^{2} \left(1 - \frac{r^{2} }{R_{\oplus}}\right) 
\end{gather*}
La massa del pianeta è allora esprimibile come $M_{\oplus} = \frac{4}{3}\pi\rho_0R_{\oplus}^{3}$.  A questo punto posso esprimere la pressione
come
\begin{gather*}
    p = p_0 + \frac{GM_{\oplus}}{R_{\oplus}}\frac{\rho_0}{2}R_{\oplus}\left(1 - \frac{r^{2} }{R_{\oplus}}\right)
\end{gather*}
Dove $\frac{GM_{\oplus}}{R_{\oplus}} = g_{\oplus}$. Per la Terra ottengo la pressione al centro di
$\approx 10^{11} \ Pa$, ossia la pressione stimata della Terra con la sua massa, 
densità e raggio come se fosse un fluido di densità costante in tutti i punti.  

\subsection{Il pianeta fatto di fluido in rotazione}
\begin{wrapfigure}{r}{0.4\textwidth}
    \centering
    \caption{Lo schema delle forze nel fluido in rotazione autogravitante}
    \begin{tikzpicture}
        \draw[->](0, 0) -- (0, 4) node[at end, left] {$z$};
        \draw(0, 0) -- (2, 3) node[midway, left] {$r$};
        \draw(0, 1) arc (90:55:1)node[midway, above] {$\theta$};
        \draw[dashed](0, 3) -- (2, 3) node [midway, above] {$h = r\sin\theta$};
        \draw[->, very thick](2, 3) -- (1, 1.5) node[at end, right] {$\vv{F_{\theta}}$ };
        \draw[->, very thick](2, 3) -- (3, 3) node[at end, right] {$\vv{F_c}$};
        \draw[->, very thick](2, 3) -- (2.5, 3.75) node[at end, right] {$\hat{r}$ };
        \draw[->, very thick](2, 3) -- (2.5, 2.75) node[at end, right] {$\hat{\theta}$ };
    \end{tikzpicture}    
\end{wrapfigure}
La distanza dall'asse di rotazione prende il nome di $h$ mentre $r$ è la distanza
dal centro della massa di fluido in rotazione. Posso esprimere 
l'angolo rispetto alla verticale del vettore posizione con
l'angolo $\theta$. E quindi posso esprimere i vettori come
\begin{gather*}
    \vv{F_g} = -\frac{\rho_0 G M(r)}{r^{2} }\hat{r} = \rho_0^{2} G\frac{4\pi}{3}r  \\
    \vv{F_c} = \rho_0 \omega^{2}h(\sin\theta \hat{r} + \cos\theta \hat{\theta}  )  = \rho_0 \omega r \sin\theta (\sin\theta \hat{r} + \cos \theta\hat{\theta}  )
\end{gather*}
Sapendo che
\begin{gather*}
    M(r) = \frac{4\pi}{3}r^{3} \rho_0 
\end{gather*}
La pressione ora deve bilanciare il contributo sia della forza di gravità che della forza centrifuga,
allora posso esprimere la derivata della pressione sia su $\theta$ che su $r$:
\begin{gather*}
    \frac{\partial p}{\partial r} = -\rho_0^{2}\frac{4\pi}{3}rG+ \rho_0 \omega^{2}r \sin^{2}\theta \\
    \frac{1}{r}\frac{\partial p}{\partial \theta} = -\rho_0^{2} \omega^{2}r\sin\theta\cos\theta       
\end{gather*}
Posso allora integrare come se $\theta$ fosse un parametro nella prima
e quindi ottengo l'espressione della pressione in funzione del raggio
e dell'angolo $\theta$ (dell'angolo $\phi$ non mi interessa in quanto sto
considerando una sfera). Posso esprimere $f(\theta)$ come l'integrale della derivata
parziale rispetto a $\theta$:
\begin{gather*}
    p(r, \theta) - p_0 = -\rho_0^{2}\frac{4\pi}{3}\frac{r^{2} }{2}G+ \rho_0 \omega^{2}\frac{r^{2} }{2} \sin^{2}\theta + f(\theta)
\end{gather*}
Allora posso derivare rispetto a $\theta$ questa espressione
\begin{gather*}
    \frac{\partial p}{\partial \theta} = \rho \omega^{2}r^{2} \sin\theta\cos\theta + f'(\theta)  
\end{gather*}
Adesso se ponessimo il termine $\frac{1}{r}$ davanti all'espressione, dovrebbe essere uguale con
l'espressione della derivata parziale trovata prima, uguagliandole:
\begin{gather*}
    f'(\theta) = 0 \ \Longrightarrow \ f(\theta) = const
\end{gather*}
La soluzione è allora considerare una generica funzione $f(\theta)$ come costante
ed esprimerla come $C$. Dato che ho scelto $p_0$ come la pressione
al centro,allora la costante sarà tolta e otterrò 
\begin{align}
    p(r, \theta) - p_0 = -\rho_0 \frac{r^2 }{2}\left(G\rho_0\frac{4\pi}{3} - \omega^{2}\sin^{2}\theta  \right)
\end{align}
Posso esprimere come si è fatto con il raggio della sfera e la massa della sfera in modo da ottenere $g_{\oplus}$,
ossia l'accelerazione di gravità e quindi trovare l'espressione della pressione
finale come
\begin{gather*}
    p - p_0 = \frac{r^{2}\rho_0 g_{\oplus} }{2R_{\oplus}} \left(1 - \frac{\omega^{2}R_{\oplus} }{g_{\oplus}}\sin^{2}\theta \right)
\end{gather*}


\end{document}
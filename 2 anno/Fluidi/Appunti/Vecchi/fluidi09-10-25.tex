\documentclass[a4paper, oneside]{article}
\usepackage{graphicx}
\usepackage{amsthm}
\usepackage{amsmath}
\usepackage{amssymb}
\usepackage[a4paper,
            bindingoffset=0.2in,
            left=2cm,
            right=2cm,
            top=2cm,
            bottom=2cm,
            footskip=.25in]{geometry}
\usepackage[italian]{babel}
\usepackage{pgfplots}
\usepackage{tabularx}
\usepackage{tikz}
\usepackage{wrapfig}
\usepackage{color}
\usepackage[d]{esvect}
\definecolor{page}{rgb}{0.129,0.157,0.212}
\pagecolor{page}
\color{white}
\graphicspath{ {./images/} }
\usetikzlibrary{shapes.geometric}
\usetikzlibrary{datavisualization}
\usetikzlibrary{datavisualization.formats.functions}
\usetikzlibrary{patterns}
\pgfplotsset{width=10cm,compat=1.9}

\title{Appunti di Fluidi}
\author{Tommaso Miliani}
\date{09-10-25}

\begin{document}
\newtheoremstyle{theoremEnv}
                {}          % Space above
                {}          % Space below
                {\slshape}  % Body font
                {}          % Indent amount
                {\bfseries} % Head font
                {.}         % Punctuation after head
                {\newline}         % Space after theorem head
                {}          % Theorem head spec
\theoremstyle{theoremEnv}

\newtheorem{definition}{Definizione}[section]
\newtheorem{theorem}{Teorema}[section]
\newtheorem{lemma}{Proposizione}[section]
\newtheorem{observation}{Osservazione}[section]
\newtheorem{corollary}{Corollario}[theorem]
\newtheorem{example}{Esempio}[section]

\maketitle

\section{Validità di Van Der Walls}
\begin{wrapfigure}{r}{0.4\textwidth}
    \centering
    \caption{}
    \begin{tikzpicture}
        \draw(0, 0) -- (4, 0);
        \draw(0, 0) -- (0, 4);
        \draw(0.5, 3.5) .. controls (0.6, 0.6) .. (0.75, 0.5);
        \draw(0.75, 0.5) .. controls (1.5, 1.8) and (2, 1.8) .. (3.5, 0.9);
        \filldraw[cyan, opacity = 0.3] (0.58, 1.25) .. controls (0.6, 0.6) .. (0.75, 0.5) .. controls (1, 1) .. (1.33, 1.25);
        \filldraw[cyan, opacity = 0.3] (1.33, 1.25) .. controls (1.5, 1.5) and (2, 1.68) .. (2.8, 1.25) -- (1.33, 1.25);
        \filldraw(0.4, 1) circle (0pt) node[anchor = center] {$A_1$};
        \filldraw(2, 1.8) circle (0pt) node[anchor = center] {$A_2$};
        \draw(0, 1.25) -- (4, 1.25); 
    \end{tikzpicture}    
\end{wrapfigure}
Possiamo riscrivere l'equazione di Van Der Walls come
\begin{gather*}
    V^{3}p - V^{2}(RT + bp) + aV -ab = 0  
\end{gather*}
Dobbiamo ora modificare questa equazione ancora per poter prevedere
la curva isoterma sotto la temperatura critica. Un modo per farlo è quello
di osservare gli esperimenti e scegliere il livello di pressione in modo tale
che per un certo valore di pressione le due aree $A_1$ e $A_2$ siano uguali.
Supponendo di metterci nel caso di temperatura inferiore alla temperatura critica
in modo tale che l'isoterma possa intersecare almeno una retta orizzontale
tre volte. MaxWell dice che si deve scegliere il livello di pressione in modo
tale che le due aree devono essere uguali. Allora i risultati che si ottengono
dalle relazioni con la temperatura critica sono congruenti con gli  esperimenti
che sono circa equivalenti ad uno (sempre minori di uno).

Si definiscono delle grandezze adimensionali 
\begin{gather*}
    \tilde{T} = \frac{T}{T_c} \qquad \tilde{V} = \frac{V}{V_c} \qquad \tilde{p} = \frac{p}{p_c}
\end{gather*}
Possiamo allora far scomparire i parametri $a, b$ dall'equazione secondo
la seguente:
\begin{gather*}
    \left(\tilde{p} + \frac{3}{\tilde{V}^{2} }\right)\left(3\tilde{V} - 1\right) = 8\tilde{T}
\end{gather*}
In questo modo con queste particolari unità si ottengono le stesse
isoterme per tutti i gas. Facendo infatti scomparire i termini $a, b$, questo vuol
dire che tutti i fluidi si comportano allo stesso modo ma semplicemente 
con scale diverse. Questa equazione prende il nome di \textbf{equazione di universalità}. 
Questa cosa non è per niente ovvia in quanto ad una prima analisi si potrebbe dire che 
i fluidi si comportino tutti in modo diverso; con questa equazione si è riassunto
il comportamento di tutti i fluidi all'interno di una unica equazione. Gli esperimenti
non confermano questa predizione approssimata: infatti se scalassi le curve sperimentali
con le temperature critiche non è vero che diventano le stesse isoterme; tuttavia
più mi avvicino al punto critico e più questa è precisa: al punto critico è esatta. 
Successivamente nel novecento si è arrivati ad una equazione più precisa che
è ci ha permesso di ottenere descrizioni molto più accurate per le 
isoterme. Ken o Wilson nel 1982 è riuscito a costruire una teoria che ha permesso
di ottenere le descrizioni per le isoterme dei gas in modo accurato. 

\section{Diagramma di fase}
\begin{wrapfigure}{r}{0.45\textwidth}
    \centering
    \caption{Diagramma di fase}
    \begin{tikzpicture}[scale = 0.9]
        \draw[->](0, 0) -- (6, 0) node[at end, below] {$T$};
        \draw[->](0, 0) -- (0, 4.25) node[at end, left] {$p$};
        \draw(0, 0) .. controls (1, 0.75) and (1.5, 1.1) .. (2, 2);
        \draw(2, 2) .. controls (2.5, 2) and (3.25, 2.2) .. (4, 3);
        \draw(2, 2) .. controls (2.3, 2.6) and (2.4, 3) .. (2.5, 3.5);
        \filldraw(1, 3) node[anchor = center] {solido};
        \filldraw(3, 1) node[anchor = north] {gas};
        \filldraw(3, 3) node[anchor = center] {liquido};
        \filldraw(2, 2) circle (1pt) node[anchor = east] {$P_3$};
        \filldraw (4, 3) circle (1pt) node[anchor = south east] {$P_c$};
        \draw[dashed](4, 0) -- (4, 4) node[at start, below] {$T_c$};
        \draw[dashed](2, 0) -- (2, 2) node[at start, below] {$T_3$};
        \draw[dashed](2.5, 3.5)  -- (2.6, 4); 
        \node[align = center] (c) at (5.5, 3.5) {fluido \\ supercritico};
    \end{tikzpicture}    
\end{wrapfigure}
Conviene utilizzare grafici temperatura-pressione. Per una
qualsiasi sostanza pura, ci sono tre regioni separate da delle curve:
queste curve delineano i passaggi di stato. Come si osserva dal grafico,
è possibile ottenere dei passaggi di stato in modo tale da
passare direttamente da solido a gassoso o viceversa senza passare
dal liquido. Al centro del grafico c'è il \textbf{punto triplo}
dove le tre fasi sono in equilibrio; sulle curve che separano i vari stati
si ha l'equilibrio degli stati adiacenti alle curve stesse. 
C'è anche un altro punto notevole sul grafico, ossia il \textbf{punto critico}
dove la curva del cambiamento di stato "finisce". Infatti prima del
punto critico io posso passare da uno stato all'altro semplicemente
modificando i parametri di temperatura e di pressione. Se io volessi prendere
i medesimi punti di inizio e arrivo ma volessi passare alla destra
del punto critico, io posso avere passaggio di stato senza nessun tipo di singolarità
diventando gradualmente da gas a liquido. La presenza del punto critico mi permette di
non ottenere passaggi bruschi (ossia le proprietà del fluido "saltano" in modo non continuo)
da una fase all'altra: quella regione oltre il punto critico
prende il nome di \textbf{regione sovracritica}, nella quale,
liquido e gas sono indistinguibili e hanno lo stesso comportamento. I fluidi
che sono in quella regione prendono il nome
di \textbf{fluido supercritico}. 
Si è detto che per la descrizione completa di un gas c'è bisogno di tre variabili: 
quelle che sono curve, nel grafico a tre dimensioni diventano superfici.

\section*{Trasformazioni termodinamiche}
\section{Come avvengono le trasformazioni}
\begin{wrapfigure}{r}{0.4\textwidth}
    \centering
    \caption{}
    \begin{tikzpicture}
        \draw[->](0, 0) -- (4, 0) node[at end, below] {$V$};
        \draw[->](0, 0) -- (0, 4) node[at end, left] {$p$};
        \filldraw(1, 2.5) circle (1pt) node[anchor = south] {$I$};
        \filldraw(3, 1) circle (1pt) node[anchor = north] {F};
        \draw[->](1, 2.5) .. controls(1.8, 2.4) and (2.4, 2) .. (3, 1);
    \end{tikzpicture}    
\end{wrapfigure}
Non sempre si è in grado di determinare cosa accade tra un passaggio
di stato all'altro anche se si è in grado di determinare le fasi 
di arrivo e partenza. Dato un sistema termodinamico con uno stato
iniziale e finale, nel mezzo non so cosa accade al sistema poiché
le variabili termodinamiche sono ben definite solamente allo
stato di equilibrio. Se si immagina di fare questo cambiamento di stato
in maniera estremamente lento, posso allora definire in maniera discreta tanti
microstati di equilibrio uno dopo l'altro come una successione di punti sul grafico
che determinano gli stati. Questo tipo di Trasformazioni termodinamiche prendono
il nome di \textbf{trasformazioni statiche}: ossia delle trasformazioni 
di durata infinita. Dato che non dispongo di tempo infinito, posso definire un certo
margine e dunque queste trasformazioni prendono il nome di \textbf{trasformazione quasi statiche}.
Quando l'istante iniziale e finale coincidono, questo tipo di Trasformazioni quasi
statiche prendono il nome di \textbf{ciclo} o \textbf{trasformazione ciclica}.

\begin{wrapfigure}{r}{0.4\textwidth}
    \centering
    \caption{Trasformazioni isocore, isobare e isoterme}
    \begin{tikzpicture}
        \draw[->](0, 0) -- (4, 0)node[at end, below] {$V$};
        \draw[->](0, 0) -- (0, 4)node[at end, left] {$p$};
        \filldraw (1, 1.5) circle (1pt) node[anchor = east] {I};
        \filldraw (3, 1.5) circle (1pt) node[anchor = south] {$F_1$};
        \filldraw (3, 0.5) circle (1pt) node[anchor = west] {$F_2$};
        \draw[dashed] (1, 1.5) .. controls (1.5, 1) and (2, 0.8) .. (3, 0.5);
        \draw[thick](1, 1.5) -- (3, 1.5) -- (3, 0.5);        
    \end{tikzpicture}    
\end{wrapfigure}
Ci sono anche dei casi particolari delle trasformazioni quasi statiche:
\begin{itemize}
    \item \textbf{Isobare}: sono delle trasformazioni quasi statiche
    a pressione costante;
    \item \textbf{Isocore}: trasformazioni quasi statiche a volume costante;
    \item \textbf{Isoterme}: trasformazioni quasi statiche a temperatura costante;
    in questo caso io non conosco l'equazione della curva isoterma ma posso
    determinare che se avviene alla stessa temperatura so che sta su di una curva.
\end{itemize}
Se la trasformazione non è quasi statica io non potrei rappresentarla
come è nel grafico poiché altrimenti non conoscerei i valori della pressione e
del volume durante la trasformazione. Un caso particolare di 
trasformazione termodinamica è la \textbf{trasformazione adiabatica}: questa particolare
trasformazione è una sottoclasse delle trasformazioni isoterme.

\subsection{Espansione libera adiabatica di un gas}
\begin{wrapfigure}{r}{0.4\textwidth}
    \centering
    \caption{Espansione libera adiabatica}
    \begin{tikzpicture}
        \draw[thick](0, 0) rectangle (2, 1);
        \draw[thick](1, 0) -- (1, 1);
        \filldraw[color = cyan, opacity = 0.3] (0, 0) rectangle (1, 1);
        \draw[thick](3, 0) rectangle (5, 1);
        \filldraw[color = cyan, opacity = 0.3] (3, 0) rectangle (5, 1);
    \end{tikzpicture}
\end{wrapfigure}
La situazione è quella di un recipiente a pareti adiabatiche: fuori dal sistema
c'è l'ambiente. L'espansione libera adiabatica di un gas è l'espansione di un
gas in una camera adiabatica divisa in due parti: se il gas sta in una
parte della parete (all'equilibrio) e si rimuove  (rapidamente) la parete che
separa le due parti della camera adiabatica, allora il gas
non ha più un muro che lo separa dall'altra parte della camera e 
dunque tenderà ad espandersi.
\begin{gather*}
    V_F > V_I 
\end{gather*}
Questo tipo di trasformazione non è quasi statica poiché
avviene molto rapidamente. Quale è la relazione tra la temperatura iniziale
e finale? In questo esperimento si osserva che la temperatura finale è \emph{poco} minore
di quella finale. Per qualunque gas ideale la temperatura è leggermente minore. Posso 
allora diminuire il gas all'interno della camera adiabatica: in questo modo
la temperatura è minore di quella iniziale ma maggiore rispetto all'esperimento di
prima. Allora si ottiene che per un gas molto rarefatto:
\begin{gather*}
    \lim_{p \to 0} \text{ gas } \ \Longrightarrow \ T_F \to T_I 
\end{gather*}


\end{document}
\documentclass[a4paper, oneside]{article}
\usepackage{graphicx}
\usepackage{amsthm}
\usepackage{amsmath}
\usepackage{amssymb}
\usepackage[a4paper,
            bindingoffset=0.2in,
            left=2cm,
            right=2cm,
            top=2cm,
            bottom=2cm,
            footskip=.25in]{geometry}
\usepackage[italian]{babel}
\usepackage{pgfplots}
\usepackage{tabularx}
\usepackage{tikz}
\usepackage{wrapfig}
\usepackage{color}
\usepackage[d]{esvect}
\definecolor{page}{rgb}{0.129,0.157,0.212}
\pagecolor{page}
\color{white}
\graphicspath{ {./images/} }
\usetikzlibrary{shapes.geometric}
\usetikzlibrary{datavisualization}
\usetikzlibrary{datavisualization.formats.functions}
\usetikzlibrary{patterns}
\pgfplotsset{width=10cm,compat=1.9}

\title{APpunti di Fluidi}
\author{Tommaso Miliani}
\date{30-09-25}

\begin{document}
\newtheoremstyle{theoremEnv}
                {}          % Space above
                {}          % Space below
                {\slshape}  % Body font
                {}          % Indent amount
                {\bfseries} % Head font
                {.}         % Punctuation after head
                {\newline}         % Space after theorem head
                {}          % Theorem head spec
\theoremstyle{theoremEnv}

\newtheorem{definition}{Definizione}[section]
\newtheorem{theorem}{Teorema}[section]
\newtheorem{lemma}{Proposizione}[section]
\newtheorem{observation}{Osservazione}[section]
\newtheorem{corollary}{Corollario}[theorem]
\newtheorem{example}{Esempio}[section]

\maketitle

\section*{Termodinamica}
\section{La termodinamica}
\subsection{Storia della termodinamica}
La termodinamica è esplosa a metà dell'ottocento con le
prime macchine termiche, ossia delle macchine che sfruttavano energia
termica per poter compiere lavoro. Lo sviluppo storico
della termodinamica parte con le macchine termiche e si sviluppa
in modo non propriamente logico ma anzi è stato frutto di un processo
molto tortuoso che ha eventualmente portato alla nascita della termodinamica così come la 
studiamo oggi. 

\subsection{Lo studio della termodinamica}
Nella parte di termodinamica ci si occupa dello studio
dei sistemi a scala macroscopica e non a scala microscopica. Le coordinate
dello spazio che si definisce nella termodinamica prendono il nome
di \textbf{coordinate termodinamiche}:  grandezze come la pressione, densità di massa sono
ottime coordinate termodinamiche in quanto riescono a descrivere le caratteristiche
del nostro sistema termodinamico. Nel caso di sistemi omogenei la quantità di massa non
varia mai e quindi utilizzare il volume o la densità è semplicemente una preferenza anche
se in generale in termodinamica si parla più in termini di volume che
in termini di densità (come invece si fa nei fluidi). Quando si considera un
sistema termodinamico $S$ possiamo fare delle considerazioni:
\begin{itemize}
    \item Si è sempre in grado di stabilire lo stato del sistema attraverso
    le coordinate termodinamiche;
    \item Si determinano i confini del sistema rispetto all'ambiente esterno con il quale il sistema interagisce (anche chiamato \textbf{ambiente});
    \item Il sistema più l'ambiente è chiamato \textbf{Universo}; non l'universo
    della cosmologia ma semplicemente un insieme: $S + A = U$;
\end{itemize}
La densità e la pressione non sono in grado di descrivere accuratamente tutti i sistemi
termodinamici ma è necessario introdurre un'altra coordinata termodinamica, ossia
la \textbf{Temperatura} $T$. La definizione operativa dell temperatura è molto complessa
ma ci permette di dire che cosa vuol dire che due sistemi hanno la stessa temperatura. 



\subsection{L'equilibrio termodinamico}
A differenza dell'equilibrio meccanico, l'equilibrio termodinamico
consiste nel definire quando un sistema è in equilibrio con un altro
sistema: per farlo bisogna determinare se le loro coordinate termodinamiche sono costanti
nel tempo. Quando in un sistema macroscopico si hanno delle coordinate termodinamiche
che dipendono dallo spazio, ma sono costanti nel tempo, allora siamo in presenza dell'effetto di
un campo di forze esterne che agisce sul sistema. Quando accade questo siamo in presenza di un
\textbf{equilibrio termodinamico locale}: l'equilibrio termodinamico è 
valido solamente per un dato punto e non per tutti i punti del sistema
nel corso del tempo proprio perché le coordinate termodinamiche variano spazialmente. 
Questo equilibrio non è considerato in questo corso se non per pochi casi. \\
La cosa non ovvia è che quando il sistema è all'equilibrio termodinamico le
variabili termodinamiche sono dipendenti l'una dall'altra: esiste allora una relazione
funzionale tale per cui una è funzione delle altre due. Questo risultato
prende il nome di \textbf{equazione di stato}
\begin{gather*}
    f(p, V, T) = 0
\end{gather*} 
Ovviamente l'espressione della funzione $f$ non si conosce quasi mai anche se si
è certi che esista una espressione esplicita per l'equazione di stato. Dato un sistema, se  questo è
lasciato senza interagire con il sistema ambiente allora non cambierà il suo equilibrio;
se interagisse invece con il sistema ambiente allora il suo equilibrio varierà.

\subsection{Estensività}
Se si considerasse un sistema $S$ e lo si suddividesse in due sistemi $S_1$ e $S_2$ tale
per cui si suppone che la quantità di sostanza  di $S_1$ sia esattamente uguale alla 
sostanza di $S_2$ e che la somma 
delle loro sostanze dia la sostanza totale di $S$; dato che all'inizio
il sistema $S$ era all'equilibrio, allora facciamo in modo che anche i due
sottosistemi siano all'equilibrio: la grandezza $X$ quantità di sostanza è estensiva
quando $X_1 = X_2 = \frac{X}{2}$: è anche la definizione di additività e di
\textbf{estensività} poiché in termodinamica si dà per scontato che questi due
termini descrivano la medesima cosa. La massa, il volume sono grandezze estensive
poiché dipendono dalla quantità di sostanza; grandezze come la temperatura sono invece estensive.
Le grandezze che non sono né estensive né intensive non sono valide coordinate termodinamiche anche se 
devo possedere almeno una grandezza estensiva per poter descrivere in maniera
completa un sistema termodinamico; se così non fosse allora non
sarei in grado di determinare se il sistema considerato sia molto grande oppure
molto piccolo.


\subsection{Pareti ed equilibrio termico}
Una \textbf{parete} è un oggetto che sia capace di separare tra loro
due sistemi termodinamici. Una parete è \textbf{adiabatica} se ai due lati della
parete possono coesistere due sistemi termodinamici all'equilibrio
qualunque essi siano: questo vuol dire che la parete è \textbf{isolante}
e non permette quindi l'interazione tra questi due sistemi. Una parete
che non è adiabatica è definita come parete \textbf{diatermica}: i due sistemi che 
si trovano separati da questa parete si accorgono l'uno della presenza dell'altro
e iniziano ad interagire: se si aspettasse molto tempo varierebbero il loro stato
di equilibrio raggiungendone uno nuovo.  Due sistemi possono non essere
all'equilibrio termico fra loro ma all'equilibrio termodinamico se sono separati da una parete
adiabatica; tuttavia se sono separati da una parete diatermica i due sistemi
saranno sia in equilibrio termico fra loro che in equilibrio termodinamico.
Dalla fisica sperimentale si ha che se un sistema $A$ è all'equilibrio
termico con $C$ e anche il sistema $B$ è anch'esso in equilibrio
termico con $C$ allora anche $A$ e $B$ sono in equilibrio termico. Potremmo
riassumere queste considerazioni mediante la seguente:
\begin{align}
    (A \sim C) \wedge (B \sim C) \ \Longrightarrow \ (A \sim B)
\end{align}
Che prende il nome di \textbf{Principio zero della termodinamica}. La relazione
di equilibrio termico è una relazione transitiva che si può dimostrare ponendo una parete adiabatica
tra $A$ e $B$ e ponendo un sistema "a ponte" tra i due sistemi. Se poi si  sostituisse 
la parete adiabatica con una parete diatermica si osserverebbe che i due sistemi $A$ e $B$
sono già all'equilibrio e quindi la relazione scritta in precedenza è valida.  

\subsection{La temperatura}
La temperatura è la grandezza che mi consente di dire se
due sistemi sono in equilibrio termico: la temperatura per due sistemi
all'equilibrio termico è infatti la stessa. Anche senza definirla so che è una grandezza intensiva in quanto abbiamo visto che non 
dipende dalla quantità di sostanza per i sistemi considerati nel 
principio zero. Perché  si utilizza la temperatura per dire se un oggetto è 
più caldo di un altro? L'esperienza sensoriale ci dice che se si pone
due oggetti uno più caldo dell'altro vicini non saremmo più in grado di determinare se uno è più 
caldo dell'altro. La sensazione dunque è una cosa che per definire il concetto di oggetto
più caldo o freddo non è sufficiente per determinare le relazioni tra due corpi; 
devo quindi utilizzare una quantità estensiva per determinare se un oggetto è più caldo 
di un altro. \\
Per definire la temperatura posso considerare un sistema $S$ in modo tale
da essere nella situazione come quella illustrata nel principio zero
della termodinamica e, considerate due variabili $x, y$ che non siano la temperatura, il fatto di essere all'equilibrio termico non è scontato poiché quando
due sistemi sono all'equilibrio termico tra di loro allora ci dovrebbe essere una relazione
tra le variabili dei vari sistemi. Posso allora esprimere la l'equazione di stato:
\begin{gather*}
    f_{A, C}(x_A, y_A, x_C, y_C) = 0 \\
    f_{B, C}(x_B, y_B, x_C, y_C) = 0
\end{gather*} 
Possiamo allora fare una assunzione sensata per cui
\begin{gather*}
    y_C = g_{A, C}(x_A, y_A, x_C) \\
    y_C = g_{B, C}(x_B, y_B, x_C)
\end{gather*}
Queste due sono uguali tra di loro: applicando allora il principio zero
si ha
\begin{gather*}
    f_{A, B}(x_A, y_A, x_B, y_B) = 0
\end{gather*}
Allora lo stato di equilibrio dei vari sistemi è caratterizzato
dall'aver assegnato le variabili a quel sistema: i valori per cui quest'ultima
è soddisfatta devono essere gli stessi che per le altre. Dato che
devono descrivere lo stesso stato, la variabile $x_C$ può essere eliminata:
la dipendenza funzionale delle funzioni $g$ non dipendono da $x_C$. 
Allora posso dire che esistono due funzioni:
\begin{gather*}
    h_A(x_A, y_A) = h_B(x_B, y_B)
\end{gather*} 
Posso allora permutare i sistemi $A, B, C$ come voglio: posso ottenere
dunque un'altra
relazione con $A$ e $C$:
\begin{gather*}
    h_A(x_A, y_A) = h_C(x_C, y_C) \ \Longrightarrow \ h_A(x_A, y_A) = h_B(x_B, y_B)= h_C(x_C, y_C)
\end{gather*}
Se due sistemi sono all'equilibrio termico allora la funzione delle variabili 
$x$ e $y$ per ciascun sistema sono in relazione con gli altri sistemi
e dunque questa funzione di variabili è esattamente la \textbf{temperatura empirica}.
Se noi fossimo in grado di determinare queste funzioni allora potremmo definire la temperatura
come funzione delle altre variabili termodinamiche (il che non è possibile anche se si sa che esiste).
La grandezza temperatura è dunque quella grandezza funzione della pressione e
del volume  che mi permette di determinare se due sistemi sono in equilibrio
termico oppure no. Se la temperatura è fissata posso immaginare di prendere, dei vari stati,
solo quegli $x, y$ che mi portino (secondo qualche relazione funzionale) ad avere quella determinata temperatura. Questi punti
appartengono ad una curva che prende il nome di \textbf{curva isotermica} e rappresentano solo una
parte ristretta degli stati totali che assumerebbe il mio sistema. 


\section{Misurare la temperatura e sua definizione operativa}
\begin{wrapfigure}{r}{0.4\textwidth}
    \centering
    \caption{Il termometro con il sistema $S$}
    \begin{tikzpicture}
        \draw(0, 0) rectangle (2, 2) node[midway] {$S, T_S$};
        \draw(-1, 0.5) rectangle (0, 1.5) node[midway] {$A, T_A$};
    \end{tikzpicture}    
\end{wrapfigure}Possiamo definire uno strumento (dal principio zero) che ci permetta di misurare
la temperatura: infatti se riesco a definire la temperatura di un sistema io posso 
a quel punto metterlo a contatto termico con un altro e dunque
ottengo la misura della temperatura. Posso quindi utilizzare un \textbf{termometro}
per poter definire operativamente la temperatura così come ho fatto per la definizione
operativa di forza mediante una grandezza minore. Posso valutare allora una proprietà
che dipende dalla temperatura, ossia una \textbf{proprietà termometrica} come
il volume. Mettendo ora a contatto il termometro $A$ con il sistema $S$ che vogliamo misurare
si può misurare la temperatura di $T_A$ che ora è la stessa di $T_S$; tuttavia la
temperatura iniziale del sistema $T_S$ non sono in grado di determinarla.
Per ovviare al problema posso costruire un sistema di termometri con masse sempre più piccole
in modo tale che possa misurare la temperatura con ogni termometro e poi riportare
il sistema alle condizioni iniziali. Posso trovare allora la temperatura di ogni termometro ed ottenere che
questi valori di temperatura sperimentalmente sono sempre
tra valori assegnati e dunque la successione è monotona e limitata e dunque
è convergente: la temperatura è data dal limite
\begin{align}
    T = \lim_{M_A \to 0}T(M_A) 
\end{align}
Ad un certo punto il mio strumento non ha più risoluzione per poter determinare la temperatura.
Qualsiasi sistema $A$ può essere utilizzato per determinare la temperatura
di un certo sistema $S$ purché la massa $m_A << m_S$ per evitare di modificare
in modo sostanziale la temperatura del sistema $S$.

\end{document}
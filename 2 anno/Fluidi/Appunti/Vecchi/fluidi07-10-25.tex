\documentclass[a4paper, oneside]{article}
\usepackage{graphicx}
\usepackage{amsthm}
\usepackage{amsmath}
\usepackage{amssymb}
\usepackage[a4paper,
            bindingoffset=0.2in,
            left=2cm,
            right=2cm,
            top=2cm,
            bottom=2cm,
            footskip=.25in]{geometry}
\usepackage[italian]{babel}
\usepackage{pgfplots}
\usepackage{tabularx}
\usepackage{tikz}
\usepackage{wrapfig}
\usepackage{color}
\usepackage[d]{esvect}
\definecolor{page}{rgb}{0.129,0.157,0.212}
\pagecolor{page}
\color{white}
\graphicspath{ {./images/} }
\usetikzlibrary{shapes.geometric}
\usetikzlibrary{datavisualization}
\usetikzlibrary{datavisualization.formats.functions}
\usetikzlibrary{patterns}
\pgfplotsset{width=10cm,compat=1.9}

\title{Appunti di fluidi}
\author{Tommaso Miliani}
\date{07-10-25}

\begin{document}
\newtheoremstyle{theoremEnv}
                {}          % Space above
                {}          % Space below
                {\slshape}  % Body font
                {}          % Indent amount
                {\bfseries} % Head font
                {.}         % Punctuation after head
                {\newline}         % Space after theorem head
                {}          % Theorem head spec
\theoremstyle{theoremEnv}

\newtheorem{definition}{Definizione}[section]
\newtheorem{theorem}{Teorema}[section]
\newtheorem{lemma}{Proposizione}[section]
\newtheorem{observation}{Osservazione}[section]
\newtheorem{corollary}{Corollario}[theorem]
\newtheorem{example}{Esempio}[section]

\maketitle

\section{Digressione sui sistemi fluidi}
\subsection{L'equazione di stato per i gas perfetti}
\begin{wrapfigure}{r}{0.4\textwidth}
    \centering
    \caption{Trasformazioni nei gas ideali}
    \begin{tikzpicture}
        \draw(0, 0) -- (5, 0) node[at end, below] {$V$};
        \draw[->](0, 0) -- (0, 3.5) node[at end, left] {$p$};
        \draw[dashed] (1, 0) -- node[at start, below] {$V_0$} (1, 1) -- (0, 1)  node[at end, left] {$p_0$};
        \filldraw(1, 1) circle (1pt);
        \filldraw(2, 3) circle (1pt);
        \draw[dashed](2, 0) -- node[at start, below] {$V_1$} (2, 3) -- (0, 3)  node[at end, left] {$p_1$};
        \draw[dashed, ->](4.5, 1) .. controls  (3.5, 1.2) and (2.75, 1.5) .. (2, 3) node[at start, right] {$T_1$};
        \filldraw(4.5, 1);
        \draw[->](1, 1) -- (4.5, 1);
        \draw[dashed](4.5, 0) -- (4.5, 1) node[at start, below] {$V_2$};
        \draw[dashed](1, 1) .. controls (1.3, 0.45) and (1.7, 0.25) .. (2, 0.25);
        \draw[dashed](0.5, 2) .. controls (0.65, 1.35) and (0.95, 1) .. (1, 1);
    \end{tikzpicture}    
\end{wrapfigure}
In ogni sistema termodinamico all'equilibrio vale una
equazione di stato $f(p, V, T) = 0$. In alcuni casi
si è in grado di ricavare l'equazione di stato attraverso
la determinazione di un modello di stato attraverso la fisica
statistica e dagli esprimenti. L'esempio più semplice è quello del
gas perfetto: dagli esperimenti che ci portano a scrivere le quattro
leggi empiriche del gas si può ricavare una equazione di stato. 
Si può, data una mole di gas perfetto, mettersi nelle condizioni
standard: 
\begin{itemize}
    \item $T_0 = 273.15 \ K$: ossia la temperatura di fusione del ghiaccio;
    \item $p_0 = 1 \ atm$: Ossia la pressione dell'atmosfera al livello del mare;
\end{itemize}
Gli esperimenti ci dicono, a queste condizioni, che il gas occupa un volume 
ben preciso:
\begin{gather*}
    V = V_0  = 2.2414 \cdot  10^{-2} \ m^{3} = 22.4 \ l
\end{gather*}   
Possiamo allora utilizzare un diagramma per rappresentare l'equilibrio
dei gas: questo stato è associato ad una certa curva di temperatura secondo
la legge di Boyle. Posso allora applicare delle trasformazioni che mi permettano di raggiungere il 
volume $V_2$ alla pressione $p_0$ e lo posso esprimere in funzione
del volume precedente e della temperatura:
\begin{gather*}
    V_2 = V_0 \frac{T_1}{T_0 }
\end{gather*}
Se volessi farlo passare dallo stato $(p_1, V_1, T_1)$ allo stato $(p_0, V_2, T_1)$, allora dovrei fare in modo che il  sistema  mantenga
costante la temperatura ma possa variare la sua pressione in modo tale 
da avere il volume che io cerco. La temperatura di questi due stati di equilibrio
è la stessa ma ho cambiato sia il volume che la temperatura proprio
secondo la legge di Mariotte, deve risultare:
\begin{gather*}
    p_1 V_1 = p_0 V_2
\end{gather*}
Allora si ha, sostituendo l'espressione del volume 2:
\begin{gather*}
    p_1 V_1 = \frac{p_0 V_0}{T_0}T_1
\end{gather*}
Posso esprimere allora la costante che moltiplica la temperatura
come 
\begin{align}
    R = \frac{p_0V_0}{T_0} = 8.3145 \ \frac{J}{K}
\end{align}
Ossia l'unità di energia per grado Kelvin che prende il nome di 
\textbf{costante dei gas} che risulta vera per solo una mole di gas. Tuttavia, 
la legge di Avogadro mi dice che $pV$ è proporzionale al numero
di moli del gas, si ottiene quindi l'equazione di stato
per i gas perfetti:
\begin{align}
    pV = nRT
\end{align}
Queste considerazioni sono fatte a livello macroscopico: già il concetto
di moli non richiede una spiegazione a livello microscopico in quanto è un
concetto indipendente  e non necessita di una trattazione a livello microscopico.

\subsection{Il limite di utilizzo dell'equazione di stato dei gas}
\begin{wrapfigure}{r}{0.4\textwidth}
    \centering
    \caption{}
    \begin{tikzpicture}
        \draw[->](0, 0) -- (4, 0) node[at end, below] {$V$};
        \draw[->](0, 0) -- (0, 3.5) node[at end, left] {$p$};
        \draw(0.5, 3.5) .. controls (1, 2) and (3, 1.5) .. (3.5, 1.5);
        \draw(0.4, 3.5) .. controls (0.5, 1.75) .. (1, 1.75);
        \draw(1, 1.75) .. controls (2, 1.85) and (2.8, 0.5) .. (3.5, 0.5); 
        \filldraw(1, 1.75) circle (1pt) node[anchor = south] {$P_C$};
        \draw(0.3, 3.5) -- (0.5, 1) --  (2.25, 1) -- (3.5, 0.35);
    \end{tikzpicture}    
\end{wrapfigure}
Se si volesse analizzare il caso di un gas non perfetto ma molto semplice,
è ragionevole trattare l'equazione di stato dei gas in maniera tale che
assomigli ad uno sviluppo di Taylor e che i termini più grandi si diventino molto
piccoli quando il volume cresce molto:
\begin{gather*}
    pV = nRT\left(1 + \frac{B_1}{V} + \frac{B_2}{V^{2} } + \dots\right)
\end{gather*}
Si osserva che per gas anche non rarefatti, i coefficienti sono funzione
sia della temperatura che della composizione chimica del gas stesso. 
Si possono determinare sperimentalmente in modo tale che questa 
curva teorica mi produca, nel modo migliore possibile, i miei dati sperimentali.
Si fa dunque una ipotesi ragionevole per questa funzione anche se questa ipotesi
è valida solamente se la temperatura è sufficientemente grande. 
Dire che esistono temperature grandi e temperature piccole vuol dire 
che il sistema ha una scala di temperatura rispetto alla quale
ha senso parlare di temperatura grande. Tuttavia ad un certo punto
lo \textbf{sviluppo del viniale}, non funzionerà più: per trovare questo limite
possiamo eseguire una serie di esperimenti per alcuni gas abbassando sempre
di più la temperatura.  Ad un certo punto, se si continua ad aggiungere termini 
e a far decrescere la temperatura, si 
raggiunge un \textbf{isoterma critica} nella quale si ha un flesso e sotto la
quale la curva isoterma non è più una curva ma presenterà due punti angolosi tra i quali
è costante e dunque non è più descrivibile analiticamente. Ogni
gas ha una sua temperatura critica $T_c$ sotto alla quale lo sviluppo
del viniale non è più corretto. 


\begin{wrapfigure}{r}{0.4\textwidth}
    \centering
    \caption{La temperatura ad una pressione critica}
    \begin{tikzpicture}
        \draw[->](0, 0) -- (4, 0) node[at end, below] {$T$};
        \draw[->](0, 0) -- (0, 3) node[at end, left] {$V$};
        \draw(0.2, 0.2) .. controls (0.6, 0.3) .. (1, 0.5);
        \draw[dashed] (1, 0) -- (1, 3) node[at start, below] {$T_{\overline{p} }$};
        \draw(1, 2) .. controls (2, 2.2) and (2.7, 2.2) .. (3.5, 3);
    \end{tikzpicture}    
\end{wrapfigure}
La parte costante tra i due punti angolosi
rappresenta esattamente un cambio di stato: il sistema sta passando da gassoso a
liquido: più si prosegue verso sinistra nel grafico volume-pressione e più il liquido necessita 
di una pressione sempre più grande per piccole variazioni di volume.
Non si è sicuri che in una relazione tra tre variabili esse siano tutte e
tre indipendenti l'una dall'altra; ci possono essere delle situazioni in cui
esiste la funzione di stato ma presenta delle singolarità. 
\begin{gather*}
    V = V(p, T)
\end{gather*}
Questa funzione è continua e derivabile se si è a pressione maggiore di quella
critica, altrimenti non è regolare ma devo prendere la funzione con una
certa pressione assegnata $\overline{p}$:
\begin{gather*}
    V = V_{\overline{p} }(T) = V(\overline{p}, T )
\end{gather*} 
Quando la temperatura supera la temperatura
fissata a pressione $T_{\overline{p}}$  allora il sistema avrà grandi incrementi
di volume per piccoli incrementi di temperatura in quanto, sopra la temperatura 
che dipende dalla pressione fissata, il sistema inizierà un cambiamento di
stato da liquido a gassoso. 

\subsection{L'equazione di Van Der Walls per la descrizione dei gas}
Dato che la legge dei gas perfetti non vale per tutti i gas, si introduce una
equazione che possa funzionare anche lontano dal limite della curva critica
chiamata \textbf{equazione di Van Der Walls}:
\begin{align}
    \left(p + \frac{q}{V^{2} }\right)(V- b) = RT
\end{align}
Questa equazione prevede l'esistenza di un punto critico (ossia il flesso)
e fu derivata da Van Der Walls e riesce
a descrivere accuratamente le isoterme dei gas. Possiamo dimostrare che questa
equazione può prevedere l'esistenza di un punto critico: ossia deve esistere una situazione
per la quale la derivata seconda rispetto al volume deve essere zero.
\begin{gather*}
    p = \frac{RT}{V - b} - \frac{a}{V^{2} } \\
    \frac{\partial p}{\partial V} = -\frac{RT}{(V - b)^{2} } + \frac{2a}{V^{3} }\\
    \frac{\partial^{2} p}{\partial V^{2} } = \frac{2RT}{(V - b)^{2} } - \frac{6a}{V^{4} }
\end{gather*}

\begin{wrapfigure}{r}{0.4\textwidth}
    \centering
    \caption{le curve isoterme rispetto alla curva di temperatura critica}
    \begin{tikzpicture}
        \draw[->](0, 0) -- (4, 0) node[at end, below] {$V$};
        \draw[->](0, 0) -- (0, 4) node[at end, left] {$p$};
        \draw(0.5, 3.5) .. controls (1, 2) and (3, 1.5) .. (3.5, 1.5) node[at end, above] {$T > T_c$};
        \draw(0.4, 3.5) .. controls (0.5, 1.75) .. (1, 1.75);
        \draw(1, 1.75) .. controls (2, 1.85) and (2.8, 0.6) .. (3.5, 0.6) node[at end, right] {$T = T_c$}; 
        \draw(0.3, 3.5) .. controls (0.4, 0.5) .. (0.5, 0.5);
        \draw(0.5, 0.5) .. controls (0.8, 1.2) and (1, 1.2) .. (1.5, 0.9);
        \draw(1.5, 0.9) .. controls (2.2, 0.5) .. (3.5, 0.3) node[at end, right] {$T < T_c$};
    \end{tikzpicture} 
\end{wrapfigure}
Dato che devono entrambe essere zero:
\begin{gather*}
    \frac{2a}{V^{3}}  = \frac{RT}{(V - b)^{2} } \\
    \frac{6a}{V^{4}} = -\frac{2RT}{(V - b)^{2} } 
\end{gather*}
Si ottiene allora le espressioni per le tre grandezze critiche:
\begin{align*}
    RT_c &= \frac{8a}{27b} \\
    V_c &= 3b \\
    p_c &= \frac{a}{27b^{2} } 
\end{align*}
Se l'equazione di Van Der Walls è valida, allora deve valere che 
\begin{gather*}
    \frac{8}{3}\frac{p_cV_c}{RT_c} = 1
\end{gather*}
Che si può testare sperimentalmente.  La terza curva è una curva impossile che tuttavia
rispetta questa equazione: se quella situazione avvenisse, allora il sistema esploderebbe.
Si deve allora introdurre la correzione di MaxWell.

\end{document}
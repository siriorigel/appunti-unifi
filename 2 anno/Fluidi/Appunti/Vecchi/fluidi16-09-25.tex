\documentclass[a4paper, oneside]{article}
\usepackage{graphicx}
\usepackage{amsthm}
\usepackage{amsmath}
\usepackage{amssymb}
\usepackage[a4paper,
            bindingoffset=0.2in,
            left=2cm,
            right=2cm,
            top=2cm,
            bottom=2cm,
            footskip=.25in]{geometry}
\usepackage[italian]{babel}
\usepackage{pgfplots}
\usepackage{tabularx}
\usepackage{tikz}
\usepackage{wrapfig}
\usepackage{color}
\definecolor{page}{rgb}{0.129,0.157,0.212}
\pagecolor{page}
\color{white}
\graphicspath{ {./images/} }
\usetikzlibrary{shapes.geometric}
\usetikzlibrary{datavisualization}
\usetikzlibrary{datavisualization.formats.functions}
\pgfplotsset{width=10cm,compat=1.9}

\title{Appunti fluidi}
\author{Tommaso Miliani}
\date{16-09-25}

\begin{document}
\newtheoremstyle{theoremEnv}
                {}          % Space above
                {}          % Space below
                {\slshape}  % Body font
                {}          % Indent amount
                {\bfseries} % Head font
                {.}         % Punctuation after head
                {\newline}         % Space after theorem head
                {}          % Theorem head spec
\theoremstyle{theoremEnv}

\newtheorem{definition}{Definizione}[section]
\newtheorem{theorem}{Teorema}[section]
\newtheorem{lemma}{Proposizione}[section]
\newtheorem{observation}{Osservazione}[section]
\newtheorem{corollary}{Corollario}[theorem]
\newtheorem{example}{Esempio}[section]

\maketitle

\section{Introduzione al corso}
Il corso di Fluidi si divide in tre parti 
\begin{enumerate}
    \item Termodinamica: studia processi non di carattere meccanico che si manifestano a livello
    macroscopico
    \item Statistica: la fisica statistica coniuga la meccanica con la statistica che si usa nelle applicazioni
    che vanno al di là della fisica classica: da un fondamento microscopico agli effetti della termodinamica.
    \item Meccanica dei fluidi: sistemi materiali macroscopici per lo studio dei fluidi.
\end{enumerate}
Prima di tutto si inizia con un cappello che collega le tre parti;
poi si inizia con statica dei fluidi, dopo termodinamica, poi statistica e infine
dinamica dei fluidi. Questo perché alcuni strumenti di analisi due saranno
utilizzati per la parte di dinamica dei fluidi e statistica. \\
L'esame è diviso in tre domande orali (una per sezione) (niente prova scritta) con esercizi
applicati alla realtà e di effetto pratico (che sono fatti a lezione).
Il libro consigliato è "Elementi di meccanica di fluidi" di Egidio Landi, per la parte di termodinamica
"Termodinamica" di Enrico Fermi e per la parte di fisica statistica le dispense del prof
in quanto non ci sono libri che facciano al caso del corso. \\
Il vantaggio di studiare tre discipline in un corso è che si riesce ad avere una idea più
completa dei sistemi che si studiano.

\section{Statica dei fluidi}
\subsection{Introduzione alla statica dei fluidi}
Il caso più semplice della fisica è quello dell'approssimazione del
punto materiale; cosa succede allora se io aumentassi il numero di punti materiali
e li mettessi tutti insieme? Se si mettono insieme in modo da poter costituire un
corpo rigido allora avrei da risolvere diverse equazioni per poter determinare il comportamento
del corpo; nessun materiale reale gode di questa proprietà ma è solo una idealizzazione
che vale entro certe approssimazioni. Il modello del corpo rigido funziona molto
bene per i corpi solidi che in prima approssimazione sono ben descrivibili con questo modello, mentre
nel caso dei fluidi questa approssimazione non è valida. \\
Nei sistemi fluidi si considera allora il movimento degli atomi neutri
all'interno del fluido (non di plasmi) e quindi si può, utilizzando le conoscenze
di Fisica I, determinare come questi si muovono.

\subsubsection{L'atomo di Feynman}
In questo studio dei fluidi utilizziamo la definizione di atomo così come la aveva data Feynman: gli atomi sono dei punti materiali
che si attraggono se sono sufficientemente vicini ma se troppo vicini si respingono.
Essenzialmente a livello macroscopico l'atomo è visto come piccole palline che costituiscono il corpo. Possiamo
allora definire diverse proprietà della materia che utilizziamo per la descrizione dei fluidi:
\begin{itemize}
    \item Gli atomi sono dei punti materiali che hanno una certa massa $m$ che obbediscono alle leggi della meccanica
    di Newton;
    \item Un campione di materia è un insieme di $N$ atomi (ossia un numero molto grande) che
    possono essere chiusi in un contenitore ideale (ossia con pareti rigide e fisse).  
    \item Gli atomi interagiscono tra di loro solo tramite forze conservative, ossia che possono
    essere derivate e che queste forze dipendono esclusivamente dalla distanza tra
    gli atomi stessi; questo vuol dire che compiono solo urti completamente elastici con il contenitore.
\end{itemize}
Posso allora dare una definizione concreta alle forze intermolecolari, iniziando definendo $r_0$ come la distanza caratteristica
che ha definito Feynman oltre la quale gli atomi si attraggono ma entro la quale si respingono.
\begin{itemize}
    \item Se la distanza tra i due atomi è maggiore ma dello stesso ordine di $r_0$,
    allora le forze sono attrattive.
    \item Se la distanza è minore di $r_0$, allora la forza è repulsiva. 
    \item Le interazioni decadono molto velocemente con l'aumentare della distanza.
\end{itemize}

\subsubsection{Il potenziale di interazione}
\begin{wrapfigure}{r}{0.4\textwidth}
    \centering
    \caption{Il potenziale di interazione}
    \begin{tikzpicture}
        \draw[->](-1, 0) -- (4, 0) node[at end, below] {$r$};
        \draw[->](0, -2) -- (0, 2) node[at end, left] {$V(r)$};
        \draw(0.5, 2) .. controls (1.2, -2) and (1.7, -2) ..  (3, 0);
        \draw(3, 0) -- (4, 0);
        \draw[dashed](4, 0) -- (5, 0);
        \draw[dashed](1.55, 3) -- (1.55, -2) node[at end, right] {$r_0$};
    \end{tikzpicture}    
\end{wrapfigure}
Posso allora definire il \textbf{potenziale di interazione} tra due atomi come $V(r)$ e 
quindi posso relazionare le tre condizioni precedenti con il potenziale, il quale deve diventare 
costante quando i due oggetti si trovano a grandi distanze in quanto la forza diventa trascurabile.
Posso fissare un sistema di riferimento inerziale e dunque
posso dire che l'energia totale del sistema sarà data da
\begin{align}
    E = K + V\ \Longrightarrow \ \frac{1}{2}\sum |\vec{v}_i|^{2}  + U(\vec{r}_1, \dots, \vec{r}_n  )
\end{align}
Data l'ipotesi che ci siano solo delle forze conservative all'interno del sistema
che studiamo, allora posso riscrivere la componente potenziale (la quale dipende da $n$ variabili),
è scrivibile come la somma di tutte le possibili coppie e dunque
\begin{align}
    U(\vec{r}_1, \dots, \vec{r}_n  ) = \sum_{i = 0}^{n} \sum_{j = i + 1}^{n}  V(\left| \vec{r}_i - \vec{r}_j   \right| )
\end{align}
Posso anche esprimere anche l'energia potenziale attraverso la sommatoria rispetto a tutte le coppie
escludendo l'elemento già considerato (quindi si dimezza la sommatoria):
\begin{align}
    U(\vec{r}_1, \dots, \vec{r}_n  ) = \frac{1}{2}\sum_{i = 1}^{n} \sum_{j = 1 (i \neq j)}^{n} V(\left| \vec{r}_i - \vec{r}_j   \right| )  
\end{align}
I termini contano solo se la distanza è minore $a$ volte $r_0$: fissato allora un istante di tempo
posso scegliere un atomo e dire quali coppie possono effettivamente contribuire
all'energia potenziale e quindi ogni atomo interagisce solo con gli atomi vicini che si trovano
nella sfera (e quindi posso esprimerlo come $a \cdot  r_0$), posso allora approssimare con un pochino meno termini
rispetto a $\frac{N(N -1 )}{2}$ in quanto ogni atomo interagisce solo con gli altri
all'interno della sfera di interazione.
\begin{gather*}
    U(\vec{r}_1, \dots, \vec{r}_n  ) = \frac{1}{2}\sum_{i = 0}^{n}\sum_{j \in S(i)}^{n}V(\left| \vec{r}_i - \vec{r}_j   \right| )  
\end{gather*}
Quanti sono allora gli atomi massimi consentiti all'interno della sfera di interazione
$S(i)$ rispetto ad un dato atomo $j$? Se prendessi allora questa sfera di raggio
$a \cdot r_0$, si ottiene una stima che dipende da $r_0$ e lo divido per il volume della singola sferetta
(il quale è sempre una approssimazione per eccesso perché non considero che siano rigide) per determinare il
numero di atomi $n_s$ che interagiscono con l'atomo.
La sfera di interazione ha raggio $a \cdot r_0$ (dove $a$ è un numero) e il singolo atomo ha raggio $\frac{r_0}{2}$
e quindi devo fare il rapporto tra i volumi delle sfere e ottengo $n_s \approx (2a)^{d}$, dove $d$ è la dimensione
(piano = 2, spazio = 3). Si ottiene allora che per $a = 2.5$, $d = 2$ si ha $n_s \approx 25$, per $d = 3$ e $n_s \approx 125$.
La dimensione è legata a partire dal volume della sfera, infatti il raggio è elevato alla dimensione
dello spazio considerato. \\
L'energia potenziale degli atomi nella materia è direttamente proporzionale al numero di atomi
infatti si ottiene che è proporzionale a $Nn_s$ invece che $N^{2}$ come si era detto prima (abbiamo allora escluso $N$ dati). Quando una grandezza soddisfa questa proprietà allora
è una grandezza \textbf{estensiva}: la maggioranza dei termini non conta e dunque l'energia di un sistema è
una grandezza estensiva, se così non fosse allora la materia non si comporterebbe come si comporta.


\subsubsection{L'interazione tra due parti di uno stesso contenitore}
\begin{wrapfigure}{r}{0.3\textwidth}
    \centering
    \caption{L'interazione tra due recipienti uniti}
    \begin{tikzpicture}
        \draw(0, 0) rectangle (3, 2);
        \filldraw (0.5, 1) circle (0pt) node[anchor = south] {$N_1$};
        \filldraw (2, 1) circle (0pt) node[anchor = south] {$N_2$};
        \draw(1, 2) -- (1, 0);
        \draw[dashed](1.2, 2) -- (1.2, 0);
        \draw[dashed](0.8, 2) -- (0.8, 0);
        \filldraw(4, 1.5) circle (0pt) node[anchor = south] {$N_1N_2 \propto L^{d} $ };
        \filldraw(4, 1) circle (0pt) node[anchor = south] {$N_{1,2} \propto L^{d - 1} $ };
    \end{tikzpicture}    
\end{wrapfigure}
Immaginando di suddividere il contenitore in due parti allora avremo un po' del fluido che sta da una parte e un
po' che sta dall'altra: questa divisione va fatta in modo tale che le due parti debbano essere macroscopiche
(non pochi atomi da una parte e il resto dall'altra). Come posso determinare l'energia di questo sistema?
Dato allora il sistema $N_1 + N_2 = N$, ossia il numero di atomi del primo più il secondo compartimento
mi dà gli atomi totali (con $N >>>> 1$).  Devo dividere il contenitore
in modo tale che il "divisore" sia di una dimensione in meno dello spazio considerato (es nello spazio un piano)
in modo tale che sia uniforme. \\
Posso allora esprimere l'energia totale come
\begin{gather*}
    E = E_1 + E_2 + E_{1,2} \\
    E_1 =  \frac{1}{2}\sum_{i = 1}^{N_1}\left| \vec{v^{1} }_i  \right|^{2} + U (\vec{r}_1^{1} , \dots \vec{r}_2^{1}   )  \propto N_1  \\
    E_2 =  \frac{1}{2}\sum_{i = 1}^{N_2}\left| \vec{v^{2} }_i  \right|^{2} + U (\vec{r}_1^{2} , \dots \vec{r}_2^{2}   )  \propto N_2
\end{gather*}
Come si nota, esiste anche un contributo di interazione tra le due parti che posso esprimere come:
\begin{gather*}
    E_{1, 2} = \frac{1}{2}\sum_{i = 1}^{N_1}\sum_{j \neq i}^{N_2} V\left(\left| \vec{r^{N_1} }_i   - \vec{r^{N_2} }_j   \right| \right)  
\end{gather*}
Il contributo di questa interazione è molto più piccola dell'interazione $E_1$ e $E_2$ in quanto solamente gli atomi
molto vicini al bordo che divide i due sistemi interagiscono tra di loro (zona tratteggiata) e dunque
più cresce $N$ allora più diminuisce questo contributo. Inoltre si osserva che sono proporzionali ad un certo $L$ che è dato
dai limiti fisici del contenitori (chiamata scala libera):
\begin{gather*}
    L \propto N_1^{\frac{1}{d}} \ \Longrightarrow \ N_{1, 2} \propto N_1^{\frac{d- 1}{d}} \qquad \text{uguale per $N_2$}
\end{gather*}
Se $N_1$ cresce, allora $N_{1, 2}$ diventa molto piccolo.  Si ha che l'energia complessiva di un sistema
che ho arbitrariamente diviso in due pezzi avrà l'energia della somma dei due contributi ignorando il
termine di interazione tra le due parti purché siano macroscopici (infatti non sarebbe vero
se una delle due componenti avesse pochi atomi).
\begin{align}
    E \approx E_1 + E_2 \qquad \qquad per \  N >>>> 1 \ \Longrightarrow \ E = E_1 + E_2
\end{align}
Allora ottengo che l'energia è una \textbf{quantità additiva} (infatti è vero per $m$ divisioni fino a che
le componenti sono macroscopiche). Tutte queste considerazioni sono valide se e solo se
io continuo a ignorare i contributi all'esterno della sfera di interazione e considerando quindi che quei contributi convergano a zero in modo
sufficientemente rapido: infatti l'analisi ci insegna che somme di piccole quantità molto grandi non sempre
convergono a numeri finiti. Si può dimostrare che devono convergere a zero più velocemente di 
$r^{-d} $ (dove $d$ è nuovamente la dimensione dello spazio) (per l'interazione gravitazionale questo non è vero infatti). 

\end{document}
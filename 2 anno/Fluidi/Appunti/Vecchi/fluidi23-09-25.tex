\documentclass[a4paper, oneside]{article}
\usepackage{graphicx}
\usepackage{amsthm}
\usepackage{amsmath}
\usepackage{amssymb}
\usepackage[a4paper,
            bindingoffset=0.2in,
            left=2cm,
            right=2cm,
            top=2cm,
            bottom=2cm,
            footskip=.25in]{geometry}
\usepackage[italian]{babel}
\usepackage{pgfplots}
\usepackage{tabularx}
\usepackage{tikz}
\usepackage{wrapfig}
\usepackage{color}
\usepackage[d]{esvect}
\definecolor{page}{rgb}{0.129,0.157,0.212}
\pagecolor{page}
\color{white}
\graphicspath{ {./images/} }
\usetikzlibrary{shapes.geometric}
\usetikzlibrary{datavisualization}
\usetikzlibrary{datavisualization.formats.functions}
\usetikzlibrary{patterns}
\pgfplotsset{width=10cm,compat=1.9}

\title{Appunti di fluidi}
\author{Tommaso Miliani}
\date{23-09-25}

\begin{document}
\newtheoremstyle{theoremEnv}
                {}          % Space above
                {}          % Space below
                {\slshape}  % Body font
                {}          % Indent amount
                {\bfseries} % Head font
                {.}         % Punctuation after head
                {\newline}         % Space after theorem head
                {}          % Theorem head spec
\theoremstyle{theoremEnv}

\newtheorem{definition}{Definizione}[section]
\newtheorem{theorem}{Teorema}[section]
\newtheorem{lemma}{Proposizione}[section]
\newtheorem{observation}{Osservazione}[section]
\newtheorem{corollary}{Corollario}[theorem]
\newtheorem{example}{Esempio}[section]

\maketitle

\section{Fluidodinamica}
\subsection{Il concetto di fluido ideale}
\begin{wrapfigure}{r}{0.4\textwidth}
    \centering
    \caption{Il fluido diviso in due}
    \begin{tikzpicture}
        \draw(0, 0) .. controls (1, -0.5) and (2.5, 0.5) .. (3, -0.5);
        \draw(3, -0.5) .. controls (2.9, -1) and (2, -0.75) .. (1, -1);
        \draw(1, -1) .. controls (-1, -1.4) and (-0.5, 0.5) .. (0, 0);
        \draw[dashed](1, -0.2) .. controls (0.8, -0.55) .. (0.75, -1);
        \draw[dashed](1.3, -0.2) .. controls (1.3, -0.55) .. (1, -1);
        \draw[->, red](1.1, -0.5) -- (1.75, -0.5) node[at end, right] {$\hat{n}$  };
        \draw[->, cyan](1.1, -0.5) -- (1.75, 0) node[at end, above] {$\vv{F}$ };
    \end{tikzpicture}    
\end{wrapfigure}
La Fluidodinamica si occupa di definire come si comporta un fluido.
Un fluido è un corpo continuo formato da tanti puntini (molecole o atomi)
con massa e dimensione finita. Supponendo di dividere il fluido in due parti
$A$ e $B$ separate da una sezione, la cui superficie indichiamo con $\Sigma$. Devo avere  una qualche forza
che esercita la sezione $A$ sulla sezione $B$ sulla superficie di contatto $\Sigma$.
\begin{gather*}
    \vv{F_{AB}} = F_{\parallel}\hat{n} + \vv{F_{\perp}}    
\end{gather*}
Con il versore $\hat{n}$ perpendicolare alla superficie $\Sigma$ considerata e
il vettore $\vv{F_{\perp}}$ perpendicolare al versore $\hat{n}$; si definisce allora fluido se 
\begin{gather*}
    |F_{\perp}| \llless |F_{\parallel}| \qquad F_{\parallel} > 0
\end{gather*}
Posso considerare ora la forza che imprime $B$ su di $A$. Dato che conosco la forza
che imprime $A$ su $B$, allora posso dire che
\begin{gather*}
    \vv{F_{BA}} = -\vv{F_{AB}}  
\end{gather*}

\begin{wrapfigure}{r}{0.4\textwidth}
    \centering
    \caption{}
    \begin{tikzpicture}
        \draw(-1, 1.5)  -- (-1, 0)-- (3, 0) -- (3, 1.5);
        \draw(0.5, 0) rectangle (1.5, 1);
        \draw[->](0.5, 0.5) -- (-0.5, 0.5);
        \draw[->](1.5, 0.5) -- (2.5, 0.5);
    \end{tikzpicture}    
\end{wrapfigure}

La definizione di $F$ parallelo è una definizione che vale sempre anche
per l'altro senso in quanto il prodotto scalare con il versore uscente è sempre lo stesso.
E' proprio una cosa fisica: si suppone di prendere un contenitore e di porvi un fluido: questo
fluido spingerà da entrambe le direzioni esercitando una forza uscente rispetto alla
normale (ossia la sua pressione) . Esiste anche una proprietà per la quale
il fluido non si scompone in pezzi ma rimane sempre unito che è la
\textbf{tensione}. Un fluido può scorrere su di una superficie proprio
perché la sua forza perpendicolare è molto piccola; tuttavia il fluido ideale scorre
senza attrito: il fluido reale invece presenta attrito viscoso di taglio. Nella trattazione
dei fluidi in fluidodinamica utilizziamo solamente l'ipotesi
di fluido ideale che scorre senza attrito.  

\subsection{Definizione del campo scalare della pressione}
\begin{wrapfigure}{r}{0.4\textwidth}
    \centering
    \caption{}
    \begin{tikzpicture}
        \draw[->](0, 0) -- (1, -0.7) node[at end, below] {$\hat{x}$ }; 
        \draw[->](0, 0) -- (-1, -0.7) node[at end, below] {$\hat{y}$ };
        \draw[->](0, 0) -- (0, 1) node[at end, left] {$\hat{z}$ };
        \draw[->](0, 0) -- (3, 1.5) node[midway, below] {$\vv{r}$ };
        \draw(2.85, 1.35) rectangle (3.15, 1.65) node[anchor = north west] {$\Delta\Sigma$};
        \draw(1.7, 1.2) .. controls (2.8, 2.15) and (3.2, 1.9) .. (3.4 , 1.8);
        \draw(3.4, 1.8) .. controls (4, 1.6) and (3.8, 1.1) .. (3.2, 1);
        \draw(3.2, 1) .. controls (2, 0) and (1.4, 1) .. (1.7, 1.2);
    \end{tikzpicture}    
\end{wrapfigure}
Riprendendo il volume arbitrario (o una porzione di fluido ideale) 
posso considerare un sistema di riferimento con terna
destrorsa di versori e, preso un punto sulla superficie
del volume di fluido,posso identificarlo con un raggio
vettore $\vv{r}$  il punto sulla superficie del fluido, sulla sezione della superficie
$\Delta \Sigma$ io identifico la spinta verso l'esterno rispetto alla massa del volume
\begin{gather*}
    \Delta \vv{R} =  \overline{P} \Delta \Sigma \hat{n}  
\end{gather*}
Dove $\overline{P}$ è una quantità positiva definita come
\begin{gather*}
    \overline{P} = \frac{\Delta R}{\Delta \Sigma} 
\end{gather*} 
Ossia la pressione esercitata dal fluido verso l'esterno. Posso allora definire la pressione come
\begin{align}
    P(x, y, z) = \lim_{\Delta \Sigma \to 0} \frac{\Delta R}{\Delta \Sigma}  \quad \frac{[F]}{[L]^{2} } = \frac{[E]}{[L]^{3} }
\end{align}
Siamo allora passati da una quantità finita ad una quantità puntiforme: questa funzione è 
quindi univocamente definita in un punto dello spazio sulla superficie del fluido stesso. 
Questo è quello che si definisce un \textbf{campo scalare}. Un campo scalare è una funzione 
che associa uno scalare a ogni punto dello spazio. 
E' utile definire la pressione come energia per unità di volume (molto utile per lavorare
con l'energia in termodinamica).  

\subsection{Definizione del campo scalare della densità}
\begin{wrapfigure}{r}{0.4\textwidth}
    \centering
    \caption{}
    \begin{tikzpicture}
                \draw[->](0, 0) -- (1, -0.7) node[at end, below] {$\hat{x}$ }; 
        \draw[->](0, 0) -- (-1, -0.7) node[at end, below] {$\hat{y}$ };
        \draw[->](0, 0) -- (0, 1) node[at end, left] {$\hat{z}$ };
        \draw[->](0, 0) -- (3, 1.5) node[midway, below] {$\vv{r}$ };
        \draw(2.85, 1.35) rectangle (3.15, 1.65) node[anchor = north west] {$\Delta V$};
        \draw(1.7, 1.2) .. controls (2.8, 2.15) and (3.2, 1.9) .. (3.4 , 1.8);
        \draw(3.4, 1.8) .. controls (4, 1.6) and (3.8, 1.1) .. (3.2, 1);
        \draw(3.2, 1) .. controls (2, 0) and (1.4, 1) .. (1.7, 1.2);
    \end{tikzpicture}    
\end{wrapfigure}
Considerato il solito sistema di riferimento stavolta, invece di intercettare un punto
sulla superficie, prendo un punto interno al volume del fluido: invece di prendere
un elemento di superficie, prenderò un elemento di volume per cui posso misurare una
certa massa: posso allora scrivere la densità media di quel cubetto di fluido come
\begin{gather*}
    \overline{\rho} = \frac{\Delta m}{\Delta V} 
\end{gather*}
Ossia la densità media. Posso trasformarla in una quantità che è definita per ogni
punto del fluido e quindi ottenere il campo scalare della densità come
\begin{gather*}
    \rho(x, y, z) = \lim_{\Delta V \to 0} \frac{\Delta m}{\Delta V} 
\end{gather*}
Si definiscono allora, data la densità, due classi di fluidi: i liquidi, che sono incomprimibili,
ed i gas che sono invece dei fluidi comprimibili.  Infatti nei liquidi se prendiamo una sezione
molto piccola di fluido la densità non varia ma è sempre costante per tutto il liquido (fluido
incomprimibile) mentre per un gas la densità può cambiare a seconda della sezione di volume
considerata(fluido comprimibile).

\subsection{Comprimibilità}
Si può stimare la comprimibilità del fluido spingendolo con una certa pressione:
\begin{gather*}
    \frac{\Delta P}{\epsilon} = - \frac{\Delta V}{V_0}
\end{gather*}
Ossia ci si aspetta una compressione del volume a seguito dell'applicazione di una
certa pressione $P$ in modo uniforme su tutta la superficie del fluido. Chiamato allora $\epsilon$ il fattore
di comprimibilità, questo definisce la comprimibilità del fluido considerato:  più è grande e più il fluido è incomprimibile. 
A condizioni standard la comprimibilità dell'aria è circa $10^{5} \ Pa$ mentre l'acqua è più
incomprimibile dell'aria con un $\epsilon = 10^{9} \ Pa$.

\subsection{La resistenza all'espansione di un fluido}
Se volessi trovare l'equilibrio delle forze del fluido potrei considerare un
sistema di riferimento ed una piccola porzione di superficie del fluido. Questa porzione del
fluido agisce in modo tale da imprimere una forza infinitesima
\begin{gather*}
    d\vv{F_{\Sigma}} = -p\hat{n}d\sigma  
\end{gather*}
per resistere alla forza che vorrebbe fare espandere il fluido, mentre $d\sigma$ è un elemento di 
superficie infinitesimo. Un fluido
cercherà allora di mantenere i legami tra le sue molecole ad una pressione esterna.
La risultante delle forze di superficie sarà allora la somma di tutti i contributi delle
forze sulla superfici. Calcolando allora la reazione che imprime la parte
interna del liquido sulla superficie totale 
\begin{gather*}
    \vv{R_{\Sigma}}  = -\oint_{\Sigma(V)} p\hat{n} \ d\sigma 
\end{gather*}
Per ogni punto posso allora associare una quantità vettoriale e dunque definisco
un campo vettoriale derivante dalle forze esterne che agiscono sul fluido:
\begin{gather*}
    d\vv{F_V} = \vv{f}(x, y, z)dV  
\end{gather*}
quindi
\begin{gather*}
    \vv{R_V} = \int_{V} \rho \vv{g}\ dV  
\end{gather*}
dove $\vv{g}$ è una accelerazione generica. 
Dato che ora voglio l'equilibrio di un volume finito di fluido, ora impongo:
\begin{gather*}
    -\oint_{\Sigma(V)} p\hat{n} \ d\sigma + \int_{V} \rho \vv{g}\ dV   = 0 
\end{gather*} 
Ossia l'equazione di \textbf{equilibrio idrostatico}. Si può risolvere in un caso particolare in cui le forze di volume sono
trascurabili rispetto a quelle di superficie (dipende da diversi fattori); generalmente se si riduce 
il volume del fluido considerato che le forze di volume diminuiscono 10  volte più velocemente
delle forze di superficie, Ci poniamo allora nelle condizioni per cui
\begin{gather*}
    \vv{R_{\Sigma}} = 0 \qquad \oint_{\Sigma(V)} p\hat{n} \ d\sigma = 0    
\end{gather*}
La seconda condizione vale se e solo se la pressione sulla superficie del fluido è costante;
dato che la superficie è arbitraria, allora la pressione è costante in tutto il fluido
poiché potrei considerare una superficie che tocca il bordo del fluido e sia parzialmente contenuta 
nel fluido stesso, allora anche questa porzione avrà pressione costante. Questa affermazione è
dimostrabile nel seguente modo:
\begin{gather*}
    \oint_{\Sigma(V)} p\hat{n} \ d\sigma = 0  \ \Longrightarrow \ p = const \ \Longrightarrow \ p\oint \hat{n} \ d\sigma = 0 
\end{gather*}
Il secondo integrale è una proprietà geometrica e dunque è per forza zero: se prendessi infatti il generico vettore
$\hat{n}$ di un certo $\delta r$, allora ottengo il volume del cilindro che avrà altezza il prodotto scalare
dei due vettori normale e $\delta \vv{r}$ e questa traslazione non cambia il volume ma si forma un cilindro che 
indica il volume dovuto alla traslazione di ogni punto del fluido:
\begin{gather*}
    dV = d\sigma \delta \vv{r} \cdot  \hat{n} \ \Longrightarrow \ \Delta V = \int_{\Sigma(V)} \delta \vv{r} \cdot  \hat{n}  \ d\sigma = \delta r_0 \oint_{\Sigma}\hat{n} \ d\sigma = 0 
\end{gather*} 
Se scelgo arbitrariamente il vettore $\vv{r}$ come $\vv{r_0}$ allora si ottiene che quell'integrale deve fare zero in quanto tutti i 
contributi all'area devono annullarsi.  (Il volume del
cilindro rappresenta l'area dello spostamento di un certo punto). Si vede anche
che a pressione costante il momento delle forze agenti sul fluido è uguale a zero:
\begin{gather*}
    \oint _{\Sigma} \vv{r} \times (p\hat{n} ) \ d\sigma = 0  
\end{gather*}  


\begin{wrapfigure}{r}{0.4\textwidth}
    \centering
    \caption{La rotazione del fluido non cambia la risultante delle forze}
    \begin{tikzpicture}
        \draw[->](0, 0) -- (1, -0.7) node[at end, below] {$\hat{x}$ }; 
        \draw[->](0, 0) -- (-1, -0.7) node[at end, below] {$\hat{y}$ };
        \draw[->](0, 0) -- (0, 1) node[at end, left] {$\hat{z}$ };
        \draw[->](0, 0) -- (1.95, 0.7) node[midway, below] {$\vv{r}$ };
        \draw[->](1.95, 0.7) -- (1.95, 1.3)node[at end, right] {$\vv{\delta r}$ };
        \draw[cyan](0, 0) -- (1.95, 1.3);
        \draw[cyan](0.95, 0.35) arc (25:42:0.8) node[at end, above] {$\delta\alpha$};
        \draw(1.7, 1.2) .. controls (2.8, 2.15) and (3.2, 1.9) .. (3.4 , 1.8);
        \draw(3.4, 1.8) .. controls (4, 1.6) and (3.8, 1.1) .. (3.2, 1);
        \draw(3.2, 1) .. controls (2, 0) and (1.4, 1) .. (1.7, 1.2);
        \filldraw(0, 0) circle (1pt) node[anchor = north] {$\Omega$};
    \end{tikzpicture}    
\end{wrapfigure}
La conservazione della quantità di moto rappresenta l'invarianza delle leggi della
fisica nel movimento. La conservazione del momento angolare rappresenta che le leggi della
fisica sono invarianti rispetto alla rotazione. La conservazione dell'energia invece
implica che le leggi della fisica sono invarianti rispetto ai cambiamenti di tempo. 
Scegliendo come polo $\Omega$ l'origine del sistema di riferimento allora posso ruotare il fluido
di un certo angolo $\delta \alpha$ ottenendo allora un certo nuovo vettore
$\vv{r'} = \vv{r} + \delta \vv{r}$, ossia il nuovo vettore è dato dal vettore
\begin{gather*}
    \delta r = \delta\alpha \hat{\Omega} \times \vv{r} 
\end{gather*}   
Con $\hat{\Omega}$ si indica il versore dell'asse di rotazione.
\begin{gather*}
    \delta V = d\sigma \delta \alpha (\hat{\Omega} \times \vv{r}  ) \hat{n} \\
    \Delta V = \int_{\Sigma(V)} d\sigma \delta\alpha(\hat{\Omega} \times \vv{r}  ) \hat{n} \ \Longrightarrow \  \Delta V = \oint_{\Sigma} \delta\alpha\hat{\Omega} \cdot (\vv{r} \times \hat{n}  ) \ d\sigma   
\end{gather*}
Allora si ottiene con le proprietà del prodotto triplo e col fatto che in una rotazione, così come
in una traslazione, il volume non cambia, allora si ottiene la seguente
\begin{gather*}
    \Delta V = \delta a \hat{\Omega} \int(\vv{r} \times \hat{n}  ) \ d\sigma = 0
\end{gather*}
Si è appena dimostrato il \textbf{principio di Pascal}, il cui enunciato dice che: un sistema
isolato si dice all'equilibrio in cui le forze di volume
sono trascurabili e la pressione è costante.

\subsection{Torchio idraulico}
\begin{wrapfigure}{r}{0.3\textwidth}
    \centering
    \caption{}
    \begin{tikzpicture}
        \draw(0, 0) -- (3, 0) -- (3, 1) -- (2, 1) -- (2, 0.5) -- (0.5, 0.5) -- (0.5, 1) -- (0, 1) -- (0, 0);
        \draw[->](-0.2, 1) -- (-0.2, 0.5) node[at end, left] {$\vv{F_1}$};
        \draw[->](2.5,  1) -- (2.5, 2) node[at end, right] {$\vv{F_2}$};
    \end{tikzpicture}    
\end{wrapfigure}
Posso utilizzare i fluidi e la loro pressione per poter applicare più forza:
infatti la spinta che sente la macchina sulla destra sarà data dalla relazione di pressione
costante 
\begin{gather*}
    p = \frac{F_1}{\Sigma_1} = \frac{F_2}{\Sigma_2} \\
    F_2 = \frac{\Sigma_2}{\Sigma_1} F_1
\end{gather*}
Con il principio di Pascal si possono risolvere i problemi a pressione costante e 
nell'ipotesi in cui i volumi di fluidi coinvolti siano sufficientemente piccoli.

\subsection{Trovare l'equilibrio con le equazioni differenziali}
Avrò i contributi delle forze che agiscono sull'asse $z$ 
sono sia quelle di superficie che quelle di volume. 
\begin{align*}
    x &:  -(p(x + \Delta x, y, z)\Delta y \Delta z - p(x, y, z) \Delta y \Delta z) = 0 \\
    y &:  -(p(x , y + \Delta y, z)\Delta y \Delta z - p(x, y, z) \Delta y \Delta z) = 0 \\
    z &: \rho g\Delta x \Delta y \Delta z - (p(x, y z + \Delta z)\Delta x \Delta y - p(x, y, z) \Delta x \Delta y) = 0
\end{align*}
Si ottiene allora le seguenti relazioni:
\begin{align*}
    x &: \frac{-(p(x + \Delta x, y, z) - p(x, y, z))}{\Delta z} = 0 \\
    y &: \frac{-(p(x, y + \Delta y, z) - (p(x, y, z)))}{\Delta y} = 0 \\
    z &: \rho g - \frac{(p(x, y, z + \Delta z)\Delta x \Delta y - p(x, y, z))}{\Delta z} = 0
\end{align*}
Posso allora eseguire la derivata parziale rispetto alle singole coordinate
\begin{gather*}
    \left\{\begin{array}{l}
        -\frac{\partial p}{\partial x} = 0 \\
        -\frac{\partial p}{\partial y} = 0 \\
        \rho z - \frac{\partial p}{\partial z} = 0  
    \end{array}\right.
\end{gather*}
Le altri componenti non hanno le componenti $\rho g$ perché stiamo
considerando solamente l'asse $z$. In assenza di forze di volume, allora
la derivate rispetto sia ad $x$ che $y$ sono uguali a zero e dunque da qualunque parte io
la prenda $p$ non cambia. anche perché se io mi sposto di una certa quantità
infinitesima 
\begin{gather*}
    d\vv{r} = dx\hat{x} + dy\hat{y} + dz\hat{z}    
\end{gather*}
Anche se mi spostassi di un certo angolo qualunque la pressione sarebbe comunque
costante. 

\end{document}
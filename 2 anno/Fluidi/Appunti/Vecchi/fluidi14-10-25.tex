\documentclass[a4paper, oneside]{article}
\usepackage{graphicx}
\usepackage{amsthm}
\usepackage{amsmath}
\usepackage{amssymb}
\usepackage[a4paper,
            bindingoffset=0.2in,
            left=2cm,
            right=2cm,
            top=2cm,
            bottom=2cm,
            footskip=.25in]{geometry}
\usepackage[italian]{babel}
\usepackage{pgfplots}
\usepackage{tabularx}
\usepackage{tikz}
\usepackage{wrapfig}
\usepackage{color}
\usepackage[d]{esvect}
\definecolor{page}{rgb}{0.129,0.157,0.212}
\pagecolor{page}
\color{white}
\graphicspath{ {./images/} }
\usetikzlibrary{shapes.geometric}
\usetikzlibrary{datavisualization}
\usetikzlibrary{datavisualization.formats.functions}
\usetikzlibrary{patterns}
\pgfplotsset{width=10cm,compat=1.9}

\title{Appunti di fluidi}
\author{Tommaso Miliani}
\date{14-10-25}

\begin{document}
\newtheoremstyle{theoremEnv}
                {}          % Space above
                {}          % Space below
                {\slshape}  % Body font
                {}          % Indent amount
                {\bfseries} % Head font
                {.}         % Punctuation after head
                {\newline}         % Space after theorem head
                {}          % Theorem head spec
\theoremstyle{theoremEnv}

\newtheorem{definition}{Definizione}[section]
\newtheorem{theorem}{Teorema}[section]
\newtheorem{lemma}{Proposizione}[section]
\newtheorem{observation}{Osservazione}[section]
\newtheorem{corollary}{Corollario}[theorem]
\newtheorem{example}{Esempio}[section]

\maketitle

\section{}
Il lavoro adiabatico su di una curva qualsiasi $\gamma_1$ è lo stesso
che di quello per un'altra curva  $\gamma_2$ se le due trasformazioni
hanno lo stesso stato inziale e finale (anche se le trasformazioni non sono
quasi statiche). 
\begin{gather*}
    L_{i \to f, \gamma_1}^{(adiabatica)} = L_{i \to f, \gamma_2}^{(adiabatica)}  
\end{gather*}
C'è anche un altro aspetto su di questo: se questa trasformazione 
è anche ciclica, allora il suo lavoro è sempre zero. 
\begin{gather*}
    L_{i \to i}^{(adiabatica)} = 0
\end{gather*}
Queste conseguenze rispecchiano
i dati sperimentali anche se non si ha una legge della fisica classica che mi permetta di 
dire questo. Questo risultato in realtà ci dice moltissimo sulla natura: questo risultato
è un vincolo su quello che accade a livello di trasformazioni. E' un qualcosa che ci da un sacco
di informazioni sulla fisica.   \\
Se io avessi una quantità di lavoro adiabatico che compie il mio sistema 
e dipende solamente dallo stato inziale e finale, allora ci deve essere necessariamente
una funzione che mi permetta di esprimere il lavoro in termini dello stato iniziale 
e dello stato finale:
\begin{align}
    L_{i \to f}^{(adiabatica)} = F(i, f) = F(p_i, V_i, p_f, V_f) 
\end{align}
Inoltre questa deve essere tale che se $i \equiv f$ allora si annulla. Questa 
funzione assomiglia molto all'energia potenziale nelle forze conservative nella
meccanica (anche se non si sta parlando di energia potenziale). Si definisce
questa funzione come \textbf{Energia interna}, che è definita come 
\begin{gather*}
    \Delta U = U(f) - U(i) = -L_{i \to f}^{(adiabatica)} 
\end{gather*}
Questa non è energia potenziale anche se utilizza lo stesso simbolo, ma è
una funzione di stato. Si dimostra che è definita a meno di una costante
poiché se si definisce $U' = U +c$, nella differenza è possibile rimuovere la costante,
anche se la funzione $U$ non è univoca. In generale dipende solamente dallo stato iniziale e finale
poiché c'è la condizione $F(i, i) = 0$.

\section{Primo principio della termodinamica}
In generale non esiste una trasformazione adiabatica che mi permetta di passare
da uno stato finale $B$ ad uno stato inziale $A$, anche se esiste
una trasformazione adiabatica che mi permetta di passare da $A$ a $B$. 
Nel gas perfetto, per esempio, so che esiste sempre una trasformazione 
adiabatica che mi permetta di passare da 
\begin{gather*}
    A(V_1, T) \to B(V_2, T) \qquad V_2 > V_1
\end{gather*}
anche se non esiste una trasformazione adiabatica che mi permetta di tornare 
indietro (in altre parole non esiste al compressione adiabatica poiché
la temperatura di un gas si alza sempre quando viene compresso).
Si può dimostrare che per solo per un caso particolare esistono
trasformazioni termodinamiche adiabatiche per entrambi i
sensi supponendo che si abbiano due trasformazioni adiabatiche
in entrambi i sensi:
\begin{gather*}
    A \to B \qquad B \to A
\end{gather*}
L'energia interna del sistema per le due trasformazioni sarà data:
\begin{gather*}
    U(B) - U(A) = -L_{A \to B}^{(adiabatica)} \\
    U(A) - U(B) = -L_{B \to A}^{(adiabatica)} 
\end{gather*}
Ossia, quando esistono entrambe le possibilità il lavoro che va da
uno stato all'altro è esattamente l'opposto del lavoro che
va dall'altro stato a quello di partenza:
\begin{gather*}
    L_{A \to B} = -L_{B \to A}
\end{gather*}
Tutte le volte che non esiste una trasformazione adiabatica in un verso, esisterà
nel verso opposto per qualsiasi stato inziale e finale e per qualsiasi trasformazione 
termodinamica. Il resto del primo principio della termodinamica è che il lavoro
del sistema non dipende dal set di trasformazioni che compie il sistema per
raggiungere uno stato finale qualsiasi da uno stato iniziale qualsiasi.
In generale posso dire che se il lavoro è esprimibile come funzione
di stato (ossia in condizioni adiabatiche) che
\begin{align}
    \Delta U = -L^{(adia)} 
\end{align}
Dato che l'energia interna del sistema è proprio il lavoro del sistema,
allora il lavoro che compie il sistema corrisponde alla diminuzione 
dell'energia all'interno del sistema e che può essere usata per compiere lavoro
meccanico sull'ambiente esterno. 


Se il lavoro non è compiuto da una trasformazione adiabatica, allora $U$ non è una funzione di stato,
allora è esprimibile sempre a prescindere dalle trasformazioni del sistema e dai suoi stati. Inoltre, la somma
tra l'energia potenziale ed il lavoro del sistema è definita come una quantità
chiamata \textbf{quantità di calore}:
\begin{align}
    \Delta U + L = Q
\end{align} 
Quando si analizza una trasformazione qualunque si ha il calore in modo univoco 
(e senza dover sapere gli stati del sistema) e dunque
posso riesprimere l'energia interna come
\begin{align}
    Q - L = \Delta U
\end{align}
E prende il nome di funzione matematica integrale del primo principio della termodinamica. 
\begin{itemize}
    \item $Q > 0$: Il sistema ha acquisito dell'energia dall'ambiente;
    \item $Q < 0$: Il sistema ha ceduto dell'energia all'ambiente.
\end{itemize}
Un sistema può scambiare con l'esterno o lavoro meccanico o lavoro.
Si può esprimere come conseguenza la forma differenziale dl primo principio
della termodinamica:
\begin{gather*}
    \delta Q - \delta L = dU
\end{gather*}
A livello matematico è notevole in quanto le due espressioni sono 
legate tra di loro.  
\begin{itemize}
    \item Adiabatica: per definizione, una trasformazione adiabatica
    ha quantità di calore zero: le pareti adiabatiche allora non permettono il  
    passaggio di energia di tipo termico e permettono solamente gli scambi energetici di tipo meccanico.
    \item Trasformazione ciclica: $\Delta U = 0$, allora $Q - L = 0$ e $Q = L$. La quantità
    di calore fornito deve equivalere al lavoro svolto dal sistema. Il lavoro sarà positivo se percorro
    la curva in senso orario o negativo se la percorro in senso antiorario.
\end{itemize}
Dato che si è data una definizione operativa di calore secondo il primo principio, posso dire che
la quantità di calore che scambio in una certa trasformazione, per definizione, è
data da:
\begin{gather*}
    Q_{A \to B} = \Delta U + L_{A \to B} \qquad \Delta U = U(B) - (A) 
\end{gather*} 
A pressione costante se si fornisse del calore senza che aumenti il volume
del sistema, allora posso dire che
\begin{gather*}
    Q = \Delta U \qquad L \approx 0
\end{gather*}

\subsection{Il mulinello di Joule}
L'esperimento che fece Joule per definire il calore utilizzava un contenitore
con all'interno un sistema di pale e sopra un mulinello collegato alle
pale con dei pesi per ogni parte. L'energia potenziale dei pesi viene trasferita al sistema
sottoforma di attrito delle pale con l'acqua, in modo tale che tutta l'energia che 
viene fornita al sistema è dovuta solo al calore e non al lavoro 
per cui
\begin{gather*}
    Q = \Delta U \ \Longrightarrow \ Q \propto \Delta T
\end{gather*} 
Il che mi suggerisce che
\begin{gather*}
    \delta Q = c\ dt
\end{gather*}
con $\mathcal{C}$ si indica la \textbf{capacità termica} che è definita come la quantità di  calore infinitesima
che devo fornire al sistema per ottenere un aumento di temperatura infinitesimo. Tuttavia
questo non è applicabile per tutti gli esprimenti (ossia quelli nei quali
si fornisce solamente calore al sistema senza fare alcun tipo di lavoro meccanico).
Dato che il calore fornito dipende dal tipo di trasformazione, 
allora devo definire la capacità termica quando fisso la pressione (per le
isobare) e quando fisso il volume (per le isocore)  in maniera diversa:
\begin{gather*}
    \mathcal{C}_p = \left(\frac{\delta Q}{dt}\right)_p \qquad c_V = \left(\frac{\delta Q}{dt}\right)_V
\end{gather*}
La capacità termica adiabatica è zero per qualsiasi sistema, mentre la capacità
termica di una trasformazione non è definita. 
Generalmente per i liquidi $\mathcal{C}_P \approx c_V$ mentre per i gas sono molto diversi. 
Si definisce il \textbf{calore specifico} come
\begin{align}
    c = \frac{\mathcal{C}}{M}
\end{align}
Ossia il calore per quantità di massa. Si può anche definire
il calore per quantità di sostanza, come 
\begin{align}
    c = \frac{\mathcal{C}}{n}
\end{align}
Ossia la capacità termica per numero di mole. IN generale, dato che è una quantità
additiva, si ha che $\mathcal{C} \geq 0$.


\end{document}
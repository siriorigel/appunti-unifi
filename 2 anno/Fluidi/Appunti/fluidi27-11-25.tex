\documentclass[a4paper, oneside]{article}
\usepackage{graphicx}
\usepackage{amsthm}
\usepackage{amsmath}
\usepackage{amssymb}
\usepackage[a4paper,
            bindingoffset=0.2in,
            left=2cm,
            right=2cm,
            top=2cm,
            bottom=2cm,
            footskip=.25in]{geometry}
\usepackage[italian]{babel}
\usepackage{pgfplots}
\usepackage{tabularx}
\usepackage{tikz}
\usepackage{wrapfig}
\usepackage{color}
\usepackage[d]{esvect}
\usepackage{chemfig}
\usepackage{mhchem}
\definecolor{page}{rgb}{0.129,0.157,0.212}
\pagecolor{page}
\color{white}
\graphicspath{ {./images/} }
\usetikzlibrary{shapes.geometric}
\usetikzlibrary{datavisualization}
\usetikzlibrary{datavisualization.formats.functions}
\usetikzlibrary{patterns}
\pgfplotsset{width=10cm,compat=1.18}

\title{Appunti Fisica statistica}
\author{Tommaso Miliani}
\date{27-11-25}

\begin{document}
\newtheoremstyle{theoremEnv}
                {}          % Space above
                {}          % Space below
                {\slshape}  % Body font
                {}          % Indent amount
                {\bfseries} % Head font
                {.}         % Punctuation after head
                {\newline}  % Space after theorem head
                {}          % Theorem head spec
\theoremstyle{theoremEnv}

\newtheorem{definition}{Definizione}[section]
\newtheorem{theorem}{Teorema}[section]
\newtheorem{lemma}{Proposizione}[section]
\newtheorem{observation}{Osservazione}[section]
\newtheorem{corollary}{Corollario}[theorem]
\newtheorem{example}{Esempio}[section]
\newtheorem{remark}{Enunciato}[section]

\maketitle

\section{Ripassino}
L'energia totale del sistema si può scrivere come la somma delle energia di tutti i
sottosistemi e dunque $\epsilon$ dipende solamente dagli stati di ogni 
sottosistema. Si può verificare che la probabilità che si verifichi
un certo stato per un singolo sottosistema come
\begin{align}
    \prod = \frac{e^{-\beta\epsilon_i}}{\sum_{\alpha} e^{-\beta \epsilon_\alpha}}
\end{align}
E dunque l'energia totale come
\begin{align}
    E = \sum_{i = 1}^{N}\epsilon_i 
\end{align}
Se si volesse calcolare il contributo all'energia media totale si utilizzerebbe una distribuzione 
del seguente tipo (con un determinato valore atteso del contributo dell'energia 
di ognuno dei sottosistemi):
\begin{align}
    \left< \epsilon \right> = \frac{\sum \epsilon_i e^{-\beta \epsilon_i}}{\sum e^{-\beta\epsilon_i}} 
\end{align}

\section{Somma dei contributi energetici interni e determinazione della funzione energetica}
Una cosa utile è che la somma sopra è una somma su tutti i possibili stati microscopici
e dunque può essere utile cercare di passare da una somma degli stati ad una somma sui singoli valori dell'energia.
La somma sugli stati si esprime come
\begin{gather*}
    \sum e^{-\beta\epsilon_i}
\end{gather*}
Invece di sommare solamente i singoli stati, posso sommare rispetto a tutti i 
contributi energetici introducendo una funzione $g(\epsilon)$ che conta per ogni 
valore dell'energia quanti sono gli stati microscopici che hanno quel determinato
valore dell'energia:
\begin{gather*}
    \sum g(\epsilon)e^{-\beta\epsilon_i}
\end{gather*}
Se invece di avere una somma continua di stati o di energie, ad un integrale su i possibili valori 
dell'energia in funzione della funzione $g(\epsilon)$, come
\begin{gather*}
    \sum g(\epsilon)e^{-\beta\epsilon_i} \to \int_{\epsilon_{min}}^{\epsilon_{max}} g(t)e^{-\beta\epsilon} \ d\epsilon 
\end{gather*}
La funzione $g(\epsilon) \ d\epsilon$ mi descriver il numero di stati nei quali 
l'energia varia tra $\epsilon$ e $\epsilon + d\epsilon$ e prende il nome di 
funzione di densità degli stati: è infatti una densità di energia . Dunque si può ottenere il valore 
atteso dell'energia dei singoli sottosistemi come
\begin{gather*}
    \left< \epsilon \right> = \frac{\int \epsilon g(\epsilon) e^{-\beta\epsilon} \ d\epsilon }{\int g(\epsilon) e^{-\beta \epsilon} } 
\end{gather*}
Si osserva, da questa funzione, che ci siamo ricondotti a integrali definiti in funzioni ad una sola variabile. Questo
è un problema che si risolve sempre (o si risolve analiticamente o con un calcolatore): quando l'integrale è
ben definito allora lo si sa sempre calcolare e dunque l'energia sui sottosistemi è risolto purché io 
riesca a determinare la funzione $g(\epsilon)$,anche se non si deve conoscere totalmente in quanto si deve
solo conoscere il suo valore a meno di una costante.

\section{Determinazione della funzione $g(\epsilon)$ quando la funzione energetica dipende 
quadraticamente dalla variabile microscopica}
E' possibile determinare questa funzione ponendoci nel caso in cui il singolo
sottosistema abbia come energia 
\begin{gather*}
    \epsilon = \frac{1}{2}mv_x^{2}
\end{gather*} 
Posso definire, nello spazio della velocità $v_x$ il numero di stati, e dunque lo stato 
infinitesimo all'interno dello spazio delle fasi è dato da
\begin{gather*}
    dN = \frac{dv_x}{\Delta v}
\end{gather*}
E dunque posso esprimere la funzione $g(\epsilon)$ come
\begin{gather*}
    g(\epsilon)d\epsilon = \frac{dN}{d\epsilon}d\epsilon = \frac{1}{\Delta v}\frac{dv_x}{d\epsilon}d\epsilon
\end{gather*}
Adesso non mi resta che calcolare la derivata della funzione $v_x$ nella seguente maniera:
\begin{gather*}
    v_x = \sqrt{\frac{2\epsilon}{m}} \ \Longrightarrow \ \frac{dx}{d\epsilon} = \frac{1}{d\epsilon}\sqrt{\frac{2\epsilon}{m}} = \frac{1}{\sqrt{2m\epsilon} }  
\end{gather*}
E allora si è determinata la funzione $g(\epsilon)$ come
\begin{gather*}
    g(\epsilon) = \frac{1}{\Delta v \sqrt{2m} }\epsilon^{-\frac{1}{2}} + c
\end{gather*}
Si può quindi determinare il valore atteso di $\epsilon$:
\begin{gather*}
        \left< \epsilon \right> = \frac{\int \epsilon^{\frac{1}{2}}  e^{-\beta\epsilon} \ d\epsilon }{\int \epsilon^{-\frac{1}{2}} e^{-\beta \epsilon} } 
\end{gather*}
Risolvendo gli integrali:
\begin{gather*}
    \left< \epsilon \right> = \frac{\frac{\sqrt{\epsilon} }{2\beta^{\frac{3}{2}}} }{\frac{\sqrt{\epsilon} }{\sqrt{\beta} }} = \frac{1}{2\beta} = \frac{1}{2}k_B T
\end{gather*}
Se invece si volesse calcolare il contributo dell'energia media degli oscillatori
armonici (e dunque l'energia non in funzione del tempo ma dello spazio), dovrei utilizzare una
funzione energetica come
\begin{gather*}
    \epsilon = \frac{1}{2}kx^{2}
\end{gather*}
E dunque dire che
\begin{gather*}
    dN = \frac{dx}{\Delta x}
\end{gather*}
Dal punto di vista matematico questo calcolo è lo stesso di quello fatto prima, purché
posso sostituire i dati che si hanno ora nelle espressioni di prima, posso dunque scrivere
direttamente il risultato:
\begin{gather*}
    g(\epsilon) = \frac{1}{\Delta x \sqrt{2k} }\epsilon^{-\frac{1}{2}}
\end{gather*}
E dunque si ottiene di nuovo lo stesso risultato
\begin{gather*}
    \left< \epsilon \right> = \frac{1}{2}k_B T 
\end{gather*}
Nel caso di un corpo rigido che ruota intorno ad un asse,
il contributo all'energia di questo corpo è dato da
\begin{gather*}
    \epsilon = \frac{1}{2}I_c \omega^{2}
\end{gather*}
Questo è uguale all'esempio di prima purché si sostituiscano le costanti:
si trova dunque che
\begin{gather*}
    g(\epsilon) = \frac{1}{\Delta \omega \sqrt{2I} }\epsilon^{-\frac{1}{2}}
\end{gather*}
Ottenendo lo stesso contributo energetico
\begin{gather*}
    \left< \epsilon \right> = \frac{1}{2}k_B T 
\end{gather*}
Questi tre casi hanno in comunque il fatto che la dipendenza dalla variabile microscopica è 
al quadrato, dunque tutte le volte che si ha un sistema nel quale le variabili 
microscopiche contribuiscono in maniera additiva e quadratica allora si ha lo stesso
contributo energetico come
\begin{gather*}
    \left< \epsilon \right> = \frac{1}{2}k_B T 
\end{gather*}

\section{Teorema di equipartizione dell'energia e spiegazione capacità termiche
per i gas mono, di e poli atomici}
Il teorema di equipartizione non vale solamente se i contributi
delle variabili microscopiche sono quadratiche, ma per semplicità
si tratterà solamente il caso in cui la dipendenza dalla funzione energetica
sia quadratica e additiva, allora ha la seguente espressione
\begin{align}
    \left< \epsilon \right> \frac{1}{2}k_B T 
\end{align}
Questo risultato permette di risolvere il mistero dei calori specifici
dei gas perfetti e perché dipenda dal fatto che siano monoatomici, diatomici o poliatomici.
Il caso monoatomico è tale che
\begin{gather*}
    E = \sum k_i \qquad k_i = \frac{1}{2}m(\vv{v} )^{2} = \frac{1}{2}m(v_x^{2} + v_y^{2} + v_z^{2})_i
\end{gather*}
Infine si trova, nel caso dell'energia interna è dato 
\begin{gather*}
    U = \left< E \right> = \sum k_i = N \left< k \right> = N \frac{3}{2}k_B T  
\end{gather*}
Si è trovato che l'energia interna del gas monoatomico è dato proprio dall'espressione sopra,
dunque la capacità termica di un gas monoatomico è data esattamente da
\begin{gather*}
    \mathcal{C}_V = \frac{dU}{dT} = \frac{3}{2}Nk_B
\end{gather*}
Nel caso in cui si abbia numero di moli $\mathcal{N}= 1$, allora si può dire che
\begin{gather*}
    \mathcal{C}_V = \frac{3}{2}R
\end{gather*} 
L'estensione degli atomi fisicamente non ci interessa (anche se hanno una estensione
ben definita) poiché la massa totale sta concentrata in una regione di spazio che è 5 volte più
piccola dell'effettiva estensione dell'atomo. Si può dunque pensare che la massa totale possa
essere tutta concentrata all'interno del nucleo per una molecola monoatomica (in buona approssimazione).

\begin{wrapfigure}{r}{0.4\textwidth}
    \centering
    \caption{Modellizazione delle molecole diatomiche}
    \begin{tikzpicture}
        \draw(0, 0) circle (0.75);
        \draw(3, 0) circle (0.4);
        \draw[->](-1, 0) -- (4, 0) node[at end, below] {$x$};
        \filldraw(0, 0) circle (1pt);
        \filldraw(3, 0) circle (1pt);
        \filldraw(1, 0) circle (1pt);
        \draw[->](1, -1) -- (1, 1) node[at end, left] {$y$};
        \node at (0, -1) {$m_1$};
        \node at (3, -1) {$m_2$};
    \end{tikzpicture}    
\end{wrapfigure}
Per una molecola diatomica invece, questo oggetto è dato da due atomi e 
dunque in prima approssimazione posso considerala come un corpo rigido:
oscillando le molecole, in realtà non è ottima l'approssimazione di corpo rigido
ma solamente se l'energia interna è piccola in quanto con poca energia gli atomi 
non oscillano molto. Si può dunque considerare la particella diatomica come se
fossero due sferette collegate da una molla (con costante elastica molto grande 
approssimabile ad una sbarretta rigida) e massa zero. Dunque devo anche
considerare i termini rotazionali delle molecole intorno al centro di massa:
\begin{gather*}
    K = K_{CM} + K_{ROT} = \frac{1}{2}m(v_x^{2} + v_y^{2} + v_z^{2}) + \frac{1}{2}\left(I_x\omega_x^{2} + I_y\omega_y^{2} + I_z\omega_z^{2}\right)
\end{gather*} 
I contributi energetici rotazionali per tutti gli assi non sono dello stesso ordine:
$I_x$ è molto più piccolo degli altri due e dunque l'energia cinetica è data
da 5 pezzi quadratici additivi, ossia
\begin{align}
    U = N\frac{5}{2}k_B T
\end{align}
E dunque vale
\begin{align}
    \mathcal{C}_V = \frac{5}{2}R
\end{align}

Nel caso di un atomo poliatomico, non si ha più un asse rispetto al 
quale il momento di inerzia è molto minore degli altri due, dunque devo tenere anche conto
della terza componente ottenendo esattamente per l'energia 
interna del sistema
\begin{align}
    U = 3Nk_B T
\end{align}
E dunque la capacità termica è data esattamente come
\begin{align}
    \mathcal{C}_V = 3R
\end{align}
Tuttavia, il contributo rispetto all'asse che passa dal centro di massa
è trascurabile nel caso in cui le molecole poliatomiche è trascurabile
e dunque ci si riconduce nel caso precedente come nel caso
della $\ce{CO2}$ che è analogo all'energia della $\ce{CO}$. 

\section{Schematizzazione energia del solido}
E' possibile schematizzare l'energia interna di un solido come 
l'energia potenziale di tanti oscillatori armonici in quanto,
quando si fornisce molta energia ad un reticolo, le molecole iniziano ad
oscillare e dunque modificano la distanza tra di loro e la distanza
dagli assi di rotazione (nel caso di e poli atomico). Dunque si
ottiene l'energia attesa delle molecole come
\begin{gather*}
    E = \sum \frac{1}{2}(v_x^{2} + v_y^{2} + v_z^{2})_i + \frac{1}{2}(k_xx_i^{2} + k_yy_i^{2} + k_zz_i^{2})
\end{gather*}
Allora l'energia interna del sistema è la somma di tutti i valori attesi dell'energia:
\begin{align}
    U = \left< E \right> = 3Nk_B T  
\end{align}
Allora la capacità termica di un solido è data da
\begin{align}
    \mathcal{C}_V \approx 3Nk_B = 3R
\end{align}
Per una mole di solido. Questa è nota come legge di Dulong-Petit. Questa
è valida per temperature intermedie, infatti per temperature  molto alte,
le molecole non si comportano più come oscillatori armonici ma le vibrazioni
diventano molto grandi. A basse temperatura invece gli effetti quantistici
diventano molto grandi e non è più valida. 

\section{Entropia in un sistema}
Dato che si è detto che per ogni sistema l'entropia è tale che
\begin{gather*}
    S = Ns
\end{gather*}
Dove $s$ è l'entropia dei singoli sottosistemi (atomi). Nella pratica
questa funzione è meglio definita se $N$ è molto grande. SI pul dunque deteminare
$s$:
\begin{align}
    s = -k_B \sum \pi_k \ln \pi_k
\end{align}
Dove si ha
\begin{gather*}
    \pi_k = \frac{e^{-\beta \epsilon_k}}{\sum_{i} e^{-\beta\epsilon_i}} = \frac{1}{z}e^{-\beta \epsilon_k} 
\end{gather*}
Si ha dunque che
\begin{gather*}
    s = -k_B \sum \pi_k (\ln z + \beta \epsilon_k) = k_B \ln z \sum \pi_k + \beta k_B \sum \epsilon_k \pi_k = \\
    k_B(\ln z + \beta \left< \epsilon \right> )
\end{gather*}
Dato però, che per un gas monoatomico si ha che $\left< \epsilon \right>  = \frac{3}{2}k_B T = \frac{3}{2\beta}$,
si ha che il contributo all'entropia di una singola particella è data esattamente da
\begin{gather*}
    s = k_B(\ln z + \frac{3}{2})
\end{gather*}
E allora il contributo totale all'entropia è data da
\begin{gather*}
    S = Ns = Nk_B(\ln z + \frac{3}{2})
\end{gather*}
Rimane dunque da definire solamente $\ln z$: questo si può fare
attraverso la seguente funzione:
\begin{gather*}
    z = \sum e^{-\beta \epsilon_k} = \int \frac{dx dy dz \ dv_x dv_y dv_z}{(\Delta x \Delta v)^{3}}e^{-\beta \epsilon(x, v)} 
\end{gather*}
Dato che dipende solamente dalla velocità, si deve necessariamente avere che
\begin{gather*}
    \int_{V}^{} \frac{dx dy dz}{(\Delta x)^{3}} \int \frac{d v_x dv_y dv_z}{(\Delta V)^{3}} e^{-\beta\frac{1}{2}m(v_x^{2} + v_y^{2} + v_z^{2})}
\end{gather*}
Il primo integrale è facile in quanto è solamente il seguente integrale:
\begin{gather*}
    \frac{V}{(\Delta x)^{3}} \left(\int_{-\infty }^{+ \infty } \frac{d\eta}{\Delta v}e^{-\frac{1}{2}\beta\eta^{2}} \right)^{3}
\end{gather*}
Allora, con la seguente sostituzione si ha che
\begin{gather*}
    x^{2}  =\frac{1}{2}m\beta\eta^{2} \qquad \frac{1}{\Delta v}\sqrt{\frac{2}{m\beta}}\int_{-\infty }^{+\infty } e^{-x^{2}} dx = \frac{1}{\Delta v} \sqrt{\frac{2\pi}{m\beta}}   
\end{gather*}
Si ottiene dunque che
\begin{gather*}
    z = \frac{V}{(\Delta x)^{3}} \frac{1}{(\Delta v)^{3}} \left(\frac{2\pi}{m\beta}\right)^{-\frac{3}{2}}
\end{gather*}
Allora si ottiene che 
\begin{gather*}
    \ln z = \ln \frac{V}{(\Delta x)^{3}} + \frac{3}{2}\ln \left(\frac{2\pi k_B}{m(\Delta v)^{2}}T\right)
\end{gather*}
Si definisce 
\begin{gather*}
    \frac{m(\Delta v)^{2}}{2\pi k_B} = T_0 \qquad (\Delta x)^{3} = V_0
\end{gather*}
E dunque l'entropia totale è data da
\begin{align}
    S = Nk_B\left(\ln \left(\frac{V}{V_0}\right) + \frac{3}{2}\ln\left(\frac{T}{T_0}\right) + \frac{3}{2}\right)
\end{align}
Questa espressione, che sembra valida, in realtà è molto sbagliata. 


\end{document}
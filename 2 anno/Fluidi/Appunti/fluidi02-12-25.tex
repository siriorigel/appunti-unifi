\documentclass[a4paper, oneside]{article}
\usepackage{graphicx}
\usepackage{amsthm}
\usepackage{amsmath}
\usepackage{amssymb}
\usepackage[a4paper,
            bindingoffset=0.2in,
            left=2cm,
            right=2cm,
            top=2cm,
            bottom=2cm,
            footskip=.25in]{geometry}
\usepackage[italian]{babel}
\usepackage{pgfplots}
\usepackage{tabularx}
\usepackage{tikz}
\usepackage{wrapfig}
\usepackage{color}
\usepackage[d]{esvect}
\usepackage{chemfig}
\usepackage{mhchem}
\definecolor{page}{rgb}{0.129,0.157,0.212}
\pagecolor{page}
\color{white}
\graphicspath{ {./images/} }
\usetikzlibrary{shapes.geometric}
\usetikzlibrary{datavisualization}
\usetikzlibrary{datavisualization.formats.functions}
\usetikzlibrary{patterns}
\pgfplotsset{width=10cm,compat=1.18}

\title{Appunti di Fluidodinamica}
\author{Tommaso Miliani}
\date{02-12-25}

\begin{document}
\newtheoremstyle{theoremEnv}
                {}          % Space above
                {}          % Space below
                {\slshape}  % Body font
                {}          % Indent amount
                {\bfseries} % Head font
                {.}         % Punctuation after head
                {\newline}  % Space after theorem head
                {}          % Theorem head spec
\theoremstyle{theoremEnv}

\newtheorem{definition}{Definizione}[section]
\newtheorem{theorem}{Teorema}[section]
\newtheorem{lemma}{Proposizione}[section]
\newtheorem{observation}{Osservazione}[section]
\newtheorem{corollary}{Corollario}[theorem]
\newtheorem{example}{Esempio}[section]
\newtheorem{remark}{Enunciato}[section]

\maketitle

\section{Flusso di massa}
La proprietà della conservazione della massa si può applicare
ad un volume finito , allora la massa totale di questo volume è
dato dall'integrale
\begin{gather*}
    M = \int_{V}^{} \rho \ dV
\end{gather*}
L'unico modo per variare la massa è che questa entra o esca
dal volume, allora posso dire che la variazione della massa è
esattamente il flusso di massa:
\begin{gather*}
    \dot{M} = -\int_{\Sigma (V)}^{} \rho \vv{u} \cdot \hat{n} \ d\sigma   
\end{gather*}

\begin{wrapfigure}{r}{0.4\textwidth}
    \centering
    \caption{}
    \begin{tikzpicture}
        \draw[->](0, 0) -- (3, 0) node[at end, below] {$z$};
        \draw[->](0, 0) -- (0, 2) node[at end, left] {$y$};
        \draw[->](0, 0) -- (-1, -1) node[at end, below] {$x$};
        \draw(-0.5, -0.5) -- (0.5, -0.5) -- (1, 0) -- (1, 1) -- (0.5, 0.5) -- (-0.5, 0.5) -- (-0.5, -0.5);
        \draw(0.5, -0.5) -- (0.5, 0.5);
        \draw(-0.5, 0.5) -- (0, 1) -- (1, 1);
    \end{tikzpicture}    
\end{wrapfigure}
Ossia tramite il vettore normale alla superficie, si può anche riscrivere
questa equazione in forma differenziale ragionando su di una piccola porzione
di volume (a piacere), dunque si può prendere un volumetto 
qualsiasi all'interno del fluido e se ne misura la variazione di massa. Dato
che il volumetto infinitesimo è definito come $dV = dx dy dz$, mentre
la massa è data da $m = \rho\ dx dy dz $.  Dunque il flusso di massa sarà dato 
da
\begin{gather*}
    \dot{M} = \frac{\partial \rho}{\partial t} \ dx dy dz  
\end{gather*}
Iniziando con la componente $x$, posso definire il flusso per le singole
componenti, dunque il flusso totale per le $x$ è 
\begin{gather*}
    \phi_x = \left((\rho u_x) \cdot (x + dx, y, z) - (\rho u_x)\cdot (x, y, z)\right) dy dz 
\end{gather*}
Per le altre due componenti è
\begin{gather*}
    \phi_y = \left((\rho u_y) \cdot (x, y + dy, z) - (\rho u_y) \cdot (x, y, z)\right) dx dz \\
    \phi_z = \left((\rho u_z) \cdot (x, y, z + dz) - (\rho u_z) \cdot (x, y, z)\right) dx dy
\end{gather*}
Adesso posso considerare il flusso totale per il piccolo volumetto:
\begin{gather*}
    \frac{\partial f}{\partial t} dx dy dz = -  \left((\rho u_x) \cdot (x + dx, y, z) - (\rho u_x)\cdot (x, y, z)\right) dy dz + \\
     - \left((\rho u_y) \cdot (x, y + dy, z) - (\rho u_y) \cdot (x, y, z)\right) dx dz - \left((\rho u_z) \cdot (x, y, z + dz) - (\rho u_z) \cdot (x, y, z)\right) dx dy
\end{gather*}
semplificando si ottiene la derivata parziale di $\frac{\partial \rho u_x}{\partial dx}$ e lo
stesso per le altre due componenti, allora si può riscrivere l'intero flusso di massa
come
\begin{gather*}
    \frac{\partial \rho}{\partial t}  = - \frac{\partial }{\partial x} \rho u_x - \frac{\partial }{\partial y}\rho u_y - \frac{\partial }{\partial z} \rho u_z   
\end{gather*} 
Ossia questo è l'operatore divergenza:
\begin{gather*}
    \frac{\partial \rho}{\partial t} + \vv{\nabla} \cdot (\rho \vv{u} ) = 0  
\end{gather*}
Ottenendo :
\begin{align}
    \left\{\begin{array}{l}
                \rho \left(\frac{\partial \vv{u} }{\partial t} + (\vv{u} \cdot  \vv{\nabla}) \vv{u}  \right) = -\vv{\nabla} p + \rho \vv{g}  \\
        \frac{\partial \rho}{\partial t} + \vv{\nabla} \cdot (\rho \vv{u} ) = 0  
    \end{array}\right.
\end{align}
Dunque si è messo a sistema anche il flusso di massa: 

\section{Risolvere il problema della quinta incognita}
Si potrebbe pensare al principio di energia, si dovrebbe pensare
a due energie diverse (l'energia cinetica) e l'energia interna del fluido,
se non si sa però come il piccolo elemento termodinamico alla posizione $x, y, z$ comunica
con quelli circostanti, non si può determinare la conservazione dell'energia
e non si sa che tipo di trasformazione compie l'elemento di volume. 
Si deve dunque scrivere le proprietà termodinamiche del sistema,
ossia la legge (barotropica), dunque si definisce la seguente funzione
\begin{gather*}
    p= p(\rho)
\end{gather*}
Ossia una legge barotropica, supponendo un fluido ideale (gas) e, dato che
in una trasformazione adiabatica si conserva 
\begin{gather*}
    pV^{\gamma} = \text{const}
\end{gather*}
Questo ragionamento si può fare anche per l'elemento infinitesimo
e si può supporre che l'elemento fluido non scambi calore con gli 
elementi circostanti, si può dunque dire che, dato che ha una massas 
infinitesima,
\begin{gather*}
    p\frac{dm}{\rho}^{\gamma } = \text{const}
\end{gather*}
E dunque
\begin{gather*}
    p\left(\frac{dm}{\rho}\right)^{\gamma} = p_0 \left(\frac{dm}{\rho_0}\right)^{\gamma}
\end{gather*}
E dunque si ottiene una relazione per la pressione 
\begin{gather*}
    p = p_0\left(\frac{\rho}{\rho_0}\right)^{\gamma}
\end{gather*}
Di fatto questa legge politropica corrisponde all'ipotesi che si compie se la
trasformazione dell'elemento fluido si evolve adiabaticamente se
$\gamma$ è il rapporto dei calori specifici.  Il caso isotermo è 
semplicemente $\gamma = 1$: se non si hanno perdite di calori si ha
che
\begin{gather*}
    T \frac{dS}{dt} = 0
\end{gather*} 
Si ha dunque il sistema completo per un fluido comprimibile. Per un fluido incomprimibile
invece la situazione è ancora più semplice: l'elemento fluido conserva la sua densità, dunque
\begin{gather*}
    \frac{d \rho}{d t} = \frac{\partial \rho}{\partial t} \sum_{k = 1}^{3} u_k \frac{\partial \rho}{\partial x_ k} = 0   
\end{gather*}
Ma questo non vuol dire che la sua densità sia costante: infatti questo 
sarebbe un caso particolare della conservazione della densità: infatti nel caso 
di olio e acqua, l'elemento di fluido ha sempre la stessa densità anche se la densità 
non è mai costante poiché olio e acqua non si mischiano. L'ipotesi 
di fluido incomprimibile mi permette allora di dire che,
sfruttando l'ipotesi di continuità,
\begin{gather*}
    \frac{\partial \rho}{\partial t} + \sum_{k = 1}^{3} \frac{\partial }{\partial x_k} (\rho u_k) = \frac{\partial \rho}{\partial t}  + \sum_{k = 1}^{3} \left( u_k \frac{\partial \rho}{\partial x_k} + \rho \frac{\partial u_k}{\partial x_k}\right)  
\end{gather*}
Adesso si può riscrivere come
\begin{gather*}
    \frac{d\rho}{dt} + \rho \sum_{k = 1}^{3} \frac{\partial u_k}{\partial x_k}  = \frac{d\rho}{dt} + \rho \vv{\nabla} \cdot \vv{u}  
\end{gather*}
Ossia la stessa equazione nel sistema di prima. Adesso,questa coincide con la divergenza di
$\vv{u}$ uguale a zero, dunque per il fluido incomprimibile
\begin{gather*}
    \left\{\begin{array}{l}
        \frac{d\rho}{dt} = \frac{\partial \rho}{\partial t} + \sum_{k = 1}^{3} u_k\frac{\partial \rho}{\partial x_k} = 0 \\
        \vv{\nabla}\cdot \vv{u} = 0     
    \end{array}\right.
\end{gather*} 

\section{Condizioni stazionarie}
In condizioni stazionarie si possono fare diverse considerazioni, 
prima di tutto posso dire che
\begin{gather*}
    \vv{\nabla}(\rho \cdot \vv{u} ) = 0 
\end{gather*}
Prendendo dunque un volume finito e facendone l'integrale della divergenza,
posso ottenere
\begin{gather*}
    \int_{V}^{} \vv{\nabla} (\rho \cdot \vv{u} ) \ dV  = 0 = \int_{\Sigma(V)}^{} \rho \vv{u} \cdot \hat{n} \ d\sigma     
\end{gather*}


\begin{wrapfigure}{r}{0.4\textwidth}
    \centering
    \caption{Tubo di flusso}
    \begin{tikzpicture}
        \draw[->](0, 0) .. controls (1, 0.25) and (2, -0.5) .. (3, -0.5);
        \draw[->](0, -1) -- (3, -1);
        \draw[->](0, -2) .. controls (1, -2.25) and (2, -1.5) .. (3, -1.5);
        \draw(0, -1) ellipse (0.25 and 1);
        \draw(3, -1) ellipse (0.2 and 0.5);
        \filldraw(0, -2) circle(1pt) node[anchor = north] {$C$};
        \filldraw(3, -1.5) circle(1pt) node[anchor = north] {$C'$};
        \node at (2, 0) {$\vv{u}$ };
    \end{tikzpicture}    
\end{wrapfigure}
Posso trasformarlo in un integrale di volume tramite l'integrale
di Gauss (oppure scomporlo come si era fatto prima in un volumetto).
In condizioni stazionarie si possono dunque definire le linee di flusso ed i 
tubi di flusso e dunque posso applicare l'integrale al tubo di flusso stesso 
secondo le due superfici $C$ e $C'$:
\begin{gather*}
    \int_{\Sigma_C}^{} \rho \vv{u} \cdot \hat{n} \ d\sigma + \int_{\Sigma_{C'}}^{}  \rho \vv{u} \cdot \hat{n} \ d\sigma + \int_{\Sigma_A}^{} \rho \vv{u} \cdot \hat{n}\ d\sigma = 0     
\end{gather*}
Deve valere zero in quanto siamo in un tubo chiuso, dove $A$ è la superficie del tubo tra
le due $C$ e $C'$. Dato che siamo in un tubo di flusso, la normale è sempre
perpendicolare alla velocità per quanto riguarda il termine che è descritto
dalla superficie $A$, mentre per quanto riguarda le due componenti 
sulle superfici $C$ e $C'$ è parallelo e dunque il prodotto scalare è uno.
\begin{gather*}
    \int_{\Sigma_C}^{} \rho \vv{u} \ d\sigma = \int_{\Sigma_{C'}}^{}  \rho \vv{u} \ d\sigma 
\end{gather*}
Se le linee di flusso sono entranti nel tubo, la componente $\hat{n}$ la considero
entrante, mentre la seconda la considero uscente, dunque si è invertito la normale e allora  
si è spiegato perché non c'è il meno nel secondo integrale.
Si definisce allora
\begin{align}
    Q_M =     \int_{\Sigma_C}^{} \rho \vv{u} \ d\sigma = \int_{\Sigma_{C'}}^{}  \rho \vv{u} \ d\sigma 
\end{align}
Come la \textbf{portata di massa} $[kg \cdot  s^{-1}]$ che è positiva in questo caso se il fluido va verso destra. La portata
inoltre si conserva, infatti, se dovesse entrare una certa quantità di massas, allora deve necessaraimente 
anche uscire dalla sezione $C'$. Chiaramente $\rho$ deve essere una costante per
i fluidi incomprimibili. In generale per qualunque fluido si conserva la portata di massa
lungo un tubo di flusso ma, per i fluidi incomprimibili, si conserva anche la \textbf{portata
di volume} $[m^{3} \cdot s^{-1}]$, in quanto la densità è costante.  Queste considerazioni prendono il 
nome di \textbf{principio di Leonardo}.


\begin{wrapfigure}{r}{0.2\textwidth}
    \centering
    \begin{tikzpicture}
        \draw[->](0, 0) -- (3, 0) node[at end, below] {$\vv{u}$ };
        \draw(0.25, 1) -- (0.75, -1) node[at start, left] {$\Sigma$};
        \draw[->](0.5, 0) -- (1.5, 0.3) node[at end, right] {$\hat{n}$ };
        \draw(1, 0.15) arc (10:0:1) node[at end, below] {$\theta$};
        \draw[cyan](0.5, 1) -- (0.5, -1) node[at end, left] {$\Sigma$};
    \end{tikzpicture}    
\end{wrapfigure}
Questa definizione macroscopica ha anche una controparte microscopica
per un tubo di flusso sottile, in questo caso si ha che
\begin{gather*}
    \rho  \vv{u} \sigma_1 \hat{n} = \rho \vv{u} \sigma_2 \hat{n}  
\end{gather*}
Se si avesse una superficie con un certo angolo rispetto alla velocità, 
basterà fare il prodotto scalare con l'angolo $\theta$ tra il versore
normale e il vettore velocità. Dunque non varia il flusso se varia 
l'inclinazione della superficie rispetto al vettore velocità.

\section{Teorema di Bernoulli}
\begin{wrapfigure}{r}{0.4\textwidth}
    \centering
    \caption{Teorema di Bernoulli applicato al tubo di flusso sottile}
    \begin{tikzpicture}
        \draw[->](0, 0) -- (4, 0);
        \draw[->](0, 0) -- (0, 3.5);
        \draw(0, 3) .. controls (0.5, 3) and (0.75, 2.5) .. (1, 1.5).. controls (1.25, 0.8) and (1.75, 0.8) .. (2, 0.8);
        \node at (2, 1.1) {$\vv{dl_2}$ };
        \node at(0.4, 3.25) {$\vv{dl_1}$ };
        \draw[dashed](0, 0.8) -- (2, 0.8) node[at start, left] {$h_2$};
        \node at (-0.3, 3) {$h_1$};
    \end{tikzpicture}    
\end{wrapfigure}
Supponendo si avere un tubo di flusso sottile su cui agisce la forza di gravità, le cui
estremità considerate sono a delle altezze $h_1 > h_2$ e con una profondità
della sezione considerata rispettivamente di $\vv{dl_1} \neq \vv{fl_2} $.  
si sa che su di esso agiscono delle forze di pressione che lo fanno muovere all'interno di
questo tubo sottile che compiono un lavoro che si può esprimere come
\begin{gather*}
    L_p = p\sigma_1 dl_1 - p\sigma_2 dl_2
\end{gather*}
Il lavoro fatto dalla gravità invece è dato da
\begin{gather*}
    L_g = \rho_1 g h_1 \sigma_1 dl_1 - \rho_2  g h_2 \sigma_2 dl_2
\end{gather*}

\subsection{Il fluido incomprimibile}
Per un fluido incomprimibile si ha solamente l'energia cinetica da considerare
come energia interna. Dunque l'energia cinetica all'inizio è diminuita in quanto
perdo massa (esce dalla superficie) mentre alla fine del tubo l'energia cinetica è
positiva in quanto entra massa nella superficie e dunque si può esprimere
\begin{gather*}
    \Delta K = \frac{1}{2}\rho_2 u_2^{2}\sigma_2 dl_2 - \frac{1}{2}\rho_1 u_1^{2} \sigma_1 dl_1 
\end{gather*}
Se non sussistono altri contributi energetici, si deve necessariamente 
avere conservazione dell'energia totale, dunque
\begin{gather*}
    (p_1 + \rho_1 g h_1 + \frac{1}{2}\rho_1 u_1^{2}) \sigma dl_1 = (p_2 + \rho_2 g h_2 + \frac{1}{2}\rho_2 u_2^{2}) \sigma_2 dl_2
\end{gather*}
Se il fluido è incomprimibile, allora si ha che 
\begin{gather*}
    \sigma_1 dl_1 = \sigma_2 dl_2 = m
\end{gather*}
Dato che si suppone anche che si consideri un solo tipo di liquido, allora
si deve avere che $\rho_1 = \rho_2$, inoltre deve valere che
\begin{gather*}
    \frac{p_1}{\rho} + g h_1 + \frac{u_1^{2}}{2} = \frac{p_2}{\rho} + gh_2 + \frac{u_2^{2}}{2}
\end{gather*}
E dunque per un fluido incomprimibile ideale vale la seguente legge
\begin{align}
    p + \rho g h + \frac{1}{2}\rho u^{2} = \text{const}
\end{align}
Con le seguenti forme equivalenti
\begin{align}
    \frac{p}{\rho} + gh + \frac{1}{2}u^{2} = \text{const} = \frac{p}{\rho g} + h + \frac{u^{2}}{2g}
\end{align}

\subsection{Fluido comprimibile}
Se il fluido è comprimibile invece, devo anche considerare che tipologia
di fluido è, dunque devo considerare l'energia interna del fluido
\begin{gather*}
    \Delta U = U_2 \sigma_2 dl_2 - U_1 S_1 dl_1
\end{gather*}
Dove $U_1$ e $U_2$ sono le energie interne per unità di volume.
Posso dunque rifare il conto visto per il fluido incomprimibile,
ossia posso porre l'energia totale come conservata, e allora si ottiene che
\begin{gather*}
    \Delta K + \Delta U = L_p + L_g
\end{gather*}
Ossia la generalizzazione del teorema delle forze vive per un
fluido comprimibile, poiché, dato che il fluido è comprimibile,
esso è soggetto a variazioni di temperatura (ossia variazioni delle'energia
interna). Allora si ottiene
\begin{gather*}
    (p_1 + \rho_1 g h_1 + \frac{1}{2}\rho_1 u_1^{2} + U_1)  \sigma_1 dl_1 =  \sigma_2 dl_2(p_2 + \rho_2 gh_2 + \frac{1}{2}\rho_2 u_2^{2} + U_2) 
\end{gather*}
Raccogliendo $\rho_1$ e $\rho_2$ da entrambe le parti, si ha che
le quantità
\begin{gather*}
    \rho_1 dV_1 = \rho_2 dV_2
\end{gather*}
Sono uguali per l'equazione di continuità, e allora si giunge alla seguente
legge per i gas comprimibili
\begin{align}
    \frac{p + U}{\rho} + gh + \frac{1}{2}u^{2} 
\end{align}
La quantità
\begin{gather*}
    \frac{p + U}{\rho}
\end{gather*}
E' l'entalpia specifica per unità di massa. $p$ è l'entalpia 
per unità di volume e $U$ è l'entalpia.

\section{Applicazioni pratiche del principio di Bernoulli}
\subsection{Effetto venturi}
\begin{wrapfigure}{r}{0.4\textwidth}
    \centering
    \caption{Effetto venturi}
    \begin{tikzpicture}
        \draw(0, 0) -- (1, 0);
        \draw(0, -2) -- (1, -2);
        \draw(1, 0) .. controls (2.5, -0.5) .. (4, -0.6);
        \draw(1, -2) .. controls (2.5, -1.5) .. (4, -1.4);
        \draw(0, 0)  -- (0, -2) node[midway, left] {$\Sigma_1$};
        \draw(4, -0.6) -- (4, -1.4) node[midway, right] {$\Sigma_2$};\
    \end{tikzpicture}    
\end{wrapfigure}
Preso un fluido incomprimibile e in condizioni stazionarie
in una conduttura che ha due superfici $\Sigma_1 \neq \Sigma_2$. 
Applicando il principio di Leonardo, si ha la seguente
\begin{gather*}
    u_1 \Sigma_1 = u_2 \Sigma_2 \ \Longrightarrow \ u_2 = \frac{\Sigma_1}{\Sigma_2}u_1
\end{gather*}
A questo punto si ha che(poiché non c'è variazione di quota tra le due superfici
di entrata e di uscita): 
\begin{gather*}
    \frac{u_1^{2}}{2} + \frac{p_1}{\rho_1} = \frac{u_2^{2}}{2} + \frac{p_2}{\rho_2}
\end{gather*}
Posso determinare $u_1$
\begin{gather*}
    \frac{p_1 - p_2}{\rho} = \left(\frac{\Sigma_1}{\Sigma_2}^{2} - 1\right)\frac{u_1^{2}}{2}
\end{gather*}
Se $\Sigma_1 > \Sigma_2$, la pressione è più alta dove la velocità 
è più bassa. Questa cosa è contro intuitiva ma è dimostrato da questo esempio
e prende il nome di \textbf{Effetto venturi}. 


\end{document}
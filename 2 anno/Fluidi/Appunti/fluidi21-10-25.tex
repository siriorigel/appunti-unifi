\documentclass[a4paper, oneside]{article}
\usepackage{graphicx}
\usepackage{amsthm}
\usepackage{amsmath}
\usepackage{amssymb}
\usepackage[a4paper,
            bindingoffset=0.2in,
            left=2cm,
            right=2cm,
            top=2cm,
            bottom=2cm,
            footskip=.25in]{geometry}
\usepackage[italian]{babel}
\usepackage{pgfplots}
\usepackage{tabularx}
\usepackage{tikz}
\usepackage{wrapfig}
\usepackage{color}
\usepackage[d]{esvect}
\usepackage{mhchem}
\usepackage{chemfig}
\definecolor{page}{rgb}{0.129,0.157,0.212}
\pagecolor{page}
\color{white}
\graphicspath{ {./images/} }
\usetikzlibrary{shapes.geometric}
\usetikzlibrary{datavisualization}
\usetikzlibrary{datavisualization.formats.functions}
\usetikzlibrary{patterns}
\pgfplotsset{width=10cm,compat=1.9}

\title{Fluidi}
\author{Tommaso Miliani}
\date{21-10-25}

\begin{document}
\newtheoremstyle{theoremEnv}
                {}          % Space above
                {}          % Space below
                {\slshape}  % Body font
                {}          % Indent amount
                {\bfseries} % Head font
                {.}         % Punctuation after head
                {\newline}         % Space after theorem head
                {}          % Theorem head spec
\theoremstyle{theoremEnv}

\newtheorem{definition}{Definizione}[section]
\newtheorem{theorem}{Teorema}[section]
\newtheorem{lemma}{Proposizione}[section]
\newtheorem{observation}{Osservazione}[section]
\newtheorem{corollary}{Corollario}[theorem]
\newtheorem{example}{Esempio}[section]

\maketitle

\section{Riprendendo l'esempio dell'atmosfera}
L'equazione fondamentale dell'idrostatica non può essere utilizzata per poter
determinare il problema dell'atmosfera di fluido. Si era
infatti scelto 
\begin{gather*}
    p = \text{const } \rho
\end{gather*}
In modo tale che
\begin{gather*}
    \rho(z) \text{ simile a }  p(z) = p_0 \exp\left(-\frac{z}{k}\right)
\end{gather*}
La prima osservazione che si può fare è che compaiono delle coordinate
termodinamiche che non sono costanti nello spazio (ma non dal tempo). Questo
non è un vero e proprio equilibrio termodinamico in quanto è vero che non dipendono
dal tempo, ma dipendono anche dallo spazio; dunque si parla di 
\textbf{equilibrio termodinamico locale}. La dipendenza dalla coordinata
$z$ è la conseguenza dell'azione di qualcosa di esterno al fluido. 
Qui è come si trattasse di tanti strati di fluido in equilibrio termodinamico
tutti leggermente diversi tra di loro. Il fluido che si studia è che 
sia l'atmosfera terrestre. E' del tutto ragionevole considerare  l'atmosfera come un gas 
perfetto in quando ha la seguente composizione:
\begin{gather*}
    \ce{O_2} \quad \ce{N_2} \to 99\% \\
    \ce{CO_2} \quad \ce{H_2O} \qquad \ce{CH_4} \to 1 \%
\end{gather*}
Un gas è approssimabile ad un gas perfetto se la sua temperatura non è confrontabile
con la sua temperatura critica. La temperatura critica dell'azoto
molecolare è circa $125 \ K$ mentre la temperatura critica dell'ossigeno è
circa $155 \ K$. Anche ai bordi della troposfera si parla di temperature
dell'ordine di $200 \ K$ (più freddo della temperatura sulla superficie ma comunque
molto superiori delle varie temperature critiche). Approssimare 
l'atmosfera ad un gas perfetto non è esatto ma è una buona approssimazione. Se
l'atmosfera è un gas perfetto, dire che l'atmosfera ha una relazione di pressione costante
con l'indice politropico uguale ad $1$. Questo significa assumere che tutte
le trasformazioni che avvengono nel gas (e che dunque ci sia una equazione che mi permetta
di determinare la pressione in funzione di una costante), equivale a dire che
tutte le trasformazioni avvengano a temperatura costante. Questo modello prende il nome
di \textbf{modello di atmosfera isotermica}. 

Il fatto che l'atmosfera sia stratificata e che dunque la pressione diminuisca con
la quota mi permette di definire dei processi che avvengono all'interno dell'atmosfera.
Tipicamente è possibile che ci siano delle porzioni di fluido che cambiano di densità
per motivi termodinamici oppure dinamici; dunque le porzioni meno 
dense salgono e,  dato che la pressione è minore, allora questa porzione di fluido
continua ad espandersi via via che sale nel fluido. Se invece succede 
che una porzione di fluido diventi più densa, allora questa viene spinta verso il basso 
e dunque continua a ricevere spinta verso il basso. Questo fenomeno prende il 
nome di \textbf{convezione}.  Mediamente non esiste una dipendenza dal tempo e
queste trasformazioni non sono descritte da una legge che vincola la pressione ad essere costante. 
Questi movimenti sono sufficientemente veloci in modo tale che non ci sia scambio di energia tra la
bolla di fluido ed il fluido intorno (di fatto il moto avviene in un ambiente adiabatico).

\subsection{La formulazione rigorosa}
Dato il primo principio e, dato che si è detto che questi movimenti non
scambiano energia con il resto del fluido, posso dire che
\begin{gather*}
    dU + pdV = 0 \ \Longrightarrow \ dU = \mathcal{C}_V dT  
\end{gather*}
E dunque 
\begin{gather*}
    dU = mc_V dT \qquad pdV = -dp\frac{V}{\gamma} \ \Longrightarrow \ mc_V dT - dp\frac{V}{\gamma}= 0
\end{gather*}
Posso quindi risolvere e ottenere
\begin{gather*}
    \frac{m}{V}c_V dT - \frac{dp}{\gamma} = 0 \ \Longrightarrow \rho c_V dT - \frac{dp}{\gamma} = 0
\end{gather*}
Sostituendo con l'espressione di $\gamma$, ottengo l'espressione per la pressione 
infinitesimo è data da
\begin{gather*}
    \rho c_V dT - \frac{c_V}{c_p}dp = 0 \ \Longrightarrow \ dp = c_P \rho dT
\end{gather*}
Dato che 
\begin{gather*}
    c_p \rho dT = -\rho g dz
\end{gather*}
Posso esprimere la variazione della temperatura rispetto al rapporto dell'accelerazione
di gravità è del calore specifico a pressione costante.
\begin{gather*}
    dp = -\rho g dz
\end{gather*}
Dunque si ottiene
\begin{align}
    \frac{dT}{dz} = -\frac{g}{c_p}
\end{align}
Quello che si trova in questo modo è che se si integrasse questa
espressione si otterrebbe l'espressione della temperatura del fluido in funzione
della quota come
\begin{align}
    T(z) = T_0 - \frac{g}{c_p}z
\end{align}
Dunque per l'atmosfera terrestre 
\begin{gather*}
    \frac{dT}{dz} \approx -0.98 \cdot 10^{-2} K \cdot m^{-1}  
\end{gather*}
Dunque questa equazione riesce a predire il profilo di temperatura del
fluido in funzione della quota: il profilo di temperatura è, con ottima
approssimazione, lineare. Quello che viene fuori è che le misure ci dicono che
se si prova a misurare quella derivata, in realtà è leggermente più piccola:
\begin{gather*}
    \left(\frac{dT}{dz}\right)_{Misurata} \approx -0.7 \cdot 10^{-2} \ K \cdot  m^{-1}  
\end{gather*}
Ossia una discrepanza del $30\%$. Questo si spiega con il fatto che c'è un cambiamento
di stato tra il vapore acqueo e l'aria: con l'effetto della convezione l'acqua
sale e si raffredda e dunque ricade sottoforam di pioggia: il punto è che non altera
il modello del fluido ma è un fluido nel quale si ha una fonte di energia che rilascia il suo calore
latente di vaporizzazione: ogni goccia d'acqua sputa energia e scalda il sistema. SI può ottenere una controprova
cerando la dipendenza dall'umidità di questa formulazione. 



\end{document}
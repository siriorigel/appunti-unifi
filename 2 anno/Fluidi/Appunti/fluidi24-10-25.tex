\documentclass[a4paper, oneside]{article}
\usepackage{graphicx}
\usepackage{amsthm}
\usepackage{amsmath}
\usepackage{amssymb}
\usepackage[a4paper,
            bindingoffset=0.2in,
            left=2cm,
            right=2cm,
            top=2cm,
            bottom=2cm,
            footskip=.25in]{geometry}
\usepackage[italian]{babel}
\usepackage{pgfplots}
\usepackage{tabularx}
\usepackage{tikz}
\usepackage{wrapfig}
\usepackage{color}
\usepackage[d]{esvect}
\definecolor{page}{rgb}{0.129,0.157,0.212}
\pagecolor{page}
\color{white}
\graphicspath{ {./images/} }
\usetikzlibrary{shapes.geometric}
\usetikzlibrary{datavisualization}
\usetikzlibrary{datavisualization.formats.functions}
\usetikzlibrary{patterns}
\pgfplotsset{width=10cm,compat=1.9}

\title{Termodinamica}
\author{Tommaso Miliani}
\date{24-10-25}

\begin{document}
\newtheoremstyle{theoremEnv}
                {}          % Space above
                {}          % Space below
                {\slshape}  % Body font
                {}          % Indent amount
                {\bfseries} % Head font
                {.}         % Punctuation after head
                {\newline}         % Space after theorem head
                {}          % Theorem head spec
\theoremstyle{theoremEnv}

\newtheorem{definition}{Definizione}[section]
\newtheorem{theorem}{Teorema}[section]
\newtheorem{lemma}{Proposizione}[section]
\newtheorem{observation}{Osservazione}[section]
\newtheorem{corollary}{Corollario}[theorem]
\newtheorem{example}{Esempio}[section]
\newtheorem{remark}{Enunciato}[section]

\maketitle

\section{Come si traduce l'equazione di diffusione nel caso realistico}
In questo caso la funzione flusso di calore si traduce semplicemente utilizzando 
la funzione sul vettore posizione: 
\begin{align}
    q(\vv{r}, t) = -k \nabla T \cdot  \vv{n} 
\end{align}
Localmente la superficie che si è considerata ha una direzione normale esterna
determinata dal vettore $\vv{n}$. E' quindi la legge di Fourier in tre dimensioni.
Il campo di temperatura si modifica nella seguente maniera:
\begin{gather*}
    \frac{\partial T}{\partial t} = D_T \nabla^{2} T  
\end{gather*} 
Questo è un esempio di fenomeno irreversibile: in generale si deve introdurre
una distinzione tra trasformazioni reversibili ed irreversibili.

\section{Trasformazioni reversibili}
Per definire una trasformazione reversibile non è sufficiente
dire che un processo può essere invertito: si deve anche poter riportare
indietro anche l'ambiente circostante. In altre parole una trasformazione
reversibile deve cancellare ogni traccia di quello che è successo sull'ambiente
per poter essere definita tale (oltre che tornare allo stato iniziale). Se si fosse
compiuto un certo $L$ oppure si fosse scambiata una certa quantità di calore
$Q$ con l'ambiente, una trasformazione reversibile deve poter
compiere un lavoro $-L$ e scambiare $-Q$ calore con l'ambiente in modo 
tale che non ci sia nessun modo per poter determinare che c'è stata una trasformazione.
Anche se questo non è possibile, entro certe approssimazioni è possibile avvicinarsi 
arbitrariamente all'ideale della trasformazione reversibile. Le trasformazioni 
che non soddisfano queste caratteristiche prendono il nome di \textbf{irreversibili}.

\subsection{Esempi di trasformazioni irreversibili}
Il raggiungimento dell'equilibrio termico tra due sistemi termodinamici è un esempio di 
trasformazione spontanea e irreversibile che annulla il gradiente del campo 
di temperatura; questa è irreversibile perché non esiste un modo per poter
riportare entrambi i sistemi alla loro temperatura iniziale utilizzando la stessa
trasformazione termodinamica poiché uno dei due sistemi non deciderà mai di assorbire 
calore dall'altro per poter tornare alle proprie temperature iniziali ma devo utilizzare
trasformazioni diverse. \\
Un'altra trasformazione irreversibile è l'espansione adiabatica: questo tipo di 
trasformazione, per poter riportare il sistema alle condizioni iniziali, devo comprimere il
gas (e dunque compiere lavoro) e utilizzare un termostato e dunque 
l'espansione adiabatica è una trasformazione irreversibile. 
La caratteristica che accomuna queste due trasformazioni è la 
non quasi staticità delle due trasformazioni. Tuttavia non è chiaro che
se una trasformazione sia quasi statica allora debba anche essere reversibile:
è una condizione necessaria ma non sufficiente. Tuttavia non è chiaro anche qui
se tutte le trasformazioni irreversibili non siano quasi statiche.

\subsection{E' possibile trovare una trasformazione quasi statica reversibile?}
\begin{wrapfigure}{r}{0.4\textwidth}
    \centering
    \caption{Trasformazione quasi statica irreversibile}
    \begin{tikzpicture}[scale = 0.8]
        \filldraw[cyan, opacity = 0.3] (0, 0) rectangle (2, 2);
        \draw[very thick](0, 0) rectangle(4, 2);
        \filldraw(2, 0) rectangle (2.2, 2);        
        \draw(2, 1) -- (5, 1);
        \draw(4.25, 0) rectangle (5.75, 0.75);
        \draw[pattern = north west lines, pattern color = white](4.5, 0.75) rectangle (5.5, 1);
        \draw[pattern = north west lines, pattern color = white](4.25, 0.6) rectangle(5.75, 0.75);
        \node at(5, 0.3) {$T$};
    \end{tikzpicture}    
\end{wrapfigure}
Per fare una trasformazione quasi statica reversibile: potrei pensare di far
avvenire una espansione di un gas adiabatica e quasi statica: per farlo posso
mettere un pistone che viene spinto dal gas istante per istante. Quello che 
accade è che il gas fa lavoro sull'ambiente esterno e può essere immagazzinato 
sottoforma di energia potenziale per esempio con una molla attaccata al pistone. 
In questo modo questa trasformazione è reversibile poiché non si è lasciata alcuna
traccia nell'ambiente e nel sistema.  
Si può ottenere un risultato analogo (ma non reversibile) utilizzando non più 
un apparato che possa immagazzinare energia potenziale ma facendo
agire degli attriti all'asta che collega il pistone in modo tale che sia l'attrito che mi permetta
di ottenere la trasformazione quasi statica. All'andata è dunque stato fatto del lavoro meccanico
sull'ambiente e le forze dissipative hanno trasformato l'energia 
meccanica del gas in energia interna dell'ambiente sottoforma di calore. In questa situazione
la quasi staticità non basta poiché al termine del processo è vero che il gas torna
nelle condizioni iniziali, ma è stato compiuto lo stesso lavoro sia 
all'andata che al ritorno dalle forze di attrito: dunque si è dissipata
una quantità di energia pari a
\begin{gather*}
    2L = Q_{d}
\end{gather*}
Se prendessi in prestito dal termostato una certa quantità di calore e,
tramite processi termodinamici, la trasformassi in lavoro, potrei ottenere 
lo stesso lavoro inverso $-L$. Tuttavia, anche se il primo principio
della termodinamica non lo vieta, non è possibile realisticamente ottenere
questo processo. Questo principio che vieta questa cosa, prende il nome di 
\textbf{secondo principio della termodinamica}.


\section{Secondo principio della termodinamica}
\begin{wrapfigure}{r}{0.4\textwidth}
    \centering
    \caption{Situazione impossibile per questo enunciato}
    \begin{tikzpicture}[scale = 1.2]
        \draw(0, 0) rectangle (2, 1);
        \node at (1, 0.5) {T};
        \draw[->, very thick](1, 0) -- (1, -0.5) node[midway, right] {$Q$};
        \draw(0.5, - 1) -- (1, -0.5) -- (1.5, -1) -- (0.5, -1);
        \draw[->, very thick](1.25, -0.75) -- (2, -0.75) node[at end, right] {$L > 0$};
    \end{tikzpicture}    
\end{wrapfigure}
Anche se si chiama secondo principio, storicamente è stato trovato prima 
del primo come spiegazione delle trasformazioni irreversibili. Inizialmente 
formulato da Kelvin, è stato successivamente riformulato da Planck
e dunque prende il nome di enunciato KP:
\begin{remark}[Enunciato KP]
    Non è possibile realizzare una trasformazione termodinamica il cui unico 
    risultato sia la completa trasformazione di una certa quantità di calore $Q$
    in lavoro meccanico $L > 0$ quando questa quantità di calore è assorbita
    da un solo termostato. 
\end{remark}
Questo enunciato è troppo generico anche se garantisce che la trasformazione di calore 
in lavoro non è l'unica trasformazione possibile. 

\begin{wrapfigure}{r}{0.4\textwidth}
    \centering
    \caption{Il lavoro del gas su di un corpo di massa $m$}
    \begin{tikzpicture}[scale = 0.85]
        \draw[very thick](0, 0) --  (2, 0) -- (2, 4);
        \draw[very thick](0, 0) -- (0, 4);
        \draw[very thick](0, 2) -- (2, 2);
        \draw(0.5, 2) rectangle (1.5, 3) node[midway] {$m$};
        \filldraw[cyan, opacity = 0.3](0, 0) rectangle (2, 2);
        \draw[|-|](-0.2, 2) -- (-0.2, 3.2) node[midway, left] {$h$};
        \draw[dashed](0, 3.2) -- (2, 3.2);
        \draw[dashed](0.5, 3.2) rectangle (1.5, 4.2) node[midway] {$m$};
        \draw (-1, -1) rectangle (3, 0) node[midway] {$T$};
        \node at (3, 2) {$L = Q$};
    \end{tikzpicture}    
\end{wrapfigure}
Si può considerare l'esperimento di Fermi per cui un gas perfetto
sopra un termostato alza una massa posta sopra di esso e dunque 
il lavoro fatto deve essere pari alla quantità di calore che gli passa il 
termostato per il primo principio.  Questo esperimento funziona perfettamente 
e posso trasformare tutto il calore in lavoro ma il punto è che 
nel farlo il dispositivo deve cambiare il suo stato. Si arriva alla
formulazione del secondo principio secondo Clausius
\begin{remark}[Enunciato C]
    Non è possibile avere una trasformazione termodinamica il cui unico risultato 
    sia il passaggio di calore da un termostato $T_B$ ad un termostato
    $T_A$ se $T_A > T_B$. 
\end{remark}
Senza la parola unico, il frigorifero sarebbe un sistema ideale, 
tuttavia se non gli fornisco energia, il frigorifero non funziona. 
Questi due enunciati non riescono a dirci se una trasformazione sia reversibile o no:
facendo un po' di strada il primo principio ci permette di dire se 
una trasformazione è reversibile o meno. 

\subsection{Gli enunciati si coimplicano}
Si può dimostrare che i due enunciati si possono implicare a vicenda
in quanto entrambi dicono che non è possibile fare determinate cose. 
Possiamo dimostrare che la negazione di Kelvin Planck implica la negazione 
dell'enunciato di Clausius: essendo che enunciano delle impossibilità ,
possiamo negare gli enunciati e ragionare con un controesempio. 
\begin{gather*}
    \begin{tikzpicture}
        \draw(0, -1) rectangle (1, 1) node[midway] {$T$};
        \draw[->, very thick](1, 0) -- (2, 0) node[midway, above] {$Q$}; 
        \draw(1.75, -0.25) -- (2.25, 0.25) -- (2.75, -0.25) -- (1.75, -0.25);
        \draw[very thick, ->](2.5, 0) -- (3.5, 0) node[midway, above] {$L$};
        \draw(3.5, -0.5) rectangle (4.5, 0.5) node[midway] {$E_{pot}$};
        \draw[->, very thick](4.5, 0) -- (5.5, 0) node[midway, above] {$L$};
        \draw(5.5, -0.5) rectangle (6.5, 0.5) node[midway] {$F_{att}$};
        \draw[->, very thick](6.5, 0) -- (7.5, 0) node[midway, above] {$Q$};
        \draw(7.5, -1) rectangle (8.5, 1) node[midway] {$T'$};
    \end{tikzpicture} 
\end{gather*}
Negando KP possiamo prendere una certa quantità di calore $Q$ da un termostato a 
temperatura $T$ e utilizzarla per produrre una
certa quantità di lavoro. Questo lavoro lo posso immagazzinare sottoforma di energia potenziale
e dunque posso utilizzarla quando voglio su di un sistema dove c'è un attrito in 
modo tale che il calore dissipato da questa forza di attrito passi sottoforma
di calore in un termostato con temperatura $T'$. Potrei allora 
decidere (dato che niente me lo vieta), la temperatura del secondo termostato
in modo tale che $T' > T$. In questo modo ho creato un insieme di processi 
termodinamici tali per cui sono riuscito a trasferire calore da un corpo più freddo ad uno 
più caldo. Per poter dimostrare l'altro verso devo introdurre le macchine termiche. 

\subsection{Macchine termiche}
Si definisce \textbf{Macchina termica} un dispositivo in grado di trasformare del calore
in lavoro attraverso delle trasformazioni termodinamiche quando scambia calore con un termostato.
Una macchina termica che rice lavoro dall'ambiente e usa questo lavoro prende il nome di \textbf{macchina frigorifera}.
Alcune macchine termiche sono \textbf{cicliche} sono delle macchine che compiono del lavoro e, terminato il processo 
di trasformazione di calore in lavoro, tornano  sempre nello stesso stato iniziale. Se
le trasformazioni che avvengono sono tutte reversibili, allora la macchina termica prende il nome di
\textbf{macchina termica reversibile}. Dato che una macchina termica prende 
del calore e lo converte in lavoro e prende il calore da un solo termostato, allora ha la seguente rappresentazione
\begin{gather*}
    \begin{tikzpicture}[scale = 0.85]
        \draw(0, 0) rectangle (2, 1) node[midway] {$T_1$};
        \draw[very thick, <->](1, 0) -- (1, -1) node[midway, right] {$Q_1$};
        \draw(1 , -2) circle (1);
        \node at (1, -2) {$M$};
        \draw[->, very thick](2, -2) -- (3, -2) node[midway, above] {$L > 0$};
        \draw(0, -5) rectangle(2, -4) node[midway] {$T_2$};
        \draw[->](-1, -4) -- (-1, 0) node[at end, left] {$T$};
        \draw[very thick, <->](1, -4) -- (1, -3) node[midway, right] {$Q_2$};
    \end{tikzpicture}
\end{gather*}
Una conseguenza del secondo principio è che una macchina termica deve necessariamente interagire
con due termostati diversi in quanto altrimenti riuscirebbe a convertire tutto il calore 
in ingresso in lavoro (e farebbe SOLO quello). In linea di principio se la macchina termica scambiasse
tutto il calore che arriva da $T_1$ e lo da a $T_2$ allora non compirà nessun lavoro. 
Dato che la quantità complessiva di calore scambiato dal sistema è 
esattamente $Q = Q_1 + Q_2$, e il lavoro di questa macchina termica è 
positivo, allora ho allora quattro possibili combinazioni
\begin{itemize}
    \item $Q_1 < 0, Q_2 < 0$: impossibile in quanto non potrebbe fare lavoro poiché la macchina
    starebbe fornendo calore ad entrambi i termostati.
    \item $Q_1 > 0, Q_2 > 0$: in questa configurazione la macchina sta ricevendo calore da entrambi
    i dispositivi. Dalla teoria io so che tra questi due termostati passa del calore se sono messi in contatto: se 
    potessi mettere quindi un conduttore tra i due termostati, allora mentre la macchina lavora
    il termostato $T_1$ cede del calore al termostato $T_2$. In questo modo 
    potrei far sì che durante un ciclo la quantità di calore che passa è $|Q_2|$. Questo mi porta a dire che
    il termostato $T_2$ riceve del calore $Q_2$ e cede del calore $Q_2$; questa situazione è equivalente alla situazione 
    in cui il termostato $T_1$ cede un calore pari a $Q_1 + Q_2$ come se il termostato $T_2$ non ci fosse. Allora questa
    macchina violerebbe il secondo principio della termodinamica.
    \item $Q_1 < 0, Q_2 > 0$: questa situazione ha due casi intermedi. \begin{itemize}
        \item $|Q_1| > |Q_2|$: ossia la situazione in cui assorbe calore da $T_2$ e lo cede 
        a $T_1$. Dato che cede più calore di quanto ne assorbe, se ponessi un collegamento 
        tra i due termostati e riuscissero a scambiare, tramite questo collegamento una quantità
        $Q_1$, allora è lo stesso ragionamento per il quale $T_1$ non esisterebbe ed è dunque 
        equivalente alla situazione nella quale $T_2$ è da solo e dunque viola di nuovo il 
        secondo principio della termodinamica.
    \end{itemize}
    \item $Q_1 > 0, Q_2 < 0$: questa è l'unica configurazione possibile in quanto è possibile cedere del
    calore al termostato più freddo prendendolo dal termostato più caldo.
\end{itemize}




\end{document}
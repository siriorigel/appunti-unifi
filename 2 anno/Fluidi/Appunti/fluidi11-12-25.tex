\documentclass[a4paper, oneside]{article}
\usepackage{graphicx}
\usepackage{amsthm}
\usepackage{amsmath}
\usepackage{amssymb}
\usepackage[a4paper,
            bindingoffset=0.2in,
            left=2cm,
            right=2cm,
            top=2cm,
            bottom=2cm,
            footskip=.25in]{geometry}
\usepackage[italian]{babel}
\usepackage{pgfplots}
\usepackage{tabularx}
\usepackage{tikz}
\usepackage{wrapfig}
\usepackage{color}
\usepackage[d]{esvect}
\usepackage{chemfig}
\usepackage{mhchem}
\definecolor{page}{rgb}{0.129,0.157,0.212}
\pagecolor{page}
\color{white}
\graphicspath{ {./images/} }
\usetikzlibrary{shapes.geometric}
\usetikzlibrary{datavisualization}
\usetikzlibrary{datavisualization.formats.functions}
\usetikzlibrary{patterns}
\pgfplotsset{width=10cm,compat=1.18}

\title{Appunti di Fluidodinamica}
\author{Tommaso Miliani}
\date{11-12-25}

\begin{document}
\newtheoremstyle{theoremEnv}
                {}          % Space above
                {}          % Space below
                {\slshape}  % Body font
                {}          % Indent amount
                {\bfseries} % Head font
                {.}         % Punctuation after head
                {\newline}  % Space after theorem head
                {}          % Theorem head spec
\theoremstyle{theoremEnv}

\newtheorem{definition}{Definizione}[section]
\newtheorem{theorem}{Teorema}[section]
\newtheorem{lemma}{Proposizione}[section]
\newtheorem{observation}{Osservazione}[section]
\newtheorem{corollary}{Corollario}[theorem]
\newtheorem{example}{Esempio}[section]
\newtheorem{remark}{Enunciato}[section]

\maketitle

\section*{Energia ed informazione attraverso un fluido}
\section{Onde di compressione}
Le onde sonore sono delle onde di compressione che
diffondono l'informazione in un fluido. Per trattarle si ipotizza
di trattare un sistema uniforme che dunque abbia le seguenti ipotesi:
\begin{itemize}
    \item $p = p_0 = \text{const}$;
    \item $\rho = \rho_0 = \text{const}$;
    \item $\vv{u} = 0$. 
    \item Gravità trascurabile per comodità.  
\end{itemize}
Si considera dunque un certo elemento fluido ad una certa posizione
$\vv{r} = x\hat{x} + y\hat{y} + z\hat{z}$ che si sposta da questa posizione
per qualche motivo. Posso descrivere il suo spostamento tramite 
lo spostamento Lagrangiano $\vv{\zeta}  (\vv{r}, t )$, il quale è una funzione di una coordinata: 
dato un punto diverso, si sposterà in modo diverso e dipenderà dal tempo. Evidentemente
il cambiamento dell'elemento fluido definisce la velocità
dell'elemento fluido, che si esprimerà come
\begin{gather*}
    \vv{u}(\vv{r}, t ) = \frac{\partial \vv{\zeta} }{\partial t}  
\end{gather*}
Si può provare, dalle equazioni di Eulero e di continuità, a descrivere i 
moti di compressione 
\begin{gather*}
    \frac{\partial p}{\partial t} + \vv{\nabla}(\rho \vv{u} ) = 0   \\
    \rho \left(\frac{\partial \vv{u} }{\partial t} + \vv{u} \cdot \vv{\nabla} \vv{u}    \right) = -\vv{\nabla}p 
\end{gather*}
Il termine nelle parentesi nella seconda è l'accelerazione nel sistema Euleriano; nell'ottica
Lagrangiana è invece la derivata seconda rispetto al tempo di $\vv{\zeta}$, allora si può
trasformare in:
\begin{gather*}
    \rho \frac{\partial^{2} \vv{\zeta} }{\partial t^{2}} = -\vv{\nabla}p 
\end{gather*} 
Si introduce ora un'altra ipotesi per la trattazione, ossia che le fluttuazioni 
siano molto piccole: dunque $\rho = \rho_0 + \delta \rho$ e
$p = p_0 \delta p$. L'ipotesi che si compie è che
\begin{gather*}
    \frac{\left| \delta \rho \right| }{\rho_0} << 1 \qquad \frac{\left| \delta p \right| }{p_0} << 1
\end{gather*}
Giustamente devo mettere il modulo poiché una variazione può essere anche negativa. Questa ipotesi,
che prende il nome di \textbf{ipotesi di piccole perturbazioni}, mi permette di definire 
l'equazione Lagrnagiana di prima come 
\begin{gather*}
    (\rho_0 - \delta \rho) \frac{\partial^{2} \vv{\zeta} }{\partial t^{2}} = - \vv{\nabla}(p_0 - \delta p) = -\vv{\nabla} \delta p   
\end{gather*}
Dato che il termine derivato è molto più piccolo della differenza $\rho_0 - \delta \rho$, è 
possibile ridurre l'equazione a 
\begin{gather*}
    \rho_0 \frac{\partial^{2} \vv{\zeta} }{\partial t^{2}} = -\vv{\nabla} \delta p  
\end{gather*}
Suppongo adesso che esita una legge Barotropica per la quale si può legare la densità alla pressione,
dunque vale che $p_0 = p(\rho_0)$ e $p = p(\rho)$. Si può anche esprimere
$\delta p = p(\rho + \delta \rho) - p(\rho_0)$. Sviluppando in Taylor il primo termine
in un intorno di $\rho_0$, è possibile dire che
\begin{gather*}
    \delta p = p(\rho_0 ) + \delta p\left(\frac{dp}{d\rho}\right)_0 + o(\delta \rho^{2}) - p (\rho_0)
\end{gather*}
Dunque nell'equazione di continuità:
\begin{gather*}
    \rho \frac{\partial^{2} \vv{\zeta} }{\partial t^{2}} = -\vv{\nabla}\left(\frac{dp}{d\rho}\right)_0 \delta p  
\end{gather*}
Posso togliere dal gradiente la quantità $\rho_0$ dunque
\begin{gather*}
    -\left(\frac{dp}{d\rho}\right)_0 \vv{\nabla} \delta p 
\end{gather*}
Dimensionalmente 
\begin{gather*}
    \left[\frac{dp}{d\rho}\right] = \left[\frac{J}{m^{3}}\right] = \left[\frac{m}{s}\right]^{2}
\end{gather*}
Allora posso esprimere l'equazione di prima come
\begin{gather*}
    \rho_0 \frac{\partial ^{2} \vv{\zeta} }{\partial t^{2}} = - C_s^{2} \vv{\nabla}\delta p  
\end{gather*}
\begin{gather*}
    \frac{\partial \delta p}{\partial t} - \vv{\nabla}(\rho_0 - \delta \rho) \frac{\partial \vv{\zeta} }{\partial t} = 0    
\end{gather*}
Dato che il termine derivato è ancora una volta più piccolo del termine posso toglierlo e posso anche
togliere 
\begin{gather*}
    \frac{\partial \delta p}{\partial t} + p_0 \vv{\nabla} + \frac{\partial \vv{\zeta} }{\partial t} = 0   
\end{gather*}
Derivando nuovamente rispetto al tempo, si ha
\begin{gather*}
    \frac{\partial ^{2}}{\partial \delta \rho} - p_0 \vv{\nabla} \frac{\partial ^{2} \vv{\zeta} }{\partial t^{2}}   = 0
\end{gather*}
L'ultimo termine è uguale a
\begin{gather*}
    p_0 \vv{\nabla} \frac{\partial ^{2} \vv{\zeta} }{\partial t^{2}} = -C_s^{2} \vv{\nabla} \vv{\nabla} \delta \rho \ \Longrightarrow \ \boxed{\frac{\partial ^{2} \delta \rho}{\partial t^{2}} - C_s^{2}\nabla^{2} \delta \rho = 0}  
\end{gather*}
Dunque si suppone che tutte le quantità dipendano esclusivamente da una coordinata
nell'ipotesi di onda piana:
\begin{gather*}
    \frac{\partial ^{2} \delta \rho}{\partial t^{2}} - C_s^{2} \frac{\partial ^{2} \delta \rho}{\partial t^{2}} = 0   
\end{gather*}
Questa equazione ha le seguenti soluzioni:
\begin{gather*}
    \delta \rho = f_\pm(x \mp C_s t)
\end{gather*}

\begin{wrapfigure}{r}{0.4\textwidth}
    \centering
    \caption{Funzione $f(x \mp C_s t)$}
    \begin{tikzpicture}
        \draw(-1, 0) -- (2, 0);
        \draw(0, -1) -- (0, 2);
        \draw(-1, 0.1) .. controls (-0.75, 0.4) and (-0.25, 1.5) .. (0, 1.5) .. controls (0.25, 1.5) and (0.75, 0.4) .. (1, 0.1);
    \end{tikzpicture}    
\end{wrapfigure}
Ossia qualunque funzione esprimibile in questa forma è soluzione della differenziale,
infatti
\begin{gather*}
    \frac{\partial f}{\partial x} = \frac{\partial f}{\partial z} \frac{\partial f}{\partial x} \qquad \frac{d^{2} f}{d x^{2}}   = \frac{d f}{dz^{2}}\left(\frac{dz}{dx}\right)^{2} = \frac{d^{2}f}{dz^{2}} 
\end{gather*}
Inoltre, rispetto al tempo
\begin{gather*}
    \frac{\partial f}{\partial t} = \frac{df}{dt} \frac{\partial t}{\partial t} \qquad \frac{\partial ^{2} f}{\partial t^{2}} = \frac{d^{2} f}{dt^{2}} C_s^{2}   
\end{gather*}
Dunque qualunque funzione fatta in questa maniera è soluzione dell'equazione,
qualunque essa sia. L'onda con $f(x + C_s t)$ è un onda che si sposta indietro,
mentre l'altra si sposta in avanti. Le soluzioni sono dunque delle coppie di
onde (o segnali) che si propagano in direzioni diverse: le onde di compressione dunque si muovono 
in avanti e indietro nel fluido per generare differenza di densità e dunque di pressione. QUesto 
non vuol dire che si debba necessariamente raggiungere la velocità di propagazione 
dell'onda nel mezzo, infatti quando si parla le corde vocali non vibrano 
alla velocità del suono ma la voce di propaga alla velocità del suono.
Molto spesso le onde sono identificate dal vettore d'onda e dalla
pulsazione. Si ha dunque il segnale che si identifica dal seno o dal coseno
e si scrive 
\begin{gather*}
    \delta \rho = \cos\left(kx \pm \omega t\right)\qquad k = \frac{2\pi}{\lambda} \qquad \omega = \frac{2\pi}{T}
\end{gather*}
Chiaramente esiste una relazione tra il vettore d'onda e la frequenza
verificando o che questa equazione è soluzione della differenziale, oppure che
\begin{gather*}
    \delta \rho = \cos\left(k \left(x \pm \frac{\omega t}{k}\right)\right) \ \Longrightarrow \ \frac{\omega}{k} = \pm C_s
\end{gather*}
Il rapporto ci da sempre la velocità di fase, ossia è una relazione di dispersione per
un sistema di onde che non sono dispersive. Si può calcolare la velocità delle onde 
di compressione secondo la seguente
\begin{gather*}
    C_s^{2} = \left(\frac{dp}{d\rho}\right)_0
\end{gather*}
Che cambia, ovviamente, dal tipo di fluido. Se volessi considerare la velocità del
suono dovrei, innanzitutto, che il gas sia perfetto e descrivere cosa succede quando 
il fluido è compresso oppure decompresso. Si pul pensare che la compressione sia
così veloce che gli elementi fluidi non riescono a scambiare calore con quelli adiacenti;
si può dire allora che il processo è adiabatico ed utilizzare una legge barotropica. 
\begin{gather*}
    p = p_0 \left(\frac{\rho}{\rho_0}\right)^{\gamma} \qquad \gamma = \frac{c_p}{c_V} 
\end{gather*} 
Se $\gamma = 1$ si ha la situazione isoterma del fas perfetto:
\begin{gather*}
    C_s^{2} = \gamma\frac{p_0}{\rho_0}
\end{gather*}
Dato che siamo in regime di gas perfetto, possiamo esprimere 
\begin{gather*}
    p = \frac{R\rho T}{\mu}  \qquad p_0 = \frac{R\rho_0 T}{\mu}
\end{gather*}
Dove $\mu$ è la massa molecolare media del gas considerato. 
Allora
\begin{gather*}
    C_s^{2} = \gamma\frac{RT_0}{\mu}
\end{gather*}
Utilizzando il coefficiente di comprimibilità
\begin{gather*}
    \frac{\Delta V}{V} = \frac{\Delta p}{\xi} \approx \frac{dV}{V} = - \frac{dp}{\xi} \qquad \frac{dV}{V} = - \frac{d\rho}{\rho}
\end{gather*}
Allora,
\begin{gather*}
    \frac{d\rho}{\rho} = \frac{dp}{\xi} \ \Longrightarrow \ \frac{dp_0}{d\rho_0} = \frac{\xi}{\rho_0}
\end{gather*}

\subsection{Geometria sferica a simmetria radiale}
In questo caso dipende solo da $r$, dunque 
\begin{gather*}
    \nabla^{2} \delta \rho = \frac{1}{r^{2}} \frac{d}{dr} r^{2} \frac{d\delta \rho}{dr}
\end{gather*}
Derivando rispetto al tempo si ha
\begin{gather*}
    \frac{\partial^{2} \delta \rho}{\partial t^{2}} = \dots 
\end{gather*}
Prendendo una funzione di appoggio $\psi$
\begin{gather*}
    \psi = r\delta \rho \qquad \frac{\partial \psi}{\partial r}  =\delta \rho + r \frac{\partial \delta \rho}{\partial r}  \qquad \frac{\partial^{2} \psi}{\partial r^{2}} = r \frac{\partial \delta \rho}{\partial r} + r \frac{\partial ^{2} \delta \rho}{\partial r^{2}}   
\end{gather*}
Sviluppando l'equazione di prima con questa funzione ausiliaria
\begin{gather*}
    \frac{\partial ^{2} \delta \rho}{\partial t^{2}} - C_s^{2} \frac{2}{r} \frac{\partial \delta \rho}{\partial r} - \frac{\partial ^{2} \delta \rho}{\partial t^{2}} = 0 \ \Longrightarrow \    \frac{\partial^{2} \psi}{\partial t^{2}} - C_s^{2} \frac{\partial ^{2} \psi}{\partial r^{2}}  = 0
\end{gather*}
Ossia si ottiene l'equazione d'onda tradizionale, dunque se si prende
$\psi = f(r \pm C_s t)$, $\psi$ mantiene la stessa ampiezza ma $\delta \rho$ 
diminuisce come $\frac{1}{r}$. Da questo si spiega come la lampadina 
diminuisce di intensità quando ci si allontana da essa. 



\end{document} 
\documentclass[a4paper, oneside]{article}
\usepackage{graphicx}
\usepackage{amsthm}
\usepackage{amsmath}
\usepackage{amssymb}
\usepackage[a4paper,
            bindingoffset=0.2in,
            left=2cm,
            right=2cm,
            top=2cm,
            bottom=2cm,
            footskip=.25in]{geometry}
\usepackage[italian]{babel}
\usepackage{pgfplots}
\usepackage{tabularx}
\usepackage{tikz}
\usepackage{wrapfig}
\usepackage{color}
\usepackage[d]{esvect}
\usepackage{chemfig}
\usepackage{mhchem}
\definecolor{page}{rgb}{0.129,0.157,0.212}
\pagecolor{page}
\color{white}
\graphicspath{ {./images/} }
\usetikzlibrary{shapes.geometric}
\usetikzlibrary{datavisualization}
\usetikzlibrary{datavisualization.formats.functions}
\usetikzlibrary{patterns}
\pgfplotsset{width=10cm,compat=1.18}

\title{Appunti di Fluidodinamica (Landi)}
\author{Tommaso Miliani}
\date{12-12-25}

\begin{document}
\newtheoremstyle{theoremEnv}
                {}          % Space above
                {}          % Space below
                {\slshape}  % Body font
                {}          % Indent amount
                {\bfseries} % Head font
                {.}         % Punctuation after head
                {\newline}  % Space after theorem head
                {}          % Theorem head spec
\theoremstyle{theoremEnv}

\newtheorem{definition}{Definizione}[section]
\newtheorem{theorem}{Teorema}[section]
\newtheorem{lemma}{Proposizione}[section]
\newtheorem{observation}{Osservazione}[section]
\newtheorem{corollary}{Corollario}[theorem]
\newtheorem{example}{Esempio}[section]
\newtheorem{remark}{Enunciato}[section]

\maketitle

\section{Onde di gravità}
\begin{wrapfigure}{r}{0.4\textwidth}
    \centering
    \caption{Onde di gravità a seguito di una perturbazione}
    \begin{tikzpicture}
        \draw[->](-1, 0) -- (3, 0) node[at end, below] {$x$};
        \draw[->](0, -1) -- (0, 3) node[at end, left] {$z$};
        \node at(1, 1) {aria};
        \node at(1, -1) {acqua};
        \draw(0, 0) .. controls (1, 1) and (2, -1) .. (3, 0);
        \draw[->](-1, 2) -- (-1, 1.2) node[at end, left] {$\vv{g}$ };
        \draw(0.4, -0.1) rectangle (0.6, 0.1);
    \end{tikzpicture}    
\end{wrapfigure}
Sono delle onde che si presentano quando un campo esterno 
agisce su di un fluido (si tratterà solo il caso di fluido
incomprimibile). SUpponendo di avere una superficie di
separazione tra due fluidi diversi che hanno densità molto diverse tra loro,
e si suppone di creare una perturbazione sulla superficie:
si crea una colonna di liquido supplementare rispetto alla quota
di riferimento che crea una pressione supplementare sul piano
$x = 0$, che si alza e che vuole tornare verso il basso. Si deve dunque
calcolare la forza che del fluido che vuole tornare giù e 
determinare la relazione che mi permette di trasmettere 
l'oscillazione. Per trattarlo si discute un elemento fluido
posizionato ad una qualunque quota e si discute il suo spostamteno
tramite lo spostamento Lagrangiano. Il vettore spostamento sarà
\begin{gather*}
    \vv{\xi} (x, z, t) = \xi_x \hat{x} + \xi_z \hat{z}  
\end{gather*}
Si può dunque descrivere il moto attraverso una funzione trigonometrica
(coseno) con una ampiezza $\xi_0$ in funzione della quota:
\begin{gather*}
    \xi_z = \xi_0 h(z) \cos(kx - \omega t)
\end{gather*}
Ovviamente si vuole che $h(0) = 1$. Studiando la cinematica
dell'oggetto, si vuole che
\begin{gather*}
    \vv{\nabla} \cdot \vv{u} = 0 \qquad \vv{u} = \frac{\partial \vv{\xi} }{\partial t}    
\end{gather*}
Si può dunque calcolare $u_z$ e $u_x$:
\begin{gather*}
    u_z = \xi_0 \omega h(z) \sin(kx - \omega t) \\
    \frac{\partial u_x}{\partial x} = - \frac{\partial u_z}{\partial z} = - \xi_0 \omega h'(z)\sin(kx - \omega t)  
\end{gather*}
Dunque $u_x$:
\begin{gather*}
    u_x =  \frac{\xi_0}{k} \omega h'(z)\cos(kx - \omega t)
\end{gather*}
Se si integra rispetto al tempos si ottiene invece $\xi_x$:
\begin{gather*}
    \xi_x = \frac{\xi_0}{k} h'(z) \sin(kx - \omega t)
\end{gather*}
Allora si ha
\begin{gather*}
    \left\{\begin{array}{l}
            \xi_z = \xi_0 h(z) \cos(kx - \omega t)\\
    \xi_x = -\frac{\xi_0}{k} h(z) \sin(kx - \omega t)\\
    u_z = \xi_0 \omega h'(z) \sin(kx - \omega t) \\
    u_x =  \frac{\xi_0}{k} \omega h'(z)\cos(kx - \omega t)
    \end{array}\right.
\end{gather*}
$\xi_x$ e $\xi_z$ mi danno l'equazione di un ellisse: dunque si è studiato
la cinematica di un fluido perturbato in due dimensioni. Possiamo ora utilizzare
lo spostamento Lagrangiano per studiare la dinamica dell'oggetto.
\begin{gather*}
    \rho_0 \frac{\partial ^{2} \vv{\xi} }{\partial t^{2}} = -\vv{\nabla}p + \rho_0 \vv{g}   
\end{gather*}
Adesso, la pressione è esprimibile come la pressione di equilibrio 
più una fluttuazione $\delta p$, come per la trattazione delle onde
sonore. La condizione di equilibrio, evidentemente, diventa
\begin{gather*}
    \vv{\nabla}p_0 = \rho_0 g 
\end{gather*}
Posso ora scrivere 
\begin{gather*}
    \rho_0 \frac{\partial ^{2} \vv{\xi} }{\partial t^{2}} = -  \vv{\nabla} \delta p   - \vv{\nabla}p_0 - \rho_0 \vv{g}  \ \Longrightarrow \ \rho_0 \frac{\partial ^{2} \vv{\xi} }{\partial t^{2}} = - \vv{\nabla} \delta p  
\end{gather*}
Dunque si riassumono le condizioni viste fino ad ora:
\begin{gather*}
\left\{\begin{array}{l}
        \frac{\partial u_x}{\partial t} = - \frac{1}{\rho_0} \frac{\partial \delta p}{\partial x} = \frac{\xi_0}{k} h'(z) \omega^{2}\sin(kx - \omega t) \\
    \frac{\partial u_z}{\partial t} = - \frac{1}{\rho_0} \frac{\partial \delta p}{\partial z} = -\xi_0 \omega^{2} h(z) \cos(kx - \omega t)
\end{array}\right.
\end{gather*}
Posso eliminare $\delta p$ derivando la prima per $z$ e la seconda per $x$, in quanto 
le derivate miste devono essere uguali:
\begin{gather*}
    \frac{\partial }{\partial z}\left(\xi_0 \frac{h'(z)}{k} \omega^{2} \sin(kx - \omega t)\right) = - \frac{\partial }{\partial x}\left(\xi_0 \omega^{2} h(z) \cos(kx - \omega t)\right)  \\
    \xi_0 \frac{h''(z)}{k}  \omega^{2}\sin(kx - \omega t) = + k\xi_0 \omega^{2} h(z) \sin(kx - \omega t)
\end{gather*}
Semplificando ora si ottiene la differenziale che descrive l'ampiezza della perturbazione
in funzione della profondità $z$
\begin{gather*}
    h''(z) = k^{2}h(z)
\end{gather*}
Le soluzioni sono dunque le combinazioni lineari della seguente:
\begin{gather*}
    h(z) = h_+e^{kz} + h_-e^{-kz}
\end{gather*}
Le due costanti $h_+$ e $h_-$ sono determinate dalle condizioni al bordo,
ossia le seguenti:
\begin{itemize}
    \item $h(0) = 1$;
    \item Il fondo sia ad una profondità fissa $h(z_\text{fondo}) = 0$.
\end{itemize}
Dunque si possono trovare le costanti:
\begin{gather*}
    \left\{\begin{array}{l}
        h(0) = h_+ + h_- = 1  \\
        h_+e^{-kz_\text{fondo}} + h_-e^{kz_\text{fondo}} = 0
    \end{array}\right.
\end{gather*}
La soluzione è dunque
\begin{gather*}
    h(z) = \frac{e^{k(z + z_\text{fondo})} - e^{-k(z + z_\text{fondo})}}{e^{kz_\text{fondo}} - e^{-kz_\text{fondo}}}
\end{gather*}
Si può ora calcolare la spinta della colonna di fluido perturbato alla quota $z = 0$. Posso 
prendere lungo $x$ 
\begin{gather*}
    \rho_0 \frac{\partial u_x}{\partial t}_{z = 0} = - \frac{\partial \delta p}{\partial x}_{z = 0}  
\end{gather*}
Dove $\delta p = \rho_0 h\xi_z$ è la pressione della colonna di fluido incomprimibile che si è aggiunta 
in seguito alla perturbazione rispetto all'interfaccia $z = 0$, dunque
si può scrivere l'espressione di prima come 
\begin{gather*}
    \rho_ 0 h'(z) \frac{\xi_0}{k}\omega^{2} \sin(kx - \omega t) = - \rho_0 g k\xi_0 h(z) \sin(kx - \omega t ) \ \Longrightarrow \ \omega^{2} = g\frac{k^{2} h(z)}{h'(z)}_{z = 0}
\end{gather*}

\section{Casi particolari}
\subsection{Limite delle acque profonde}
Nel limite delle acque profonde $L \to +\infty$, dunque 
\begin{gather*}
    h(z) = e^{kz} \ \Longrightarrow \ h'(z) = ke^{kz}
\end{gather*}
Che tende a zero quando $z \to -L$. Si può ottenere una relazione per la dispersione delle
onde di gravità quando le acque sono molto profonde (ossia che $kL$ molto maggiore di $1$, ossia
nel limite in cui la lunghezza d'onda è molto più piccola della profondità del fluido):
\begin{gather*}
    \omega^{2} = kg
\end{gather*}
Dato che $h'(z) = h(z)k$, allora vuol dire che $\xi_x = \xi_z$, dunque
l'elemento fluido percorre delle circonferenze con raggio via via
più piccolo quando si va in profondità invece che delle ellisse. La velocità
di fase (ossia la velocità di propagazione dell'onda a quella determinata lunghezza
d'onda) è dunque
\begin{align}
    v_\phi = \frac{\omega}{k} = \sqrt{\frac{g}{k}} 
\end{align}
Le onde lunghe, in profondità, vanno più veloci rispetto alle onde corte. 


\subsection{Limite delle acque basse}
Nel limite delle acque basse $L \to 0$, dunque 
\begin{gather*}
    \frac{h'(z)}{h(z)}_{z = 0} = \frac{k}{\tanh(kL)}
\end{gather*}
Dunque 
\begin{gather*}
    \omega^{2} = kg\tanh(kL)
\end{gather*}
Dato che $kL << 1$, si può sviluppare con Taylor al primo ordine
per ottenere il \textbf{limite della acque basse} (non dispersiva perché
la velocità di fase non dipende dalla lunghezza d'onda):
\begin{gather*}
    \omega^{2} = k^{2}gL
\end{gather*}
La velocità di fase è dunque
\begin{gather*}
    v_\phi = \sqrt{gL} 
\end{gather*}
Il motivo per il quale le onde sembrano arrivare frontali rispetto alla riva
è che il fronte d'onda, anche se inclinato rispetto alla riva, inizia a ruotare poiché
la parte del fronte più vicino alla riva va più lento della parte più lontana e dunque
il fronte sembra ruotare in modo tale che arriva sempre frontale. 

\section{Flussi di fluidi comprimibili in sezioni convergenti e divergenti}
\begin{wrapfigure}{r}{0.4\textwidth}
    \centering
    \caption{}
    \begin{tikzpicture}
        \draw(0, 1.5) .. controls (1, 0.5) and (2, 0.5) .. (3, 1.5);
        \draw(0, -1.5) .. controls (1, -0.5)  and (2, -0.5).. (3, -1.5);
        \draw[->](-1, 0) -- (4, 0) node[at end, below] {$x$};
        \draw[very thick](0.65, 1) -- (0.65, -1) node[at start, above] {$A(x)$};
    \end{tikzpicture}    
\end{wrapfigure}
Si considerano condizioni stazionarie e si suppone avere una conduttura 
con una strozzatura, con le equazioni di Leonardo (e immaginando che tutto
dipenda da $x$ e che la velocità del fluido in arrivo alla strozzatura è $u_x$):
\begin{gather*}
    \vv{\nabla}(\rho \vv{u} ) = 0 \ \Longrightarrow \ \frac{d}{dx}(\rho u_x A_x ) = 0 
\end{gather*}
Ossia, presa una sezione $A_x$ della strozzatura e calcolandone il flusso, questo 
deve essere uguale al flusso della sezione $A_x + dx$ in quanto il flusso di una sezione
chiusa deve essere costante. Questo vuol dire che la derivata deve essere
zero in quanto non c'è cambiamento nel flusso di fluido. 
Posso sviluppare la derivata logaritmica in modo tale che
\begin{gather*}
    \frac{1}{\rho} \frac{d \rho }{dx } + \frac{1}{u} \frac{du}{dx} + \frac{1}{A} \frac{dA}{dx} = 0 
\end{gather*}
Posso esprimere 
\begin{gather*}
    \frac{\partial \vv{u} }{\partial t} + (\vv{u} \cdot \vv{\nabla} ) \vv{u} = - \frac{1}{\rho}\vv{\nabla} p   
\end{gather*}
Posso trascurare gli effetti della gravità (per esempio per un razzo
nello spazio) e anche la priam derivata in quanto siamo in condizioni stazionarie:
\begin{gather*}
    u\frac{du}{dx} = - \frac{1}{\rho}\frac{dp}{dx}
\end{gather*}
Posso impostare adesso l'ipotesi barotropica, dunque posso relazionare 
la pressione in funzione della densità:
\begin{gather*}
    p = p(\rho) \qquad \frac{dp}{dx} = \frac{dp}{d\rho} \frac{d\rho}{dx}
\end{gather*}
Adesso, utilizzando la velocità di propagazione delle onde di compressione,
si può ottenere una nuova espressione per l'equazione di continuità:
\begin{gather*}
    u \frac{du}{dx} = - \frac{1}{\rho} \frac{dp}{dx} = - C_s^{2} \frac{1}{\rho}\frac{dp}{dx}
\end{gather*}
Dunque
\begin{gather*}
    -\frac{1}{\rho}\frac{d\rho}{dx} = \frac{1}{u}\frac{du}{dx} - \frac{1}{A}\frac{dA}{dx} \ \Longrightarrow \ u\frac{du}{dx}= - C_s^{2}\left(\frac{1}{u}\frac{du}{dx}\right) + \frac{C_s^{2}}{A}\frac{dA}{dx}
\end{gather*}
Posso dunque riscrivere il tutto come
\begin{gather*}
    \left(\frac{u^{2}}{C_s^{2}} - 1\right) \frac{1}{u}\frac{du}{dx} = \frac{1}{A} \frac{dA}{dx}
\end{gather*}
Il rapporto tra la velocità è la velocità del suono prende il nome di 
\textbf{numero di Mach} $M$:
\begin{gather*}
    (M^{2} - 1)\frac{1}{u}\frac{du}{dx} = \frac{1}{A} \frac{dA}{dx}
\end{gather*}
Si riprende ora l'equazione di continuità, 
\begin{gather*}
    \frac{dp}{dx} = - \rho u^{2} \frac{1}{u}\frac{du}{dx} = - \frac{\rho u^{2}}{M^{2} - 1} \frac{1}{A} \frac{dA}{dx}
\end{gather*}
Posso cambiare di segno e ottenere la seguente relazione
\begin{gather*}
    (1 - M^{2}) \frac{dp}{dx} = \rho u^{2} \frac{1}{A} \frac{dA}{dx}
\end{gather*}
L'andamento della velocità dipende dunque dal profilo della conduttura: supponendo di
avere un gas caldo con velocità molto piccola rispetto alla velocità del suono, 
il flusso subsonico comincia ad accelerare in una conduttura convergente in quanto 
se diminuisce l'area, aumenta la velocità.  Se $\frac{dA}{dx} = 0$, non accelera,
dunque il fluido ha la velocità del suono localmente e a destra della strozzatura 
il fluido può o continuare ad accelerare in maniera indefinita a velocità supersonica
e diminuire la propria pressione, oppure può rallentare e aumentare la propria pressione
rispetto all'ambiente esterno. 
Questo oggetto prende il nome di \textbf{strozzatura di De-Laval}. 


\end{document}
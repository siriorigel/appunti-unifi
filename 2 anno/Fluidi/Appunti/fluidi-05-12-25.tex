\documentclass[a4paper, oneside]{article}
\usepackage{graphicx}
\usepackage{amsthm}
\usepackage{amsmath}
\usepackage{amssymb}
\usepackage[a4paper,
            bindingoffset=0.2in,
            left=2cm,
            right=2cm,
            top=2cm,
            bottom=2cm,
            footskip=.25in]{geometry}
\usepackage[italian]{babel}
\usepackage{pgfplots}
\usepackage{tabularx}
\usepackage{tikz}
\usepackage{wrapfig}
\usepackage{color}
\usepackage[d]{esvect}
\usepackage{chemfig}
\usepackage{mhchem}
\definecolor{page}{rgb}{0.129,0.157,0.212}
\pagecolor{page}
\color{white}
\graphicspath{ {./images/} }
\usetikzlibrary{shapes.geometric}
\usetikzlibrary{datavisualization}
\usetikzlibrary{datavisualization.formats.functions}
\usetikzlibrary{patterns}
\pgfplotsset{width=10cm,compat=1.18}

\title{Appunti di Fluidodinamica}
\author{Tommaso Miliani}
\date{05-12-25}

\begin{document}
\newtheoremstyle{theoremEnv}
                {}          % Space above
                {}          % Space below
                {\slshape}  % Body font
                {}          % Indent amount
                {\bfseries} % Head font
                {.}         % Punctuation after head
                {\newline}  % Space after theorem head
                {}          % Theorem head spec
\theoremstyle{theoremEnv}

\newtheorem{definition}{Definizione}[section]
\newtheorem{theorem}{Teorema}[section]
\newtheorem{lemma}{Proposizione}[section]
\newtheorem{observation}{Osservazione}[section]
\newtheorem{corollary}{Corollario}[theorem]
\newtheorem{example}{Esempio}[section]
\newtheorem{remark}{Enunciato}[section]

\maketitle

\section{Riassuntino altra volta}
La \textbf{pressione ram} è l'espressione della portanza
degli arei: il flusso con una certa velocità $\vv{u}$ che sbatte contro 
un ostacolo è deviato e cambia la sua quantità di moto, 
quando viene spinto verso il basso, l'ostacolo subisce una forza verso l'alto.
Un applicazione semplice del problema può essere la seguente: supponendo 
di avere una conduttura che ha sezione uguale $\Sigma$ da entrambe le parti 
ed un flusso entrante ed uscente, chiaramente, se il gomito della conduttura
non è inchiodato, il fluido esercita sul gomito una certa spinta che si determina 
attraverso il teorema di Bernoulli. Per il principio di Leonardo si ha che la quantità di 
massa che passa istantaneamente è:
\begin{gather*}
    Q_m = \rho u_k \Sigma
\end{gather*}
La quantità di moto è possibile calcolarlo prendendo una sezione parallela
al flusso:
\begin{gather*}
    \vv{F_1} = \rho \vv{u_1} \cdot u_1 \Sigma   
\end{gather*}
Allora 
\begin{gather*}
    \int_{V}^{} \frac{\partial \vv{p_u} }{\partial t} dV + \int_{V}^{}  \vv{\nabla} (\rho\vv{u}\cdot u  ) = -\int_{V}^{}  \vv{\nabla}p \ dV + \int_{V}^{}  \rho \vv{g} \ dV 
\end{gather*}
Si ottiene dunque la seguente uguaglianza
\begin{gather*}
    \frac{dQ}{dt} + \int_{\Sigma(V)}^{} (\rho \vv{u} \cdot \vv{u}  ) \cdot \hat{u} \ d \sigma = - \int_{\Sigma}^{}   \rho \hat{u} \ d\sigma = - \vv{\nabla} p + \rho \vv{u}   
\end{gather*}

Invece nel punto 2:
\begin{gather*}
    \vv{F_2} = \rho \vv{u_2} \cdot u_2  \Sigma 
\end{gather*}
$u_1 = u_2$ e $\vv{u_2} = -\vv{u_1}$  dato che deve valere il principio di Leonardo,
si ha che la variazione della quantità di moto è esattamente legata alla forza che
applica il fluido sulla conduttura
\begin{gather*}
    \vv{F_T} = - q_{1, 2} \vv{u_1} - q_{1, 2} \vv{u_2} = -2q_{1, 2} \vv{u_1}    
\end{gather*}
Dunque il fluido ha un flusso negativo di quantità di moto: perde quantità
di moto entrando nella conduttura e dunque 
\begin{gather*}
    F = 2q\vv{u} 
\end{gather*} 

\section{Pala eolica}
Nelle pale eoliche le pale girano poiché sono messe in moto dal fluido che
colpisce la pala eolica: supponendo di conoscere la superficie della pala, 
la velocità del vento e la potenza erogata dalla pala. A questo punto si
può determinare il rendimento della pala, la velocità del vento dopo la pala e la
spinta che il flusso esercita sulla pala. 
\begin{gather*}
    Q_M = \rho u_1 \Sigma
\end{gather*}
Mi posso calcolare il flusso di energia cinetica che 
entra nella pala, ossia la potenza che agisce sulla pala. 
\begin{gather*}
    P = Q_M \frac{u_1^{2}}{2}
\end{gather*}
Il rendimento è dunque la differenza tra la potenza iniziale meno quella 
assorbita:
\begin{gather*}
    \eta = \frac{P_E}{P_1}
\end{gather*}
Dove $P_E$ è la potenza assorbita dalla pala (ossia l'energia per unità di tempo 
sottratta al vento) e $P_1$ è la potenza in entrata. Dato che tutto si deve
conservare, allora si ha che la potenza in uscita è data
\begin{gather*}
    P_2 = P_1 - P_E
\end{gather*}
Dato che il flusso di massa si conserva:
\begin{gather*}
    Q_M \frac{u^{2}_2}{2} = Q_M\frac{u_1^{2}}{2}(1 - \eta)
\end{gather*}
A questo punto posso ottenere la velocità del vento dopo la pala 
in funzione del rendimento della pala
\begin{gather*}
    u_2 = u_1\sqrt{1 - \eta} 
\end{gather*}
Dato che la variazione di quantità di moto è data da
\begin{gather*}
    -\rho u_1 \Sigma_1 \vv{u_1} - \rho u_2 \Sigma_2 \vv{u_2}  
\end{gather*}
Allora la forza esercitata sulla pala è proprio
\begin{gather*}
    \vv{F} = -Q_M\vv{u_1}  \cdot (1 - \sqrt{q - \eta} )
\end{gather*}

\section{Ricavare la prima legge della dinamica per Fluidodinamica}
Date le leggi della Fluidodinamica
\begin{gather*}
    \frac{\partial \rho}{\partial t} + \vv{\nabla} (\rho \vv{u} ) = 0 \\
    \vv{u} \left(\rho\dots\right) 
\end{gather*}
si può ricavare la prima legge della dinamica per la Fluidodinamica lavorando
per un fluido incomprimibile
\begin{gather*}
    -\vv{\nabla}p \cdot \vv{u} - \rho \vv{\nabla}p\phi_G \cdot \vv{u}    
\end{gather*}
Dato che
\begin{gather*}
    \vv{\nabla} (a \cdot \vv{B} ) = \vv{B} \cdot \vv{\nabla}a + a\vv{\nabla}\cdot \vv{B}     
\end{gather*}
Da cui si può semplificare quella sopra come 
\begin{gather*}
    -\vv{\nabla}(p \vv{u} ) = \vv{u}\vv{\nabla}p + p\vv{\nabla}\vv{u}     
\end{gather*}
Ossia
\begin{gather*}
    -\vv{u}\vv{\nabla}p +p\vv{\nabla}\vv{u}   - \vv{\nabla}(\rho \phi_G \vv{u}  ) + \phi_G\vv{\nabla}(p\vv{u} ) 
\end{gather*}
Dato che si è in regime incomprimibile, allora si deve avere che la divergenza del vettore
$\vv{u}$ è nulla, allora posso riscrivere come 
\begin{gather*}
    -\vv{\nabla}(p\vv{u} )- \vv{\nabla}(\rho \phi_G \vv{u} ) - \frac{\partial \rho\phi_G}{\partial t}   
\end{gather*} 
Si svolge il seguente
\begin{gather*}
    \rho\vv{u}((\vv{u} \vv{\nabla}  )\vv{u} ) \ \Longrightarrow \ \sum_i \rho u_i \sum _k u_k\frac{\partial u}{\partial x_k} 
\end{gather*}
Si può provare ora a fare la divergenza di
\begin{gather*}
    \vv{\nabla}\left(\rho \frac{u^{2}}{2} \vv{u} \right) \ \Longrightarrow \ \sum_{k} \frac{\partial }{\partial x_k}  \rho \frac{u^{2}}{2}u_k = \frac{u^{2}}{2} \sum_k \frac{\partial \rho u_k}{\partial x_k} + \rho \sum _k u_k \frac{\partial }{\partial x_k}\frac{u^{2}}{2}  
\end{gather*}
Questa diventa proprio
\begin{gather*}
    \frac{u^{2}}{2}  \vv{\nabla} (\rho \vv{u} ) + \rho \sum_k u_k \sum_i \frac{\partial u_i^{2}}{2\partial x_k} = \frac{u^{2}}{2}\vv{\nabla}(\rho \vv{u} )   + \sum_k \rho u_k \sum_i u_i \frac{\partial u_i}{\partial x_k} 
\end{gather*}
Ottenendo
\begin{gather*}
    \frac{u^{2}}{2} \vv{\nabla}(\rho \vv{u} ) + \sum_i \rho u_i \sum_k u_k \frac{\partial u_i}{\partial x_k}  
\end{gather*}
L'ultimo termine è proprio
\begin{gather*}
    \rho\vv{u}((\vv{u} \vv{\nabla}  )\vv{u} )
\end{gather*}
Dunque
\begin{gather*}
    \vv{\nabla}\left(\rho \frac{u^{2}}{2} \vv{u} \right) = \frac{u^{2}}{2} \vv{\nabla}(\rho \vv{u} )+  \rho\vv{u}((\vv{u} \vv{\nabla}  )\vv{u} )
\end{gather*}
Combinando nell'equazione di continuità
\begin{gather*}
    \frac{\partial }{\partial t} \left(\frac{\rho}{2}u^{2} + \rho \phi_G\right) - \vv{\nabla} \left(\left(\frac{\rho}{2}u^{2} + p + \rho \phi_g\right) \vv{u} \right) = 0  
\end{gather*}
Si ottiene allora il principio di conservazione di energia della Fluidodinamica,
per cui si ottiene l'energia interna e nella seconda espressione, rispettivamente, il flusso
di energia cinetica del gas, il lavoro delle forze di pressione e di quella di gravità.
In condizioni stazionarie l'espressione dell'energia diventa
\begin{gather*}
    \vv{\nabla}\left(\left(\frac{\rho}{2}u^{2} + p + \rho \phi_G\right) \vv{u} \right) = 0 
\end{gather*}
Dato che lungo una linea di flusso deve essere costante, allora devo necessariamente avere
che
\begin{gather*}
    \vv{\nabla} \left(\rho \vv{u}\left(\frac{u^{2}}{2} + \frac{p}{\rho} + \phi_G\right) \right) 
\end{gather*}
È il teorema di Bernoulli applicato alla linea di flusso: ossia una conduttura con superficie
piccola che si approssima a linea di flusso. 

\section{Conservazione dell'energia meccanica in condizioni in cui esiste una legge barotropica}
Dall'equazione di continuità e dall'equazione di moto, supponendo di avere una legge
barotropica $p = p(\rho)$, è possibile definire una quantità differenziale
dividendo per la densità $w = \frac{dp}{\rho}$, ossia
\begin{gather*}
    \vv{\nabla} \frac{p}{\rho}  =\vv{\nabla} w
\end{gather*}
Allora l'equazione di moto
\begin{gather*}
    \left(- \rho \vv{\nabla}w + \phi_G \vv{\nabla}p  \right) \vv{u} \ \Longrightarrow \ \rho \vv{u} \cdot \vv{\nabla} w = \vv{\nabla}(\rho w \vv{u} ) - w\vv{\nabla}(\rho \vv{u} )   = \vv{\nabla}(\rho w \vv{u} ) + w\frac{\partial \rho}{\partial t}    
\end{gather*}
Ossia
\begin{gather*}
    \vv{\nabla}(\rho w \vv{u} ) + \frac{\partial \rho w}{\partial t} - \rho \frac{\partial w}{\partial t}    
\end{gather*}
Data ora la definizione di derivata rispetto al tempo di $w$, si ha che
\begin{gather*}
    \vv{\nabla}(\rho w \vv{u} ) + \frac{\partial (\rho w - p)}{\partial t}  
\end{gather*}
Adesso dentro l'equazione di moto si ottiene la seguente
\begin{gather*}
    \frac{\partial }{\partial t}(\frac{\rho}{2} u^{2} + \rho \phi_G - \rho w - p) + \vv{\nabla} \left(\left(\frac{\rho}{2} u^{2} + p\phi_G + \rho w\right) \vv{u} \right) = 0  
\end{gather*}
Posso dunque definire l'energia interna per unità di volume come 
\begin{gather*}
    U = \rho w - p
\end{gather*}
allora posso definire
\begin{gather*}
    w = \frac{U}{\rho} + \frac{p}{\rho}
\end{gather*}
Dunque $w$ è l'\textbf{entalpia specifica per unità di massa}.
Per un gas perfetto si ha che è definita come 
\begin{gather*}
    w = \frac{\gamma}{1 - \gamma} \frac{p}{\rho}
\end{gather*}



\end{document}
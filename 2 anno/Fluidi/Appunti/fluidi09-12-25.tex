\documentclass[a4paper, oneside]{article}
\usepackage{graphicx}
\usepackage{amsthm}
\usepackage{amsmath}
\usepackage{amssymb}
\usepackage[a4paper,
            bindingoffset=0.2in,
            left=2cm,
            right=2cm,
            top=2cm,
            bottom=2cm,
            footskip=.25in]{geometry}
\usepackage[italian]{babel}
\usepackage{pgfplots}
\usepackage{tabularx}
\usepackage{tikz}
\usepackage{wrapfig}
\usepackage{color}
\usepackage[d]{esvect}
\usepackage{chemfig}
\usepackage{mhchem}
\definecolor{page}{rgb}{0.129,0.157,0.212}
\pagecolor{page}
\color{white}
\graphicspath{ {./images/} }
\usetikzlibrary{shapes.geometric}
\usetikzlibrary{datavisualization}
\usetikzlibrary{datavisualization.formats.functions}
\usetikzlibrary{patterns}
\pgfplotsset{width=10cm,compat=1.18}

\title{Appunti di Fluidodinamica (Landi)}
\author{Tommaso Miliani}
\date{09-12-25}

\begin{document}
\newtheoremstyle{theoremEnv}
                {}          % Space above
                {}          % Space below
                {\slshape}  % Body font
                {}          % Indent amount
                {\bfseries} % Head font
                {.}         % Punctuation after head
                {\newline}  % Space after theorem head
                {}          % Theorem head spec
\theoremstyle{theoremEnv}

\newtheorem{definition}{Definizione}[section]
\newtheorem{theorem}{Teorema}[section]
\newtheorem{lemma}{Proposizione}[section]
\newtheorem{observation}{Osservazione}[section]
\newtheorem{corollary}{Corollario}[theorem]
\newtheorem{example}{Esempio}[section]
\newtheorem{remark}{Enunciato}[section]

\maketitle

\section{Vorticità}
La vorticità ($\vv{\omega} $), è definita come il rotore del campo
di velocità:
\begin{gather*}
    \vv{\omega} = \vv{\nabla} \times \vv{u}   
\end{gather*} 
Dove il rotore del vettore velocità è dato da
\begin{gather*}
    \vv{\nabla} \times \vv{a} = \det \begin{pmatrix}
        \hat{x} & \hat{y} & \hat{z} \\
        \partial_x & \partial_y & \partial_z \\
        a_x & a_y & a_z   
    \end{pmatrix}  = (\partial_y a_z - \partial_z a_z) \hat{x} + (\partial_z a_x - \partial_x a_z) \hat{y} + (\partial_x a_y - \partial_y a_x) \hat{z}   
\end{gather*}
Dove $\vv{a}$ è il vettore velocità generico e $\partial_i$ indica la derivata
parziale rispetto alla componente $i$ specificata. Se un fluido è in rotazione, allora la vorticità è diversa da zero: indica 
dunque se c'è rotazione in un fluido.
\begin{wrapfigure}{r}{0.4\textwidth}
    \centering
    \caption{}
    \begin{tikzpicture}
        \draw[->](-1, 0) -- (3, 0) node[at end, below] {$x$};
        \draw[->](0, -1) -- (0, 3) node[at end, left] {$y$};
        \draw[->](1, 0.5) arc (0:60:1);
    \end{tikzpicture}    
\end{wrapfigure}
Supponendo di avere un fluido che sta ruotando (rigidamente) intorno ad un asse,
posso dunque definire la velocità del fluido come
\begin{gather*}
    \vv{u} = \vv{\Omega} \times \vv{r} = \Omega \hat{z} \times (x\hat{x} + y\hat{y}  )    = \Omega x\hat{y} - \Omega y \hat{x}   
\end{gather*}
Dove il vettore posizione è dato da
\begin{gather*}
    \vv{r} = x\hat{x} -y \hat{y}   
\end{gather*}
Posso dunque ricavare il rotore del vettore velocità per definire la vorticità: (ricordando
che non c'è velocità lungo $\hat{z}$ ):
\begin{gather*}
    \vv{\omega}  = (\partial_x u_y - \partial_y a_x) \hat{z}  = (\Omega + \Omega) \hat{z} = 2\Omega \hat{z}  
\end{gather*}
Se si ha una situazione in cui si ha un campo che è funzione
solo della distanza, si può dimostrare che il rotore della velocità è 
identicamente nullo. Analogamente, un flusso lungo una certa direzione $x$, e che è funzione di $x$, 
sta accelerando o decelerando ha rotore nullo poiché il rotore compie 
sempre derivate incrociate. Nella presenza di una struttura a getto, la vorticità
è non nulla: questo perché la componente lungo $x$ dipende da quella 
lungo $y$: il suo termine di vorticità non nullo è sempre diretto 
lungo $z$. 

\subsection{Evoluzione dinamica della vorticità}
Tramite l'equazione di Eulero:
\begin{gather*}
    \rho \left(\frac{\partial \vv{u} }{\partial t} - (\vv{u} \cdot \vv{\nabla}  ) \vv{u} \right) = - \vv{\nabla} p - \rho \vv{\nabla} \phi_G 
\end{gather*}
Dove l'ultimo termine è un campo potenziale qualsiasi. Posso dunque riscriverla come
\begin{gather*}
    \frac{\partial \vv{u} }{\partial t} - (\vv{u} \cdot \vv{\nabla}  ) \vv{u}  = - \vv{\nabla} w - \vv{\nabla} \phi_G  
\end{gather*}
Dove $dw = \frac{dp}{\rho}$ è l'entalpia specifica per unità di massa. Da questa
posso utilizzare delle applicazioni di rotori e divergenze per ottenere una forma che 
mi permetta di descrivere dinamicamente l'evoluzione della vorticità. Posso utilizzare
la seguente identità:
\begin{gather*}
    \vv{\nabla} \times \vv{u} \times \vv{u}  = - \vv{\nabla} \frac{u^{2}}{2} - (\vv{u} \cdot \vv{\nabla}  ) \vv{u}  
\end{gather*}
Sulla componente $x$ posso ottenere
\begin{gather*}
   \left. (\vv{\nabla} \times \vv{u}  ) \times \vv{u} \right|_x = (\vv{\nabla} \times u )_y u_z - (\vv{\nabla} \times \vv{u}  )_z u_y = (\partial_z u_x - \partial_x u_z) u_z - (\partial_x u_y - \partial_y u_x) u_y 
\end{gather*}
Posso ora aggiungere e togliere $u_x \partial_x u_x$ e si ottiene dunque
l'espressione rispetto al gradiente:
\begin{gather*}
    u_x \partial_x u_x + u_y \partial_y u_x + u_z \partial_z u_x - u_x \partial_x u_x - u_y\partial_x u_y - u_z \partial_x u_z 
\end{gather*}
Che diventa, dato che è il gradiente e la divergenza rispetto a $x$:
\begin{gather*}
    (\vv{u} \cdot \vv{\nabla}  )u_x - \frac{\partial }{\partial x} \frac{u^{2}}{2} = (\vv{u} \cdot \vv{\nabla}  )u_x - \vv{\nabla} \frac{u^{2}}{2}_x  
\end{gather*}
Da questa relazione posso ottenere la seguente:
\begin{gather*}
    \frac{\partial \vv{u} }{\partial t} + (\vv{\nabla} \times \vv{u}  ) \times \vv{u}  = -\vv{\nabla}\left(\frac{u^{2}}{2} + w + \phi_G\right) 
\end{gather*}
Supponendo di essere ora in condizioni stazionarie, si fa il prodotto scalare con $\vv{u}$. Dunque posso espreimere
\begin{gather*}
    \vv{u} \cdot \left((\vv{\nabla} \times \vv{u}  ) \times \vv{u} \right) = - \vv{u} \cdot \vv{\nabla} \left(\frac{u^{2}}{2} + w + \phi_G\right)   
\end{gather*} 
Dunque, essendo prodotto triplo di vettori paralleli tra di loro, si ha la seguente 
\begin{gather*}
    \vv{u} \cdot \vv{\nabla} \left(\frac{u^{2}}{2} + w + \phi_G\right)   = 0
\end{gather*}
Il gradiente di questa espressione è dunque perpendicolare ad $\vv{u}$, ossia
$\vv{u}$ è parallelo all'isolivello della funzione $f$ descritta da quello dentro
le parentesi. Dunque lungo la linea di isolivello, questa quantità è costante in condizioni
stazionarie. Si può trascrivere ora la derivata rispetto al tempo 
del vettore velocità in modo tale da fare apparire la vorticità:
\begin{gather*}
    \frac{\partial \vv{u} }{\partial t} + \vv{\omega} \times \vv{u} = - \vv{\nabla} \left(\frac{u^{2}}{2} + w + \phi_G\right)    
\end{gather*}
Applicando l'operatore rotore a tutti e due i membri (questa operazione è legale) , è 
possibile invertire la derivata rispetto al tempo con l'operatore rotore (cioè scambiare l'ordine
degli operatori, che ,in questo caso, non cambia il risultato).
\begin{gather*}
    \frac{\partial }{\partial f} \vv{\omega} + \vv{\nabla} \times (\vv{\omega} \times \vv{u}  ) = - \vv{\nabla} \times \vv{\nabla} \left(\frac{u^{2}}{2} + w + \phi_G\right)     
\end{gather*}
Il secondo membro può essere espresso come il rotore del
gradiente di una funzione scalare, ossia zero (si può dimostrare ):
\begin{gather*}
    \vv{\nabla} \times \vv{\nabla}f =  \det \begin{pmatrix}
        \hat{x} & \hat{y} & \hat{z} \\
        \partial_x & \partial_y & \partial_z \\
        \partial_x f & \partial_y f & \partial_z f   
    \end{pmatrix} = (\partial_y \partial_z f + \partial_z \partial_y f)\hat{x}  + (\partial_z \partial_x f - \partial_x \partial_z f)\hat{y} + (\partial_x \partial_y f - \partial_y \partial_x f)\hat{z}  = 0
\end{gather*}
Dato che ci si pone nelle condizioni nel quale le derivate sono interscambiabili, allora 
il primo termine è nullo, così come il secondo e dunque anche il terzo. Questo vale sempre
per funzioni regolari. Dunque si ottiene il seguente risultato:
\begin{gather*}
    \frac{\partial \vv{\omega} }{\partial t} + \vv{\nabla} \times (\vv{\omega} \times \vv{u}  ) = 0  
\end{gather*}
Nelle condizioni in cui non c'è vorticità in tutto il fluido per un certo tempo, allora non si può generare vorticità: se il fluido non ha vorticità,
allora non se ne può generare poiché la derivata rispetto al tempo del vettore nullo è
esattamente nulla. Si ha inoltre un'altra identità vettoriale
\begin{gather*}
    \vv{\nabla} \times (\vv{A} \times \vv{B}  ) = - (\vv{A} \cdot \vv{\nabla}  ) \vv{B} + (\vv{B} \cdot \vv{\nabla}  ) \vv{A} + \vv{A} (\vv{\nabla} \cdot \vv{B}  ) - \vv{B}(\vv{\nabla} \cdot \vv{A}  )    
\end{gather*}
Senza dimostrazione. Utilizzando questa identità nella relazione precedente si ottiene:
\begin{gather*}
    \frac{\partial \vv{\omega} }{\partial t} + (\vv{u} \times \vv{\nabla}   ) \vv{\omega} - (\vv{\omega} \cdot \vv{\nabla}  )\vv{u} - \vv{u}(\vv{\nabla} \cdot \vv{\omega}  )   = 0
\end{gather*}
I primi due termini sono la derivata sostanziale, dunque, dato che
la divergenza del vettore $\vv{\omega}$, ossia la divergenza di un rotore di un campo vettoriale
$\vv{A}$ qualsiasi è zero (questo vuol dire che il vettore $\vv{\omega}$ non ha sorgente e
le linee di campo sono dunque sempre chiuse) allora si ottiene che
\begin{gather*}
    \frac{d\vv{\omega} }{dt} - (\vv{\omega} \vv{\nabla} )\vv{u} + \omega (\vv{\nabla} \cdot \vv{u}  ) = 0
\end{gather*}

\section{Teorema di Kelvin}
\begin{wrapfigure}{r}{0.4\textwidth}
    \centering
    \caption{}
    \begin{tikzpicture}
        \draw[->](0, 0) -- (3, 0);
        \draw[->](0, 0) -- (0, 3);
        \draw[->](0, 0) -- (-1, -1);
        \draw(2, 2) circle (1);
        \node at (2.75, 2) {$C(z)$};
    \end{tikzpicture}    
\end{wrapfigure}
Sulla curva $C$ si calcola $\Phi$, ossia il flusso della vorticità:
\begin{gather*}
    \Phi = \int_{\Sigma(C)}^{} \vv{\omega} \cdot \hat{n} \ d\sigma  
\end{gather*}
Dato che la divergenza di $\vv{\omega}$ è uguale a zero, allora non
cambia il tipo di superficie che si utilizza per quella determinata curva $C$
scelta.  Supponendo di avere una carica positiva con campo uscente, supponendo di
prendere una curva qualsiasi e di calcolare il flusso attraverso quella curva su 
di una superficie, in questo caso cambia il flusso poiché il campo 
in questione ha una sorgente ben definita. Si può ora
far evolvere il sistema con le equazioni di eulero e con le equazioni di identità,
gli elementi fluidi si sono mossi In generale, mi ricostruiranno una seconda curva $C$ al tempo
$t + \Delta t$. Dunque il flusso
\begin{gather*}
    \Phi (t + \Delta t) = \int_{\Sigma (C(t + \Delta t))}^{} \vv{\omega} \cdot \hat{n} \ d\sigma   
\end{gather*}
Si può dimostrare dunque, a partire da quell'equazione, il \textbf{Teorema di Kelvin}: ossia si può dimostrare
che il flusso $\Phi(t + \Delta t) = \Phi(t)$. Ossia il flusso della curva comovente
con gli elementi fluidi è uguale a quello della curva iniziale $C(t)$ iniziale. 
\begin{gather*}
    \Phi(t) = \int_{\Sigma(C)}^{} \vv{\omega} \cdot \vv{n} d \sigma = \int_{\Sigma(C)}^{}   (\vv{\nabla} \cdot \vv{u}  ) \vv{n} \ d\sigma 
\end{gather*}
Si può ora applicare il teorema di Stokes in modo tale da fare la somma di tutti i contributi lungo la curva
e dunque si ottiene il flusso al tempo $t$ come
\begin{gather*}
    \int_{C}^{} \vv{u}\cdot \vv{dl}   
\end{gather*}
Ossia la circuitazione di $\vv{u}$ intorno alla curva chiusa $C$. Supponendo di avere
un fluido che, invece di avere solo una caduta libera da un rubinetto, ha anche un moto 
rotatorio: prendendo una sezione circolare ad una certa altezza ed una ad un punto più basso ,evidentemente
il circuito comovente è più piccolo sulla superficie inferiore. Si può allora definire
l'integrale come
\begin{gather*}
    \int_{C(t)}^{} \vv{u} \cdot dl = \int_{C(t + \Delta t )}^{}  \vv{u} \cdot dl 
\end{gather*}
Dunque, se si accorcia la rotazione , dato che si conserva il momento angolare,
il fluido deve necessariamente accelerare la sua rotazione. 

\subsection{Fluido irrotazionale}
In un fluido irrotazionale
\begin{gather*}
    \vv{\omega} = \vv{\nabla} \times \vv{u} = 0   
\end{gather*}
Posso allora scrivere il vettore velocità come $\vv{u} = \nabla \psi$. Se 
trovo dunque una equazione per la funzione $\psi$, posso ottenere un 
fluido irrotazionale. In altre parole, se riesco a definire la funzione
$\psi$ che mi permette di costruire un campo $\vv{u}$ tale che il suo 
rotore sia nullo, allora avrò costruito un campo irrotazionale: 
ossia un fluido che non ruota. Una condizione che posso usare è che il fluido 
sia incomprimibile: ossia che la divergenza del vettore $\vv{u}$ sia nulla. 
Questa è una possibile soluzione:
\begin{gather*}
    \vv{\nabla}(\vv{\nabla} \psi ) = 0 
\end{gather*} 
Questo mi costruisce le linee di campo in modo che sia irrotazionale
ed incomprimibile. Posso esprimere questo come il Laplaciano quadro di $\psi$:
\begin{gather*}
    \vv{\nabla}(\vv{\nabla} \psi )  = \partial_x(\partial_x \psi) + \partial_y(\partial_y \psi) + \partial_z(\partial_z \psi) =  \partial_x^{2} \psi + \partial_y^{2} \psi + \partial_z^{2} \psi = \nabla^{2} \psi
\end{gather*}
Se è incomprimibile, allora è uguale a zero. Dunque ho definito delle 
linee di campo incomprimibili e irrotazionali. 

\section{Fluido che incontra un ostacolo (stazionario, irrotazionale e incomprimibile)}
\begin{wrapfigure}{r}{0.4\textwidth}
    \centering
    \caption{}
    \begin{tikzpicture}
        \draw[->](-2, 0) -- (2, 0) node[at end, below] {$x$};
        \draw[->](0, -2) -- (0, 2) node[at end, left] {$y$};
        \draw(0, 0) circle (1);
        \draw[->](-3, 1) -- (-2.5, 1) node[at end, below] {$\vv{u}$ };
    \end{tikzpicture}    
\end{wrapfigure}
Si hanno le seguenti condizioni per il fluido ideale considerato:
\begin{itemize}
    \item Il fluido ha una velocità $\vv{u}$ per distanze $d >> r_0$.
    \item Si vuole che $\left.\vv{u} \cdot \hat{r}\right|_ {r = r_0} = 0$.
    Se cambiassi oggetto mi cambierebbe la condizione del bordo. 
\end{itemize} 
La soluzione è
\begin{gather*}
    \psi = u_0 x \left(1 + \frac{r_0^{2}}{r^{2}}\right)
\end{gather*}
Dato che $u_x = \partial_x \psi$, allora
\begin{gather*}
    u_x = \partial_x \psi = u_0 \left(1 + \frac{r_0^{2}}{r^{2}}\right) - \frac{2u_0x^{2}r_0^{2}}{r^{4}} = u_0 \left(1 + \frac{(y^{2} - x^{2})r_0^{2}}{r^{2}}\right)
\end{gather*}
Dove $r$ è il modulo della distanza dall'origine del sistema di riferimento cartesiano. 
Se $\theta$ è l'angolo di impatto, si ha che
\begin{gather*}
    u_x = u_0 \left(1 + (\sin^{2}\theta - \cos^{2}\theta) \frac{r_0^{2}}{r^{2}}\right) \\
    u_y = -\frac{2u_0 xyr_0^{2}}{r^{4}} = -2u_0 \sin\theta\cos\theta \frac{r_0^{2}}{r^{4}}
\end{gather*}
In quel punto $u_y = 0$ così come $u_x$ quando $\theta = \pi$. Dunque
quello prende il nome di \textbf{punto di stagnazione}: il fluido si ferma, così
come nel caso di $\theta = 0$. Per $\theta = \frac{\pi}{2}$ invece, si ha che
solo moto orizzontale ma la velocità è due volte quella di partenza del fluido:
il fluido dunque accelera quando impatta con la sfera e decelera quando si allontana fino
a tornare alla stessa velocità di quella di partenza. Quando si è 
lontani dall'ostacolo il fluido si accorge sempre meno dell'ostacolo.
Dato che il fluido è in condizioni stazionarie, si deve avere che
la pressione rispetto al punto lontano aumenta nel punto in cui la velocità diminuisce;
nel punto tangente invece è più bassa. 
\begin{gather*}
    
\end{gather*}

\end{document}
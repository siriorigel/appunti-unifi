\documentclass[a4paper, oneside]{article}
\usepackage{graphicx}
\usepackage{amsthm}
\usepackage{amsmath}
\usepackage{amssymb}
\usepackage[a4paper,
            bindingoffset=0.2in,
            left=2cm,
            right=2cm,
            top=2cm,
            bottom=2cm,
            footskip=.25in]{geometry}
\usepackage[italian]{babel}
\usepackage{pgfplots}
\usepackage{tabularx}
\usepackage{tikz}
\usepackage{wrapfig}
\usepackage{color}
\usepackage[d]{esvect}
\definecolor{page}{rgb}{0.129,0.157,0.212}
\pagecolor{page}
\color{white}
\graphicspath{ {./images/} }
\usetikzlibrary{shapes.geometric}
\usetikzlibrary{datavisualization}
\usetikzlibrary{datavisualization.formats.functions}
\usetikzlibrary{patterns}
\pgfplotsset{width=10cm,compat=1.9}

\title{Appunti di fluidi}
\author{Tommaso Miliani}
\date{16-10-25}

\begin{document}
\newtheoremstyle{theoremEnv}
                {}          % Space above
                {}          % Space below
                {\slshape}  % Body font
                {}          % Indent amount
                {\bfseries} % Head font
                {.}         % Punctuation after head
                {\newline}         % Space after theorem head
                {}          % Theorem head spec
\theoremstyle{theoremEnv}

\newtheorem{definition}{Definizione}[section]
\newtheorem{theorem}{Teorema}[section]
\newtheorem{lemma}{Proposizione}[section]
\newtheorem{observation}{Osservazione}[section]
\newtheorem{corollary}{Corollario}[theorem]
\newtheorem{example}{Esempio}[section]

\maketitle

\subsection{Ridefinizioni di termostato e temperatura}
Il termostato diventa quindi un sistema la cui capacità termica
$\mathcal{C} \to \infty $ in quanto il termostato deve avere la 
capacità di non mutare la propria temperatura pur scambiando quantità di
temperature molto grande.  Un termometro, invece, è un sistema termodinamico
$\mathcal{C} \to 0$, in quanto deve poter scambiare quantità di calore molto piccole
e causare una sua variazione di temperatura grande. Si dice allora che
un qualsiasi sistema termodinamico $S_1$ è un termostato rispetto ad un
sistema termodinamico $S_2$ se $\mathcal{C}(S_1) >> \mathcal{C}(S_2)$; mentre
un sistema termodinamico $S_1$ si dice termometro rispetto ad un sistema
termodinamico $S_2$ se $\mathcal{C}{S_1} << \mathcal{C}(S_2)$. 

\section{Cosa accade agli scambi termici durante le trasformazioni}
\begin{wrapfigure}{r}{0.4\textwidth}
    \centering
    \caption{Coesistenza delle fasi durante un passaggio di fase}
    \begin{tikzpicture}
        \draw[->](0, 0) -- (4, 0) node[at end, below] {$t$};
        \draw[->](0, 0) -- (0, 4) node[at end, left] {$T$};
        \draw(0, 0) --(1, 1) -- (2.5, 1) -- (4, 2);
        \draw[dashed](1, 1) -- (1, 0);
        \draw[dashed](2.5, 1) -- (2.5, 0);
        \filldraw(1.75, 0) node[anchor = north] {coesistenza};
    \end{tikzpicture}    
\end{wrapfigure}
Se si vuole far avvenire un passaggio di stato si deve fornire dell'energia
o sottrarre dell'energia al materiale per cui vogliamo far avvenire il passaggio di 
stato. Durante il passaggio di stato, il fluido è in stato di coesistenza
tra le due fasi tra cui avviene il passaggio. Durante quel tempo, si definisce
il \textbf{calore latente} ($l$), ossia l'energia che si deve fornire al fluido
durante quel periodo di coesistenza per far sì che il corpo finisca
il passaggio di stato. I valori per il passaggio da liquido a gas e da solido a liquido
sono diversi tra di loro e si indicano con $l_v$ e $l_F$ rispettivamente e sono
proprie di ogni sostanza. Il calore latente di vaporizzazione è sempre più grande
di quello di fusione in quanto gli atomi hanno più movimenti e possono,
in media, avere pià legami tra di loro in quanto sono più liberi di muoversi; inoltre
per ottenere un gas devo necessariamente romperli tutti i legami
intermolecolari mentre per il passaggio da solido a liquido basta che
se ne rompano una quantità sufficiente tale per cui gli strati
del reticolo possano scorrere uno sopra l'altro. 

\subsection{La relazione tra energia interna e capacità termica a volume costante}
Se si decidesse di considerare l'energia interna come funzione
solamente del volume e della temperatura. Se io volessi scrivere il
differenziale di questa funzione:
\begin{gather*}
    U(V, T) \qquad dU = \left(\frac{\partial U}{\partial V} \right)_T dV + \left(\frac{\partial U}{\partial T} \right)_V dT
\end{gather*}
Il fatto che il valore di $T$ sia fissato quando derivo parzialmente il volume
è dovuto al fatto che le funzioni di stato potrebbero essere funzioni di
diverse variabili, allora devo specificare se la derivata parziale rispetto
al volume devo dire se è stata fatta a temperatura costante oppure a pressione
costante in quanto ho scelto la dipendenza funzionale di $U$ rispetto
a sole due variabili. Dato che
\begin{gather*}
    dU = \delta Q - \delta L \ \Longrightarrow \ \delta Q - \delta L = \left(\frac{\partial U}{\partial V} \right)_T dV + \left(\frac{\partial U}{\partial T} \right)_V dT
\end{gather*} 
Nel caso particolare in cui il volume sia costante, allora non compirò alcun lavoro sull'ambiente esterno e 
dunque posso esprimere il calore come 
\begin{gather*}
    \delta Q  = \left(\frac{\partial U}{\partial V} \right)_T dV + \left(\frac{\partial U}{\partial T} \right)_V dT
\end{gather*}
Dato che sono a volume costante, posso utilizzare la capacità termica e devo anche considerare
che $dV = 0$ poiché è una trasformazione a volume costante:
\begin{align}
        \mathcal{C}_V  dT = \left(\frac{\partial U}{\partial T} \right)_V dT \ \Longrightarrow \ \mathcal{C}_V  = \left(\frac{\partial U}{\partial T} \right)_V 
\end{align}
Questo risultato mi permette di dire che esiste una relazione diretta tra la
capacità termica a volume costante e la derivata parziale dell'energia interna
rispetto alla temperatura calcolata a volume costante.
Posso allora dire che questa è una funzione di stato (ma solo in questo caso particolare per un fluido). 


\section{Conseguenze del primo principio per i gas perfetti}
\subsection{Trasformazioni adiabatiche}
\begin{wrapfigure}{r}{0.4\textwidth}
    \centering
    \caption{Trasformazione adiabatica}
    \begin{tikzpicture}[scale = 0.5]
        \draw[very thick](0, 0) rectangle (4, 2);
        \draw[thick](1.5, 0) -- (1.5, 2);
        \filldraw[cyan, opacity = 0.3](0, 0) rectangle (1.5, 2);
        \draw[->](4.2, 1) -- (4.8, 1);
        \draw[very thick](5, 0) rectangle (9, 2);
        \filldraw[cyan, opacity = 0.3](5, 0) rectangle (9, 2);
        \node at(0.75, -0.5) {$V_A$};
        \node at(7, -0.5) {$V_B$};
    \end{tikzpicture}    
\end{wrapfigure}
Ricordando cosa succede nell'espansione libera di un gas il lavoro compiuto dal
sistema è esattamente zero e, dato che non viene fornito nessun calore al
gas, allora l'energia interna è anch'essa zero: vale
\begin{gather*}
    \Delta U = Q - L = 0
\end{gather*}
Se lo facessi con un gas qualunque e non un gas perfetto, allora la temperatura
varierebbe di un pochino. Nel caso di gas perfetto con $\Delta T = 0$ possiamo 
esprimere la trasformazione come:
\begin{gather*}
    (V_A, T) \to (V_B, T) \\
    (p_A, T) \to (p_B, T)
\end{gather*}
Se è vero che $\Delta U \neq 0$, allora vuol dire che la funzione
\begin{gather*}
    U(V_A, T) = U(V_B, T)
\end{gather*}
Siccome lo stato è variato ma l'energia interna non è variata, allora l'energia
interna non dipenderà da $V$. Posso fare il medesimo ragionamento per la pressione e
ottenere che
\begin{gather*}
    U(p_A, T) = U(p_B, T)
\end{gather*}
E dunque l'energia interna non dipenderà dalla pressione. L'energia interna è 
allora solo funzione delle temperatura per un gas perfetto. Se la temperatura
è costante allora l'energia interna è nulla per un gas perfetto e si ottiene l'utile
relazione
\begin{align}
    Q = L = \int_{V_i}^{V_f} p \ dV \ \Longrightarrow \ nRT\ln\left(\frac{V_f}{V_i}\right) = Q 
\end{align}
Nel caso di un gas perfetto il risultato ottenuto per la capacità termica
rispetto all'energia interna è generalizzabile come
\begin{align}
    \mathcal{C}_V = \frac{dU}{dT}
\end{align}
Dato che si è dimostrato che l'energia interna è funzione della temperatura,
posso dire che anche la capacità termica è funzione della temperatura.
\begin{align}
    \mathcal{C}_V = \mathcal{C}_V(T)
\end{align} 
Questo mi porta a dire che dalla relazione per l'energia
interna
\begin{gather*}
    dU = \delta Q - \delta L
\end{gather*}
Posso ottenere, per un gas perfetto le seguenti sostituzioni:
\begin{gather*}
    \mathcal{C}_V dT = \delta Q - pdV 
\end{gather*}
Quando la pressione è costante si ottiene
una relazione che lega le capacità termiche a 
volume e pressione costante:
\begin{gather*}
    \mathcal{C}_V dT = \mathcal{C}_P dT - pdV \\
    \ \Longrightarrow \ (\mathcal{C}_P - \mathcal{C}_V)dT = pdV
\end{gather*}
Adesso devo trovare una relazione che possa legare il volume alle altre
due variabili termodinamiche. Posso vedere che il volume è una funzione
con la seguente espressione:
\begin{gather*}
    V = V(p, T) \ \Longrightarrow \ dV = \left(\frac{\partial V}{\partial p} \right)_T dp + \left(\frac{\partial V}{\partial T} \right)_p dT
\end{gather*}
Allora la relazione tra le capacità termiche diventa (ricordando che siamo
a pressione costante):
\begin{gather*}
    \mathcal{C}_P - \mathcal{C}_V = p\left(\frac{\partial V}{\partial T} \right)_p 
\end{gather*}
Adesso posso anche esplicitare l'espressione per il volume ricordando
che 
\begin{gather*}
    V = \frac{nRT}{p} \ \Longrightarrow \ \mathcal{C}_P - \mathcal{C}_V = nR
\end{gather*}
Per un gas perfetto si ottiene allora il risultato generale:
\begin{gather*}
    \mathcal{C}_P - \mathcal{C}_V \geq 0
\end{gather*}
Questo è dovuto al fatto che se $\mathcal{C}_V$ è una certa funzione
rispetto alla temperatura, allora lo è anche $\mathcal{C}_P$ secondo 
la seguente:
\begin{gather*}
    \mathcal{C}_V = g(T) \ \Longrightarrow \ \mathcal{C}_P = g(T) + nR
\end{gather*}
Quello che ci dice la termodinamica per un gas perfetto, queste considerazioni
non mi bastano per poter determinare una equazione di stato.



\end{document}
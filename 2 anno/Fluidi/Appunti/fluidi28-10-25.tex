\documentclass[a4paper, oneside]{article}
\usepackage{graphicx}
\usepackage{amsthm}
\usepackage{amsmath}
\usepackage{amssymb}
\usepackage[a4paper,
            bindingoffset=0.2in,
            left=2cm,
            right=2cm,
            top=2cm,
            bottom=2cm,
            footskip=.25in]{geometry}
\usepackage[italian]{babel}
\usepackage{pgfplots}
\usepackage{tabularx}
\usepackage{tikz}
\usepackage{wrapfig}
\usepackage{color}
\usepackage[d]{esvect}
\definecolor{page}{rgb}{0.129,0.157,0.212}
\pagecolor{page}
\color{white}
\graphicspath{ {./images/} }
\usetikzlibrary{shapes.geometric}
\usetikzlibrary{datavisualization}
\usetikzlibrary{datavisualization.formats.functions}
\usetikzlibrary{patterns}
\pgfplotsset{width=10cm,compat=1.9}

\title{Termodinamica}
\author{Tommaso Miliani}
\date{28-10-25}

\begin{document}
\newtheoremstyle{theoremEnv}
                {}          % Space above
                {}          % Space below
                {\slshape}  % Body font
                {}          % Indent amount
                {\bfseries} % Head font
                {.}         % Punctuation after head
                {\newline}         % Space after theorem head
                {}          % Theorem head spec
\theoremstyle{theoremEnv}

\newtheorem{definition}{Definizione}[section]
\newtheorem{theorem}{Teorema}[section]
\newtheorem{lemma}{Proposizione}[section]
\newtheorem{observation}{Osservazione}[section]
\newtheorem{corollary}{Corollario}[theorem]
\newtheorem{example}{Esempio}[section]
\newtheorem{remark}{Enunciato}[section]

\maketitle

\section{Finire la dimostrazione}
\begin{wrapfigure}{r}{0.4\textwidth}
    \centering
    \caption{La macchina anti Clausius}
    \begin{tikzpicture}
        \draw(-0.5, 0) rectangle (1.5, 1) node[midway] {$\sim C + M$};
        \draw[very thick, ->](1.5, 0.5) -- (2.5, 0.5) node[at end, right] {$L = Q + Q_2$};
        \draw(-0.5, -1) rectangle (1.5, -2) node[midway] {$T_2$};
        \draw[very thick, ->](0.5, -1) -- (0.5, 0) node[midway, right] {$Q + Q_2$};
    \end{tikzpicture}    
\end{wrapfigure}
Si presuppone l'esistenza di un dispositivo anti Clausius che permetta
di trasferire calore da un termostato a temperatura inferiore a quello
con calore superiore che produce un certo lavoro. Dunque
questa macchina si riassume come se fosse un dispositivo Anti-Clausius 
che produce del lavoro dato del calore $Q + Q_2$, dove $Q_2 < 0$. Allora abbiamo dimostrato che se si nega
uno dei due teoremi, allora è negato anche l'altro. Dunque entrambe le
macchine che si sono create allora violano entrambi i principi: questo vuol
dire che le due formulazioni sono, in qualche modo, collegate tra di loro. 
\begin{gather*}
    \boxed{KP \ \Longleftrightarrow \ C}
\end{gather*}

\section{Il rendimento di una macchina termica: il rapporto tra il calore totale e quello assorbito}
Si è detto che la macchina termica può compiere solo scambi in un
certo verso, che succede però nel caso in cui una macchina
termica scambi calore con diversi termostati? Ci deve essere almeno un
termostato al quale cede del calore, altrimenti si violerebbero gli enunciati
del secondo principio della termodinamica; mentre è consentito prendere
calore da molteplici termostati. 
\begin{gather*}
    Q_1, \dots, Q_j \geq 0 \qquad  Q_{j + 1}, \dots, Q_n < 0 
\end{gather*}
Per i termostati $T_1, \dots, T_n$ con $ 1 < j  < n - 1$. Posso chiamare 
allora \textbf{calore assorbito}, il calore che la macchina termica prende dai
termostati
\begin{align}
    Q_a = \sum_{i = 0}^{j}  Q_i 
\end{align}
E \textbf{calore ceduto} il calore che la macchina termia cede a certi termostati
\begin{align}
    \sum_{i = j +1}^{n} Q_i 
\end{align}
Una macchina termica deve dunque convertire energia termica in energia meccanica, è quindi
possibile definire una quantità che mi permetta di dire quanto bene una macchina
termica riesce a trasformare il calore in lavoro che prende il nome
di \textbf{rendimento} o \textbf{efficienza} definito come 
\begin{align}
    \eta = \frac{L}{Q_a}
\end{align}
Se non si fa nessuna ipotesi su questa macchina, io posso solo sapere che
l'efficienza di una qualsiasi macchina termica sia positiva, se si suppone invece che
la macchina sia ciclica (come sarà da qui in poi), allora l'efficienza è uguagliabile
al rapporto tra il calore complessivo e quello assorbito, ossia
\begin{align}
    \eta = \frac{L}{Q_a} = \frac{Q}{Q_a} = \frac{Q_a + Q_e}{Q_a} = 1 + \frac{Q_c}{Q_a} = 1 - \frac{\left| Q_c \right| }{Q_a} 
\end{align}
Ossia l'efficienza dipende da quanto calore cede ed assorbe la macchina per una
macchina ciclica. Per il secondo principio della dinamica, dato che $Q_c \neq 0$, allora
l'efficienza di una macchina ciclica non potrà mai essere il $100\%$, dunque 
il rendimento per una macchina termica ciclica deve essere
\begin{gather*}
    0 \leq \eta < 1
\end{gather*}

\section{Macchine termiche che scambiano calore con due soli termostati}
Nel caso in cui una macchina termica (ciclica) scambi solo calore con due termostati,
in questa situazione è possibile definire il \textbf{Teorema di Carnot}.
\begin{remark}[Teorema di Carnot]
Fra tutte le macchine cicliche che compiono lavoro tra $T_1 < T_2$ si considerano le macchine
reversibili il cui rendimento è sempre 
\begin{align}
    \eta_R \geq \eta
\end{align}
Inoltre, se due macchine diverse sono entrambe reversibili, allora il
loro rendimento è equivalente:
\begin{align}
    \eta_{R'} = \eta_R
\end{align}
\end{remark}

\begin{wrapfigure}{r}{0.4\textwidth}
    \centering
    \caption{Macchina $R$ reversibile e macchina $M$ non reversibile}
    \begin{tikzpicture}
        \draw(0, 2) rectangle (3, 3) node[midway] {$T_1$};
        \draw(0, -1) rectangle (3, -2) node[midway] {$T_2$};
        \draw(0.5, 0.5) circle (0.5) node[anchor = center] {$R$};
        \draw[very thick, ->](0.5, 2) -- (0.5, 1) node[midway, left] {$Q_{1R}$};
        \draw[very thick, ->](0.5, 0) -- (0.5, -1) node[midway, left] {$Q_{2R}$};
        \draw(2.5, 0.5) circle (0.5) node[anchor = center] {$M$};
        \draw[very thick, ->](2.5, 0) -- (2.5, -1) node[midway, right] {$Q_{2M}$};
        \draw[very thick, ->](2.5, 1) -- (2.5, 2) node[midway, right] {$Q_{1M}$};
        \draw[very thick, ->](3, 0.5) -- (4, 0.5) node[at end, right] {$L_M$};
        \draw[very thick, ->](1, 0.5) -- (1.5, 0.5) node[midway, above] {$L_R$};
    \end{tikzpicture}    
\end{wrapfigure}
E' cruciale dire che questa cosa vale solamente nell'insieme
di queste due temperature assegnate. Se così non fosse, sarebbe
possibile costruire macchine termiche reversibili che hanno rendimento estremamente
piccolo, prossimo a zero, in questo modo posso dire che tutte le altre macchine
hanno rendimento zero, il che sarebbe assurdo.
Si dimostra ora determinando un certo numero di cicli per ogni 
macchina termica come $N_R$ il numero di cicli della macchina reversibile
e $N_M$ il numero di cicli della macchina non reversibile. Data la ciclicità delle
macchine, posso dire che la macchina reversibile assorbe un certo
calore e produce un certo lavoro
\begin{gather*}
    Q_{1R}' = N_R Q_{1R} \quad L'_R = N_R L_R \quad Q'_{2R} = N_R Q_{2R} \\
    Q_{1M}' = N_R Q_{1M} \quad L'_M = N_M L_M \quad Q'_{2M} = N_M Q_{2M} \\
\end{gather*}
E lo stesso per la macchina non reversibile. Adesso posso fare in modo che il calore che 
è assorbito dalla prima macchina sia lo stesso di quello della macchina non 
reversibile per ogni ciclo, allora posso aggiustare il numero di cicli per le macchine
per ottenere il calore assorbito da ogni macchina in quel numero di cicli e quindi
\begin{gather*}
    Q_{1R}' = N_R Q_{1R} = N_MQ_{1M} = Q'_{1M}
\end{gather*}
Posso allora chiamare i vari rendimenti totali come
\begin{gather*}
    \eta_M' = \frac{L_M'}{Q_{1M}'} = \frac{N_ML_M}{N_MQ_{1M}} = \eta_M
\end{gather*}
E la stessa cosa per la macchina rendimento. Si è dimostrato un corollario
del teorema:
\begin{observation}
    L'efficienza di una macchina non dipende dal numero di cicli che si esegue
    ma è una caratteristica intrinseca della macchina.
\end{observation}

\begin{wrapfigure}{r}{0.4\textwidth}
    \centering
    \caption{La macchina risultante}
    \begin{tikzpicture}
        \draw(0,0) ellipse (1 and 0.5) node[midway] {$M - R$};
        \draw[very thick, <->](1, 0) -- (2, 0) node[at end, above] {$L_M - L_R$};
        \draw[very thick, <->](0, -0.5) -- (0, -1.5) node[midway, right] {$Q_{2M} - Q_{2R}$};
        \draw(-1, -1.5) rectangle (1, -2.5) node[midway] {$T_2$};
    \end{tikzpicture}    
\end{wrapfigure}
Sfruttando l'ipotesi della macchina reversibile, la macchina $R$ subisce lavoro 
e dunque i segni dei calori si invertono. Questa diventa dunque una macchina frigorifera
che è in grado di cedere calore al termostato più caldo prendendolo da quello 
più freddo. L'effetto complessivo della macchina al contrario è quello della macchina
che produce come lavoro il lavoro complessivo e scambia come calore il calore complessivo.
Tuttavia, dato che abbiamo imposto che $Q_{1R} = Q_{1M}$, questa macchina scambia
solo calore con il termostato $T_2$, allora, per non andare contro al secondo principio
della termodinamica, devo necessariamente avere che
\begin{gather*}
    L_M - L_R \leq 0
\end{gather*}
E dunque
\begin{gather*}
    \frac{L_M}{Q_1} \leq \frac{L_R}{Q_1} \ \Longrightarrow \ \eta_R \geq \eta_M
\end{gather*}
Si dimostra ora la seconda parte dell'enunciato. Supponiamo dunque che quella macchina
$M$ sia reversibile, allora posso fare esattamente la stessa dimostrazione invertendo 
solamente la macchina $M$ e non la macchina $R$, ottenendo
\begin{gather*}
    \eta_M \geq \eta_R 
\end{gather*}
Ma dato che prima ho dimostrato il contrario, allora deve necessariamente risultare che
\begin{gather*}
    \eta_M = \eta_R
\end{gather*}

\section{Il ciclo di Carnot}
\begin{wrapfigure}{r}{0.4\textwidth}
    \centering
    \caption{Il ciclo di Carnot}
    \begin{tikzpicture}
        \draw[->](0, 0) -- (4, 0) node[at end, below] {$V$};
        \draw[->](0, 0) -- (0, 4) node[at end, left] {$p$};
        \filldraw(1, 3) circle (1pt) node[anchor = south] {$A$};
        \filldraw(2.25, 2.25) circle (1pt) node[anchor = south] {$B$};
        \filldraw(3.5, 1) circle (1pt) node[anchor = west] {$C$};
        \filldraw(1.5, 1.75) circle (1pt) node[anchor = north] {$D$};
        \draw[very thick, ->](1, 3) .. controls (1.5, 2.45) and (1.8, 2.35) .. (2.25, 2.25);
        \draw[very thick, ->](2.25, 2.25) .. controls (2.6, 1.5) and (3.25, 1) .. (3.5 ,1);
        \draw[very thick, ->](3.5, 1) .. controls (2.5, 1) and (2, 1.4) .. (1.5, 1.75);
        \draw[very thick, ->](1.5, 1.75) .. controls (1.4, 1.8) and (1.2, 2) .. (1, 3);
    \end{tikzpicture}    
\end{wrapfigure}
Presa una macchina termica reversibile ciclica, possiamo farle 
fare il ciclo di Carnot. I questa macchina termica si ha un gas perfetto
che dal punto $A$ si espande in maniera quasi statica (compiendo 
allora un arco di iperbole a temperatura costante) e dopo una espansione
adiabatica da $B$ a $C$ in modo tale che la sua temperatura non 
segua più una isoterma. Successivamente eseguo una compressione
quasi statica per cui la trasformazione $CD$ sia a temperatura costante
e chiudo il ciclo con una trasformazione $DA$ adiabatica quasi statica.
Quello che si sa è che l'energia interna $AB$ non cambia, così come 
l'energia interna per la trasformazione $CD$:
\begin{gather*}
    U(A) = U(B) \qquad Q_1 = L_{AB} = nRT_1\ln\frac{V_B}{V_A} \\
    U(C) = U(D) \qquad Q_2 = L_{CD} = nRT_2\ln\frac{V_D}{V_C}
\end{gather*}
Ossia la quantità $Q_1$ è il calore scambiato con il termostato a $T_1$ ed
il calore $Q_2$ quello con il termostato $T_2$. 
Si può ora determinare il rendimento della macchina come 
\begin{gather*}
    \eta = 1 + \frac{Q_C}{Q_A} = 1 + \frac{Q_2}{Q_1}
\end{gather*}
Per cui
\begin{gather*}
    \frac{Q_2}{Q_1} = \frac{T_2}{T_1} \frac{\ln \frac{V_D}{V_C}}{\ln\frac{V_B}{V_A}}
\end{gather*}
Si utilizza l'informazione della quasi staticità per poter trovare informazioni sulle relazioni
tra i due volumi. Conviene dunque utilizzare le informazioni sulle 
temperature per le trasformazioni adiabatiche per il gas perfetto ($TV^{\gamma - 1}$) e dunque 
posso esprimere 
\begin{gather*}
    T_1V_A^{\gamma - 1} = T_2V_D^{\gamma - 1}  \\
    T_1V_B^{\gamma - 1} = T_2B_C^{\gamma - 1}  
\end{gather*}
Allora posso ottenere il rapporto dei volumi come 
\begin{gather*}
    \left(\frac{V_A}{V_B}\right)^{\gamma -1} = \left(\frac{V_D}{V_C}\right)^{\gamma - 1}  \ \Longrightarrow \ \ln\frac{V_D}{V_C} = \ln\frac{V_B}{V_A}
\end{gather*}
Allora si conclude che
\begin{gather*}
    \frac{Q_2}{Q_1} = -\frac{T_2}{T_1}
\end{gather*}
Si esprime allora il rendimento in maniera esplicita per una macchina che compie il ciclo di 
Carnot:
\begin{align}
    \eta_R = 1 - \frac{T_2}{T_1}
\end{align}
Questa vale anche per i gas non perfetti (anche se non sono 
in grado di trovare i lavori per le varie trasformazioni).
Se si esegue il ciclo di Carnot in senso opposto, allora le quantità di calore
sono al contrario ed il lavoro è contrario e dunque la trasformazione 
è reversibile. Se il gas è perfetto le quantità di calore sono semplicemente date 
dalla relazione di stato dei gas perfetti e dunque è possibile determinare sia il calore 
scambiato che il lavoro eseguito.

\section{Misurare la temperatura termodinamica assoluta}
Con il termometro a gas perfetto non posso misurare tutte le temperature: non posso 
misurare temperature sotto la soglia della temperatura critica del gas con cui eseguo la misura.
Con una macchina termica è possibile però eseguire una misura della temperatura di un termostato
con temperatura incognita $T$ se affianco la macchina ad un termostato a 
temperatura $T_0$:
\begin{gather*}
    T = T_0 \frac{|Q|}{|Q_0|}
\end{gather*}
In linea di principio si potrebbe scegliere questa come definizione
operativa di temperatura che si distingue dall'altra 
poiché al posto di $T$ si usa $\theta$:
\begin{gather*}
    \theta = \theta_0 \frac{|Q|}{|Q_0|}
\end{gather*}
Adesso posso utilizzare il rendimento
\begin{gather*}
    \eta = \ - \frac{|Q_2|}{Q_1} = 1 - \frac{\theta_2}{\theta_1}
\end{gather*}
Che non è altro che la definizione di temperatura utilizzando il rendimento
(ossia la temperatura termodinamica assoluta)
di una macchina di Carnot. Dunque questi due metodi per definire la
temperatura stanno tra di loro come
\begin{gather*}
    \frac{\theta_2}{\theta_1} = \frac{T_2}{T_1}
\end{gather*}
Ponendo $\theta_0$ una temperatura comoda di riferimento (come il punto triplo dell'acqua),
posso utilizzare una macchina termica che lavori tra questa sorgente e $\theta_0$ e
la sorgente che voglio determinare a temperatura $\theta$. 
Non è possibile raggiungere lo zero termico nella scala Kelvin in quanto, se fosse possibile,
una macchina termica che compie del lavoro ed è posta a contatto con due
termostati, con il termostato allo zero termico scambierebbe esattamente zero calore e dunque
violerebbe il secondo principio della termodinamica poiché lavorerebbe cedendo calore 
alla sorgente a temperatura $\theta_0$. D'ora in poi si utilizza $\theta = T$. 

\end{document}
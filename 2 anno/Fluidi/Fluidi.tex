\documentclass[a4paper, oneside]{book}
\usepackage{graphicx}
\usepackage{amsthm}
\usepackage{amsmath}
\usepackage{amssymb}
\usepackage[a4paper,
            bindingoffset=0.2in,
            left=2cm,
            right=2cm,
            top=2cm,
            bottom=2cm,
            footskip=.25in]{geometry}
\usepackage[italian]{babel}
\usepackage{pgfplots}
\usepackage{tabularx}
\usepackage{tikz}
\usepackage{wrapfig}
\usepackage{color}
\usepackage[d]{esvect}
\definecolor{page}{rgb}{0.129,0.157,0.212}
\pagecolor{page}
\color{white}
\graphicspath{ {./images/} }
\usetikzlibrary{shapes.geometric}
\usetikzlibrary{datavisualization}
\usetikzlibrary{datavisualization.formats.functions}
\pgfplotsset{width=10cm,compat=1.9}

\title{Appunti di Fluidi/Termodinamica/Statistica}
\author{Tommaso Miliani}
\date{2025/2026}

\begin{document}
\newtheoremstyle{theoremEnv}
                {}          % Space above
                {}          % Space below
                {\slshape}  % Body font
                {}          % Indent amount
                {\bfseries} % Head font
                {.}         % Punctuation after head
                {\newline}         % Space after theorem head
                {}          % Theorem head spec
\theoremstyle{theoremEnv}

\newtheorem{definition}{Definizione}[chapter]
\newtheorem{theorem}{Teorema}[chapter]
\newtheorem{lemma}{Proposizione}[chapter]
\newtheorem{observation}{Osservazione}[chapter]
\newtheorem{corollary}{Corollario}[theorem]
\newtheorem{example}{Esempio}[chapter]

\maketitle

\tableofcontents

\chapter*{Introduzione al corso}
Il corso di Fluidi si divide in tre parti 
\begin{enumerate}
    \item Termodinamica: studia processi non di carattere meccanico che si manifestano a livello
    macroscopico
    \item Statistica: la fisica statistica coniuga la meccanica con la statistica che si usa nelle applicazioni
    che vanno al di là della fisica classica: da un fondamento microscopico agli effetti della termodinamica.
    \item Meccanica dei fluidi: sistemi materiali macroscopici per lo studio dei fluidi.
\end{enumerate}
Prima di tutto si inizia con un cappello che collega le tre parti;
poi si inizia con statica dei fluidi, dopo termodinamica, poi statistica e infine
dinamica dei fluidi. Questo perché alcuni strumenti di analisi due saranno
utilizzati per la parte di dinamica dei fluidi e statistica. \\
L'esame è diviso in tre domande orali (una per sezione) (niente prova scritta) con esercizi
applicati alla realtà e di effetto pratico (che sono fatti a lezione).
Il libro consigliato è "Elementi di meccanica di fluidi" di Egidio Landi, per la parte di termodinamica
"Termodinamica" di Enrico Fermi e per la parte di fisica statistica le dispense del prof
in quanto non ci sono libri che facciano al caso del corso. \\
Il vantaggio di studiare tre discipline in un corso è che si riesce ad avere una idea più
completa dei sistemi che si studiano.

\part{Statica dei fluidi}
\chapter{Lo studio dei fluidi}
\section{Introduzione alla statica dei fluidi}
Il caso più semplice della fisica è quello dell'approssimazione del
punto materiale; cosa succede allora se io aumentassi il numero di punti materiali
e li mettessi tutti insieme? Se si mettono insieme in modo da poter costituire un
corpo rigido allora avrei da risolvere diverse equazioni per poter determinare il comportamento
del corpo; nessun materiale reale gode di questa proprietà ma è solo una idealizzazione
che vale entro certe approssimazioni. Il modello del corpo rigido funziona molto
bene per i corpi solidi che in prima approssimazione sono ben descrivibili con questo modello, mentre
nel caso dei fluidi questa approssimazione non è valida. \\
Nei sistemi fluidi si considera allora il movimento degli atomi neutri
all'interno del fluido (non di plasmi) e quindi si può, utilizzando le conoscenze
di Fisica I, determinare come questi si muovono.

\subsection{L'atomo di Feynman}
In questo studio dei fluidi utilizziamo la definizione di atomo così come la aveva data Feynman: gli atomi sono dei punti materiali
che si attraggono se sono sufficientemente vicini ma se troppo vicini si respingono.
Essenzialmente a livello macroscopico l'atomo è visto come piccole palline che costituiscono il corpo. Possiamo
allora definire diverse proprietà della materia che utilizziamo per la descrizione dei fluidi:
\begin{itemize}
    \item Gli atomi sono dei punti materiali che hanno una certa massa $m$ che obbediscono alle leggi della meccanica
    di Newton;
    \item Un campione di materia è un insieme di $N$ atomi (ossia un numero molto grande) che
    possono essere chiusi in un contenitore ideale (ossia con pareti rigide e fisse).  
    \item Gli atomi interagiscono tra di loro solo tramite forze conservative, ossia che possono
    essere derivate e che queste forze dipendono esclusivamente dalla distanza tra
    gli atomi stessi; questo vuol dire che compiono solo urti completamente elastici con il contenitore.
\end{itemize}
Posso allora dare una definizione concreta alle forze intermolecolari, iniziando definendo $r_0$ come la distanza caratteristica
che ha definito Feynman oltre la quale gli atomi si attraggono ma entro la quale si respingono.
\begin{itemize}
    \item Se la distanza tra i due atomi è maggiore ma dello stesso ordine di $r_0$,
    allora le forze sono attrattive.
    \item Se la distanza è minore di $r_0$, allora la forza è repulsiva. 
    \item Le interazioni decadono molto velocemente con l'aumentare della distanza.
\end{itemize}

\subsection{Il potenziale di interazione}
Posso allora definire il \textbf{potenziale di interazione} tra due atomi come $V(r)$ e 
quindi posso relazionare le tre condizioni precedenti con il potenziale, il quale deve diventare 
costante quando i due oggetti si trovano a grandi distanze in quanto la forza diventa trascurabile.
Posso fissare un sistema di riferimento inerziale e dunque
posso dire che l'energia totale del sistema sarà data da
\begin{align}
    E = K + V\ \Longrightarrow \ \frac{1}{2}\sum |\vv{v}_i|^{2}  + U(\vv{r}_1, \dots, \vv{r}_n  )
\end{align}

\begin{wrapfigure}{r}{0.4\textwidth}
    \centering
    \caption{Il potenziale di interazione}
    \begin{tikzpicture}
        \draw[->](-1, 0) -- (4, 0) node[at end, below] {$r$};
        \draw[->](0, -2) -- (0, 2) node[at end, left] {$V(r)$};
        \draw(0.5, 2) .. controls (1.2, -2) and (1.7, -2) ..  (3, 0);
        \draw(3, 0) -- (4, 0);
        \draw[dashed](4, 0) -- (5, 0);
        \draw[dashed](1.55, 3) -- (1.55, -2) node[at end, right] {$r_0$};
    \end{tikzpicture}    
\end{wrapfigure}
Data l'ipotesi che ci siano solo delle forze conservative all'interno del sistema
che studiamo, allora posso riscrivere la componente potenziale (la quale dipende da $n$ variabili),
è scrivibile come la somma di tutte le possibili coppie e dunque
\begin{align}
    U(\vv{r}_1, \dots, \vv{r}_n  ) = \sum_{i = 0}^{n} \sum_{j = i + 1}^{n}  V(\left| \vv{r}_i - \vv{r}_j   \right| )
\end{align}
Posso anche esprimere anche l'energia potenziale attraverso la sommatoria rispetto a tutte le coppie
escludendo l'elemento già considerato (quindi si dimezza la sommatoria):
\begin{align}
    U(\vv{r}_1, \dots, \vv{r}_n  ) = \frac{1}{2}\sum_{i = 1}^{n} \sum_{j = 1 (i \neq j)}^{n} V(\left| \vv{r}_i - \vv{r}_j   \right| )  
\end{align}
I termini contano solo se la distanza è minore $a$ volte $r_0$: fissato allora un istante di tempo
posso scegliere un atomo e dire quali coppie possono effettivamente contribuire
all'energia potenziale e quindi ogni atomo interagisce solo con gli atomi vicini che si trovano
nella sfera (e quindi posso esprimerlo come $a \cdot  r_0$), posso allora approssimare con un pochino meno termini
rispetto a $\frac{N(N -1 )}{2}$ in quanto ogni atomo interagisce solo con gli altri
all'interno della sfera di interazione.
\begin{gather*}
    U(\vv{r}_1, \dots, \vv{r}_n  ) = \frac{1}{2}\sum_{i = 0}^{n}\sum_{j \in S(i)}^{n}V(\left| \vv{r}_i - \vv{r}_j   \right| )  
\end{gather*}
Quanti sono allora gli atomi massimi consentiti all'interno della sfera di interazione
$S(i)$ rispetto ad un dato atomo $j$? Se prendessi allora questa sfera di raggio
$a \cdot r_0$, si ottiene una stima che dipende da $r_0$ e lo divido per il volume della singola sferetta
(il quale è sempre una approssimazione per eccesso perché non considero che siano rigide) per determinare il
numero di atomi $n_s$ che interagiscono con l'atomo.
La sfera di interazione ha raggio $a \cdot r_0$ (dove $a$ è un numero) e il singolo atomo ha raggio $\frac{r_0}{2}$
e quindi devo fare il rapporto tra i volumi delle sfere e ottengo $n_s \approx (2a)^{d}$, dove $d$ è la dimensione
(piano = 2, spazio = 3). Si ottiene allora che per $a = 2.5$, $d = 2$ si ha $n_s \approx 25$, per $d = 3$ e $n_s \approx 125$.
La dimensione è legata a partire dal volume della sfera, infatti il raggio è elevato alla dimensione
dello spazio considerato. \\
L'energia potenziale degli atomi nella materia è direttamente proporzionale al numero di atomi
infatti si ottiene che è proporzionale a $Nn_s$ invece che $N^{2}$ come si era detto prima (abbiamo allora escluso $N$ dati). Quando una grandezza soddisfa questa proprietà allora
è una grandezza \textbf{estensiva}: la maggioranza dei termini non conta e dunque l'energia di un sistema è
una grandezza estensiva, se così non fosse allora la materia non si comporterebbe come si comporta.


\subsection{L'interazione tra due parti di uno stesso contenitore}
\begin{wrapfigure}{r}{0.3\textwidth}
    \centering
    \caption{L'interazione tra due recipienti uniti}
    \begin{tikzpicture}
        \draw(0, 0) rectangle (3, 2);
        \filldraw (0.5, 1) circle (0pt) node[anchor = south] {$N_1$};
        \filldraw (2, 1) circle (0pt) node[anchor = south] {$N_2$};
        \draw(1, 2) -- (1, 0);
        \draw[dashed](1.2, 2) -- (1.2, 0);
        \draw[dashed](0.8, 2) -- (0.8, 0);
        \filldraw(4, 1.5) circle (0pt) node[anchor = south] {$N_1N_2 \propto L^{d} $ };
        \filldraw(4, 1) circle (0pt) node[anchor = south] {$N_{1,2} \propto L^{d - 1} $ };
    \end{tikzpicture}    
\end{wrapfigure}
Immaginando di suddividere il contenitore in due parti allora avremo un po' del fluido che sta da una parte e un
po' che sta dall'altra: questa divisione va fatta in modo tale che le due parti debbano essere macroscopiche
(non pochi atomi da una parte e il resto dall'altra). Come posso determinare l'energia di questo sistema?
Dato allora il sistema $N_1 + N_2 = N$, ossia il numero di atomi del primo più il secondo compartimento
mi dà gli atomi totali (con $N >>>> 1$).  Devo dividere il contenitore
in modo tale che il "divisore" sia di una dimensione in meno dello spazio considerato (es nello spazio un piano)
in modo tale che sia uniforme. \\
Posso allora esprimere l'energia totale come
\begin{gather*}
    E = E_1 + E_2 + E_{1,2} \\
    E_1 =  \frac{1}{2}\sum_{i = 1}^{N_1}\left| \vv{v^{1} }_i  \right|^{2} + U (\vv{r}_1^{1} , \dots \vv{r}_2^{1}   )  \propto N_1  \\
    E_2 =  \frac{1}{2}\sum_{i = 1}^{N_2}\left| \vv{v^{2} }_i  \right|^{2} + U (\vv{r}_1^{2} , \dots \vv{r}_2^{2}   )  \propto N_2
\end{gather*}
Come si nota, esiste anche un contributo di interazione tra le due parti che posso esprimere come:
\begin{gather*}
    E_{1, 2} = \frac{1}{2}\sum_{i = 1}^{N_1}\sum_{j \neq i}^{N_2} V\left(\left| \vv{r^{N_1} }_i   - \vv{r^{N_2} }_j   \right| \right)  
\end{gather*}
Il contributo di questa interazione è molto più piccola dell'interazione $E_1$ e $E_2$ in quanto solamente gli atomi
molto vicini al bordo che divide i due sistemi interagiscono tra di loro (zona tratteggiata) e dunque
più cresce $N$ allora più diminuisce questo contributo. Inoltre si osserva che sono proporzionali ad un certo $L$ che è dato
dai limiti fisici del contenitori (chiamata scala libera):
\begin{gather*}
    L \propto N_1^{\frac{1}{d}} \ \Longrightarrow \ N_{1, 2} \propto N_1^{\frac{d- 1}{d}} \qquad \text{uguale per $N_2$}
\end{gather*}
Se $N_1$ cresce, allora $N_{1, 2}$ diventa molto piccolo.  Si ha che l'energia complessiva di un sistema
che ho arbitrariamente diviso in due pezzi avrà l'energia della somma dei due contributi ignorando il
termine di interazione tra le due parti purché siano macroscopici (infatti non sarebbe vero
se una delle due componenti avesse pochi atomi).
\begin{align}
    E \approx E_1 + E_2 \qquad \qquad per \  N >>>> 1 \ \Longrightarrow \ E = E_1 + E_2
\end{align}
Allora ottengo che l'energia è una \textbf{quantità additiva} (infatti è vero per $m$ divisioni fino a che
le componenti sono macroscopiche). Tutte queste considerazioni sono valide se e solo se
io continuo a ignorare i contributi all'esterno della sfera di interazione e considerando quindi che quei contributi convergano a zero in modo
sufficientemente rapido: infatti l'analisi ci insegna che somme di piccole quantità molto grandi non sempre
convergono a numeri finiti. Si può dimostrare che devono convergere a zero più velocemente di 
$r^{-d} $ (dove $d$ è nuovamente la dimensione dello spazio) (per l'interazione gravitazionale questo non è vero infatti). 

\section{Prototipo di funzioni che descrivono il potenziale di interazione}
\subsection{Funzioni di Leonard - Jones}
\begin{wrapfigure}{r}{0.4\textwidth}
    \centering
    \caption{La funzione di potenziale}
    \begin{tikzpicture}
        \draw[->] (-1, 0) -- (4, 0) node[at end, below] {$\frac{r}{r_0}$};
        \draw[->](0, -1) -- (0, 3) node[at end, left] {$\frac{V}{\epsilon}$};
        \draw(0.75, 3) ..controls (1, -2.5) and (1.35, -0.25) .. (2, -0.05);
        \draw(2, -0.05) -- (3.5, -0.05);
        \draw[dashed](3.5, -0.05) -- (4, -0.05);
        \filldraw(1, 0) circle (1pt) node[] {$\sigma$};
        \draw(1.2, 0) -- (1.2, -0.65) node[at end, below] {$r_0$};
    \end{tikzpicture}    
\end{wrapfigure}
Il potenziale di interazione è dato dalla seguente relazione
\begin{align}
    V(r) = \epsilon \left(\left(\frac{r_0}{r}\right)^{12} - 2\left(\frac{r_0}{r}\right)^{6} \right)
\end{align}
Il modello funziona molto
bene per i gas nobili in quanto sono atomi neutri che non interagiscono con nessun'altro atomo.
Anche per lo studio di casi più complessi il modello da seguire è essenzialmente questo entro certe approssimazioni;
spesso si riscrive in modo leggermente diversa questa formulazione, ossia nella seguente
forma 
\begin{align}
    V(r) = 4\epsilon \left(\left(\frac{\sigma}{r}\right)^{12} - \left(\frac{\sigma}{r}\right)^{6} \right)
\end{align}
Dove si ha che $r_0 = 2^{1/6} \cdot  \sigma \approx 1.12 \sigma$.
Sigma è il valore oltre al quale il potenziale diventa negativo, ossia il punto in cui
si azzera l'energia potenziale; il valore di $\epsilon$ per i gas nobili è dato da
\begin{gather*}
    \epsilon = 10^{-3} \sim 10^{-2} eV   
\end{gather*}
mentre il range di $\sigma$ (sempre per i gas nobili) è 
\begin{gather*}
    \sigma = 0.2 \sim 0.4 \ nm
\end{gather*}
L'interazione alla sesta è data dai termini di Van Der Walls che dipendono
dalle interazioni tra le nubi elettroniche; è possibile ricavare che il contributo attrattivo sia
di potenza $6$ mentre il contributo repulsivo è di potenza $12$ in quanto 
è computazionalmente comoda. Questo modello puramente meccanico ci permette di descrivere le interazioni
atomiche a livello approssimativo poiché non si può usare le leggi della meccanica classica.

\section{Studio del minimo dell'energia totale}
\subsection{Configurazione a minore energia}
Nello studio di sistemi conservativi si possono imparare un sacco di cose
dallo studio della funzione dell'energia potenziale senza andare a derivare;
per esempio in uno oscillatore armonico nel punto più
basso dell'energia potenziale il sistema sta fermo, se però si dà un po' di energia
al sistema, allora il sistema avrà una regione all'interno della quale si può muovere.
Dall'espressione dell'energia totale 
\begin{gather*}
    E = \frac{1}{2}m\sum_{i = 1}^{n}|\vv{v}_i|^{2} + U(\vv{r}_1, \dots, \vv{r}_n  )     
\end{gather*}
si può trovare il minimo dell'energia. Il livello minimo più intuitivo si ha quando tutte le particelle sono ferme,
tuttavia esisterà anche un altro punto di minimo che è più complicato da ricavare.
Riesprimendo il potenziale come
\begin{gather*}
    U = \frac{1}{2}\sum_{i = 1}^{n}\sum_{j = 1}^{n}V(|\vv{r}_i - \vv{r}_j|  )_{i \neq j}  
\end{gather*}
Trascurando l'effetto della differenza tra due e tre dimensioni e, al posto di utilizzare metodi analitici, cerchiamo di trovare il minimo attraverso
metodi fisici e matematici. Lo scopo è allora capire come si dispongono gli atomi nello spazio
per poter determinare il minimo dell'interazione tra di essi. \\
Possiamo innanzitutto ricondurci ad un caso molto semplice con soli due atomi $N = 2$.
L'energia potenziale è allora minima quando $r = r_0$, che si ottiene centrando il
sistema di riferimento nella particella 1 e ponendo la seconda a distanza $r_0$.\\
Se avessi ora tre particelle, avrei tre contributi energetici per il potenziale dato da
\begin{gather*}
    U = V(r_{1, 2}) + V(r_{1, 3}) + V(r_{2, 3})
\end{gather*}
\begin{wrapfigure}{r}{0.4\textwidth}
    \centering
    \caption{La configurazione a minore energia: il reticolo regolare}
    \begin{tikzpicture}
        \filldraw(0, 1) circle (0pt);
        \filldraw(0, 0) circle (2pt);
        \filldraw(1, 0) circle (2pt);
        \filldraw(-1, 0) circle (2pt);
        \filldraw(0.5, 0.75) circle (2pt);
        \filldraw(-0.5, 0.75) circle (2pt);
        \filldraw(0.5, -0.75) circle (2pt);
        \filldraw(-0.5, -0.75) circle (2pt);
    \end{tikzpicture}    
\end{wrapfigure}
Se la distanza tra tutte le particelle è esattamente $r_0$, allora il mio sistema
avrà energia potenziale minima. Da $N > 4$ le cose cambiano radicalmente
in quanto, essendo vincolati ad un piano, non c'è nessun modo per far sì
che le distanze tra tutte le particelle sia uguale. Devo disporre quindi tutte le particelle in modo
tale che stiano secondo uno schema simmetrico e regolare in quanto le particelle da una parte della configurazione
esercitano una certa forza mentre quelle che stanno dall'altra esercitano una forza
uguale e contraria per cui il contributo netto sulle singole
particelle è zero. Con questa disposizione regolare allora avremmo che la funzione presenta sicuramente un punto
stazionario anche se, secondo l'analisi, si avranno altri punti stazionari. Dobbiamo fare in modo che questa configurazione regolare si 
contribuisca con la minore energia possibile e quindi posso ripetere il pattern del triangolo
equilatero per poter ottenere la minore energia potenziale. \\
Si può dimostrare che il reticolo regolare è la configurazione con minore energia potenziale
purché il numero di particelle considerate sia molto grande rispetto ad $1$ anche se il minimo
vero è diverso poiché stiamo ignorando il contributo potenziale con
la parete. Possiamo ottenere tutte le posizioni delle particelle in un reticolo
come se fosse un sottospazio vettoriale con generatori 
\begin{gather*}
    \vv{a} = r_0 \hat{i} \\
    \vv{b} = \frac{r_0}{2}\left(-\hat{i} + \sqrt{3}\hat{j}   \right)\\
    \ \Longrightarrow \ \vv{r}_{n, m} = n\vv{a} + m\vv{b} \quad  n, m \in \mathbb{Z}      
\end{gather*}
Si può allora ottenere che per determinare il minimo per ogni particella si considerano le particelle
all'interno dell'esagono più vicino poiché abbiamo dimostrato che, in valore assoluto, hanno contributo maggiore
e dunque si ottiene che il minimo è proprio:
\begin{align}
    E_{mn} \approx \frac{1}{2}\sum_{i = 1}^{N} \sum_{j \neq i}^{6}V(r_0) = -3N\epsilon  
\end{align}
Il \textbf{cristallo} è la disposizione di minima energia possibile in quanto
la fisica quantistica fa prediligere le formazioni cristalline come
configurazione di energia minore possibile.

\subsection{Validità del modello entro un certo $\Delta E$ rispetto al minimo}
Dobbiamo ora cercare di convincerci che questo modello non 
funzioni solo per un modello con poche particelle ma anche per configurazioni abbastanza simili:
\begin{gather*}
    E = E_{min} + \Delta E
\end{gather*}
Posso applicare allora un po' di energia cinetica al nostro
modello attraverso il calore per poter ottenere un innalzamento dell'energia totale
(utilizzando l'approssimazione per la quale l'energia che applichiamo al sistema
è uniformemente condivisa tra tutti gli atomi). 
Possiamo prendere l'$i$-esimo atomo con energia pari a
\begin{gather*}
    \frac{1}{2}m|\vv{v}_i|^{2} \approx \frac{\Delta E}{N} \qquad \Delta E << E_{min} \qquad  \frac{1}{2}m|\vv{v}_i|^{2} << \epsilon    
\end{gather*}
Se sviluppassi con Taylor al secondo ordine la funzione di potenziale rispetto a $r$
si ottiene che
\begin{gather*}
    V(r_{i, j}) \approx -\epsilon + \frac{1}{2}V''(r_0)(r_j - r_0)^{2}  \qquad V''(r_0) = 72\frac{\epsilon}{r_0^{2} }
\end{gather*}
Tutte le coppie di atomi si comportano allora come oscillatori armonici e quindi si può approssimare
il modello come un network di oscillatori armonici trascurando le molle più lontane
e considerando solo quelle dei 6 atomi più vicini. Macroscopicamente si comporta come un solido in quanto per poterlo deformare
bisogna applicare forze molto grandi. Se si continua a far crescere l'energia, ad un certo
punto dobbiamo aggiungere termini successivi allo sviluppo di Taylor, il quale diventa più grande,
e il modello necessita dell'introduzione dei termini successivi. Da questo
si ha che più un sistema è caldo (e quindi più calore forniamo ad un sistema) 
e più questo si dilata. 

\subsection{Gli effetti di $\Delta E$ molto grandi}
\begin{wrapfigure}{r}{0.4\textwidth}
    \centering
    \caption{La buca di potenziale}
    \begin{tikzpicture}
        \draw[->] (-1, 0) -- (4, 0) node[at end, below] {$\frac{r}{r_0}$};
        \draw[->](0, -1) -- (0, 3) node[at end, left] {$\frac{V}{\epsilon}$};
        \draw(0.75, 3) ..controls (1, -2.5) and (1.35, -0.25) .. (2, -0.05);
        \draw(2, -0.05) -- (3.5, -0.05);
        \draw[dashed](3.5, -0.05) -- (4, -0.05);
        \draw[<->](1.3, 0) -- (1.3, -0.65);
        \draw[dashed](-0.5, 1) -- (3, 1);
    \end{tikzpicture}    
\end{wrapfigure}Aumentando sempre di più $\Delta E$ il modello avrà un asintoto verticale e
quindi se applico una energia maggiore della differenza tra il minimo e zero
allora gli atomi non saranno più legati tra di loro e quindi il sistema inizierà
a comportarsi come un gas e ogni particella inizierà a muoversi
indipendentemente dalle altre.
Come si vede dal grafico, se aumenta l'energia, i valori possibili di energia
per $r_0$ tende ad essere uno solo, ossia nella parte del grafico dove l'energia
potenziale è repulsiva. \\
Il sistema si comporterà come un solido cristallino se e solo se l'energia delle
particelle è minore di $\epsilon$ e non supera la buca di potenziale; se superassi allora questa
buca di potenziale si comporterà come un gas. Ovviamente esiste anche la situazione di mezzo
per cui il corpo si comporta come un liquido.
A livello macroscopico la differenza tra un liquido ed un solido
non è la comprimibilità ma lo scorrimento delle molecole l'una sulle altre.

\section{Le proprietà microscopiche dello stato liquido}
\begin{wrapfigure}{r}{0.4\textwidth}
    \centering
    \caption{}
    \begin{tikzpicture}
        \filldraw(0, 1) circle (0pt);
        \filldraw(0, 0) circle (2pt);
        \filldraw(1, 0) circle (2pt);
        \filldraw(-1, 0) circle (2pt);
        \filldraw(0.5, 0.75) circle (2pt);
        \filldraw(-0.5, 0.75) circle (2pt);
        \filldraw(0.5, -0.75) circle (2pt);
        \filldraw(-0.5, -0.75) circle (2pt);
    \end{tikzpicture}    
\end{wrapfigure}
Considerando un esagono di atomi dal reticolo cristallino e riprendendo allora la definizione che si è data al reticolo,
ci poniamo dunque nelle ipotesi di avere ancora un solido.
Concentrandosi sull'atomo centrale, possiamo allora chiedersi come si può muovere questo assumendo che gli altri
atomi siano in una posizione fissata. Questo diventa il problema del moto di un punto
materiale in due dimensioni: posso esprimere l'energia potenziale del sistema
intorno all'atomo considerato come
\begin{gather*}
    U(x, y) = \sum_{i = 1}^{n} V(|\vv{r} - \vv{r_i}|  ) 
\end{gather*}
In questo problema si ha una funzione di due variabili che dipende parametricamente
anche dalle altre dodici variabili (che posso allora fissare come voglio dato che
sono parametri e non variabili). L'energia meccanica sarà allora data dal contributo anche
cinetico
\begin{gather*}
    E = \frac{1}{2}m(v_x^{2}  + v_y^{2} ) + U(x, y)
\end{gather*}
Quale è la regione di piano accessibile al moto del nostro punto centrale? La regione accessibile
dall'atomo centrale è quella in cui l'energia potenziale sia minore o uguale all'energia meccanica totale. L'energia
cinetica in questo modello semplificato può essere applicata solamente all'atomo centrale: questo moto è descritto allora
da questa regione di movimento specifica. Che succede allora se gli altri sei atomi siano fermi ed in un
reticolo a bassa energia? La particella non si muove molto poiché è ingabbiata dagli altri atomi
anche nel caso in cui si fornisca una energia molto grande di $10\epsilon$.

\begin{wrapfigure}{r}{0.4\textwidth}
    \centering
    \caption{Situazione fisica}
    \begin{tikzpicture}
        \draw[->](-1.5, -2) -> (-1.5, 2);
        \draw[->](-2, -1.5) -> (2, -1.5);
        \draw (0,0.5) circle (0.25);
        \draw(1, 0) ..controls(0.6, -0.4).. (0.5, -0.8);
        \draw(0.5, -0.8) ..controls(0, -0.6).. (-0.5, -0.8);
        \draw(-1, 0) ..controls(-0.8, -0.2).. (-0.5, -0.8);
        \draw(-1, 0) ..controls(-0.8, 0.4).. (-1, 0.8);
        \draw(1, 0.8) ..controls(0.8, 0.5).. (1, 0);
        \draw(-0.3, 2) ..controls(-0.3, 1.2).. (-1, 0.8);
        \draw(0.3, 2) ..controls(0.3, 1.2).. (1, 0.8);
    \end{tikzpicture}    
\end{wrapfigure}
Se si considera la situazione più realistica secondo la quale tutti gli atomi
ricevono energia cinetica: in questo caso gli atomi tenderanno tutti ad agitarsi e dunque il raggio $r_0$
di interazione tenderà a crescere e il moto della particella risulterà maggiore rispetto
alla situazione ideale anche con solo $\frac{1}{2}\epsilon$. La particella ha quindi molto più margine
di movimento in quanto gli atomi in media sono più distanti tra di loro e ciascuno si muove di più rispetto
agli altri. Tra tutte le fluttuazioni possibili se due degli atomi del reticolo
intorno si muovono in direzioni opposte, allora il margine di movimento della particella ha 
di punto in bianco una regione di movimento maggiore e si crea un canale di uscita; 
anche se la situazione più probabile è quella che la particella centrale si scambia di posizione con una del reticolo intorno.
Se si aspetta abbastanza ogni particella potrebbe percorrere tutto il reticolo (e questa è proprio la situazione
che accade nel liquido!) senza dispendio di energia anche se, in un dato istante, si volesse osservare cosa accade,
ogni particella apparirà circondata da altre 6 particelle. 

\subsection {Riassunto delle proprietà delle fasi e ipotesi di validità}
Per ogni fase noi consideriamo una energia minima in cui si ha il
reticolo perfetto di un materiale e aggiungiamo un certo $\Delta E$: 
\begin{align*}
    &\text{Solido} &\frac{\Delta E}{N} &\llless \epsilon \\
    &\text{Liquido} &\frac{\Delta E}{N} &\approx \epsilon \\
    &\text{Gas} &\frac{\Delta E}{N} &\gtrapprox 3\epsilon 
\end{align*}
Tutte queste ipotesi valgono solo quando la distribuzione degli
atomi è omogenea e la distribuzione di energia applicata al sistema sia
uniforme a tutti gli atomi. Gli stati della materia dunque sono omogenei
in quanto sono interazioni a corto raggio mentre interazioni a lungo raggio
come la forza di gravità non è omogenea. Questo modello vale
solo ed esclusivamente nel caso in cui gli atomi sono approssimabili a piccole sferette
anche se a livello macroscopico potremmo anche assumere gli atomi come 
puntiformi in quanto in fisica, gli eventi che avvengono ad una data scala sono
influenzati solo dalle scale immediatamente vicine ed ignorano (salvo varie
eccezioni) ciò che accade alle scale più lontane. Le dimensioni contano in quanto è
possibile ricavare le leggi per livelli superiori partendo dalle leggi del 
piccolo però dalle leggi macroscopiche non si può ricavare le leggi microscopiche. 

\subsection{Determinare gli effetti probabilistici nel sistema}
Considerando la forza applicata ad ogni oggetto
\begin{gather*}
    \vv{F_i} = m\vv{a_i} \qquad i = 1, \dots, N \qquad N \ggg 1  
\end{gather*}
Per esempio in un bicchiere d'acqua ci sono $8.4 \cdot 10^{24}$ atomi all'interno
del bicchiere e quindi questo problema è perfettamente determinato attraverso le leggi che
abbiamo definito. L'unico problema è che abbiamo $10^{25}$ punti materiali nel nostro
sistema e computazionalmente è molto scomodo. Supponiamo che per ogni punto materiale utilizziamo
$1$Byte di informazioni, ossia $10^{13}TB$ solo per tenere in memoria le posizioni dei punti materiali.
Le prime persone che hanno provato a calcolare le posizioni attraverso un calcolatore è
stato negli anni 50 con un set di dati di $N =32$ e $N = 64$ solo nel moto unidimensionale
per cercare di studiare la dinamica molecolare. \\
Con un supercomputer si potrebbe ora simulare sistemi dell'ordine di $10^{6}$, se il nostro sistema stesse in un cubo, allora segue
\begin{gather*}
    N \propto V = L^{3} \ \Longrightarrow \ L \propto N^{1/3}  
\end{gather*}
Allora il numero di atomi che sta sul bordo è proporzionale a $L^{2}$ e quindi 
$N_{bordo} \propto L^{2} \propto N^{2/3}$ e quindi
\begin{gather*}
    \frac{N_{bordo}}{N} \propto \frac{N^{2/3} }{N} = N^{1/3} 
\end{gather*}  
Più il sistema è piccolo e più gli effetti del brodo contano, anche
con soli $10^{6}$ atomi, avrò un errore di circa $10^{-2}$ nello studio
della dinamica molecolare. Se noi facciamo la dinamica molecolare in un piano
con punti materiali allora dovremmo rivedere le nostre relazioni utilizzando un 
software per poter simulare sistemi molecolari.  


\chapter{La fluidostatica}
\section{Piccola parentesi sulla fluidodinamica}
\subsection{Il concetto di fluido ideale}
\begin{wrapfigure}{r}{0.4\textwidth}
    \centering
    \caption{Il fluido diviso in due}
    \begin{tikzpicture}
        \draw(0, 0) .. controls (1, -0.5) and (2.5, 0.5) .. (3, -0.5);
        \draw(3, -0.5) .. controls (2.9, -1) and (2, -0.75) .. (1, -1);
        \draw(1, -1) .. controls (-1, -1.4) and (-0.5, 0.5) .. (0, 0);
        \draw[dashed](1, -0.2) .. controls (0.8, -0.55) .. (0.75, -1);
        \draw[dashed](1.3, -0.2) .. controls (1.3, -0.55) .. (1, -1);
        \draw[->, red](1.1, -0.5) -- (1.75, -0.5) node[at end, right] {$\hat{n}$  };
        \draw[->, cyan](1.1, -0.5) -- (1.75, 0) node[at end, above] {$\vv{F}$ };
    \end{tikzpicture}    
\end{wrapfigure}
La Fluidodinamica si occupa di definire come si comporta un fluido.
Un fluido è un corpo continuo formato da tanti puntini (molecole o atomi)
con massa e dimensione finita. Supponendo di dividere il fluido in due parti
$A$ e $B$ separate da una sezione, la cui superficie indichiamo con $\Sigma$. Devo avere  una qualche forza
che esercita la sezione $A$ sulla sezione $B$ sulla superficie di contatto $\Sigma$.
\begin{gather*}
    \vv{F_{AB}} = F_{\parallel}\hat{n} + \vv{F_{\perp}}    
\end{gather*}
Con il versore $\hat{n}$ perpendicolare alla superficie $\Sigma$ considerata e
il vettore $\vv{F_{\perp}}$ perpendicolare al versore $\hat{n}$; si definisce allora fluido se 
\begin{gather*}
    |F_{\perp}| \llless |F_{\parallel}| \qquad F_{\parallel} > 0
\end{gather*}
Posso considerare ora la forza che imprime $B$ su di $A$. Dato che conosco la forza
che imprime $A$ su $B$, allora posso dire che
\begin{gather*}
    \vv{F_{BA}} = -\vv{F_{AB}}  
\end{gather*}

\begin{wrapfigure}{r}{0.4\textwidth}
    \centering
    \caption{}
    \begin{tikzpicture}
        \draw(-1, 1.5)  -- (-1, 0)-- (3, 0) -- (3, 1.5);
        \draw(0.5, 0) rectangle (1.5, 1);
        \draw[->](0.5, 0.5) -- (-0.5, 0.5);
        \draw[->](1.5, 0.5) -- (2.5, 0.5);
    \end{tikzpicture}    
\end{wrapfigure}

La definizione di $F$ parallelo è una definizione che vale sempre anche
per l'altro senso in quanto il prodotto scalare con il versore uscente è sempre lo stesso.
E' proprio una cosa fisica: si suppone di prendere un contenitore e di porvi un fluido: questo
fluido spingerà da entrambe le direzioni esercitando una forza uscente rispetto alla
normale (ossia la sua pressione) . Esiste anche una proprietà per la quale
il fluido non si scompone in pezzi ma rimane sempre unito che è la
\textbf{tensione}. Un fluido può scorrere su di una superficie proprio
perché la sua forza perpendicolare è molto piccola; tuttavia il fluido ideale scorre
senza attrito: il fluido reale invece presenta attrito viscoso di taglio. Nella trattazione
dei fluidi in fluidodinamica utilizziamo solamente l'ipotesi
di fluido ideale che scorre senza attrito.  

\subsection{Definizione del campo scalare della pressione}
\begin{wrapfigure}{r}{0.4\textwidth}
    \centering
    \caption{}
    \begin{tikzpicture}
        \draw[->](0, 0) -- (1, -0.7) node[at end, below] {$\hat{x}$ }; 
        \draw[->](0, 0) -- (-1, -0.7) node[at end, below] {$\hat{y}$ };
        \draw[->](0, 0) -- (0, 1) node[at end, left] {$\hat{z}$ };
        \draw[->](0, 0) -- (3, 1.5) node[midway, below] {$\vv{r}$ };
        \draw(2.85, 1.35) rectangle (3.15, 1.65) node[anchor = north west] {$\Delta\Sigma$};
        \draw(1.7, 1.2) .. controls (2.8, 2.15) and (3.2, 1.9) .. (3.4 , 1.8);
        \draw(3.4, 1.8) .. controls (4, 1.6) and (3.8, 1.1) .. (3.2, 1);
        \draw(3.2, 1) .. controls (2, 0) and (1.4, 1) .. (1.7, 1.2);
    \end{tikzpicture}    
\end{wrapfigure}
Riprendendo il volume arbitrario (o una porzione di fluido ideale) 
posso considerare un sistema di riferimento con terna
destrorsa di versori e, preso un punto sulla superficie
del volume di fluido,posso identificarlo con un raggio
vettore $\vv{r}$  il punto sulla superficie del fluido, sulla sezione della superficie
$\Delta \Sigma$ io identifico la spinta verso l'esterno rispetto alla massa del volume
\begin{gather*}
    \Delta \vv{R} =  \overline{P} \Delta \Sigma \hat{n}  
\end{gather*}
Dove $\overline{P}$ è una quantità positiva definita come
\begin{gather*}
    \overline{P} = \frac{\Delta R}{\Delta \Sigma} 
\end{gather*} 
Ossia la pressione esercitata dal fluido verso l'esterno. Posso allora definire la pressione come
\begin{align}
    P(x, y, z) = \lim_{\Delta \Sigma \to 0} \frac{\Delta R}{\Delta \Sigma}  \quad \frac{[F]}{[L]^{2} } = \frac{[E]}{[L]^{3} }
\end{align}
Siamo allora passati da una quantità finita ad una quantità puntiforme: questa funzione è 
quindi univocamente definita in un punto dello spazio sulla superficie del fluido stesso. 
Questo è quello che si definisce un \textbf{campo scalare}. Un campo scalare è una funzione 
che associa uno scalare a ogni punto dello spazio. 
E' utile definire la pressione come energia per unità di volume (molto utile per lavorare
con l'energia in termodinamica).  

\subsection{Definizione del campo scalare della densità}
\begin{wrapfigure}{r}{0.4\textwidth}
    \centering
    \caption{}
    \begin{tikzpicture}
                \draw[->](0, 0) -- (1, -0.7) node[at end, below] {$\hat{x}$ }; 
        \draw[->](0, 0) -- (-1, -0.7) node[at end, below] {$\hat{y}$ };
        \draw[->](0, 0) -- (0, 1) node[at end, left] {$\hat{z}$ };
        \draw[->](0, 0) -- (3, 1.5) node[midway, below] {$\vv{r}$ };
        \draw(2.85, 1.35) rectangle (3.15, 1.65) node[anchor = north west] {$\Delta V$};
        \draw(1.7, 1.2) .. controls (2.8, 2.15) and (3.2, 1.9) .. (3.4 , 1.8);
        \draw(3.4, 1.8) .. controls (4, 1.6) and (3.8, 1.1) .. (3.2, 1);
        \draw(3.2, 1) .. controls (2, 0) and (1.4, 1) .. (1.7, 1.2);
    \end{tikzpicture}    
\end{wrapfigure}
Considerato il solito sistema di riferimento stavolta, invece di intercettare un punto
sulla superficie, prendo un punto interno al volume del fluido: invece di prendere
un elemento di superficie, prenderò un elemento di volume per cui posso misurare una
certa massa: posso allora scrivere la densità media di quel cubetto di fluido come
\begin{gather*}
    \overline{\rho} = \frac{\Delta m}{\Delta V} 
\end{gather*}
Ossia la densità media. Posso trasformarla in una quantità che è definita per ogni
punto del fluido e quindi ottenere il campo scalare della densità come
\begin{gather*}
    \rho(x, y, z) = \lim_{\Delta V \to 0} \frac{\Delta m}{\Delta V} 
\end{gather*}
Si definiscono allora, data la densità, due classi di fluidi: i liquidi, che sono incomprimibili,
ed i gas che sono invece dei fluidi comprimibili.  Infatti nei liquidi se prendiamo una sezione
molto piccola di fluido la densità non varia ma è sempre costante per tutto il liquido (fluido
incomprimibile) mentre per un gas la densità può cambiare a seconda della sezione di volume
considerata(fluido comprimibile).

\subsection{Comprimibilità}
Si può stimare la comprimibilità del fluido spingendolo con una certa pressione:
\begin{gather*}
    \frac{\Delta P}{\epsilon} = - \frac{\Delta V}{V_0}
\end{gather*}
Ossia ci si aspetta una compressione del volume a seguito dell'applicazione di una
certa pressione $P$ in modo uniforme su tutta la superficie del fluido. Chiamato allora $\epsilon$ il fattore
di comprimibilità, questo definisce la comprimibilità del fluido considerato:  più è grande e più il fluido è incomprimibile. 
A condizioni standard la comprimibilità dell'aria è circa $10^{5} \ Pa$ mentre l'acqua è più
incomprimibile dell'aria con un $\epsilon = 10^{9} \ Pa$.

\subsection{La resistenza all'espansione di un fluido}
Se volessi trovare l'equilibrio delle forze del fluido potrei considerare un
sistema di riferimento ed una piccola porzione di superficie del fluido. Questa porzione del
fluido agisce in modo tale da imprimere una forza infinitesima
\begin{gather*}
    d\vv{F_{\Sigma}} = -p\hat{n}d\sigma  
\end{gather*}
per resistere alla forza che vorrebbe fare espandere il fluido, mentre $d\sigma$ è un elemento di 
superficie infinitesimo. Un fluido
cercherà allora di mantenere i legami tra le sue molecole ad una pressione esterna.
La risultante delle forze di superficie sarà allora la somma di tutti i contributi delle
forze sulla superfici. Calcolando allora la reazione che imprime la parte
interna del liquido sulla superficie totale 
\begin{gather*}
    \vv{R_{\Sigma}}  = -\oint_{\Sigma(V)} p\hat{n} \ d\sigma 
\end{gather*}
Per ogni punto posso allora associare una quantità vettoriale e dunque definisco
un campo vettoriale derivante dalle forze esterne che agiscono sul fluido:
\begin{gather*}
    d\vv{F_V} = \vv{f}(x, y, z)dV  
\end{gather*}
quindi
\begin{gather*}
    \vv{R_V} = \int_{V} \rho \vv{g}\ dV  
\end{gather*}
dove $\vv{g}$ è una accelerazione generica. 
Dato che ora voglio l'equilibrio di un volume finito di fluido, ora impongo:
\begin{gather*}
    -\oint_{\Sigma(V)} p\hat{n} \ d\sigma + \int_{V} \rho \vv{g}\ dV   = 0 
\end{gather*} 
Ossia l'equazione di \textbf{equilibrio idrostatico}. Si può risolvere in un caso particolare in cui le forze di volume sono
trascurabili rispetto a quelle di superficie (dipende da diversi fattori); generalmente se si riduce 
il volume del fluido considerato che le forze di volume diminuiscono 10  volte più velocemente
delle forze di superficie, Ci poniamo allora nelle condizioni per cui
\begin{gather*}
    \vv{R_{\Sigma}} = 0 \qquad \oint_{\Sigma(V)} p\hat{n} \ d\sigma = 0    
\end{gather*}
La seconda condizione vale se e solo se la pressione sulla superficie del fluido è costante;
dato che la superficie è arbitraria, allora la pressione è costante in tutto il fluido
poiché potrei considerare una superficie che tocca il bordo del fluido e sia parzialmente contenuta 
nel fluido stesso, allora anche questa porzione avrà pressione costante. Questa affermazione è
dimostrabile nel seguente modo:
\begin{gather*}
    \oint_{\Sigma(V)} p\hat{n} \ d\sigma = 0  \ \Longrightarrow \ p = const \ \Longrightarrow \ p\oint \hat{n} \ d\sigma = 0 
\end{gather*}
Il secondo integrale è una proprietà geometrica e dunque è per forza zero: se prendessi infatti il generico vettore
$\hat{n}$ di un certo $\delta r$, allora ottengo il volume del cilindro che avrà altezza il prodotto scalare
dei due vettori normale e $\delta \vv{r}$ e questa traslazione non cambia il volume ma si forma un cilindro che 
indica il volume dovuto alla traslazione di ogni punto del fluido:
\begin{gather*}
    dV = d\sigma \delta \vv{r} \cdot  \hat{n} \ \Longrightarrow \ \Delta V = \int_{\Sigma(V)} \delta \vv{r} \cdot  \hat{n}  \ d\sigma = \delta r_0 \oint_{\Sigma}\hat{n} \ d\sigma = 0 
\end{gather*} 
Se scelgo arbitrariamente il vettore $\vv{r}$ come $\vv{r_0}$ allora si ottiene che quell'integrale deve fare zero in quanto tutti i 
contributi all'area devono annullarsi.  (Il volume del
cilindro rappresenta l'area dello spostamento di un certo punto). Si vede anche
che a pressione costante il momento delle forze agenti sul fluido è uguale a zero:
\begin{gather*}
    \oint _{\Sigma} \vv{r} \times (p\hat{n} ) \ d\sigma = 0  
\end{gather*}  


\begin{wrapfigure}{r}{0.4\textwidth}
    \centering
    \caption{La rotazione del fluido non cambia la risultante delle forze}
    \begin{tikzpicture}
        \draw[->](0, 0) -- (1, -0.7) node[at end, below] {$\hat{x}$ }; 
        \draw[->](0, 0) -- (-1, -0.7) node[at end, below] {$\hat{y}$ };
        \draw[->](0, 0) -- (0, 1) node[at end, left] {$\hat{z}$ };
        \draw[->](0, 0) -- (1.95, 0.7) node[midway, below] {$\vv{r}$ };
        \draw[->](1.95, 0.7) -- (1.95, 1.3)node[at end, right] {$\vv{\delta r}$ };
        \draw[cyan](0, 0) -- (1.95, 1.3);
        \draw[cyan](0.95, 0.35) arc (25:42:0.8) node[at end, above] {$\delta\alpha$};
        \draw(1.7, 1.2) .. controls (2.8, 2.15) and (3.2, 1.9) .. (3.4 , 1.8);
        \draw(3.4, 1.8) .. controls (4, 1.6) and (3.8, 1.1) .. (3.2, 1);
        \draw(3.2, 1) .. controls (2, 0) and (1.4, 1) .. (1.7, 1.2);
        \filldraw(0, 0) circle (1pt) node[anchor = north] {$\Omega$};
    \end{tikzpicture}    
\end{wrapfigure}
La conservazione della quantità di moto rappresenta l'invarianza delle leggi della
fisica nel movimento. La conservazione del momento angolare rappresenta che le leggi della
fisica sono invarianti rispetto alla rotazione. La conservazione dell'energia invece
implica che le leggi della fisica sono invarianti rispetto ai cambiamenti di tempo. 
Scegliendo come polo $\Omega$ l'origine del sistema di riferimento allora posso ruotare il fluido
di un certo angolo $\delta \alpha$ ottenendo allora un certo nuovo vettore
$\vv{r'} = \vv{r} + \delta \vv{r}$, ossia il nuovo vettore è dato dal vettore
\begin{gather*}
    \delta r = \delta\alpha \hat{\Omega} \times \vv{r} 
\end{gather*}   
Con $\hat{\Omega}$ si indica il versore dell'asse di rotazione.
\begin{gather*}
    \delta V = d\sigma \delta \alpha (\hat{\Omega} \times \vv{r}  ) \hat{n} \\
    \Delta V = \int_{\Sigma(V)} d\sigma \delta\alpha(\hat{\Omega} \times \vv{r}  ) \hat{n} \ \Longrightarrow \  \Delta V = \oint_{\Sigma} \delta\alpha\hat{\Omega} \cdot (\vv{r} \times \hat{n}  ) \ d\sigma   
\end{gather*}
Allora si ottiene con le proprietà del prodotto triplo e col fatto che in una rotazione, così come
in una traslazione, il volume non cambia, allora si ottiene la seguente
\begin{gather*}
    \Delta V = \delta a \hat{\Omega} \int(\vv{r} \times \hat{n}  ) \ d\sigma = 0
\end{gather*}
Si è appena dimostrato il \textbf{principio di Pascal}, il cui enunciato dice che: un sistema
isolato si dice all'equilibrio in cui le forze di volume
sono trascurabili e la pressione è costante.

\subsection{Torchio idraulico}
\begin{wrapfigure}{r}{0.3\textwidth}
    \centering
    \caption{}
    \begin{tikzpicture}
        \draw(0, 0) -- (3, 0) -- (3, 1) -- (2, 1) -- (2, 0.5) -- (0.5, 0.5) -- (0.5, 1) -- (0, 1) -- (0, 0);
        \draw[->](-0.2, 1) -- (-0.2, 0.5) node[at end, left] {$\vv{F_1}$};
        \draw[->](2.5,  1) -- (2.5, 2) node[at end, right] {$\vv{F_2}$};
    \end{tikzpicture}    
\end{wrapfigure}
Posso utilizzare i fluidi e la loro pressione per poter applicare più forza:
infatti la spinta che sente la macchina sulla destra sarà data dalla relazione di pressione
costante 
\begin{gather*}
    p = \frac{F_1}{\Sigma_1} = \frac{F_2}{\Sigma_2} \\
    F_2 = \frac{\Sigma_2}{\Sigma_1} F_1
\end{gather*}
Con il principio di Pascal si possono risolvere i problemi a pressione costante e 
nell'ipotesi in cui i volumi di fluidi coinvolti siano sufficientemente piccoli.

\subsection{Trovare l'equilibrio con le equazioni differenziali}
Avrò i contributi delle forze che agiscono sull'asse $z$ 
sono sia quelle di superficie che quelle di volume. 
\begin{align*}
    x &:  -(p(x + \Delta x, y, z)\Delta y \Delta z - p(x, y, z) \Delta y \Delta z) = 0 \\
    y &:  -(p(x , y + \Delta y, z)\Delta y \Delta z - p(x, y, z) \Delta y \Delta z) = 0 \\
    z &: \rho g\Delta x \Delta y \Delta z - (p(x, y z + \Delta z)\Delta x \Delta y - p(x, y, z) \Delta x \Delta y) = 0
\end{align*}
Si ottiene allora le seguenti relazioni:
\begin{align*}
    x &: \frac{-(p(x + \Delta x, y, z) - p(x, y, z))}{\Delta z} = 0 \\
    y &: \frac{-(p(x, y + \Delta y, z) - (p(x, y, z)))}{\Delta y} = 0 \\
    z &: \rho g - \frac{(p(x, y, z + \Delta z)\Delta x \Delta y - p(x, y, z))}{\Delta z} = 0
\end{align*}
Posso allora eseguire la derivata parziale rispetto alle singole coordinate
\begin{gather*}
    \left\{\begin{array}{l}
        -\frac{\partial p}{\partial x} = 0 \\
        -\frac{\partial p}{\partial y} = 0 \\
        \rho z - \frac{\partial p}{\partial z} = 0  
    \end{array}\right.
\end{gather*}
Le altri componenti non hanno le componenti $\rho g$ perché stiamo
considerando solamente l'asse $z$. In assenza di forze di volume, allora
la derivate rispetto sia ad $x$ che $y$ sono uguali a zero e dunque da qualunque parte io
la prenda $p$ non cambia. anche perché se io mi sposto di una certa quantità
infinitesima 
\begin{gather*}
    d\vv{r} = dx\hat{x} + dy\hat{y} + dz\hat{z}    
\end{gather*}
Anche se mi spostassi di un certo angolo qualunque la pressione sarebbe comunque
costante. 

\section{Isolivelli}
\begin{wrapfigure}{r}{0.35\textwidth}
    \centering
    \caption{}
    \begin{tikzpicture}
        \draw[->](-1, 0) -- (3, 0);
        \draw[->](0, -1) -- (0, 3);
        \draw(0.5, 1.5) .. controls (1.5, 2.2) and (2.2, 1.7) .. (2.5, 1);
        \draw(0.5, 1.5) .. controls (0.5, 1)  and  (1.5, 1.5) .. (2, 0.75);
        \draw(2, 0.75) .. controls (2, 0.2) .. (2.5, 1);
        \draw[dashed](1.5, 0) -- (1.5, 1.8) node[at start, below] {$x_0$};
        \draw[dashed](0, 1.8) -- (1.5, 1.8) node[at start, left] {$y_0$};
        \draw[thick, red, ->] (1.5, 1.8) -- (1.5, 2.5) node[at end, right] {$\vv{\nabla}p$ };
    \end{tikzpicture}    
\end{wrapfigure}
Posso vedere che il rotore della pressione è il piccolo vettorino
sulla superficie della curva chiusa per cui posso scrivere
l'equazione per la mia superficie come
\begin{align}
    -\vv{\nabla}p + \rho \vv{g} = 0 
\end{align}
Ossia l'equazione fondamentale dell'idrostatica. Ogni vettorino
$\vv{\nabla}p$ è ortogonale alla superficie del fluido. Si ottengono allora
gli \textbf{isolivelli} per $p$ sulla superficie del fluido: ossia la pressione
del fluido è costante su tutta la sua superficie.  
Dato che la densità non si conosce a priori, devo derivarla dalla pressione:
posso prendere un gas e, comprimendolo, se riduce il suo volume allora deve necessariamente
aumentare la densità.

\section{Legge di Stevino}
\begin{wrapfigure}{r}{0.4\textwidth}
    \centering
    \caption{}
    \begin{tikzpicture}
        \draw[->](0, 0) -- (4, 0) node[at end, below] {$x$};
        \draw[->](0, 1) -- (0, -3) node[at end, left] {$z$};
        \draw[->](-0.2, -0) -- (-0.2, -1) node[at end, left] {$\vv{g}$ };
        \draw(-0.1, -1.75) -- (0.1, -1.75) node[at start, left] {$z$};
        \draw(1, -1) ..  controls (1.5, -1) and (2.2, -1.5) .. (2.5, -2);
        \draw(1, - 1) .. controls (0.5, -1.5) and (1, -2) .. (1.5, -2.5);
        \draw(1.5, -2.5) .. controls (2, -2.7) .. (2.5, -2);
        \filldraw(2, -1.75) circle (1pt);
    \end{tikzpicture}    
\end{wrapfigure}
Il caso più semplice della risoluzione dell'equazione è con $\vv{g}$ costante e nel caso
in cui il fluido sia \textbf{incomprimibile} ($\rho = \rho_0$). Tutte le derivate parziali
della pressione rispetto ai vari assi sono allora zero
tranne che per l'asse $z$ (perché l'unica forza è proprio
lungo la direzione di $z$, ossia la forza gravitazionale), per cui l'equazione fondamentale 
dell'idrostatica ci darà
\begin{gather*}
    \frac{\partial p}{\partial z} = \rho g 
\end{gather*}
A questo punto la funzione varia solo per $z$. La derivata parziale diventa una derivata
totale. 
\begin{gather*}
    \int_{p_0}^{p}\frac{dp}{dz}dz = \int_{z_0}^{z}\rho g \ dz 
\end{gather*}
Si ha quindi la \textbf{legge di Stevino} per cui la differenza di pressione
è legata unicamente alla profondità all'interno del fluido. 
\begin{align}
    p - p_0 = \rho g (z - z_0)
\end{align}
In un fluido incomprimibile la pressione di un fluido dipende interamente
dalla sua altezza. La pressione spinge in tutte le direzioni del contenitore
e più ci si addentra dentro al fluido e più diventa grande la pressione: questo
è dovuto proprio alla legge di Stevino.

\subsection{Stima per un caso specifico}
Quanta pressione esercita un metro d'acqua:
data la densità $ \rho_a \approx 1 \ g /cm^{3}$ e preso $g \approx 10$,
in modo tale che la pressione dell'acqua è $10^{4} N /m^{2}$. In 10
metri si avrebbe invece $10^{5} N / m^{2}$. L'esperimento di
Pascal consiste nel prendere un piccolo tubo di una sezione molto piccola
tale per cui per $3$ metri di tubo si ha un litro d'acqua (per esempio) 
per esercire una grossa pressione nel contenitore sottostante.

\subsection{Contenitore con liquido}
\begin{wrapfigure}{r}{0.4\textwidth}
    \centering
    \caption{}
    \begin{tikzpicture}
        \draw(0, 1) -- (0, 0) -- (3, 0) -- (3, 1);
        \draw(0, 0.8) -- (1.2, 0.8);
        \draw(1.8, 0.8) -- (3, 0.8);
        \draw(1.2, 0.5) -- (1.2, 1.5) -- (1.8, 1.5) -- (1.8, 0.5);
    \end{tikzpicture}    
\end{wrapfigure}
Bernoulli in questo esperimento ha osservato che la pressione che agisce sul 
fluido all'interfaccia con l'aria deve bilanciarsi con la pressione dell'aria.
Per la legge di Stevino posso osservare che
\begin{gather*}
    p = p_{\perp} g h
\end{gather*}
Questa pressione dovrà coincidere con la pressione atmosferica in quanto i due fluidi
devono essere in equilibrio sull'interfaccia. L'interfaccia tra liquido
ed aria non si sposta in quanto le pressioni che intercorrono
tra le due facce sono in equilibrio. Bernoulli con questo esperimento
è riuscito a calcolare la pressione dell'aria di $10^{5}Pa$ al livello 
del mare (e coincide con la sua comprimibilità). 
Posso definire allora 
\begin{gather*}
    10^{5} \ N/ m = 1 \ Bar \approx 1 \ Atm  (1, 03 \cdot  10^{5} \ Pa ) \approx 1 \ kg / cm^{2} 
\end{gather*}
Se si ipotizzasse che dentro al contenitore ci sia acqua, allora
la legge di Stevino diventerebbe
\begin{gather*}
    p - p_0 = \rho_{acqua} g h
\end{gather*}

\begin{wrapfigure}{r}{0.4\textwidth}
    \centering
    \caption{}
    \begin{tikzpicture}
        \draw(0, 0) -- (0, 3) -- (1, 3);
        \draw(1, 3) .. controls (1.2, 1) .. (2, 0);
        \draw(0,0) -- (4, 0);
        \draw(1, 2.5) -- (4, 2.5);
        \draw[->](1.2, 2) -- (0.5, 2) node[midway, above] {$\vv{F}_d$ };
        \draw[->](1.2, 1.8) -- (0.2, 1.8) node[midway, below] {$\vv{F_a}$ };
    \end{tikzpicture}    
\end{wrapfigure}
Quando si è alla profondità dell'ordine di un chilometro
la comprimibilità dell'acqua è da tenere in considerazione
e la densità dell'acqua aumenta a causa della pressione del liquido sovrastante.
Le dighe infatti non sono dei muri dritti
Il pezzo sotto della diga deve resistere sia alla pressione
del cemento sovrastante che alla pressione dell'acqua 
che cresce con la profondità.
La pressione totale a cui deve resistere la diga con una certa quota
di acqua rispetto alla base della diga è
l'integrale della pressione
\begin{gather*}
    R(z) = \int_{0}^{z} \rho g z' \ dz' = \frac{\rho}{2}g z^{2} 
\end{gather*}

\subsection{Fluido immerso in un fluido}
\begin{wrapfigure}{r}{0.4\textwidth}
    \centering
    \caption{}
    \begin{tikzpicture}
        \draw(0, 3) -- (0, 0) -- (3, 0) -- (3, 3);
        \draw[->](2, 1.4) -- (2, 0.5) node[at end, right] {$M_F \vv{g}$ };
        \draw[->](2, 1.6) -- (2, 2.5) node[near end, right] {$-M_F \vv{g}$ };
        \draw(1, 1.5) .. controls (1.5, 2.5) and (1.7, 2) .. (1.75, 1.5);
        \draw(1, 1.5) .. controls (0.8, 1) and (1.25, 0.3) .. (1.75, 1.5);
        \draw(0, 2.5) -- (3, 2.5);
        \filldraw(1, 2) circle (0pt) node[anchor = north] {$F$};
        \filldraw(1.5, 2) circle (0pt) node[anchor = north] {$S$};
    \end{tikzpicture}    
\end{wrapfigure}
Nel caso in cui si immerga un fluido all'interno di un fluido,
se il fluido immerso è lo stesso fluido contenitore $F$, allora la risultante è
diretta lungo la stessa direzione della gravità ma con verso
opposto per avere l'equilibrio
\begin{gather*}
    \rho_F g V = R_{\Sigma}
\end{gather*}
La risultante nel caso in cui il fluido $S$ contenuto nel
fluido $F$ sia diverso
sarà allora data dall'integrale rispetto al volume
\begin{gather*}
    R_{\Sigma} = - \int_{V} \rho \ d V \vv{g} = -M_F \vv{g}  
\end{gather*}

\section{Esempi di esercizi dei fluidi}
In linea generale per la risoluzione dei problemi in
statica dei fluidi è trovare una relazione tra le pressione
e la densità. Quando la pressione è funzione della densità si parla di \textbf{Legge barotropica}:
\begin{align}
     p = p(\rho) 
\end{align}
Questo permette di ottenere una relazione tipo
\begin{gather*}
    \vv{\nabla} p = \rho \vv{g}  
\end{gather*}
Si può anche assumere una certa legge di proporzionalità
sulla potenza della densità:
\begin{align}
    p \propto C \rho^{\gamma} \ \Longrightarrow \  \frac{p}{p_0} = \left(\frac{\rho}{\rho_0}\right)^{\gamma} 
\end{align}
Questa legge prende il nome di \textbf{Legge politropica}. Ci sono dei motivi fisici
che ci portano a determinare questa relazione: in quanto il valore $\gamma$ è dato
dal rapporto dei calori specifici (che si tratterà più avanti). Possiamo considerare
il seguente esempio per determinare perché questa relazione è valida:
\begin{gather*}
    \frac{\partial p}{\partial x} = 0 \qquad \frac{\partial p}{\partial y} = 0 \qquad \frac{\partial p}{\partial z} = -\rho g   
\end{gather*}
Allora posso dire anche in questo caso che la derivata parziale rispetto a $z$ è esattamente
la derivata della pressione sulla costante $z$ poiché rispetto agli altri assi
è nulla la pressione. Allora posso considerare il caso semplice in cui non si abbia la potenza
delle $\rho$ con la legge di Stevino:
\begin{gather*}
    p = p_0 \left(\frac{\rho}{\rho_0}\right) \ \Longrightarrow \ \rho = \rho_0 \frac{p}{p_0}
\end{gather*}
Nell'espressione della derivata:
\begin{gather*}
    \frac{dp}{dz} = -\rho_0 \frac{p}{p_0}g \ \Longrightarrow \ \int_{p_0}^{p}\frac{dp'}{p'} = - \int_{0}^{z} \frac{\rho_0g}{p_0}dz'
\end{gather*}
Gli apici mi servono per differenziare i differenziali rispetto agli estremi di
integrazione, quindi ho separato le variabili. Posso quindi risolvere l'equazione
e ottenere:
\begin{gather*}
    \ln \frac{p}{p_0} = -\frac{\rho_0g}{p_0}z \ \Longrightarrow \ \boxed{p(z) = p_0 \exp\left(-\frac{\rho_0g}{p_0}z\right)}
\end{gather*}
Ho ottenuto una relazione esponenziale per la pressione rispetto alla pressione,
se chiamassi allora
\begin{gather*}
    H = \frac{p_0}{\rho_0g}
\end{gather*}
Allora posso dire che la legge politropica si esprime come
\begin{align}
     p = p_0 \exp\left(-\frac{z}{H}\right)
\end{align}
In questo modo ho trovato una relazione esponenziale della variazione
della pressione a seconda della quota $z$: il fatto che sia una espressione
esponenziale (e non lineare) è coerente a livello fisico poiché se si avesse avuto una espressione
lineare, questa ad un certo punto sarebbe diventata negativa, il che non avrebbe avuto alcun senso.
Per la Terra posso ora stimare il fattore $H$ e dire quando la pressione diminuisce
per un fattore $\frac{1}{e}$, ossia dopo circa $8.5 km$, la pressione si dimezza invece
dopo $\approx 6 \ km$ rispetto al livello del mare (assumendo che $g$ non cambi). 

\subsection{Anticipazione legge dei gas perfetti}
La legge dei Gas perfetti ci dice che
\begin{gather*}
    pV = nRT
\end{gather*}
Posso determinare con il numero di Avogadro il numero di molecole
all'interno del gas:
\begin{gather*}
    pV = \frac{NRT}{N_A} \qquad \text{con} \qquad N = nN_A
\end{gather*}
Posso anche esprimere la relazione tramite la costante di Boltzmann ($k_B = \frac{R}{N_A}$)
\begin{gather*}
    pV = Nk_BT
\end{gather*}
Per un volume infinitesimo $dV$ ci sarà un numero infinitesimo di particelle
contenute all'interno di questo volumetto di gas. Se io lo moltiplicassi
per la massa media delle particelle $\overline{m}$:
\begin{gather*}
    pdV = \overline{m}dN \frac{k_BT}{\overline{m} } 
\end{gather*} 
Si ha allora che il termine $\overline{m}dN$ è esattamente la massa infinitesima,
allora posso riesprimere la relazione dei gas perfetti per un certo volume infinitesimo
di gas come
\begin{gather*}
    p = \frac{dm}{dV} \frac{k_BT}{\overline{m} } \ \Longrightarrow \  p = \frac{k_BT}{\overline{m} } \rho
\end{gather*} 
Allora, assumendo che la temperatura sia la stessa da qui a $8.5km$, allora posso esprimere il rapporto
tra la pressione e la densità come e ottenere allora una approssimazione (non molto corretta) 
per determinare la pressione dell'atmosfera a $8.5km$ secondo questo modello invece che
secondo la relazione (più precisa) che si era trovata prima.
\begin{gather*}
    \frac{p}{\rho} = \frac{k_BT}{\overline{m} }
\end{gather*}
Essendo questa una grossa approssimazione, in futuro sarà rivista e trattata in modo
più completo nella parte di termodinamica e nella parte di fisica statistica. 

\chapter{Le forze apparenti nei fluidi}
\section{Fluido in rotazione e isolivelli della pressioneS}
\begin{wrapfigure}{r}{0.5\textwidth}
    \centering
    \caption{Forze apparenti nei fluidi}
    \begin{tikzpicture}[domain=0:2]
        \draw[->](-1, 0) -- (3, 0) node[at end, below] {$x$};
        \draw[->](0, -1) -- (0, 3) node[at end, left] {$z$};
        \draw[color=cyan]  plot (\x, {(\x)^2 / 2 });
        \draw(1.3, 0.9) rectangle (1.5, 1.1);
        \draw[->](1.4, 1) -- (2.2, 1) node[at end, right] {$\vv{F_{cf}}$ };
        \draw[->](1.4, 1) -- (0.8, 1.6) node[at end, above] {$\vv{\nabla}p dV$ };
        \draw[->](1.4, 1) -- (1.4, 0.25) node[at end, right] {$\vv{g} \rho dV$};
    \end{tikzpicture}    
\end{wrapfigure}
Abbandonando le ipotesi di barotropica e politropica, supponiamo di avere un fluido
incomprimibile e che sia in rotazione rispetto all'asse $z$, posso esprimere il
vettore velocità angolare e l'accelerazione di gravità come:
\begin{align*}
    \vv{\omega} &= \omega \hat{z} \\
    \vv{g} &= -g \hat{z}    
\end{align*}
La quota $z_0$ è l'altezza del liquido nell'asse $z_0$. Preso un piccolo elemento
di fluido piccolo a piacere, questo sentirà l'effetto di due forze: un effetto da
parte della forza di gravità e anche la forza centrifuga (se osservato in un SdR inerziale).
Ponendosi in coordinate cilindriche, posso definire $\vv{r}$ la distanza dall'asse di rotazione
e quindi esprimere la forza centrifuga come:
\begin{gather*}
    \vv{F}_{cf} = \rho_0 dV \omega^{2}r \hat{r}   
\end{gather*} 
Ossia diretta lungo la direzione radiale. La somma delle due forze può essere
bilanciata solamente da una forza $\vv{\nabla} pdV$. Non ci resta
altro che scrivere le relazioni:
\begin{gather*}
    \vv{\nabla}p  =\rho \vv{g}  
\end{gather*}
Posso esprimere il gradiente della forza da coordinate cartesiane
a coordinate cilindriche come
\begin{gather*}
    \vv{\nabla}F = \frac{\partial F}{\partial x}\hat{x} + \frac{\partial F}{\partial y}\hat{y} + \frac{\partial F}{\partial z} \hat{z}     \\
    \ \Longrightarrow \ \vv{\nabla}F = \frac{\partial F}{\partial r}\hat{r} + \frac{1}{r}\frac{\partial F}{\partial \theta}\hat{\theta} + \frac{\partial f}{\partial z}\hat{z}       
\end{gather*}
Posso allora esprimere la derivata parziale rispetto alla pressione come
\begin{gather*}
    \left\{\begin{array}{l}
        \frac{\partial p}{\partial r} = \rho_0 \omega^{2}r \\
        \frac{\partial p}{\partial \theta} = 0 \\
        \frac{\partial p}{\partial z} = -\rho_0 g    
    \end{array}\right.
\end{gather*}
La pressione allora dipenderà solamente dalla distanza rispetto all'asse di rotazione
e dalla quota $z$ mentre non dipende dall'angolo di rotazione. Posso assumere
allora che la pressione possa essere divisa in una funzione solo di $r$ e una solo di $z$:
\begin{gather*}
    p(r, z)  - p_0 = f_1(r) - f_2(z)  = \rho_0 \frac{\omega^{2} }{p}r^{2} - \rho_0gz + c_1 + c_2 
\end{gather*}
Posso allora esprimere le derivate parziali rispetto alle due funzioni, che derivano proprio da
dalla derivata parziale della pressione rispetto a $r$ e $z$.
\begin{gather*}
    \left\{\begin{array}{l}
        \frac{\partial f_1}{\partial r} = \rho_0 \omega^{2} r  \ \Longrightarrow \ f_1 = \rho_0 \frac{\omega^{2} }{2}r^{2} + c_1 \\
        \frac{\partial f_2}{\partial z} = -\rho_0 g \ \ \Longrightarrow \ f_2 = -\rho_0 gz + c_2   
    \end{array}\right.
\end{gather*}
Per comodità posso escludere le costanti (dipende dal fatto che viene considerata la pressione $p_0$
nell'espressione della pressione in funzione del raggio e della quota): infatti quando $r = 0$
posso considerare le costanti come zero. infatti potrei esprimere la funzione
della pressione come
\begin{align}
    \boxed{p(r, z) = \rho_0 \frac{\omega^{2} }{2}r^{2} - \rho_0 gz + p_0 }
\end{align}
Se volessi trovare il valore di $p_0$, allora troverei la condizione
\begin{gather*}
    \rho_0 \frac{\omega^{2} }{2}r^{2} = \rho_0 gz 
\end{gather*}
E dunque il valore di $z$ per cui si ha $p_0$ sono tutti i punti con
\begin{gather*}
    z = \frac{\omega^{2} }{2}\frac{r^{2} }{g} 
\end{gather*}
Tutte le isosuperfici sono date da una parabola nel grafico (ossia tutte le curve
lungo le quali la pressione è la stessa). Quando un liquido ruota la superficie di un 
liquido tende a formare un paraboloide proprio per questo motivo analitico. Il pelo del liquido
allora (ossia la superficie di contatto tra il fluido ed un altro fluido) tende a formare
un vortice quando ruota. 

\section{Autogravitazione dei fluidi nel caso di una sfera di fluido}
\begin{wrapfigure}{r}{0.4\textwidth}
    \centering
    \caption{La sfera di fluido}
    \begin{tikzpicture}
        \draw(0, 0) circle (2);
        \draw(0, 0) -- (0, 2) node[midway, left] {$r$};
        \filldraw[red](-0.1, 2) rectangle (0.1, 2.2) node[anchor = west] {$dV$};
        \draw[->, red, thick](0, 2) -- (0, 1) node[near end, right] {$\vv{F_g}$};
        \draw(0, 0) -- (-2, 2) node[midway, below] {$R$};
        \draw(0, 0) circle (2.85);
        \draw[->, red, thick](0, 2.2) -- (0, 3.2) node[at end, right] {$\vv{\nabla}p$ };
    \end{tikzpicture}    
\end{wrapfigure}
Fino ad ora si era considerato equilibri tra forza di pressione e forza esterna (come la gravità)
generate da corpi esterni. E' l'equilibrio tra le pressioni e la forza di gravità che
mantiene intatti i corpi celesti. La forza esercitata su unità di massa è data da:
\begin{gather*}
    \frac{\vv{F} }{m} = - \frac{GM(r)}{r^{2} } \rho dV \hat{r}  
\end{gather*}
Sto cercando di determinare la forza che agisce su di una parte del fluido $dV$ da parte della
gravità del fluido interno ad una distanza $r$ dal centro. La parte di fluido di raggio $r$ agisce su di una certa porzione
del fluido $dV$ ad una certa distanza $r$ in \textbf{autogravità}, posso esprimere la massa 
a distanza $r$ dall'origine della sfera
\begin{gather*}
    M(r) = \int_{0}^{r} 4\pi r^{2}\rho (r) \ dr
\end{gather*}
Posso cambiare sistema di riferimento ed utilizzare le coordinate sferiche ed esprimere il gradiente
della pressione in funzione del versore $\hat{r}$ come
\begin{gather*}
    \vv{\nabla}p = \rho \vv{g}(r) = -\frac{GM(r)}{r^{2} }\hat{r}   
\end{gather*} 
Il gradiente di una forza in questo caso sarà dato dalla seguente relazione 
(devo trasformarlo dal sistema di riferimento cartesiano a quello sferico) utilizzando
l'angolo $\theta$ rispetto alla verticale e l'angolo $\phi$ rispetto ad uno dei due assi perpendicolari
a quello verticale ed il raggio $r$ come la distanza dal centro della sfera si ha:
\begin{gather*}
    \vv{\nabla}F = \frac{\partial F}{\partial r}\hat{r} + \frac{1}{r}\frac{\partial F}{\partial \theta} \hat{\theta} + \frac{1}{r\sin\theta}\frac{\partial F}{\partial \phi} \hat{\phi}       
\end{gather*}
Posso allora derivare parzialmente la pressione (tuttavia considero solamente rispetto al raggio
in quanto, essendo a simmetria sferica, le derivate rispetto a $\phi$ e $\theta$ sono zero):
\begin{gather*}
    \frac{\partial p}{\partial r} = -G \rho \frac{M(r)}{r^{2} } = -\frac{4}{3}\pi G\rho_0^{2}r 
\end{gather*}
Posso integrare da entrambe le parti con separazione di variabili e dunque  ottenere l'integrale:
\begin{gather*}
    \int_{p_0}^{p} dp = -\int_{R_{\oplus}}^{r} \frac{4\pi}{3}G\rho_0^{2}r' \ dr \ \Longrightarrow \ -\frac{4\pi}{3}\rho_0^{2}\frac{G}{2} (r^{2} - R^{2}_{\oplus}  ) 
\end{gather*}
Dove $R_{\oplus}$ è il raggio della sfera. Posso allora esprimere la pressione
con la seguente funzione:
\begin{gather*}
    p = p_0 + \frac{4}{3}\pi\rho_0^{2}\frac{G}{2}(R_{\oplus}^{2} -r^{2}  ) = p_0 + \frac{4}{3}\pi\frac{G}{2}R_{\oplus}^{2}\rho_0^{2} \left(1 - \frac{r^{2} }{R_{\oplus}}\right) 
\end{gather*}
La massa del pianeta è allora esprimibile come $M_{\oplus} = \frac{4}{3}\pi\rho_0R_{\oplus}^{3}$.  A questo punto posso esprimere la pressione
come
\begin{gather*}
    p = p_0 + \frac{GM_{\oplus}}{R_{\oplus}}\frac{\rho_0}{2}R_{\oplus}\left(1 - \frac{r^{2} }{R_{\oplus}}\right)
\end{gather*}
Dove $\frac{GM_{\oplus}}{R_{\oplus}} = g_{\oplus}$. Per la Terra ottengo la pressione al centro di
$\approx 10^{11} \ Pa$, ossia la pressione stimata della Terra con la sua massa, 
densità e raggio come se fosse un fluido di densità costante in tutti i punti.  

\subsection{Il pianeta fatto di fluido in rotazione}
\begin{wrapfigure}{r}{0.4\textwidth}
    \centering
    \caption{Lo schema delle forze nel fluido in rotazione autogravitante}
    \begin{tikzpicture}
        \draw[->](0, 0) -- (0, 4) node[at end, left] {$z$};
        \draw(0, 0) -- (2, 3) node[midway, left] {$r$};
        \draw(0, 1) arc (90:55:1)node[midway, above] {$\theta$};
        \draw[dashed](0, 3) -- (2, 3) node [midway, above] {$h = r\sin\theta$};
        \draw[->, very thick](2, 3) -- (1, 1.5) node[at end, right] {$\vv{F_{\theta}}$ };
        \draw[->, very thick](2, 3) -- (3, 3) node[at end, right] {$\vv{F_c}$};
        \draw[->, very thick](2, 3) -- (2.5, 3.75) node[at end, right] {$\hat{r}$ };
        \draw[->, very thick](2, 3) -- (2.5, 2.75) node[at end, right] {$\hat{\theta}$ };
    \end{tikzpicture}    
\end{wrapfigure}
La distanza dall'asse di rotazione prende il nome di $h$ mentre $r$ è la distanza
dal centro della massa di fluido in rotazione. Posso esprimere 
l'angolo rispetto alla verticale del vettore posizione con
l'angolo $\theta$. E quindi posso esprimere i vettori come
\begin{gather*}
    \vv{F_g} = -\frac{\rho_0 G M(r)}{r^{2} }\hat{r} = \rho_0^{2} G\frac{4\pi}{3}r  \\
    \vv{F_c} = \rho_0 \omega^{2}h(\sin\theta \hat{r} + \cos\theta \hat{\theta}  )  = \rho_0 \omega r \sin\theta (\sin\theta \hat{r} + \cos \theta\hat{\theta}  )
\end{gather*}
Sapendo che
\begin{gather*}
    M(r) = \frac{4\pi}{3}r^{3} \rho_0 
\end{gather*}
La pressione ora deve bilanciare il contributo sia della forza di gravità che della forza centrifuga,
allora posso esprimere la derivata della pressione sia su $\theta$ che su $r$:
\begin{gather*}
    \frac{\partial p}{\partial r} = -\rho_0^{2}\frac{4\pi}{3}rG+ \rho_0 \omega^{2}r \sin^{2}\theta \\
    \frac{1}{r}\frac{\partial p}{\partial \theta} = -\rho_0^{2} \omega^{2}r\sin\theta\cos\theta       
\end{gather*}
Posso allora integrare come se $\theta$ fosse un parametro nella prima
e quindi ottengo l'espressione della pressione in funzione del raggio
e dell'angolo $\theta$ (dell'angolo $\phi$ non mi interessa in quanto sto
considerando una sfera). Posso esprimere $f(\theta)$ come l'integrale della derivata
parziale rispetto a $\theta$:
\begin{gather*}
    p(r, \theta) - p_0 = -\rho_0^{2}\frac{4\pi}{3}\frac{r^{2} }{2}G+ \rho_0 \omega^{2}\frac{r^{2} }{2} \sin^{2}\theta + f(\theta)
\end{gather*}
Allora posso derivare rispetto a $\theta$ questa espressione
\begin{gather*}
    \frac{\partial p}{\partial \theta} = \rho \omega^{2}r^{2} \sin\theta\cos\theta + f'(\theta)  
\end{gather*}
Adesso se ponessimo il termine $\frac{1}{r}$ davanti all'espressione, dovrebbe essere uguale con
l'espressione della derivata parziale trovata prima, uguagliandole:
\begin{gather*}
    f'(\theta) = 0 \ \Longrightarrow \ f(\theta) = const
\end{gather*}
La soluzione è allora considerare una generica funzione $f(\theta)$ come costante
ed esprimerla come $C$. Dato che ho scelto $p_0$ come la pressione
al centro,allora la costante sarà tolta e otterrò 
\begin{align}
    p(r, \theta) - p_0 = -\rho_0 \frac{r^2 }{2}\left(G\rho_0\frac{4\pi}{3} - \omega^{2}\sin^{2}\theta  \right)
\end{align}
Posso esprimere come si è fatto con il raggio della sfera e la massa della sfera in modo da ottenere $g_{\oplus}$,
ossia l'accelerazione di gravità e quindi trovare l'espressione della pressione
finale come
\begin{gather*}
    p - p_0 = \frac{r^{2}\rho_0 g_{\oplus} }{2R_{\oplus}} \left(1 - \frac{\omega^{2}R_{\oplus} }{g_{\oplus}}\sin^{2}\theta \right)
\end{gather*}

\part{Termodinamica}
\chapter{La termodinamica}
\section{Introduzione alla termodinamica}
\subsection{Storia della termodinamica}
La termodinamica è esplosa a metà dell'ottocento con le
prime macchine termiche, ossia delle macchine che sfruttavano energia
termica per poter compiere lavoro. Lo sviluppo storico
della termodinamica parte con le macchine termiche e si sviluppa
in modo non propriamente logico ma anzi è stato frutto di un processo
molto tortuoso che ha eventualmente portato alla nascita della termodinamica così come la 
studiamo oggi. 

\subsection{Lo studio della termodinamica}
Nella parte di termodinamica ci si occupa dello studio
dei sistemi a scala macroscopica e non a scala microscopica. Le coordinate
dello spazio che si definisce nella termodinamica prendono il nome
di \textbf{coordinate termodinamiche}:  grandezze come la pressione, densità di massa sono
ottime coordinate termodinamiche in quanto riescono a descrivere le caratteristiche
del nostro sistema termodinamico. Nel caso di sistemi omogenei la quantità di massa non
varia mai e quindi utilizzare il volume o la densità è semplicemente una preferenza anche
se in generale in termodinamica si parla più in termini di volume che
in termini di densità (come invece si fa nei fluidi). Quando si considera un
sistema termodinamico $S$ possiamo fare delle considerazioni:
\begin{itemize}
    \item Si è sempre in grado di stabilire lo stato del sistema attraverso
    le coordinate termodinamiche;
    \item Si determinano i confini del sistema rispetto all'ambiente esterno con il quale il sistema interagisce (anche chiamato \textbf{ambiente});
    \item Il sistema più l'ambiente è chiamato \textbf{Universo}; non l'universo
    della cosmologia ma semplicemente un insieme: $S + A = U$;
\end{itemize}
La densità e la pressione non sono in grado di descrivere accuratamente tutti i sistemi
termodinamici ma è necessario introdurre un'altra coordinata termodinamica, ossia
la \textbf{Temperatura} $T$. La definizione operativa dell temperatura è molto complessa
ma ci permette di dire che cosa vuol dire che due sistemi hanno la stessa temperatura. 


\section{L'equilibrio termodinamico}
A differenza dell'equilibrio meccanico, l'equilibrio termodinamico
consiste nel definire quando un sistema è in equilibrio con un altro
sistema: per farlo bisogna determinare se le loro coordinate termodinamiche sono costanti
nel tempo. Quando in un sistema macroscopico si hanno delle coordinate termodinamiche
che dipendono dallo spazio, ma sono costanti nel tempo, allora siamo in presenza dell'effetto di
un campo di forze esterne che agisce sul sistema. Quando accade questo siamo in presenza di un
\textbf{equilibrio termodinamico locale}: l'equilibrio termodinamico è 
valido solamente per un dato punto e non per tutti i punti del sistema
nel corso del tempo proprio perché le coordinate termodinamiche variano spazialmente. 
Questo equilibrio non è considerato in questo corso se non per pochi casi. \\
La cosa non ovvia è che quando il sistema è all'equilibrio termodinamico le
variabili termodinamiche sono dipendenti l'una dall'altra: esiste allora una relazione
funzionale tale per cui una è funzione delle altre due. Questo risultato
prende il nome di \textbf{equazione di stato}
\begin{gather*}
    f(p, V, T) = 0
\end{gather*} 
Ovviamente l'espressione della funzione $f$ non si conosce quasi mai anche se si
è certi che esista una espressione esplicita per l'equazione di stato. Dato un sistema, se  questo è
lasciato senza interagire con il sistema ambiente allora non cambierà il suo equilibrio;
se interagisse invece con il sistema ambiente allora il suo equilibrio varierà.

\section{Estensività}
Se si considerasse un sistema $S$ e lo si suddividesse in due sistemi $S_1$ e $S_2$ tale
per cui si suppone che la quantità di sostanza  di $S_1$ sia esattamente uguale alla 
sostanza di $S_2$ e che la somma 
delle loro sostanze dia la sostanza totale di $S$; dato che all'inizio
il sistema $S$ era all'equilibrio, allora facciamo in modo che anche i due
sottosistemi siano all'equilibrio: la grandezza $X$ quantità di sostanza è estensiva
quando $X_1 = X_2 = \frac{X}{2}$: è anche la definizione di additività e di
\textbf{estensività} poiché in termodinamica si dà per scontato che questi due
termini descrivano la medesima cosa. La massa, il volume sono grandezze estensive
poiché dipendono dalla quantità di sostanza; grandezze come la temperatura sono invece estensive.
Le grandezze che non sono né estensive né intensive non sono valide coordinate termodinamiche anche se 
devo possedere almeno una grandezza estensiva per poter descrivere in maniera
completa un sistema termodinamico; se così non fosse allora non
sarei in grado di determinare se il sistema considerato sia molto grande oppure
molto piccolo.


\section{Pareti ed equilibrio termico}
Una \textbf{parete} è un oggetto che sia capace di separare tra loro
due sistemi termodinamici. Una parete è \textbf{adiabatica} se ai due lati della
parete possono coesistere due sistemi termodinamici all'equilibrio
qualunque essi siano: questo vuol dire che la parete è \textbf{isolante}
e non permette quindi l'interazione tra questi due sistemi. Una parete
che non è adiabatica è definita come parete \textbf{diatermica}: i due sistemi che 
si trovano separati da questa parete si accorgono l'uno della presenza dell'altro
e iniziano ad interagire: se si aspettasse molto tempo varierebbero il loro stato
di equilibrio raggiungendone uno nuovo.  Due sistemi possono non essere
all'equilibrio termico fra loro ma all'equilibrio termodinamico se sono separati da una parete
adiabatica; tuttavia se sono separati da una parete diatermica i due sistemi
saranno sia in equilibrio termico fra loro che in equilibrio termodinamico.
Dalla fisica sperimentale si ha che se un sistema $A$ è all'equilibrio
termico con $C$ e anche il sistema $B$ è anch'esso in equilibrio
termico con $C$ allora anche $A$ e $B$ sono in equilibrio termico. Potremmo
riassumere queste considerazioni mediante la seguente:
\begin{align}
    (A \sim C) \wedge (B \sim C) \ \Longrightarrow \ (A \sim B)
\end{align}
Che prende il nome di \textbf{Principio zero della termodinamica}. La relazione
di equilibrio termico è una relazione transitiva che si può dimostrare ponendo una parete adiabatica
tra $A$ e $B$ e ponendo un sistema "a ponte" tra i due sistemi. Se poi si  sostituisse 
la parete adiabatica con una parete diatermica si osserverebbe che i due sistemi $A$ e $B$
sono già all'equilibrio e quindi la relazione scritta in precedenza è valida.  

\section{La temperatura}
La temperatura è la grandezza che mi consente di dire se
due sistemi sono in equilibrio termico: la temperatura per due sistemi
all'equilibrio termico è infatti la stessa. Anche senza definirla so che è una grandezza intensiva in quanto abbiamo visto che non 
dipende dalla quantità di sostanza per i sistemi considerati nel 
principio zero. Perché  si utilizza la temperatura per dire se un oggetto è 
più caldo di un altro? L'esperienza sensoriale ci dice che se si pone
due oggetti uno più caldo dell'altro vicini non saremmo più in grado di determinare se uno è più 
caldo dell'altro. La sensazione dunque è una cosa che per definire il concetto di oggetto
più caldo o freddo non è sufficiente per determinare le relazioni tra due corpi; 
devo quindi utilizzare una quantità estensiva per determinare se un oggetto è più caldo 
di un altro. \\
Per definire la temperatura posso considerare un sistema $S$ in modo tale
da essere nella situazione come quella illustrata nel principio zero
della termodinamica e, considerate due variabili $x, y$ che non siano la temperatura, il fatto di essere all'equilibrio termico non è scontato poiché quando
due sistemi sono all'equilibrio termico tra di loro allora ci dovrebbe essere una relazione
tra le variabili dei vari sistemi. Posso allora esprimere la l'equazione di stato:
\begin{gather*}
    f_{A, C}(x_A, y_A, x_C, y_C) = 0 \\
    f_{B, C}(x_B, y_B, x_C, y_C) = 0
\end{gather*} 
Possiamo allora fare una assunzione sensata per cui
\begin{gather*}
    y_C = g_{A, C}(x_A, y_A, x_C) \\
    y_C = g_{B, C}(x_B, y_B, x_C)
\end{gather*}
Queste due sono uguali tra di loro: applicando allora il principio zero
si ha
\begin{gather*}
    f_{A, B}(x_A, y_A, x_B, y_B) = 0
\end{gather*}
Allora lo stato di equilibrio dei vari sistemi è caratterizzato
dall'aver assegnato le variabili a quel sistema: i valori per cui quest'ultima
è soddisfatta devono essere gli stessi che per le altre. Dato che
devono descrivere lo stesso stato, la variabile $x_C$ può essere eliminata:
la dipendenza funzionale delle funzioni $g$ non dipendono da $x_C$. 
Allora posso dire che esistono due funzioni:
\begin{gather*}
    h_A(x_A, y_A) = h_B(x_B, y_B)
\end{gather*} 
Posso allora permutare i sistemi $A, B, C$ come voglio: posso ottenere
dunque un'altra
relazione con $A$ e $C$:
\begin{gather*}
    h_A(x_A, y_A) = h_C(x_C, y_C) \ \Longrightarrow \ h_A(x_A, y_A) = h_B(x_B, y_B)= h_C(x_C, y_C)
\end{gather*}
Se due sistemi sono all'equilibrio termico allora la funzione delle variabili 
$x$ e $y$ per ciascun sistema sono in relazione con gli altri sistemi
e dunque questa funzione di variabili è esattamente la \textbf{temperatura empirica}.
Se noi fossimo in grado di determinare queste funzioni allora potremmo definire la temperatura
come funzione delle altre variabili termodinamiche (il che non è possibile anche se si sa che esiste).
La grandezza temperatura è dunque quella grandezza funzione della pressione e
del volume  che mi permette di determinare se due sistemi sono in equilibrio
termico oppure no. Se la temperatura è fissata posso immaginare di prendere, dei vari stati,
solo quegli $x, y$ che mi portino (secondo qualche relazione funzionale) ad avere quella determinata temperatura. Questi punti
appartengono ad una curva che prende il nome di \textbf{curva isotermica} e rappresentano solo una
parte ristretta degli stati totali che assumerebbe il mio sistema. 


\subsection{Misurare la temperatura e sua definizione operativa}
\begin{wrapfigure}{r}{0.4\textwidth}
    \centering
    \caption{Il termometro con il sistema $S$}
    \begin{tikzpicture}
        \draw(0, 0) rectangle (2, 2) node[midway] {$S, T_S$};
        \draw(-1, 0.5) rectangle (0, 1.5) node[midway] {$A, T_A$};
    \end{tikzpicture}    
\end{wrapfigure}Possiamo definire uno strumento (dal principio zero) che ci permetta di misurare
la temperatura: infatti se riesco a definire la temperatura di un sistema io posso 
a quel punto metterlo a contatto termico con un altro e dunque
ottengo la misura della temperatura. Posso quindi utilizzare un \textbf{termometro}
per poter definire operativamente la temperatura così come ho fatto per la definizione
operativa di forza mediante una grandezza minore. Posso valutare allora una proprietà
che dipende dalla temperatura, ossia una \textbf{proprietà termometrica} come
il volume. Mettendo ora a contatto il termometro $A$ con il sistema $S$ che vogliamo misurare
si può misurare la temperatura di $T_A$ che ora è la stessa di $T_S$; tuttavia la
temperatura iniziale del sistema $T_S$ non sono in grado di determinarla.
Per ovviare al problema posso costruire un sistema di termometri con masse sempre più piccole
in modo tale che possa misurare la temperatura con ogni termometro e poi riportare
il sistema alle condizioni iniziali. Posso trovare allora la temperatura di ogni termometro ed ottenere che
questi valori di temperatura sperimentalmente sono sempre
tra valori assegnati e dunque la successione è monotona e limitata e dunque
è convergente: la temperatura è data dal limite
\begin{align}
    T = \lim_{M_A \to 0}T(M_A) 
\end{align}
Ad un certo punto il mio strumento non ha più risoluzione per poter determinare la temperatura.
Qualsiasi sistema $A$ può essere utilizzato per determinare la temperatura
di un certo sistema $S$ purché la massa $m_A << m_S$ per evitare di modificare
in modo sostanziale la temperatura del sistema $S$.
\section{Come è fatto un termometro}       
Nei primi termometri si utilizzava il  mercurio in quanto
a temperatura ambiente è liquido ed ha un coefficiente di dilatazione
molto grande e soprattutto molto più lineare rispetto ad altri fluidi
come l'acqua. L'idea di un termometro è in generale quella di trovare la temperatura
di un dato sistema date come variabili $x, y$, che corrispondono rispettivamente
al volume e alla pressione. Posso allora decidere (dato che è comodo) di trovare una funzione
lineare prototipo del tipo:
\begin{gather*}
    T(x) = ax   + b
\end{gather*}
\begin{wrapfigure}{r}{0.4\textwidth}
    \centering
    \caption{}
    \begin{tikzpicture}
        \draw[->](0, 0) -- (4, 0) node[at end, below] {$x$};
        \draw[->](0, 0) -- (0, 4) node[at end, left] {$y$};
        \draw(1, 1) .. controls (1.4, 0.9) and (1.9, 2.2) .. (2, 3) node[at end, above] {$h(x, y) = T_1$};
        \draw(0, 2) -- (4, 2) node[at end, right] {$Y = Y_0$};
        \draw[dashed](1.78, 0) -- (1.78, 2) node[at start, below] {$x_1$};
        \draw(2, 1) .. controls (2.4, 0.9) and (2.9, 2.2) .. (3, 3) node[at end, right] {$h(x, y) = T_2$};
        \draw[dashed](2.78, 0) -- (2.78, 2) node[at start, below] {$x_2$};
    \end{tikzpicture}    
\end{wrapfigure}
Adesso devo trovare un modo per poter fissare queste costanti: se io so che
ci sono delle situazioni che si verificano sempre alla stessa temperatura, allora
sono in grado di determinarle. Esistono quindi dei punti chiamati \textbf{punti fissi
termodinamici} : questi punti esistono e corrispondono proprio ai cambi di stato. Quando un materiale
passa da uno stato all'altro si verificano sempre alla stessa temperatura purché sia fissata almeno una
delle due variabili da cui ricavo la temperatura. Storicamente per determinare le scale dei
termometri si utilizzano i punti termodinamici fissi dell'acqua come pressione fissata a $p = 1 \ atm$.
\begin{gather*}
    T_1 = ax_1 + b \\
    T_2 = ax_2 + b
\end{gather*}
Posso allora ricavare le costanti come
\begin{gather*}
    a = \frac{T_2 - T_1}{x_2 - x_1} \\
    b = T_1 - x_1\frac{T_2 - T_1}{x_2 - x_1}
\end{gather*}
Tutti questi procedimenti sono validi se e solo se si rimane
ad un certo $Y$ assegnato. Questa è solo una definizione per una
curva che modellizzi la curva isoterma (è solamente una approssimazione
e non una legge). Utilizzando ora i punti fissi dell'acqua si ha
\begin{gather*}
    t(x) = 100\text{°}\frac{x - x_1}{x_2 - x_1}
\end{gather*}
La variabile $x$ è la temperatura dell'oggetto che si vuole
misurare mentre $x_1$ e $x_2$ sono la temperatura della scala iniziale e finale.

\subsection{Le varie scale}
La scala Fahrenheit è una scala che utilizza altri punti fissi rispetto all'acqua
utilizzando una miscela di acqua ghiaccio e ammoniaca. Si può ricavare una relazione tra
la temperatura Fahrenheit e la temperatura Celsius:
\begin{align}
    t_F[^{\circ}F ] = \frac{9}{5}t[^{\circ}C ] + 32
\end{align}
Quando misuro una temperatura con due termometri che utilizzano fluidi diversi
ma la stessa scala termometrica questi leggeranno sempre misure diverse
per tutte le temperature intermedie tra i due punti fissi in quanto la funzione
che ho utilizzato per approssimare la temperatura non è congruente con la curva di
temperatura effettiva. Un'altro problema per i termometri è la lenta evaporazione del 
fluido utilizzato che rende piano piano il termometro sempre meno preciso.


\chapter{I Gas}
\section{La storia della teoria dei gas e la definizione di gas perfetto}
Quando si fornisce tanta energia ad un sistema questo diventa un gas: si
è scoperto che i gas obbediscono a delle leggi molto semplici. Robert Boyle
ha scoperto, misurando pressioni e volume, che variando quelle grandezze il loro prodotto risultava
essere costante 
\begin{gather*}
    pV =  \text{const}
\end{gather*}
Successivamente il fisico Mariotte ha scoperto che tale costante è 
la temperatura. Molti anni dopo il fisico Gay-Lussac si è domandato come
possa dipendere una variabile dall'altra e ha trovato 
una dipendenza della pressione dalla temperatura, e lo stesso per il volume,  secondo le seguenti relazioni:
\begin{gather*}
    p = p_0 (1 + \beta T) \ (V\text{ const}) \\
    V = V_0 (1 + \gamma T) \ (p \text{ const})
\end{gather*}
Allora si vede che entro gli errori queste due costanti sono molto vicine tra
di loro e non dipendono dalla sostanza dalla quale si è fatto le misurazioni. 
Queste leggi valgono se e solo se si hanno piccole variazioni di temperatura 
e dunque si ottiene che
\begin{gather*}
    \beta \approx \gamma = \frac{1}{273}\ ^{\circ}C^{-1}  
\end{gather*}
Poco dopo il chimico Avogadro scopre che 
\begin{gather*}
    p = \text{const} \quad T = \text{const} \ \Longrightarrow \  V \propto n
\end{gather*}
Ossia il volume è direttamente proporzionale al numero di moli di un gas se e solo se il gas
è mantenuto a pressione e temperatura costanti.
Nonostante queste leggi siano storiche, queste sono tanto più valide tanto più un gas è
rarefatto: posso allora definire un limite secondo il quale
\begin{gather*}
    \lim_{p \to 0} \ \Longrightarrow \  \text{gas perfetto} 
\end{gather*}
Il gas si dice \textbf{perfetto} quando la pressione tende a zero e dunque le leggi scritte
per la pressione ed il volume valgono ed il valore dei coefficienti è esattamente:
\begin{gather*}
    \beta = \gamma = \frac{1}{273.15 \ ^\circ C }
\end{gather*}
Posso allora determinare la temperatura in funzione della pressione
quando il volume è costante:
\begin{gather*}
    T = \frac{273.15}{p_0}p - 273.15 \ [^{\circ}C ]
\end{gather*}
In questo modo posso rendere un gas sempre più rarefatto e misurarne la temperatura:
così posso estrapolare la curva di temperatura con volume costante 
in funzione della variazione della pressione. Se volessimo
una scala più comoda potremmo definire una nuova scala di temperatura
chiamata \textbf{scala Kelvin} (grado Kelvin non è giusto!) :
\begin{align}
    T = 273.15 + T(^{\circ}C ) \ [K]
\end{align}
\begin{wrapfigure}{r}{0.4\textwidth}
    \centering
    \caption{Il punto triplo}
    \begin{tikzpicture}
        \draw[->](0, 0) -- (4, 0) node[at end, below] {$T$};
        \draw[->](0, 0) -- (0, 4) node[at end, left] {$p$};
        \draw(0, 0) .. controls (0.5, 0.2)  and (0.8, 0.6).. (1, 1);
        \draw(1, 1) .. controls (2, 1.2) and (3, 1.7) .. (4, 3);
        \draw(1, 1) -- (0.5, 3.5);
        \filldraw[red](1, 1) circle (1pt);
    \end{tikzpicture}    
\end{wrapfigure}
Allora la temperatura assoluta definita a volume costante
utilizzando la scala Kelvin sarà:
\begin{gather*}
    T = \frac{p}{p_0}T_0
\end{gather*}
Esistono, oltre ai due punti fissi di fusione e ebollizione, esiste anche un unico
valore in certe condizioni di pressione, volume e temperatura nel 
quale le tre fasi coesistono simultaneamente che prende il nome di \textbf{punto triplo}.
Le condizioni del punto triplo dell'acqua sono:
\begin{gather*}
    T_{\text{triplo}} = 0.01\  ^{\circ}C = 273.16 \ K \\
    p_{\text{triplo}} = 611 \ Pa \approx 10^{-3} \ atm \\
\end{gather*}
Dato che il gas prefetto non esiste, io posso determinare la temperatura di un oggetto
con un termometro che contiene un gas perfetto è data da:
\begin{gather*}
    T = \frac{p}{p_3}T_3
\end{gather*}
Dato che il gas perfetto non esiste, io prendo una successione di pressioni
che convergono verso zero, di conseguenza la temperatura sarà sempre diversa:
la $p_3$ è la pressione che misura il termometro che misura quando lo metto in contatto
con l'acqua alla temperatura del punto triplo e cambierà sempre di poco. In questo modo posso ottenere una procedura
univoca per determinare la temperatura degli oggetti che tutti possono ripetere. 

\section{Digressione sui sistemi fluidi}
\subsection{L'equazione di stato per i gas perfetti}
\begin{wrapfigure}{r}{0.4\textwidth}
    \centering
    \caption{Trasformazioni nei gas ideali}
    \begin{tikzpicture}
        \draw(0, 0) -- (5, 0) node[at end, below] {$V$};
        \draw[->](0, 0) -- (0, 3.5) node[at end, left] {$p$};
        \draw[dashed] (1, 0) -- node[at start, below] {$V_0$} (1, 1) -- (0, 1)  node[at end, left] {$p_0$};
        \filldraw(1, 1) circle (1pt);
        \filldraw(2, 3) circle (1pt);
        \draw[dashed](2, 0) -- node[at start, below] {$V_1$} (2, 3) -- (0, 3)  node[at end, left] {$p_1$};
        \draw[dashed, ->](4.5, 1) .. controls  (3.5, 1.2) and (2.75, 1.5) .. (2, 3) node[at start, right] {$T_1$};
        \filldraw(4.5, 1);
        \draw[->](1, 1) -- (4.5, 1);
        \draw[dashed](4.5, 0) -- (4.5, 1) node[at start, below] {$V_2$};
        \draw[dashed](1, 1) .. controls (1.3, 0.45) and (1.7, 0.25) .. (2, 0.25);
        \draw[dashed](0.5, 2) .. controls (0.65, 1.35) and (0.95, 1) .. (1, 1);
    \end{tikzpicture}    
\end{wrapfigure}
In ogni sistema termodinamico all'equilibrio vale una
equazione di stato $f(p, V, T) = 0$. In alcuni casi
si è in grado di ricavare l'equazione di stato attraverso
la determinazione di un modello di stato attraverso la fisica
statistica e dagli esprimenti. L'esempio più semplice è quello del
gas perfetto: dagli esperimenti che ci portano a scrivere le quattro
leggi empiriche del gas si può ricavare una equazione di stato. 
Si può, data una mole di gas perfetto, mettersi nelle condizioni
standard: 
\begin{itemize}
    \item $T_0 = 273.15 \ K$: ossia la temperatura di fusione del ghiaccio;
    \item $p_0 = 1 \ atm$: Ossia la pressione dell'atmosfera al livello del mare;
\end{itemize}
Gli esperimenti ci dicono, a queste condizioni, che il gas occupa un volume 
ben preciso:
\begin{gather*}
    V = V_0  = 2.2414 \cdot  10^{-2} \ m^{3} = 22.4 \ l
\end{gather*}   
Possiamo allora utilizzare un diagramma per rappresentare l'equilibrio
dei gas: questo stato è associato ad una certa curva di temperatura secondo
la legge di Boyle. Posso allora applicare delle trasformazioni che mi permettano di raggiungere il 
volume $V_2$ alla pressione $p_0$ e lo posso esprimere in funzione
del volume precedente e della temperatura:
\begin{gather*}
    V_2 = V_0 \frac{T_1}{T_0 }
\end{gather*}
Se volessi farlo passare dallo stato $(p_1, V_1, T_1)$ allo stato $(p_0, V_2, T_1)$, allora dovrei fare in modo che il  sistema  mantenga
costante la temperatura ma possa variare la sua pressione in modo tale 
da avere il volume che io cerco. La temperatura di questi due stati di equilibrio
è la stessa ma ho cambiato sia il volume che la temperatura proprio
secondo la legge di Mariotte, deve risultare:
\begin{gather*}
    p_1 V_1 = p_0 V_2
\end{gather*}
Allora si ha, sostituendo l'espressione del volume 2:
\begin{gather*}
    p_1 V_1 = \frac{p_0 V_0}{T_0}T_1
\end{gather*}
Posso esprimere allora la costante che moltiplica la temperatura
come 
\begin{align}
    R = \frac{p_0V_0}{T_0} = 8.3145 \ \frac{J}{K}
\end{align}
Ossia l'unità di energia per grado Kelvin che prende il nome di 
\textbf{costante dei gas} che risulta vera per solo una mole di gas. Tuttavia, 
la legge di Avogadro mi dice che $pV$ è proporzionale al numero
di moli del gas, si ottiene quindi l'equazione di stato
per i gas perfetti:
\begin{align}
    pV = nRT
\end{align}
Queste considerazioni sono fatte a livello macroscopico: già il concetto
di moli non richiede una spiegazione a livello microscopico in quanto è un
concetto indipendente  e non necessita di una trattazione a livello microscopico.

\subsection{Il limite di utilizzo dell'equazione di stato dei gas}
\begin{wrapfigure}{r}{0.4\textwidth}
    \centering
    \caption{}
    \begin{tikzpicture}
        \draw[->](0, 0) -- (4, 0) node[at end, below] {$V$};
        \draw[->](0, 0) -- (0, 3.5) node[at end, left] {$p$};
        \draw(0.5, 3.5) .. controls (1, 2) and (3, 1.5) .. (3.5, 1.5);
        \draw(0.4, 3.5) .. controls (0.5, 1.75) .. (1, 1.75);
        \draw(1, 1.75) .. controls (2, 1.85) and (2.8, 0.5) .. (3.5, 0.5); 
        \filldraw(1, 1.75) circle (1pt) node[anchor = south] {$P_C$};
        \draw(0.3, 3.5) -- (0.5, 1) --  (2.25, 1) -- (3.5, 0.35);
    \end{tikzpicture}    
\end{wrapfigure}
Se si volesse analizzare il caso di un gas non perfetto ma molto semplice,
è ragionevole trattare l'equazione di stato dei gas in maniera tale che
assomigli ad uno sviluppo di Taylor e che i termini più grandi si diventino molto
piccoli quando il volume cresce molto:
\begin{gather*}
    pV = nRT\left(1 + \frac{B_1}{V} + \frac{B_2}{V^{2} } + \dots\right)
\end{gather*}
Si osserva che per gas anche non rarefatti, i coefficienti sono funzione
sia della temperatura che della composizione chimica del gas stesso. 
Si possono determinare sperimentalmente in modo tale che questa 
curva teorica mi produca, nel modo migliore possibile, i miei dati sperimentali.
Si fa dunque una ipotesi ragionevole per questa funzione anche se questa ipotesi
è valida solamente se la temperatura è sufficientemente grande. 
Dire che esistono temperature grandi e temperature piccole vuol dire 
che il sistema ha una scala di temperatura rispetto alla quale
ha senso parlare di temperatura grande. Tuttavia ad un certo punto
lo \textbf{sviluppo del viniale}, non funzionerà più: per trovare questo limite
possiamo eseguire una serie di esperimenti per alcuni gas abbassando sempre
di più la temperatura.  Ad un certo punto, se si continua ad aggiungere termini 
e a far decrescere la temperatura, si 
raggiunge un \textbf{isoterma critica} nella quale si ha un flesso e sotto la
quale la curva isoterma non è più una curva ma presenterà due punti angolosi tra i quali
è costante e dunque non è più descrivibile analiticamente. Ogni
gas ha una sua temperatura critica $T_c$ sotto alla quale lo sviluppo
del viniale non è più corretto. 


\begin{wrapfigure}{r}{0.4\textwidth}
    \centering
    \caption{La temperatura ad una pressione critica}
    \begin{tikzpicture}
        \draw[->](0, 0) -- (4, 0) node[at end, below] {$T$};
        \draw[->](0, 0) -- (0, 3) node[at end, left] {$V$};
        \draw(0.2, 0.2) .. controls (0.6, 0.3) .. (1, 0.5);
        \draw[dashed] (1, 0) -- (1, 3) node[at start, below] {$T_{\overline{p} }$};
        \draw(1, 2) .. controls (2, 2.2) and (2.7, 2.2) .. (3.5, 3);
    \end{tikzpicture}    
\end{wrapfigure}
La parte costante tra i due punti angolosi
rappresenta esattamente un cambio di stato: il sistema sta passando da gassoso a
liquido: più si prosegue verso sinistra nel grafico volume-pressione e più il liquido necessita 
di una pressione sempre più grande per piccole variazioni di volume.
Non si è sicuri che in una relazione tra tre variabili esse siano tutte e
tre indipendenti l'una dall'altra; ci possono essere delle situazioni in cui
esiste la funzione di stato ma presenta delle singolarità. 
\begin{gather*}
    V = V(p, T)
\end{gather*}
Questa funzione è continua e derivabile se si è a pressione maggiore di quella
critica, altrimenti non è regolare ma devo prendere la funzione con una
certa pressione assegnata $\overline{p}$:
\begin{gather*}
    V = V_{\overline{p} }(T) = V(\overline{p}, T )
\end{gather*} 
Quando la temperatura supera la temperatura
fissata a pressione $T_{\overline{p}}$  allora il sistema avrà grandi incrementi
di volume per piccoli incrementi di temperatura in quanto, sopra la temperatura 
che dipende dalla pressione fissata, il sistema inizierà un cambiamento di
stato da liquido a gassoso. 

\subsection{L'equazione di Van Der Walls per la descrizione dei gas}
Dato che la legge dei gas perfetti non vale per tutti i gas, si introduce una
equazione che possa funzionare anche lontano dal limite della curva critica
chiamata \textbf{equazione di Van Der Walls}:
\begin{align}
    \left(p + \frac{q}{V^{2} }\right)(V- b) = RT
\end{align}
Questa equazione prevede l'esistenza di un punto critico (ossia il flesso)
e fu derivata da Van Der Walls e riesce
a descrivere accuratamente le isoterme dei gas. Possiamo dimostrare che questa
equazione può prevedere l'esistenza di un punto critico: ossia deve esistere una situazione
per la quale la derivata seconda rispetto al volume deve essere zero.
\begin{gather*}
    p = \frac{RT}{V - b} - \frac{a}{V^{2} } \\
    \frac{\partial p}{\partial V} = -\frac{RT}{(V - b)^{2} } + \frac{2a}{V^{3} }\\
    \frac{\partial^{2} p}{\partial V^{2} } = \frac{2RT}{(V - b)^{2} } - \frac{6a}{V^{4} }
\end{gather*}

\begin{wrapfigure}{r}{0.4\textwidth}
    \centering
    \caption{le curve isoterme rispetto alla curva di temperatura critica}
    \begin{tikzpicture}
        \draw[->](0, 0) -- (4, 0) node[at end, below] {$V$};
        \draw[->](0, 0) -- (0, 4) node[at end, left] {$p$};
        \draw(0.5, 3.5) .. controls (1, 2) and (3, 1.5) .. (3.5, 1.5) node[at end, above] {$T > T_c$};
        \draw(0.4, 3.5) .. controls (0.5, 1.75) .. (1, 1.75);
        \draw(1, 1.75) .. controls (2, 1.85) and (2.8, 0.6) .. (3.5, 0.6) node[at end, right] {$T = T_c$}; 
        \draw(0.3, 3.5) .. controls (0.4, 0.5) .. (0.5, 0.5);
        \draw(0.5, 0.5) .. controls (0.8, 1.2) and (1, 1.2) .. (1.5, 0.9);
        \draw(1.5, 0.9) .. controls (2.2, 0.5) .. (3.5, 0.3) node[at end, right] {$T < T_c$};
    \end{tikzpicture} 
\end{wrapfigure}
Dato che devono entrambe essere zero:
\begin{gather*}
    \frac{2a}{V^{3}}  = \frac{RT}{(V - b)^{2} } \\
    \frac{6a}{V^{4}} = -\frac{2RT}{(V - b)^{2} } 
\end{gather*}
Si ottiene allora le espressioni per le tre grandezze critiche:
\begin{align*}
    RT_c &= \frac{8a}{27b} \\
    V_c &= 3b \\
    p_c &= \frac{a}{27b^{2} } 
\end{align*}
Se l'equazione di Van Der Walls è valida, allora deve valere che 
\begin{gather*}
    \frac{8}{3}\frac{p_cV_c}{RT_c} = 1
\end{gather*}
Che si può testare sperimentalmente.  La terza curva è una curva impossile che tuttavia
rispetta questa equazione: se quella situazione avvenisse, allora il sistema esploderebbe.
Si deve allora introdurre la correzione di MaxWell.

\section{Validità di Van Der Walls}
\begin{wrapfigure}{r}{0.4\textwidth}
    \centering
    \caption{L'equazione di van Der Walls}
    \begin{tikzpicture}
        \draw[->](0, 0) -- (4, 0) node[at end, below] {$V$};
        \draw[->](0, 0) -- (0, 4) node[at end, left] {$p$};
        \draw(0.5, 3.5) .. controls (0.6, 0.6) .. (0.75, 0.5);
        \draw(0.75, 0.5) .. controls (1.5, 1.8) and (2, 1.8) .. (3.5, 0.9);
        \filldraw[cyan, opacity = 0.3] (0.58, 1.25) .. controls (0.6, 0.6) .. (0.75, 0.5) .. controls (1, 1) .. (1.33, 1.25);
        \filldraw[cyan, opacity = 0.3] (1.33, 1.25) .. controls (1.5, 1.5) and (2, 1.68) .. (2.8, 1.25) -- (1.33, 1.25);
        \filldraw(0.4, 1) circle (0pt) node[anchor = center] {$A_1$};
        \filldraw(2, 1.8) circle (0pt) node[anchor = center] {$A_2$};
        \draw(0, 1.25) -- (4, 1.25); 
    \end{tikzpicture}    
\end{wrapfigure}
Possiamo riscrivere l'equazione di Van Der Walls come
\begin{gather*}
    V^{3}p - V^{2}(RT + bp) + aV -ab = 0  
\end{gather*}
Dobbiamo ora modificare questa equazione ancora per poter prevedere
la curva isoterma sotto la temperatura critica. Un modo per farlo è quello
di osservare gli esperimenti e scegliere il livello di pressione in modo tale
che per un certo valore di pressione le due aree $A_1$ e $A_2$ siano uguali.
Supponendo di metterci nel caso di temperatura inferiore alla temperatura critica
in modo tale che l'isoterma possa intersecare almeno una retta orizzontale
tre volte, MaxWell dice che si deve scegliere il livello di pressione in modo
tale che le due aree devono essere uguali. Allora i risultati che si ottengono
dalle relazioni con la temperatura critica sono congruenti con gli  esperimenti
che sono circa equivalenti ad uno (sempre minori di uno).

Si definiscono delle grandezze adimensionali 
\begin{gather*}
    \tilde{T} = \frac{T}{T_c} \qquad \tilde{V} = \frac{V}{V_c} \qquad \tilde{p} = \frac{p}{p_c}
\end{gather*}
Possiamo allora far scomparire i parametri $a, b$ dall'equazione secondo
la seguente:
\begin{gather*}
    \left(\tilde{p} + \frac{3}{\tilde{V}^{2} }\right)\left(3\tilde{V} - 1\right) = 8\tilde{T}
\end{gather*}
In questo modo con queste particolari unità si ottengono le stesse
isoterme per tutti i gas. Facendo infatti scomparire i termini $a, b$, questo vuol
dire che tutti i fluidi si comportano allo stesso modo ma semplicemente 
con scale termometriche diverse. Questa equazione prende il nome di \textbf{equazione di universalità}. 
Questa cosa non è per niente ovvia in quanto ad una prima analisi si potrebbe dire che 
i fluidi si comportino tutti in modo diverso; con questa equazione si è riassunto
il comportamento di tutti i fluidi all'interno di una unica equazione. Gli esperimenti
non confermano questa predizione approssimata: infatti se scalassi le curve sperimentali
con le temperature critiche non è vero che diventano le stesse isoterme. Tuttavia
più mi avvicino al punto critico e più questa equazione è precisa: al punto critico descrive, infatti,
perfettamente il comportamento di un gas. 

Successivamente, nel novecento, si è arrivati ad una equazione più precisa che
ci ha permesso di ottenere descrizioni molto più accurate per le 
curve isoterme dei gas. Ken o' Wilson nel 1982 è riuscito a costruire una teoria che ha permesso
di ottenere le descrizioni per le isoterme dei gas in modo accurato. 

\section{Diagramma di fase}
\begin{wrapfigure}{r}{0.45\textwidth}
    \centering
    \caption{Diagramma di fase}
    \begin{tikzpicture}[scale = 0.9]
        \draw[->](0, 0) -- (6, 0) node[at end, below] {$T$};
        \draw[->](0, 0) -- (0, 4.25) node[at end, left] {$p$};
        \draw(0, 0) .. controls (1, 0.75) and (1.5, 1.1) .. (2, 2);
        \draw(2, 2) .. controls (2.5, 2) and (3.25, 2.2) .. (4, 3);
        \draw(2, 2) .. controls (2.3, 2.6) and (2.4, 3) .. (2.5, 3.5);
        \filldraw(1, 3) node[anchor = center] {solido};
        \filldraw(3, 1) node[anchor = north] {gas};
        \filldraw(3, 3) node[anchor = center] {liquido};
        \filldraw(2, 2) circle (1pt) node[anchor = east] {$P_3$};
        \filldraw (4, 3) circle (1pt) node[anchor = south east] {$P_c$};
        \draw[dashed](4, 0) -- (4, 4) node[at start, below] {$T_c$};
        \draw[dashed](2, 0) -- (2, 2) node[at start, below] {$T_3$};
        \draw[dashed](2.5, 3.5)  -- (2.6, 4); 
        \node[align = center] (c) at (5.5, 3.5) {fluido \\ supercritico};
    \end{tikzpicture}    
\end{wrapfigure}
Per la rappresentazione degli stati di una sostanza si utilizzano 
dei grafici temperatura-pressione. Per una
qualsiasi sostanza pura esistono tre regioni separate da delle curve:
queste curve delineano i passaggi di stato. Come si osserva dal grafico,
è possibile ottenere dei passaggi di stato in modo tale da
passare direttamente da solido a gassoso o viceversa senza passare
dal liquido. Al centro del grafico c'è il \textbf{punto triplo}
dove le tre fasi sono in equilibrio; sulle curve che separano i vari stati
si ha l'equilibrio degli stati adiacenti alle curve stesse. 
C'è anche un altro punto notevole sul grafico: il \textbf{punto critico},
dove la curva del cambiamento di stato "finisce". Infatti prima del
punto critico io posso passare da uno stato all'altro semplicemente
modificando i parametri di temperatura e di pressione. Se io volessi prendere
i medesimi punti di inizio e arrivo ma volessi passare alla destra
del punto critico, io posso avere passaggio di stato senza nessun tipo di singolarità, per cui
il la sostanza può gradualmente passare da gas a liquido. La presenza del punto critico mi permette di
non ottenere passaggi bruschi (ossia le proprietà del fluido non "saltano" in modo non continuo)
da una fase all'altra: quella regione oltre il punto critico
prende il nome di \textbf{regione sovracritica}, nella quale
liquido e gas sono indistinguibili e hanno lo stesso comportamento. I fluidi
che sono in quella regione prendono il nome
di \textbf{fluido supercritico}. 
Si è detto che per la descrizione completa di un gas c'è bisogno di tre variabili: 
quelle che sono curve, nel grafico a tre dimensioni diventano superfici.

\chapter{Trasformazioni termodinamiche}
\section{Come avvengono le trasformazioni}
\begin{wrapfigure}{r}{0.4\textwidth}
    \centering
    \caption{Una trasformazione termodinamica}
    \begin{tikzpicture}
        \draw[->](0, 0) -- (4, 0) node[at end, below] {$V$};
        \draw[->](0, 0) -- (0, 4) node[at end, left] {$p$};
        \filldraw(1, 2.5) circle (1pt) node[anchor = south] {$I$};
        \filldraw(3, 1) circle (1pt) node[anchor = north] {F};
        \draw[->](1, 2.5) .. controls(1.8, 2.4) and (2.4, 2) .. (3, 1);
    \end{tikzpicture}    
\end{wrapfigure}
Non sempre si è in grado di determinare cosa accade tra un passaggio
di stato all'altro anche se si è in grado di determinare le fasi 
di arrivo e partenza. Dato un sistema termodinamico con uno stato
iniziale e finale, nel mezzo non so cosa accade al sistema poiché
le variabili termodinamiche sono ben definite solamente allo
stato di equilibrio. Se si immagina di fare questo cambiamento di stato
in maniera estremamente lenta, posso definire in maniera discreta tanti
microstati di equilibrio uno dopo l'altro come una successione di punti sul grafico
che determinano gli stati. Questo tipo di trasformazioni termodinamiche prendono
il nome di \textbf{trasformazioni statiche}: ossia delle trasformazioni 
di durata infinita. Dato che non dispongo di tempo infinito, posso definire un certo
margine entro il quale posso determinare queste trasformazioni come statiche; dunque queste trasformazioni,
se avvengono in maniera sufficientemente lenta, prendono il nome di \textbf{trasformazione quasi statiche}.
Quando l'istante iniziale e finale coincidono, questo tipo di trasformazioni quasi
statiche prendono il nome di \textbf{ciclo} o \textbf{trasformazione ciclica}.

\begin{wrapfigure}{r}{0.4\textwidth}
    \centering
    \caption{Trasformazioni isocore, isobare e isoterme}
    \begin{tikzpicture}
        \draw[->](0, 0) -- (4, 0)node[at end, below] {$V$};
        \draw[->](0, 0) -- (0, 4)node[at end, left] {$p$};
        \filldraw (1, 1.5) circle (1pt) node[anchor = east] {I};
        \filldraw (3, 1.5) circle (1pt) node[anchor = south] {$F_1$};
        \filldraw (3, 0.5) circle (1pt) node[anchor = west] {$F_2$};
        \draw[dashed] (1, 1.5) .. controls (1.5, 0.9) and (2, 0.7) .. (3, 0.5);
        \draw[thick](1, 1.5) -- (3, 1.5) -- (3, 0.5);        
    \end{tikzpicture}    
\end{wrapfigure}
Ci sono anche dei casi particolari delle trasformazioni quasi statiche:
\begin{itemize}
    \item \textbf{Isobare}: sono delle trasformazioni quasi statiche
    a pressione costante (quelle orizzontali nel grafico);
    \item \textbf{Isocore}: trasformazioni quasi statiche a volume costante ( quelle
    verticali nel grafico);
    \item \textbf{Isoterme}: trasformazioni quasi statiche a temperatura costante;
    in questo caso io non conosco l'equazione della curva isoterma ma posso
    determinare che se avviene alla stessa temperatura so che sta su di una curva.
\end{itemize}
Se la trasformazione non è quasi statica io non potrei rappresentarla
come è nel grafico poiché non conoscerei i valori della pressione e
del volume durante la trasformazione. Un caso particolare di 
trasformazione termodinamica è la \textbf{trasformazione adiabatica}: questa particolare
trasformazione è una sottoclasse delle trasformazioni isoterme.

\clearpage
\section{Espansione libera adiabatica di un gas}
\begin{wrapfigure}{r}{0.4\textwidth}
    \centering
    \caption{Espansione libera adiabatica}
    \begin{tikzpicture}
        \draw[thick](0, 0) rectangle (2, 1);
        \draw[thick](1, 0) -- (1, 1);
        \filldraw[color = cyan, opacity = 0.3] (0, 0) rectangle (1, 1);
        \node at(0.5, -0.3) {$V_I$};
        \draw[->](2.2, 0.5) -- (2.8, 0.5);
        \draw[thick](3, 0) rectangle (5, 1);
        \filldraw[color = cyan, opacity = 0.3] (3, 0) rectangle (5, 1);
        \node at(4, -0.3) {$V_F$};
    \end{tikzpicture}
\end{wrapfigure}
La situazione è quella di un recipiente a pareti adiabatiche: fuori dal sistema
c'è l'ambiente. L'espansione libera adiabatica di un gas è l'espansione di un
gas in una camera adiabatica divisa in due parti: se il gas sta in una
parte della parete (all'equilibrio) e si rimuove  (rapidamente) la parete che
separa le due parti della camera adiabatica, allora il gas
non ha più un muro che lo separa dall'altra parte della camera e 
dunque tenderà ad espandersi.
\begin{gather*}
    V_F > V_I 
\end{gather*}
Questo tipo di trasformazione non è quasi statica poiché
avviene molto rapidamente. Quale è la relazione tra la temperatura iniziale
e finale? In questo esperimento si osserva che la temperatura finale è \emph{poco} minore
di quella iniziale: per qualunque gas ideale la temperatura finale 
in una espansione libera adiabatica è sempre leggermente minore di quella iniziale.
Diminuendo il gas all'interno della camera adiabatica si ha che
la temperatura finale sarà minore di quella iniziale ma maggiore rispetto a quando
è presente molto più gas. Allora si ottiene che per un gas molto rarefatto:
\begin{gather*}
    \lim_{p \to 0} \text{ gas } \ \Longrightarrow \ T_F \to T_I 
\end{gather*}

\subsection{Espansione libera adiabatica quasi statica}
\begin{wrapfigure}{r}{0.4\textwidth}
    \centering
    \caption{Il pistone}
    \begin{tikzpicture}
        \draw[very thick](0, 0) rectangle (2, 2);
        \filldraw[cyan, opacity = 0.3] (0, 0) rectangle (2, 2);
        \draw[very thick] (2, 2) -- (3.5, 2 );
        \draw[very thick](2, 0) -- (3.5, 0);
        \draw[->, thick] (3, 1) -- (2, 1) node[midway, above] {$\vv{F}$ };
    \end{tikzpicture}    
\end{wrapfigure}
Per far si che l'espansione libera adiabatica sia quasi statica, io
non posso rimuovere la barriera immediatamente ma devo applicare una forza
sulla parete che sia in grado di equilibrarla esattamente: se la diminuissi
istante per istante, allora il gas si espanderebbe lentamente. In questo
modo riesco ad ottenere una espansione libera adiabatica di un gas in 
maniera quasi statica. Si potrebbe ottenere la \textbf{compressione} adiabatica
del sistema in maniera quasi statica se invece facessi crescere istante per istante
la forza necessaria. Il contenitore in figura prende il nome di \textbf{pistone adiabatico}. 
Si ottengono allora le seguenti situazioni:
\begin{itemize}
    \item $\Delta V > 0 \ \Longrightarrow \ \Delta T < 0$;
    \item $\Delta V < 0 \ \Longrightarrow \ \Delta T > 0$; 
    \item $\Delta V \neq 0 \ \Longrightarrow \ \Delta T \neq 0$. 
\end{itemize}
Non esistono solamente trasformazioni adiabatiche. 

\section{Trasformazioni non adiabatiche}
\subsection{Il termostato}
\begin{wrapfigure}{r}{0.4\textwidth}
    \centering
    \caption{Termostato}
    \begin{tikzpicture}[scale=0.85]
        \draw(-2, -1) rectangle (0, 3) node[midway] {$T$};
        \draw(0, 0) rectangle (2, 2);
        \filldraw[cyan, opacity = 0.3] (0, 0) rectangle (2, 2);
        \draw[very thick] (0, 2) -- (3.5, 2);
        \draw[very thick](0, 0) -- (3.5, 0);
        \draw[very thick](2, 0) -- (2, 2);
        \draw[->, thick] (3, 1) -- (2, 1) node[midway, above] {$\vv{F}$ };
    \end{tikzpicture}    
\end{wrapfigure}
Per studiare trasformazioni non adiabatiche si introduce il concetto di
\textbf{termostato}: ossia un sistema termodinamico tale che la sua
temperatura possa sempre rimanere costante qualsiasi cosa gli venga fatta. Ovviamente
il termostato perfetto non esiste anche se è possibile definire termostati rispetto
ad altri sistemi se e solo se la temperatura del termostato varia molto meno rispetto al sistema
a cui è messo a contatto. Si può creare un termostato con un sistema di massa molto 
grande rispetto all'altro sistema con il quale lo stiamo facendo interagire.
Se $M$ è la massa del termostato e $m$ è la massa del sistema si deve avere
\begin{align}
    M >> m
\end{align}
La parete con cui il pistone è a contatto con il termostato è una parete diatermica:
per costruzione allora $\Delta T = 0$, ossia la compressione o espansione
quasi statica avviene sempre a temperatura costante. Se io attaccassi un sistema
ad un termostato anche se la trasformazione non avviene in maniera quasi statica,
la temperatura iniziale e quella finale coincideranno poiché,
data la natura del termostato, la temperatura rimane la stessa (anche se
durante la trasformazione può cambiare). Queste tipologie di trasformazioni prendono
il nome di \textbf{isoterme}. 

Se volessi fare delle trasformazioni a volume costante con aumento di temperatura 
in maniera quasi statica, potrei introdurre una successione di termostati con temperatura
sempre crescente in modo tale da raggiungere l'equilibrio istante per istante. Se non facessi tutti  i passaggi
intermedi, allora non sarebbe una trasformazione quasi statica e dunque non potrei dire che
la pressione cresce in maniera proporzionale rispetto alla temperatura.

\subsection{Il lavoro delle forze di pressione di un fluido}
\begin{wrapfigure}{r}{0.4\textwidth}
    \centering
    \caption{Il lavoro delle forze di pressione}
    \begin{tikzpicture}
        \draw[very thick](0, 0) rectangle (2, 2);
        \draw[very thick](2, 2) -- (4, 2);
        \draw[very thick](2, 0) -- (4, 0);
        \draw[<->](2, -0.2) -- (2.4, -0.2) node[midway, below] {$dl$};
        \filldraw[cyan, opacity = 0.3] (0, 0) rectangle(2, 2);
        \draw[dashed](2.4, 0) -- (2.4, 2);
    \end{tikzpicture}    
\end{wrapfigure}
L'unica forza che può imprimere un fluido che è compresso o che si espande è
la forza che deriva dalla pressione del gas. Posso quantificare il lavoro 
(in maniera meccanica) della pressione su di un gas considerando un pistone
con all'interno un fluido. Dato che è un pistone, allora il gas è in grado di essere
compresso oppure è in grado di espandersi a seconda della forza che vi 
è sulla parete mobile del pistone. Se si considerasse il caso in cui il gas si espande di un certo 
$dl$, allora, dato $\vv{dl} = dl \hat{i}$:
\begin{gather*}
    V \to V + dV
\end{gather*}  
Le forze di pressione agiscono sulla parete e dunque compiono un lavoro
infinitesimo che è dato da:
\begin{gather*}
    \delta L = \vv{F} \cdot  \vv{dr}  
\end{gather*}
Possiamo esprimere allora la forza dovuta dalla pressione come
\begin{gather*}
    \vv{F} = pA \hat{i}  
\end{gather*}
Dove $A$ è la superficie di contatto della parete mobile del pistone 
lungo il versore $\hat{i}$. Si ottiene allora l'espressione del lavoro infinitesimo
come :
\begin{gather*}
    \delta L = pA \hat{i} \cdot  dl \hat{i} \ \Longrightarrow \ pAdl 
\end{gather*} 
Ricordando allora che $Adl$ è esattamente il volume infinitesimo di cui si è espanso il corpo:
\begin{align}
    \delta L = pdV
\end{align}
Il lavoro è positivo se il fluido si espande ed è negativo se 
invece è compresso. Questa espressione del lavoro vale per qualunque
trasformazione infinitesima di un fluido quando varia di un certo  $dV$
ben definito. 

\begin{wrapfigure}{r}{0.4\textwidth}
    \centering
    \caption{L'espansione libera di un fluido}
    \begin{tikzpicture}
        \draw(0, 0) ellipse (2 and 1) node[midway] {$V$};
        \draw[dashed](0, 0) ellipse (2.2 and 1.1);
        \draw[->](2.5, -1.2) -- (1.8, -0.8) node[at start, right] {$V + dV$};
    \end{tikzpicture}    
\end{wrapfigure}
Dato un fluido racchiuso in una forma arbitraria, posso applicargli una
trasformazione in maniera tale che il fluido possa variare il suo volume
in modo tale che il volume occupato dall'oggetto è allora $V + dV$. Se io
considerassi una parte del fluido, potrei approssimare la curva arbitraria
come se ci fossero tanti pistoncini che si stanno espandendo dalla curva
normale a quella tratteggiata. La sezione del pistone non importa; posso
allora pensare che ognuno di questi pistoncini si espandano con $dl_1, dl_2\dots$
tutti diversi tra di loro. Posso allora considerare un solo pistoncino per il quale
il lavoro compiuto dalla pressione dal gas per l'espansione infinitesima:
\begin{gather*}
    \delta L_i = p  \ dA_i \ dl_i
\end{gather*}
Allora il lavoro totale dei pistoncini diventerà la sommatoria
dei lavori infinitesimi
\begin{gather*}
    \delta L = \sum p \ dA_i \ dl_i 
\end{gather*}
Dato che $p$ è costante per tutta la superficie del fluido, e che
$dV_i = dA_i \ dl_i$, si ha il lavoro totale come:
\begin{gather*}
    \delta L = p\sum dV_i
\end{gather*}

\begin{wrapfigure}{r}{0.4\textwidth}
    \centering
    \caption{Il lavoro lungo una curva generica}
    \begin{tikzpicture}
        \draw[->](0, 0) -- (3, 0) node[at end, below] {$V$};
        \draw[->](0, 0) -- (0, 3) node[at end, left] {$p$};
        \draw(0.5, 1) .. controls (1, 1.4) and (1.5, 1.4) .. (2.5, 1);
        \filldraw(0.5, 1) circle (1pt) node[anchor = south] {I};
        \filldraw(2.5, 1) circle (1pt) node[anchor = south] {F};
        \draw[dashed](0.5, 0) -- (0.5, 1);
        \draw[dashed](2.5, 0) -- (2.5, 1);
        \filldraw[cyan, opacity = 0.3] (2.5, 1) -- (2.5, 0) -- (0.5, 0) -- (0.5, 1) .. controls (1, 1.4) and (1.5, 1.4) .. (2.5, 1);
    \end{tikzpicture}    
\end{wrapfigure}
Se il volume del fluido variasse di una quantità finita, allora
non saprei dire il lavoro della forza di pressione in quanto la trasformazione non
sarebbe quasi statica. Se fosse quasi statica, per ogni stato di equilibrio,
dovrei avere fissata la pressione del fluido e dunque il lavoro della pressione in una 
trasformazione quasi statica è dato da:
\begin{align}
    L = \int_{V_i}^{V_f}p \ d V
\end{align}  
dove $p$ è esattamente una funzione del volume  $p(V)$ (che non so). Non
posso però considerare $\delta L$ come $dL$, ossia non posso dire che
il lavoro infinitesimo sia il differenziale esatto in quanto
le funzioni $p(V)$ possono essere infinite e dunque il lavoro non è
una funzione di stato. Durante le espansioni libere, il lavoro delle forze di
pressione è nullo in quanto la parete è rimossa istantaneamente e sull'ambiente 
esterno non si è spostato nulla. 


\subsection{Trasformazioni quasi statiche a pressione costante}
Dato che la pressione è costante, allora è valida la relazione 
del lavoro (anche perché è quasi statica) e dunque il lavoro 
è proprio l'integrale
\begin{align}
    L = p\int_{V_i}^{V_f}
\end{align}

\subsection{Trasformazioni quasi statiche isoterme}
Dato che è isoterma, allora $T$ è costante: non so per un sistema
qualunque quanto può fare il lavoro delle forze di pressione, anche
se il sistema è quasi statico. Tuttavia, nel caso in cui la funzione di stato
allora posso ricavarmi una relazione per poter esprimere l'equazione della curva
isoterma: allora in quel caso posso determinare il lavoro delle forze di pressione.
Dato che, per un gas perfetto, $pV = nRT$, allora
\begin{align}
    L = \int_{V_i}^{V_f} \frac{nRT}{V} \ dV \ \Longrightarrow \ L = nRT \cdot  \ln\frac{V_f}{V_i}
\end{align}

\subsection{Lavoro del sistema e dell'ambiente}
\begin{wrapfigure}{r}{0.4\textwidth}
    \centering
    \caption{Il lavoro del gas su di un corpo di massa $m$}
    \begin{tikzpicture}
        \draw[very thick](0, 0) --  (2, 0) -- (2, 4);
        \draw[very thick](0, 0) -- (0, 4);
        \draw[very thick](0, 2) -- (2, 2);
        \draw(0.5, 2) rectangle (1.5, 3) node[midway] {$m$};
        \filldraw[cyan, opacity = 0.3](0, 0) rectangle (2, 2);
        \draw[|-|](-0.2, 2) -- (-0.2, 3.2) node[midway, left] {$h$};
        \draw[dashed](0, 3.2) -- (2, 3.2);
        \draw[dashed](0.5, 3.2) rectangle (1.5, 4.2) node[midway] {$m$};
    \end{tikzpicture}    
\end{wrapfigure}
Posso dare alcune definizioni tecniche:
\begin{itemize}
    \item $\delta L > 0$: il sistema compie del lavoro sull'ambiente esterno
    \item $\delta L < 0$: l'ambiente esterno compie del lavoro sul sistema.
\end{itemize}
Il lavoro è il modo meccanico per far sì che ambiente e 
sistema si scambino energia: questo vuol dire che è possibile anche
immagazzinare energia sottoforma di energia potenziale.  Nell'esempio semplice
di un peso su di un pistone, posso dire che l'energia potenziale dell'ambiente
è variata secondo
\begin{gather*}
    \Delta U(A) = mgh
\end{gather*}
Posso esprimere il lavoro che ha compiuto la pressione
come
\begin{gather*}
    L(S) = p\Delta V \ \Longrightarrow \ L = \frac{mg}{B} hB \ \Longrightarrow \ L(S) = mgh
\end{gather*}
Dove $B$ è la superficie del pistone, $F$ è la forza che scaturisce la pressione
del gas e $h$ la variazione di quota dovuta alla pressione del gas.

\clearpage
\section{Lavoro del sistema durante una trasformazione adiabatica}
\begin{wrapfigure}{r}{0.4\textwidth}
    \centering
    \caption{Ambiente adiabatico}
    \begin{tikzpicture}
        \filldraw[cyan, opacity = 0.3](0, 0) rectangle (1, 2);
        \draw[very thick](0, 0) rectangle (5, 2);
        \draw[very thick](1, 0) -- (1, 2);
        \draw[very thick](2, 0) -- (2, 2);
        \draw(2, 1) -- (5.5, 1);
    \end{tikzpicture}    
\end{wrapfigure}
Per arrivare a definire il lavoro del sistema in una trasformazione adiabatica
si può partire da considerare una situazione semplice in cui il sistema è
in espansione. Dato che non esiste un unico modo per poter passare da uno
stato iniziale I ad uno stato finale F, allora posso utilizzare una
scatola, con pareti adiabatiche, che mi permetta di ottenere 3 trasformazioni
per riempire tutta la scatola. Considerando un gas perfetto e molto rarefatto
all'interno di una scatola che presenta due pareti: la prima parete, così come la seconda, presentano delle aperture che
possono essere controllate dall'esterno senza interferire sullo stato del sistema.
La seconda parete, è controllata dall'esterno tramite un'asta, in modo tale che possa muoversi
dalla sua posizione iniziale fino a tutta la lunghezza della scatola. 


\begin{wrapfigure}{r}{0.4\textwidth}
    \centering
    \caption{La schematizzazione delle trasformazioni}
    \begin{tikzpicture}
        \draw[->](0, 0) -- (4, 0) node[at end, below] {$V$};
        \draw[->](0, 0) -- (0, 4) node[at end, left] {$p$};
        \draw[dashed](1, 0) -- (1, 3.5) node[at start, below] {$V_0$};
        \draw[dashed](2, 0) -- (2, 3) node[at start, below] {$V_1$};
        \draw[dashed](3, 0) -- (3, 1.5) node[at start, below] {$V_2$};
        \draw[dashed](4, 0) -- (4, 1.3) node[at start, below] {$V_3$};
        \filldraw(1, 3.5) circle (1pt);
        \filldraw(2, 3) circle (1pt);
        \filldraw(3, 1.5) circle (1pt);
        \filldraw(4, 1.3) circle (1pt);
        \draw[thick, ->](1, 3.5) .. controls (1.3, 3.2) and (1.7, 3) .. (2, 3) node[at end, right] {$T_0 = T_1$};
        \draw[dashed](1, 2.5) .. controls (2, 1.7) and (3, 1.45) .. (4, 1.3) node[at end, right] {$T_2 = T_3$};
        \draw[thick, ->](2, 3)  -- (3, 1.5);
        \draw[thick, ->](3, 1.5) .. controls (3.4, 1.4) and (3.7, 1.35) .. (4, 1.3);
    \end{tikzpicture}    
\end{wrapfigure}
Posso riassumere tutte le trasformazioni nel grafico volume-pressione,
di seguito sono elencate in ordine le trasformazioni: 
\begin{itemize}
    \item \textbf{I trasformazione}: la prima trasformazione è una espansione libera
    adiabatica (la cui trasformazione sta sull'isoterma che comprende le due temperature
    iniziali e finali):
    \begin{gather*}
        V_1 > V_0 \quad T_0 = T_1 \ \Longrightarrow \ L_{0, 1} = 0
    \end{gather*}
    \item \textbf{II trasformazione}: La seconda trasformazione consiste nell'espansione del 
    volume del gas in maniera quasi statica, controllando il movimento della seconda parete. 
    Questa trasformazione, essendo quasi statica, è una curva sul piano che non si sa disegnare,
    anche se si sa che questa curva deve essere tale che la temperatura dello stato in cui si atterra
    deve essere una temperatura diversa da quella che si aveva all'inizio.
    \begin{gather*}
        V_2 > V_1 \quad T_2 < T_1 \ \Longrightarrow \ L_{1, 2} = \int_{V_1}^{V_2}p \ dV
    \end{gather*}
    \item \textbf{III trasformazione}: A questo punto, apro la seconda parete, così inizia unn'altra
    espansione libera del gas, per cui
    \begin{gather*}
        V_3 > V_2 \quad T_3 = T_2 \ \Longrightarrow \ L_{2, 3} = 0
    \end{gather*}
\end{itemize}
Dato che questa è solamente una trasformazione di tutte le infinite trasformazioni possibili, 
è possibile ricavare sempre altre tre trasformazioni per cui è possibile passare dallo stesso stato iniziale allo stesso stato finale.
Qualunque sia l'insieme delle trasformazioni applicate al sistema,
il lavoro complessivo sarà sempre dato dal solo contributo
dell'espansione del volume del gas in maniera quasi statica dovuto al
controllo del movimento della seconda parete. 
Il lavoro adiabatico su di una curva qualsiasi $\gamma_1$ è lo stesso
di quello per un'altra curva $\gamma_2$ se le due trasformazioni
hanno lo stesso stato iniziale e finale (anche se le trasformazioni non sono
quasi statiche). 
\begin{gather*}
    L_{I \to F, \gamma_1}^{(adiabatica)} = L_{I \to F, \gamma_2}^{(adiabatica)}  
\end{gather*}
C'è anche un altro aspetto su di questo: se questa trasformazione 
è anche ciclica, allora il suo lavoro è sempre zero. 
\begin{gather*}
    L_{I \to I}^{(adiabatica)} = 0
\end{gather*}
Queste conseguenze rispecchiano i dati sperimentali, anche se non mi permettono di
determinare una legge fisica che mi permetta di esprimere questa situazione. 
Se io avessi una quantità di lavoro adiabatico che compie il mio sistema 
che dipende solamente dallo stato iniziale e finale, allora ci deve essere necessariamente
una funzione che mi permetta di esprimere il lavoro in termini dello stato iniziale 
e dello stato finale:
\begin{align}
    L_{i \to f}^{(adiabatica)} = F(i, f) = F(p_i, V_i, p_f, V_f) 
\end{align}
Inoltre questa deve essere tale che se $i \equiv f$ allora si annulla. Questa 
funzione assomiglia molto all'energia potenziale nelle forze conservative nella
meccanica (anche se non si sta parlando di energia potenziale). Si definisce
questa funzione come \textbf{Energia interna}, che è definita come 
\begin{gather*}
    \Delta U = U(f) - U(i) = -L_{i \to f}^{(adiabatica)} 
\end{gather*}
Questa non è energia potenziale anche se utilizza lo stesso simbolo, ma è
una funzione di stato. Si dimostra che è definita a meno di una costante
poiché se si definisce $U' = U +c$, nella differenza è possibile rimuovere la costante,
anche se la funzione $U$ non è univoca. In generale dipende solamente dallo stato iniziale e finale
poiché c'è la condizione $F(i, i) = 0$.

\chapter{Primo principio della termodinamica}
In generale non esiste una trasformazione adiabatica che mi permetta di passare
da uno stato finale $B$ ad uno stato iniziale $A$, anche se esiste
una trasformazione adiabatica che mi permetta di passare da $A$ a $B$. 
Nel gas perfetto, per esempio, so che esiste sempre una trasformazione 
adiabatica che mi permetta di passare da 
\begin{gather*}
    A(V_1, T) \to B(V_2, T) \qquad V_2 > V_1
\end{gather*}
anche se non esiste una trasformazione adiabatica che mi permetta di tornare 
indietro (in altre parole non esiste la compressione adiabatica poiché
la temperatura di un gas si alza sempre quando viene compresso).
Si può dimostrare che per solo per un caso particolare esistono
trasformazioni termodinamiche adiabatiche per entrambi i
sensi; supponendo che si abbiano due trasformazioni adiabatiche
in entrambi i sensi:
\begin{gather*}
    A \to B \qquad B \to A
\end{gather*}
L'energia interna del sistema per le due trasformazioni sarà data:
\begin{gather*}
    U(B) - U(A) = -L_{A \to B}^{(adiabatica)} \\
    U(A) - U(B) = -L_{B \to A}^{(adiabatica)} 
\end{gather*}
Ossia, quando esistono entrambe le possibilità il lavoro che va da
uno stato all'altro è esattamente l'opposto del lavoro che
va dall'altro stato a quello di partenza:
\begin{gather*}
    L_{A \to B} = -L_{B \to A}
\end{gather*}
Tutte le volte che non esiste una trasformazione adiabatica in un verso, esisterà
nel verso opposto per qualsiasi stato iniziale e finale e per qualsiasi trasformazione 
termodinamica. Il resto del primo principio della termodinamica è che il lavoro
del sistema non dipende dal set di trasformazioni che compie il sistema per
raggiungere uno stato finale qualsiasi da uno stato iniziale qualsiasi.
In generale posso dire che se il lavoro è esprimibile come funzione
di stato (ossia in condizioni adiabatiche) che
\begin{align}
    \Delta U = -L^{(adiabatica)} 
\end{align}
Dato che l'energia interna del sistema è proprio il lavoro del sistema,
allora il lavoro che compie il sistema corrisponde alla diminuzione 
dell'energia all'interno del sistema e che può essere usata per compiere lavoro
meccanico sull'ambiente esterno. 

Se il lavoro non è compiuto da una trasformazione adiabatica, $U$ non è una funzione di stato ed
è quindi esprimibile sempre a prescindere dalle trasformazioni del sistema e dai suoi stati. Inoltre, la somma
tra l'energia potenziale ed il lavoro del sistema è definita come una quantità
chiamata \textbf{quantità di calore}:
\begin{align}
    \Delta U + L = Q
\end{align} 
Quando si analizza una trasformazione qualunque si ha il calore in modo univoco 
(e senza dover sapere gli stati del sistema) e dunque
posso riesprimere l'energia interna come
\begin{align}
    Q - L = \Delta U
\end{align}
E prende il nome di funzione matematica integrale del primo principio della termodinamica. 
\begin{itemize}
    \item $Q > 0$: Il sistema ha acquisito dell'energia dall'ambiente;
    \item $Q < 0$: Il sistema ha ceduto dell'energia all'ambiente.
\end{itemize}
Un sistema può scambiare con l'esterno o lavoro meccanico o lavoro.
Si può esprimere come conseguenza la forma differenziale del primo principio
della termodinamica:
\begin{gather*}
    \delta Q - \delta L = dU
\end{gather*}
A livello matematico è notevole in quanto le due espressioni sono 
legate tra di loro.  
\begin{itemize}
    \item Adiabatica: per definizione, una trasformazione adiabatica
    ha quantità di calore zero: le pareti adiabatiche allora non permettono il  
    passaggio di energia di tipo termico e permettono solamente gli scambi energetici di tipo meccanico.
    \item Trasformazione ciclica: $\Delta U = 0$, allora $Q - L = 0$ e $Q = L$. La quantità
    di calore fornito deve equivalere al lavoro svolto dal sistema. Il lavoro sarà positivo se percorro
    la curva in senso orario o negativo se la percorro in senso antiorario.
\end{itemize}
Dato che si è data una definizione operativa di calore secondo il primo principio, posso dire che
la quantità di calore che scambio in una certa trasformazione, per definizione, è
data da:
\begin{gather*}
    Q_{A \to B} = \Delta U + L_{A \to B} \qquad \Delta U = U(B) - (A) 
\end{gather*} 
A pressione costante se si fornisse del calore senza che aumenti il volume
del sistema, allora posso dire che
\begin{gather*}
    Q = \Delta U \qquad L \approx 0
\end{gather*}

\section{La capacità termica}
\subsection{Il mulinello di Joule e la definizione di capacità termica}
L'esperimento che fece Joule per definire il calore utilizzava un contenitore
con all'interno un sistema di pale e sopra un mulinello collegato alle
pale con dei pesi per ogni parte. L'energia potenziale dei pesi viene trasferita al sistema
sottoforma di attrito delle pale con l'acqua, in modo tale che tutta l'energia che 
viene fornita al sistema è dovuta solo al calore e non al lavoro 
per cui
\begin{gather*}
    Q = \Delta U \ \Longrightarrow \ Q \propto \Delta T
\end{gather*} 
Il che mi suggerisce che
\begin{gather*}
    \delta Q = c\ dt
\end{gather*}
con $\mathcal{C}$ si indica la \textbf{capacità termica} che è definita come la quantità di  calore infinitesima
che devo fornire al sistema per ottenere un aumento di temperatura infinitesimo. Tuttavia
questo non è applicabile per tutti gli esprimenti (ossia quelli nei quali
si fornisce solamente calore al sistema senza fare alcun tipo di lavoro meccanico).
Dato che il calore fornito dipende dal tipo di trasformazione, 
allora devo definire la capacità termica quando fisso la pressione (per le
isobare) e quando fisso il volume (per le isocore)  in maniera diversa:
\begin{gather*}
    \mathcal{C}_p = \left(\frac{\delta Q}{dt}\right)_p \qquad c_V = \left(\frac{\delta Q}{dt}\right)_V
\end{gather*}
La capacità termica adiabatica è zero per qualsiasi sistema, mentre la capacità
termica di una trasformazione non è definita. 
Generalmente per i liquidi $\mathcal{C}_P \approx \mathcal{C}_V$ mentre per i gas sono molto diversi. 
Si definisce il \textbf{calore specifico} come
\begin{align}
    c = \frac{\mathcal{C}}{M}
\end{align}
Ossia il calore per quantità di massa. Si può anche definire
il calore per quantità di sostanza, come 
\begin{align}
    c = \frac{\mathcal{C}}{n}
\end{align}
Ossia la capacità termica per numero di mole. In generale, dato che è una quantità
additiva, si ha che $\mathcal{C} \geq 0$.




\end{document}
\documentclass[a4paper, oneside]{article}
\usepackage{graphicx}
\usepackage{amsthm}
\usepackage{amsmath}
\usepackage{amssymb}
\usepackage[a4paper,
            bindingoffset=0.2in,
            left=2cm,
            right=2cm,
            top=2cm,
            bottom=2cm,
            footskip=.25in]{geometry}
\usepackage[italian]{babel}
\usepackage{pgfplots}
\usepackage{tabularx}
\usepackage{tikz}
\usepackage{wrapfig}
\usepackage{color}
\usepackage[d]{esvect}
\definecolor{page}{rgb}{0.129,0.157,0.212}
\pagecolor{page}
\color{white}
\graphicspath{ {./images/} }
\usetikzlibrary{shapes.geometric}
\usetikzlibrary{datavisualization}
\usetikzlibrary{datavisualization.formats.functions}
\usetikzlibrary{patterns}
\pgfplotsset{width=10cm,compat=1.9}

\title{Appunti di Ottica}
\author{Tommaso Miliani}
\date{20-10-25}

\begin{document}
\newtheoremstyle{theoremEnv}
                {}          % Space above
                {}          % Space below
                {\slshape}  % Body font
                {}          % Indent amount
                {\bfseries} % Head font
                {.}         % Punctuation after head
                {\newline}         % Space after theorem head
                {}          % Theorem head spec
\theoremstyle{theoremEnv}

\newtheorem{definition}{Definizione}[section]
\newtheorem{theorem}{Teorema}[section]
\newtheorem{lemma}{Proposizione}[section]
\newtheorem{observation}{Osservazione}[section]
\newtheorem{corollary}{Corollario}[theorem]
\newtheorem{example}{Esempio}[section]

\maketitle

\section{Sommare due lunghezze d'onda diverse}
\begin{wrapfigure}{r}{0.4\textwidth}
    \centering
    \caption{L'interferometro di Machelson}
    \begin{tikzpicture}
        \draw[->](0, 0.2) -- (2.25, 0.2);
        \draw[->](2.25, 0.2) -- (2.25, 2);
        \draw[->](2.5, 2) -- (2.5, 0.75);
        \draw[->](2.5, 0.75) -- (1, 0.75);
        \draw[->](2.5, 0.75) -- (2.7, 0.35) -- (2.7, -2);
        \draw[->](2.25, 0.2) -- (2.55, 0) -- (4.5, 0);
        \draw[->](4.5, -0.25) -- (2.42, -0.25);
        \draw[->](2.42, -0.25) -- (2.42, -2);
        \draw[->](2.42, -0.25) -- (2.15, -0.1) -- (0, -0.1);
        \draw(1.75, -1) -- (2, -1.2) -- (3, 1) -- (2.75, 1.2);
        \draw[very thick](2.75, 1.2) -- (1.75, -1);
        \draw[pattern = north west lines, pattern color = white](1.75, 2) rectangle (3, 2.5);
        \draw[pattern = north west lines, pattern color = white](4.5, 0.75) rectangle (5, -1);
        \filldraw (2.1, -2) rectangle (3, -2.25);
        \draw[|-|, red](2, 1.9) -- (2, 0.3) node[midway, left] {$L_1$};
        \draw[|-|, red](2.5, -0.5) -- (4.4, -0.5) node[midway, below] {$L_2$};
    \end{tikzpicture}
\end{wrapfigure}
Se i due fasci in ingresso sono di lunghezze d'onde diverse, si è già detto 
l'intensità totale come è ottenuta. Se le intensità sono equivalenti per entrambi i
fasci di luce, allora posso esprimere le intensità come
\begin{gather*}
    I_1 = I_2 = I_0
\end{gather*}
Sfruttando le formule di Prostaferesi e le considerazioni trigonometriche
sul seno quadro per cui $\sin^{2}(\alpha) = \frac{1 - \cos(\alpha)}{2}$, posso esprimere
l'intensità totale dei due fasci luminosi come 
\begin{gather*}
    I_0 \left(1 - \cos((k_1 + k_2) \Delta L) \cdot  \cos((k_1 - k_2) \Delta L)\right)
\end{gather*}
Il primo contributo del coseno rappresenta l'oscillazione più lunga mentre il secondo
è una oscillazione più corta (spazialmente) per cui la variazione di lunghezza d'onda
$\delta l$ per entrambi i fasci sono:
\begin{gather*}
    \delta l_1 = \frac{2\pi}{k_1 + k_2} \\
    \delta l_2 = \frac{2\pi}{k_1 - k_2}
\end{gather*}
L'ampiezza del secondo termine determina la modulazione dell'onda che introduce
una dipendenza spaziale all'oscillazione delle onde. Tramite l'interferometro
e spostando gli specchi di un certo $\Delta L$, la luce in ingresso nell'interferometro 
risulta modulata.

\subsection{Lo spettro continuo}
\begin{wrapfigure}{r}{0.4\textwidth}
    \centering
    \caption{L'intensità luminosa in funzione della pulsazione}
    \begin{tikzpicture}[domain=0.5:3.5]
        \draw[->](0, 0) -- (4, 0) node[at end, below] {$\omega$};
        \draw[->](0, 0) -- (0, 3) node[at end, left] {$I(\omega)$};
        \draw plot (\x, {3* e^(-(\x - 2)^2)});
        \draw[dashed](2, 3) -- (2, 0) node[at end, below] {$\omega_0$};
        \draw[|-|](2.1, 1.5) -- (2.75, 1.5) node[midway,below] {$\Delta \omega$};
    \end{tikzpicture}    
\end{wrapfigure}
Nel caso in cui si utilizzi una sorgente luminosa di luce bianca: questa conterrà tutte le
lunghezze d'onda del visibile in maniera continua. Si può ora andare ad analizzare
cosa accade nell'interferometro al variare di $\Delta L$. 
La luce emessa da un led bianco è centrato intorno ad un valore di $\omega_0$ e $\lambda_0$:
con una certa distribuzione dell'intensità luminosa, io so che
\begin{gather*}
    \omega_0 = ck_0 \ \Longrightarrow \ \omega_0 = c \frac{2\pi}{\lambda_0}
\end{gather*}
Dato che il led è costruito in modo tale che sia visibile all'occhio umano,
io posso fare in modo che il $\Delta \lambda$ sia circa uguale a $150 nm$ per
poter stare nel visibile. Per trovare il corrispettivo intervallo
in termini di $\omega$ posso dire che, dato che esiste una relazione lineare tra
$\omega$ e $\lambda$,
\begin{gather*}
    \frac{\Delta \lambda}{\lambda_0} = \frac{\Delta \omega}{\omega_0}
\end{gather*}
Conosciuto $\Delta \lambda$, allora è possibile trovare $\Delta \omega$, ossia
l'intervallo della pulsazione della luce. Il $\Delta \lambda$ e $\Delta \omega$ sono
delle stime qualitative della distanza del centro della distribuzione
gaussiana della lunghezza d'onda dal punto di metà altezza . Con
queste considerazioni posso determinare l'intensità luminosa che
riceve il rilevatore in funzione dello scostamento delle distanze
dei due specchi $\Delta L$ in maniera discreta:
\begin{gather*}
    I(\Delta L) = \sum I_i \sin^{2}(k_i \Delta L) 
\end{gather*}
Posso allora passare al caso continuo attraverso l'integrale
nella seguente maniera:
\begin{gather*}
    I(\Delta L) = \int I(\omega)\Delta \omega \sin^{2}(k_\omega \Delta L) = \int I(\omega) \sin^{2}\left(\frac{\omega}{c}\Delta L\right)d\omega  
\end{gather*}
Il led che si ha non ha necessariamente una distribuzione gaussiana delle lunghezze
d'onda centrate intorno ad un certo $\lambda_0$ ma supponiamo che
lo sia entro certe approssimazioni. Posso allora cercare di risolvere l'integrale 
della funzione dell'intensità luminosa secondo la distribuzione Gaussiana:
\begin{gather*}
    I(\omega) = \frac{I_0}{\sqrt{\pi} \Delta\omega}\exp\left(-\frac{\omega - \omega_0}{\Delta \omega}\right)^{2} 
\end{gather*}
Di conseguenza la risoluzione dell'integrale di $I(\Delta L)$ è:
\begin{gather*}
    \frac{I_0}{2}\left(1 - \exp\left(-\frac{(\Delta L \Delta \omega)^{2} }{c}\right)\cos\left(\frac{2\Delta L\omega_0}{2}\right)\right)
\end{gather*}
Da questo di può vedere cosa succede quando $\Delta L \to 0$: 
\begin{gather*}
    \frac{I_0}{2} \left(1 - \cos\left(2\frac{\Delta L\omega_0}{2}\right)\right) \ \Longrightarrow \ \frac{I_0}{2}\sin^{2}(k_0\Delta L) 
\end{gather*}
Per $\Delta L$ molto piccoli allora i fasci luminosi oscillano 
normalmente mentre per $\Delta L$ non piccoli invece c'è da considerare anche il termine
esponenziale. Il coefficiente $\frac{I_0}{\sqrt{\pi}\Delta \omega }$ è tale per cui
si ottiene
\begin{gather*}
    \int I(\omega) d\omega = I_0
\end{gather*}
Posso determinare il valore minimo di $\Delta L$ tale per cui
io posso osservare l'oscillazione dell'intensità luminosa. Questa lunghezza
è detta \textbf{lunghezza di coerenza}, e si indica con $L_c$ e si
definisce come l'ampiezza di oscillazione tale per cui
il termine $L_c$ è tale per cui l'esponente della $e$ diventa $1$,
ossia
\begin{gather*}
    \frac{L_c \cdot  \Delta \omega}{c} = 1
\end{gather*}
Si ha quindi
\begin{align}
    L_c = \frac{c}{\Delta \omega} \ \Longrightarrow \ L_c = \frac{c}{\omega_0}\frac{\lambda_0}{\Delta \lambda}
\end{align}
Ho sostituito $\Delta \omega$ in funzione di $\lambda_0$ e inoltre posso esprimere anche
$\omega_0$ in funzione di $\lambda _0$ come $\omega_0 = c\frac{2\pi }{\lambda_0}$. Dunque
\begin{align}
        L_c = \frac{\lambda_0^{2} }{2\pi \cdot  \Delta \lambda}
\end{align}
Si scopre, facendo i conti, che per un led che emette luce bianca nel
visibile, la lunghezza di coerenza è di $400 \ nm$. Se si vuole dunque
vedere l'oscillazione della luce per un led a luce bianca bisogna utilizzare una
lunghezza di coerenza di questa dimensione. 


\section*{L'esperienza dell'ottica geometrica}
\section{Introduzione: obiettivi e finalità}
\begin{wrapfigure}{r}{0.4\textwidth}
    \centering
    \caption{}
    \begin{tikzpicture}
        \draw(0, 0) -- (5, 0);
        \draw[<->](2.5, -1.5) -- (2.5, 1.5);
        \filldraw(1, 0) circle (1pt);
        \draw(1, 0) -- (2.5, 1) -- (4, 0);
        \draw(1, 0) -- (2.5, 0.5) -- (4, 0);
        \draw(1, 0) -- (2.5, -0.5) -- (4, 0);
        \draw(1, 0) -- (2.5, -1) -- (4, 0);
        \draw[|-|](1, 1.5) -- (2.4, 1.5) node[midway, above] {$p$};
        \draw[|-|](2.6, 1.5) -- (4, 1.5) node[midway, above] {$q$};
    \end{tikzpicture}    
\end{wrapfigure}
L'esperienza dell'ottica geometrica è la verifica della legge delle
lenti sottili e di una misura della focale di una lente convergente incognita. 
Ricordando la legge delle lenti sottili
\begin{gather*}
    \frac{1}{p} + \frac{1}{q} = \frac{1}{f}
\end{gather*}
Si eseguono seguenti procedure in laboratorio
\begin{enumerate}
    \item Ogni sperimentatore misura
    $f \pm \Delta f$ ad un valore diverso $p$ (e di conseguenza di $q$);
    \item Se tutte le misure sono consistenti 
    entro le barre di errore, allora si sarà dimostrata la legge delle lenti sottili. 
    \item Determinare la migliore stima di $f$ mediante la media pesata
    del valore di $f$ trovato da ogni sperimentatore.
\end{enumerate}

\section{L'apparato sperimentale}
\begin{wrapfigure}{r}{0.5\textwidth}
    \centering
    \caption{Rappresentazione apparato Lente di servizio e oculare-occhio}
    \begin{tikzpicture}[scale=1.5]
        \draw(0, 0) -- (4,0);
        \draw[thin](0.5 ,1.25) -- (4, 1.25) node[at end, above] {asse ottico};
        \draw(0, 0) -- (0, 1.5);
        \draw(0.5, 0) -- (0.5, 1.5);
        \draw[thick](0.5, 1) -- (0.5, 1.5); 
        \draw[|-|](0.5, 1.7) -- (1.5, 1.7) node[midway, above] {$d$};
        \draw(1.2, 0) rectangle (1.8, 0.4);
        \draw(1.4, 0.4) rectangle (1.6, 1);
        \draw(1.5, 1.25) ellipse (0.1 and 0.25);
        \draw(0.5 , 1.25) -- (1.5, 1.5) -- (2.5, 1.25);
        \draw(0.5, 1.25) -- (1.5, 1) -- (2.5, 1.25);
        \filldraw(2.5, 1.25) circle (1pt);
        \node at (2.5, 1.75) {sorgente virtuale};
        \draw(2.3, 1) rectangle (3, 1.5);
    \end{tikzpicture}    
\end{wrapfigure}
L'apparato sperimentale consiste in un regolo lungo circa
un metro e mezzo con precisione di un millimetro. Una lente
è posta ad una certa distanza con una scala graduata (nonio)
che mi permette di ottenere una precisione di posizionamento
del decimo di millimetro. La sorgente è realizzata mediante un led che 
illumina una diapositiva in modo tale che il led giaccia esattamente
sull'asse ottico rispetto alla lente e parallelo alla guida.
\begin{enumerate}
    \item Per determinare la distanza della diapositiva dalla lente, devo fare in
    modo di creare una immagine virtuale utilizzando la lente convergente con $f = 50 \ mm$
    nella configurazione $2f - 2f$ (dove $f$ è quella della lente conosciuta). 
    La lente che si utilizza prende il nome di \textbf{lente di servizio} per cui:
    \begin{gather*}
        \frac{1}{q} = \frac{1}{2f}
    \end{gather*}
    Per cui $d = 2f$. L'incertezza associata a questa distanza non è fondamentale
    in quanto l'immagine nella sorgente virtuale si formerà (più o meno
    sfuocata) a prescindere dalla precisione. 
\end{enumerate}

\begin{wrapfigure}{r}{0.4\textwidth}
    \centering
    \caption{Schematizzazione del sistema oculare-occhio}
    \begin{tikzpicture}
        \draw(0, 0) -- (5, 0);
        \draw[red, thick](0.5, -1) -- (0.5, 1);
        \draw[thin](0.5, 0) -- (1.5, 1) -- (3, 1) -- (4, 0);
        \draw[thin](0.5, 0) -- (1.5, -1) -- (3, -1) -- (4, 0);
        \draw[<->, thick](1.5, -1) -- (1.5, 1);
        \draw[<->, thick](3, -1) -- (3, 1);
    \end{tikzpicture}    
\end{wrapfigure}
Ad una certa distanza c'è un oculare che mi permette di osservare
l'immagine ottenuta. Sull'oculare c'è un crocefilo che mi permette di osservare
l'allineamento dell'immagine sull'oculare. Il fatto che si utilizza
un oculare con una lente davanti (che è possibile ruotare con una ghiera),
vuol dire che si utilizza l'occhio per osservare l'immagine. 
Si dispone l'oculare in modo tale che il crocefilo sia visibile e anche l'immagine
(questo accade quando il mio occhio li vede nitidi entrambi) e in modo
tale che l'immagine sia sovrapposta esattamente sul crocefilo.
La schematizzazione oculare-occhio comprende il crocefilo con una lente
associata ed il cristallino dell'occhio (che deve essere completamente rilassato
in modo tale da avere il fuoco all'infinito).
Nel disegno il tratto rosso è il crocefilo, le due lenti convergenti sono invece la
lente prima dell'occhio e poi il cristallino. 


\begin{wrapfigure}{r}{0.65\textwidth}
    \centering
    \caption{L'apparato sperimentale totale}  
        \begin{tikzpicture}[scale=1.2]
        \draw(0, 0) -- (7.5,0);
        \draw[thin](0.5 ,1.25) -- (7.5, 1.25) node[at end, above] {asse ottico};
        \draw(0, 0) -- (0, 1.5);
        \draw(0.5, 0) -- (0.5, 1.5);
        \draw[thick](0.5, 1) -- (0.5, 1.5); 
        \draw[|-|](0.5, 1.7) -- (1.5, 1.7) node[midway, above] {$d$};
        \draw(1.2, 0) rectangle (1.8, 0.4);
        \draw(1.4, 0.4) rectangle (1.6, 1);
        \draw(1.5, 1.25) ellipse (0.1 and 0.25);
        \draw(0.5 , 1.25) -- (1.5, 1.5) -- (2.5, 1.25) -- (4.25, 0.9) -- (6, 1.25);
        \draw(0.5, 1.25) -- (1.5, 1) -- (2.5, 1.25) -- (4.25, 1.6) -- (6, 1.25);
        \filldraw(2.5, 1.25) circle (1pt) node[anchor = north, align = center] {sorgente \\ virtuale};
        \draw(3.95, 0) rectangle (4.55, 0.4);
        \draw(4.15, 0.4) rectangle (4.35, 0.9);
        \draw(4.25, 1.25) ellipse (0.15 and 0.35);
        \draw[|-|](2.5, 1.7) -- (4.2, 1.7) node[midway, above] {$p$};
        \draw[|-|](4.3, 1.7) -- (6, 1.7) node[midway, above] {$q$};
        \draw(5.8, 1) rectangle (6.5, 1.5);
    \end{tikzpicture}    
\end{wrapfigure}
Adesso posso spostare l'oculare dal punto dove si forma
l'immagine virtuale e mettere una lente ad una certa distanza
$p$ dall'immagine virtuale ed una certa distanza $q$ dall'oculare nuovo:



\end{document}
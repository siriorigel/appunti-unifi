\documentclass[a4paper, oneside]{article}
\usepackage{graphicx}
\usepackage{amsthm}
\usepackage{amsmath}
\usepackage{amssymb}
\usepackage[a4paper,
            bindingoffset=0.2in,
            left=2cm,
            right=2cm,
            top=2cm,
            bottom=2cm,
            footskip=.25in]{geometry}
\usepackage[italian]{babel}
\usepackage{pgfplots}
\usepackage{tabularx}
\usepackage{tikz}
\usepackage{wrapfig}
\usepackage{color}
\usepackage[d]{esvect}
\definecolor{page}{rgb}{0.129,0.157,0.212}
\pagecolor{page}
\color{white}
\graphicspath{ {./images/} }
\usetikzlibrary{shapes.geometric}
\usetikzlibrary{datavisualization}
\usetikzlibrary{datavisualization.formats.functions}
\usetikzlibrary{patterns}
\pgfplotsset{width=10cm,compat=1.18}

\title{Appunti ottica}
\author{Tommaso Miliani}
\date{27-10-25}

\begin{document}
\newtheoremstyle{theoremEnv}
                {}          % Space above
                {}          % Space below
                {\slshape}  % Body font
                {}          % Indent amount
                {\bfseries} % Head font
                {.}         % Punctuation after head
                {\newline}         % Space after theorem head
                {}          % Theorem head spec
\theoremstyle{theoremEnv}

\newtheorem{definition}{Definizione}[section]
\newtheorem{theorem}{Teorema}[section]
\newtheorem{lemma}{Proposizione}[section]
\newtheorem{observation}{Osservazione}[section]
\newtheorem{corollary}{Corollario}[theorem]
\newtheorem{example}{Esempio}[section]

\maketitle

\section*{Esperienza della polarizzazione}
\section{Scopi e finalità}
\subsection{Il fit sinusoidale}
Nella polarizzazione si deve validare le leggi della polarizzazione
per le lenti $\frac{\lambda}{4}$ e per le lenti $\frac{\lambda}{2}$. 
Si deve realizzare un fit (non lineare) di una funzione sinusoidale:
per realizzarlo dobbiamo fare le seguenti assunzioni (così come per i fit lineari):
\begin{itemize}
    \item l'incertezza $\sigma$ sulla variabile aleatoria $x$ sia trascurabile: questo vuol dire che
    si ha uno strumento che misura ole $x$ con una precisione di diversi ordini di grandezza superiore delle $y$. 
    \item La funzione $g$, ossia la funzione del fit, abbia una distribuzione
    gaussiana e con i vari $\sigma_{y_i}$ costanti per tutte
    le misure $ \ \Longrightarrow \ \sigma_{y_i} \approx \sigma_y$. 
    \item Devo minimizzare il $\chi^{2}$.  
\end{itemize} 

\begin{wrapfigure}{r}{0.4\textwidth}
    \centering
    \caption{La funzione generica}
    \begin{tikzpicture}
        \draw[->](0, 0) -- (4, 0);
        \draw[->](0, 0) -- (0, 4);
        \draw(0.5, 1) ..  controls (1, 1.9) and (1.2, 1.9) .. (1.7, 1.3);
        \draw(1.7, 1.3) .. controls (2.2, 0.7) and (2.7, 1.8) .. (3.5, 1);
        \draw(1, 0.1) -- (1, -0.1) node[at end, below] {$x_1$};
        \draw(2, 0.1) -- (2, -0.1) node[at end, below] {$x_2$};
        \draw(3, 0.1) -- (3, -0.1) node[at end, below] {$x_3$};
    \end{tikzpicture}    
\end{wrapfigure}
Nel nostro caso non posso applicare le regole del fit lineare ma posso supporre di avere una
funzione generica ottenuta con dei certi valori sperimentali
$(x_i, y_i)$ dove i dati sono legati tra di loro mediante una determinata legge
\begin{gather*}
    f(a, b, c, \dots)
\end{gather*}
Tuttavia questa funzione non è completamente arbitraria ma è del tipo
\begin{gather*}
    f = a\cos^{2}(bx + \phi) + \text{offset} 
\end{gather*}
Questa funzione sarà ovviamente una funzione dell'angolo $x$ (ossia la rotazione
della lamina) e descrive l'intensità luminosa del cubo in analisi
in trasmissione e riflessione. Dunque posso determinare il $\chi^{2}$ come
\begin{gather*}
    \chi^{2} = \frac{\sum (y_i - f(a, b, \phi, \text{offset}, x))^{2} }{\sigma_y^{2} } 
\end{gather*}  
Posso ottenere la minimizzazione del $\chi^{2}$ mediante l'utilizzo di un calcolatore 
e delle $\overline{a}, \overline{b}, \overline{c}, \overline{d}$. Inoltre è capace di ottenere
anche le incertezze associate alle singole variabili. 

\subsection{Le lamine $\frac{\lambda}{4}$ e $\frac{\lambda}{2}$ nell'esperienza}
\begin{wrapfigure}{r}{0.4\textwidth}
    \centering
    \caption{La lamina $\frac{\lambda}{2}$}
    \begin{tikzpicture}
        \draw[red, thick, ->](0, 0) -- (2, 0) node[at end, right] {asse fast};
        \draw[cyan, thick, ->](0, 0) -- (0, 2) node[at end, below] {asse slow};
        \draw[thick, ->](0, 0) -- (2, 1) node[at end, right] {in};
        \draw[thick, ->](0, 0) -- (-2, 1) node[at end, above] {out};
        \draw(0, 0.5) arc (90:30:0.5) node[midway, above] {$\theta$};
        \draw(0, 0.5) arc (90: 150:0.5) node[midway, above] {$\theta$};
    \end{tikzpicture}    
\end{wrapfigure}
Le lamine possono riflettere il campo elettrico lungo le due direzioni
\begin{gather*}
    E_{z} = E_{z0}\cos(kx - \omega t + \phi_z) \\
    E_{y} = E_{y0}\cos(kx - \omega t + \phi_y)
\end{gather*}
Con la lamina $\frac{\lambda}{4}$ posso cambiare la spanciatura dell'ellisse che 
descrive la polarizzazione ellittica generica senza modificare la direzione degli assi:
per quello si utilizza la $\frac{\lambda}{2}$. 

\section{Apparato sperimentale}
Si ha un tavolo con una breadboard: ossia un piano di 
alluminio con dei fori che permette di fissare gli 
strumenti ottici. La polarizzazione in uscita dal laser non la
conosciamo e potrebbe aver qualsiasi tipo di polarizzazione 
ma si conosce la lunghezza d'onda che è di $532 \ nm$. 
\begin{gather*}
        \begin{tikzpicture}
        \draw[->](-1, 0) -- (10, 0) node[at end, below] {$x$};
        \draw(0, -0.25) rectangle (1, 0.25);
        \draw(-1, 0) circle (0.2);
        \filldraw(-1, 0) circle (1pt) node[anchor = south] {$y$}; 
        \draw[->](-1, 0) -- (-1, -2) node[at end, right] {$z$};
        \draw(2, 0) circle (0.2);
        \draw[->](2, 0) -- (2, 1) node[at end, right] {$\vv{E_{\parallel}}$};
        \draw(1.9, 0.1) -- (2.1, -0.1) node[at end, below] {$\vv{E_\perp}$ };
        \draw(1.9, -0.1) -- (2.1, 0.1);
        \draw(3, -0.75) rectangle (3.3, 0.75);
        \draw (4, -0.75) rectangle  (5.5, 0.75);
        \draw[->](6, 0) -- (6, -1) node[at end, below] {$\vv{E_\parallel}$ };
        \draw(6.5, -0.75) rectangle (6.8, 0.75); 
        \draw[->](7.5, 0) -- (7.5, -1) node[at end, right] {$\vv{E_{\parallel}}$};
        \draw(7.4, 0.1) -- (7.6, -0.1) node[at start, above] {$\vv{E_\perp}$ };
        \draw(7.4, -0.1) -- (7.6, 0.1);
        \draw(7.5, 0) circle (0.2);
        \draw(8, -0.75) rectangle (9.5, 0.75);
        \draw(8.75, 0) -- (8.75, -2);
        \draw(10, -0.5) rectangle (11, 0.5);
    \end{tikzpicture}  
\end{gather*}
SI fa passare il campo 
attraverso un filtro variabile che permette di cambiare (variando il
suo angolo di rotazione) l'intensità luminosa che passa. La luce passa poi
attraverso un cubo di pulizia, che permette di filtrare il campo perpendicolare 
e di far passare solo il campo $\vv{E_\parallel}$ che risulta avere
verso contrario a quello di entrata.  \\
Si utilizza un supporto con goniometro che permette di ruotare la lamina; dato che
il laser passa dentro la lamina, posso fare in modo che l'asse slow e quello fast
possano ruotare rispetto alla scala goniometrica. Questo componente ottico sta dopo il cubo
di pulizia e permette di cambiare la polarizzazione in base all'angolo di rotazione $\theta$.
Dopo si pone un altro cubo di pulizia che permette di filtrare nuovamente  il
campo perpendicolare (che prende il nome di cubo di analisi) e porre i due rilevatori. 

\section{Funzioni del fit}
\subsection{Calibrazione dei laser}
Si presentano vari problemi nella raccolta dei dati sperimentali:
\begin{itemize}
    \item Il rilevatore non è perfetto e dunque non riesce a misurare
    perfettamente tutta l'intensità;
    \item Il rilevatore ha una sensibilità diversa da quella dell'altro 
    rilevatore
    \item I rilevatori sono sporchi oppure tarati male
    \item L'allineamento dell'apparato ottico non è perfetto 
\end{itemize}
E' possibile ovviare alla calibrazione dei rilevatori sfruttando la 
lamina $\frac{\lambda}{2}$: dopo aver posizionato i rilevatori si ruota la
lamina $\frac{\lambda}{2}$ e si misura il valore massimo che si ottiene in ognuno
dei rilevatori. 

\subsection{Lamina $\frac{\lambda}{2}$}
Si può ora descrivere le leggi di Malus per i vari assi
\begin{gather*}
    I_\parallel(\theta) = I_0 \cos^{2}(2\theta) \\
    I_\perp(\theta) = I_0 \sin^{2}(2\theta)  
\end{gather*}
SI ottiene allora le seguenti condizioni
\begin{gather*}
    \begin{tikzpicture}
        \draw[red, ->](0, 0) -- (2, 0) node[at end, below] {$E$};
        \draw[cyan,->](0, 0) -- (0, 2);
        \node at (1.5, 1.5) {$\theta = 0$};
        \draw(5, 0) -- (7, 0) node[at end, below] {$E$};
        \draw[cyan, ->](5, 0) -- (5, 2);
        \draw(5, 0) -- (7, 2);
        \draw(5, 0) -- (4, 2); 
        \node at (7, 1.5) {$\theta = \frac{\pi}{4}$};
    \end{tikzpicture}
\end{gather*}
Facendo ora ruotare le polarizzazioni, si ottengono tanti dati sperimentali:
in funzione dell'angolo si può allora disegnare il grafico della legge di Malus
(circa 20 punti per gli angoli da zero a $360$° equispaziati, ossia ogni $\sim 20$°)
Mentre si collezionano i dati per l'intensità trasmessa, si collezionano anche i dati
per l'intensità riflessa e si disegnerà anche il grafico attraverso quei punti.
La funzione del fit sarà ora data da
\begin{gather*}
    f(a, b, \phi, \text{offset}, x) = a\cos^{2}(bx + \phi) + \text{offset} 
\end{gather*}
Supponendo ora di avere questa funzione da farne il fit, dividendo $I_\parallel(\theta)$
e $I_\perp(\theta)$ per i loro valori massimi ci si aspetta che l'ampiezza
dell'oscillazione del coseno e del seno sia $\overline{a} \approx 1$. Posso invece dire che $\overline{b} \approx 2$ in quanto 
nella legge di Malus deve risultare $2\theta$ e dove $\text{offset} = 0$. Tra la fase del coseno e del seno ci si
aspetta una differenza di circa $\approx \frac{\pi}{2}$, e dunque, arbitrariamente, posso dire che $\overline{\phi} \approx 0$.  
Riassumendo
\begin{gather*}
    \overline{a} \approx 1 \qquad \overline{b} \approx 2 \qquad \overline{\phi} \approx 0 \qquad \text{offset} \approx 0  
\end{gather*}
Tuttavia in condizioni non ideali l'offset del laser non sarà zero ma 
sarà un valore molto piccolo da dover sottrarre a tutti i valori
trovati in laboratorio. Si può ora vedere che
\begin{gather*}
    \cos^{2}(2\theta) = 1 - \sin^{2}(2\theta) = 1 - \cos^{2} (2\theta + \frac{pi}{2})  
\end{gather*}
Data questa legge, si osserva che si trova un altro set di valori con 
\begin{gather*}
    \overline{a} \approx -1 \qquad \overline{b} \approx 2 \qquad \overline{\phi} \approx = \frac{\pi}{2}  \qquad \text{offset} = 1 
\end{gather*}
In questo caso si vede che $\chi^{2}$ ha due minimi e permette di ottenere due
funzioni dal programma di fit.  

\subsection{La lamina $\frac{\lambda}{4}$}
Con la lamina $\frac{\lambda}{4}$ si per la legge di Malus che
\begin{gather*}
    I_\parallel = I_0\left(1 - \frac{1}{2}\sin^{2}(2\theta) \right) \\
    I_\perp =   \frac{I_0}{2}\left(\sin^{2}(2\theta) \right)
\end{gather*}
Dato che questa lamina mi permette di spanciare l'ellisse, se ruotassi la lamina
di un certo angolo $\theta = \frac{\pi}{2}$, il ritardo di fase è lungo una direzione dove il campo elettrico non
oscilla neanche, allora la polarizzazione rimane tutta $\vv{E_\parallel}$ e dunque
tutta la potenza è trasmessa.
\begin{gather*}
    \begin{tikzpicture}
        \draw[red, ->](0, 0) -- (2, 0) node[at end, below] {E};
        \draw[cyan,->](0, 0) -- (0, 2);
        \node at (1.5, 1.5) {$\theta = 0$};
        \draw[->](4, 0) -- (6, 0);
    \end{tikzpicture}
\end{gather*} 

\begin{wrapfigure}{r}{0.4\textwidth}
    \centering
    \caption{}
    \begin{tikzpicture}[domain=0:4]
        \draw[->](0, 0) -- (4, 0) node[right] {$\theta$};
        \draw[->](0, 0) -- (0, 2) node[right] {$I_0$};
        \draw[samples = 100, red] plot (\x, {sin(2 *\x r) * sin(2 *\x r)}) node[left] {$I_\perp$};
        \draw[samples = 100, cyan] plot (\x, {2 - sin(2 * \x r) * sin(2 * \x r)}) node[left] {$I_\parallel$};
    \end{tikzpicture}    
\end{wrapfigure}
L'ellisse si spancia solamente quando $\theta = \frac{\pi}{4}$. Allora, la polarizzazione
in uscita diventa circolare.
\begin{gather*}
    \theta = \frac{\pi}{4} \qquad I_\parallel = \frac{I_0}{2} \qquad I_\perp = \frac{I_0}{2}
\end{gather*}
Dato che la lamina $\frac{\lambda}{2}$ permette di ruotare il campo di un certo angolo
$\theta$ e, quando la luce passa nuovamente  all'interno della lamina viene riportata alla polarizzazione 
iniziale, tutta la luce che torna sul cubo analisi sarà totalmente riflessa. Invece, una $\frac{\lambda}{4}$
attraversata due volte si comporta come una $\frac{\lambda}{2}$: l'effetto complessivo 
del doppio passaggio è una riflessione rispetto all'asse slow della lamina. 

\end{document}
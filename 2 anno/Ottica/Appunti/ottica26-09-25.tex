\documentclass[a4paper, oneside]{article}
\usepackage{graphicx}
\usepackage{amsthm}
\usepackage{amsmath}
\usepackage{amssymb}
\usepackage[a4paper,
            bindingoffset=0.2in,
            left=2cm,
            right=2cm,
            top=2cm,
            bottom=2cm,
            footskip=.25in]{geometry}
\usepackage[italian]{babel}
\usepackage{pgfplots}
\usepackage{tabularx}
\usepackage{tikz}
\usepackage{wrapfig}
\usepackage{color}
\usepackage[d]{esvect}
\definecolor{page}{rgb}{0.129,0.157,0.212}
\pagecolor{page}
\color{white}
\graphicspath{ {./images/} }
\usetikzlibrary{shapes.geometric}
\usetikzlibrary{datavisualization}
\usetikzlibrary{datavisualization.formats.functions}
\usetikzlibrary{patterns}
\pgfplotsset{width=10cm,compat=1.9}

\title{Appunti di Ottica}
\author{Tommaso Miliani}
\date{26-09-25}

\begin{document}
\newtheoremstyle{theoremEnv}
                {}          % Space above
                {}          % Space below
                {\slshape}  % Body font
                {}          % Indent amount
                {\bfseries} % Head font
                {.}         % Punctuation after head
                {\newline}         % Space after theorem head
                {}          % Theorem head spec
\theoremstyle{theoremEnv}

\newtheorem{definition}{Definizione}[section]
\newtheorem{theorem}{Teorema}[section]
\newtheorem{lemma}{Proposizione}[section]
\newtheorem{observation}{Osservazione}[section]
\newtheorem{corollary}{Corollario}[theorem]
\newtheorem{example}{Esempio}[section]

\maketitle

\section{Funzionamento dell'occhio umano}
\begin{wrapfigure}{r}{0.4\textwidth}
    \centering
    \caption{}
    \begin{tikzpicture}
        \draw(0, 0) -- (4, 0);
        \draw[<->](2, -1) -- (2, 1);
        \draw(3, -1) -- (3, 1) node[near end, right] {Piano immagine};
        \draw[|-|](2.1, -1) -- (2.9, -1) node[midway, below] {$q$};
    \end{tikzpicture}    
\end{wrapfigure}
L'occhio ha una forma sferica con un indice di rifrazione di 
circa $n = 1.4$. Sulla parte posteriore abbiamo la retina, ossia
il rilevatore dell'occhio in grado di raccogliere la luce in entrata dal cristallino
e poi trasferita al nostro cervello tramite il nervo ottico. Sulla
retina si forma l'immagine che noi vediamo e dunque la retina è il piano immagine. Poiché
non si riesce a cambiare la distanza $q$ tra il cristallino e la retina,
io posso mettere a fuoco piani sorgenti diversi utilizzando dei muscoli
che contraggono il cristallino o lo rilassano e contribuisce per
solo un terzo del potere convergente dell'occhio. Gli altri due terzi
del potere convergente dell'occhio derivano dalla cornea posta
davanti al cristallino ed immersa nell'humor acqueo mentre l'occhio 
è immerso nell'humor vitreo (ognuno con i suoi indici di rifrazione).
La distanza $q = 0.02 m$. E' l'analogo del sistema ottico in figura.

\begin{wrapfigure}{r}{0.4\textwidth}
    \centering
    \caption{}
    \begin{tikzpicture}
        \draw(0, 0) -- (6, 0);
        \draw(3, -1) -- (3, 1);
        \draw(4, -1) -- (4, 1);
        \draw(0, 1) -- (3, 0) -- (4, -0.33);
        \draw(2, 0.33) arc(160:180:1) node[midway, left] {$\theta$};
        \draw(0, 0) -- (0, 1) node[midway, right] {$h$};
        \draw[cyan](4, 0) -- (4, -0.33) node[midway, right] {$h'$};
    \end{tikzpicture}    
\end{wrapfigure}
La distanza minima di messa a fuoco aumenta con l'età in quanto
diminuisce la capacità di strizzare il cristallino da parte dei muscoli del cristallino: la
distanza minima è di $p_{min} \approx 0.3m$. Un occhio sano riesce inoltre a mettere a fuoco
fino ad una distanza $p_{max} = +\infty $. La distanza $q$ dell'occhio è $0.02m$. 
La distanza focale dell'occhio è di $0.02m$, voglio vedere cosa succede
alla focale dell'occhio quando si vuole osservare un oggetto alla distanza minima:
\begin{gather*}
    \frac{1}{f} = \frac{q + p}{pq} \ \Longrightarrow \  f = \frac{pq}{p + q} \ \Longrightarrow \ f \approx 19.2 mm
\end{gather*}
Posso allora ricavare l'ingrandimento dell'occhio: più gli oggetti sono vicini e più sono grandi all'interno del
nostro occhio.
\begin{gather*}
    I = \frac{h'}{h} = -\frac{p}{q}
\end{gather*}

\section{Lente di ingrandimento}
\begin{wrapfigure}{r}{0.5\textwidth}
    \centering
    \caption{Lente di ingrandimento}
    \begin{tikzpicture}
        \draw(-3, 0) -- (4, 0);
        \draw[<->](0, -2) -- (0, 2);
        \filldraw (-1.5, 0) circle (1pt) node[anchor = north] {$F'$};
        \filldraw (1.5, 0) circle (1pt) node[anchor = north] {$F$};
        \draw[<->](2.5,-1) -- (2.5, 1);
        \draw(3.5, -1) -- (3.5, 1);
        \draw(-1, 0) -- (-1, 1) node[midway, left] {$h$};
        \draw[red](-1, 1) -- (2.5, -2.5);
        \draw[red](-1, 1) -- (0,1) -- (1.5, 0) -- (2.5, -0.75);
        \draw[red](-1, 1) -- (-2.5, 2.5);
        \draw[red](-2.5, 2.5) -- (0, 1);
        \draw[red](-2.5, 0) -- (-2.5, 2.5) node[midway, left] {$h'$};
        \draw[green](-2.5, 2.5) -- (3.5, -0.5);
        \draw[green](-1, 1) -- (0, 1.25);
        \draw[red](0.75, 0) arc (180:145:0.75) node[midway, left] {$\theta$};
        \draw[green](1.5, 0.5) arc (140:180:0.75) node[at start, above] {$\theta'$};
    \end{tikzpicture}    
\end{wrapfigure}
Per poter allora ingrandire gli oggetti, si utilizzano delle lenti in grado di avere
un potere di ingrandimento tale per cui si possa vedere meglio oggetti molto piccoli.
Ponendo un oggetto prima della focale della lente, si osserva che i raggi luminosi dell'oggetto
non si incontrano; l'oggetto apparirà quindi più grande
quando arriva all'occhio umano ad una certa distanza $d$. Data  la legge delle lenti sottili,
con $p < f$ allora sappiamo che 
\begin{gather*}
    \frac{1}{q} = \frac{p - f}{fp} < 0
\end{gather*}
I raggi luminosi per l'occhio, appaiono come se si formino prima dell'oggetto
stesso: l'occhio umano osserva allora la sorgente virtuale dell'immagine. Posso allora
studiare l'angolo con cui l'immagine virtuale $h'$ si intersechi 
con l'occhio umano anche se in realtà il raggio verde è generato dall'oggetto $h$.
La distanza dove si forma l'immagine virtuale adesso è proprio $q$; voglio ora trovare l'angolo
verde $\theta'$; sappiamo che, in assenza della lente, l'oggetto con altezza $h$
formerebbe un angolo
\begin{gather*}
    \tan \theta = \frac{h}{p + d}
\end{gather*}
Invece, l'angolo $\theta'$ che si forma sarà dato da:
\begin{align}
    \tan\theta' = \frac{h'}{q + d} \qquad q < 0
\end{align}
Possiamo ora determinare quale sarà l'angolo più grande. Dall'espressione
dell'ingrandimento possiamo ricavare $h'$ in funzione della lente e di $h$:
\begin{gather*}
    \frac{h'}{h} = -\frac{q}{p} = - \frac{pf}{p - f}\frac{1}{p} \ \Longrightarrow \ h' = \frac{f}{f  - p}h
\end{gather*}
Allora posso combinare le due espressioni della tangente per vedere quale è l'angolo
più grande
\begin{gather*}
    \tan\theta' = \frac{f}{f - p}\frac{h}{q + d}
\end{gather*}
Posso allora
riscrivere $q$ attraverso le relazioni con $f$ e $p$ (ho invertito $p- f$ con
$f - p$ a causa del segno di $q$):
\begin{gather*}
    q = \frac{pf}{f - p} \ \Longrightarrow \ \tan\theta' = \frac{f}{f - p}\frac{h}{\frac{pf + df -dp}{f - p}} \\
\end{gather*}
Allora posso confrontare le due espressioni direttamente:
\begin{gather*}
    \tan\theta = \frac{h}{p + d} \ \Longrightarrow \ \tan\theta' = \frac{h}{p + d - \left(\frac{dp}{f}\right)}
\end{gather*}
Si ottiene allora che $\theta' > \theta$ proprio perché il
denominatore è più piccolo e allora l'immagine si ingrandisce.

\section{Sistema di lenti}
\begin{wrapfigure}{r}{0.4\textwidth}
    \centering
    \caption{}
    \begin{tikzpicture}
        \draw(-3, 0) -- (3, 0);
        \draw[<->](0, -1.5) -- (0, 1.5) node[at start, below] {$f_1$};
        \draw[<->](1, -1.5) -- (1, 1.5) node[at start, below] {$f_2$};
        \draw[|-|](-2, -0.2) -- (0, -0.2) node[midway, below] {$p$};
        \draw[|-|](0.1, 0.1) -- (0.9, 0.1) node[midway, above] {$d_{1,2}$};
        \draw[|-|](1.1, -0.2) -- (1.9, -0.2) node[midway, below] {$q'$};
    \end{tikzpicture}    
\end{wrapfigure}
Si vuole dimostrare il comportamento di due lenti convergente:
l'ipotesi è che aumenti il potere convergente e che quindi il sistema di lenti
si possa comportare come una unica lente con un potere
convergente maggiore per cui la focale totale sarà la somma della
focale. Tuttavia questa cosa è sbagliata in quanto
il potere convergente è maggiore se la focale della lente è piccola:
allora il potere convergente mi aspetto che sia direttamente proporzionale all'inverso
della lente
\begin{gather*}
    \frac{1}{f} \approx \frac{1}{f_1} + \frac{1}{f_2}
\end{gather*}
Questo è dimostrabile a partire dell'applicazione della legge delle lenti
sottili a tutto il sistema di lenti (così come si era fatto per l'occhio) e
quindi vale per qualsiasi distanza $d_{1, 2}$ tra le due lenti. 
\begin{gather*}
    d_{1, 2} = q + p' 
\end{gather*}
E quindi si ottiene 
\begin{align}
    \frac{1}{f} = \frac{1}{f_1} + \frac{1}{f_2}
\end{align}
Esperimento a lezione con 4 lenti:
\begin{align*}
    f_1 &= 50 mm \\
    f_2 &= -100 mm \\
    f_3 &= 300 mm \\
    f_4 &= -500 mm
\end{align*}



\end{document}
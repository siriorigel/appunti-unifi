\documentclass[a4paper, oneside]{article}
\usepackage{graphicx}
\usepackage{amsthm}
\usepackage{amsmath}
\usepackage{amssymb}
\usepackage[a4paper,
            bindingoffset=0.2in,
            left=2cm,
            right=2cm,
            top=2cm,
            bottom=2cm,
            footskip=.25in]{geometry}
\usepackage[italian]{babel}
\usepackage{pgfplots}
\usepackage{tabularx}
\usepackage{tikz}
\usepackage{wrapfig}
\usepackage{color}
\usepackage[d]{esvect}
\definecolor{page}{rgb}{0.129,0.157,0.212}
\pagecolor{page}
\color{white}
\graphicspath{ {./images/} }
\usetikzlibrary{shapes.geometric}
\usetikzlibrary{datavisualization}
\usetikzlibrary{datavisualization.formats.functions}
\usetikzlibrary{patterns}
\pgfplotsset{width=10cm,compat=1.9}

\title{Appunti di ottica}
\author{Tommaso Miliani}
\date{03-10-25}

\begin{document}
\newtheoremstyle{theoremEnv}
                {}          % Space above
                {}          % Space below
                {\slshape}  % Body font
                {}          % Indent amount
                {\bfseries} % Head font
                {.}         % Punctuation after head
                {\newline}         % Space after theorem head
                {}          % Theorem head spec
\theoremstyle{theoremEnv}

\newtheorem{definition}{Definizione}[section]
\newtheorem{theorem}{Teorema}[section]
\newtheorem{lemma}{Proposizione}[section]
\newtheorem{observation}{Osservazione}[section]
\newtheorem{corollary}{Corollario}[theorem]
\newtheorem{example}{Esempio}[section]

\maketitle

\section{Funzionamento di un cubo polarizzatore}
\begin{wrapfigure}{r}{0.4\textwidth}
    \centering
    \caption{Schematizzazione del campo elettrico}
    \begin{tikzpicture}
        \draw[->](0, 0) -- (2, 0) node[at end, below] {$y$};
        \draw[->](0, 0) -- (0, 2) node[at end, left] {$z$};
        \draw[->, thick](0, 0) -- (1, 1) node[at end, right] {$\vv{E}$ };
        \draw[->, thick](0, 0) -- (1, 0) node[at end, below] {$\vv{E_y}$};
        \draw[->, thick](0, 0) -- (0, 1) node[at end, left] {$\vv{E_z}$ };
    \end{tikzpicture}    
\end{wrapfigure}
Alcuni materiali dielettrici sono detti \textbf{birifrangenti}: ossia
hanno due indici di rifrazione per due polarizzazione lineari ortogonali
della luce che li attraversa. Una applicazione è il \textbf{cubo polarizzatore},
il cui scopo è proprio quello di separare due polimerizzazioni ortogonali di un
fascio incidente sul cubo con polarizzazione  generica.
Si può ora analizzare il funzionamento del cubo polarizzatore attraverso il
secondo modello: tra le due facce del cubo è inserito un film sottile birifrangente
per cui l'indice di rifrazione per gli assi ortogonali è diverso. Si avrà allora 
che gli indici di rifrazione $n_{\parallel} \neq n_{\perp}$. Voglio ora che
l'angolo di incidenza a $45$ gradi sia l'angolo di rifrazione totale 
per $\vv{E_{\perp}}$.


\begin{wrapfigure}{r}{0.4\textwidth}
    \centering
    \caption{Il cubo polarizzatore}
    \begin{tikzpicture}
        \draw(0, 0) -- (2, 0);
        \draw[->, thick](0.5, 0) -- (1.5, 0) node[at end, above] {$\vv{E_{\perp}}$};
        \draw[->, thick](0.5, 0) -- (0.5, 1) node[at end, left] {$\vv{E_{\parallel}}$ };
        \filldraw(0.5, 0) circle (1pt);
        \draw(0.5, 0) circle (0.15);
        \draw(2, 1) -- (4, -1) -- (2, -1) -- (2, 1); 
        \draw(2.2, 1) -- (4.2, -1) -- (4.2, 1) -- (2.2, 1);
        \filldraw[pattern = north east lines, pattern color = white](2.2, 1) -- (4.2, -1) -- (4, -1) -- (2, 1) -- (2.2, 1);
        \draw[dashed](2, 0) -- (3, 0) -- (2, -1);
    \end{tikzpicture}    
\end{wrapfigure}
Posso elencare le ipotesi:
\begin{itemize}
    \item Scelgo $n_{\perp}$ in modo tale che $\frac{\pi}{4}$ sia l'angolo di 
    riflessione totale interna per luce con polarizzazione  $\vv{E_{\perp}}$.
    \item Scelgo $n_{\parallel}$ in modo tale che l'angolo $\frac{\pi}{4}$ sia l'angolo di Proust
    per luce con polarizzazione $\vv{E_{\parallel}}$.
\end{itemize}
L'angolo di rifrazione è dunque molto grande per questo coincide
con l'interfaccia, allora si ha riflessione totale interna secondo la 
legge di snell:
\begin{gather*}
    n_{vetro} \sin\frac{\pi}{4} = n_{\perp} \sin\frac{\pi}{2}
\end{gather*}
L'indice di rifrazione $n_{\perp}$ deve allora soddisfare questa condizione
per poter essere un angolo di riflessione interna totale
per la luce con polarizzazione $\vv{E}_{\perp}$:
\begin{gather*}
    n_{\perp} = n_{vetro}\frac{\sqrt{2} }{2}
\end{gather*} 
Mentre $n_{\parallel}$ del film birifrangente è in relazione con l'indice
di rifrazione del vetro secondo la seguente:
\begin{gather*}
    \tan\theta = \frac{n_{\parallel}}{n_{vetro}} \ \Longrightarrow \ n_{\parallel} = n_{vetro}
\end{gather*}
Dato che l'angolo è per costruzione è $\frac{\pi}{4}$ allora la relazione a destra vale.

\section{Polarizzatore a fili metallici}
\begin{wrapfigure}{r}{0.4\textwidth}
    \centering
    \caption{Polarizzatore a fili metallici}
    \begin{tikzpicture}
        \draw(0, 0) rectangle (2, 2);
        \draw(0.1, 0) -- (0.1, 2);
        \draw(0.2, 0) -- (0.2, 2);
        \draw(0.3, 0) -- (0.3, 2);
        \draw(0.4, 0) -- (0.4, 2);
        \draw(0.5, 0) -- (0.5, 2);
        \draw(0.6, 0) -- (0.6, 2);
        \draw(0.7, 0) -- (0.7, 2);
        \draw(0.8, 0) -- (0.8, 2);
        \draw(0.9, 0) -- (0.9, 2);
        \draw(1.0, 0) -- (1.0, 2);
        \draw(1.1, 0) -- (1.1, 2);
        \draw(1.2, 0) -- (1.2, 2);
        \draw(1.3, 0) -- (1.3, 2);
        \draw(1.4, 0) -- (1.4, 2);
        \draw(1.5, 0) -- (1.5, 2);
        \draw(1.6, 0) -- (1.6, 2);
        \draw(1.7, 0) -- (1.7, 2);
        \draw(1.8, 0) -- (1.8, 2);
        \draw(1.9, 0) -- (1.9, 2);
        \draw(2.0, 0) -- (2, 2);
        \draw[->, thick](-0.5, 0.5) -- (-0.5, 1.5) node[at end, left] {$\vv{E_{\parallel}}$ };
        \draw[->, thick](0.5, -0.5) -- (1.5, -0.5) node[at end, below] {$\vv{E_{\perp}}$ };
    \end{tikzpicture}    
\end{wrapfigure}
Un altro strumento per poter ottenere una sola polarizzazione da un 
fascio di luce lineare è detto polarizzatore a \textbf{fili metallici}.
Il polarizzatore a fili metallici è una lastra di materiale sulla quale ci si
può depositare dei fili metallici sottili  lungo una direzione ben definita
(in genere ognuno con un  diametro di $10\sim100 \ nm$). Supponendo di avere
una luce incidente al filo metallico con componente sia parallela che
perpendicolare al fascio di fili e di farla passare attraverso il polarizzatore.
La luce che passa dentro questo polarizzatore è quella con $\vv{E_{\perp}}$ poiché il campo parallelo
mette in moto gli elettroni nei fili sottili e, dato che sono più alti che larghi,
allora c'è più movimento degli elettroni che creano un campo
elettrico con interferenza distruttiva per la componente parallela. Le onde
perpendicolari, invece, passano in quanto il diametro dei microfili è
molto piccolo e dunque non c'è abbastanza spazio per poter mettere
in movimento gli elettroni per poter creare interferenza distruttiva.

\section{Lamine di ritardo}
\begin{wrapfigure}{r}{0.4\textwidth}
    \centering
    \caption{Ritardo di fase}
    \begin{tikzpicture}
        \draw[->](0, 0) --(2, 0) node[at end, below] {$x$};
        \draw[->](0, 0) -- ( 0, 2) node[at end, left] {$y$};
        \filldraw(0, 0) circle (1pt) node[anchor = north] {$O$}; 
        \filldraw(1.5, 0) circle (1pt) node[anchor = north] {$O'$};
        \draw[->](1.5, 0)-- (3.5, 0) node[at end, below] {$x'$};
        \draw[->](1.5, 0) -- (1.5, 2) node[at end, left] {$y$};
        \draw[|-|](0.1, 0.2) -- (1.4, 0.2) node[midway, above] {$d$};
        \draw[|-|](1.6, 0.2) -- (2.5, 0.2) node[midway, above] {$x'$};
        \draw[|-|](0, -0.7) -- (2.5, -0.7) node[midway, below] {$x$};
    \end{tikzpicture}    
\end{wrapfigure}
Le \textbf{lamine di ritardo} si basano su materiali birifrangenti e
servono a modificare la polarizzazione della luce.
Supponiamo di scrivere un campo elettrico lungo una sola
direzione con 
\begin{gather*}
    \vv{E_{\parallel}} = E_{0z}\cos(kx - \omega t)\hat{z} 
\end{gather*}  
Potrei chiedermi come l'onda, rispetto all'origine $O$,
cambia rispetto al tempo il campo elettrico? Quindi posso 
considerare un altro sistema di riferimento $O'$ descritto con
assi paralleli a quelli del sistema di riferimento $O$. Concentrandosi
ora sul sistema di riferimento $O'$, dopo un certo istante di
tempo l'onda avrà camminato per una distanza $x$ rispetto a $O$
e per una distanza $x'$ rispetto a $O'$. Posso quindi dire che il
campo elettrico è
\begin{gather*}
    \vv{E}(x', t) = E_{0z}\cos(k(d + x') - \omega t)\hat{z}  
\end{gather*}
Per l'onda, dopo essersi propagata per una distanza $d$, posso 
considerare un sistema di riferimento $O'$ attraverso un termine di fase
$kd$ che prende il nome di \textbf{ritardo di fase} che è dovuto
alla propagazione lungo il tratto di lunghezza $d$ che,
nel caso si tratti di un mezzo dielettrico con indice di
rifrazione $n$, questo termine diventa $knd$. 
Si considera ora una lamina di spessore $d$ con materiale
birifrangente ed un campo elettrico incidente generico dato 
dalla seguente:
\begin{gather*}
    \vv{E_{in}} = E_{0y}\cos(kx - \omega t)\hat{y} + E_{0z}\cos(kx - \omega t)\hat{z}   
\end{gather*}
In questo caso le ampiezze possono essere anche diverse ma le fasi
sono le stesse e dunque sono lineari. Adesso mi chiedo che cosa accade al campo
elettrico dopo che ha attraversato la lamina in funzione di un sistema
di riferimento $O'$ posto dopo la lamina. Se il materiale è birifrangente, allora
dovrà necessariamente cambiare $n$ a seconda della direzione di oscillazione. Il
campo elettrico di uscita dalla lamina allora sarà:
\begin{gather*}
    \vv{E_{out}}(x', t) = E_{0y}\cos(kx' -\omega t + kn_yd)\hat{y} + E_{0z}\cos(kx' - \omega t + kn_z d)\hat{z}   
\end{gather*}
La polarizzazione in uscita sarà quindi ellittica.  La velocità nel mezzo
dipende da $\frac{\lambda}{T}$, ma la lunghezza d'onda dipende da $n$: nel 
mezzo birifrangente esisterà un asse lungo il quale la velocità dell'onda va più lenta (e quindi
si ha un indice di rifrazione maggiore) ed un asse lento con indice di rifrazione minore. \\
Cambiando l'origine dei tempi, posso aggiungere o togliere a piacere
un termine di fase comune su due coseni. Se ponessi come termine di fase $-kn_yd$, 
allora il campo elettrico uscente sarà
\begin{gather*}
    \vv{E_{out}}(x', t) = E_{0y}\cos(kx'- \omega t)\hat{y} + E_{0z}\cos(kx' - \omega t + kd(n_z - n_y))\hat{z}  
\end{gather*}
E quindi la fase dipende dalla differenza tra gli indici di rifrazione.


\subsection{La lamina $\frac{\lambda}{2}$}
La prima lamina di interessa  è quella la cui fase è $\pi$:
\begin{gather*}
    kd(n_z - n_y) = \pi
\end{gather*}
Se esplicitassi $k$ 
\begin{gather*}
    k = \frac{2\pi}{\lambda} \ \Longrightarrow \ d(n_z - n_y) = \frac{\lambda}{2}
\end{gather*}
La differenza tra i due indici di rifrazione che moltiplicano la distanza prende
il nome di \textbf{cammino ottico}. La differenza tra i cammini ottici in questa lamina è pari alla metà
della lunghezza d'onda in uscita: l'effetto di questa lamina è quello di introdurre
un ritardo di fase uguale a $\pi$ lungo l'asse lento. Inoltre, se la polarizzazione
in ingresso è lineare, la luce trasmessa ha polarizzazione ancora lineare ma lungo l'asse
riflesso rispetto all'asse veloce della lamina. Dato che la fase è $\pi$, allora 
\begin{align}
    \vv{E_{out}}(x', t) = E_{0y}\cos(kx'- \omega t + \frac{\pi}{2})\hat{y} - E_{0z}\cos(kx' - \omega t)\hat{z}  
\end{align}


\begin{wrapfigure}{r}{0.4\textwidth}
    \centering
    \caption{}
    \begin{tikzpicture}
        \draw(-2, 0) -- (2, 0) node[at end, below] {$z$};
        \draw[->](0, -2) -- (0, 2) node[at end, left] {$y$};
        \draw(0, 0)circle (1.5);
        \filldraw(0, 0) circle (1pt);
        \draw(0, 0) circle (0.15);
        \draw[->, green](0.15, 0.1) -- (1.2,0.7);
        \draw[->, purple](0.15, 0.15) -- (1, 1);
        \draw[->, cyan](0.1, 0.2) -- (0.7, 1.25);
        \draw[->, green](-0.15, 0.1) -- (-1.2,0.7);
        \draw[->, purple](-0.15, 0.15) -- (-1, 1);
        \draw[->, cyan](-0.1, 0.2) -- (-0.7, 1.25);
    \end{tikzpicture}    
\end{wrapfigure}
Nel caso di una polarizzazione circolare sinistra $\sigma^{+}$:
\begin{gather*}
    \vv{E_{out}} = E_{0}\cos(kx - \omega t + \frac{\pi}{2})\hat{y} + E_{0}\cos(kx - \omega t)\hat{z}   
\end{gather*} 
Aggiungendo ora il termine di fase anche per l'asse lento, allora si può scrivere
\begin{gather*}
    \vv{E_{out}} = E_0 \cos(kx - \omega t + \frac{\pi}{2})\hat{y} + E_0 \cos(kx - \omega t + \pi)\hat{z}   
\end{gather*}
Sottraggo ora un termine $-\pi$ rispetto ad entrambi i termini e quindi
\begin{gather*}
        \vv{E_{out}}(x', t) = E_{0y}\cos(kx'- \omega t - \frac{\pi}{2})\hat{y} - E_{0z}\cos(kx' - \omega t)\hat{z}  
\end{gather*}

\subsection{La lamina $\frac{\lambda}{4}$}
La lamina di ritardo $\frac{\lambda}{4}$ è molto simile a quella
di $\frac{\lambda}{2}$ anche se causa un ritardo che è la metà, ossia
produce un ritardo di fase 
\begin{gather*}
    kd(n_z - n_y)  = \frac{\pi}{2}
\end{gather*}
La differenza dei cammini ottici è allora $\frac{\lambda}{4}$

\subsection{Polarizzazione nei due casi}
In generale la polarizzazione sarà ellittica, l'unico caso semplice
però è quello in cui la polarizzazione lineare incidente ha le due componenti
$E_{0y} = E_{0z}$. In questo caso semplice, la polarizzazione lineare
incidente sarebbe lungo l'asse a $\frac{\pi}{4}$ rispetto
agli assi $y, z$ solamente se il mio campo elettrico è allineato
alla bisettrice del primo quadrante.\\
E' possibile verificare che, se ho una polarizzazione ellittica,
scegliendo un angolo opportuno in una lamina $\frac{\lambda}{4}$ posso ottenere
una polarizzazione lineare e agire sulla spanciatura dell'ellisse. Poi con 
una $\frac{\lambda}{2}$ posso ottenere una polarizzazione lineare rispetto
all'asse preferito. 

\subsection{Cosa succede se si utilizza una lamina progettata per una data lunghezza
d'onda con un altra lunghezza d'onda}
La mia lamina è stata progettata in modo tale che 
\begin{gather*}
    kd(n_z - n_y) = \pi + 2i\pi \qquad i \in \mathbb{N}
\end{gather*}
Supponendo che vogliamo utilizzare luce con  $\lambda '= \lambda + \Delta \lambda$
e supponendo che non ci sia tanta differenza tra i due indici di rifrazione, allora 
voglio determinare il ritardo di fase $\phi'$:
\begin{gather*}
    \phi' = k'd(n_z -n_y) = \frac{2\pi}{\lambda'}d(n_z - n_y) = \frac{2\pi}{\lambda + \Delta \lambda} d(n_z - n_y) 
\end{gather*}
Se la variazione d'onda ora è molto piccola, posso dire che il termine
al denominatore è sviluppabile con Taylor:
\begin{gather*}
    \phi' = \frac{2\pi}{\lambda} d(n_z - n_y) \left(1 - \frac{\Delta \lambda}{\lambda}\right)
\end{gather*}
Allora posso riscrivere 
\begin{gather*}
    (\pi + 2i \pi)\left(1 - \frac{\Delta \lambda}{\lambda}\right)
\end{gather*}
Si possono allora considerare i seguenti casi
\begin{itemize}
    \item Se $i = 0$ prendono il nome di zero order: sono molto sensibili e costano
    molto ma si adattano a tutte le lunghezze d'onda;
    \item $i >> 0$ prendono il nome di \textbf{multiple order} e sono economiche ma
    non funzionano con tutte le lunghezze d'onda ma solo per piccole variazioni
\end{itemize}

\end{document}
\documentclass[a4paper, oneside]{article}
\usepackage{graphicx}
\usepackage{amsthm}
\usepackage{amsmath}
\usepackage{amssymb}
\usepackage[a4paper,
            bindingoffset=0.2in,
            left=2cm,
            right=2cm,
            top=2cm,
            bottom=2cm,
            footskip=.25in]{geometry}
\usepackage[italian]{babel}
\usepackage{pgfplots}
\usepackage{tabularx}
\usepackage{tikz}
\usepackage{wrapfig}
\usepackage{color}
\definecolor{page}{rgb}{0.129,0.157,0.212}
\pagecolor{page}
\color{white}
\graphicspath{ {./images/} }
\usetikzlibrary{shapes.geometric}
\usetikzlibrary{datavisualization}
\usetikzlibrary{datavisualization.formats.functions}
\pgfplotsset{width=10cm,compat=1.9}

\title{Appunti ottica}
\author{Tommaso Miliani}
\date{19-09-25}

\begin{document}
\newtheoremstyle{theoremEnv}
                {}          % Space above
                {}          % Space below
                {\slshape}  % Body font
                {}          % Indent amount
                {\bfseries} % Head font
                {.}         % Punctuation after head
                {\newline}         % Space after theorem head
                {}          % Theorem head spec
\theoremstyle{theoremEnv}

\newtheorem{definition}{Definizione}[section]
\newtheorem{theorem}{Teorema}[section]
\newtheorem{lemma}{Proposizione}[section]
\newtheorem{observation}{Osservazione}[section]
\newtheorem{corollary}{Corollario}[theorem]
\newtheorem{example}{Esempio}[section]

\maketitle

\section{Onde elettromagnetiche}
\subsection{I metalli }
All'interno dei metalli gli elettroni si muovono liberamente e, mentre
oscillano sotto un certo campo elettrico esterno, essi iniziano ad oscillare
e generano un campo elettrico in \textbf{controfase} al campo elettrico incidente
annullando il campo elettrico in arrivo e incontrando
resistenza nel loro movimento; l'effetto è la liberazione di energia sottoforma di calore nel metallo. L'effetto complessivo
è che certe lunghezze d'onda sono assorbite mentre altre sono
riflesse. 

\subsection{Incidenza di onde piane ad un certo angolo}
\begin{wrapfigure}{r}{0.5\textwidth}
    \centering
    \caption{Relazione tra fronti d'onda e angolo di incidenza}
    \begin{tikzpicture}
        \draw(2, 4) -- (2, -2.5);
        \filldraw(0.5, 4) circle (0pt) node[anchor = north] {Aria $n = 1$};
        \filldraw(3.5, 4) circle (0pt) node[anchor = north] {Vetro $n = 1.5$};
        \draw[dashed](-1, 0) -- (5, 0);
        \draw(-1, 2.5) -- (2, 0);
        \draw(1, 0) arc (180:140:1)  node[at start, below] {$\theta_i$};
        \draw(2, 0) -- (5, -1);
        \draw(4, 0) arc (0:-18:2) node[midway, right] {$\theta_f$};
        \draw(1.5, -0.75) -- (2, 0);
        \draw(2, 0) -- (2.75, 3);
        \draw(0.5, 0)-- (2, 2);
        \draw(2, 2) -- (2.5, 4);
        \draw(-1, 0) -- (2, 4);
        \draw[<->](2.75, 3) -- (2.3, 3.15) node[midway, above] {$\lambda'$};
        \draw[<->](0.75, 2.25) -- (1.7, 1.6) node[midway, above] {$\lambda$};
        \draw[cyan](2, 0) -- (2, 2) node[midway, left] {$D$};
        \draw(2, 2) arc (90:60:1) node[midway, above] {$\theta_f$};
        \draw(1.7, 1.6) arc(245:270:0.6) node[midway, below] {$\theta_i$};
    \end{tikzpicture}    
\end{wrapfigure}
Preso un sistema di riferimento piano, da una parte abbiamo l'aria (con indice di rifrazione
molto simile al vuoto) e dall'altra del vetro (o qualsiasi
altro materiale con $n_2 \neq n_1$). Possiamo allora vedere cosa succede ai fronti d'onda che attraversano
l'interfaccia tra i due materiali con un angolo di incidenza diverso da zero. Si può vedere che l'onda nel dielettrico è un'onda
piana anch'essa in quanto lavoriamo sempre nella stessa ipotesi di onda piana. So inoltre che
i fronti d'onda nel dielettrico sono gli stessi che al di fuori del dielettrico ma con
lunghezza d'onda minore. Per far sì che i fronti d'onda coincidono, fisicamente succede che l'onda si inclina
rispetto all'angolo di incidenza $\theta_i$ di un certo angolo $\theta_f$ chiamato
di \textbf{rifrazione}.  L'ipotenusa identificata con $D$ al centro del disegno è
in relazione alle lunghezze d'onda e agli angoli di incidenza e di rifrazione secondo la seguente formulazione 
\begin{gather*}
    D \sin \theta_i = \lambda \\
    D \sin \theta_f = \lambda'
\end{gather*}
Se combiniamo le due equazioni si ottiene l'utile relazione (considerando che
l'indice di rifrazione dell'aria è $n_1$ e quello del vetro è $n_2$)
\begin{gather*}
    \frac{\lambda}{n_2} \frac{1}{\sin\theta_f} = \frac{\lambda}{n_1} \frac{1}{\sin\theta_i} \ \Longrightarrow \ \frac{n_1}{n_2} = \frac{\sin\theta_f}{\sin\theta_i}
\end{gather*}
Ossia la \textbf{Legge di Snell}:
\begin{align}
    n_2 \sin\theta_f = n_1 \sin\theta_i
\end{align}
Le onde che sono riflesse, vengono riflesse con un angolo di riflessione $\theta_r = \theta_i$
in quanto si riflettono all'interno dello stesso dielettrico e dunque per la legge di Snell 
si riflettono con lo stesso angolo con cui incidono. 

\subsection{Riflessione totale interna e prisma}
\begin{wrapfigure}{r}{0.4\textwidth}
    \centering
    \caption{Riflessione totale interna}
    \begin{tikzpicture}
        \draw[dashed](-2, 0) -- (2, 0);
        \draw(0, -2) -- (0, 2);
        \draw(-2, 0.5) -- (0, 0);
        \draw(0, 0) -- (2, -1.5);
        \draw(-1, 0.25) arc (160:180:0.75) node[midway, left] {$\theta_i$};
        \draw(-2, 2.5) -- (0, 0);
        \draw(-0.5, 0.62) arc(120:180:0.7) node[midway, above] {$\theta_c$};
        \draw(0, 0) -- (-1,-1.25);
        \draw[dashed](-1 , -1.25) -- (-2, -2.5);
        \filldraw(-1, 2) circle (0pt) node[anchor = center] {$n_2=$ vetro};
        \filldraw(1, 2) circle(0pt) node[anchor = center] {$n_1=$ aria};
    \end{tikzpicture}    
\end{wrapfigure}
Supponendo di avere sempre la stessa interfaccia, ma stavolta
le onde arrivano da dentro il dielettrico con indice di rifrazione maggiore,
quindi quando arrivano all'interfaccia si rifrangono con un angolo
molto grande. Il raggio di luce dunque, sopra un certo angolo, tenderà a rifrangersi
con un angolo sempre più grande: in questo caso si arriva a dire che non si rifrangerà nessun raggio
di luce all'esterno del dielettrico; si parla allora di \textbf{riflessione totale interna}.
Questo si ha quando $\theta_i > \theta_c$ ossia l'angolo critico oltre al quale
non si rifrangerà più alcun raggio di luce. L'angolo critico si ricava dalla legge di Snell:
\begin{gather*}
    n_2 \sin\theta_i = n_1 \sin\theta_f \\
        n_2 \sin\theta_c = n_1 \sin \frac{\pi}{2}
\end{gather*}
Allora l'angolo critico sarà
\begin{align}
    \sin\theta_c = \frac{n_1}{n_2}
\end{align}
Dove $n_2 > n_1$.  Nel caso del prisma, quando entra il laser dalla parte perpendicolare al
fascio di luce, la luce subisce un accorciamento della sua lunghezza d'onda mentre quando dall'interno
colpisce le pareti inclinate del prisma (a $45$ gradi e quindi maggiore dell'angolo critico),
e quindi riflette solo all'interno rimbalzando fino a che non esce dalla parte nuovamente perpendicolare
al fascio iniziale. Ruotando il prisma posso ottenere una parziale rifrazione della luce in modo da inclinare il fascio 
uscente.

\subsection{Uscita da un mezzo dielettrico}
\begin{wrapfigure}{r}{0.4\textwidth}
    \centering
    \caption{Deviazione della luce}
    \begin{tikzpicture}
        \draw(2, 2) -- (2, -2);
        \draw(3, 2) -- (3, -2);
        \filldraw(2.5, 2) node[anchor = center] {vetro};
        \draw(1, 1) -- (2, 0);
        \draw(2, 0) -- (3, -0.5);
        \draw(3, -0.5) -- (4, -1.5);
        \draw[dashed, very thin](1, 1) -- (4, -2);
    \end{tikzpicture}    
\end{wrapfigure}
Quando un fascio di luce entra in un mezzo e poi ne esce,
questo esce con lo stesso angolo in entrata ma con un discostamento
rispetto alla direzione iniziale di entrata nel mezzo. Questo è dato dal fatto
che la legge di Snell mi modifica l'angolo in entrata e dunque l'angolo in uscita si ottiene
applicando nuovamente la legge al contrario. \\
All'interno di un prisma è possibile invece far cambiare la direzione della luce completamente
in quanto si utilizza la rifrazione per creare raggi con angoli diversi
a seconda della lunghezza d'onda in entrata: il fascio di luce bianca che entra dentro
un prisma è scomposto in tanti fasci luminosi di lunghezza d'onda diversa
tutti distinti ed osservabili.


\section{Le lenti}
\subsection{Introduzione alle lenti sferiche}
\begin{wrapfigure}{r}{0.6\textwidth}
    \centering
    \caption{Le lenti}
    \begin{tikzpicture}
        \draw(0, 2) arc (90:270: 2);
        \draw[dashed](-5, 0) -- (2, 0);
        \filldraw(-5, 0) circle (1pt) node[anchor = east] {$P$};
        \draw[green](-5, 0) -- (-1, 1.73);
        \filldraw(-1, 1.73) circle (1pt) node[anchor = south] {$C$};
        \filldraw(0, 0) circle (1pt) node[anchor = north] {$O$};
        \filldraw(2, 0) circle (1pt) node[anchor = west] {$Q$};
        \draw(-1, 1.73) -- (2, 0);
        \draw(1.5, 0) arc(180:150:0.5) node[at start, below] {$\alpha$};
        \draw[cyan](-1, 1.73) -- (-1, 0) node[near end, left] {$h$};
        \draw[green](-4, 0.45) arc(25:0:1.1) node[midway, right] {$\theta$};
        \draw[<->, green](-5, -0.5) -- (-1.1, -0.5) node[midway, below] {$p$};
        \draw[<->, cyan](-0.9, -0.5) -- (2, -0.5) node[midway, below] {$q$};
        \draw[dashed](0, 0) -- (-1.4, 2.4);
        \draw(-1.25, 2.2) arc(110:195:0.6) node[midway, left] {$\theta_i$};
        \draw[cyan](-0.2, 0.32) arc(120:180:0.35) node[midway, left] {$\phi$};
        \draw(-0.27, 1.25) arc(-25:-60:0.7) node[midway, right] {$\theta_r$};
    \end{tikzpicture}    
\end{wrapfigure}
Supponendo di avere una sfera di vetro curvilinea e all'esterno aria; voglio vedere che succede se le
onde piane della luce lo colpiscono con un certo angolo rispetto alla direzione del raggio della sfera.
Chiamo $\theta$ l'angolo di inclinazione rispetto alla congiungente
sorgente-centro della circonferenza della lente e $p$ la lunghezza dalla sorgente alla congiungente
della verticale dal punto di contatto all'orizzontale.
Si può supporre che se l'angolo $\theta << 1$, allora
posso approssimare la lunghezza $p$ come la distanza da $P$
alla superficie di contatto della lente (ipotesi \textbf{parassiale}). Chiamo allora $\theta_i$ l'angolo
di incidenza sulla lente e $\theta_r$ l'angolo di rifrazione dovuto alla lente rispetto alla
congiungente punto di contatto $C$ al centro $O$ della circonferenza. Ottengo
degli altri angoli: $\phi$, ossia l'angolo tra l'orizzontale e la congiungente $OC$ e 
$\alpha$, ossia l'angolo che forma la congiungente $PQ$ rispetto all'orizzontale.\\
Posso chiamare $q$ come $PQ - p$ e $Q$ il punto di contatto tra il fascio
e l'orizzontale e l'angolo $\alpha$ di incidenza del fascio rifranto sull'orizzontale. 
Si ottiene allora la seguente relazione vicino al punto di contatto $C$:
\begin{gather*}
    \theta_i + \left(\frac{\pi}{2} - \theta\right) + \left(\frac{\pi}{2} - \phi\right) = \pi \ \Longrightarrow \ \theta_i = \theta +\phi \qquad (1)
\end{gather*}
Inoltre si ottiene un'altra relazione
\begin{gather*}
    \left(\frac{\pi}{2} - \phi\right) + \theta_r = \frac{\pi}{2} - \alpha \ \Longrightarrow \ \alpha = \phi - \theta_r  \quad (2)
\end{gather*}
Inoltre si ha che
\begin{gather*}
    h = q\tan\alpha = p\tan\theta = R\sin\phi
\end{gather*}
Data l'ipotesi parassiale le relazioni per $h$ sono tutte uguali l'una dall'altra
e posso togliere le varie tangenti e seni in quanto per $\theta << 1$ si ha che $\sin\theta, \tan\theta \approx \theta$:
\begin{gather*}
    h = q\alpha = p\theta = R\phi \qquad (3, 4, 5)
\end{gather*}
Rifacendosi alla legge di Snell e ricordando l'ipotesi parassiale, si ottiene che
\begin{gather*}
    n_{aria}\sin\theta_i = n_{vetro}\sin\theta_r \ \Longrightarrow \ n_{aria} \theta_i = n_{vetro} \theta_r
\end{gather*}
Posso far sparire glia angoli utilizzando le varie relazioni:
posso utilizzare intanto la $(1)$, poi la $(2)$ e ottenere:
\begin{gather*}
    n_{aria}(\theta + \phi) = n_{vetro}(\phi - \alpha)
\end{gather*}
Posso allora utilizzare le relazioni dell'altezza per eliminare definitivamente
gli angoli:
\begin{gather*}
    n_{aria} \left(\frac{h}{p} + \frac{h}{R}\right) = n_{vetro} \left(\frac{h}{R} - \frac{h}{q}\right)
\end{gather*}
Adesso posso semplificare $h$ dappertutto:
\begin{align}
    n_{aria}\left(\frac{1}{p} + \frac{1}{R}\right) = n_{vetro} \left(\frac{1}{R} - \frac{1}{q}\right)
\end{align}
Con questa equazione io vedo che tutti i raggi che emetto dalla mia sorgente
$P$ giungono tutti nel medesimo punto $Q$ con angoli sufficientemente
piccoli (ossia nel limite parassiale). Se si ripetesse il conto con un'altra
interfaccia sferica vetrosa di raggio $R_2 < R$, allora scoprirò che il
punto di convergenza di tutti i raggi numerosi è dato da
\begin{gather*}
    \frac{1}{p} + \frac{1}{q} = \left(\frac{1}{R} - \frac{1}{R_2}\right) \frac{n _{vetro} - n_{aria}}{n_{aria}} + \frac{n_{vetro}d}{p(p - d)}
\end{gather*} 



\end{document}
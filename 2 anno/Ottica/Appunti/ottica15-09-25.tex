\documentclass[a4paper, oneside]{article}
\usepackage{graphicx}
\usepackage{amsthm}
\usepackage{amsmath}
\usepackage{amssymb}
\usepackage[a4paper,
            bindingoffset=0.2in,
            left=2cm,
            right=2cm,
            top=2cm,
            bottom=2cm,
            footskip=.25in]{geometry}
\usepackage[italian]{babel}
\usepackage{pgfplots}
\usepackage{tabularx}
\usepackage{tikz}
\usepackage{wrapfig}
\usepackage{color}
\definecolor{page}{rgb}{0.129,0.157,0.212}
\pagecolor{page}
\color{white}
\graphicspath{ {./images/} }
\usetikzlibrary{shapes.geometric}
\usetikzlibrary{datavisualization}
\usetikzlibrary{datavisualization.formats.functions}
\pgfplotsset{width=10cm,compat=1.9}

\title{Appunti ottica}
\author{Tommaso Miliani}
\date{15-09-25}

\begin{document}
\newtheoremstyle{theoremEnv}
                {}          % Space above
                {}          % Space below
                {\slshape}  % Body font
                {}          % Indent amount
                {\bfseries} % Head font
                {.}         % Punctuation after head
                {\newline}         % Space after theorem head
                {}          % Theorem head spec
\theoremstyle{theoremEnv}

\newtheorem{definition}{Definizione}[section]
\newtheorem{theorem}{Teorema}[section]
\newtheorem{lemma}{Proposizione}[section]
\newtheorem{observation}{Osservazione}[section]
\newtheorem{corollary}{Corollario}[theorem]
\newtheorem{example}{Esempio}[section]

\maketitle

\section{Introduzione al corso}
Il corso di ottica si propone di studiare i quattro fenomeni principali dell'ottica:
\begin{enumerate}
    \item Ottica geometrica: approssimazione della luce come un insieme
    di raggi luminosi e come essi si propagano nel vuoto (lenti e formazioni
    di immagini);
    \item Polarizzazione: Dato che la luce è una onda elettromagnetica, e dato che
    il campo è un campo di vettori, il campo elettrico e luminoso è un vettore
    che oscilla nel tempo e questa oscillazione è proprio la polarizzazione;
    \item Diffrazione: il fenomeno per il quale la luce si diffonde;
    \item Interferenza: il fenomeno più complicato e bello della luce: due sorgenti luminose
    con campi opposti l'uno rispetto all'altro possono interferire distruttivamente
    e quindi non creare alcuna luce, ovviamente esiste anche l'interferenza costruttiva.
\end{enumerate}
Il peso di ogni relazione è un decimo sul totale del voto. Il restante sessanta percento
del voto è legato ad un compito in classe (molto simili l'uno all'altro). La media pesata di 
questi voti dà il voto finale (senza orale).
Non si potrà essere super rigorosi in questo corso, ma si daranno le basi dell'ottica attraverso
un approccio più pragmatico e semplice in quanto sarebbe necessario studiare prima fisica
due e fisica quantistica.

\section{Ottica Geometrica}
\subsection{Le onde elettromagnetiche}
\begin{wrapfigure}{r}{0.4\textwidth}
    \centering
    \caption{}
    \begin{tikzpicture}
        \filldraw(0, 0) circle (1pt) node[anchor=south] {$e_1$};
        \filldraw(3, 0) circle (1pt) node[anchor=south] {$e_2$};
        \draw[->](0, 0.05) -- (-1, 0.05) node[at end, above] {$\vec{F}$};
        \draw[->](3, 0) -- (4, 0) node[at end, above] {$\vec{F}$};
        \draw[->, red](0, -0.05) -- (-1, -0.05) node[at end, below] {$\vec{\Delta S}$ };
        \draw[->] (0, 0) -- (1, 0) node[at end, above] {$\vec{r}$ };
    \end{tikzpicture}    
\end{wrapfigure}
Prendiamo per esempio due elettroni, noi sappiamo che si respingono
poiché tra di loro agisce la forza di Coulomb, che ha modulo:
\begin{gather*}
    \vec{F} = k\frac{q_1q_2}{r^{2} } \hat{r}  
\end{gather*}
Questa forza dipende dalla carica delle due particelle e dalla direzione $r$ congiungente.
Se io spostassi la carica $1$ di un certo $\Delta S$, la carica due non si accorgerebbe
di questo spostamento in modo immediato e quindi la relazione precedente non 
sarebbe più valida: in generale la relazione vale solamente nella statica, ossia se le due
particelle sono immobili. Avendo allora un ritardo nel cambiamento dell'interazione rispetto 
all'istante in cui è avvenuto lo spostamento, siamo allora costretti ad introdurre due campi vettoriali che si propagano con una
velocità finita. Questi campi sono il campo magnetico ed il campo elettrico ($\vec{B}$ e $\vec{E}$  )
secondo delle leggi definite da Maxwell. \\
La forza che agisce sulle due cariche è un vettore e dunque per esprimere
come cambia un vettore, ho bisogno di un vettore per descrivere questa variazione:
posso esprimere allora con la legge di Lorentz come una carica risente della presenza di un
campo elettrico e di un campo magnetico attraverso la seguente:
\begin{align}
    \vec{F}_q = q(\vec{E} + \vec{v} \times \vec{B}  ) 
\end{align}
Dalle equazioni di Maxwell nel vuoto (ossia in assenza di cariche e correnti di cariche), introduciamo
quindi l'operatore \textbf{Laplaciano quadro}, ossia un operatore differenziale
che si applica al campo elettrico e anche a quello magnetico secondo la seguente: 
\begin{align}
    \vec{ \nabla}^{2} \vec{E} &= \epsilon_0 \mu_0 \frac{\partial^{2}  \vec{E} }{\partial t^{2} } \\
    \vec{ \nabla}^{2} \vec{B} &= \epsilon_0 \mu_0 \frac{\partial^{2}  \vec{B} }{\partial t^{2} } 
\end{align} 
L'operatore Laplaciano agisce su ogni componente del vettore campo elettrico o campo magnetico.
Possiamo vedere cosa succede per $\vec{E}_x$:
\begin{gather*}
    \epsilon_0 \mu_0 \frac{\partial^{2}  \vec{E}_x }{\partial t^{2} } = \frac{\partial ^{2} E_x }{\partial x^{2} } + \frac{\partial ^{2} E_x }{\partial y^{2} } + \frac{\partial^{2}  E_x}{\partial z^{2} } 
\end{gather*} 
Questa equazione mi esprime il contributo rispetto ad un singolo asse e va sotto il nome di \textbf{equazione delle onde} ed è analoga all'equazione delle onde
per tutte le onde che si studiano (sonore, luminose, radio ...):
\begin{align}
    \frac{1}{v^{2} } \frac{\partial^{2} \psi(x, t)}{\partial t^{2} } = \frac{\partial ^{2} \psi(x, t) }{\partial x^{2} }  
\end{align}
In questo caso $v$ rappresenta la velocità di propagazione dell'onda rispetto ad un sistema
di riferimento inerziale (per ora in quanto non si considera la relatività). Si può trovare una
soluzione generale a questa equazione come 
\begin{gather*}
    \psi(x, t) = f(x - vt)
\end{gather*}
Ossia prendo una funzione $f$ a caso in modo tale che abbia la stessa forma di $\psi$ e dunque
se calcolata in $x - vt$ sarà soluzione della mia equazione. Devo allora verificare che sia soluzione
attraverso la doppia derivata e vedere che effettivamente sia soluzione:
\begin{gather*}
    \frac{\partial f}{\partial x} = f'(x - vt) \\
    \frac{\partial ^{2} f }{\partial x^{2} } = f''(x - vt)  \\
    \frac{\partial f}{\partial t} f'(x - vt) \cdot (-v)  \\
    \frac{\partial ^{2} f }{\partial t^{2} } = f''(x - vt) \cdot  (-v) \cdot (-v)
\end{gather*}
Risolvendo l'equazione con $\psi$ posso allora dire che
\begin{gather*}
    f''(x - vt) = \frac{1}{v^{2} } v^{2} f''(x - vt) 
\end{gather*}

\begin{wrapfigure}{r}{0.4\textwidth}
    \centering
    \caption{Il grafico della funzione generica $f(x)$}
    \begin{tikzpicture}
        \draw[->](-3, 0) -- (4, 0) node[at end, below] {$x$};
        \draw[->](0, -1) -- (0, 3) node[at end, left] {$f(x)$};
        \draw[cyan](-3, 0) .. controls(-2, 3.5)  and (0, 0) ..  (2, 0);
        \draw[cyan](2, 0) .. controls  (2.5, 1)  and (3, 2) .. (4, 0);
        \draw[dashed](-2.6, 0) -- (-2.6, 1) node[at start, below] {$x = \alpha$};
        \draw[dashed, red] (-2.6, 1) -- (2.75, 1) node[midway, above] {$f(\alpha)$};
        \draw[dashed](2.75, 1) -- (2.75, 0) node[at end, below] {$\alpha + vt$};
        \draw[red](-2.6, 0) -- (2.75, 0);
    \end{tikzpicture}    
\end{wrapfigure}

E quindi ho dimostrato che è soluzione generica dell'equazione.
Al tempo $t = 0$ io calcolo la funzione $f(x - vt)$ e dunque mi chiedo
dopo un tempo $t$ in quale punto dello spazio la funzione $f$ assumerà
lo stesso valore $f(\alpha)$ ? Ossia nel punto in cui
$x - vt = \alpha$ e allora l'$x$ cercato è $\alpha + vt$. In qualche modo mi
sono spostato di una certa quantità $vt$ e dunque in quel punto il campo assumerà
nuovamente il valore che aveva nel punto $\alpha$. E' per questo che si parla di equazione
delle onde che si muovono di velocità $v$ poiché queste onde si propagano nello spazio
secondo $vt$.\\
Le costanti che ho utilizzato fino ad ora sono
\begin{align}
    \epsilon_0= 8.85 \cdot  10^{-12}\  \frac{C^{2} }{m^{2}N } 
\end{align}
E la permeabilità magnetica nel vuoto la costante dielettrica nel vuoto:
\begin{align}
    \mu_0 = 4\pi \cdot  10^{-7} \ \frac{T\cdot m}{A} 
\end{align}
La velocità di propagazione delle onde elettromagnetiche nel vuoto è proprio
\begin{align}
    c = \sqrt{\frac{1}{\epsilon_0 \mu_0}} = 299792458\ \frac{m}{s}  
\end{align}

\subsection{La soluzione semplice: le onde piane}
\begin{wrapfigure}{r}{0.4\textwidth}
    \centering
    \caption{L'ipotesi graficamente}
    \begin{tikzpicture}
        \draw[->](0, 0) -- (2, 0) node[at end, below] {$x$};
        \draw[->](0, 0) -- (-1, -1) node[at end, below] {$z$};
        \draw[->](0, 0) -- (0, 2) node[at end, left] {$y$};
        \draw(-0.25, -0.25) -- (-0.25, 1.75);
        \draw(-0.75, -0.75) -- (-0.75, 1.25);
        \draw(0, 0.25) -- (-1, -0.75);
        \draw(0, 0.75) -- (-1, -0.25);
        \draw[->, cyan](-0.75, 0.5) -- (-0.25, 1) node[at end, above] {$\vec{E} $};
    \end{tikzpicture}    
\end{wrapfigure}
La soluzione più semplice all'equazione delle onde è l'onda piana:
questa equazione ha come ipotesi che $\vec{E}$ e $\vec{B}$ siano uniformi
in un piano perpendicolare alla direzione di propagazione. Se considerassimo
l'asse $x$ come l'asse di propagazione e consideriamo che il nel piano $yz$ entrambi i campi
siano uniformi e quindi che le derivate rispetto agli assi $y$ e $z$ siano zero.
Ottengo allora, nel caso di onda piana e quindi nel caso più semplice:
\begin{gather*}
    \epsilon_0 \mu_0 \frac{\partial^{2} E_x}{\partial x^{2} } = \frac{1}{c^{2} } \frac{\partial ^{2} E_x}{\partial t^{2} }  \\
    \epsilon_0 \mu_0 \frac{\partial^{2} E_y}{\partial y^{2} } = \frac{1}{c^{2} } \frac{\partial ^{2} E_y}{\partial t^{2} }  \\
    \epsilon_0 \mu_0 \frac{\partial^{2} E_z}{\partial z^{2} } = \frac{1}{c^{2} } \frac{\partial ^{2} E_z}{\partial t^{2} } 
\end{gather*}  
Possiamo ora assumere che dalle equazioni di Maxwell il vettore
campo elettrico non può essere diretto lungo la direzione della propagazione dell'onda
e quindi le uniche componenti che rimangono sono quelle perpendicolari all'asse di propagazione
e dunque $E_x = 0$. Si tratteranno d'ora in poi onde sinusoidali, ossia le onde che hanno
la seguente espressione:
\begin{align}
    E_{y, z} = A\cos\left(k(x - vt) + \phi\right)
\end{align}

\begin{wrapfigure}{r}{0.4\textwidth}
    \centering
    \caption{L'onda elettromagnetica}
    \begin{tikzpicture}[scale=0.5]
        \draw[->](-4, 0) -- (4, 0) node[at end, below] {$x$};
        \draw[->](0, -4) -- (0, 4) node[at end, left] {$E_{y, z}$};
        \draw[<->](-3, -1) --( 3, -1) node[midway, below] {$\lambda$};
        \draw[color=cyan]   plot (\x,{cos(\x r)});

    \end{tikzpicture}    
\end{wrapfigure}

Il periodo spaziale dopo il quale la mia onda torna ad assumere lo stesso
valore (ossia dopo $2\pi$) posso ottenerlo imponendo che 
\begin{gather*}
    k \cdot  \lambda = 2\pi \ \Longrightarrow \ k = \frac{2\pi}{\lambda}
\end{gather*}
E quindi chiamerò d'ora in poi $\lambda$ come \textbf{lunghezza d'onda} e $k$ il
\textbf{vettore d'onda}, le quali sono intrinsecamente legate l'una all'altra. L'altra quantità
che si usa per descrivere le onde elettromagnetiche è la frequenza e la pulsazione. Concentrandosi 
su di un certo valore di $x$ e lasciando che il tempo trascorra, il campo elettrico nel punto $x$ prefissato
inizierà ad oscillare e dunque questo avviene in un certo tempo
\begin{gather*}
    kv T = 2\pi \ \Longrightarrow \ T = \frac{2\pi}{c \cdot  k} \ \Longrightarrow \ T = \frac{2\pi}{\omega}
\end{gather*}
Chiamo allora $\omega = c \cdot  k$ la \textbf{pulsazione}. Sostituendo con l'espressione
della lunghezza d'onda ottengo
\begin{gather*}
    T = \frac{\lambda}{c} \ \Longrightarrow \ \omega = c \cdot  \frac{2\pi}{\lambda}
\end{gather*}
La direzione del campo elettrico è detta \textbf{polarizzazione} e nel caso più semplice di onda
piana noi abbiamo che la direzione del campo e del modulo è costante e quindi posso dire che il vettore
\begin{gather*}
    \vec{E} = \vec{E}_0 \cos\left(kx - \omega t + \phi\right)
\end{gather*}  
In questo caso l'onda piana con la direzione del campo elettrico non cambia nel tempo ed è 
nel piano $yz$ sempre costante e quindi vale la precedente. Se volessi esprimerla nella direzione
generica per cui 
\begin{gather*}
    \vec{r} = x \hat{u}_x + y\hat{u}_y + z\hat{u}_z    \\
    \vec{E} = \vec{E}_0 \cos\left(\hat{k} \cdot \vec{r} - \omega t + \phi \right)  
\end{gather*}

\begin{wrapfigure}{r}{0.3\textwidth}
    \centering
    \caption{}
    \begin{tikzpicture}[scale=1.5]
        \draw[->](-0.5, -0.5) -- (1, 1) node[at end, below] {$\hat{k}$ };
        \draw[|-|](0.25, -0.25) -- (0.5, 0) node[midway, below] {$\lambda$};
        \draw(0.25, -0.25) -- (-0.25, 0.25);
        \draw(0.5, -0) -- (0, 0.5);
        \draw(0.75, 0.25) -- (0.25, 0.75);
    \end{tikzpicture}    
\end{wrapfigure}
In questo caso il vettore $\hat{k}$ mi indica la direzione di propagazione
della mia onda e dunque posso risolvere il prodotto scalare con il vettore
direzione $\vec{r}$ e ottenere che la direzione di propagazione ed il suo verso
sono proprio quelli di $\hat{k}$.    Posso rappresentare le onde utilizzando solo il 
fronte d'onda (ossia il picco dell'onda) con larghezza $\lambda$.
Maxwell considera che il campo magnetico oscilla in fase rispetto al campo elettrico 
e quindi tutte le volte che si ha un onda che si propaga nello spazio, oltre che al campo
elettrico che oscilla insieme all'onda, si ha anche il campo magnetico che oscilla
insieme all'onda in fase (ossia traslato rispetto al campo elettrico), inoltre
il campo magnetico oscilla perpendicolarmente rispetto al campo magnetico.



%Inserire immagine delle onde qui
%\begin{wrapfigure}{r}{0.4\textwidth}
%    \centering
%    \caption{}
%    \includegraphics[width=0.4\textwidth]{}
%\end{wrapfigure}
Sempre secondo Maxwell il modulo del campo magnetico si rapporta al modulo del campo
elettrico secondo la relazione
\begin{gather*}
    \left|\vec{B}\right| = \frac{\left| \vec{E}  \right| }{c} 
\end{gather*}
Dato che la forza su di una carica è esprimibile come
\begin{gather*}
    \vec{F} = q\left(\vec{E} + \vec{v} \times \vec{B}   \right) 
\end{gather*}
E quindi posso esprimere la forza sulla particella come come
$q\left| \vec{E}  \right| $ poiché nell'espressione del prodotto 
vettoriale si ottiene che il prodotto dei moduli sullo stesso asse diventa
\begin{gather*}
    q\left| \vec{v}  \right| \left| \vec{B}  \right| \sin \theta = q \frac{|\vec{v}| }{c}\left| \vec{E}  \right| \sin\theta   
\end{gather*} 
Data l'espressione del modulo del campo magnetico posso trascurare 
il termine $\frac{\left| \vec{v}  \right| }{c}$ solo se la velocità dell'onda è molto piccola (in generale solo nei solidi).

\subsection{Materiali dielettrici}
\begin{wrapfigure}{r}{0.4\textwidth}
    \centering
    \caption{Materiali dielettrici}
    \begin{tikzpicture}
        \draw(0, 0) circle (0.25);
        \draw(0, 0) circle (0.5);
        \filldraw(0, 0) circle (0pt) node[] {$+$};
        \filldraw(0, 0.5) circle (2 pt) node[anchor = south] {$e^ -$};
        \draw(3, 0) circle (0.25);
        \draw(3.5, 0) ellipse (1 and 0.5);
        \filldraw(3, 0) circle (0pt) node[] {$+$};
        \filldraw(3.5, 0.5) circle (2 pt) node[anchor = south] {$e^ -$};
        \draw[->](4, -1) -- (3, -1) node[midway, below] {$\vec{E}$};
    \end{tikzpicture}    
\end{wrapfigure}Il mezzo dielettrico è un mezzo nel quale gli elettroni sono liberi di muoversi in modo molto
limitato e quindi, avendo una ridotta mobilità, la componente del campo magnetico è molto piccola rispetto al
campo elettrico. Un campo elettrico applicato alle cariche tende a far allontanare
gli elettroni dai nuclei e quindi si distribuiscono diversamente nello spazio
rispetto allo stato di quiete in quanto la forza elettrica riesce ad influenzare
la distribuzione di carica senza farle muovere liberamente (nei metalli le cariche si muovono
liberamente ma nei materiali dielettrici come il vetro, non lo fanno)  \\
Quando un onda entra in un materiale dielettrico essenzialmente modifica la sua lunghezza
d'onda e quindi la pulsazione e dunque la lunghezza d'onda nel vuoto si relaziona
a quella nel materiale dielettrico secondo la relazione
\begin{align}
    \lambda' = \frac{\lambda_v}{n}
\end{align}
Dove $\lambda_v$ è la lunghezza d'onda nel vuoto e $n$ la \textbf{costante dielettrica} del materiale in questione.


\end{document}
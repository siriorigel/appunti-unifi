\documentclass[a4paper, oneside]{article}
\usepackage{graphicx}
\usepackage{amsthm}
\usepackage{amsmath}
\usepackage{amssymb}
\usepackage[a4paper,
            bindingoffset=0.2in,
            left=2cm,
            right=2cm,
            top=2cm,
            bottom=2cm,
            footskip=.25in]{geometry}
\usepackage[italian]{babel}
\usepackage{pgfplots}
\usepackage{tabularx}
\usepackage{tikz}
\usepackage{wrapfig}
\usepackage{color}
\usepackage[d]{esvect}
\usepackage{chemfig}
\usepackage{mhchem}
\definecolor{page}{rgb}{0.129,0.157,0.212}
\pagecolor{page}
\color{white}
\graphicspath{ {./images/} }
\usetikzlibrary{shapes.geometric}
\usetikzlibrary{datavisualization}
\usetikzlibrary{datavisualization.formats.functions}
\usetikzlibrary{patterns}
\pgfplotsset{width=10cm,compat=1.18}

\title{Appunti Ottica}
\author{Tommaso Miliani}
\date{21-11-25}

\begin{document}
\newtheoremstyle{theoremEnv}
                {}          % Space above
                {}          % Space below
                {\slshape}  % Body font
                {}          % Indent amount
                {\bfseries} % Head font
                {.}         % Punctuation after head
                {\newline}  % Space after theorem head
                {}          % Theorem head spec
\theoremstyle{theoremEnv}

\newtheorem{definition}{Definizione}[section]
\newtheorem{theorem}{Teorema}[section]
\newtheorem{lemma}{Proposizione}[section]
\newtheorem{observation}{Osservazione}[section]
\newtheorem{corollary}{Corollario}[theorem]
\newtheorem{example}{Esempio}[section]
\newtheorem{remark}{Enunciato}[section]

\maketitle

\section*{Esperienza Interferometro di Machelson}
\section{Apparato sperimentale}
\begin{wrapfigure}{r}{0.4\textwidth}
    \centering
    \caption{L'interferometro di Machelson}
    \begin{tikzpicture}
        \draw[->](0, 0.2) -- (2.25, 0.2);
        \draw[->](2.25, 0.2) -- (2.25, 2);
        \draw[->](2.5, 2) -- (2.5, 0.75);
        \draw[->](2.5, 0.75) -- (1, 0.75);
        \draw[->](2.5, 0.75) -- (2.7, 0.35) -- (2.7, -2);
        \draw[->](2.25, 0.2) -- (2.55, 0) -- (4.5, 0);
        \draw[->](4.5, -0.25) -- (2.42, -0.25);
        \draw[->](2.42, -0.25) -- (2.42, -2);
        \draw[->](2.42, -0.25) -- (2.15, -0.1) -- (0, -0.1);
        \draw(1.75, -1) -- (2, -1.2) -- (3, 1) -- (2.75, 1.2);
        \draw[very thick](2.75, 1.2) -- (1.75, -1);
        \draw[pattern = north west lines, pattern color = white](1.75, 2) rectangle (3, 2.5);
        \draw[pattern = north west lines, pattern color = white](4.5, 0.75) rectangle (5, -1);
        \filldraw (2.1, -2) rectangle (3, -2.25);
    \end{tikzpicture}
\end{wrapfigure}
L'interferometro di Machelson è un beam splitter e non un polarizer:
dunque non polarizza la luce ma semplicemente divide il fascio 
luminoso in base  al campo incidente. Il laser che si utilizza è un laser
che spara una luce a lunghezza d'onda ignota. Dunque il primo scopo dell'esperienza
è proprio quello di misurare la lunghezza d'onda incognita che viene
sparata dal laser elio-neon. Si pul esprimere l'intensità
totale del campo elettrico in funzione della distanza dei due rilevatori
come
\begin{gather*}
    I = I_0 \sin^{2}\left(\frac{\pi(L_1 - L_2)}{\frac{\lambda}{2}}\right)   
\end{gather*}
La lunghezza d'onda incognita è quella del laser verde utilizzato per l'esperienza
della polarizzazione. Il segnale del fotodiodo sui rilevatori è misurato 
tramite un oscilloscopio digitale che misura una tensione analogica
in Volt a 12 bit. \\
Montando uno dei due specchi su di una ceramica piezoelettrica: ossia 
una ceramica al quale applico una tensione variabile, lo spessore della ceramica
cambia: si dilata dunque in base alla tensione che gli è applicata. Questo fenomeno 
permette di cambiare dunque la lunghezza di uno dei due cammini ottici
attraverso della tensione. Si deve dunque misurare, in funzione della tensione, 
l'intensità del fascio di luce e tracciare un fit sinusoidale attraverso i dati 
raccolti dall'oscilloscopio digitale. Si genera dunque una rampa lineare della tensione applicata
alla ceramica piezoelettrica che dovrebbe, in linea teorica, determinare un allungamento
$\Delta L$ lineare, anche se in realtà è un andamento monotono che non si conosce 
(ma sicuramente non lineare).
\begin{gather*}
    \begin{tikzpicture}
        \draw[->](0, 0) -- (3, 0) node[at end, below] {$t$};
        \draw[->](0, 0) -- (0, 2) node[at end, left] {$V$};
        \draw(0, 0) -- (3, 2);
        \draw[->](5, 0) -- (8, 0) node[at end, below] {$V$};
        \draw[->](5, 0) -- (5, 2) node[at end, left] {$I$};
        \draw[domain=5:8, dashed] plot (\x, {sin(\x * 3 r) * sin(\x * 3 r)});
        \draw[->](10, 0) -- (13, 0) node[at end, below] {$V$};
        \draw[->](10, 0) -- (10, 2) node[at end, left] {$\Delta L$};
        \draw(10, 0) -- (13, 2) node[at end, right] {No};
        \draw[domain=10:13, red] plot (\x, {((\x - 10) * (\x- 10)) /  3});
        \draw[domain=10:13, cyan, samples = 50] plot (\x, {sqrt(\x - 10)});
    \end{tikzpicture}
\end{gather*}
Sul grafico si deve avere sia la tensione in uscita dal rilevatore di luce, che la
tensione alla ceramica piezoelettrica: dunque sullo schermo del computer
si hanno entrambe le tensioni. Il grafico è tratteggiato in quanto la sinusoide è
calcolata ottenendo dati ogni pochi millisecondi. I dati sono 
caricati sul programma scritto in Mathematica in modo tale che si analizzi i punti per trovare 
i massimi locali per determinare la curva $V - \Delta L$.
\begin{gather*}
    \begin{tikzpicture}
        \draw[->](0, 0) -- (3, 0);
        \draw[->](0, 0) -- (0, 2);
        \filldraw[red](0.7, 0.5) circle (1pt);
        \filldraw[green](0.5, 0.5) circle (1pt);
        \filldraw[red](1.4, 0.75) circle (1pt);
        \filldraw[green](1.1, 0.75) circle (1pt);
        \filldraw[red](2.4, 1.25) circle (1pt);
        \filldraw[green](1.9, 1.25) circle(1pt);
    \end{tikzpicture}
\end{gather*}
Si toglie ora il laser
elio-neon e si sostituisce con il laser verde dell'esperienza della polarizzazione
in modo tale che si possa ottenere un grafico in funzione della tensione l'intensità
e si ottiene un grafico simile a quello ricavato con la lampada
elio-neon. I grafici dell'intensità sono leggermente sfasati tra di loro.
L'intensità del seno quadro del laser rosso ha un massimo per
\begin{gather*}
    I_R(\Delta L(v)) = I_0 \sin^{2}\left(\frac{\pi \Delta L(v)}{\frac{\lambda_R}{2}}\right) \ \Longrightarrow \ \max \frac{\pi \Delta L (v)}{\frac{\lambda_R}{2}} = \frac{\pi}{2} + m_R \pi
\end{gather*}
Mentre per il laser verde ha un max per
\begin{gather*}
    I_R(\Delta L(v)) = I_0 \sin^{2}\left(\frac{\pi \Delta L(v)}{\frac{\lambda_R}{2}}\right) \ \Longrightarrow \ \max \frac{\pi \Delta L(v)}{\frac{\lambda_V}{2}} = \frac{\pi}{2} + m_V \pi
\end{gather*}
Si possono dunque ottenere le curve dei massimi delle due sorgenti
rosse e verde.
\begin{gather*}
    \left\{\begin{array}{l}
        m_R = \frac{\Delta L(v)}{\frac{\lambda_R}{2}} - \frac{1}{2} \\
        m_V = \frac{\Delta L(v)}{\frac{\lambda_V}{2}} - \frac{1}{2}
    \end{array}\right.
\end{gather*}
Si deve immaginare ogni volta che ci sia un certo valore di tensione 
della ceramica piezoelettrica in modo tale che si possa far coincidere la curva 
di tensione rossa con quella verde. Matematicamente è possibile farlo attraverso
$m_R \frac{\lambda_R}{\lambda_V}$ per cui la curva rossa viene moltiplicata per il 
rapporto tra le lunghezze d'onda ottenendo
\begin{gather*}
    \frac{\Delta L (v)}{\frac{\lambda_V}{2}} - \frac{1}{2}\frac{\lambda_R}{\lambda_V}
\end{gather*}
Per far coincidere le curve devo aggiungere un certo offset opportuno,
ossia $\frac{1}{2}\frac{\lambda_R}{\lambda_V} - \frac{1}{2}$:
\begin{gather*}
    \frac{\Delta L(v)}{\frac{\lambda_V}{2}} - \frac{1}{2}
\end{gather*}
Si \underbar{misura} il coefficiente $F$, ossia il rapporto tra le lunghezze d'onda: si misura dunque questo 
coefficiente per qualche volta in modo da ricavare anche lo scarto massimo su questo: il $\lambda_R$ è dato 
dalle caratteristiche della lampada a elio neon.
\begin{gather*}
    \left< \lambda_V \right> = \frac{\left< \lambda_R \right> }{\left< F \right> } \ \Longrightarrow \ \frac{\Delta \lambda_V}{\lambda_V} = \frac{\Delta \lambda_R}{\lambda_R} + \frac{\Delta F}{F}
\end{gather*} 
In generale lo spettro è centrato intorno ad una certa lunghezza d'onda con un certo 
intervallo: si stabilisce dunque tramite l'interferometro di Machelson
questa ampiezza: per farlo si deve ricordare la teoria. In presenza di una intensità luminosa
che è distributa secondo una gaussiana con parametro di larghezza $\sigma = \Delta \omega$: 
se questa è spedita attraverso Machelson, si vede che l'intensità in funzione del $Delta L $ modificheà
la Gaussiana totalmente:
\begin{align}
    I(\Delta L) = \frac{I_0}{2}\left(1 - \exp\left(-\left(\frac{\Delta L - \Delta \omega}{c}\right)^{2}\right) \cos\left(\frac{2\Delta L \omega_0}{c}\right)\right)
\end{align}
Il grafico che si vede facendo questo esperimento è il seguente:
\begin{gather*}
    \begin{tikzpicture}
        \draw[->](0, 0) -- (4, 0) node[at end, below] {$\Delta L$};
        \draw[->](0, 0) -- (0, 3) node[at end, left] {$I(\Delta L$)};
        %\draw[domain = 0.1:4, samples = 50] plot (\x, {2 * sin(3 *\x r) * sin(3 * \x r) / (\x * 0.5)});
    \end{tikzpicture}
\end{gather*}

\end{document}
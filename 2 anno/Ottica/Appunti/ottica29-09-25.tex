\documentclass[a4paper, oneside]{article}
\usepackage{graphicx}
\usepackage{amsthm}
\usepackage{amsmath}
\usepackage{amssymb}
\usepackage[a4paper,
            bindingoffset=0.2in,
            left=2cm,
            right=2cm,
            top=2cm,
            bottom=2cm,
            footskip=.25in]{geometry}
\usepackage[italian]{babel}
\usepackage{pgfplots}
\usepackage{tabularx}
\usepackage{tikz}
\usepackage{wrapfig}
\usepackage{color}
\usepackage[d]{esvect}
\definecolor{page}{rgb}{0.129,0.157,0.212}
\pagecolor{page}
\color{white}
\graphicspath{ {./images/} }
\usetikzlibrary{shapes.geometric}
\usetikzlibrary{datavisualization}
\usetikzlibrary{datavisualization.formats.functions}
\usetikzlibrary{patterns}
\pgfplotsset{width=10cm,compat=1.9}

\title{Appunti di ottica}
\author{Tommaso Miliani}
\date{29-09-25}

\begin{document}
\newtheoremstyle{theoremEnv}
                {}          % Space above
                {}          % Space below
                {\slshape}  % Body font
                {}          % Indent amount
                {\bfseries} % Head font
                {.}         % Punctuation after head
                {\newline}         % Space after theorem head
                {}          % Theorem head spec
\theoremstyle{theoremEnv}

\newtheorem{definition}{Definizione}[section]
\newtheorem{theorem}{Teorema}[section]
\newtheorem{lemma}{Proposizione}[section]
\newtheorem{observation}{Osservazione}[section]
\newtheorem{corollary}{Corollario}[theorem]
\newtheorem{example}{Esempio}[section]

\maketitle

\section{Principio di funzionamento del telescopio}
\begin{wrapfigure}{r}{0.55\textwidth}
    \centering
    \caption{Osservare un corpo celeste ad occhio nudo}
    \begin{tikzpicture}
        \draw(0, 0) -- (8, 0);
        \draw(7, -0.5) -- (7, 0.5);
        \draw[->](0, 0) -- (0, 0.5);
        \draw[cyan](0, 0.5) -- (7, 0);
        \draw(2, 0.38) arc (160:180:1) node[midway, left] {$\alpha$};
    \end{tikzpicture}    
\end{wrapfigure}
Per poter osservare oggetti molto lontani da noi, si fa utilizzo di un apparato
ottico che prende il nome di \textbf{telescopio}: il telescopio è composto 
da due lenti convergenti poste ad una certa distanza tra di loro in modo tale da ingrandire 
l'oggetto osservato. Consideriamo allora un asse ottico molto lungo alle cui estremità
abbiamo l'oggetto da osservare e dall'altra l'occhio: possiamo tracciare
quindi i fasci di luce che partono dall'oggetto e intersecano il centro del cristallino
dell'occhio. Esiste allora un certo angolo $\alpha$ tra i due fasci di luce: a seconda
dell'angolo il corpo celeste avrà una certa dimensione sull'occhio. Se si avesse un
sistema ottico (ossia il telescopio) tra l'occhio e l'oggetto, potremmo posizionare due lenti in modo tale
che il fuoco primario della prima lente coincida con il fuoco secondario
della seconda lente; le lenti sono scelte in modo tale che $f_1 > f_2$.


\begin{wrapfigure}{r}{0.5\textwidth}
    \centering
    \caption{Osservare un corpo celeste tramite un telescopio}
    \begin{tikzpicture}[scale=1.4]
        \draw(0, 0) -- (5, 0);
        \draw[<->](2, -1) -- (2, 1);
        \draw[<->](4, -1) -- (4, 1);
        \draw[red](0, 0.75) -- (2, 0.75) -- (3.5, 0);
        \draw[cyan](0, 0.75) -- (2, 0) -- (3.5, -0.75) -- (4, -0.75) -- (5, 0.75);
        \draw[cyan](0, 1.5) -- (2, 0.75) -- (3.5, -0.75) -- (4.5, 0.75);
        \draw[cyan](3.5, 0) -- (3.5, -0.75) node[midway, left] {$h$};
        \draw[cyan](0.5, 0.5) arc (140:180:0.75) node[midway, left] {$\alpha$}; 
        \filldraw(4.5, 0) circle (1pt) node[anchor = south] {$F_2$};
        \filldraw(3.5, 0) circle (1pt)node[anchor = south] {$F_1$};
        \draw[cyan](4.75, 0.38) arc (60:0:0.4) node[midway, right] {$\beta$};
        \draw[very thick](2, 0) -- (3.5, 0);
        \draw[very thick, cyan](2, 0) -- (3.5, -0.75);
        \draw[very thick, cyan](3.5, 0) -- (3.5, -0.75);
        \draw[very thick](3.5, 0) -- (4, 0);
        \draw[very thick, cyan](3.5, -0.75) -- (4, 0);
    \end{tikzpicture}    
\end{wrapfigure}
Così la distanza tra le due lenti è esattamente $d = f_1 + f_2$. Adesso si deve rappresentare
i raggi che arrivano dal corpo celeste sulle due lenti convergenti. Analizzando ora il telescopio,
i raggi molto vicini all'asse ottico appaiono paralleli tra di loro. Possiamo considerare il
raggio blu che arriva con l'angolo $\alpha$: nel momento in cui pongo un sistema di lenti
davanti all'occhio, allora i raggi vicini a quello blu arriveranno sulla lente
primaria del telescopio venendo fatti convergere sotto l'asse ottico. Per studiare ora
cosa accade ai raggi del corpo celeste alla destra del fuoco primario della prima lente, è sufficiente immaginare
$h$ (ossia l'immagine del corpo celeste dopo la prima lente) come sorgente posta prima della seconda lente
del telescopio. Osservando dunque il disegno si vede che i raggi blu che venivano dall'estremità finale della
corpo celeste arrivano sul nostro occhio con un nuovo angolo $\beta$ più grande.  
Si ottengono le seguenti relazioni per i triangoli in figura:
\begin{gather*}
    \frac{h}{f_1} = \tan\alpha \\
    \frac{h}{f_2} = \tan\beta
\end{gather*}
Si ha l'utile relazione come il rapporto tra le tangenti:
\begin{align}
    \frac{\tan\beta}{\tan\alpha} = \frac{f_1}{f_2}
\end{align}
Se il rapporto tra le tangenti è maggiore di 1 allora
\begin{gather*}
    \beta > \alpha
\end{gather*}


\section{Il principio di funzionamento del microscopio}
\begin{wrapfigure}{r}{0.5\textwidth}
    \centering
    \caption{Schematizzazione del microscopio}
    \begin{tikzpicture}
        \draw(0, 0) -- (6, 0);
        \draw(0, 0) -- (0, 0.5) node[midway, left] {$h$};
        \draw[<->](1, -1) -- (1, 1);
        \draw[<->](4, -2) -- (4, 2);
        \draw[cyan](0, 0.5) -- (1, 0.5) -- (4, -1) -- (6, -1);
        \draw[cyan](0, 0.5) -- (1, 0) -- (4, -1.5) -- (6, -1);
        \draw[green](6, 0) -- (6, -1) node[midway, right] {$h'$};
        \draw[dashed](2, -2) -- (2, 2);
        \filldraw[cyan](1, 0) -- (1, 0.5) -- (2, 0);
        \filldraw[cyan](2, 0) -- (4, 0) -- (4, -1);
    \end{tikzpicture}    
\end{wrapfigure}
Quando si vuole osservare un oggetto molto piccolo, invece di usare una sola
lente di ingrandimento, possiamo utilizzare un sistema di lenti
con l'oggetto posto nel fuoco secondario della prima lente convergente in modo
tale che $p=f_1$. Scelgo allora una seconda lente a focale lunga (l'inverso del telescopio)
così che la distanza tra le due lenti sia $d = f_1 + f_2$ e che quindi il fuoco
primario della prima lente coincida con il fuoco secondario della seconda lente.
Posso stimare geometricamente l'ingrandimento dell'immagine come
\begin{gather*}
    I = \frac{h'}{h} = \frac{f_2}{f_1}
\end{gather*}
Dato che $f_2 > f_1$, segue che $I > 1$ e dunque l'immagine sarà
ingrandita. Se volessi tenere conto del fatto che l'immagine sia ribaltata,
allora dovrei mettere un meno davanti al rapporto per tenere conto del ribaltamento.

\section*{Analisi e manipolazione della polarizzazione delle onde elettromagnetiche}
\section{Definizione di polarizzazione}
\begin{wrapfigure}{r}{0.4\textwidth}
    \centering
    \caption{}
    \begin{tikzpicture}
        \draw[->](0, 0) -- (4, 0) node[at end, below] {$x$};
        \draw[->](0, 0) --(0, 1) node[at end, left] {$z$};
        \draw[->](0, 0) -- (1, 1) node[at end, right] {$y$};
        \draw[->, green](0, 0) -- (0.45, 0.9) node[at end, above] {$\vv{E}$};
    \end{tikzpicture}    
\end{wrapfigure}
Si definisce polarizzazione la direzione del campo elettrico dell'onda elettromagnetica, si ricorda
che per una onda piana la polarizzazione giace in un piano che è perpendicolare
alla direzione di propagazione dell'onda $\hat{k}$. Il campo elettrico dell'onda sarà dato
in funzione sia della posizione che del tempo. Matematicamente il campo elettrico
si propagherà solamente sul piano $zy$ e dunque posso esprimere
\begin{gather*}
    \vv{E}(x, t) = E_{0z} \cos(kx - \omega t + \phi_z) \hat{z} + E_{0y}\cos(kx - \omega t + \phi_y) \hat{y}    
\end{gather*} 
Dove $\phi_z$ e $\phi_y$ sono le fasi del campo rispetto ai due assi. In questo caso
l'onda elettromagnetica si propaga lungo la direzione $\hat{x} \equiv \hat{k}$.

\subsection{Polarizzazione lineare}
\begin{wrapfigure}{r}{0.4\textwidth}
    \centering
    \caption{Campo elettrico sul piano $zy$, l'asse $x$ è uscente
    dal piano}
    \begin{tikzpicture}
        \draw[->](0, 0) -- (2, 0) node[at end, below] {$y$};
        \draw[->](0, 0) -- (0, 2) node[at end, left] {$z$};
        \draw[->](0, 0) -- (1.5, 1.5) node[at end, right] {$\vv{E}$ };
        \draw[dashed, thin](0, 1.5) -- (1.5, 1.5) node[at start, left] {$E_{0z}$};
        \draw[dashed, thin](1.5, 0) -- (1.5, 1.5) node[at start, below] {$E_{0y}$};
        \draw[<-](0.7, 0.7) arc (45:0:1) node[midway, right] {$\alpha$};
    \end{tikzpicture}    
\end{wrapfigure}
La polarizzazione lineare consiste nella polarizzazione tra due onde con
la stessa fase $\phi_y = \phi_z$. Le due componenti $z, y$ del campo elettrico
oscillano in fase sia spazialmente che temporalmente e quindi la direzione
del campo elettrico totale rimane costante.  Posso esprimere allora la tangente
dell'angolo in funzione del tempo come
\begin{gather*}
    \tan\alpha(t) = \frac{E_{0z} \cos(kx - \omega t + \phi_z)}{E_{0y}\cos(kx - \omega t + \phi_y)} = \frac{E_{0z}}{E_{0y}}
\end{gather*}
La seconda uguaglianza vale perché le fasi sono le stesse:

\subsection{}
E' un altro caso particolare per cui $E_{0y} = E_{0z}$ ma nel caso in
cui le onde siano sfasate di $\phi_y = \phi_z \pm \frac{\pi}{2}$. Vado a vedere
ancora la tangente di alfa in funzione del tempo:
\begin{gather*}
    \tan\alpha(t) = \frac{E_{0z} \cos(kx - \omega t + \phi_z)}{E_{0y}\cos(kx - \omega t + \phi_y)} = \frac{\cos(kx - \omega t + \phi_z)}{\cos(kx - \omega t + \phi_z \pm \frac{\pi}{2})} = \frac{ \cos(kx - \omega t + \phi_z)}{\mp\sin(kx - \omega t + \phi_z)}
\end{gather*}
Per le proprietà trigonometriche del coseno, quando si introduce uno sfasamento di $\frac{\pi}{2}$, il coseno
si trasforma nel seno dell'angolo $\phi_z$ quindi vale la seconda uguaglianza.
Se chiamassi l'argomento dentro al seno e al coseno $\theta$, otterrei che
\begin{gather*}
    \tan \theta = \frac{\sin \theta}{\cos\theta} = \frac{\cos(\frac{\pi}{2} - \theta)}{\sin(\frac{\pi}{2} - \theta)} = \frac{\cos(\theta - \frac{\pi}{2})}{-\sin(\theta - \frac{\pi}{2})} = - \frac{1}{\tan(\theta - \frac{\pi}{2})}
\end{gather*}
Allora con questi passaggi trigonometrici si ottiene
\begin{gather*}
    \tan\alpha (t) = \pm \tan \left(kx - \omega t + \phi_z + \frac{\pi}{2}\right)
\end{gather*}
Allora si ha che l'angolo $\alpha$ varia nel tempo, con il simbolo $\pm$ in base alla variazione
di fase che sia positiva o negativa secondo la seguente legge
\begin{align}
    \alpha (t) = \pm\left(kx - \omega t + \phi_z + \frac{\pi}{2}\right)
\end{align}
Le due polarizzazioni si chiameranno \textbf{circolari}:
\begin{align*}
    +:& \ \alpha (t) = \cos t - \omega t \\
    -:& \  \alpha (t) = \cos t + \omega t 
\end{align*} 
Nel primo caso gira in senso orario e dunque prende il nome di polarizzazione circolare sinistra 
e nel caso meno, girando in senso antiorario, prenderà il nome di polarizzazione circolare destra. 

\end{document}
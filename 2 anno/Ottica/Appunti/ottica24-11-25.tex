\documentclass[a4paper, oneside]{article}
\usepackage{graphicx}
\usepackage{amsthm}
\usepackage{amsmath}
\usepackage{amssymb}
\usepackage[a4paper,
            bindingoffset=0.2in,
            left=2cm,
            right=2cm,
            top=2cm,
            bottom=2cm,
            footskip=.25in]{geometry}
\usepackage[italian]{babel}
\usepackage{pgfplots}
\usepackage{tabularx}
\usepackage{tikz}
\usepackage{wrapfig}
\usepackage{color}
\usepackage[d]{esvect}
\usepackage{chemfig}
\usepackage{mhchem}
\definecolor{page}{rgb}{0.129,0.157,0.212}
\pagecolor{page}
\color{white}
\graphicspath{ {./images/} }
\usetikzlibrary{shapes.geometric}
\usetikzlibrary{datavisualization}
\usetikzlibrary{datavisualization.formats.functions}
\usetikzlibrary{patterns}
\pgfplotsset{width=10cm,compat=1.18}

\title{Appunti di Ottica}
\author{Tommaso Miliani}
\date{24-11-25}

\begin{document}
\newtheoremstyle{theoremEnv}
                {}          % Space above
                {}          % Space below
                {\slshape}  % Body font
                {}          % Indent amount
                {\bfseries} % Head font
                {.}         % Punctuation after head
                {\newline}  % Space after theorem head
                {}          % Theorem head spec
\theoremstyle{theoremEnv}

\newtheorem{definition}{Definizione}[section]
\newtheorem{theorem}{Teorema}[section]
\newtheorem{lemma}{Proposizione}[section]
\newtheorem{observation}{Osservazione}[section]
\newtheorem{corollary}{Corollario}[theorem]
\newtheorem{example}{Esempio}[section]
\newtheorem{remark}{Enunciato}[section]

\maketitle

\section*{Fine esperienza interferenza}
\section{Sorgente luminosa non monocromatica}
Se si avesse una sorgente di luce non monocromatica, 
l'intensità in uscita dall'interferometro di Machelson
è data dalla seguente (con le formule di bisezione):
\begin{gather*}
    I(\Delta L)  = I_0 \sin^{2}\left(\frac{\Delta L \omega_0}{c}\right)
\end{gather*}
E si riscrive nel modo più canonico:
\begin{gather*}
    I_0\sin^{2}\left(\pi\frac{\Delta L}{\frac{\lambda_0}{2}}\right)
\end{gather*}
L'uscita dipende dal seno al quadrato perché l'onda riflessa acquisisce
un termine di fase negativo, dunque il campo elettrico, e di conseguenza
l'intensità, sono zero. Per $\Delta L << 1$ è come se la sorgente fosse monocromatica.
Quando invece
\begin{gather*}
    \frac{\Delta L\Delta \omega}{c} \approxeq 1
\end{gather*}
Allora questa lunghezza $l_c = \Delta L$ per cui è uguale ad uno prende 
il nome di lunghezza di coerenza: questa fa sì che il coseno sia molto 
piccolo e dunque si ottiene che l'ampiezza dell'oscillazione tende
all'asintoto $\frac{I_0}{2}$. Dunque si sceglie questa lunghezza di coerenza
in modo tale che
\begin{gather*}
    I(l_c) = \frac{I_0}{2}\left(1 + e^{-1}\right)
\end{gather*}
Ossia che si trovi un massimo per la funzione. Per stimare la lunghezza
di coerenza devo anche dare una stima per  $\Delta \omega$: dunque si può relazionare
questa grandezza con la lunghezza d'onda 
\begin{gather*}
    \frac{\Delta \omega}{\omega_0} = \frac{\Delta \lambda}{\lambda_0} \ \Longrightarrow \ \Delta \omega = \frac{\Delta \lambda \omega_0}{\lambda_0} 
\end{gather*}
Allora
\begin{gather*}
    l_c = \frac{c\lambda_0}{\Delta \lambda} \frac{\lambda_0}{2\pi c}
\end{gather*}
Ponendo $\lambda_0 \approx 550 \ nm$ e $\Delta \lambda \approx 150 \ nm$, si ha che
$l_c \approx 350 \ nm$. È fondamentale non toccare lo specchio con la ceramica
piezoelettrica in quanto se si sposta dalla condizione di $\Delta L = 0$ non si vedrà più niente.
Sperimentalmente per trovare $l_c$, si deve misurare $\lambda_0$ e poi si utilizza
il programma in Mathematica. 
\begin{itemize}
    \item Si ricava una traccia per $I(\Delta L)$ attorno a $\Delta L \approx 0$. 
    \item Si determina la lunghezza d'onda $\lambda_0$ 
    \item Si identificano il numero di oscillazioni $n$ affinché si ottenga il valore massimo
    dell'oscillazione sia da $I_0$ a $\frac{I_0}{2}(1 + e^{-1})$ e moltiplico
    per $\frac{\lambda_0}{2}$ per trovare $l_c$.
    \item Si trova $\Delta \lambda_0$ con la formula per la lunghezza di coerenza.  
\end{itemize}
La luce è filtrata da un filtro verde in modo tale che la lunghezza d'onda sia centrata
intorno a $\lambda_0 \approx 552 \pm 2 \ nm$ e che ha un intervallo di 
misura di circa $\Delta \lambda \approx 5 \pm 2 \ nm$. In questo modo il rilevatore
vede una lunghezza d'onda molto stretta e dunque la lunghezza di coerenza aumenta. 


\section*{Approfondimenti dell'ottica}
\section{Interferenza da film sottili}
Consideriamo cosa succede ad un onda piana che attraversa
un sottile strato di materiale dielettrico a facce piane e parallele
con indice di rifrazione $n > 1$. 
\subsection{Legge di Fresnel}
La legge di Fresnel sulla riflessione e trasmissione ad una interfaccia
stabilisce la quantità di luce riflessa e quella trasmessa: 
\begin{align}
    r = \frac{n_1 - n_2}{n_1+ n_2}
\end{align}
Questo coefficiente stabilisce che il campo riflesso e quello trasmesso
è dato dalle seguenti relazioni
\begin{gather*}
    E_r = r E_{in} \qquad E_t = t E_{in}
\end{gather*}
Già da qui si scopre che nel caso tipico Aria-Vetro, si scopre che
$r = -0.2$ e dunque si può quantificare l'intensità e le relazioni tra 
l'intensità riflessa e quella trasmessa:
\begin{gather*}
    I_r = 0.04 I_{in}
\end{gather*} 
Ossia si ha una riflessione dell'ordine del $4\%$. Posso dunque 
indicare con $R$ il coefficiente di riflessione dell'intensità luminosa;
per le regole di conservazione dell'energia $T = 1 - R$, dunque
\begin{gather*}
    I_T = TI_{in}
\end{gather*}
Allora posso dire che
\begin{gather*}
    T = 1 - R = 1 - r^{2} \qquad t = \sqrt{1 - r^{2}} 
\end{gather*}
Ottenendo allora la relazione generale che lega il coefficiente di trasmissione 
a quello di riflessione: nel caso specifico, dato $R = 0.04$, allora
si può dire che $t = 0.98$: dunque il campo elettrico trasmesso 
nel dielettrico è ben il $98\%$ del campo totale. 

\subsection{Applicazione della legge di Fresnel}
\begin{wrapfigure}{r}{0.4\textwidth}
    \centering
    \caption{Interfaccia}
    \begin{tikzpicture}
        \draw(0, 0) -- (4, 0);
        \draw(0, -0.5) -- (4, -0.5);
        \node at (-0.25, 0.5) {$n_1$};
        \node at (-0.25, -0.25) {$n_2$};
        \node at (-0.25, -0.75) {$n_1$};
        \draw(1.5, 2) -- (2, 0) -- (2.5, 2);
        \draw[thick, ->](1.5, 2) -- (1.75, 1) node[at end, left] {$\vv{E_0} $};
        \draw[dashed] (2, 0) -- (2, 2);
        \draw(2, 0) -- (2.4, -0.5) -- (2.8, 0) -- (3.3, 2);
        \draw[red, thick](2, 0) -- (2.5, 2);
        \draw[cyan, thick](2.8, 0) -- (3.3, 2);
        \draw[dashed](2, 0) -- (2, -0.5);
        \draw (2, -0.25) arc (-90:-70:0.5) node[at start, left] {$\theta'$};
        \draw(2.4, -0.5) -- (2.65, -1);
        \draw(2.8, 0) -- (3.2, -0.5) -- (3.45, -1);
        \draw(3.2, -0.5) -- (3.6, 0);
        \draw[green, thick](3.6, 0) -- (4.1, 2);
    \end{tikzpicture}    
\end{wrapfigure}
Presa l'interfaccia dielettrica con indice di rifrazione $n_2$
immerso nell'aria, supponendo che i raggi di luce incidano con un angolo
$\theta << 1$, sull'interfaccia aria-vetro si ha una prima riflessione e
trasmissione: per cui si ottiene che una parte del fascio di luce 
viene trasmessa e una parte riflessa, così accade per tutti i fasci riflessi all'interno del 
vetro che provano ad uscire. Dunque, considerato che la luce che viene riflessa 
all'interno del vetro, devo aggiungere un ritardo di fase per tenere conto 
di questo cammino extra che è stato compiuto. Il cammino che compie in più il campo che esce dal vetro 
è dato da
\begin{gather*}
    l = 2\frac{d}{\cos\theta'}
\end{gather*}
Il fascio rosso dunque ha modulo del campo:
\begin{gather*}
    -|r|E_0 \cos(kx - \omega t)
\end{gather*}
Il fascio rosso invece ha come modulo (dato dalla differenza del cammino):
\begin{gather*}
    E_0 t^{2}r\cos\left(kx - \omega t + 2\frac{kn_2 d}{\cos\theta'}\right)
\end{gather*}
Mentre il fascio verde è dato da
\begin{gather*}
    E_0t^{2}r^{3}\cos\left(kx - \omega t + 4\frac{kn_2d}{\cos\theta'}\right)
\end{gather*}
Dunque per studiare riflessione e trasmissione basterà studiare semplicemente i primi 
due fasci in quanto dal terzo in poi i contributi sono molto piccoli e dunque valgono i
risultati che si sono ottenuti nell'interferometro di Machelson. Il motivo per il quale
le bolle di sapone risultano colorate (dei colori dello spettro luminoso) è dato dal
fatto che la luce che esce e rientra nel sapone genera interferenza. Tutte le volte che 
si usa un laser si deve considerare che ogni volta che il fascio di laser passa
attraverso i componenti ottici si perde il $4\%$ dell'intensità. 

\begin{wrapfigure}{r}{0.4\textwidth}
    \centering
    \caption{Il coating delle lenti a bassa riflettività}
    \begin{tikzpicture}
        \draw(0, 0) -- (4, 0);
        \draw(0, -1.5) -- (4, -1.5);
        \filldraw[cyan, opacity = 0.3](0, 0) rectangle (4, -0.25);
        \node at (4.6, -0.15) {$n_1 > 1.5$};
        \node at (2, -0.75) {$n_{vetro} = 1.5$};
        \draw(2, 1) -- (2.5, 0) -- (2.75, -0.25) -- (3, 0) -- (3.5, 1);
        \draw(2.5, 0) -- (3, 1);
    \end{tikzpicture}    
\end{wrapfigure}
Per ovviare al problema
si può depositare sulla superficie delle lenti un materiale dielettrico molto sottile 
con un indice di rifrazione $n_1 > n_2$, ossia maggiore dell'indice di rifrazione
del vetro. Lo spessore del \textbf{coating} fatto da questo materiale dielettrico lo scelgo 
in modo tale che tra i due fasci riflessi ci sia interferenza distruttiva.
Dunque l'intensità di luce che viene riflessa dopo che entra nel dielettrico
(che deve essere in interferenza con quella inizialmente riflessa) è data da
\begin{gather*}
    I_r = I_0 t^{4}r^{6}
\end{gather*}
E dunque si perde solamente lo $0.06\%$ della luce. 

\subsection{Coating ad alta riflettività}
\begin{wrapfigure}{r}{0.4\textwidth}
    \centering
    \caption{Coating ad alta riflettività}
    \begin{tikzpicture}
        \draw(0, 0) -- (4, 0) node[at start, above] {$n_{aria}$};
        \draw (0, -1) -- (4, -1);
        \draw[<->](0, 0) -- (0, -1) node[midway, left] {$\frac{\lambda}{4n_1}$};
        \draw(0, -2.5) -- (4, -2.5);
        \draw[<->](0, -1) -- (0, -2.5) node[midway, left] {$\frac{\lambda}{4n_2}$};
        \node at (4.2, -0.5) {$n_1$};
        \node at (4.8, -1.75) {$1 < n_2 < n_1$};
        \node at (4.2, -3) {$n_1$};
        \draw(0, -3.5) -- (4, -3.5);
        \draw(1.5, 0.5) -- (2, 0) -- (2.5, 0.5) node[at end, right] {1};
        \draw(2, 0) -- (2.5, -1) -- (3, 0) -- (3.5, 0.5) node[at end, right] {2};
        \draw(2.5, -1) -- (3, -2.5) -- (3.5, -1) -- (4, 0) -- (4.5, 0.5) node[at end, right] {3};
    \end{tikzpicture}    
\end{wrapfigure}
Si considera un coating ad alta riflettività: si può sfruttare la legge di Snell, e
di Fresnel e l'interferenza costruttiva in riflessione per creare specchi 
ad elevata riflettività. Si utilizzano film sottili alternati a vetro in modo tale che
si possa sfruttare l'interferenza costruttiva. Si definiscono allora le proprietà dei seguenti 
raggi luminosi (quando ovviamente $\theta << 1$):
\begin{enumerate}
    \item Ha uno shift di $\pi$ dovuto alla riflessione dovuta all'interfaccia 
    aria-mezzo dielettrico con $n_{aria} < n_{d}$. 
    \item Il secondo raggio ha un ritardo di fase di $\pi$ dovuto alla doppia propagazione
    all'interno del dielettrico di spessore di $\frac{\lambda}{4n_1}$.
    \item Il terzo fascio ha un ritardo di fase di $\pi$ rispetto al fascio 2 dovuto alla propagazione
    all'interno del film spesso $\frac{\lambda}{4n_2}$ ed un ritardo di fase di $\pi$ dovuto
    alla riflessione $n_2 \to n_1$. 
\end{enumerate}
Anche questo coating deve necessariamente essere utilizzato con una certa
lunghezza d'onda e con un determinato angolo di incidenza. 

\end{document}
\documentclass[a4paper, oneside]{article}
\usepackage{graphicx}
\usepackage{amsthm}
\usepackage{amsmath}
\usepackage{amssymb}
\usepackage[a4paper,
            bindingoffset=0.2in,
            left=2cm,
            right=2cm,
            top=2cm,
            bottom=2cm,
            footskip=.25in]{geometry}
\usepackage[italian]{babel}
\usepackage{pgfplots}
\usepackage{tabularx}
\usepackage{tikz}
\usepackage{wrapfig}
\usepackage{color}
\definecolor{page}{rgb}{0.129,0.157,0.212}
\pagecolor{page}
\color{white}
\graphicspath{ {./images/} }
\usetikzlibrary{shapes.geometric}
\usetikzlibrary{datavisualization}
\usetikzlibrary{datavisualization.formats.functions}
\pgfplotsset{width=10cm,compat=1.9}

\title{Appunti di ottica}
\author{Tommaso Miliani}
\date{22-09-25}

\begin{document}
\newtheoremstyle{theoremEnv}
                {}          % Space above
                {}          % Space below
                {\slshape}  % Body font
                {}          % Indent amount
                {\bfseries} % Head font
                {.}         % Punctuation after head
                {\newline}         % Space after theorem head
                {}          % Theorem head spec
\theoremstyle{theoremEnv}

\newtheorem{definition}{Definizione}[section]
\newtheorem{theorem}{Teorema}[section]
\newtheorem{lemma}{Proposizione}[section]
\newtheorem{observation}{Osservazione}[section]
\newtheorem{corollary}{Corollario}[theorem]
\newtheorem{example}{Esempio}[section]

\maketitle

\section{Le lenti sottili}
\begin{wrapfigure}{r}{0.4\textwidth}
    \centering
    \caption{La lente sottile}
    \begin{tikzpicture}
        \draw[dashed](0, 0) -- (5, 0);
        \draw(0, 0) -- (1.15, 1.15);
        \draw(1.5, 1.5) arc (130:230:2);
        \filldraw(0, 0) node[anchor = east] {$P$};
        \draw(1.7, 1.5) arc (160:200:4.5);
        \draw[<->, thick, red] (0.8, 0) -- (1.4, 0) node[midway, above] {$d$};
        \draw(1.15, 1.15) -- (1.6, 1.4) -- (4.5, 0);
        \draw[<->, cyan](0, -0.1) -- (0.75, -0.1) node[midway, below] {$p$};
        \draw[<->, cyan](1.45, -0.1) -- (4.5, -0.1) node[midway, below] {$q$};
    \end{tikzpicture}    
\end{wrapfigure}
Possiamo ora andare ad analizzare il comportamento di una lente
sottile e non di una sfera di vetro come si era trattato prima.
Data sempre una sorgente posta in $P$ e come $p$ la distanza
tra la sorgente e la prima interfaccia; posso chiamare $d$ come
lo spessore della lente e $q$ come la distanza tra il punto di arrivo
sull'orizzontale del fascio di luce e la seconda interfaccia.
$R_1$ e $R_2$ sono $>0$ se il centro delle curvature si trova
a destra della superficie sferica. Si ottiene da questo modello
la seguente relazione
\begin{gather*}
    \frac{1}{p} + \frac{1}{q} = \frac{n_2 - n_1}{n_1}\left(\frac{1}{R_1} + \frac{1}{R_2}\right)+ \frac{n_2 d}{p(p - d)}
\end{gather*}
Data l'ipotesi di lente sottile $d$ è molto piccolo: in questo modo posso semplificare
la formula in questione con la seguente:
\begin{align}
    \frac{1}{p} +\frac{1}{q} = \frac{n_2 - n_1}{n_1} \left(\frac{1}{R_1} + \frac{1}{R_2}\right) = \frac{1}{f}
\end{align} 
Dove $f$ è chiamata \textbf{lunghezza focale} e questa relazione si chiama
\textbf{formula del costruttore di lenti} poiché per realizzare una lente
con una certa focale devo sapere i raggi di curvatura delle interfacce e il loro
coefficiente di rifrazione. Si ottiene anche la legge delle lenti sottili che dovremmo verificare in laboratorio
ossia il primo membro uguale al terzo. 

\subsection{La lente convergente}
\begin{wrapfigure}{r}{0.4\textwidth}
    \centering
    \caption{Caso A}
    \begin{tikzpicture}
        \draw(-2.5, 0) -- (2.5, 0);
        \draw[<->](0, -2) -- (0, 2) node[at end, above] {$f > 0$};
        \draw[->](-2, 2) -- (-1, 2);
        \draw[->](-2, 1) -- (-1, 1);
        \draw[->](-2, -1) -- (-1, -1);       
        \draw[->](-2, -2) -- (-1, -2);
        \filldraw (-1, 2 ) circle (0pt) node[anchor = south] {$p \to \infty $};
        \filldraw(1.5, 0) circle (1pt) node[anchor = south] {$Q$};
        \draw(-1, 2) -- (0, 2);
        \draw(-1, 1) -- (0, 1);
        \draw(-1, -1) -- (0, -1);
        \draw(-1, -2) -- (0, -2);
        \draw(0, 2) -- (1.5, 0);
        \draw(0, 1) -- (1.5, 0);
        \draw(0, -1) -- (1.5, 0);
        \draw(0, -2) -- (1.5, 0);
    \end{tikzpicture}    
\end{wrapfigure}
Nel caso di $f > 0$ si parla di \textbf{lente convergente}. Possiamo allora studiare 
i casi limite per capire il significato fisico di $f$. Una lente convergente
è una lente sottile e possiamo distinguere diversi casi.
\begin{itemize}
    \item Il primo caso è il caso in cui la distanza rispetto alla lente è infinita 
    e ha la caratteristica in cui tutti i raggi convergono nel medesimo punto $Q$.
    \item Il secondo caso si ha quando $p$ sia molto piccoli (siamo sempre nell'ipotesi parassiale)
    allora il termine $q$ nella formula si annulla e la lente mi fa convergere i fasci
    luminosi all'infinito. 
    \item Il terzo caso è quello della configurazione $2f - 2f$, ossia la configurazione
    in cui $p = 2f$ 
    \begin{gather*}
        \frac{1}{2f} + \frac{1}{q} = \frac{1}{f} \ \Longrightarrow \ \frac{1}{q} = \frac{1}{f} - \frac{1}{2f} \ \Longrightarrow \ \frac{1}{q} = \frac{1}{2f}
    \end{gather*}
    E quindi anche $q = 2f$, in questo modo i raggi che partono da $P$
    convergono tutti sul punto simmetrico rispetto alla lente $Q$.
    Per una lente \textbf{convergente} i punti a distanza $f$ dalla lente (supponendo sempre che
    la sorgente sia a sinistra rispetto alla lente) per cui il punto a destra della lente si chiamerà
    \textbf{fuoco primario} della lente convergente, mentre il fuoco a sinistra della lente
    è chiamato \textbf{fuoco secondario}. 
    \item Posso considerare infine un ultima configurazione:
    se $p < f$ posso ottenere intanto l'espressione per $q$:
    \begin{gather*}
        \frac{1}{q} = \frac{p - f}{fp} \ \Longrightarrow \ \frac{fp}{p - f} = q
    \end{gather*}
    La lente non riesce a far convergere i fasci di luce nel fuoco
    ma riesce solo a defletterli. Per cui la direzione dei raggi deflessi
    mi individua un punto in cui convergono le loro direzioni $Q$ e la distanza
    $q$ è allora negativa.
    \end{itemize}
Se il punto focale della sorgente è sinistra della lente si dice che
    si ha una \textbf{immagine reale della sorgente}, altrimenti se il punto
    focale è a sinistra della lente si parla di \textbf{immagine virtuale della sorgente}.
    L'immagine è definita come il luogo dei punti dove si intersecano fisicamente
    i raggi provenienti da fuori della lente; nel caso dell'immagine virtuale sono i
    prolungamenti.

\subsection{La lente divergente}
\begin{wrapfigure}{r}{0.4\textwidth}
    \centering
    \caption{La lente divergente}
    \begin{tikzpicture}
        \draw(-3, 0) -- (3, 0);
        \draw[>-<](0, 2) -- (0, -2);
        \draw[->](-2, 1) -- (-1, 1);
        \draw[->](-2, -1) -- (-1, -1);
        \draw(-1, 1) -- (0, 1);
        \draw(-1, -1) -- (0, -1);
        \draw[->](0, 1) -- (2, 2);
        \draw[->](0, -1) -- (2, -2);
        \draw[red](0, 1) -- (-2, 0);
        \draw[red](0, -1) -- (-2, 0);
        \filldraw[red] (-2, 0) node[anchor = south] {$Q$};
    \end{tikzpicture}    
\end{wrapfigure}
Le lenti con $f < 0$ si chiamano \textbf{divergenti}. La legge delle lenti
sottili è sempre valida e studiamo solo un caso per la lente divergente con
sorgente all'infinito. La legge delle lenti sottili mi mette in relazione
\begin{gather*}
    \frac{1}{q} = \frac{1}{f} \qquad f < 0 \\ \Longrightarrow \ q=f
\end{gather*}
Allora la mia lente produrrà delle immagini virtuali sulla sinistra della lente
sul fuoco secondario: allora i fuochi si invertono e il fuoco secondario
diventerà il fuoco primario e viceversa. Non esistono invece delle immagini reali create
da questa lente in quanto i raggi che giungono dall'infinito
vengono sempre deflessi (ossia toccano la lente e vengono rifratti a differenza della
lente convergente che invece fa convergere i fasci luminosi in un medesimo punto $Q$).



\section{Il problema della formazione delle immagini}
\begin{wrapfigure}{r}{0.4\textwidth}
    \centering
    \caption{}
    \begin{tikzpicture}        
        \filldraw[purple] (0, 0) -- (1.7, -0.7) -- (0, -0.7);
        \filldraw[purple] (0, 0) -- (0, 1) -- (-2.5, 1);
        \filldraw[red](0, 0) -- (0, 1) -- (1, 0);
        \filldraw[red](1, 0) -- (1.7, -0.7) -- (1.7, 0);
        \filldraw[green](-1, 0) -- (0, 0) -- (0, -0.7);
        \filldraw[green](-1, 0) -- (-2.5, 0) -- (-2.5, 1);
        \draw(-3, 0) -- (3, 0);
        \draw[<->](0, -2) -- (0, 2);
        \filldraw(1, 0) circle(1pt) node[anchor = north] {$F$};
        \filldraw(-1, 0) circle (1pt) node[anchor = north] {$F'$};
        \draw[->](-2.5, 0) -- (-2.5, 1) node[midway, left] {$h$};
        \filldraw(-2.5, 1) circle (1pt) node[anchor = south] {$P'$};
        \draw(-2.5, 1) -- (0, 1);
        \draw(0, 1) -- (2, -1) node[at end, right] {$1$}; 
        \draw(-2.5, 1) -- (0, -0.7);
        \draw(0, -0.7) -- (2, -0.7);
        \filldraw(1.7, -0.7) circle (1pt) node[anchor = north east] {$Q'$};
        \filldraw(-2.5, 0) circle (1pt) node[anchor = north] {$P$};
        \filldraw(1.5, 0) circle (1pt) node[anchor = south] {$Q$};
        \draw[<->](1.7, 0) -- (1.7, -0.7) node[midway, right] {$h'$};
        \draw(-2.5, 1) -- (1.7, -0.7);
    \end{tikzpicture}    
\end{wrapfigure}
Studiamo ora una sorgente $P'$ posta fuori dalla'asse ottico per
studiare il problema della formazione delle immagini da parte di una lente. 
Il punto $P'$ sta quindi ad una certa distanza rispetto all'asse ottico. Sappiamo
che tutti i raggi che vengono dall'infinito convergono sul fuoco: il raggio $1$
deve passare per il fuoco primario. Il raggio $2$ che invece è sparato da $P'$ passando per
il fuoco $F'$ deve necessariamente diventare perpendicolare alla lente. Il punto $Q$
dipende dalla distanza della sorgente rispetto alla lente (il punto $Q$ coincide solo con $F$ quando
i raggi vengono dall'infinito). Posso considerare quindi $q$ come la distanza dal punto $Q$
e $q'$ come la distanza da $Q'$. Posso considerare i due triangoli simili rossi e vedere che vale la seguente
\begin{gather*}
    \frac{q' - f}{f} = \frac{h'}{h}
\end{gather*}
Per i triangoli verdi invece si ottiene la seguente relazione:
\begin{gather*}
    \frac{p - f}{f} = \frac{h}{h'}
\end{gather*}
Se le pongo uguali le espressioni per il rapporto di $h$ e $h'$ allora si
ha
\begin{gather*}
    \frac{q' - f}{f} = \frac{f}{p - f} \ \Longrightarrow \ q' - f = \frac{f^{2} }{p - f} \ \Longrightarrow \ q' = \frac{fp}{p - f}
\end{gather*}
Si può allora arrangiare l'equazione come
\begin{gather*}
    \frac{1}{q'} = \frac{p - f}{fp'} = \frac{1}{f} - \frac{1}{p} = \frac{1}{q}
\end{gather*}
Allora ho dimostrato che $q = q'$ e dunque i punti $Q$ e $Q'$ giacciono
sullo stesso piano perpendicolare all'asse ottico: anche per immagini estese
questa relazione vale ed è proprio come vede l'occhio umano.
I triangoli simili viola si ottengono perché il fascio di luce che viene da $P'$ 
e passa per il centro della lente giunge in $Q'$ in quanto non c'è alcuna
deflessione nel caso di una lente sottile. 
\begin{gather*}
    \frac{p -f}{f} = \frac{h}{h'}
\end{gather*}
Se si moltiplica per $\frac{p}{p}$ allora si ottiene la formula dell'\textbf{ingrandimento}:
\begin{align}
    \frac{h'}{h} = -\frac{q}{p}
\end{align}
Il segno meno mi ricorda che l'immagine si ribalta rispetto alla sorgente
nel caso in cui sia $p$ che $q$ siano positivi.

\subsection{Sbagliare il piano d'immagine}
\begin{wrapfigure}{r}{0.4\textwidth}
    \centering
    \caption{}
    \begin{tikzpicture}
        \draw(-3, 0) -- (3, 0);
        \draw[<->](0, -2) -- (0, 2);
        \draw(-2, 1) -- (0, -1);
        \draw(-2, 1) -- (0, 1);
        \draw(0, 1) -- (2, -1);
        \draw(0, -1) -- (2, -1);
        \draw(-2, 1) -- (2, -1);
        \filldraw(1, 0) circle (1pt);
        \filldraw(-1, 0) circle (1pt);
        \draw[orange, thick](0, 1) -- (0, -1) node[midway, anchor = south west] {$D$};
        \draw(2.5, 2) -- (2.5, -2);
        \draw(2, -1) -- (2.5, -1.5);
        \draw(2, -1) -- (2.5, -1.25);
        \draw(2, -1) -- (2.5, -1);
        \draw[thick, orange] (2.5, -1) -- (2.5, -1.5) node[midway, right] {$d$};
        \draw[dashed] (2, 2) -- (2, -2);
        \filldraw(2.25, 2) node[anchor = south] {$\delta x$};
        \draw[<->](0.1, 1) -- (1.9, 1) node[midway, above] {$q$};
    \end{tikzpicture}    
\end{wrapfigure}
Quando si fa una foto non riesco a mettere a fuoco tutti gli
oggetti di una scena ma devo concentrarmi su oggetti che hanno una certa distanza dal
piano immagine per poter essere rappresentati sul questo piano. Se sbagliassi il piano dell'immagine, 
potrei chiamare la distanza del punto di formazione dell'immagine dalla lente
e $D$ come la dimensione della lente; analogamente posso chiamare $d$ la
dimensione dell'immagine attraverso la lente. Posso anche avere un piano
su cui giace $d$ (ossia il piano sbagliato dell'immagine perché
magari ho sbagliato a mettere a fuoco l'immagine).
L'unico modo che ho per mettere a fuoco l'immagine ora è
ridurre la dimensione della lente 
\begin{gather*}
    \frac{d}{D} = \frac{\delta x}{q} \ \Longrightarrow \ d = \frac{\delta x}{q}D
\end{gather*}
Dove $\delta x$ è la distanza tra il piano dell'immagine "giusto" e quello
sbagliato su cui c'è $d$: in questo modo posso ridurre lo spread
dei raggi luminosi ed ottenere una immagine nitida. E' simile a quando si strizza l'occhio anche se in quel caso
si modifica anche il cristallino. Come è possibile che in alcune immagini sia sempre tutto a fuoco? 
Si può spingere al limite il concetto di messa a fuoco attraverso una \textbf{camera stenopeica}: ossia una camera
che mi permette di avere tutto a fuoco e tutto sullo stesso piano immagine attraverso una dimensione della
lente $D$ molto piccola. Il prezzo da pagare è che passa pochissima luce all'interno della
camera perdendo così tanta luminosità dell'immagine.

\section{La legge del costruttore di lenti e i vari tipi di lenti}
La legge del costruttore di lenti ci dice che
\begin{gather*}
    f = \frac{n_1}{n_2 - n_1}\left(\frac{R_1R_2}{R_2 - R_1}\right) 
\end{gather*}
Si possono ricavare tre tipi fondamentali di lenti da questa formula
\subsection{Lente piano convessa}
\begin{wrapfigure}{r}{0.2\textwidth}
    \centering
    \begin{tikzpicture}
        \draw(0, 0) -- (2, 0);
        \draw(1, 1) -- (1, -1);
        \draw(1, 1) arc(150:210:2);
        \draw(1.5, 0) -- (0.78, 0.55) node[midway, above] {$R$}; 
    \end{tikzpicture}    
\end{wrapfigure}
Questa lente ha una interfaccia con raggio di curvatura
normale e un raggio di curvatura assimilabile a infinito;
si ottiene allora la formulazione per la lunghezza focale 
dalla legge del costruttore di lenti
\begin{gather*}
    f = \frac{n_1}{n_2 - n_1} R
\end{gather*}
Più la lente è piccola e più la lunghezza focale è piccola e la lente ha
allora la capacità di far convergere molto i fasci luminosi. 

\subsection{Lente biconvessa}
\begin{wrapfigure}{r}{0.2\textwidth}
    \centering
    \begin{tikzpicture}
        \draw(-0.2, 1) arc(150:210:2);
        \draw(0.2, 1) arc(30:-30:2);
        \draw(-0.2, 1) -- (0.2, 1);
        \draw(-0.2, -1) -- (0.2, -1);
        \draw(-1, 0) -- (1, 0);
    \end{tikzpicture}    
\end{wrapfigure}
La lente biconvessa è una lente che non ha più una superficie piana ma ha due
superficie sferiche con $R_1 > 0$ e $R_2 < 0$. Nel caso in cui $|R_1| = |R_2| = |R|$ si
ha la seguente relazione:
\begin{gather*}
    f = \frac{n_1}{n_2 - n_1} \left(\frac{R(-R)}{-2R}\right)
\end{gather*}
Complessivamente la mia focale sarà data da 
\begin{gather*}
    f = \frac{n_1}{n_2 - n_1}\frac{R}{2} 
\end{gather*}
Il potere convergente è ora maggiore in quanto riesce a convergere
più vicino alla lente (dato che la focale è la metà dell'altra lente).


\subsection{La lente piano concava}
\begin{wrapfigure}{r}{0.2\textwidth}
    \centering
    \begin{tikzpicture}
        \draw(0, 0) -- (2, 0);
        \draw(1, -1) -- (2, -1);
        \draw(2, -1) -- (2, 1);
        \draw(2, 1) -- (1, 1);
        \draw(1, 1) arc(30:-30:2);
    \end{tikzpicture}    
\end{wrapfigure}
Con questo tipo do lente si ha che la focale  
è negativa e dunque ho realizzato una lente 
convessa che avrà in modulo la stessa lunghezza focale della
lente piano convessa ma con segno opposto
\begin{gather*}
    f = -\frac{n_1}{n_2 - n_1}R
\end{gather*}

\end{document}
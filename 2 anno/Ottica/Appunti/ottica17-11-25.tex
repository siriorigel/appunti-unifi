\documentclass[a4paper, oneside]{article}
\usepackage{graphicx}
\usepackage{amsthm}
\usepackage{amsmath}
\usepackage{amssymb}
\usepackage[a4paper,
            bindingoffset=0.2in,
            left=2cm,
            right=2cm,
            top=2cm,
            bottom=2cm,
            footskip=.25in]{geometry}
\usepackage[italian]{babel}
\usepackage{pgfplots}
\usepackage{tabularx}
\usepackage{tikz}
\usepackage{wrapfig}
\usepackage{color}
\usepackage[d]{esvect}
\usepackage{chemfig}
\usepackage{mhchem}
\definecolor{page}{rgb}{0.129,0.157,0.212}
\pagecolor{page}
\color{white}
\graphicspath{ {./images/} }
\usetikzlibrary{shapes.geometric}
\usetikzlibrary{datavisualization}
\usetikzlibrary{datavisualization.formats.functions}
\usetikzlibrary{patterns}
\pgfplotsset{width=10cm,compat=1.18}

\title{Appunti di Ottica}
\author{Tommaso Miliani}
\date{17-11-25}

\begin{document}
\newtheoremstyle{theoremEnv}
                {}          % Space above
                {}          % Space below
                {\slshape}  % Body font
                {}          % Indent amount
                {\bfseries} % Head font
                {.}         % Punctuation after head
                {\newline}  % Space after theorem head
                {}          % Theorem head spec
\theoremstyle{theoremEnv}

\newtheorem{definition}{Definizione}[section]
\newtheorem{theorem}{Teorema}[section]
\newtheorem{lemma}{Proposizione}[section]
\newtheorem{observation}{Osservazione}[section]
\newtheorem{corollary}{Corollario}[theorem]
\newtheorem{example}{Esempio}[section]
\newtheorem{remark}{Enunciato}[section]

\maketitle

\section*{Telescopi}
\section{Principio di funzionamento di un telescopio}
Il telescopio è uno strumento ottico in grado di poter
osservare degli oggetti molto lontani dall'occhio umano. 
Dato che se volessi un ingrandimento dell'angolo, dovrei avere
che $f_1 > f_2$. I raggi di luce che vengono dalla sorgente luminosa hanno
un certo angolo $\alpha$ rispetto all'asse ottico. Dunque, il raggio che
viene dall'estremità della luna si proietta sul punto focale della 
lente del telescopio e dunque, con la seconda lente lente formerà un angolo
$\beta > \alpha$, ottenendo così un ingrandimento dell'immagine stessa. 
La dimensione dell'immagine si esprime come
\begin{gather*}
    f_1 \tan \alpha = f_2 \tan \beta
\end{gather*}  
Si conclude allora che
\begin{gather*}
    \frac{f_1}{f_2} = \frac{\beta}{\alpha} > 1
\end{gather*}
E dunque sul mio occhio la Luna ha una dimensione maggiore 
rispetto a quando non utilizzo un telescopio. 

\section{L'effetto della diffrazione}
Ci si deve immaginare che ci siano altri raggi 
che impattano sulla lente e che risultano deflessi di un certo angolo
rispetto ai raggi che arrivano dalla sorgente luminosa
che dipende dalla lunghezza d'onda. L'immagine in corrispondenza
del fuoco primario della prima lente ha un allargamento (spread), a causa
della diffrazione della prima lente pari a
\begin{gather*}
    \delta y_1 = f_1 \cdot \frac{\lambda}{D_1}
\end{gather*}
Tutti i dettagli dell'oggetto osservato non sono più definiti ma 
l'immagine tende dunque a sfocarsi. Posso pensare inoltre che esista 
anche uno spread angolare associato all'angolo che risulta sull'occhio:
\begin{gather*}
    \delta \beta_1 = f_1\frac{\lambda}{D_1} \frac{1}{f_2}
\end{gather*}
E dunque ottengo che
\begin{align}
    \delta y_1 = f_2 \delta \beta_1
\end{align}
Dunque sulla seconda lente 
\begin{gather*}
    \delta \beta_2 = \frac{\lambda}{D_2}
\end{gather*}
Dunque posso ottenere l'effetto combinato delle
due lenti sullo spread dei raggi luminosi come
\begin{gather*}
    \delta \beta = \frac{\lambda}{D_1} \frac{f_1}{f_2} + \frac{\lambda}{D_2}
\end{gather*}
Dove si è usato $\tan \beta_1 \approx \beta_1$ e $\tan \beta_2 \approx \beta_2$
secondo l'approssimazione parassiale. Inoltre posso trascurare il secondo termine
dello spread su $\beta$ in quanto è molto piccolo e considerare solamente il termine 
dovuto dal rapporto delle focali. Si ha inoltre che
\begin{gather*}
    f_1 \delta \alpha = f_2 \delta \beta \ \Longrightarrow \ \delta \beta = \frac{f_1}{f_2}\delta \alpha = \frac{\lambda}{D_1}
\end{gather*}

\section*{Esperienza diffrazione}
\section{Scopo dell'Esperienza}
Gli scopi dell'esperienza sono i seguenti:
\begin{itemize}
    \item Verifica delle legge di una onda piana da una fenditura 
    nel limite di campo lontano (legge di Fraunhofer);
    \begin{gather*}
        I(\theta) = I_0 sinc^{2}\left(\frac{KD}{2}\sin \theta\right)
    \end{gather*}
\end{itemize}

\section{L'apparato sperimentale}
Il laser che si utilizza è un laser a $668,66 \pm 0.01 \ nm$ che punta su
di un filtro in assorbimento variabile. Dopo il filtro c'è una
fenditura larga circa $30 - 40 \ \mu m$ che è posta ad una distanza
$x$ rispetto al rilevatore. Ogni posizione $y'$ sul rilevatore corrisponde ad un certo
angolo $\theta$ per il seno cardinale. Nel limite di Fraunhofer, ossia quando
$x >> D$, in più ci si mette nel limite in cui $x >> y'$ (che  è possibile
in quanto il rilevatore ha una dimensione di circa un centimetro). 
Dunque
\begin{gather*}
    \theta << 1 \ \Longrightarrow \ \sin \theta \approx \tan \theta \approx \theta
\end{gather*}
Il rilevatore è composto da $1032 \times 1032$ pixel ognuno di 
una dimensione di $d \approx 5 \mu m$. Ogni raggio di luce eccita gli elettroni
dei led e dunque misurare l'intensità incidente in base alla quantità di elettroni
incidenti. Dato che ogni pixel è in grado di restituire un valore di intensità
al massimo di $2^{8}$, bisogna regolare il filtro in modo tale che
nessun led possa saturare; se i pixel saturassero, non riuscirei a determinare
se l'intensità è effettivamente massima oppure il led arriva a saturazione
troppo presto. \\
Si utilizza un programma in Mathematica per poter fittare il seno cardinale 
in funzione degli indici dei pixel. La coordinata $y'$ di ogni pixel dipende
dalla larghezza del pixel e dal suo indice. Dunque l'intensità
diventerà la seguente funzione:
\begin{gather*}
    I(y') = I_0 sinc^{2}\left(\frac{KD}{2}\sin\theta\right) = I_0 sinc^{2}\left(\frac{KD}{2} \frac{d}{x}y\right)
\end{gather*}
Si introduce inoltre un parametro che mi possa centrare il seno cardinale
proprio sull'asse che congiunge il centro del rilevatore e la sorgente del 
fascio di luce:
\begin{gather*}
    I(y') = I_0 sinc^{2}\left(\frac{KD}{2}\sin\theta\right) = I_0 sinc^{2}\left(\frac{KD}{2} \frac{d}{x}(y - y_0)\right)
\end{gather*}
Dal fit non lineare del profilo di intensità misurato con la CCD ricavo
$B \pm \Delta B$, dove $B = \frac{KD}{2} \frac{d}{x}$. L'idea è ora
confrontare tale misura con il valore teorico nel quale sostituisco i valori 
di $k$ conoscendo $\lambda$ e le loro incertezze. I dati da trovare sono dunque:
\begin{itemize}
    \item $k \pm \Delta k$: si ricava conoscendo $\lambda  \pm \Delta \lambda$;
    \item $D \pm \Delta D$: $\left\{\begin{array}{l}
            40 \pm 2 \ \mu m \\
            30 \pm 2 \ \mu m 
        \end{array}\right.$.
    \item $d = 5.20 \pm 0.01 \ \mu m$.
    \item $x \pm \Delta x$: si misura con il calibro.  
\end{itemize} 
Infine confronto i $B_i \pm \Delta B_i$ sperimentali ottenuti
e li confronto con quello teorico verificando che siano consistenti per poter dimostrare la legge. 
I valori $B_i$ sono tutti diversi ognuno per ogni riga di pixel poiché la fenditura
non è perfetta, inoltre ogni valore dipende dalla posizione rispetto alla fenditura. 
Inoltre, se vi sono $n$ fenditure con larghezza $D$, il seno cardinale
presenterà anche un termine in funzione delle $n$ fenditure spaziate
di $a$:
\begin{gather*}
    I(\theta) = I_0 sinc^{2}\left(\frac{KD}{2}\sin\theta\right) \frac{\sin^{2} \left(\frac{naK}{2}\sin\theta\right)}{n\sin^{2}\left(\frac{ka}{2}\sin\theta\right)}
\end{gather*}
E dunque
\begin{gather*}
    \frac{ka}{2}\sin\theta = m \pi \ \Longrightarrow \ \sin\theta = \frac{2m\pi}{ka} \ \Longrightarrow \ m\frac{\lambda}{a}
\end{gather*}
In corrispondenza di multipli interi di  questa quantità si osservano
i picchi dell'intensità. \\
Utilizzando dunque la CCD come un reticolo
di diffrazione in riflessione, il singolo pixel è fatto di una parte in silicio
che riflette la luce ed una parte che contorna il pixel che invece assorbe
la luce. La distanza tra i pixel è quindi $a_p$ e la larghezza
della parte riflettente è $D_p$. Si vedrà dunque in riflessione il pattern
dell'intensità sulla parete del laboratorio.  Si può dunque stimare l'angolo di
riflessione del primo ordine rifratto e di quello $-1$. Posso prendere
la distanza sulla parete $l$ tra i due ordini uno e meno uno e la distanza
tra la parete e la CCD come $L$: 
\begin{gather*}
    \sin\theta \approx \tan\theta \approx \frac{\frac{l}{2}}{L}
\end{gather*}

\section{Seconda parte delle'esperienza: verifica della legge di diffrazione}
Nella seconda parte dell'esperienza si verifica la legge di diffrazione di un'onda
piana incidente su di un reticolo diffrazione. Misuro allora 
$\theta_i$ per valori diversi di $\lambda_i$ ricavando $a_i$. Mi assicuro allora che
i vari valori di $a_i$ trovati siano consistenti tra di loro. Utilizzando una
lampada a gas di Mercurio, si prende una cella di vetro all'interno della quale ci si mette un gas
di interesse: ci si mette un catodo ed un anodo e ci si attacca un generatore ad alta tensione.
Mediante degli impulsi ad alta tensione si eccitano gli elettroni  degli 
gli atomi di Mercurio, i quali saltano da un orbitale all'alto emettendo
dei fotoni che hanno energia
\begin{gather*}
    \Delta E = h \nu 
\end{gather*}
In corrispondenza di un angolo più piccolo si avrà dunque una luce più violacea
e, più cresce l'angolo, più la luce tenderà verso il rosso. Per ogni $\lambda_i$ per lo
spettro goniometro si va ad osservare un $\theta_i^{+}$ ed un $\theta_i^{-}$, per trovare dunque
l'angolo per una data lunghezza d'onda si avrà
\begin{gather*}
    \theta = \frac{\theta_i^{+} - \theta_i^{-}}{2} \ \Longrightarrow \ \sin \theta = m \frac{\lambda}{a}
\end{gather*}
Questa misura $\theta_i^{+}$ e $\theta_i^{-}$ si fa perché, dato che
la lettura è fatta con un crocefilo, vicino allo zero la luce è talmente intensa
che il crocefilo non è visibile, dunque si osserva cosa accade a destra e a sinistra del laser
per ottenere $\theta$. Si ottiene le incertezza sia per $\theta_i^{+}$ che per $\theta_i^{-}$. 

\end{document}
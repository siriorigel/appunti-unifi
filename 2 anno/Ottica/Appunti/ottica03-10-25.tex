\documentclass[a4paper, oneside]{article}
\usepackage{graphicx}
\usepackage{amsthm}
\usepackage{amsmath}
\usepackage{amssymb}
\usepackage[a4paper,
            bindingoffset=0.2in,
            left=2cm,
            right=2cm,
            top=2cm,
            bottom=2cm,
            footskip=.25in]{geometry}
\usepackage[italian]{babel}
\usepackage{pgfplots}
\usepackage{tabularx}
\usepackage{tikz}
\usepackage{wrapfig}
\usepackage{color}
\usepackage[d]{esvect}
\definecolor{page}{rgb}{0.129,0.157,0.212}
\pagecolor{page}
\color{white}
\graphicspath{ {./images/} }
\usetikzlibrary{shapes.geometric}
\usetikzlibrary{datavisualization}
\usetikzlibrary{datavisualization.formats.functions}
\usetikzlibrary{patterns}
\pgfplotsset{width=10cm,compat=1.9}

\title{Appunti di ottica}
\author{Tommaso Miliani}
\date{03-10-25}

\begin{document}
\newtheoremstyle{theoremEnv}
                {}          % Space above
                {}          % Space below
                {\slshape}  % Body font
                {}          % Indent amount
                {\bfseries} % Head font
                {.}         % Punctuation after head
                {\newline}         % Space after theorem head
                {}          % Theorem head spec
\theoremstyle{theoremEnv}

\newtheorem{definition}{Definizione}[section]
\newtheorem{theorem}{Teorema}[section]
\newtheorem{lemma}{Proposizione}[section]
\newtheorem{observation}{Osservazione}[section]
\newtheorem{corollary}{Corollario}[theorem]
\newtheorem{example}{Esempio}[section]

\maketitle

\section{Polarizzazione su di uno specchio metallico}
\begin{wrapfigure}{r}{0.4\textwidth}
    \centering
    \caption{La situazione dopo $t = \epsilon$ }
    \begin{tikzpicture}
        \draw(0, 0) -- (1, 0);
        \draw(0, 0) -- (0, 2);
        \draw(0, 0) -- (1, 2);
    \end{tikzpicture}    
\end{wrapfigure}
Dato un campo elettrico che si propaga lungo l'asse $x$, il campo
elettrico generico nel piano $yz$ si può scrivere come
\begin{gather*}
    \vv{E} = E_{0z}\cos(kx - \omega t + \phi_z)\hat{u}_z + E_{0y}\cos(kx - \omega t + \phi_y)\hat{u}_y   
\end{gather*}
Con i moduli $E_{0z} = E_{0y}$ tale che i vettori sono in fase con $\phi_y = \phi_z + \frac{\pi}{2}$, scelgo 
che $\phi_z = 0$ e dunque $\phi_y = \frac{\pi}{2}$, allora posso dire che il vettore
campo elettrico si può esprimere come
\begin{gather*}
    \vv{E} = E_0 \cos(kx - \omega t + \frac{\pi}{2})\hat{u}_y + E_0 \cos(kx - \omega t)\hat{u}_z   
\end{gather*}
La componente lungo $y$ diventa leggermente positiva mentre la componente lungo
$z$ è leggermente ridotta dopo un certo tempo $\epsilon$, che scelgo molto piccolo.
Allora risulta che il campo elettrico è ruotato rispetto alla direzione iniziale e sarà polarizzato
verso sinistra.


\begin{wrapfigure}{r}{0.4\textwidth}
    \centering
    \caption{}
    \begin{tikzpicture}[scale = 0.5]
        \draw(0, -1.5) -- (0, 1.5);
        \draw[color=cyan]   plot (\x,{cos(\x r)});    
    \end{tikzpicture}    
\end{wrapfigure}
Voglio vedere che succede se mando una certa onda piana
con una certa polarizzazione $\sigma$ su di uno specchio. 
L'interferenza distruttiva causata dal metallo non permette di
avere campo elettrico dentro lo specchio ma solo fuori come 
riflesso. Gli elettroni che oscillano nel mezzo metallico, oltre che
a generare il campo elettrico in controfase, generano un onda anche 
verso la direzione di provenienza del campo elettrico e dunque i due 
campi elettrici non si eliminano.
Dobbiamo scrivere il campo elettrico che si propaga verso sinistra 
cambiando la fase del campo di una fase $\pi$:
\begin{gather*}
    \vv{E} = -E_{0}\cos(-kx -\omega t + \frac{\pi}{2} )\hat{u}_y + E_0 \cos(-kx -\omega t)\hat{u}_z   
\end{gather*}
L'effetto dello specchio metallico è dunque quello di invertire il valore
del campo elettrico in virtù del fatto che il campo generato è in controfase.
Quindi i due campi sono opposti in $x = 0$ ed il nuovo campo si propaga verso
sinistra e non verso destra; la polarizzazione quando la luce incide su di
uno specchio viene invertita per cui se prima era a polarizzazione circolare
sinistra, l'onda uscente avrà una polarizzazione circolare destra.

\section{L'energia dell'onda elettromagnetica}
\begin{wrapfigure}{r}{0.3\textwidth}
    \centering
    \caption{}
    \begin{tikzpicture}
        \draw[->](0, 0) -- (1, 0) node[at end, below] {$\hat{k}$ };
        \draw(3, -1) -- (3, 1) node[at end, above] {$A$};
        \draw(3, 1) -- (2.5, 0) -- (2.5, -2) -- (3, -1);
    \end{tikzpicture}    
\end{wrapfigure}
Perché ci si scalda al sole? Quando una onda elettromagnetica
incide sui nostri elettroni li mette in accelerazione e dunque
guadagnano energia cinetica e la rilasciano sottoforma di calore.
Come si può determinare l'energia emessa da una data onda elettromagnetica? 
Posso determinare \textbf{l'intensità luminosa}, ossia l'energia che
passa per una data superficie $A$:
\begin{align}
    \frac{\Delta E}{\Delta t} \cdot \frac{1}{A} = I = \epsilon_0 c \vv{E}(t)^{2}  
\end{align}
Ossia l'energia che attraversa una certa superficie in un certo intervallo di
tempo:
\begin{align}
    \Delta E = I A \Delta t
\end{align}
Dato che il campo elettrico oscilla sempre, ci sono degli istanti in cui
il campo elettrico è nullo: l'intensità luminosa dunque oscilla anch'essa
con un certo periodo e con una certa fase rispetto al campo elettrico. Dato che
\begin{gather*}
    \omega = \frac{2\pi}{T} = \frac{2\pi c}{\lambda}
\end{gather*}
Dove $\lambda$ non è altro che la lunghezza d'onda della radiazione
considerata: se si considera la radiazione luminosa $\sim 0.6 \cdot  10^{-6} \ m$, si ha
che il periodo di oscillazione è $0.2 \cdot  10^{-14} \ s$. La nostra intensità luminosa
è allora la media dell'intensità luminosa su un periodo di oscillazione
del campo elettrico:
\begin{gather*}
    <I> = c\epsilon_0 \frac{1}{T}\int_{0}^{T}\vv{E}^{2} \ dt  \ \Longrightarrow \ <I> = c\epsilon_0 \frac{1}{T}\int_{0}^{T}E_0^{2}\cos^{2}(kx - \omega t) \ dt  
\end{gather*}
Allora posso risolvere la media rispetto al coseno alla seconda, che non è altro che
un mezzo, allora posso dire che
\begin{align}
    <I> = \frac{1}{2}c\epsilon_0 E^{2} 
\end{align}
Di conseguenza la media dell'intensità luminosa non è altro che la metà
dell'intensità luminosa totale in valore assoluto. Se la polarizzazione
del campo elettrico in modulo rimane sempre $E_0$ allora l'intensità
varia poiché si ha il doppio dell'intensità: questa onda polarizzata è la somma del
contributo dell'onda piana iniziale e del contributo sfasato rispetto
a questa onda. 

\section{Il fenomeno della trasmissione di un mezzo dielettrico}
\begin{wrapfigure}{r}{0.4\textwidth}
    \centering
    \caption{L'interfaccia tra dielettrici}
    \begin{tikzpicture}
        \draw(0, -1) -- (0, 1) node[at end, left] {$n_1$} node[at end, right] {$n_2 > n_1$};
        \draw[dashed](-3, 0) -- (3, 0);
        \draw(-3, 1) -- (0, 0);
        \draw(-1.5, 0.5) arc (150:180:1) node[midway, left] {$\theta$};
        \draw[->, thick, red](-2.25, 0.75) -- (-2, 1.35) node[at end, right] {$\vv{E_{\parallel}}$};
        \filldraw[red](-2.25, 0.75) circle (1pt) node[anchor = north east] {$\vv{E_{\perp}}$ };
        \draw(-2.25, 0.75) circle (0.1);
        \draw(0, 0) -- (3, -0.5);
        \draw[red, thick, ->](1.5, -0.25) -- (1.7, 0.5) node[at end, right] {$\vv{E_{\parallel}^{T} }$ };
        \draw(0, 0) -- (-3, -1);
        \draw[->, thick, red](-2.25, -0.75) -- (-2, -1.35) node[at end, right] {$\vv{E_{\parallel}}$};
        \filldraw[red](-2.25, -0.75) circle (1pt) node[anchor = south east] {$\vv{E_{\perp}}$ };
        \draw(-2.25, -0.75) circle (0.1);
    \end{tikzpicture}    
\end{wrapfigure}
I fenomeni di trasmissione e di riflessione alle interfacce tra dielettrici dipendono
dalla polarizzazione delle onde rispetto al piano
di incidenza. Il \textbf{piano di incidenza} è il piano che contiene il \textbf{raggio
incidente} definito dal vettore $\hat{k}$ e la direzione normale alla superficie
nel punto di incidenza: nel disegno il piano di incidenza coincide
con il foglio. Il vettore campo elettrico è dunque scomponibile in due componenti
una parallela ed una perpendicolare al piano di incidenza (il cui verso, ossia se entrante
o uscente è arbitrario). Un fenomeno particolare che si incontra è
quando una onda incide un dielettrico senza essere riflessa (ossia la luce
se ha una polarizzazione con solo $\vv{E_{\parallel}}$) viene  solo trasmessa. 

\begin{wrapfigure}{r}{0.4\textwidth}
    \centering
    \caption{L'angolo di Brewester}
    \begin{tikzpicture}
        \draw[dashed](-3 , 0) -- (3, 0);
        \draw(0, -2) -- (0, 2);
        \draw(-1, 2) -- (0, 0) -- (2, -1);
        \draw(0, 0) -- (-1, -2);
        \draw(-0.5, 1) arc (110:180:1) node[midway, left] {$\theta_1$};
        \draw(1, -0.5) arc (-30:0:1) node[midway, right] {$\theta_2$};
        \draw[->, green](0, 0) -- (0.25, 0.5) node[at end, above] {$\vv{d_E}$ };
        \draw[->, green](0.25, -0.12) -- (0.5, 0.35);
        \draw[->, green](0.5, -0.25) -- (0.75, 0.25);
    \end{tikzpicture}    
\end{wrapfigure}
Se avessi un dipolo elettrico oscillante con il campo orientato in una
certa direzione, allora esso emetterà solamente campo elettrico con
direzione perpendicolare alla direzione del dipolo elettrico oscillante;
l'emissione lungo la direzione di oscillazione è nulla. Il risultato delle onde
in trasmissione o riflessione sono date dalle cariche all'interno del
dielettrico: si immagina di avere la condizione per cui si ha una interfaccia 
con un raggio di luce che incide ad un certo angolo sul dielettrico in modo tale che
si crei un angolo di $90^{\circ}$ tra i raggi riflessi e trasmessi: in questo
caso si creano dipoli elettrici nel dielettrico che oscillano perpendicolarmente alla
direzione dell'onda trasmessa. Questi dipoli possono creare solo delle onde nella
direzione dell'onda riflessa e dunque l'onda riflessa non può esistere.
Questo angolo particolare prende il nome di \textbf{angolo di Brewester} e questo vale solo
quando $\vv{E}$ giace sul piano di incidenza e dunque si può verificare
solo per una polarizzazione lineare e vale dunque per $\vv{E_{\parallel}}$.   
La relazione che ottengo tra gli angoli sotto mi permette di ricavare l'angolo di Brewester:
\begin{gather*}
    \left\{\begin{array}{l}
        \theta_1 + \theta_2 + \frac{\pi}{2} = \pi \\
        n_1 \sin \theta_1 = n_2 \sin\theta_2
    \end{array}\right. \ \Longrightarrow \ \left\{\begin{array}{l}
        \theta_2 = \frac{\pi}{2} - \theta_1 \\
        n_1 \sin \theta_1 = n_2 \sin (\frac{\pi}{2} - \theta_1) = n_2 \cos\theta_1
    \end{array}\right. 
\end{gather*}
Si ottiene allora l'angolo critico per il quale si ha questa condizione: 
\begin{align}
    \tan \theta_1 = \frac{n_2}{n_1}
\end{align}
L'angolo tra aria e vetro di Brewester è esattamente $\theta = 56^{\circ}$. Se della luce
con polarizzazione ellittica (casuale) incide con l'angolo di Brewester su di una
superficie dielettrica, la riflessione emerge con polarizzazione lineare e perpendicolare.
L'interfaccia si comporta come se fosse un polarizzatore: solo la luce con una data polarizzazione
può percorrere il cammino di riflessione. 
Una polarizzazione \textbf{ellittica} è una polarizzazione tale per cui
\begin{gather*}
    E_{0y} \neq E_{0z} \qquad \text{se} \qquad \phi_y \neq \phi_z
\end{gather*}
E dunque la punta del campo elettrico sul piano $zy$ segue la traiettoria di una ellisse.


\end{document}
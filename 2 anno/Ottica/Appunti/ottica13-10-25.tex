\documentclass[a4paper, oneside]{article}
\usepackage{graphicx}
\usepackage{amsthm}
\usepackage{amsmath}
\usepackage{amssymb}
\usepackage[a4paper,
            bindingoffset=0.2in,
            left=2cm,
            right=2cm,
            top=2cm,
            bottom=2cm,
            footskip=.25in]{geometry}
\usepackage[italian]{babel}
\usepackage{pgfplots}
\usepackage{tabularx}
\usepackage{tikz}
\usepackage{wrapfig}
\usepackage{color}
\usepackage[d]{esvect}
\definecolor{page}{rgb}{0.129,0.157,0.212}
\pagecolor{page}
\color{white}
\graphicspath{ {./images/} }
\usetikzlibrary{shapes.geometric}
\usetikzlibrary{datavisualization}
\usetikzlibrary{datavisualization.formats.functions}
\usetikzlibrary{patterns}
\pgfplotsset{width=10cm,compat=1.9}

\title{Appunti di Ottica}
\author{Tommaso Miliani}
\date{13-10-25}

\begin{document}
\newtheoremstyle{theoremEnv}
                {}          % Space above
                {}          % Space below
                {\slshape}  % Body font
                {}          % Indent amount
                {\bfseries} % Head font
                {.}         % Punctuation after head
                {\newline}         % Space after theorem head
                {}          % Theorem head spec
\theoremstyle{theoremEnv}

\newtheorem{definition}{Definizione}[section]
\newtheorem{theorem}{Teorema}[section]
\newtheorem{lemma}{Proposizione}[section]
\newtheorem{observation}{Osservazione}[section]
\newtheorem{corollary}{Corollario}[theorem]
\newtheorem{example}{Esempio}[section]

\maketitle

\section{Aberrazioni delle lamine in condizioni non ideali}
\begin{wrapfigure}{r}{0.4\textwidth}
    \centering
    \caption{}
    \begin{tikzpicture}
        \draw[->](0, 0) -- (3, 0) node[at end, below] {$\vv{E_\parallel}$};
        \draw[->](0, 0) -- (2, 2) node[at end, above] {$a$};
        \draw[->](0, 0) -- (0, 3) node[at end, left] {$\vv{E_\perp} $};
        \draw[->](0, 0) -- (-1, 2) node[at end, above] {$b$};
        \draw(0.5, 0) arc (0:45:0.5) node[midway, right] {$\theta$};
    \end{tikzpicture}    
\end{wrapfigure}
Quando si opera con delle lamine $\frac{\lambda}{4}$ e $\frac{\lambda}{2}$
che non hanno ritardi di fase ideali, ossia quando la polarizzazione non è
lineare oppure si è in condizioni di umidità e temperatura che differiscono
da quelle specificate dal costruttore, si introducono inevitabilmente delle aberrazioni. 
Si è detto, fino ad ora, che esistono degli assi relativi ai cubi paralleli e
degli assi ortogonali relativi alla lamina. Si potrebbe avere che la mia lamina di
ritardo ha gli assi lenti e veloci inclinati di un certo angolo 
$\theta$ rispetto agli assi paralleli al polarizzatore. Per poter analizzare 
questa situazione, si ricorda l'espressione dei complessi in forma esponenziale
\begin{gather*}
    e^{i\theta} = \cos\theta + i\sin\theta 
\end{gather*}
Ricordando ora l'espressione del campo elettrico:
\begin{gather*}
    \vv{E} = E_{0z}\cos(kx - \omega t + \phi_z)\hat{k}  + E_{0y} \cos(kx - \omega t + \phi_y)\hat{y}  
\end{gather*}
Utilizzando i complessi posso esprimere il campo elettrico come una componente lungo l'asse
reale e l'altra lungo l'asse degli immaginari. Mi ricordo però di prendere
solamente la parte reale di questa espressione e dunque:
\begin{gather*}
    \vv{E} = Re\left(E_{0z}e^{i(kx - \omega t + \phi_z)}\hat{z} + E_{0y}e^{i(kx - \omega t + \phi_y)} \hat{y} \right) 
\end{gather*}
Posso pensare di avere una matrice $2\times 1$ ed esprimere il campo
in forma matriciale:
\begin{gather*}
    \begin{pmatrix}
    E_{0z}e^{i(kx - \omega t + \phi_z)} \\
    E_{0y}e^{i(kx - \omega t + \phi_y)}
    \end{pmatrix}
\end{gather*}
Allora lungo gli assi $a, b$, che sono orientati di un certo $\theta$
rispetto agli assi paralleli ed ortogonali, si esprime il campo
elettrico come:
\begin{gather*}
    \vv{E}_{a, b} = E_{0z} \cos(kx - \omega t + \phi_z) \left(\cos\theta \hat{a} - \sin\theta \hat{b}  \right) + E_{0y} \cos(kx - \omega t + \phi_z) \left(\sin\theta \hat{a} + \cos\theta \hat{b}  \right) 
\end{gather*}
Questa espressione si può allora riscrivere come le coordinate del campo rispetto
ai versori $\hat{a}$ e $\hat{b}$. Posso dunque pensare a questa espressione come
il prodotto tra una matrice due per due e il vettore del campo elettrico nelle sue componenti:
\begin{align}
    Re\begin{pmatrix}
        \cos\theta & \sin\theta \\
        -\sin\theta & \cos\theta 
    \end{pmatrix}\begin{pmatrix}
        E_{0z} e^{i(kx - \omega t + \phi_z)} \\
        E_{0y} e^{i(kx - \omega t + \phi_y)}  
    \end{pmatrix}
\end{align}   
La matrice a sinistra prende il nome di matrice delle rotazioni in funzione
dell'angolo $\theta$, questa matrice permette di ottenere l'espressione di un vettore
rispetto a degli assi ruotati di un certo angolo $\theta$, e si indica con $R(\theta)$. 
La matrice che serve per tornare agli assi di partenza è la matrice inversa (ossia quella
coi segni invertiti per il seno).  Adesso, data l'espressione
del campo elettrico generico rispetto alla base $\hat{a}, \hat{b} $
\begin{gather*}
    \vv{E}_{a, b} E_{0a}\cos(kx - \omega t + \phi_a) \hat{a} + R_{0b} \cos(kx - \omega t + \phi_b)\hat{b} 
\end{gather*} 
Posso esprimerlo attraverso i numeri complessi nel vettore
\begin{gather*}
    Re \begin{pmatrix}
        E_{0a}e^{i(kx - \omega t + \phi_a)} \\
        E_{0b}e^{i(kx - \omega t + \phi_b)}  
    \end{pmatrix}
\end{gather*}
Posso introdurre sull'asse lento un ritardo di fase. Per cui
il campo elettrico uscente è dato dal campo rispetto al versore $\hat{b}$ e 
dal campo rispetto al versore $\hat{a}$ con un ritardo di fase:  
\begin{gather*}
    \vv{E_{a, b}^{OUT} } = E_{0a}\cos(kx - \omega t + \phi_a + \delta \phi)\hat{a}  + E_{0b}\cos(kx - \omega t + \phi_b)\hat{b}   
\end{gather*}
Partendo dal vettore di ingresso nella lamina, posso ottenere l'espressione del campo in uscita
attraverso il prodotto tra matrici come
\begin{gather*}
    Re\begin{pmatrix}
        e^{\delta\phi} & 0  \\
        0 & 1 
    \end{pmatrix} \begin{pmatrix}
        E_{0a}e^{i(kx - \omega t + \phi_a)} \\
        E_{0b}e^{i(kx - \omega t + \phi_b)}  
    \end{pmatrix}
\end{gather*}
Analogamente a quanto detto prima, se si volesse ottenere il campo elettrico in entrata a partire dal campo
elettrico in uscita, si può applicare la matrice inversa. Si riassume che l'espressione
del campo elettrico in uscita da una lamina, che è entrato con un certo angolo, 
ha espressione
\begin{align}
    \vv{E_{a, b}^{OUT} } = Re \left(R(-\theta)\begin{pmatrix}
        e^{\delta \phi} & 0 \\
        0 & 1 
    \end{pmatrix}R(\theta) \begin{pmatrix}
        E_{0z}e^{i(kx - \omega t + \phi_z)}\\
        E_{0y}e^{i(kx - \omega t + \phi_y)}  
    \end{pmatrix}\right) 
\end{align}

\section{Interferenza}
L'Interferenza si verifica quando sono presenti due o più campi elettromagnetici
con fasi differenti. Di fatto, quando si sommano i campi elettromagnetici
si utilizzano le formule di prostaferesi per il coseno:
\begin{align}
    \cos\alpha + \cos\beta = 2\cos\left(\frac{\alpha + \beta}{2}\right) \cdot  \cos\left(\frac{\alpha - \beta}{2}\right)
\end{align}
Dato che ci sono molte variabili per cui i campi potrebbero differire, possiamo
partire dall'analisi del caso semplice in cui solamente la fase differisce per i 
due campi elettrici.

\subsection{Interferenza semplice}
Il caso semplice di Interferenza si ha quando i due campi elettromagnetici
hanno la stessa direzione di propagazione e quindi stessa frequenza (o pulsazione) ; inoltre,
per rimanere nel caso semplice, hanno anche stessa polarizzazione e stessa ampiezza.
L'unica differenza è la fase differente. 
\begin{gather*}
    \vv{E_1} = E_0\cos(kx - \omega t + \phi_1)\hat{y}  \\
    \vv{E_2} = E_0\cos(kx - \omega t + \phi_2)\hat{y}   
\end{gather*}
Posso ottenere allora il campo elettrico totale come la somma tra i due
contributi secondo la formula di prostaferesi:
\begin{gather*}
    \vv{E_{TOT}} = 2E_0\cos\left(kx - \omega t + \frac{\phi_1 + \phi_2}{2}\right) \cdot  \cos\left(\frac{\phi_1 - \phi_2}{2}\right) 
\end{gather*}
Il totale dell'intensità mediata nel tempo, si ottiene come 
\begin{align}
    \left< I_{TOT}  \right> = c\epsilon_0 \left< \vv{E}_{TOT}  \right>  = c\epsilon_0\frac{1}{T}\int_{0}^{T}4E_0^{2}\cos^{2}\left(kx - \omega t + \frac{\phi_1 + \phi_2}{2}\right) \cos^{2}\left(\frac{\phi_1 + \phi_2}{2}\right) \ dt
\end{align}
Ossia la media temporale dell'integrale del campo elettrico mi dà l'intensità totale media
in un certo intervallo di tempo $\Delta T = T$. Questo mi permette allora di esprimere
l'intensità delle due onde come l'intensità di una unica onda "somma". 
\begin{gather*}
    \left< I_{TOT} \right> =  4I_0\cos^{2} \left(\frac{\phi_1 - \phi_2}{2}\right)
\end{gather*}
Quando $\phi_1 = \phi_2$ allora c'è un istante in cui l'intensità è molto maggiore dell'intensità
generata dalla semplice somma delle onde, mentre c'è un istante in cui si "distrugge"
l'energia delle due onde, ossia quando $\phi_1 = \pi + \phi_2$. Si riassume con
\begin{itemize}
    \item Se $\phi_1 - \phi_2 = 2\pi n$: le onde sono in \textbf{fase}, e si ha interferenza costruttiva;
    \item Se $\phi_1 - \phi_2 = 2\pi n + \pi$: le onde sono in \textbf{controfase} e si ha interferenza distruttiva.
\end{itemize}

\subsection{Oscillazione con polarizzazioni diverse}
\begin{wrapfigure}{r}{0.4\textwidth}
    \centering
    \caption{}
    \begin{tikzpicture}
        \draw[->](0, 0) -- (4, 0) node[at end, below] {$x$};
        \draw[->](0, 0) -- (1.5, 1.5) node[at end, above] {$\vv{E_1}$};
        \draw[->](0, 0) -- (0, 2) node[at end, right] {$y$} node[at end, left] {$\vv{E_2}$};
        \draw(0, 0.5) arc (90:45:0.5) node[midway, above] {$\alpha$};
        \draw[->](0, 0) -- (-1, -1) node[at end, right] {$z$};
    \end{tikzpicture}    
\end{wrapfigure}
In questo caso l'ampiezza e la direzione dei campi elettrici non
sono mai gli stessi per le due onde. Supponiamo ancora che abbiano
lo stesso $\hat{k}$ ma i due hanno polarizzazioni lineari diverse. 
\begin{gather*}
    \vv{E_1} = \vv{E_{01}}\cos(kx - \omega t + \phi_1) \\
    \vv{E_2} = \vv{E_{02}}\cos(kx - \omega t + \phi_2)    
\end{gather*}
Si può ora scrivere il campo elettrico totale e poi si va a farne il modulo
al quadrato per determinarne l'intensità. Dato che i due campi adesso hanno
stessa ampiezza e direzione ma con polarizzazioni diverse, posso determinare
la somma totale del vettore campo elettrico come il modulo
\begin{gather*}
    \left| \vv{E_{TOT}}  \right|^{2} = \left| \vv{E_1} + \vv{E_2}  \right|^{2}    = \left| \vv{E_1}  \right|^{2} + \left| \vv{E_2}  \right|^{2} + 2\vv{E_1} \cdot  \vv{E_2}      
\end{gather*}
Questo deve essere equivalente alla seguente espressione, ossia il modulo
del vettore campo elettrico totale:
\begin{gather*}
    |\vv{ E_{01}}|^{2} \cos^{2}(kx - \omega t + \phi_1) + \left| \vv{E_{02}}  \right|^{2}\cos^{2} (kx - \omega t + \phi_2) + 2\vv{E_{01}} \cdot  \vv{E_{02}}\cos(kx - \omega t + \phi_1) \cdot  \cos(kx - \omega t + \phi_2)     
\end{gather*}
Il terzo termine si può pensare come il prodotto scalare tra i due vettori che compongono 
il campo elettrico:
\begin{gather*}
    \left| \vv{E_{01}}  \right| \cdot  \left| \vv{E_{02}}  \right| \cos\alpha  
\end{gather*}
Dove $\alpha $ è l'angolo compreso tra i due vettori del campo elettrico come
nel disegno. Con le formule di prostaferesi inverse posso esprimere il coseno che moltiplica il 
terzo termine in funzione di angoli generici $\beta$ e $\gamma$ (gli argomenti dei due coseni):
\begin{gather*}
    \cos\beta \cdot  \cos\gamma = \frac{1}{2}\left(\cos(\beta + \gamma) + \cos(\beta - \gamma)\right)
\end{gather*}
Allora il terzo termine del modulo del campo totale è esprimibile come:
\begin{gather*}
    2\left| \vv{E_{01}}  \right| \cdot  \left| \vv{E_{02}}  \right|\cos\alpha + \frac{1}{2} \left(\cos(2kx - \omega t + \phi_1 + \phi_2) + \cos(\phi_1 - \phi_2)\right) 
\end{gather*}
Si può esprimere allora l'intensità totale del campo elettrico come la somma delle intensità delle due onde
più l'intensità dovuta al terzo termine dell'espressione del modulo totale (Ossia
l'espressione di prima). 
\begin{gather*}
    \left< I_{TOT} \right> = I_1 + I_2 + c\epsilon_0 \frac{2\left| \vv{E_{01}}  \right| \left| \vv{E_{02}}  \right|\cos\alpha}{2} \cdot  \cos(\phi_1 - \phi_2)  
\end{gather*}
L'intensità $I_1$  ($I_2$ è analoga) si esprime come:
\begin{gather*}
    I_1 = c\epsilon_0 \cdot  \frac{\left| \vv{E_{01}}  \right|^{2}  }{2} \ \Longrightarrow \ \sqrt{I_1} = \sqrt{c}\sqrt{\epsilon_0} \frac{|\vv{E_{10}}| }{\sqrt{2} }   
\end{gather*}
Che posso sostituire nell'espressione dell'intensità totale al terzo termine:
\begin{gather*}
    \left< I_{TOT} \right> = I_1 + I_2 + 2\sqrt{I_1}\sqrt{I_2}\cos\alpha \cos(\phi_1 - \phi_2)   
\end{gather*}
Il terzo termine dell'intensità totale è chiamato \textbf{termine di interferenza}. Questo termine
determina se c'è interferenza oppure no.
\begin{itemize}
    \item Se $\alpha = \frac{\pi}{2}$: non c'è alcuna Interferenza poiché le polarizzazioni
    sono perpendicolari tra di loro.
    \item Se $I_1 = I_2$, allora si ritrova il caso trattato nel paragrafo precedente:
    \begin{gather*}
        I_{TOT} = 2I_0 (1 + \cos(\phi_1 - \phi_2)) \ \Longrightarrow \ 4I_0 \cos^{2}\left(\frac{\phi_1 - \phi_2}{2}\right) 
    \end{gather*}
    \item Se $\alpha = 0$: si ha il caso con Interferenza trattato ora.  
\end{itemize} 

\section{L'onda stazionaria}
\begin{wrapfigure}{r}{0.4\textwidth}
    \centering
    \caption{}
    \begin{tikzpicture}[scale = 1.3]
        \draw[->](-2, 0) -- (2, 0) node[at end, below] {$x$};
        \draw[dashed](0, 1.5) -- (0, -1.5);
        \draw(-1, -1) -- (0, 0) -- (1, -1);
        \draw(-0.3, -0.3) arc (225:315:0.4) node[midway, below ] {$\theta$}; 
        \draw[very thick, ->](-0.8, -0.8) -- (-0.3, -0.3) node[midway, left] {$\vv{k_1}$};
        \draw[very thick, ->](0.8, -0.8) -- (0.3, -0.3) node[midway, right] {$\vv{k_2}$};
    \end{tikzpicture}    
\end{wrapfigure}
L'onda stazionaria si ottiene nel caso in cui si ha interferenza di vettori
d'onda delle due onde elettromagnetiche che interferiscono tali che $\vv{k_1}$ e $\vv{k_2}$
non sono più paralleli: si ottiene un'onda stazionaria. 
\begin{gather*}
    \vv{E_1} = E_0 \hat{y}\cos(\vv{k_1} \cdot  \vv{r} - \omega t + \phi_1  )  \\
    \vv{E_2} = E_0 \hat{y}\cos(\vv{k_2} \cdot  \vv{r} - \omega t + \phi_2  )  
\end{gather*}  
Nel caso del modulo totale del campo elettrico:
\begin{gather*}
    \left| \vv{E_{TOT}}  \right| E_0^{2} \left(\cos(\vv{k_1} \cdot  \vv{r} - \omega t + \phi_1  ) + \cos(\vv{k_2} \cdot  \vv{r} - \omega t + \phi_2   )\right)^{2}  
\end{gather*}
Applico nuovamente le formule di prostaferesi al contrario ottenendo 
\begin{gather*}
    4E_0^{2}\left(\cos^{2} \left(\frac{(\vv{k_1}  + \vv{k_2}) \cdot  \vv{r}  }{2} - \omega t + \frac{\phi_1 + \phi_2}{2}\right) \cdot  \cos^{2} \left(\frac{(\vv{k_1} - \vv{k_2}  )\cdot \vv{r} }{2} + \frac{\phi_1 - \phi_2}{2}\right)\right) 
\end{gather*}
C'è una parte del campo che ha dipendenza temporale ed un apart del campo che non ha dipendenza temporale. Quando
calcolo l'intensità totale mediata nel tempo allora la parte che dipende dal tempo è mediata, l'altra è costante:
\begin{gather*}
    \left< I_{TOT} \right> = 4c\epsilon_0 \frac{E_0^{2} }{2}\cos^{2}\left(\frac{\vv{k_1} - \vv{k_2}  }{2} \cdot \vv{r} + \frac{\phi_1 - \phi_2}{2} \right) = 4I_0\cos^{2}\left(\frac{\vv{k_1} - \vv{k_2}  }{2} \cdot \vv{r} + \frac{\phi_1 - \phi_2}{2}\right)  
\end{gather*}
Si trova allora che l'intensità dipende dalla posizione in cui si posiziona il rilevatore: l'intensità
è diventata funzione della posizione e non dipende dal tempo (dunque è una onda stazionaria).
Potrei esprimere questa espressione attraverso il vettore differenza $\vv{\Delta k}$ tra i
vettori d'onda delle due onde:
\begin{gather*}
    \left< I_{TOT} \right> = 4I_0\cos^{2}\left(\frac{|\vv{\Delta k}| }{2} \cdot  x + \frac{\phi_1 - \phi_2}{2}\right)  
\end{gather*} 
Posso esprimere allora il prodotto del vettore d'onda come $\delta x$, ossia la periodicità spaziale
dell'onda stazionaria. Se si considera $\theta$ l'angolo tra i due vettori d'onda, 
allora il vettore dentro il coseno è esattamente:
\begin{gather*}
    \left| \vv{k}  \right| \sin\frac{\theta}{2} \cdot  \delta x = \pi 
\end{gather*}
Allora, ricordando il modulo di $\vv{k}$, posso esprimere
\begin{gather*}
    \frac{2}{\lambda}\sin\frac{\theta}{2} \delta x = 1
\end{gather*} 
Dunque possiamo scegliere $\delta x$ in modo arbitrario
\begin{gather*}
    \delta x = \frac{\lambda}{2\sin\frac{\theta}{2}} \qquad \theta << 1 \ \Longrightarrow \ \delta x = \frac{\lambda}{\theta}
\end{gather*}
Se l'angolo tra le due onde è molto piccolo, allora si può approssimare con l'argomento e dunque
la lunghezza d'onda si amplifica. 

\end{document}
\documentclass[a4paper, oneside]{article}
\usepackage{graphicx}
\usepackage{amsthm}
\usepackage{amsmath}
\usepackage{amssymb}
\usepackage[a4paper,
            bindingoffset=0.2in,
            left=2cm,
            right=2cm,
            top=2cm,
            bottom=2cm,
            footskip=.25in]{geometry}
\usepackage[italian]{babel}
\usepackage{pgfplots}
\usepackage{tabularx}
\usepackage{tikz}
\usepackage{wrapfig}
\usepackage{color}
\usepackage[d]{esvect}
\definecolor{page}{rgb}{0.129,0.157,0.212}
\pagecolor{page}
\color{white}
\graphicspath{ {./images/} }
\usetikzlibrary{shapes.geometric}
\usetikzlibrary{datavisualization}
\usetikzlibrary{datavisualization.formats.functions}
\usetikzlibrary{patterns}
\pgfplotsset{width=10cm,compat=1.9}

\title{APpunti di OTtica}
\author{Tommaso Miliani}
\date{17-10-25}

\begin{document}
\newtheoremstyle{theoremEnv}
                {}          % Space above
                {}          % Space below
                {\slshape}  % Body font
                {}          % Indent amount
                {\bfseries} % Head font
                {.}         % Punctuation after head
                {\newline}         % Space after theorem head
                {}          % Theorem head spec
\theoremstyle{theoremEnv}

\newtheorem{definition}{Definizione}[section]
\newtheorem{theorem}{Teorema}[section]
\newtheorem{lemma}{Proposizione}[section]
\newtheorem{observation}{Osservazione}[section]
\newtheorem{corollary}{Corollario}[theorem]
\newtheorem{example}{Esempio}[section]

\maketitle

\section{Riassuntino volta scorsa}
\begin{wrapfigure}{r}{0.4\textwidth}
    \centering
    \caption{Grafico interferenza dell'altra volta}
    \begin{tikzpicture}[domain=0:6, scale=0.7, samples = 50]
        \draw[->](0, 0) -- (6, 0) node[at end, below] {$x$};
        \draw[->](0, 0) -- (0, 3) node[at end, left] {$I_0$};
        \draw[|-|](0, -0.2) -- (3.14, -0.2) node[midway, below] {$\delta x$};
        \filldraw (0,  2) circle (0pt) node[anchor = east] {$4I_0$};
        \draw plot (\x,{cos(2* \x r) + 1});
    \end{tikzpicture}    
\end{wrapfigure}
Dato che l'onda stazionaria ha una lunghezza d'onda
\begin{gather*}
    \delta x = \frac{\lambda}{2\sin\left(\frac{\theta}{2}\right)}
\end{gather*}
A sinistra dell'interfaccia le interferenze dell'onda in ingresso
e quella in uscita generano un onda complessiva che è ferma nello spazio
ma oscilla nel tempo che prende proprio il nome di onda stazionaria. Questa onda
ha sempre dei punti in cui il campo elettrico è totalmente nullo e 
dunque l'intensità media viene diversa da zero. \\

\section{Interferometro di Machelson}
\begin{wrapfigure}{r}{0.4\textwidth}
    \centering
    \caption{L'interferometro di Machelson}
    \begin{tikzpicture}
        \draw[->](0, 0.2) -- (2.25, 0.2);
        \draw[->](2.25, 0.2) -- (2.25, 2);
        \draw[->](2.5, 2) -- (2.5, 0.75);
        \draw[->](2.5, 0.75) -- (1, 0.75);
        \draw[->](2.5, 0.75) -- (2.7, 0.35) -- (2.7, -2);
        \draw[->](2.25, 0.2) -- (2.55, 0) -- (4.5, 0);
        \draw[->](4.5, -0.25) -- (2.42, -0.25);
        \draw[->](2.42, -0.25) -- (2.42, -2);
        \draw[->](2.42, -0.25) -- (2.15, -0.1) -- (0, -0.1);
        \draw(1.75, -1) -- (2, -1.2) -- (3, 1) -- (2.75, 1.2);
        \draw[very thick](2.75, 1.2) -- (1.75, -1);
        \draw[pattern = north west lines, pattern color = white](1.75, 2) rectangle (3, 2.5);
        \draw[pattern = north west lines, pattern color = white](4.5, 0.75) rectangle (5, -1);
        \filldraw (2.1, -2) rectangle (3, -2.25);
    \end{tikzpicture}
\end{wrapfigure}
L'\textbf{interferometro di Machelson} è l'interferometro che si utilizza
durante l'esperienza dell'interferenza. Una prima onda piana è inviata su di un
componente ottico, ossia il \textbf{separatore di fascio}, che è un componente che
ha un substrato di materiale dielettrico su una delle sue superfici:
il materiale dielettrico ha la caratteristica per cui la luce ha il
$50\%$ di probabilità di essere riflessa o trasmessa.  \\
Il fascio riflesso compie un cammino $L_1$ prima di incontrare uno specchio e
venire riflesso e torna nuovamente sul separatore. Il fascio che inizialmente era
stato trasmesso procede per un cammino $L_2$ fino ad un nuovo specchio e poi torna indietro.
Questo interferometro mi permette di ottenere da un solo fascio di luce 
ben 4 fasci di luce distinti. \\
Sotto all'interferometro è posto un rilevatore per determinare
l'interferenza tra le due onde che giungono al rilevatore stesso.
Dato che il campo elettrico va come il quadrato mediato nel tempo, ogni volta che il
fascio attraversa l'interferometro  (da sinistra verso destra) il suo campo elettrico
diventa $E_r = -\frac{E_0}{\sqrt{2} }$ per il fascio riflesso (che accumula un ritardo di fase di $\pi$)
mentre per la parte trasmessa $E_t = \frac{E_0}{\sqrt{2} }$ che non accumula ritardo di fase. Ogni interferometro
è tarato per trasmettere o riflettere una certa percentuale del campo elettrico in entrata. 
Se invece propagassi il fascio di luce da destra verso sinistra si ha che sia la
parte riflessa, che quella trasmessa, hanno il campo elettrico 
\begin{gather*}
    E_r = E_t = \frac{E_0}{\sqrt{2} }
\end{gather*}

\begin{wrapfigure}{r}{0.4\textwidth}
    \centering
    \caption{Fascio riflesso}
    \begin{tikzpicture}
        \draw[->](0, 0.2) -- (2.25, 0.2);
        \draw[->](2.25, 0.2) -- (2.25, 2);
        \draw[->](2.5, 2) -- (2.5, 0.75);
        \draw[->](2.5, 0.75) -- (1, 0.75);
        \draw[->](2.5, 0.75) -- (2.7, 0.35) -- (2.7, -1);
        \draw(1.75, -1) -- (2, -1.2) -- (3, 1) -- (2.75, 1.2);
        \draw[very thick](2.75, 1.2) -- (1.75, -1);
        \draw[pattern = north west lines, pattern color = white](1.75, 2) rectangle (3, 2.5);
        \draw[|-|](2, 1.8) -- (2, 0.4) node[midway, left] {$L_1$};
    \end{tikzpicture}    
\end{wrapfigure}
La definizione di destra o sinistra dipende da dove è messo il materiale dielettrico 
prima del substrato di vetro. Questa tipologia di interferometro è utilizzato all'interno
dei rilevatori di onde gravitazionali; è anche uno degli strumenti più sensibili
mai costruiti dall'uomo.  A partire dal campo elettrico posso determinare
il primo contributo che va verso il rivelatore
\begin{gather*}
    E_0\cos(\vv{k}\cdot \vv{x} - \omega t  )
\end{gather*}
L'onda inizialmente riflessa avrà come espressione del campo elettrico
\begin{gather*}
    -\frac{E_0}{\sqrt{2} }
\end{gather*}
Dato che alla fine si formano quattro fasci, il fascio che inizialmente è stato riflesso e
poi trasmesso avrà come modulo del campo elettrico
\begin{gather*}
    \frac{E_0}{\sqrt{2}\sqrt{2}  }\cos(\vv{k} \cdot \vv{r} - \omega t + \pi + 2kL_1 + \pi  )
\end{gather*}
Il primo $\pi$ è dovuto al ritardo di fase dovuto alla riflessione
dell'interferometro, mentre il termine $2kL_1$ è il termine di ritardo di fase dovuto alla
riflessione sullo specchio sopra all'interfaccia, ossia la distanza che percorre
la luce prima di tornare all'interfaccia. Il termine $\pi$ è dovuto invece al ritardo
di fase dovuto alla riflessione sullo specchio a distanza $L_1$. 

\begin{wrapfigure}{r}{0.4\textwidth}
    \centering
    \caption{Fascio trasmesso}
    \begin{tikzpicture}
        \draw[->](0, 0.2) -- (2.25, 0.2);
        \draw[->](2.25, 0.2) -- (2.55, 0) -- (4.5, 0);
        \draw[->](4.5, -0.25) -- (2.42, -0.25);
        \draw[->](2.42, -0.25) -- (2.42, -1);
        \draw[->](2.42, -0.25) -- (2.15, -0.1) -- (0, -0.1);
        \draw(1.75, -1) -- (2, -1.2) -- (3, 1) -- (2.75, 1.2);
        \draw[very thick](2.75, 1.2) -- (1.75, -1);
        \draw[pattern = north west lines, pattern color = white](4.5, 0.75) rectangle (5, -1);
        \draw[|-|] (2.6, -0.4) -- (4.3, -0.4) node[midway, below] {$L_2$};
    \end{tikzpicture}    
\end{wrapfigure}
L'altro contributo è quello del fascio che prima è trasmesso e poi è
riflesso allo specchio a distanza $L_2$ e poi è riflesso sull'interfaccia.
Posso quindi dire che il campo elettrico di quel fascio di luce
che giunge sul rilevatore è 
\begin{gather*}
    \frac{E_0}{\sqrt{2} \sqrt{2} }\cos(\vv{k} \cdot  \vv{r} - \omega t  + \pi + 2kL_2)
\end{gather*}
Qui abbiamo il termine $\pi$ che è il ritardo di fase dovuto alla riflessione sullo
specchio a distanza $L_2$ e anche il contributo $2kL_2$ dovuto alla
distanza dall'interfaccia dello specchio. Per questo fascio di luce non c'è il termine
$\pi$ in quanto il fascio in riflessione non accumula ritardo di fase.
A questo punto devo fare la somma dei campi elettrici complessivi in modo
tale da poter ottenere il campo elettrico risultante al rilevatore:
\begin{gather*}
    \frac{E_0}{2}\left(\cos(\vv{k}\cdot  \vv{r} - \omega t + 2kL_2  + \pi) + \cos(\vv{k} \cdot  \vv{r} - \omega t + 2kL_1  )\right)
\end{gather*}
Adesso posso applicare le formule di Prostaferesi per ottenere il campo elettrico totale
come
\begin{gather*}
    E_{TOT} = \frac{E_0}{2} \cdot  \left(2\cos\left(\vv{k} \cdot  \vv{r} - \omega t + k(L_1 + L_2) + \frac{\pi}{2}  \right) \cdot  \cos\left(k(L_2 - L_1) + \frac{\pi}{2}\right)\right)
\end{gather*}
Adesso posso ottenere il quadrato del campo elettrico e mediato nel tempo 
mi dà l'intensità del fascio luminoso medio:
\begin{gather*}
    \left< I_{TOT} \right> = c\epsilon_0 \left< \left|\vv{E_{TOT}}\right|^{2}   \right> = I_0\cos^{2}\left(k(L_2 - L_1) + \frac{\pi}{2}\right) = I_0\sin^{2}(k(L_2 - L_1))    
\end{gather*}
Si può ottenere ora una nuova espressione per l'intensità esplicitando $k$:
\begin{gather*}
    I_{TOT} = I_0\sin^{2}\left(\pi \frac{L_2 - L_1}{\frac{\lambda}{2}}\right) 
\end{gather*}
Basta che cambi la differenza tra i cammini ottici di $\frac{\lambda}{2}$ che il seno ha fatto una 
oscillazione completa. Basta dunque cambiare la distanza degli specchi di una lunghezza d'onda
per poter accorgersi della variazione dell'intensità del campo elettrico.

\subsection{Lo specchio con la camera a vuoto}
Supponendo di avere una impostazione simile, se lo specchio su cui deve riflettersi la luce
è nel vuoto ed è lasciato cadere, la luce che torna indietro dai due specchi interferirà
sul rilevatore. Si può esprimere lo spostamento dello specchio come
\begin{gather*}
    \Delta L(t) = L_1 - L_2 = -\frac{1}{2}gt^{2} + \cos t  
\end{gather*}
\begin{wrapfigure}{r}{0.4\textwidth}
    \centering
    \caption{Il grafico dei valori di $t$}
    \begin{tikzpicture}[domain=0:3.5]
        \draw[->](0, 0) -- (4, 0) node[at end, below] {$t$};
        \draw[->](0, 0) -- (0, 4) node[at end, left] {$\frac{gt^{2} }{\lambda}$};
        \draw plot (\x, {(\x * \x) / 4});
    \end{tikzpicture}    
\end{wrapfigure}
Allora l'intensità dovrà anch'essa dipendere dal tempo,
ottenendo la seguente espressione:
\begin{gather*}
    I(t) = I_0 \sin^{2}\left(\pi\frac{\Delta L(t)}{\frac{\lambda}{2}}\right) = \sin^{2}\left(\pi\frac{gt^{2} }{\lambda}\right)  
\end{gather*}
Immaginando di avere uno oscilloscopio al rilevatore, io mi aspetto che l'intensità possa
sempre variare tra zero ed uno e man mano mi aspetto che
il periodo di oscillazione avvenga sempre più velocemente.
Questo vuol dire che il termine dentro al seno è tale per cui
\begin{gather*}
    \frac{gt_1^{2} }{\lambda} = 1 \qquad \frac{gt_2^{2} }{\lambda} = 2, \qquad \dots
\end{gather*}
Se volessi riportare tutti i $t_i$ nei quali l'intensità del campo elettrico è zero, otterrei
un andamento congruente con quello di una parabola che dipende esattamente da $g$.

\section{La realizzazione di uno spettrografo}
\subsection{Due lunghezze d'onda}
Supponendo di mandare dentro all'interferometro di Machelson due fasci di
luce con due lunghezze d'onda differenti, ci si aspetterebbe una lettura diversa
al rilevatore. Quello che accade però è che il campo elettrico non subisce
alcuna modifica in quanto i due fasci con lunghezze d'onda diverse
non interferiscono tra di loro: infatti i loro campi elettrici totali sono 
esattamente:
\begin{gather*}
    E_{TOT_1} = \frac{E_{01}}{2} \cdot  \left(2\cos\left(\vv{k_1} \cdot  \vv{r} - \omega t + k_1(L_1 + L_2) + \frac{\pi}{2}  \right) \cdot  \cos\left(k_1(L_2 - L_1) + \frac{\pi}{2}\right)\right) \\
    E_{TOT_2}= \frac{E_{02}}{2} \cdot  \left(2\cos\left(\vv{k_2} \cdot  \vv{r} - \omega t + k_2(L_1 + L_2) + \frac{\pi}{2}  \right) \cdot  \cos\left(k_2(L_2 - L_1) + \frac{\pi}{2}\right)\right)
\end{gather*}
Adesso mi interessa solamente della dipendenza temporale del campo elettrico. Dato che
alcuni dei termini sono fissati, allora posso utilizzare una certa approssimazione
per evidenziare questa dipendenza temporale ottenendo,
rispettivamente, per ii due campi totali:
\begin{gather*}
    A_1\cos(\omega_1 t + \phi_1) \\
    A_2\cos(\omega_2 t + \phi_2)
\end{gather*}
Ottengo allora il seguente campo sul rilevatore:
\begin{gather*}
      E_{TOT} = A_1\cos(\omega_1 + \phi_1) + A_2 \cos(\omega_2  + \phi_2)
\end{gather*}
Il modulo del campo elettrico totale è dunque:
\begin{gather*}
    \left| \vv{E_{TOT}}  \right|^{2} = A_1^{2}\cos^{2}(\omega_1 t + \phi_1) + A_2^{2}\cos^{2}(\omega_2 t + \phi_2) + 2A_1A_2 \cos(\omega_1 t + \phi_1) \cos(\omega_2 t + \phi_2)      
\end{gather*}
Il terzo termine è esattamente l'interferenza tra le due lunghezze d'onda nel tempo. 
Posso ora dimostrare che quel terzo termine mediato nel tempo è nullo: infatti posso, da prodotto,
trasformalo in una somma di coseni attraverso la formula di Prostaferesi:
\begin{gather*}
     \cos(\omega_1 t + \phi_1) \cos(\omega_2 t + \phi_2)= \frac{1}{2}\left(\cos\left(\frac{\omega_1 + \omega_2}{2}  t + \frac{\phi_1 + \phi_2}{2}\right) + \cos\left(\frac{\omega_1 - \omega_2}{2}t + \frac{\phi_1 -\phi_2}{2}\right)\right)
\end{gather*}
Mediando ora temporalmente questa espressione, si osserva che è zero.
Adesso l'intensità totale del campo mediata nel tempo sarà
\begin{gather*}
    I_{TOT}(t) = I_1 \sin^{2}\left(\pi\frac{\Delta L (t)}{\frac{\lambda_1}{2}}\right) + I_2\sin^{2}\left(\pi \frac{\Delta L(t)}{\frac{\lambda_2}{2}}\right)  
\end{gather*}




\end{document}
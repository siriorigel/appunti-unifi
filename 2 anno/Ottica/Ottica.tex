\documentclass[a4paper, oneside]{book}
\usepackage{graphicx}
\usepackage{amsthm}
\usepackage{amsmath}
\usepackage{amssymb}
\usepackage[a4paper,
            bindingoffset=0.2in,
            left=2cm,
            right=2cm,
            top=2cm,
            bottom=2cm,
            footskip=.25in]{geometry}
\usepackage[italian]{babel}
\usepackage{pgfplots}
\usepackage{tabularx}
\usepackage{tikz}
\usepackage{wrapfig}
\usepackage{color}
\usepackage[d]{esvect}
\definecolor{page}{rgb}{0.129,0.157,0.212}
\pagecolor{page}
\color{white}
\graphicspath{ {./images/} }
\usetikzlibrary{shapes.geometric}
\usetikzlibrary{datavisualization}
\usetikzlibrary{patterns}
\usetikzlibrary{datavisualization.formats.functions}
\usetikzlibrary{external}
\tikzexternalize[prefix=tikz/]
\pgfplotsset{width=10cm,compat=1.9}

\title{Appunti di Ottica}
\author{Tommaso Miliani}
\date{2025/2026}

\begin{document}
\newtheoremstyle{theoremEnv}
                {}          % Space above
                {}          % Space below
                {\slshape}  % Body font
                {}          % Indent amount
                {\bfseries} % Head font
                {.}         % Punctuation after head
                {\newline}         % Space after theorem head
                {}          % Theorem head spec
\theoremstyle{theoremEnv}

\newtheorem{definition}{Definizione}[chapter]
\newtheorem{theorem}{Teorema}[chapter]
\newtheorem{lemma}{Proposizione}[chapter]
\newtheorem{observation}{Osservazione}[chapter]
\newtheorem{corollary}{Corollario}[theorem]
\newtheorem{example}{Esempio}[chapter]

\maketitle

\tableofcontents

\chapter*{Introduzione al corso} 
Il corso di ottica si propone di studiare i quattro fenomeni principali dell'ottica:
\begin{enumerate}
    \item Ottica geometrica: approssimazione della luce come un insieme
    di raggi luminosi e come essi si propagano nel vuoto (lenti e formazioni
    di immagini);
    \item Polarizzazione: Dato che la luce è una onda elettromagnetica, e dato che
    il campo è un campo di vettori, il campo elettrico e luminoso è un vettore
    che oscilla nel tempo e questa oscillazione è proprio la polarizzazione;
    \item Diffrazione: il fenomeno per il quale la luce si diffonde;
    \item Interferenza: il fenomeno più complicato e bello della luce: due sorgenti luminose
    con campi opposti l'uno rispetto all'altro possono interferire distruttivamente
    e quindi non creare alcuna luce, ovviamente esiste anche l'interferenza costruttiva.
\end{enumerate}
Il peso di ogni relazione è un decimo sul totale del voto. Il restante sessanta percento
del voto è legato ad un compito in classe (molto simili l'uno all'altro). La media pesata di 
questi voti dà il voto finale (senza orale).
Non si potrà essere super rigorosi in questo corso, ma si daranno le basi dell'ottica attraverso
un approccio più pragmatico e semplice in quanto sarebbe necessario studiare prima fisica
due e fisica quantistica.

\part{Ottica geometrica}
\chapter{Le onde elettromagnetiche}
\section{Introduzione alle onde elettromagnetiche}
\begin{wrapfigure}{r}{0.4\textwidth}
    \centering
    \caption{Forze agenti tra due elettroni}
    \begin{tikzpicture}
        \filldraw(0, 0) circle (1pt) node[anchor=south] {$e_1$};
        \filldraw(3, 0) circle (1pt) node[anchor=south] {$e_2$};
        \draw[->](0, 0.05) -- (-1, 0.05) node[at end, above] {$\vv{F}$};
        \draw[->](3, 0) -- (4, 0) node[at end, above] {$\vv{F}$};
        \draw[->, red](0, -0.05) -- (-1, -0.05) node[at end, below] {$\vv{\Delta S}$ };
        \draw[->] (0, 0) -- (1, 0) node[at end, above] {$\vv{r}$ };
    \end{tikzpicture}    
\end{wrapfigure}
Prima di poter introdurre il concetto di onde elettromagnetiche è necessario, anche
se ancora non si è iniziato il corso di Fisica II, introdurre il concetto di elettromagnetismo:
prendiamo per esempio due elettroni, dall'elettromagnetismo si sa che si respingono
poiché tra di loro agisce una forza, chiamata forza di Coulomb, che ha modulo:
\begin{gather*}
    \vv{F} = k\frac{q_1q_2}{r^{2} } \hat{r}  
\end{gather*}
Questa forza dipende dalla carica delle due particelle e dalla direzione $r$ congiungente.
Se io spostassi il primo elettrone di un certo $\Delta S$, il secondo elettrone non si accorgerebbe
di questo spostamento in modo immediato. La relazione precedente non 
è dunque più valida se la particella si muove. Essendoci dunque un ritardo nel cambiamento dell'interazione rispetto 
all'istante in cui è avvenuto lo spostamento, siamo costretti ad introdurre due campi vettoriali che si propagano con una
velocità finita. Questi campi sono il campo magnetico ed il campo elettrico ($\vv{B}$ ed $\vv{E}$)
secondo delle leggi definite da Maxwell.
La forza che agisce sulle due cariche è un vettore e dunque si fa uso di un vettore per studiarne
il cambiamento: è possibile esprimere, tramite la legge di Lorentz, come una carica risente della presenza di un
campo elettrico e di un campo magnetico attraverso la seguente relazione:
\begin{align}
    \vv{F_q} = q(\vv{E} + \vv{v} \times \vv{B}  ) 
\end{align}
Dalle equazioni di Maxwell nel vuoto (ossia in assenza di cariche e correnti di cariche), si introduce
quindi l'operatore \textbf{Laplaciano quadro}, ossia un operatore differenziale
che si applica al campo elettrico e anche a quello magnetico secondo le seguenti: 
\begin{align}
    \vv{ \nabla}^{2} \vv{E} &= \epsilon_0 \mu_0 \frac{\partial^{2}  \vv{E} }{\partial t^{2} } \\
    \vv{ \nabla}^{2} \vv{B} &= \epsilon_0 \mu_0 \frac{\partial^{2}  \vv{B} }{\partial t^{2} } 
\end{align} 
L'operatore Laplaciano agisce su ogni componente del vettore campo elettrico o campo magnetico.
Si può vedere cosa succede per $\vv{E}_x$:
\begin{gather*}
    \epsilon_0 \mu_0 \frac{\partial^{2}  \vv{E}_x }{\partial t^{2} } = \frac{\partial ^{2} E_x }{\partial x^{2} } + \frac{\partial ^{2} E_x }{\partial y^{2} } + \frac{\partial^{2}  E_x}{\partial z^{2} } 
\end{gather*} 
Questa equazione esprime il contributo rispetto ad un singolo asse e prende il nome di \textbf{equazione delle onde} ed è analoga all'equazione delle onde
per tutte le onde che si studiano (sonore, luminose, radio ...):
\begin{align}
    \frac{1}{v^{2} } \frac{\partial^{2} \psi(x, t)}{\partial t^{2} } = \frac{\partial ^{2} \psi(x, t) }{\partial x^{2} }  
\end{align}
In questo caso $v$ rappresenta la velocità di propagazione dell'onda rispetto ad un sistema
di riferimento inerziale (per ora in quanto non si considera la relatività). Si può trovare una
soluzione generale a questa equazione come 
\begin{gather*}
    \psi(x, t) = f(x - vt)
\end{gather*}
Si considera una funzione generica $f$ in modo tale che abbia la stessa forma di $\psi$, dunque,
se calcolata in $x - vt$, sarà soluzione della funzione attraverso la doppia derivata rispetto
alle $x$:
\begin{gather*}
    \frac{\partial f}{\partial x} = f'(x - vt) \\
    \frac{\partial ^{2} f }{\partial x^{2} } = f''(x - vt)  \\
    \frac{\partial f}{\partial t} f'(x - vt) \cdot (-v)  \\
    \frac{\partial ^{2} f }{\partial t^{2} } = f''(x - vt) \cdot  (-v) \cdot (-v)
\end{gather*}
Risolvendo l'equazione con $\psi$ si può definire 
\begin{gather*}
    f''(x - vt) = \frac{1}{v^{2} } v^{2} f''(x - vt) 
\end{gather*}

\begin{wrapfigure}{r}{0.5\textwidth}
    \centering
    \caption{Il grafico della funzione generica $f(x)$}
    \begin{tikzpicture}
        \draw[->](-3, 0) -- (4, 0) node[at end, below] {$x$};
        \draw[->](0, -1) -- (0, 3) node[at end, left] {$f(x)$};
        \draw[cyan](-3, 0) .. controls(-2, 3.5)  and (0, 0) ..  (2, 0);
        \draw[cyan](2, 0) .. controls  (2.5, 1)  and (3, 2) .. (4, 0);
        \draw[dashed](-2.6, 0) -- (-2.6, 1) node[at start, below] {$x = \alpha$};
        \draw[dashed, red] (-2.6, 1) -- (2.75, 1) node[midway, above] {$f(\alpha)$};
        \draw[dashed](2.75, 1) -- (2.75, 0) node[at end, below] {$\alpha + vt$};
        \draw[red](-2.6, 0) -- (2.75, 0);
    \end{tikzpicture}    
\end{wrapfigure}
E quindi si è dimostrato che è soluzione generica dell'equazione.
Al tempo $t = 0$ si calcola la funzione $f(x - vt)$ e ci si chiede
dopo quale tempo $t$, e in quale punto dello spazio, la funzione $f$ assumerà
lo stesso valore. Se al tempo $f(\alpha)$ la funzione assume un 
determinato valore, assumerà lo stesso valore al punto $\alpha + vt$ dopo un certo
istante di tempo $t$ data la velocità di propagazione $v$. Dunque
l'onda si sarà mossa di una certa quantità $vt$ e dunque in quel punto il campo assumerà
nuovamente il valore che aveva nel punto $\alpha$. E' per questo che si parla di equazione
delle onde che si muovono di velocità $v$ poiché queste onde si propagano nello spazio
secondo $vt$.\\
Le costanti utilizzate fino ad ora sono la costante dielettrica nel vuoto
\begin{align}
    \epsilon_0= 8.85 \cdot  10^{-12}\  \frac{C^{2} }{m^{2}N } 
\end{align}
E la permeabilità magnetica nel vuoto
\begin{align}
    \mu_0 = 4\pi \cdot  10^{-7} \ \frac{T\cdot m}{A} 
\end{align}
La velocità di propagazione delle onde elettromagnetiche nel vuoto è proprio
\begin{align}
    c = \sqrt{\frac{1}{\epsilon_0 \mu_0}} = 299792458\ \frac{m}{s}  
\end{align}

\clearpage
\section{La soluzione semplice: le onde piane}
\begin{wrapfigure}{r}{0.4\textwidth}
    \centering
    \caption{Rappresentazione grafica dell'ipotesi}
    \begin{tikzpicture}
        \draw[->](0, 0) -- (2, 0) node[at end, below] {$x$};
        \draw[->](0, 0) -- (-1, -1) node[at end, below] {$z$};
        \draw[->](0, 0) -- (0, 2) node[at end, left] {$y$};
        \draw(-0.25, -0.25) -- (-0.25, 1.75);
        \draw(-0.75, -0.75) -- (-0.75, 1.25);
        \draw(0, 0.25) -- (-1, -0.75);
        \draw(0, 0.75) -- (-1, -0.25);
        \draw[->, cyan](-0.75, 0.5) -- (-0.25, 1) node[midway, above] {$\vv{E} $};
    \end{tikzpicture}    
\end{wrapfigure}
La soluzione più semplice all'equazione delle onde è l'onda piana:
questa equazione ha come ipotesi che $\vv{E}$ e $\vv{B}$ siano uniformi
in un piano perpendicolare alla direzione di propagazione. Considerando 
l'asse $x$ come l'asse di propagazione dell'onda e il piano $yz$ il piano sul quale entrambi i campi
sono uniformi, e dove le derivate rispetto agli assi $y$ e $z$ siano zero,
si ha l'ipotesi di onda piana
\begin{gather*}
    \epsilon_0 \mu_0 \frac{\partial^{2} E_x}{\partial x^{2} } = \frac{1}{c^{2} } \frac{\partial ^{2} E_x}{\partial t^{2} }  \\
    \epsilon_0 \mu_0 \frac{\partial^{2} E_y}{\partial y^{2} } = \frac{1}{c^{2} } \frac{\partial ^{2} E_y}{\partial t^{2} }  \\
    \epsilon_0 \mu_0 \frac{\partial^{2} E_z}{\partial z^{2} } = \frac{1}{c^{2} } \frac{\partial ^{2} E_z}{\partial t^{2} } 
\end{gather*}  
Si può ora assumere, dalle equazioni di Maxwell, che il vettore
campo elettrico non sia diretto lungo la direzione di propagazione dell'onda
($E_x = 0$), dunque le uniche componenti che rimangono sono le componenti
sul piano $yz$. 

Si tratteranno d'ora in poi onde sinusoidali, ossia le onde che hanno la seguente espressione:
\begin{align}
    E_{y, z} = A\cos\left(k(x - vt) + \phi\right)
\end{align}
\begin{wrapfigure}{r}{0.4\textwidth}
    \centering
    \caption{L'onda elettromagnetica}
    \begin{tikzpicture}[scale=0.5]
        \draw[->](-4, 0) -- (4, 0) node[at end, below] {$x$};
        \draw[->](0, 0) -- (0, 4) node[at end, left] {$E_{y, z}$};
        \draw[<->](-3, -1) --( 3, -1) node[midway, below] {$\lambda$};
        \draw[color=cyan]   plot (\x,{cos(\x r)});
    \end{tikzpicture}    
\end{wrapfigure}
Il periodo spaziale dopo il quale l'onda torna ad assumere lo stesso
valore (ossia dopo $2\pi$) si può ottenere imponendo che 
\begin{align}
    k \cdot  \lambda = 2\pi \ \Longrightarrow \ k = \frac{2\pi}{\lambda}
\end{align}
Si chiameranno, d'ora in poi, $\lambda$ come \textbf{lunghezza d'onda} e $k$ il
\textbf{vettore d'onda}, le quali sono intrinsecamente legate l'una all'altra. L'altra quantità
che si usa per descrivere le onde elettromagnetiche è la \textbf{frequenza} e la \textbf{pulsazione}.
Fissando spazialmente un certo valore di $x$ e lasciando che il tempo trascorra, 
il campo elettrico nel punto $x$ inizierà ad oscillare nel tempo con un periodo di
\begin{gather*}
    kv T = 2\pi \ \Longrightarrow \ T = \frac{2\pi}{c \cdot  k} \ \Longrightarrow \ T = \frac{2\pi}{\omega}
\end{gather*}
Chiamo allora $\omega = c \cdot  k$ la \textbf{pulsazione}. Sostituendo con l'espressione
della lunghezza d'onda ottengo
\begin{gather*}
    T = \frac{\lambda}{c} \ \Longrightarrow \ \omega = c \cdot  \frac{2\pi}{\lambda}
\end{gather*}
La direzione del campo elettrico è detta \textbf{polarizzazione} e, nel caso più semplice di onda
piana, la polarizzazione del campo, così come il modulo, è costante: si può dunque dire che il vettore
campo elettrico ha la seguente espressione:
\begin{gather*}
    \vv{E} = \vv{E}_0 \cos\left(kx - \omega t + \phi\right)
\end{gather*}  

\begin{wrapfigure}{r}{0.3\textwidth}
    \centering
    \caption{Propagazione dell'onda e fronti d'onda}
    \begin{tikzpicture}[scale=1.5]
        \draw[->](-0.5, -0.5) -- (1, 1) node[at end, below] {$\hat{k}$ };
        \draw[|-|](0.25, -0.25) -- (0.5, 0) node[midway, below] {$\lambda$};
        \draw(0.25, -0.25) -- (-0.25, 0.25);
        \draw(0.5, -0) -- (0, 0.5);
        \draw(0.75, 0.25) -- (0.25, 0.75);
    \end{tikzpicture}    
\end{wrapfigure}
In questo caso l'onda piana con la direzione del campo elettrico non cambia nel tempo ed è 
sempre costante nel piano $yz$, per cui vale la precedente. Se si volesse esprimerla nella direzione
generica, si avrebbe la seguente espressione per il vettore posizione e per il vettore campo elettrico:
\begin{gather*}
    \vv{r} = x \hat{u}_x + y\hat{u}_y + z\hat{u}_z    \\
    \vv{E} = \vv{E}_0 \cos\left(\hat{k} \cdot \vv{r} - \omega t + \phi \right)  
\end{gather*}
In questo caso il vettore $\hat{k}$ indica la direzione di propagazione
dell'onda; dunque si può risolvere il prodotto scalare con il vettore
direzione $\vv{r}$ e ottenere che la direzione di propagazione, ed il suo verso,
sono proprio quelli di $\hat{k}$. È possibile, inoltre, rappresentare le onde utilizzando solo il 
fronte d'onda (ossia il picco dell'onda) con larghezza $\lambda$.
Maxwell considera che il campo magnetico oscilla in fase rispetto al campo elettrico 
e quindi, tutte le volte che si ha un onda che si propaga nello spazio, oltre all'oscillazione
del campo elettrico, si ha anche l'oscillazione del campo magnetico, il quale oscilla
insieme all'onda in fase, ossia traslato rispetto al campo elettrico,
perpendicolarmente rispetto a quest'ultimo.



\begin{wrapfigure}{r}{0.4\textwidth}
    \centering
    \caption{Relazione tra campo elettrico e magnetico}
    \includegraphics[width=0.4\textwidth]{onde.png}
 \end{wrapfigure}
Sempre secondo le equazioni di Maxwell, il modulo del campo magnetico si rapporta al modulo del campo
elettrico secondo la seguente relazione:
\begin{align}
        \left|\vv{B}\right| = \frac{\left| \vv{E}  \right| }{c} 
\end{align}
Dato che la forza su di una carica è esprimibile come
\begin{gather*}
    \vv{F} = q\left(\vv{E} + \vv{v} \times \vv{B}   \right) 
\end{gather*}
È possibile esprimere la forza sulla particella come
$q\left| \vv{E}  \right| $ poiché nell'espressione del prodotto 
vettoriale si ottiene che il prodotto dei moduli sullo stesso asse diventa
\begin{gather*}
    q\left| \vv{v}  \right| \left| \vv{B}  \right| \sin \theta = q \frac{|\vv{v}| }{c}\left| \vv{E}  \right| \sin\theta   
\end{gather*} 
Data l'espressione del modulo del campo magnetico si può trascurare 
il termine $\frac{\left| \vv{v}  \right| }{c}$ solo se la velocità dell'onda è molto piccola (in generale solo nei solidi).

\subsection{Materiali dielettrici}
\begin{wrapfigure}{r}{0.4\textwidth}
    \centering
    \caption{Materiali dielettrici}
    \begin{tikzpicture}
        \draw(0, 0) circle (0.25);
        \draw(0, 0) circle (0.5);
        \filldraw(0, 0) circle (0pt) node[] {$+$};
        \filldraw(0, 0.5) circle (2 pt) node[anchor = south] {$e^ -$};
        \draw(3, 0) circle (0.25);
        \draw(3.5, 0) ellipse (1 and 0.5);
        \filldraw(3, 0) circle (0pt) node[] {$+$};
        \filldraw(3.5, 0.5) circle (2 pt) node[anchor = south] {$e^ -$};
        \draw[->](4, -1) -- (3, -1) node[midway, below] {$\vv{E}$};
    \end{tikzpicture}    
\end{wrapfigure}
Si può definire un \textbf{mezzo dielettrico} un mezzo nel quale gli elettroni sono liberi di muoversi in modo molto
limitato e dunque, avendo una ridotta mobilità, la componente del campo magnetico è molto più piccola rispetto al
campo elettrico. Un campo elettrico applicato alle cariche tende a far allontanare
gli elettroni dai nuclei, i quali si distribuiscono diversamente nello spazio
rispetto allo stato di quiete: la forza elettrica riesce dunque ad influenzare
la distribuzione delle cariche senza farle muovere liberamente (nei metalli, invece,
le cariche si possono muovere liberamente). \\
Quando un'onda entra in un materiale dielettrico questo modifica la lunghezza
d'onda e quindi la pulsazione dell'onda. La nuova lunghezza
d'onda, rispetto a quella che avrebbe l'onda nel vuoto, è data secondo la seguente relazione:
\begin{align}
    \lambda' = \frac{\lambda_v}{n}
\end{align}
Dove $\lambda_v$ è la lunghezza d'onda nel vuoto e $n$ \textbf{l'indice di rifrazione} del dielettrico in questione. 
Se l'indice di rifrazione fosse minore di $1$, allora la luce andrebbe più veloce
di $c$ anche se la sua intensità si attenuerebbe molto in quanto questi
dielettrici hanno un forte assorbimento. 
Quando l'angolo di inclinazione è perfettamente perpendicolare, allora il fronte
d'onda esterno al dielettrico è lo stesso che si ha all'interno del dielettrico.

\subsection{I metalli }
All'interno dei metalli gli elettroni si muovono liberamente e, se esposti ad un certo campo elettrico esterno, essi inizieranno ad oscillare
generando un campo elettrico in \textbf{controfase} al campo elettrico incidente,
annullando il campo elettrico in arrivo e incontrando
resistenza nel loro movimento; l'effetto è la liberazione di energia sottoforma di calore nel metallo. L'effetto complessivo
è che certe lunghezze d'onda sono assorbite mentre altre sono
riflesse. 

\clearpage
\subsection{Incidenza di onde piane ad un certo angolo e legge di Snell}
\begin{wrapfigure}{r}{0.5\textwidth}
    \centering
    \caption{Relazione tra fronti d'onda e angolo di incidenza}
    \begin{tikzpicture}
        \draw(2, 4) -- (2, -2.5);
        \filldraw(0.5, 4) circle (0pt) node[anchor = north] {Aria $n_1 = 1$};
        \filldraw(3.78, 4) circle (0pt) node[anchor = north] {Vetro $n_2 = 1.5$};
        \draw[dashed](-1, 0) -- (5, 0);
        \draw(-1, 2.5) -- (2, 0);
        \draw(1, 0) arc (180:140:1)  node[at start, below] {$\theta_i$};
        \draw(2, 0) -- (5, -1);
        \draw(4, 0) arc (0:-18:2) node[midway, right] {$\theta_f$};
        \draw(1.5, -0.75) -- (2, 0);
        \draw(2, 0) -- (2.75, 3);
        \draw(0.5, 0)-- (2, 2);
        \draw(2, 2) -- (2.5, 4);
        \draw(-1, 0) -- (2, 4);
        \draw[<->](2.75, 3) -- (2.3, 3.15) node[midway, above] {$\lambda'$};
        \draw[<->](0.75, 2.25) -- (1.7, 1.6) node[midway, above] {$\lambda$};
        \draw[cyan](2, 0) -- (2, 2) node[midway, left] {$D$};
        \draw(2, 2) arc (90:60:1) node[midway, above] {$\theta_f$};
        \draw(1.7, 1.6) arc(245:270:0.6) node[midway, below] {$\theta_i$};
    \end{tikzpicture}    
\end{wrapfigure}
Considerando un sistema di riferimento piano e un interfaccia a sinistra
della quale si ha l'aria (con indice di rifrazione
molto simile al vuoto), mentre dall'altra si ha il vetro (o qualsiasi
altro materiale con $n_2 \neq n_1$). Si può studiare cosa accade ai fronti d'onda che attraversano
l'interfaccia tra i due materiali con un angolo di incidenza diverso da zero. Si può vedere che l'onda nel dielettrico è un'onda
piana anch'essa in quanto si è sempre nella stessa ipotesi di onda piana. Si ha, inoltre, che
i fronti d'onda nel dielettrico sono gli stessi che al di fuori del dielettrico ma con
lunghezza d'onda minore. Per far sì che i fronti d'onda coincidano, fisicamente l'onda si inclina
rispetto all'angolo di incidenza $\theta_i$ di un certo angolo $\theta_f$ chiamato
angolo di \textbf{rifrazione}. L'ipotenusa identificata con $D$ al centro del disegno è
in relazione alle lunghezze d'onda e agli angoli di incidenza e di rifrazione secondo la seguente formulazione 
\begin{gather*}
    D \sin \theta_i = \lambda \\
    D \sin \theta_f = \lambda'
\end{gather*}
Combinando le due equazioni si ottiene l'utile relazione (considerando che
l'indice di rifrazione dell'aria sia $n_1$ e quello del vetro $n_2$):
\begin{gather*}
    \frac{\lambda}{n_2} \frac{1}{\sin\theta_f} = \frac{\lambda}{n_1} \frac{1}{\sin\theta_i} \ \Longrightarrow \ \frac{n_1}{n_2} = \frac{\sin\theta_f}{\sin\theta_i}
\end{gather*}
Ossia la \textbf{Legge di Snell}:
\begin{align}
    n_2 \sin\theta_f = n_1 \sin\theta_i
\end{align}
Le onde che sono riflesse sull'interfaccia, invece, vengono riflesse con un angolo di riflessione $\theta_f = \theta_i$ in quanto,
essendo questa una riflessione all'interno dello stesso dielettrico, secondo la legge di Snell se $n_1 = n_2$, allora 
l'angolo di riflessione è lo stesso con cui incidono. 

\subsection{Riflessione totale interna e prisma}
\begin{wrapfigure}{r}{0.4\textwidth}
    \centering
    \caption{Riflessione totale interna}
    \begin{tikzpicture}
        \draw[dashed](-2, 0) -- (2, 0);
        \draw(0, -2) -- (0, 2);
        \draw(-2, 0.5) -- (0, 0);
        \draw(0, 0) -- (2, -1.5);
        \draw(-1, 0.25) arc (160:180:0.75) node[midway, left] {$\theta_i$};
        \draw(-2, 2.5) -- (0, 0);
        \draw(-0.5, 0.62) arc(120:180:0.7) node[midway, above] {$\theta_c$};
        \draw(0, 0) -- (-1,-1.25);
        \draw[dashed](-1 , -1.25) -- (-2, -2.5);
        \filldraw(-1, 2) circle (0pt) node[anchor = center] {$n_2=$ vetro};
        \filldraw(1, 2) circle(0pt) node[anchor = center] {$n_1=$ aria};
    \end{tikzpicture}    
\end{wrapfigure}
Supponendo di avere sempre la stessa interfaccia e lo stesso sistema di riferimento,
supponiamo adesso che le onde arrivino da dentro il dielettrico con indice di rifrazione maggiore.
Quando queste arrivano all'interfaccia si rifrangono con un angolo
molto grande. Il raggio di luce dunque, sopra un certo angolo, tenderà a rifrangersi
con un angolo sempre più grande: in questo caso si arriva a dire che non si rifrangerà nessun raggio
di luce all'esterno del dielettrico; si parla allora di \textbf{riflessione totale interna}.
Questo si ha quando $\theta_i > \theta_c$, dove $\theta_c$ è l'angolo oltre al quale accade questo fenomeno,
che prende il nome di \textbf{angolo critico} oltre al quale
non si rifrangerà più alcun raggio di luce. L'angolo critico si ricava dalla legge di Snell:
\begin{gather*}
    n_2 \sin\theta_i = n_1 \sin\theta_f \\
        n_2 \sin\theta_c = n_1 \sin \frac{\pi}{2}
\end{gather*}
Allora l'angolo critico sarà
\begin{align}
    \sin\theta_c = \frac{n_1}{n_2}
\end{align}
Dove $n_2 > n_1$.  Nel caso del prisma, quando entra il laser dalla parte perpendicolare al
fascio di luce, la luce subisce un accorciamento della sua lunghezza d'onda mentre quando dall'interno
colpisce le pareti inclinate del prisma (a $45$ gradi e quindi maggiore dell'angolo critico),
essa è riflessa all'interno rimbalzando fino a che non esce dalla parte nuovamente perpendicolare
al fascio iniziale. Ruotando il prisma si può ottenere una parziale rifrazione della luce in modo da inclinare il fascio 
uscente.

\subsection{Uscita da un mezzo dielettrico}
\begin{wrapfigure}{r}{0.4\textwidth}
    \centering
    \caption{Deviazione della luce}
    \begin{tikzpicture}
        \draw(2, 2) -- (2, -2);
        \draw(3, 2) -- (3, -2);
        \filldraw(2.5, 2) node[anchor = center] {vetro};
        \draw(1, 1) -- (2, 0);
        \draw(2, 0) -- (3, -0.5);
        \draw(3, -0.5) -- (4, -1.5);
        \draw[dashed, very thin](1, 1) -- (4, -2);
        \draw[|-|] (4, -2) -- (4.25, -1.75) node[midway, right] {$S$};
        \draw[<->](2.1, -2) -- (2.9, -2) node[midway, above] {$L$};
    \end{tikzpicture}    
\end{wrapfigure}
Quando un fascio di luce entra in un mezzo e poi ne esce,
questo esce con lo stesso angolo in entrata ma con un discostamento
rispetto alla direzione iniziale di entrata nel mezzo. Questo è dato dal fatto
che la legge di Snell modifica l'angolo in entrata e dunque l'angolo in uscita si ottiene
applicando nuovamente la legge al contrario. \\
All'interno di un prisma è possibile far cambiare la direzione della luce completamente
in quanto si utilizza la rifrazione per creare raggi con angoli diversi
a seconda della lunghezza d'onda in entrata: il fascio di luce bianca che entra dentro
un prisma è scomposto in tanti fasci luminosi di lunghezza d'onda diversa
tutti distinti ed osservabili. Si può ottenere lo  scostamento (ossia la distanza
tra il raggio di luce rifranto e quello che sarebbe passato indisturbato) in funzione 
della larghezza del dielettrico, del suo indice di rifrazione e dell'angolo di incidenza:
\begin{align}
    S = \frac{D \sin(\theta_i - \theta_f)}{\cos\theta_f}
\end{align}
Dove $\theta_f$ si ricava attraverso la legge di Snell:
\begin{gather*}
    \theta_f = \arcsin \left(\frac{n_1\sin\theta_i}{n_2}\right)
\end{gather*}


\chapter{Le lenti}
\section{Introduzione alle lenti sferiche}
\begin{wrapfigure}{r}{0.5\textwidth}
    \centering
    \caption{Le lenti}
    \begin{tikzpicture}
        \draw(0, 2) arc (90:270: 2);
        \draw[dashed](-5, 0) -- (2, 0);
        \filldraw(-5, 0) circle (1pt) node[anchor = east] {$P$};
        \draw[green](-5, 0) -- (-1, 1.73);
        \filldraw(-1, 1.73) circle (1pt) node[anchor = south] {$C$};
        \filldraw(0, 0) circle (1pt) node[anchor = north] {$O$};
        \filldraw(2, 0) circle (1pt) node[anchor = west] {$Q$};
        \draw(-1, 1.73) -- (2, 0);
        \draw(1.5, 0) arc(180:150:0.5) node[at start, below] {$\alpha$};
        \draw[cyan](-1, 1.73) -- (-1, 0) node[near end, left] {$h$};
        \draw[green](-4, 0.45) arc(25:0:1.1) node[midway, right] {$\theta$};
        \draw[<->, green](-5, -0.5) -- (-1.1, -0.5) node[midway, below] {$p$};
        \draw[<->, cyan](-0.9, -0.5) -- (2, -0.5) node[midway, below] {$q$};
        \draw[dashed](0, 0) -- (-1.4, 2.4);
        \draw(-1.25, 2.2) arc(110:195:0.6) node[midway, left] {$\theta_i$};
        \draw[cyan](-0.2, 0.32) arc(120:180:0.35) node[midway, left] {$\phi$};
        \draw(-0.27, 1.25) arc(-25:-60:0.7) node[midway, right] {$\theta_r$};
    \end{tikzpicture}    
\end{wrapfigure}
Supponendo di avere una superficie sferica di vetro curvilinea all'aria, si vuole vedere cosa accade se le
onde piane della luce la colpiscono con un certo angolo rispetto alla direzione del raggio della sfera.
Chiamo $\theta$ l'angolo di inclinazione rispetto alla congiungente
sorgente-centro della circonferenza della lente e $p$ la lunghezza dalla sorgente alla congiungente
della verticale dal punto di contatto all'orizzontale.
Si può supporre che se l'angolo $\theta << 1$, allora
si può approssimare la lunghezza $p$ come la distanza da $P$
alla superficie di contatto della lente (che prende il nome di ipotesi \textbf{parassiale}). 
Chiamato allora $\theta_i$ l'angolo
di incidenza sulla lente e $\theta_r$ l'angolo di rifrazione dovuto alla lente rispetto alla
congiungente tra il punto di contatto $C$ e il centro $O$ della circonferenza. Si ottengono
degli altri angoli: $\phi$, ossia l'angolo tra l'orizzontale e la congiungente $OC$ e 
$\alpha$, ossia l'angolo che forma la congiungente $PQ$ rispetto all'orizzontale. Chiamato ora
 $q$ come $PQ - p$ e $Q$ come il punto di contatto tra il fascio
e l'orizzontale e l'angolo $\alpha$ di incidenza del fascio rifranto sull'orizzontale,
si ottiene allora la seguente relazione vicino al punto di contatto $C$:
\begin{gather*}
    \theta_i + \left(\frac{\pi}{2} - \theta\right) + \left(\frac{\pi}{2} - \phi\right) = \pi \ \Longrightarrow \ \theta_i = \theta +\phi \qquad (1)
\end{gather*}
Inoltre si ottiene un'altra relazione
\begin{gather*}
    \left(\frac{\pi}{2} - \phi\right) + \theta_r = \frac{\pi}{2} - \alpha \ \Longrightarrow \ \alpha = \phi - \theta_r  \qquad\qquad\   (2)
\end{gather*}
Inoltre si ha che
\begin{gather*}
    h = q\tan\alpha = p\tan\theta = R\sin\phi
\end{gather*}
Dato che si è nell'ipotesi parassiale, le relazioni per $h$ sono tutte uguali l'una all'altra, infatti, per $\theta << 1$
si ha che $\sin\theta \approx \theta, \ \tan\theta \approx \theta$:
\begin{gather*}
    h = q\alpha = p\theta = R\phi \qquad (3, 4, 5)
\end{gather*}
Rifacendosi alla legge di Snell, e ricordando l'ipotesi parassiale, si ottiene che
\begin{gather*}
    n_{\text{aria}}\sin\theta_i = n_{\text{vetro}}\sin\theta_r \ \Longrightarrow \ n_{\text{aria}} \theta_i = n_{\text{vetro}} \theta_r
\end{gather*}
È possibile far sparire gli angoli utilizzando le varie relazioni:
utilizzando intanto la $(1)$, poi la $(2)$ si ottiene la seguente:
\begin{gather*}
    n_{\text{aria}}(\theta + \phi) = n_{\text{vetro}}(\phi - \alpha)
\end{gather*}
Utilizzando la $(3, 4, 5)$ si eliminano definitivamente
gli angoli:
\begin{gather*}
    n_{\text{aria}} \left(\frac{h}{p} + \frac{h}{R}\right) = n_{\text{vetro}} \left(\frac{h}{R} - \frac{h}{q}\right)
\end{gather*}
Semplificando dunque $h$ si ha:
\begin{align}
    n_{\text{aria}}\left(\frac{1}{p} + \frac{1}{R}\right) = n_{\text{vetro}} \left(\frac{1}{R} - \frac{1}{q}\right)
\end{align}
Con questa equazione tutti i raggi emessi dalla sorgente
$P$ giungono tutti nel medesimo punto $Q$ (solamente nel limite parassiale).
Se si ripetesse il conto con un'altra interfaccia sferica vetrosa di raggio $R_2 < R$, allora si scoprirà che il
punto di convergenza di tutti i raggi luminosi è dato dalla seguente relazione:
\begin{gather*}
    \frac{1}{p} + \frac{1}{q} = \left(\frac{1}{R_1} - \frac{1}{R_2}\right) \frac{n _{\text{vetro}} - n_{\text{aria}}}{n_{\text{aria}}} + \frac{n_{\text{vetro}}d}{p(p - d)}
\end{gather*} 

\section{Le lenti sottili}
\begin{wrapfigure}{r}{0.4\textwidth}
    \centering
    \caption{La lente sottile}
    \begin{tikzpicture}
        \draw[dashed](0, 0) -- (5, 0);
        \draw(0, 0) -- (1.15, 1.15);
        \draw(1.5, 1.5) arc (130:230:2);
        \filldraw(0, 0) node[anchor = east] {$P$};
        \draw(1.7, 1.5) arc (160:200:4.5);
        \draw[<->, thick, green] (0.8, 0) -- (1.4, 0) node[midway, above] {$d$};
        \draw(1.15, 1.15) -- (1.6, 1.4) -- (4.5, 0);
        \draw[<->, cyan](0, -0.1) -- (0.75, -0.1) node[midway, below] {$p$};
        \draw[<->, cyan](1.45, -0.1) -- (4.5, -0.1) node[midway, below] {$q$};
    \end{tikzpicture}    
\end{wrapfigure}
Si può ora analizzare il comportamento di una \textbf{lente sottile}, ossia una lente che
ha ha uno spessore molto sottile. Considerando sempre una sorgente posta in $P$ e detta $p$ la distanza
tra la sorgente e la prima interfaccia; si può chiamare $d$ come
lo spessore della lente e $q$ come la distanza tra il punto di arrivo
sull'orizzontale del fascio di luce e la seconda interfaccia.
$R_1$ e $R_2$ sono $>0$ se il centro delle curvature si trova
a destra della superficie sferica. Si ottiene da questo modello
la seguente relazione
\begin{gather*}
    \frac{1}{p} + \frac{1}{q} = \frac{n_2 - n_1}{n_1}\left(\frac{1}{R_1} - \frac{1}{R_2}\right)+ \frac{n_2 d}{p(p - d)}
\end{gather*}
Dove $n_2$ è il coefficiente di rifrazione della lente e $n_1$ quello dell'aria. 
Dato che si sta lavorando con l'ipotesi di lente sottile, questo vuol dire che $d$ è molto piccolo: in questo modo si
semplifica la formula in questione con la seguente:
\begin{align}
    \frac{1}{p} +\frac{1}{q} = \frac{n_2 - n_1}{n_1} \left(\frac{1}{R_1} - \frac{1}{R_2}\right) = \frac{1}{f}
\end{align} 
Dove $f$ è chiamata \textbf{lunghezza focale} e questa relazione prende il nome di
\textbf{formula del costruttore di lenti} poiché per realizzare una lente
con una certa focale si devono conoscere i raggi di curvatura delle interfacce e il loro
coefficiente di rifrazione. Si ottiene anche la \textbf{legge delle lenti sottili}
\begin{align}
    \frac{1}{p}+ \frac{1}{q} = \frac{1}{f}
\end{align}
Dato il segno di $f$ si possono definire due tipologie di lenti:
\begin{itemize}
    \item Se $f > 0$ le lenti prendono il nome di \textbf{lenti convergenti}.
    \item Se $f < 0$ le lenti prendono il nome di \textbf{lenti divergenti}.
\end{itemize}

\subsection{La lente convergente}
\begin{wrapfigure}{r}{0.4\textwidth}
    \centering
    \caption{Caso A}
    \begin{tikzpicture}
        \draw(-2.5, 0) -- (2.5, 0);
        \draw[<->](0, -2) -- (0, 2) node[at end, above] {$f > 0$};
        \draw[->](-2, 2) -- (-1, 2);
        \draw[->](-2, 1) -- (-1, 1);
        \draw[->](-2, -1) -- (-1, -1);       
        \draw[->](-2, -2) -- (-1, -2);
        \filldraw (-1, 2 ) circle (0pt) node[anchor = south] {$p \to \infty$};
        \filldraw(1.5, 0) circle (1pt) node[anchor = south] {$Q$};
        \draw(-1, 2) -- (0, 2);
        \draw(-1, 1) -- (0, 1);
        \draw(-1, -1) -- (0, -1);
        \draw(-1, -2) -- (0, -2);
        \draw(0, 2) -- (1.5, 0);
        \draw(0, 1) -- (1.5, 0);
        \draw(0, -1) -- (1.5, 0);
        \draw(0, -2) -- (1.5, 0);
    \end{tikzpicture}    
\end{wrapfigure}
Le lenti convergenti sono un tipo di lente in grado di convergere i raggi luminosi in un unico punto, chiamato fuoco.
Si possono studiare i casi limite per questa tipologia di lenti per capire il significato fisico di $f$ e di come
le lenti convergenti, a seconda della distanza della sorgente
luminosa, riescano a far convergere i raggi luminosi. Si distinguono 
dunque quattro casi limite ben definiti: 
\begin{itemize}
    \item $p \to +\infty $: tutti i raggi convergono nel medesimo punto $Q$:
    \begin{gather*}
        \frac{1}{f} = \frac{1}{q} \ \Longrightarrow \ f = q
    \end{gather*}
    \item $p = f$: la lente non riesce a 
    far convergere i fasci luminosi ma solo a defletterli in modo tale che $q \to +\infty$:
    \begin{gather*}
        \frac{1}{q} = 0 \ \Longrightarrow \ q \to +\infty 
    \end{gather*}
\end{itemize}\mbox{}
\begin{itemize}
    \item Il terzo caso è quello della configurazione $2f - 2f$, ossia la configurazione
    in cui $p = 2f$ 
    \begin{gather*}
        \frac{1}{2f} + \frac{1}{q} = \frac{1}{f} \ \Longrightarrow \ \frac{1}{q} = \frac{1}{f} - \frac{1}{2f} \ \Longrightarrow \ \frac{1}{q} = \frac{1}{2f}
    \end{gather*}
    E quindi anche $q = 2f$, in questo modo i raggi che partono da $P$
    convergono tutti sul punto simmetrico rispetto alla lente $Q$.
    \item L'ultima configurazione è quella corrisponde al caso $p < f$: esplicitando $q$:
    \begin{gather*}
        \frac{1}{q} = \frac{p - f}{fp} \ \Longrightarrow \ \frac{fp}{p - f} = q
    \end{gather*}
    Dato che $q < 0$, il punto $Q$ si troverà (se la sorgente è a sinistra della lente)
    a sinistra della lente stessa, individuando un punto di \textbf{immagine virtuale} 
    dove convergono le direzioni di tutti i fasci deflessi.
    La lente non riesce dunque a far convergere i fasci di luce nel fuoco
    ma riesce solo a defletterli. 
\end{itemize}
Per una lente \textbf{convergente} i punti a distanza $f$ dalla lente (supponendo sempre che
la sorgente sia a sinistra rispetto alla lente) prendono il nome di \textbf{fuochi}: in particolare
il punto a destra della lente si chiamerà
\textbf{fuoco primario} della lente convergente, mentre il fuoco a sinistra della lente
è chiamato \textbf{fuoco secondario}.
Se la formazione dell'immagine è a destra della lente si dice che
si ha una \textbf{immagine reale della sorgente}, altrimenti se la formazione
avviene a sinistra della lente si parla di \textbf{immagine virtuale della sorgente}.
L'immagine è definita come il luogo dei punti dove si intersecano fisicamente
i raggi provenienti da fuori della lente; nel caso dell'immagine virtuale sono i
prolungamenti dei fasci di luce deflessi. 

\subsection{La lente divergente}
\begin{wrapfigure}{r}{0.4\textwidth}
    \centering
    \caption{La lente divergente}
    \begin{tikzpicture}
        \draw(-3, 0) -- (3, 0);
        \draw[>-<](0, 2) -- (0, -2);
        \draw[->](-2, 1) -- (-1, 1);
        \draw[->](-2, -1) -- (-1, -1);
        \draw(-1, 1) -- (0, 1);
        \draw(-1, -1) -- (0, -1);
        \draw[->](0, 1) -- (2, 2);
        \draw[->](0, -1) -- (2, -2);
        \draw[red](0, 1) -- (-2, 0);
        \draw[red](0, -1) -- (-2, 0);
        \filldraw[red] (-2, 0) node[anchor = south] {$Q$};
    \end{tikzpicture}    
\end{wrapfigure}
Le lenti divergenti sono un tipo di lente ottica che ha la capacità di disperdere i raggi luminosi che le attraversano.
La legge delle lenti sottili è una valida descrizione anche per questa tipologia di lenti ma,
rispetto alla lenti convergenti, viene studiato solamente il caso in cui la 
sorgente sia posta all'infinito. La legge delle lenti sottili mette in relazione
\begin{gather*}
    \frac{1}{q} = \frac{1}{f} \Longrightarrow \ q=f \quad q < 0
\end{gather*}
La lente produrrà dunque delle immagini virtuali sul fuoco secondario: per questo 
tipo di lenti dunque i fuochi si invertono, così che il fuoco secondario
diventerà il fuoco primario e viceversa. Non esistono invece delle immagini reali create
da questa lente in quanto i raggi che giungono dall'infinito
vengono sempre deflessi (ossia toccano la lente e vengono rifratti a differenza della
lente convergente, la quale, invece, fa sì che i fasci luminosi convergano in un medesimo punto $Q$).
La schematizzazione in figura permette di capire perché non esistono immagini
reali generati da questa tipologia di lenti.

\clearpage
\section{Il problema della formazione delle immagini}
\begin{wrapfigure}{r}{0.4\textwidth}
    \centering
    \caption{Lente convergente e formazione dell'immagine $h'$}
    \begin{tikzpicture}        
        \filldraw[purple] (0, 0) -- (1.7, -0.7) -- (0, -0.7);
        \filldraw[purple] (0, 0) -- (0, 1) -- (-2.5, 1);
        \filldraw[red](0, 0) -- (0, 1) -- (1, 0);
        \filldraw[red](1, 0) -- (1.7, -0.7) -- (1.7, 0);
        \filldraw[green](-1, 0) -- (0, 0) -- (0, -0.7);
        \filldraw[green](-1, 0) -- (-2.5, 0) -- (-2.5, 1);
        \draw(-3, 0) -- (3, 0);
        \draw[<->](0, -2) -- (0, 2);
        \filldraw(1, 0) circle(1pt) node[anchor = north] {$F$};
        \filldraw(-1, 0) circle (1pt) node[anchor = north] {$F'$};
        \draw[->](-2.5, 0) -- (-2.5, 1) node[midway, left] {$h$};
        \filldraw(-2.5, 1) circle (1pt) node[anchor = south] {$P'$};
        \draw(-2.5, 1) -- (0, 1);
        \draw(0, 1) -- (2, -1) node[at end, right] {$1$}; 
        \draw(-2.5, 1) -- (0, -0.7);
        \draw(0, -0.7) -- (2, -0.7);
        \filldraw(1.7, -0.7) circle (1pt) node[anchor = north east] {$Q'$};
        \filldraw(-2.5, 0) circle (1pt) node[anchor = north] {$P$};
        \filldraw(1.5, 0) circle (1pt) node[anchor = south] {$Q$};
        \draw[<->](1.7, 0) -- (1.7, -0.7) node[midway, right] {$h'$};
        \draw(-2.5, 1) -- (1.7, -0.7);
    \end{tikzpicture}    
\end{wrapfigure}
Fino ad ora si è considerata la sorgente luminosa come puntiforme, cosa
accade dunque alla formazione delle immagini da parte di una lente per una 
immagine estesa? Consideriamo una sorgente $P'$ posta fuori dall'asse ottico,
esso sta quindi ad una certa distanza $h$ rispetto all'asse ottico. Si sa, per le proprietà delle lenti convergenti,
che tutti i raggi che vengono dall'infinito convergono sul fuoco: il raggio che viene da $P'$,
ed è parallelo all'asse ottico, deve passare per il fuoco $F$. Il raggio che invece parte da $P'$ passando per
il fuoco $F'$ deve necessariamente diventare perpendicolare alla lente. Il punto $Q$
dipende dalla distanza della sorgente rispetto alla lente (il punto $Q$ coincide solo con $F$ quando
i raggi vengono dall'infinito). Si può considerare quindi $q$ come la distanza dal punto $Q$
e $q'$ come la distanza da $Q'$ (i quali non sono necessariamente sullo stesso piano). Concentrandosi sui due triangoli simili rossi si ricava la seguente relazione:
\begin{gather*}
    \frac{q' - f}{f} = \frac{h'}{h}
\end{gather*}
Concentrandosi sui triangoli verdi si ottiene invece
\begin{gather*}
    \frac{p - f}{f} = \frac{h}{h'}
\end{gather*}
Uguagliandole si ottiene la seguente espressione per $q'$:
\begin{gather*}
    \frac{q' - f}{f} = \frac{f}{p - f} \ \Longrightarrow \ q' - f = \frac{f^{2} }{p - f} \ \Longrightarrow \ q' = \frac{fp}{p - f}
\end{gather*}
Dato che, dalla legge delle lenti sottili, si ottiene il valore di $q$ come
\begin{gather*}
    q = \frac{pf}{p - f} \ \Longrightarrow \ q' = q 
\end{gather*}
Si è dunque dimostrato che $q = q'$ e dunque i punti $Q$ e $Q'$ giacciono
sullo stesso piano perpendicolare all'asse ottico.
I triangoli simili viola si ottengono poiché il fascio di luce che viene da $P'$, 
passando per il centro della lente, giunge in $Q'$ dato che non c'è alcuna
deflessione nel caso di una lente sottile. Considerando l'equazione trovata prima per
il rapporto tra $h$ e $h'$ 
\begin{gather*}
    \frac{p -f}{f} = \frac{h}{h'}
\end{gather*}
Moltiplicando per $\frac{p}{p}$ si ottiene la formula dell'\textbf{ingrandimento}:
\begin{align}
    I = \frac{h'}{h} = -\frac{q}{p}
\end{align}
Il segno meno ricorda che l'immagine si ribalta rispetto alla sorgente
nel caso in cui sia $p$ che $q$ siano positivi.

\clearpage
\subsection{Sbagliare il piano d'immagine}
\begin{wrapfigure}{r}{0.4\textwidth}
    \centering
    \caption{Piano di immagine scostato di $\delta x$ rispetto al piano 
    su cui l'immagine osservata risulta a fuoco}
    \begin{tikzpicture}
        \draw(-3, 0) -- (3, 0);
        \draw[<->](0, -2) -- (0, 2);
        \draw(-2, 1) -- (0, -1);
        \draw(-2, 1) -- (0, 1);
        \draw(0, 1) -- (2, -1);
        \draw(0, -1) -- (2, -1);
        \draw(-2, 1) -- (2, -1);
        \filldraw(1, 0) circle (1pt);
        \filldraw(-1, 0) circle (1pt);
        \draw[orange, thick](0, 1) -- (0, -1) node[midway, anchor = south west] {$D$};
        \draw(2.5, 2) -- (2.5, -2);
        \draw(2, -1) -- (2.5, -1.5);
        \draw(2, -1) -- (2.5, -1.25);
        \draw(2, -1) -- (2.5, -1);
        \draw[thick, orange] (2.5, -1) -- (2.5, -1.5) node[midway, right] {$d$};
        \draw[dashed] (2, 2) -- (2, -2);
        \filldraw(2.25, 2) node[anchor = south] {$\delta x$};
        \draw[<->](0.1, 1) -- (1.9, 1) node[midway, above] {$q$};
    \end{tikzpicture}    
\end{wrapfigure}
Quando si scatta una foto non si riesce a mettere a fuoco tutti gli
oggetti di una scena ma ci si deve concentrare o, in termini di lenti, \textbf{mettere a fuoco} solo 
gli oggetti che risiedono su di un piano immagine ad una determinata distanza
dall'obiettivo della fotocamera. Sbagliando il piano dell'immagine, 
si potrebbe chiamare la distanza del punto di formazione dell'immagine dalla lente 
come $q + \delta x$ e $D$ come la dimensione della lente; analogamente si può chiamare $d$ la
dimensione dell'immagine che risiede sul piano immagine sbagliato. 
L'unico modo che si ha per mettere a fuoco l'immagine ora è
ridurre la dimensione della lente, infatti, per considerazioni geometriche,
si ottiene la seguente relazione:
\begin{gather*}
    \frac{d}{D} \approx \frac{\delta x}{q} \ \Longrightarrow \ d = \frac{\delta x}{q}D
\end{gather*}
Dove $\delta x$ è la distanza tra il piano dell'immagine "giusto" e quello
sbagliato su cui c'è $d$: in questo modo si può ridurre lo spread
dei raggi luminosi ed ottenere un'immagine nitida. E' simile a quando si cerca di
strizzare l'occhio per vedere meglio, anche se in quel caso
si modifica anche il cristallino. Come è possibile che in alcune immagini sia sempre tutto a fuoco? 
È possibile spingere al limite il concetto di messa a fuoco attraverso una \textbf{camera stenopeica}: ossia una camera
che permette di avere tutti gli oggetti di una scena a fuoco sul medesimo 
piano immagine. Essa si basa sul diminuire la dimensione della lente $D$ in modo tale che tutto risulti a fuoco.
Il prezzo da pagare è che all'interno di questo strumento passa pochissima luce, perdendo dunque
molta della luminosità dell'immagine. 

\section{La legge del costruttore di lenti e i vari tipi di lenti}
Dalla legge del costruttore di lenti si può ricavare la lunghezza della focale
secondo la seguente relazione:
\begin{gather*}
    f = \frac{n_1}{n_2 - n_1}\left(\frac{R_1R_2}{R_2 - R_1}\right) 
\end{gather*}
Da questa relazione esistono tre tipologie di lenti

\subsection{Lente piano convessa}
\begin{wrapfigure}{r}{0.2\textwidth}
    \centering
    \begin{tikzpicture}
        \draw(0, 0) -- (2, 0);
        \draw(1, 1) -- (1, -1);
        \draw(1, 1) arc(150:210:2);
        \draw(1.5, 0) -- (0.78, 0.55) node[midway, above] {$R$}; 
    \end{tikzpicture}    
\end{wrapfigure}
Questa lente ha una interfaccia con raggio di curvatura
definito ed una interfaccia il cui raggio di curvatura è assimilabile a
infinito. Dal costruttore di lenti si ottiene dunque la 
lunghezza focale come
\begin{gather*}
    f = \frac{n_1}{n_2 - n_1} R
\end{gather*}
Più la lente è piccola e più la lunghezza focale è piccola e dunque più la lente ha
la capacità di far convergere i fasci luminosi. In altre parole, il potere
convergente della lente è inversamente proporzionale alla lunghezza focale.

\subsection{Lente biconvessa}
\begin{wrapfigure}{r}{0.2\textwidth}
    \centering
    \begin{tikzpicture}
        \draw(-0.2, 1) arc(150:210:2);
        \draw(0.2, 1) arc(30:-30:2);
        \draw(-0.2, 1) -- (0.2, 1);
        \draw(-0.2, -1) -- (0.2, -1);
        \draw(-1, 0) -- (1, 0);
    \end{tikzpicture}    
\end{wrapfigure}
La lente biconvessa è una lente che non ha più una superficie piana ma ha due
superficie sferiche con $R_1 > 0$ e $R_2 < 0$. Nel caso in cui $|R_1| = |R_2| = |R|$ si
ha la seguente relazione:
\begin{gather*}
    f = \frac{n_1}{n_2 - n_1} \left(\frac{R(-R)}{-2R}\right)
\end{gather*}
Complessivamente la mia focale sarà data dalla seguente:
\begin{gather*}
    f = \frac{n_1}{n_2 - n_1}\frac{R}{2} 
\end{gather*}
Il potere convergente è due volte maggiore rispetto ad una lente con
raggio $R$ (dato che la focale è la metà dell'altra lente). 

\clearpage
\subsection{La lente piano concava}
\begin{wrapfigure}{r}{0.2\textwidth}
    \centering
    \begin{tikzpicture}
        \draw(0, 0) -- (2, 0);
        \draw(1, -1) -- (2, -1);
        \draw(2, -1) -- (2, 1);
        \draw(2, 1) -- (1, 1);
        \draw(1, 1) arc(30:-30:2);
    \end{tikzpicture}    
\end{wrapfigure}
Questa tipologia di lente ha focale negativa ed è esattamente
l'opposta di una lente piano convessa: infatti la focale di questa lente sarà 
la stessa, in modulo, di una piano convessa ma con segno invertito:
\begin{gather*}
    f = -\frac{n_1}{n_2 - n_1}R
\end{gather*}

\section{Funzionamento dell'occhio umano}
\begin{wrapfigure}{r}{0.4\textwidth}
    \centering
    \begin{tikzpicture}
        \draw(1, 0) -- (4, 0);
        \draw[<->](2, -1) -- (2, 1);
        \draw(3, -1) -- (3, 1) node[near end, right] {retina};
        \draw[|-|](2.1, -1) -- (2.9, -1) node[midway, below] {$q$};
    \end{tikzpicture}    
\end{wrapfigure}
È possibile schematizzare l'occhio umano attraverso una lente con indice di rifrazione di 
circa $n = 1.4$. Sulla parte posteriore dell'occhio si ha la retina, ossia
il rilevatore dell'occhio, che è in grado di raccogliere la luce in entrata dal cristallino
che poi è trasferita al nostro cervello tramite il nervo ottico. La retina dell'occhio
umano è dunque equivalente al piano immagine. Poiché
non si riesce a cambiare la distanza $q = 0.02$ m tra il cristallino e la retina,
la messa a fuoco è possibile utilizzando solo dei muscoli
che contraggono il cristallino, il quale contribuisce solo per
un terzo del potere convergente dell'occhio. Gli altri due terzi
del potere convergente dell'occhio derivano dalla cornea, la quale è posta
davanti al cristallino ed immersa nell'humor acqueo mentre l'occhio 
è immerso nell'humor vitreo (ognuno con i suoi indici di rifrazione).
L'occhio è dunque l'analogo del sistema ottico in figura.

\begin{wrapfigure}{r}{0.4\textwidth}
    \centering
    \begin{tikzpicture}
        \draw(0, 0) -- (4.5, 0);
        \draw(3, -1) -- (3, 1);
        \draw(4, -1) -- (4, 1);
        \draw(0, 1) -- (3, 0) -- (4, -0.33);
        \draw(2, 0.33) arc(160:180:1) node[midway, left] {$\theta$};
        \draw(0, 0) -- (0, 1) node[midway, right] {$h$};
        \draw[cyan](4, 0) -- (4, -0.33) node[midway, right] {$h'$};
    \end{tikzpicture}    
\end{wrapfigure}
La distanza minima di messa a fuoco aumenta con l'età in quanto
diminuisce la capacità di strizzare il cristallino da parte dei muscoli del cristallino: la
distanza minima è di $p_{\text{min}} \approx 0.3$ m. Un occhio sano riesce inoltre a mettere a fuoco
fino ad una distanza $p_{\text{max}} = +\infty $.
La distanza focale dell'occhio è di $\sim0.02$ m per una sorgente
posta all'infinito. Invece, per un oggetto che si trova alla distanza
minima di messa a fuoco, la focale diminuisce a
\begin{gather*}
    \frac{1}{f} = \frac{q + p}{pq} \ \Longrightarrow \  f = \frac{pq}{p + q} \ \Longrightarrow \ f \approx 19.2 \text{ mm}
\end{gather*}
Si può ricavare l'ingrandimento dell'occhio tramite la formula di ingrandimento 
conoscendo la distanza $p$ dell'oggetto osservato. Dunque, banalmente, più gli oggetti sono vicini e più sono grandi all'interno del
nostro occhio dato che $q$ è una distanza fissa:
\begin{gather*}
    I = \frac{h'}{h} = -\frac{q}{p}
\end{gather*}

\clearpage
\section{Lenti di ingrandimento e sistema di lenti}
\subsection{Il principio di funzionamento di una lente di ingrandimento}
\begin{wrapfigure}{r}{0.5\textwidth}
    \centering
    \caption{Lente di ingrandimento}
    \begin{tikzpicture}
        \draw(-3, 0) -- (4, 0);
        \draw[<->](0, -2) -- (0, 2);
        \filldraw (-1.5, 0) circle (1pt) node[anchor = north] {$F'$};
        \filldraw (1.5, 0) circle (1pt) node[anchor = north] {$F$};
        \draw[<->](2.5,-1) -- (2.5, 1);
        \draw(3.5, -1) -- (3.5, 1);
        \draw(-1, 0) -- (-1, 1) node[midway, left] {$h$};
        \draw[red](-1, 1) -- (2.5, -2.5);
        \draw[red](-1, 1) -- (0,1) -- (1.5, 0) -- (2.5, -0.75);
        \draw[red](-1, 1) -- (-2.5, 2.5);
        \draw[red](-2.5, 2.5) -- (0, 1);
        \draw[red](-2.5, 0) -- (-2.5, 2.5) node[midway, left] {$h'$};
        \draw[green](-2.5, 2.5) -- (3.5, -0.5);
        \draw[green](-1, 1) -- (0, 1.25);
        \draw[cyan](-1, 1) -- (2.5, 0);
        \draw[cyan](0.75, 0) arc (180:145:0.75) node[midway, left] {$\theta$};
        \draw[green](1.5, 0.5) arc (140:180:0.75) node[at start, above] {$\theta'$};
    \end{tikzpicture}    
\end{wrapfigure}
Per poter ingrandire gli oggetti si utilizzano dei sistemi di lenti, ossia degli apparati
ottici in grado di combinare diverse lenti per poter ingrandire oggetti molto piccoli
o molto distanti in modo tale che questi abbiano una dimensione sufficiente per cui il nostro occhio riesce ad osservarli senza problemi.
Nel caso di una sola lente, se si ponesse un oggetto prima della focale, si osserverebbe che i raggi luminosi dell'oggetto
non si incontrano; l'oggetto apparirà quindi più grande
quando arriva all'occhio umano ad una certa distanza $d$. Data  la legge delle lenti sottili,
con $p < f$ si ha
\begin{gather*}
    \frac{1}{q} = \frac{p - f}{fp} < 0 
\end{gather*}
I raggi luminosi, per l'occhio, appaiono come se si formino prima dell'oggetto
stesso: l'occhio umano osserva allora la sorgente virtuale dell'immagine. Si può
studiare l'angolo con cui l'immagine virtuale $h'$ si interseca 
con l'occhio umano anche se il raggio verde è generato dall'oggetto $h$.
La distanza dove si forma l'immagine virtuale è $q$, si vuole ora trovare l'angolo
verde $\theta'$: si sa che, in assenza della lente, l'oggetto con altezza $h$
formerebbe un angolo
\begin{gather*}
    \tan \theta = \frac{h}{p + d}
\end{gather*}
L'angolo $\theta'$, invece, sarà dato dalla seguente espressione (con $q < 0$ in quanto
l'immagine $h'$ si forma a sinistra della lente):
\begin{align}
    \tan\theta' = \frac{h'}{q + d}
\end{align}
Possiamo ora determinare quale sarà l'angolo più grande. Dall'espressione
dell'ingrandimento possiamo ricavare $h'$ in funzione della lente e di $h$:
\begin{gather*}
    \frac{h'}{h} = -\frac{q}{p} = - \frac{pf}{p - f}\frac{1}{p} \ \Longrightarrow \ h' = \frac{f}{f  - p}h
\end{gather*}
Combinando le due espressioni della tangente si può determinare quale è l'angolo più grande
\begin{gather*}
    \tan\theta' = \frac{f}{f - p}\frac{h}{q + d}
\end{gather*}
Si riscrive $q$ attraverso le relazioni con $f$ e $p$ (si inverte $p- f$ con
$f - p$ a causa del segno di $q$):
\begin{gather*}
    q = \frac{pf}{f - p} \ \Longrightarrow \ \tan\theta' = \frac{f}{f - p}\frac{h}{\frac{pf + df -dp}{f - p}}
\end{gather*}
Confrontando le due espressioni direttamente
\begin{gather*}
    \tan\theta' = \frac{h}{p + d - \left(\frac{dp}{f}\right)} > \tan\theta = \frac{h}{p + d} 
\end{gather*}
Si osserva che $\theta' > \theta$ proprio perché il
denominatore è più piccolo, dunque l'immagine si ingrandisce.

\clearpage
\subsection{Sistema di lenti}
\begin{wrapfigure}{r}{0.4\textwidth}
    \centering
    \caption{Sistema di due lenti convergenti}
    \begin{tikzpicture}
        \draw(-3, 0) -- (3, 0);
        \draw[<->](0, -1.5) -- (0, 1.5) node[at start, below] {$f_1$};
        \draw[<->](1, -1.5) -- (1, 1.5) node[at start, below] {$f_2$};
        \draw[|-|](-2, -0.2) -- (-0.1, -0.2) node[midway, below] {$p$};
        \draw[|-|](0.1, 0.1) -- (0.9, 0.1) node[midway, above] {$d_{1,2}$};
        \draw[|-|](1.1, -0.2) -- (1.9, -0.2) node[midway, below] {$q'$};
    \end{tikzpicture}    
\end{wrapfigure}
In questo paragrafo si vuole dimostrare il comportamento di un sistema di lenti
composto da due lenti convergenti: l'ipotesi è che tale sistema aumenti il potere convergente
e che quindi il sistema di lenti si possa comportare come una unica lente con un potere
convergente maggiore. Ci si aspetta dunque la focale totale sia la somma delle focali, tuttavia questa cosa
è sbagliata in quanto il potere convergente è tanto maggiore quanto più piccola la focale della lente:
allora il potere convergente deve essere direttamente proporzionale all'inverso
delle focali delle lenti
\begin{gather*}
    \frac{1}{f} \approx \frac{1}{f_1} + \frac{1}{f_2}
\end{gather*}
Questo è dimostrabile a partire dell'applicazione della legge delle lenti
sottili a tutto il sistema di lenti (così come si era fatto per l'occhio) e
quindi vale per qualsiasi distanza $d_{1, 2}$ tra le due lenti. 
\begin{gather*}
    d_{1, 2} = q + p' 
\end{gather*}
E quindi si ottiene 
\begin{align}
    \frac{1}{f} = \frac{1}{f_1} + \frac{1}{f_2}
\end{align}

\section{Principio di funzionamento del telescopio}
\begin{wrapfigure}{r}{0.55\textwidth}
    \centering
    \caption{Osservare un corpo celeste ad occhio nudo}
    \begin{tikzpicture}
        \draw(0, 0) -- (8, 0);
        \draw(7, -0.5) -- (7, 0.5);
        \draw[->](0, 0) -- (0, 0.5);
        \draw[cyan](0, 0.5) -- (7, 0);
        \draw(2, 0.38) arc (160:180:1) node[midway, left] {$\alpha$};
    \end{tikzpicture}    
\end{wrapfigure}
Per poter osservare oggetti molto lontani si fa utilizzo di un apparato
ottico che prende il nome di \textbf{telescopio}: il telescopio è composto 
da due lenti convergenti poste ad una certa distanza tra di loro in modo tale da ingrandire 
oggetti molto distanti dallo strumento. Consideriamo allora un asse ottico molto lungo alle cui estremità
abbiamo l'oggetto da osservare e dall'altra l'occhio: possiamo tracciare
quindi i fasci di luce che partono dall'oggetto e intersecano il centro del cristallino
dell'occhio. Esiste allora un certo angolo $\alpha$ tra i due fasci di luce: a seconda
dell'angolo, il corpo celeste avrà una certa dimensione sull'occhio. Se si avesse un
sistema ottico (ossia il telescopio) tra l'occhio e l'oggetto, potremmo posizionare due lenti in modo tale
che il fuoco primario della prima lente coincida con il fuoco secondario
della seconda lente. Inoltre, le lenti sono scelte in modo tale che $f_1 > f_2$.


\begin{wrapfigure}{r}{0.5\textwidth}
    \centering
    \caption{Osservare un corpo celeste tramite un telescopio}
    \begin{tikzpicture}[scale=1.4]
        \draw(0, 0) -- (5, 0);
        \draw[<->](2, -1) -- (2, 1);
        \draw[<->](4, -1) -- (4, 1);
        \draw[red](0, 0.75) -- (2, 0.75) -- (4, -0.25) -- (4.75, -0.25);
        \draw[cyan](0, 0.75) -- (2, 0) -- (3.5, -0.75) -- (4, -0.75) -- (5, 0.75);
        \draw[cyan](0, 1.5) -- (2, 0.75) -- (3.5, -0.75) -- (4.5, 0.75);
        \draw[very thick, cyan](3.5, 0) -- (3.5, -0.75) node[midway, left] {$h$};
        \draw[cyan](0.5, 0.55) arc (135:180:0.75) node[midway, left] {$\alpha$}; 
        \draw[cyan](0.5, 1.3) arc (135:180:0.75) node[midway, left] {$\alpha$}; 
        \filldraw(4.5, 0) circle (1pt) node[anchor = north west] {$F_2$};
        \filldraw(3.5, 0) circle (1pt)node[anchor = south] {$F_1$};
        \draw[cyan](4.75, 0.38) arc (62:0:0.4) node[midway, right] {$\beta$};
        \draw[cyan](4.25, 0.38) arc (62:0:0.4) node[midway, right] {$\beta$};
        \draw[very thick] (2, 0) -- (4, 0);
        \draw[very thick, cyan](2, 0) -- (3.5, -0.75);
        \draw[very thick, cyan](3.5, -0.75) -- (4, 0);
    \end{tikzpicture}    
\end{wrapfigure}
Così la distanza tra le due lenti è esattamente $d = f_1 + f_2$. Adesso si deve rappresentare
i raggi che arrivano dal corpo celeste sulle due lenti convergenti. Analizzando ora il telescopio,
i raggi molto vicini all'asse ottico appaiono paralleli tra di loro. Possiamo considerare il
raggio blu che arriva con l'angolo $\alpha$: nel momento in cui si pone un sistema di lenti
davanti all'occhio, i raggi vicini a quello blu arriveranno sulla lente
primaria del telescopio venendo fatti convergere sotto l'asse ottico. Per studiare ora
cosa accade ai raggi del corpo celeste alla destra del fuoco primario della prima lente, è sufficiente immaginare
$h$ (ossia l'immagine del corpo celeste dopo la prima lente) come sorgente posta prima della seconda lente
del telescopio. Osservando dunque il disegno si vede che i raggi blu che venivano dall'estremità finale del
corpo celeste, arrivano sul nostro occhio con un nuovo angolo $\beta$ più grande.  
Si ottengono le seguenti relazioni per i triangoli evidenziati dal tratto più spesso
\begin{gather*}
    \frac{h}{f_1} = \tan\alpha \qquad  \frac{h}{f_2} = \tan\beta
\end{gather*}
Si ha l'utile relazione come il rapporto tra le tangenti:
\begin{align}
    \frac{\tan\beta}{\tan\alpha} = \frac{f_1}{f_2}
\end{align}
Se il rapporto tra le tangenti è maggiore di 1 allora
\begin{gather*}
    \beta > \alpha \ \Longrightarrow \ I > 1
\end{gather*}
Dunque il telescopio riesce ad ingrandire l'oggetto che si sta osservando.  


\section{Principio di funzionamento del microscopio}
\begin{wrapfigure}{r}{0.5\textwidth}
    \centering
    \caption{Schematizzazione del microscopio}
    \begin{tikzpicture}
        \draw(0, 0) -- (6, 0);
        \draw(0, 0) -- (0, 0.5) node[midway, left] {$h$};
        \draw[<->](1, -1) -- (1, 1);
        \draw[<->](4, -2) -- (4, 2);
        \draw[cyan](0, 0.5) -- (1, 0.5) -- (4, -1) -- (6, -1);
        \draw[cyan](0, 0.5) -- (1, 0) -- (4, -1.5) -- (6, -1);
        \draw[green](6, 0) -- (6, -1) node[midway, right] {$h'$};
        \draw[dashed](2, -2) -- (2, 2);
        \filldraw[cyan](1, 0) -- (1, 0.5) -- (2, 0);
        \filldraw[cyan](2, 0) -- (4, 0) -- (4, -1);
        \draw[<->](2.1, 1) -- (3.9, 1) node[midway, above] {$f_2$};
        \draw[<->](1.1, 1) -- (1.9, 1) node[midway, above] {$f_1$};
    \end{tikzpicture}    
\end{wrapfigure}
Quando si vuole osservare un oggetto molto piccolo, invece di usare una sola
lente di ingrandimento, possiamo utilizzare un sistema di lenti
con l'oggetto posto nel fuoco secondario della prima lente convergente in modo
tale che $p=f_1$. Scelgo allora una seconda lente a focale lunga (l'inverso del telescopio)
così che la distanza tra le due lenti sia $d = f_1 + f_2$ e che quindi il fuoco
primario della prima lente coincida con il fuoco secondario della seconda lente.
Si può stimare geometricamente l'ingrandimento dell'immagine, secondo gli stessi procedimenti 
fatti per il telescopio, ottenendo la seguente espressione:
\begin{align}
    I = \frac{h'}{h} = \frac{f_2}{f_1}.
\end{align}
Dato che $f_2 > f_1$, segue che $I > 1$ e dunque l'immagine sarà
ingrandita. Se si volesse tenere conto del fatto che l'immagine sia ribaltata,
allora si dovrebbe mettere un meno davanti al rapporto per tenere conto del ribaltamento.

\part{Polarizzazione della luce}
\chapter{Polarizzazione delle onde elettromagnetiche}
\section{Definizione di polarizzazione}
\begin{wrapfigure}{r}{0.4\textwidth}
    \centering
    \caption{Polarizzazione di un onda che
    il cui vettore di propagazione è $\hat{x}$}
    \begin{tikzpicture}
        \draw[->](0, 0) -- (2, 0) node[at end, below] {$x$};
        \draw[->](0, 0) --(0, 1.5) node[at end, left] {$z$};
        \draw[->](0, 0) -- (-1, -1) node[at end, right] {$y$};
        \draw[->, green](0, 0) -- (-0.5, 0.75) node[at end, above] {$\vv{E}(x, t)$};
    \end{tikzpicture}    
\end{wrapfigure}
Si definisce \textbf{polarizzazione} la direzione del campo elettrico dell'onda elettromagnetica, si ricorda
che per una onda piana la polarizzazione giace in un piano che è perpendicolare
alla direzione di propagazione dell'onda $\hat{k}$. Il campo elettrico dell'onda sarà dato
in funzione sia della posizione che del tempo. Matematicamente il campo elettrico
si propagherà solamente sul piano $zy$ e dunque si può esprimere come 
\begin{gather*}
    \vv{E}(x, t) = E_{0z} \cos(kx - \omega t + \phi_z) \hat{z} + E_{0y}\cos(kx - \omega t + \phi_y) \hat{y}    
\end{gather*} 
Dove $\phi_z$ e $\phi_y$ sono le fasi del campo rispetto ai due assi. In questo caso
l'onda elettromagnetica si propaga lungo la direzione $\hat{x} \equiv \hat{k}$. Esistono
tre tipi di polarizzazione.

\subsection{Polarizzazione lineare}
\begin{wrapfigure}{r}{0.4\textwidth}
    \centering
    \caption{Campo elettrico sul piano $zy$, l'asse $x$ è uscente
    dal piano}
    \begin{tikzpicture}
        \draw[->](0, 0) -- (2, 0) node[at end, below] {$y$};
        \draw[->](0, 0) -- (0, 2) node[at end, left] {$z$};
        \draw[->, thick](0, 0) -- (1.5, 1.5) node[at end, right] {$\vv{E}$ };
        \draw[dashed, thin](0, 1.5) -- (1.5, 1.5) node[at start, left] {$E_{0z}$};
        \draw[dashed, thin](1.5, 0) -- (1.5, 1.5) node[at start, below] {$E_{0y}$};
        \draw[<-, cyan](0.7, 0.7) arc (45:0:1) node[midway, right] {$\alpha$};
    \end{tikzpicture}    
\end{wrapfigure}
La polarizzazione lineare consiste nella polarizzazione di onde il cui
campo elettrico, sul piano $zy$, oscilla con la stessa fase 
 $\phi_y = \phi_z$. Le due componenti $z, y$ del campo elettrico
oscillano in fase sia spazialmente che temporalmente e quindi la direzione
del campo elettrico totale rimane costante. Si esprime la tangente
dell'angolo in funzione del tempo come
\begin{gather*}
    \tan\alpha(t) = \frac{E_{0z} \cos(kx - \omega t + \phi_z)}{E_{0y}\cos(kx - \omega t + \phi_y)} = \frac{E_{0z}}{E_{0y}}
\end{gather*}
La seconda uguaglianza vale perché le fasi sono le stesse e dunque
l'angolo dipende solamente dal modulo delle componenti sulle $y$ e sulle $z$.

\subsection{Polarizzazione circolare}
La polarizzazione circolare è un altro caso particolare nel quale $E_{0y} = E_{0z}$, mentre le fasi 
risultano differire di $\frac{\pi}{2}$, per cui $\phi_y = \phi_z \pm \frac{\pi}{2}$. Si osserva cosa accade
alla tangente dell'angolo in funzione del tempo:
\begin{gather*}
    \tan\alpha(t) = \frac{E_{0z} \cos(kx - \omega t + \phi_z)}{E_{0y}\cos(kx - \omega t + \phi_y)} = \frac{\cos(kx - \omega t + \phi_z)}{\cos(kx - \omega t + \phi_z \pm \frac{\pi}{2})} = \frac{ \cos(kx - \omega t + \phi_z)}{\mp\sin(kx - \omega t + \phi_z)}
\end{gather*}
Per le proprietà trigonometriche del coseno, quando si introduce uno sfasamento di $\frac{\pi}{2}$, il coseno
si trasforma nel seno dell'angolo $\phi_z$, quindi vale la seconda uguaglianza.
Se si indicasse l'argomento dentro al seno e al coseno come $\theta$, si otterrebbe la seguente
\begin{gather*}
    \tan \theta = \frac{\sin \theta}{\cos\theta} = \frac{\cos(\frac{\pi}{2} - \theta)}{\sin(\frac{\pi}{2} - \theta)} = \frac{\cos(\theta - \frac{\pi}{2})}{-\sin(\theta - \frac{\pi}{2})} = - \frac{1}{\tan(\theta - \frac{\pi}{2})}
\end{gather*}
Allora con questi passaggi trigonometrici si ottiene
\begin{gather*}
    \tan\alpha (t) = \pm \tan \left(kx - \omega t + \phi_z + \frac{\pi}{2}\right)
\end{gather*}
Dunque si ottiene che l'angolo $\alpha$ varia nel tempo ed il simbolo $\pm$ determina se la variazione
di fase è positiva o negativa:
\begin{align}
    \alpha (t) = \pm\left(kx - \omega t + \phi_z + \frac{\pi}{2}\right)
\end{align}
Le due polarizzazioni prenderanno il nome di polarizzazioni \textbf{circolari}:
\begin{align*}
    +:& \ \alpha (t) = k t - \omega t \\
    -:& \  \alpha (t) = k t + \omega t 
\end{align*} 
Nel primo caso gira in senso orario e dunque prende il nome di \textbf{polarizzazione circolare sinistra} 
(anche indicata con $\sigma^{+}$), mentre l'altra, girando in senso antiorario, prenderà il nome di 
\textbf{polarizzazione circolare destra} (anche indicata con $\sigma^{-}$). 

\subsection{Polarizzazione ellittica}
La polarizzazione ellittica è il caso "generale" della polarizzazione,
ossia quello nel quale 
\begin{gather*}
    E_{0y} \neq E_{0z} \qquad \phi_y \neq \phi_z
\end{gather*}
Si chiama ellittica proprio perché il campo elettrico,
oscillando, disegna un ellisse sul piano perpendicolare 
alla direzione di propagazione dell'onda. 

\section{Riflessione su di uno specchio metallico}
Se si avesse un campo elettrico con polarizzazione circolare sinistra, dunque tale
che i moduli $E_{0z} = E_{0y}$ e tale che i vettori siano in fase con $\phi_y = \phi_z + \frac{\pi}{2}$. Se si
scegliesse $\phi_z = 0$, e dunque $\phi_y = \frac{\pi}{2}$, allora si potrebbe esprimere il campo elettrico come
il seguente vettore:
\begin{gather*}
    \vv{E} = E_0 \cos\left(kx - \omega t + \frac{\pi}{2}\right)\hat{u}_y + E_0 \cos(kx - \omega t)\hat{u}_z   
\end{gather*}
Se si mandasse questa onda elettromagnetica con questo determinato campo elettrico
su di uno specchio metallico, si osserverebbe che l'interferenza distruttiva causata
dal metallo non permetterebbe di avere campo elettrico dentro lo specchio.
Gli elettroni che oscillano nel mezzo metallico, oltre che
a generare il campo elettrico in controfase, generano un onda anche 
verso la direzione di provenienza del campo elettrico e dunque i due 
campi elettrici si eliminano all'interno del metallo ma non all'esterno di esso.
Si scrive dunque il campo elettrico che si propaga verso sinistra 
cambiando la fase del campo di $\pi$:
\begin{gather*}
    \vv{E} = -E_{0}\cos\left(-kx -\omega t + \frac{\pi}{2} \right)\hat{u}_y + E_0 \cos(-kx -\omega t)\hat{u}_z   
\end{gather*}
L'effetto dello specchio metallico è dunque quello di invertire il valore
del campo elettrico in virtù del fatto che il campo generato è in controfase.
Quindi i due campi sono opposti in $x = 0$ ed il nuovo campo si propaga verso
destra e non verso sinistra: la polarizzazione quando la luce incide su di
uno specchio viene invertita. Nel caso di polarizzazione lineare, si invertono i segni.
\section{L'energia dell'onda elettromagnetica}
\begin{wrapfigure}{r}{0.3\textwidth}
    \centering
    \begin{tikzpicture}
        \draw[->](0, 0) -- (1, 0) node[at end, below] {$\hat{k}$ };
        \draw(3, -1) -- (3, 1) node[at end, above] {$A$};
        \draw(3, 1) -- (2.25, 0) -- (2.25, -2) -- (3, -1);
    \end{tikzpicture}    
\end{wrapfigure}
Perché ci si scalda al sole? Quando una onda elettromagnetica
incide sui nostri elettroni, essa li mette in accelerazione facendo si
che essi guadagnino energia cinetica, rilasciandola poi sottoforma di calore.
Come si può determinare l'energia emessa da una data onda elettromagnetica? 
si può determinare \textbf{l'intensità luminosa}, ossia l'energia che
passa per una data superficie $A$:
\begin{align}
    \frac{\Delta E}{\Delta t} \cdot \frac{1}{A} = I = \epsilon_0 c \vv{E}(t)^{2}  
\end{align}
Si può dunque determinare l'energia che attraversa una certa superficie in un certo intervallo di tempo come
\begin{align}
    \Delta E = I A \Delta t.
\end{align}
Dato che il campo elettrico oscilla sempre, ci sono sia degli istanti in cui
il campo elettrico è nullo ma anche degli istanti in cui il campo elettrico è massimo: l'intensità luminosa dunque oscilla anch'essa
con un certo periodo e con una certa fase rispetto al campo elettrico. Dato che
sussiste la seguente uguaglianza,
\begin{gather*}
    \omega = \frac{2\pi}{T} = \frac{2\pi c}{\lambda}
\end{gather*}
dove $\lambda$ non è altro che la lunghezza d'onda della radiazione
considerata, allora l'intensità luminosa è la media del modulo del campo elettrico
durante un suo periodo completo di oscillazione:
\begin{gather*}
    \left<I\right> = c\epsilon_0 \frac{1}{T}\int_{0}^{T}\vv{E}^{2} \ dt  \ \Longrightarrow \ \left<I\right> = c\epsilon_0 \frac{1}{T}\int_{0}^{T}E_0^{2}\cos^{2}(kx - \omega t) \ dt  
\end{gather*}
Allora si può risolvere la media rispetto al coseno alla seconda e ottenere
l'intensità luminosa media di una certa onda elettromagnetica come
\begin{align}
    \left<I\right> = \frac{1}{2}c\epsilon_0 E^{2} 
\end{align}
Di conseguenza la media dell'intensità luminosa non è altro che la metà
dell'intensità luminosa totale in valore assoluto. Se la polarizzazione
del campo elettrico in modulo rimane sempre $E_0$ allora l'intensità
varia poiché si ha il doppio dell'intensità: questa onda polarizzata è la somma del
contributo dell'onda piana iniziale e del contributo sfasato rispetto
a questa onda. 

\section{Il fenomeno della trasmissione di un mezzo dielettrico}
\begin{wrapfigure}{r}{0.4\textwidth}
    \centering
    \caption{L'interfaccia tra dielettrici}
    \begin{tikzpicture}
        \draw(0, -1) -- (0, 1) node[at end, left] {$n_1$} node[at end, right] {$n_2 > n_1$};
        \draw[dashed](-3, 0) -- (3, 0);
        \draw(-3, 1) -- (0, 0);
        \draw(-1.5, 0.5) arc (150:180:1) node[midway, left] {$\theta$};
        \draw[->, thick, red](-2.25, 0.75) -- (-2, 1.35) node[at end, right] {$\vv{E_{\parallel}}$};
        \filldraw[red](-2.25, 0.75) circle (1pt) node[anchor = north east] {$\vv{E_{\perp}}$ };
        \draw(-2.25, 0.75) circle (0.1);
        \draw(0, 0) -- (3, -0.5);
        \draw[red, thick, ->](1.5, -0.25) -- (1.7, 0.5) node[at end, right] {$\vv{E_{\parallel}^{T} }$ };
        \draw(0, 0) -- (-3, -1);
        \draw[->, thick, red](-2.25, -0.75) -- (-2, -1.35) node[at end, right] {$\vv{E_{\parallel}}$};
        \filldraw[red](-2.25, -0.75) circle (1pt) node[anchor = south east] {$\vv{E_{\perp}}$ };
        \draw(-2.25, -0.75) circle (0.1);
    \end{tikzpicture}    
\end{wrapfigure}
I fenomeni di trasmissione e di riflessione alle interfacce tra dielettrici dipendono
dalla polarizzazione delle onde rispetto al piano
di incidenza. Il \textbf{piano di incidenza} è il piano che contiene il \textbf{raggio
incidente} definito dal vettore $\hat{k}$ e la direzione normale alla superficie
nel punto di incidenza: nel disegno il piano di incidenza coincide
con il foglio. Il vettore campo elettrico è dunque scomponibile in due componenti:
una parallela ed una perpendicolare al piano di incidenza il cui verso, ossia se entrante
o uscente, è arbitrario. Un fenomeno particolare che si incontra è
quando una onda incide un dielettrico senza essere riflessa (ossia la luce
se ha una polarizzazione con solo $\vv{E_{\parallel}}$) viene  solo trasmessa. 

\begin{wrapfigure}{r}{0.4\textwidth}
    \centering
    \caption{L'angolo di Brewester}
    \begin{tikzpicture}
        \draw[dashed](-3 , 0) -- (3, 0);
        \draw(0, -2) -- (0, 2);
        \draw(-1, 2) -- (0, 0) -- (2, -1);
        \draw(0, 0) -- (-1, -2);
        \draw(-0.5, 1) arc (110:180:1) node[midway, left] {$\theta_1$};
        \draw(1, -0.5) arc (-30:0:1) node[midway, right] {$\theta_2$};
        \draw[->, green](0, 0) -- (0.25, 0.5) node[at end, above] {$\vv{d_E}$ };
        \draw[->, green](0.25, -0.12) -- (0.5, 0.35);
        \draw[->, green](0.5, -0.25) -- (0.75, 0.25);
    \end{tikzpicture}    
\end{wrapfigure}
Se avessi un dipolo elettrico oscillante con il campo orientato in una
certa direzione, allora esso emetterà solamente campo elettrico con
direzione perpendicolare alla direzione del dipolo elettrico oscillante;
l'emissione lungo la direzione di oscillazione è nulla. Il risultato delle onde
in trasmissione o riflessione sono date dalle cariche all'interno del
dielettrico: si immagina di avere la condizione per cui si ha una interfaccia 
con un raggio di luce che incide ad un certo angolo sul dielettrico in modo tale che
si crei un angolo di $90^{\circ}$ tra i raggi riflessi e trasmessi: in questo
caso si creano dipoli elettrici nel dielettrico che oscillano perpendicolarmente alla
direzione dell'onda trasmessa. Questi dipoli possono creare solo delle onde nella
direzione dell'onda riflessa e dunque l'onda riflessa non può esistere.
Questo angolo particolare prende il nome di \textbf{angolo di Brewester} e questo vale solo
quando $\vv{E}$ giace sul piano di incidenza e dunque si può verificare
solo per una polarizzazione lineare e vale dunque per $\vv{E_{\parallel}}$.   
La relazione che si ottiene tra gli angoli sotto mi permette di ricavare l'angolo di Brewester:
\begin{gather*}
    \left\{\begin{array}{l}
        \theta_1 + \theta_2 + \frac{\pi}{2} = \pi \\
        n_1 \sin \theta_1 = n_2 \sin\theta_2
    \end{array}\right. \ \Longrightarrow \ \left\{\begin{array}{l}
        \theta_2 = \frac{\pi}{2} - \theta_1 \\
        n_1 \sin \theta_1 = n_2 \sin (\frac{\pi}{2} - \theta_1) = n_2 \cos\theta_1
    \end{array}\right. 
\end{gather*}
Si ottiene allora l'angolo critico per il quale si ha questa condizione: 
\begin{align}
    \tan \theta_1 = \frac{n_2}{n_1}
\end{align}
L'angolo tra aria e vetro di Brewester è esattamente $\theta = 56^{\circ}$. Se della luce
con polarizzazione ellittica incide con l'angolo di Brewester su di una
superficie dielettrica, la riflessione emerge con polarizzazione lineare e perpendicolare.
L'interfaccia si comporta come se fosse un polarizzatore: solo la luce con una data polarizzazione
può percorrere il cammino di riflessione. \section{Polarizzatori e lamine di ritardo}
\subsection{Funzionamento di un cubo polarizzatore}
\begin{wrapfigure}{r}{0.4\textwidth}
    \centering
    \caption{Schematizzazione del campo elettrico}
    \begin{tikzpicture}
        \draw[->](0, 0) -- (2, 0) node[at end, below] {$y$};
        \draw[->](0, 0) -- (0, 2) node[at end, left] {$z$};
        \draw[->, thick](0, 0) -- (1, 1) node[at end, right] {$\vv{E}$ };
        \draw[->, thick](0, 0) -- (1, 0) node[at end, below] {$\vv{E_y}$};
        \draw[->, thick](0, 0) -- (0, 1) node[at end, left] {$\vv{E_z}$ };
    \end{tikzpicture}    
\end{wrapfigure}
Alcuni materiali dielettrici sono detti \textbf{birifrangenti}: ossia
hanno due indici di rifrazione per due polarizzazione lineari ortogonali
della luce che li attraversa. Una applicazione è il \textbf{cubo polarizzatore},
il cui scopo è proprio quello di separare due polarizzazioni ortogonali di un
fascio incidente sul cubo con polarizzazione  generica.
Si può ora analizzare il funzionamento del cubo polarizzatore attraverso il
secondo modello: tra le due facce del cubo è inserito un film sottile birifrangente
per cui l'indice di rifrazione per gli assi ortogonali è diverso. Si avrà allora 
che gli indici di rifrazione $n_{\parallel} \neq n_{\perp}$. Si vuole ora che
l'angolo di incidenza a $45$ gradi sia l'angolo di riflessione totale 
per $\vv{E_{\perp}}$.


\begin{wrapfigure}{r}{0.4\textwidth}
    \centering
    \caption{Il cubo polarizzatore}
    \begin{tikzpicture}
        \draw(0, 0) -- (2, 0);
        \draw[->, thick](0.5, 0) -- (1.5, 0) node[at end, above] {$\vv{E_{\perp}}$};
        \draw[->, thick](0.5, 0) -- (0.5, 1) node[at end, left] {$\vv{E_{\parallel}}$ };
        \filldraw(0.5, 0) circle (1pt);
        \draw(0.5, 0) circle (0.15);
        \draw(2, 1) -- (4, -1) -- (2, -1) -- (2, 1); 
        \draw(2.2, 1) -- (4.2, -1) -- (4.2, 1) -- (2.2, 1);
        \filldraw[pattern = north east lines, pattern color = white](2.2, 1) -- (4.2, -1) -- (4, -1) -- (2, 1) -- (2.2, 1);
        \draw[dashed](2, 0) -- (3, 0) -- (3, -1) node[midway, left] {$\vv{E_\perp}$};
        \draw[dashed](3, 0) -- (4, 0) node[midway, above] {$\vv{E_\parallel}$};
    \end{tikzpicture}    
\end{wrapfigure}
Si possono dunque elencare le ipotesi:
\begin{itemize}
    \item Si sceglie $n_{\perp}$ in modo tale che $\frac{\pi}{4}$ sia l'angolo di 
    riflessione totale interna per luce con polarizzazione  $\vv{E_{\perp}}$.
    \item Si sceglie $n_{\parallel}$ in modo tale che l'angolo $\frac{\pi}{4}$ sia l'angolo di Brewester
    per luce con polarizzazione $\vv{E_{\parallel}}$.
\end{itemize}
L'angolo di rifrazione è dunque molto grande e per questo coincide
con l'interfaccia e dunque, secondo la legge di Snell, si ha riflessione totale 
interna:
\begin{gather*}
    n_{\text{vetro}} \sin\frac{\pi}{4} = n_{\perp} \sin\frac{\pi}{2}
\end{gather*}
L'indice di rifrazione $n_{\perp}$ deve allora soddisfare questa condizione
per poter essere un angolo di riflessione totale interna
per la luce con polarizzazione $\vv{E}_{\perp}$:
\begin{gather*}
    n_{\perp} = n_{\text{vetro}}\frac{\sqrt{2} }{2}
\end{gather*} 
Mentre $n_{\parallel}$ del film birifrangente è in relazione con l'indice
di rifrazione del vetro secondo la seguente:
\begin{gather*}
    \tan\theta = \frac{n_{\parallel}}{n_{\text{vetro}}} \ \Longrightarrow \ n_{\parallel} = n_{\text{vetro}}
\end{gather*}
Dato che l'angolo, per costruzione, è $\frac{\pi}{4}$ allora la relazione a destra vale.

\subsection{Polarizzatore a fili metallici}
\begin{wrapfigure}{r}{0.3\textwidth}
    \centering
    \begin{tikzpicture}
        \draw(0, 0) rectangle (2, 2);
        \draw(0.1, 0) -- (0.1, 2);
        \draw(0.2, 0) -- (0.2, 2);
        \draw(0.3, 0) -- (0.3, 2);
        \draw(0.4, 0) -- (0.4, 2);
        \draw(0.5, 0) -- (0.5, 2);
        \draw(0.6, 0) -- (0.6, 2);
        \draw(0.7, 0) -- (0.7, 2);
        \draw(0.8, 0) -- (0.8, 2);
        \draw(0.9, 0) -- (0.9, 2);
        \draw(1.0, 0) -- (1.0, 2);
        \draw(1.1, 0) -- (1.1, 2);
        \draw(1.2, 0) -- (1.2, 2);
        \draw(1.3, 0) -- (1.3, 2);
        \draw(1.4, 0) -- (1.4, 2);
        \draw(1.5, 0) -- (1.5, 2);
        \draw(1.6, 0) -- (1.6, 2);
        \draw(1.7, 0) -- (1.7, 2);
        \draw(1.8, 0) -- (1.8, 2);
        \draw(1.9, 0) -- (1.9, 2);
        \draw(2.0, 0) -- (2, 2);
        \draw[->, thick](-0.5, 0.5) -- (-0.5, 1.5) node[at end, left] {$\vv{E_{\parallel}}$ };
        \draw[->, thick](0.5, -0.5) -- (1.5, -0.5) node[at end, below] {$\vv{E_{\perp}}$ };
    \end{tikzpicture}    
\end{wrapfigure}
Un altro strumento per poter ottenere una sola polarizzazione da un 
fascio di luce lineare è detto polarizzatore a \textbf{fili metallici}.
Il polarizzatore a fili metallici è una lastra di materiale sulla quale ci si
può depositare dei fili metallici sottili  lungo una direzione ben definita
(in genere ognuno con un  diametro di $10\sim100 $ nm). Supponendo di avere
una luce incidente al filo metallico con componente sia parallela che
perpendicolare al fascio di fili e di farla passare attraverso il polarizzatore.
La luce che passa dentro questo polarizzatore è quella con $\vv{E_{\perp}}$ poiché il campo parallelo
mette in moto gli elettroni nei fili sottili e, dato che sono più alti che larghi,
allora c'è più movimento degli elettroni che creano un campo
elettrico con interferenza distruttiva per la componente parallela. Le onde
perpendicolari, invece, passano in quanto il diametro dei microfili è
molto piccolo e dunque non c'è abbastanza spazio per poter mettere
in movimento gli elettroni per poter creare interferenza distruttiva
per quella determinata polarizzazione.

\section{Lamine di ritardo}
\begin{wrapfigure}{r}{0.4\textwidth}
    \centering
    \caption{Ritardo di fase}
    \begin{tikzpicture}
        \draw[->](0, 0) --(2, 0) node[at end, below] {$x$};
        \draw[->](0, 0) -- ( 0, 2) node[at end, left] {$y$};
        \filldraw(0, 0) circle (1pt) node[anchor = north] {$O$}; 
        \filldraw(1.5, 0) circle (1pt) node[anchor = north] {$O'$};
        \draw[->](1.5, 0)-- (3.5, 0) node[at end, below] {$x'$};
        \draw[->](1.5, 0) -- (1.5, 2) node[at end, left] {$y$};
        \draw[|-|](0.1, 0.2) -- (1.4, 0.2) node[midway, above] {$d$};
        \draw[|-|](1.6, 0.2) -- (2.5, 0.2) node[midway, above] {$x'$};
        \draw[|-|](0, -0.7) -- (2.5, -0.7) node[midway, below] {$x$};
    \end{tikzpicture}    
\end{wrapfigure}
Le \textbf{lamine di ritardo} si basano su materiali birifrangenti e
servono a modificare la polarizzazione della luce.
Supponendo di scrivere un campo elettrico lungo una sola
direzione con 
\begin{gather*}
    \vv{E_{\parallel}} = E_{0z}\cos(kx - \omega t)\hat{z} 
\end{gather*}  
Ci si potrebbe chiedere come sia l'onda, rispetto all'origine $O$,
sia il campo elettrico, cambino rispetto al tempo. Si considera
un altro sistema di riferimento $O'$ descritto con
assi paralleli a quelli del sistema di riferimento $O$. Concentrandosi
ora sul sistema di riferimento $O'$, dopo un certo istante di
tempo, l'onda avrà camminato per una distanza $x$ rispetto a $O$
e per una distanza $x'$ rispetto a $O'$. Si esprime dunque il campo elettrico
rispetto a $O'$ come
\begin{gather*}
    \vv{E}(x', t) = E_{0z}\cos(k(d + x') - \omega t)\hat{z}  
\end{gather*}
Per l'onda, dopo essersi propagata per una distanza $d$, si può 
considerare un sistema di riferimento $O'$ attraverso un termine di fase
$kd$ che prende il nome di \textbf{ritardo di fase}, il quale è dovuto
alla propagazione lungo il tratto di lunghezza $d$ che,
nel caso si tratti di un mezzo dielettrico con indice di
rifrazione $n$, diventa $knd$. 
Si considera ora una lamina di spessore $d$ con materiale
birifrangente ed un campo elettrico incidente generico 
di una onda piana con polarizzazione lineare
\begin{gather*}
    \vv{E_{\text{in}}} = E_{0y}\cos(kx - \omega t)\hat{y} + E_{0z}\cos(kx - \omega t)\hat{z}   
\end{gather*}
Adesso ci si chiede che cosa accade al campo
elettrico dopo che ha attraversato la lamina in funzione di un sistema
di riferimento $O'$ posto dopo la lamina. Se il materiale è birifrangente, allora
dovrà necessariamente cambiare $n$ a seconda della direzione di oscillazione. Il
campo elettrico di uscita dalla lamina allora sarà:
\begin{gather*}
    \vv{E_{\text{out}}}(x', t) = E_{0y}\cos(kx' -\omega t + kn_yd)\hat{y} + E_{0z}\cos(kx' - \omega t + kn_z d)\hat{z}   
\end{gather*}
La polarizzazione in uscita sarà quindi ellittica.  La velocità nel mezzo
dipende da $\frac{\lambda}{T}$, ma la lunghezza d'onda dipende da $n$: nel 
mezzo birifrangente esisterà un asse lungo il quale la velocità dell'onda va più lenta (e quindi
si ha un indice di rifrazione maggiore) ed un asse veloce con indice di rifrazione minore. \\
Cambiando l'origine dei tempi, si potrebbe aggiungere o togliere a piacere
un termine di fase comune su due coseni. Se si ponesse come termine di fase $-kn_yd$, 
allora il campo elettrico uscente sarà
\begin{gather*}
    \vv{E_{\text{out}}}(x', t) = E_{0y}\cos(kx'- \omega t)\hat{y} + E_{0z}\cos(kx' - \omega t + kd(n_z - n_y))\hat{z}  
\end{gather*}
E quindi la fase dipende dalla differenza tra gli indici di rifrazione.


\subsection{La lamina $\frac{\lambda}{2}$}
Supponendo di mandare una onda elettromagnetica con polarizzazione lineare,
la prima lamina di interesse è quella la cui differenza di fase tra l'asse fast e quello slow è $\pi$,
\begin{gather*}
   \delta \phi = \pi \qquad  kd(n_z - n_y) = \pi \Longrightarrow \ d\Delta n = \frac{\lambda}{2}
\end{gather*}
La differenza tra i due indici di rifrazione che moltiplicano la distanza prende
il nome di \textbf{cammino ottico}. La differenza tra i cammini ottici in questa lamina è pari alla metà
della lunghezza d'onda in uscita: l'effetto di questa lamina è quello di introdurre
un ritardo di fase uguale a $\pi$ lungo l'asse lento. Inoltre, se la polarizzazione
in ingresso è lineare, la luce trasmessa ha polarizzazione ancora lineare ma lungo l'asse
riflesso rispetto all'asse veloce della lamina. Dato che la fase è $\pi$, allora la componente lungo $z$
cambia di segno a causa del ritardo di fase:
\begin{align}
    \vv{E_{\text{out}}}(x', t) = E_{0y}\cos\left(kx'- \omega t \right)\hat{y} - E_{0z}\cos(kx' - \omega t)\hat{z}  
\end{align}
\begin{wrapfigure}{r}{0.4\textwidth}
    \centering
    \caption{Visualizzazione della polarizzazione in uscita}
    \begin{tikzpicture}
        \draw[->](-1.75 0) -- (1.75, 0) node[at end, below] {$z$};
        \draw[->](0, -1.5) -- (0, 1.5) node[at end, left] {$y$};
        \draw(0, 0)circle (1);
        \filldraw(0, 0) circle (1pt);
        \draw[thick, ->](0, 0) -- (1.2, 0.8) node[at end, above] {$\vv{E_{in}}$ };
        \draw[thick, ->](0, 0) -- (-1.2, 0.8) node[at end, above] {$\vv{E_{out}}$};
        \end{tikzpicture}    
\end{wrapfigure}
Analizzando ora il caso di una polarizzazione circolare sinistra $\sigma^{+}$:
\begin{gather*}
    \vv{E_{\text{in}}} = E_{0}\cos\left(kx - \omega t + \frac{\pi}{2}\right)\hat{y} + E_{0}\cos(kx - \omega t)\hat{z}   
\end{gather*} 
Aggiungendo ora il termine di fase anche per l'asse lento,
\begin{gather*}
    \vv{E_{\text{out}}} = E_0 \cos\left(kx - \omega t + \frac{\pi}{2}\right)\hat{y} + E_0 \cos(kx - \omega t + \pi)\hat{z}   
\end{gather*}
Sottraggo ora un termine $-\pi$ rispetto ad entrambi i termini e quindi
\begin{gather*}
        \vv{E_{\text{out}}}(x', t) = E_{0y}\cos\left(kx'- \omega t - \frac{\pi}{2}\right)\hat{y} - E_{0z}\cos(kx' - \omega t)\hat{z}  
\end{gather*}

\subsection{La lamina $\frac{\lambda}{4}$}
La lamina di ritardo $\frac{\lambda}{4}$ è molto simile a quella
di $\frac{\lambda}{2}$ anche se causa un ritardo che è la metà, ossia
produce un ritardo di fase $kd\Delta n  = \frac{\pi}{2}$ per cui la differenza dei cammini ottici sarà $\frac{\lambda}{4}$.
Se la polarizzazione in ingresso fosse lineare con angolo di $\frac{\lambda}{4}$ rispetto all'asse slow, essa uscirebbe
con polarizzazione circolare destra.

\subsection{Polarizzazione nei due casi}
In generale la polarizzazione in ingresso, ed in uscita, sarà ellittica, anche se è possibile verificare che, se si avesse una polarizzazione ellittica e
scegliendo un angolo opportuno in una lamina $\frac{\lambda}{4}$, si potrebbe ottenere
una polarizzazione lineare e agire sulla spanciatura dell'ellisse. Poi con 
una $\frac{\lambda}{2}$ si può ottenere una polarizzazione lineare rispetto
all'asse preferito. 

\subsection{Lamine zero e multiple order: cosa accade se si utilizza una lamina con diverse lunghezze d'onda}
Si studia ora che accade se si prova ad utilizzare una determinata lamina, che è stata progettata per
funzionare solamente con determinate lunghezze d'onda, se si utilizzano onde 
diverse da quelle specificate. Una lamina birifrangente è progettata in modo tale che il ritardo di fase tra le due
polarizzazioni principali sia
\begin{align}
\phi = k d (n_z - n_y) = \frac{2\pi}{\lambda} d \Delta n = \pi + 2\pi i \qquad i \in \mathbb{N}.
\end{align}
Nel caso di una lamina $\frac{\lambda}{2}$, $i$ indica l'ordine della lamina. Se si volesse ora
utilizzare questa lamina con una lunghezza d'onda differente $\lambda '= \lambda + \Delta \lambda$,
supponendo che non ci sia molta differenza tra i due ordini di rifrazione, si può determinare
il ritardo di fase $\phi'$:
\begin{gather*}
   \phi' = \frac{2\pi}{\lambda'} d \Delta n =  \frac{2\pi}{\lambda + \Delta \lambda} d\Delta n
\end{gather*}
Se la variazione d'onda è molto piccola, il termine
al denominatore è sviluppabile con Taylor:
\begin{gather*}
    \frac{1}{\lambda + \Delta \lambda} \approx \frac{1}{\lambda}\left(1 - \frac{\Delta \lambda}{\lambda}\right)
\end{gather*}
Ottenendo dunque
\begin{gather*}
    \phi' \approx \frac{2\pi}{\lambda} d\Delta n \left(1 - \frac{\Delta \lambda}{\lambda}\right) \ \Longrightarrow \ \phi' \approx \phi \left(1 - \frac{\Delta \lambda}{\lambda}\right)
\end{gather*}
Data la prima equazione, si trasforma in
\begin{gather*}
    \phi' \approx (\pi + 2i \pi)\left(1 - \frac{\Delta \lambda}{\lambda}\right) 
\end{gather*}
Dunque l'errore sul ritardo sarà dato dalla seguente:
\begin{align}
    \Delta \phi \approx \phi \frac{\Delta \lambda}{\lambda} \ \Longrightarrow \ \Delta \phi \propto (\pi + 2\pi i)
\end{align}
Si possono allora considerare i seguenti casi
\begin{itemize}
    \item $i = 0$: prendono il nome di \textbf{lamine zero-order} ed hanno uno spessore ridotto e risultano
    poco sensibili alle variazioni di lunghezza d'onda e di angolo di incidenza;
    \item $i \gg 1$: prendono il nome di  \textbf{lamine multiple-order} e sono più spesse ed economiche,
    ma presentano un'elevata sensibilità alle variazioni di lunghezza d'onda e
    di incidenza.
\end{itemize}

\clearpage
\section{Aberrazioni delle lamine in condizioni non ideali}
\begin{wrapfigure}{r}{0.4\textwidth}
    \centering
    \caption{Scomposizione del campo elettrico}
    \begin{tikzpicture}
        \draw[->](0, 0) -- (3, 0) node[at end, below] {$\vv{E_\parallel}$};
        \draw[->](0, 0) -- (2, 2) node[at end, above] {$a$};
        \draw[->](0, 0) -- (0, 3) node[at end, left] {$\vv{E_\perp} $};
        \draw[->](0, 0) -- (-1, 2) node[at end, above] {$b$};
        \draw(0.5, 0) arc (0:45:0.5) node[midway, right] {$\theta$};
    \end{tikzpicture}    
\end{wrapfigure}
Quando si opera con delle lamine $\frac{\lambda}{4}$ e $\frac{\lambda}{2}$
che non hanno ritardi di fase ideali, ossia quando la polarizzazione non è
lineare oppure si è in condizioni di umidità e temperatura che differiscono
da quelle specificate dal costruttore, si introducono inevitabilmente delle aberrazioni. 
Si è detto, fino ad ora, che esistono degli assi relativi ai cubi paralleli e
degli assi ortogonali relativi alla lamina. Si potrebbe avere che la lamina di
ritardo abbia gli assi lenti e veloci inclinati di un certo angolo 
$\theta$ rispetto agli assi paralleli al polarizzatore. Per poter analizzare 
questa situazione, si ricorda l'espressione dei complessi in forma esponenziale
\begin{gather*}
    e^{i\theta} = \cos\theta + i\sin\theta 
\end{gather*}
Ricordando ora l'espressione del campo elettrico generico
\begin{gather*}
    \vv{E} = E_{0z}\cos(kx - \omega t + \phi_z)\hat{k}  + E_{0y} \cos(kx - \omega t + \phi_y)\hat{y}  
\end{gather*}
Utilizzando i complessi si esprimere il campo elettrico come una componente lungo l'asse
reale e l'altra lungo l'asse degli immaginari. Ricordandosi di prendere
solamente la parte reale di questa espressione si ha 
\begin{gather*}
    \vv{E} = Re\left(E_{0z}e^{i(kx - \omega t + \phi_z)}\hat{z} + E_{0y}e^{i(kx - \omega t + \phi_y)} \hat{y} \right) 
\end{gather*}
Utilizzando una matrice $2\times 1$ si esprime il campo elettrico in forma matriciale:
\begin{gather*}
    \begin{pmatrix}
    E_{0z}e^{i(kx - \omega t + \phi_z)} \\
    E_{0y}e^{i(kx - \omega t + \phi_y)}
    \end{pmatrix}
\end{gather*}
Allora lungo gli assi $a, b$, che sono orientati di un certo $\theta$
rispetto agli assi paralleli ed ortogonali, si esprime il campo
elettrico come:
\begin{gather*}
    \vv{E}_{a, b} = E_{0z} \cos(kx - \omega t + \phi_z) \left(\cos\theta \hat{a} - \sin\theta \hat{b}  \right) + E_{0y} \cos(kx - \omega t + \phi_z) \left(\sin\theta \hat{a} + \cos\theta \hat{b}  \right) 
\end{gather*}
Questa espressione si può allora riscrivere come le coordinate del campo rispetto
ai versori $\hat{a}$ e $\hat{b}$. Si può dunque pensare a questa espressione come
il prodotto tra una matrice due per due e il vettore del campo elettrico nelle sue componenti:
\begin{align}
    Re\begin{pmatrix}
        \cos\theta & \sin\theta \\
        -\sin\theta & \cos\theta 
    \end{pmatrix}\begin{pmatrix}
        E_{0z} e^{i(kx - \omega t + \phi_z)} \\
        E_{0y} e^{i(kx - \omega t + \phi_y)}  
    \end{pmatrix}
\end{align}   
La matrice a sinistra prende il nome di matrice delle rotazioni in funzione
dell'angolo $\theta$, questa matrice permette di ottenere l'espressione di un vettore
rispetto a degli assi ruotati di un certo angolo $\theta$, e si indica con $R(\theta)$. 
La matrice che serve per tornare agli assi di partenza è la matrice inversa (ossia quella
coi segni invertiti per il seno).  Adesso, data l'espressione
del campo elettrico generico rispetto alla base $\hat{a}, \hat{b} $
\begin{gather*}
    \vv{E}_{a, b} =  E_{0a}\cos(kx - \omega t + \phi_a) \hat{a} + E_{0b} \cos(kx - \omega t + \phi_b)\hat{b} 
\end{gather*} 
Attraverso i numeri complessi, si esprime il vettore parte reale del campo come
\begin{gather*}
    Re \begin{pmatrix}
        E_{0a}e^{i(kx - \omega t + \phi_a)} \\
        E_{0b}e^{i(kx - \omega t + \phi_b)}  
    \end{pmatrix}
\end{gather*}
Se si introducesse sull'asse lento un ritardo di fase, il campo elettrico uscente è dato dal campo rispetto al versore $\hat{b}$ e 
dal campo rispetto al versore $\hat{a}$ con un certo ritardo di fase $\delta \phi$:  
\begin{gather*}
    \vv{E_{a, b}^{OUT} } = E_{0a}\cos(kx - \omega t + \phi_a + \delta \phi)\hat{a}  + E_{0b}\cos(kx - \omega t + \phi_b)\hat{b}   
\end{gather*}
Si ottiene dunque l'espressione del campo in uscita
attraverso il prodotto tra matrici
\begin{gather*}
    Re\begin{pmatrix}
        e^{\delta\phi} & 0  \\
        0 & 1 
    \end{pmatrix} \begin{pmatrix}
        E_{0a}e^{i(kx - \omega t + \phi_a)} \\
        E_{0b}e^{i(kx - \omega t + \phi_b)}  
    \end{pmatrix}
\end{gather*}
Analogamente a quanto detto prima, se si volesse ottenere il campo elettrico in entrata a partire dal campo
elettrico in uscita, si può applicare la matrice inversa. Si riassume che l'espressione
del campo elettrico in uscita da una lamina, che è entrato con un certo angolo, 
ha espressione
\begin{align}
    \vv{E_{a, b}^{OUT} } = Re \left(R(-\theta)\begin{pmatrix}
        e^{\delta \phi} & 0 \\
        0 & 1 
    \end{pmatrix}R(\theta) \begin{pmatrix}
        E_{0z}e^{i(kx - \omega t + \phi_z)}\\
        E_{0y}e^{i(kx - \omega t + \phi_y)}  
    \end{pmatrix}\right) 
\end{align}

\part{Interferenza}
\chapter{Aberrazioni e interferenze}
\section{Il concetto di interferenza}
L'Interferenza si verifica quando sono presenti due o più campi elettromagnetici
con fasi differenti. Di fatto, quando si sommano i campi elettromagnetici
si utilizzano le formule di prostaferesi per il coseno:
\begin{align}
    \cos\alpha + \cos\beta &= 2\cos\left(\frac{\alpha + \beta}{2}\right) \cdot  \cos\left(\frac{\alpha - \beta}{2}\right) \\
    \cos\alpha\cos\beta &= \frac{1}{2}\left(\cos(\alpha + \beta) + \cos(\alpha - \beta)\right)
\end{align}
Dato che ci sono molte variabili per cui i campi potrebbero differire, possiamo
partire dall'analisi del caso semplice in cui solamente la fase differisce per i 
due campi elettrici.

\subsection{Interferenza semplice}
Il caso semplice di interferenza si ha quando i due campi elettromagnetici
hanno la stessa direzione di propagazione e quindi stessa frequenza (o pulsazione); inoltre,
per rimanere nel caso semplice, hanno anche stessa polarizzazione e stessa ampiezza.
L'unica differenza è la fase. 
\begin{gather*}
    \vv{E_1} = E_0\cos(kx - \omega t + \phi_1)\hat{y}  \\
    \vv{E_2} = E_0\cos(kx - \omega t + \phi_2)\hat{y}   
\end{gather*}
Secondo le formule di prostaferesi si ottiene il campo elettrico totale:
\begin{gather*}
    \vv{E_{TOT}} = 2E_0\cos\left(kx - \omega t + \frac{\phi_1 + \phi_2}{2}\right) \cdot  \cos\left(\frac{\phi_1 - \phi_2}{2}\right) 
\end{gather*}
L'intensità totale mediata nel tempo si ottiene come 
\begin{align}
    \left< I_{TOT}  \right> = c\epsilon_0 \left< \vv{E}_{TOT}  \right>  = c\epsilon_0\frac{1}{T}\int_{0}^{T}4E_0^{2}\cos^{2}\left(kx - \omega t + \frac{\phi_1 + \phi_2}{2}\right) \cos^{2}\left(\frac{\phi_1 + \phi_2}{2}\right) \ dt
\end{align}
Ossia la media temporale dell'integrale del campo elettrico è l'intensità totale media
in un certo intervallo di tempo $\Delta t = T$, dove $T$ corrisponde al periodo di oscillazione del campo. Questo permette di esprimere
l'intensità delle due onde come l'intensità di una unica onda "somma". 
\begin{align}
     \left< I_{TOT} \right> =  4I_0\cos^{2} \left(\frac{\phi_1 - \phi_2}{2}\right)
\end{align}
Se $\phi_1 = \phi_2$ c'è un istante in cui l'intensità è molto maggiore dell'intensità
generata dalla semplice somma delle onde, mentre c'è un istante in cui si "distrugge"
l'energia delle due onde, ossia quando $\phi_1 = \pi + \phi_2$. Si può dunque riassumere che
\begin{itemize}
    \item Se $\phi_1 - \phi_2 = 2\pi n$: le onde sono in \textbf{fase}, e si ha interferenza costruttiva;
    \item Se $\phi_1 - \phi_2 = 2\pi n + \pi$: le onde sono in \textbf{controfase} e si ha interferenza distruttiva.
\end{itemize}

\clearpage 
\subsection{Oscillazione di due onde con polarizzazioni diverse}
\begin{wrapfigure}{r}{0.4\textwidth}
    \centering
    \caption{Campo elettrico lungo gli assi $y$ e $z$}
    \begin{tikzpicture}
        \draw[->](0, 0) -- (4, 0) node[at end, below] {$x$};
        \draw[->](0, 0) -- (1.5, 1.5) node[at end, above] {$\vv{E_1}$};
        \draw[->](0, 0) -- (0, 2) node[at end, right] {$y$} node[at end, left] {$\vv{E_2}$};
        \draw(0, 0.5) arc (90:45:0.5) node[midway, above] {$\alpha$};
        \draw[->](0, 0) -- (-1, -1) node[at end, right] {$z$};
    \end{tikzpicture}    
\end{wrapfigure}
Si considera il caso in cui due onde oscillino insieme ma con polarizzazioni diverse:
l'ampiezza e la direzione dei campi elettrici non
sono mai gli stessi per le due onde. Supponiamo ancora che abbiano
lo stesso $\hat{k}$ ma con due polarizzazioni lineari diverse: 
\begin{gather*}
    \vv{E_1} = \vv{E_{01}}\cos(kx - \omega t + \phi_1) \\
    \vv{E_2} = \vv{E_{02}}\cos(kx - \omega t + \phi_2)    
\end{gather*}
Si può ora scrivere il campo elettrico totale e poi determinarne l'intensità totale.
Dato che i due campi adesso hanno
stessa ampiezza e direzione ma con polarizzazioni diverse, si può determinare
la somma totale del vettore campo elettrico come il modulo
\begin{gather*}
    \left| \vv{E_{TOT}}  \right|^{2} = \left| \vv{E_1} + \vv{E_2}  \right|^{2}    = \left| \vv{E_1}  \right|^{2} + \left| \vv{E_2}  \right|^{2} + 2\vv{E_1} \cdot  \vv{E_2}      
\end{gather*}
Questo deve essere equivalente alla seguente espressione, ossia il modulo
del vettore campo elettrico totale:
\begin{gather*}
    |\vv{ E_{01}}|^{2} \cos^{2}(kx - \omega t + \phi_1) + \left| \vv{E_{02}}  \right|^{2}\cos^{2} (kx - \omega t + \phi_2) + 2\vv{E_{01}} \cdot  \vv{E_{02}}\cos(kx - \omega t + \phi_1) \cdot  \cos(kx - \omega t + \phi_2)     
\end{gather*}
Il terzo termine si può pensare come il prodotto scalare tra i due vettori che compongono 
il campo elettrico:
\begin{gather*}
    \left| \vv{E_{01}}  \right| \cdot  \left| \vv{E_{02}}  \right| \cos\alpha  
\end{gather*}
Dove $\alpha $ è l'angolo compreso tra i due vettori del campo elettrico come
nel disegno. Con le formule di prostaferesi inverse si esprime il coseno che moltiplica il 
terzo termine in funzione di angoli generici $\beta$ e $\gamma$ (gli argomenti dei due coseni) e 
ottenendo il terzo termine del modulo del campo totale come
\begin{gather*}
    2\left| \vv{E_{01}}  \right| \cdot  \left| \vv{E_{02}}  \right|\cos\alpha + \frac{1}{2} \left(\cos(2kx - \omega t + \phi_1 + \phi_2) + \cos(\phi_1 - \phi_2)\right) 
\end{gather*}
Si può esprimere allora l'intensità totale del campo elettrico come la somma delle intensità delle due onde
più l'intensità dovuta al terzo termine dell'espressione del modulo totale (Ossia
l'espressione di prima). 
\begin{gather*}
    \left< I_{TOT} \right> = I_1 + I_2 + c\epsilon_0 \frac{2\left| \vv{E_{01}}  \right| \left| \vv{E_{02}}  \right|\cos\alpha}{2} \cdot  \cos(\phi_1 - \phi_2)  
\end{gather*}
Sapendo ora che le due intensità si esprimono come
\begin{gather*}
    I_1 = c\epsilon_0 \cdot  \frac{\left| \vv{E_{01}}  \right|^{2}  }{2} \qquad I_2 = c\epsilon_0 \cdot \frac{\left| \vv{E_{02}}  \right|^{2} }{2}
\end{gather*}
Si può determinare l'intensità del terzo termine attraverso l'integrale
\begin{gather*}
    c\epsilon_0 \frac{1}{T}\int_{0}^{T}\left| \vv{E_{01}}  \right| \left| \vv{E_{02}}  \right|\cos\alpha\cos(\phi_1 - \phi_2 )dt = 2\sqrt{I_1}\sqrt{I_2}\cos\alpha\cos(\phi_1 - \phi_2)  
\end{gather*}
Che si può sostituire nell'espressione dell'intensità totale al terzo termine:
\begin{gather*}
    \left< I_{TOT} \right> = I_1 + I_2 + 2\sqrt{I_1}\sqrt{I_2}\cos\alpha \cos(\phi_1 - \phi_2)   
\end{gather*}
Il terzo termine dell'intensità totale è chiamato \textbf{termine di interferenza}. Questo termine
determina se c'è interferenza oppure no. Si possono dunque riassumere tre casi:
\begin{itemize}
    \item Se $\alpha = \frac{\pi}{2}$: non c'è alcuna interferenza poiché le polarizzazioni
    sono perpendicolari tra di loro.
    \item Se $I_1 = I_2$ e $\alpha = 0$, allora si ritrova il caso trattato nel paragrafo precedente:
    \begin{gather*}
        I_{TOT} = 2I_0 (1 + \cos(\phi_1 - \phi_2)) \ \Longrightarrow \ 4I_0 \cos^{2}\left(\frac{\phi_1 - \phi_2}{2}\right) 
    \end{gather*}
    \item Se $\alpha \neq  0$ e $\neq \frac{\pi}{2}$: si ha il caso con interferenza generale ricavato ora. 
\end{itemize} 


\clearpage
\section{L'onda stazionaria}
\begin{wrapfigure}{r}{0.4\textwidth}
    \centering
    \caption{Onda stazionaria}
    \begin{tikzpicture}[scale = 1.3]
        \draw[->](-2, 0) -- (2, 0) node[at end, below] {$x$};
        \draw[dashed](0, 1.5) -- (0, -1.5);
        \draw(-1, -1) -- (1, 1);
        \draw(1, -1) -- (-1, 1);
        \draw(-0.3, -0.3) arc (225:315:0.4) node[midway, below ] {$\theta$}; 
        \draw[very thick, ->](-0.8, -0.8) -- (-0.3, -0.3) node[midway, left] {$\vv{k_1}$};
        \draw[very thick, ->](0.8, -0.8) -- (0.3, -0.3) node[midway, right] {$\vv{k_2}$};
        \draw[very thick, ->](0, 0) -- (0.5, 0.5) node[midway, below] {$\vv{k_1}$};
        \draw[very thick, ->](0, 0) -- (-0.5, 0.5) node[midway, below] {$\vv{k_2}$ };
        \draw[very thick, ->, red](-0.5, 0.5) -- (0.5, 0.5) node[midway, above] {$\vv{\Delta k}$};
    \end{tikzpicture}    
\end{wrapfigure}
Considerate due onde elettromagnetiche che abbiano stessa polarizzazione
e i cui vettori d'onda $\vv{k_1}$ e $\vv{k_2}$ non sono paralleli tra di loro, se queste
onde interferiscono tra loro si ottiene un'\textbf{onda stazionaria}.  
\begin{gather*}
    \vv{E_1} = E_0 \hat{y}\cos(\vv{k_1} \cdot  \vv{r} - \omega t + \phi_1  )  \\
    \vv{E_2} = E_0 \hat{y}\cos(\vv{k_2} \cdot  \vv{r} - \omega t + \phi_2  )  
\end{gather*}  
Nel caso del modulo totale del campo elettrico:
\begin{gather*}
    \left| \vv{E_{TOT}}  \right|  = E_0^{2} \left(\cos(\vv{k_1} \cdot  \vv{r} - \omega t + \phi_1  ) + \cos(\vv{k_2} \cdot  \vv{r} - \omega t + \phi_2   )\right)^{2}  
\end{gather*}
Applicando nuovamente le formule di prostaferesi al contrario si ha la seguente espressione per il campo elettrico totale:
\begin{gather*}
    E_{TOT} = 4E_0^{2}\left(\cos^{2} \left(\frac{(\vv{k_1}  + \vv{k_2}) \cdot  \vv{r}  }{2} - \omega t + \frac{\phi_1 + \phi_2}{2}\right) \cdot  \cos^{2} \left(\frac{(\vv{k_1} - \vv{k_2}  )\cdot \vv{r} }{2} + \frac{\phi_1 - \phi_2}{2}\right)\right) 
\end{gather*}
C'è una parte del campo che ha dipendenza temporale ed una parte del campo che non ha dipendenza temporale. Se si volesse calcolare
l'intensità totale mediata nel tempo, si osserverà che la parte dipendente dal tempo 
risulterà essere mediata (e non dipenderà più dal tempo), mentre rimane la parte dipendente dallo spazio
\begin{gather*}
    \left< I_{TOT} \right> = 4c\epsilon_0 \frac{E_0^{2} }{2}\cos^{2}\left(\frac{\vv{k_1} - \vv{k_2}  }{2} \cdot \vv{r} + \frac{\phi_1 - \phi_2}{2} \right) = 4I_0\cos^{2}\left(\frac{\vv{k_1} - \vv{k_2}  }{2} \cdot \vv{r} + \frac{\phi_1 - \phi_2}{2}\right)  
\end{gather*}
Si trova allora che l'intensità dipende dalla posizione in cui si posiziona il rilevatore: l'intensità
è diventata funzione della posizione e non dipende dal tempo (dunque è una onda stazionaria).
Si potrebbe esprimere questa espressione attraverso il vettore differenza $\vv{\Delta k}$ tra i
due vettori d'onda (si noti come abbia direzione lungo $\hat{x}$ ):
\begin{gather*}
    \left< I_{TOT} \right> = 4I_0\cos^{2}\left(\frac{|\vv{\Delta k}| }{2} \cdot  x + \frac{\phi_1 - \phi_2}{2}\right)  
\end{gather*} 
Si esprime il prodotto del vettore d'onda come $\delta x$, ossia la periodicità spaziale
dell'onda stazionaria. Se si considera $\theta$ l'angolo tra i due vettori d'onda, 
allora il vettore dentro il coseno è esattamente:
\begin{gather*}
    \left| \vv{k}  \right| \sin\frac{\theta}{2} \cdot  \delta x = \pi  \ \Longrightarrow \ \frac{2}{\lambda}\sin\frac{\theta}{2} \delta x = 1
\end{gather*}
Dunque possiamo scegliere $\delta x$ in modo arbitrario
\begin{gather*}
    \delta x = \frac{\lambda}{2\sin\frac{\theta}{2}} \qquad \theta << 1 \ \Longrightarrow \ \delta x = \frac{\lambda}{\theta}
\end{gather*}


\begin{wrapfigure}{r}{0.4\textwidth}
    \centering
    \caption{Grafico interferenza }
    \begin{tikzpicture}[domain=0:6, scale=0.7, samples = 50]
        \draw[->](0, 0) -- (6, 0) node[at end, below] {$x$};
        \draw[->](0, 0) -- (0, 3) node[at end, left] {$I_0$};
        \draw[|-|](0, -0.2) -- (3.14, -0.2) node[midway, below] {$\delta x$};
        \filldraw (0,  2) circle (0pt) node[anchor = east] {$4I_0$};
        \draw plot (\x,{cos(2* \x r) + 1});
    \end{tikzpicture}    
\end{wrapfigure}
Se l'angolo tra le due onde è molto piccolo, allora si può approssimare con l'argomento e dunque
la lunghezza d'onda si amplifica.  Dato che l'onda stazionaria ha una lunghezza d'onda
\begin{gather*}
    \delta x = \frac{\lambda}{2\sin\left(\frac{\theta}{2}\right)}
\end{gather*}
A sinistra dell'interfaccia le interferenze dell'onda in ingresso
e quella in uscita generano un onda complessiva che è ferma nello spazio
ma oscilla nel tempo e prende proprio il nome di onda stazionaria. Questa onda
ha sempre dei punti in cui il campo elettrico è totalmente nullo e 
dunque l'intensità media viene diversa da zero. Se invece $\theta = \pi$,
le due onde sono \textbf{copropaganti} e di periodicità pari a $\frac{\lambda}{2}$.

\clearpage
\section{Interferometro di Michelson}
\begin{wrapfigure}{r}{0.4\textwidth}
    \centering
    \caption{L'interferometro di Michelson}
    \begin{tikzpicture}
        \draw[->](0, 0.2) -- (2.25, 0.2);
        \draw[->](2.25, 0.2) -- (2.25, 2);
        \draw[->](2.5, 2) -- (2.5, 0.75);
        \draw[->](2.5, 0.75) -- (1, 0.75);
        \draw[->](2.5, 0.75) -- (2.7, 0.35) -- (2.7, -2);
        \draw[->](2.25, 0.2) -- (2.55, 0) -- (4.5, 0);
        \draw[->](4.5, -0.25) -- (2.42, -0.25);
        \draw[->](2.42, -0.25) -- (2.42, -2);
        \draw[->](2.42, -0.25) -- (2.15, -0.1) -- (0, -0.1);
        \draw(1.75, -1) -- (2, -1.2) -- (3, 1) -- (2.75, 1.2);
        \draw[very thick](2.75, 1.2) -- (1.75, -1);
        \draw[pattern = north west lines, pattern color = white](1.75, 2) rectangle (3, 2.5);
        \draw[pattern = north west lines, pattern color = white](4.5, 0.75) rectangle (5, -1);
        \filldraw (2.1, -2) rectangle (3, -2.25);
    \end{tikzpicture}
\end{wrapfigure}
L'\textbf{interferometro di Michelson} è l'interferometro che si utilizza
durante l'esperienza dell'interferenza. Una prima onda piana è inviata su di un
componente ottico, che prende il nome di \textbf{separatore di fascio}, che è un componente che
ha un substrato di materiale dielettrico su una delle sue superfici:
il materiale dielettrico ha la caratteristica per cui la luce ha il
$50\%$ di probabilità di essere riflessa o trasmessa.  \\
Il fascio riflesso compie un cammino $L_1$ prima di incontrare uno specchio e
venire riflesso e tornare nuovamente sul separatore. Il fascio che inizialmente era
stato trasmesso procede per un cammino $L_2$ fino ad un nuovo specchio e poi torna indietro.
Questo interferometro permette di ottenere da un solo fascio di luce 
ben 4 fasci di luce distinti. \\
Sotto all'interferometro è posto un rilevatore per determinare
l'interferenza tra le due onde che giungono al rilevatore stesso.
Dato che il campo elettrico va come il quadrato mediato nel tempo, ogni volta che il
fascio attraversa l'interferometro  (da sinistra verso destra) il suo campo elettrico
diventa $E_r = -\frac{E_0}{\sqrt{2} }$ per il fascio riflesso (che accumula un ritardo di fase di $\pi$)
mentre per la parte trasmessa $E_t = \frac{E_0}{\sqrt{2} }$, la quale non accumula ritardo di fase. Ogni interferometro
è tarato per trasmettere o riflettere una certa percentuale del campo elettrico in entrata. 
Se invece si propagasse il fascio di luce da destra verso sinistra si ha che sia la
parte riflessa, che quella  trasmessa, hanno come campo elettrico
\begin{gather*}
    E_r = E_t = \frac{E_0}{\sqrt{2} }
\end{gather*}
La definizione di destra o sinistra dipende da dove è messo il materiale dielettrico 
prima del substrato di vetro. Questa tipologia di interferometro è utilizzato all'interno
dei rilevatori di onde gravitazionali; è anche uno degli strumenti più sensibili
mai costruiti dall'uomo. 
\paragraph{Fascio riflesso}
\begin{wrapfigure}{r}{0.4\textwidth}
    \centering
    \caption{Fascio riflesso}
    \begin{tikzpicture}
        \draw[->](0, 0.2) -- (2.25, 0.2);
        \draw[->](2.25, 0.2) -- (2.25, 2);
        \draw[->](2.5, 2) -- (2.5, 0.75);
        \draw[->](2.5, 0.75) -- (1, 0.75);
        \draw[->](2.5, 0.75) -- (2.7, 0.35) -- (2.7, -1);
        \draw(1.75, -1) -- (2, -1.2) -- (3, 1) -- (2.75, 1.2);
        \draw[very thick](2.75, 1.2) -- (1.75, -1);
        \draw[pattern = north west lines, pattern color = white](1.75, 2) rectangle (3, 2.5);
        \draw[|-|](2, 1.8) -- (2, 0.4) node[midway, left] {$L_1$};
    \end{tikzpicture}    
\end{wrapfigure}
Il campo elettrico che inizialmente va verso l'interferometro
ha la seguente espressione:
\begin{gather*}
    E_0\cos(\vv{k}\cdot \vv{x} - \omega t  )
\end{gather*}
L'onda inizialmente riflessa avrà come modulo del campo elettrico $-\frac{E_0}{\sqrt{2} }$.
Dato che alla fine si formano quattro fasci, il fascio che inizialmente è stato riflesso, e
poi trasmesso, avrà come modulo del campo elettrico
\begin{gather*}
    \frac{E_0}{\sqrt{2}\sqrt{2}  }\cos(\vv{k} \cdot \vv{r} - \omega t + \pi + 2kL_1 + \pi  )
\end{gather*}
Il primo $\pi$ è dovuto al ritardo di fase dovuto alla riflessione
dell'interferometro, mentre il termine $2kL_1$ è il termine di ritardo di fase dovuto alla
riflessione sullo specchio sopra all'interfaccia, ossia la distanza che percorre
la luce prima di tornare all'interfaccia. Il termine $\pi$ è dovuto invece al ritardo
di fase dovuto alla riflessione sullo specchio a distanza $L_1$. 

\paragraph{Fascio trasmesso} 
\begin{wrapfigure}{r}{0.4\textwidth}
    \centering
    \caption{Fascio trasmesso}
    \begin{tikzpicture}
        \draw[->](0, 0.2) -- (2.25, 0.2);
        \draw[->](2.25, 0.2) -- (2.55, 0) -- (4.5, 0);
        \draw[->](4.5, -0.25) -- (2.42, -0.25);
        \draw[->](2.42, -0.25) -- (2.42, -1);
        \draw[->](2.42, -0.25) -- (2.15, -0.1) -- (0, -0.1);
        \draw(1.75, -1) -- (2, -1.2) -- (3, 1) -- (2.75, 1.2);
        \draw[very thick](2.75, 1.2) -- (1.75, -1);
        \draw[pattern = north west lines, pattern color = white](4.5, 0.75) rectangle (5, -1);
        \draw[|-|] (2.6, -0.4) -- (4.3, -0.4) node[midway, below] {$L_2$};
    \end{tikzpicture}    
\end{wrapfigure}
L'altro contributo è quello del fascio che prima è trasmesso e poi è
riflesso allo specchio a distanza $L_2$ e poi è riflesso sull'interfaccia.
Si può dunque esprimere il campo elettrico di quel fascio di luce
che giunge sul rilevatore come 
\begin{gather*}
    \frac{E_0}{\sqrt{2} \sqrt{2} }\cos(\vv{k} \cdot  \vv{r} - \omega t  + \pi + 2kL_2)
\end{gather*}
Qui si ha un termine $\pi$, che è il ritardo di fase dovuto alla riflessione sullo
specchio a distanza $L_2$, e anche il contributo $2kL_2$ che invece è dovuto alla
distanza dall'interfaccia dello specchio. Per questo fascio di luce non c'è il termine
$\pi$ in quanto il fascio in riflessione non accumula ritardo di fase.
A questo punto si deve fare la somma dei campi elettrici complessivi in modo
tale da poter ottenere il campo elettrico risultante sul rilevatore come 
\begin{gather*}
    \frac{E_0}{2}\left(\cos(\vv{k}\cdot  \vv{r} - \omega t + 2kL_2  + \pi) + \cos(\vv{k} \cdot  \vv{r} - \omega t + 2kL_1  )\right)
\end{gather*}
Adesso si applicano le formule di Prostaferesi per ottenere il campo elettrico totale come
\begin{gather*}
    E_{TOT} = \frac{E_0}{2} \cdot  \left(2\cos\left(\vv{k} \cdot  \vv{r} - \omega t + k(L_1 + L_2) + \frac{\pi}{2}  \right) \cdot  \cos\left(k(L_2 - L_1) + \frac{\pi}{2}\right)\right)
\end{gather*}
Mediando il modulo del campo elettrico nel tempo si ottiene l'intensità media del fascio:
\begin{gather*}
    \left< I_{TOT} \right> = c\epsilon_0 \left< \left|\vv{E_{TOT}}\right|^{2}   \right> = I_0\cos^{2}\left(k(L_2 - L_1) + \frac{\pi}{2}\right) = I_0\sin^{2}(k(L_2 - L_1))    
\end{gather*}
Infine, si ottiene una nuova espressione per l'intensità esplicitando $k$:
\begin{align}
    I_{TOT} = I_0\sin^{2}\left(\pi \frac{L_2 - L_1}{\frac{\lambda}{2}}\right) 
\end{align}
Da questa si evince come basta cambiare la differenza tra i cammini ottici di solo $\frac{\lambda}{2}$
affinché il seno compia una oscillazione completa. Basta dunque cambiare la distanza degli specchi di meno di una lunghezza d'onda
per potersi accorgere della variazione dell'intensità del campo elettrico.

\subsection{Lo specchio con la camera a vuoto}
Supponendo di avere una impostazione simile, se lo specchio su cui deve riflettersi la luce
è nel vuoto ed è lasciato cadere, la luce che torna indietro dai due specchi interferirà
sul rilevatore. Si può esprimere lo spostamento dello specchio come
\begin{gather*}
    \Delta L(t) = L_1 - L_2 = -\frac{1}{2}gt^{2} + \cos t  
\end{gather*}
\begin{wrapfigure}{r}{0.4\textwidth}
    \centering
    \caption{Il grafico dei valori di $t$}
    \begin{tikzpicture}[domain=0:3.5]
        \draw[->](0, 0) -- (4, 0) node[at end, below] {$t$};
        \draw[->](0, 0) -- (0, 4) node[at end, left] {$\frac{gt^{2} }{\lambda}$};
        \draw plot (\x, {(\x * \x) / 4});
    \end{tikzpicture}    
\end{wrapfigure}
Allora l'intensità dovrà anch'essa dipendere dal tempo,
secondo la seguente espressione:
\begin{gather*}
    I(t) = I_0 \sin^{2}\left(\pi\frac{\Delta L(t)}{\frac{\lambda}{2}}\right) = \sin^{2}\left(\pi\frac{gt^{2} }{\lambda}\right)  
\end{gather*}
Immaginando di avere uno oscilloscopio al rilevatore, ci si aspetta che l'intensità possa
sempre variare tra zero ed uno e man mano ci si aspetta che
il periodo di oscillazione avvenga sempre più velocemente.
Questo vuol dire che il termine dentro al seno è tale per cui
\begin{gather*}
    \frac{gt_1^{2} }{\lambda} = 1 \qquad \frac{gt_2^{2} }{\lambda} = 2, \qquad \dots
\end{gather*}
Se si volesse riportare tutti i $t_i$ nei quali l'intensità del campo elettrico è zero, si otterrebbe
un andamento congruente con quello di una parabola, dal quale è possibile ricavare
un valore di $g$ con precisione fino all'ottava cifra decimale.


\subsection{Due lunghezze d'onda differenti}
Supponendo di mandare dentro all'interferometro di Michelson due fasci di
luce con due lunghezze d'onda differenti, ci si aspetterebbe una lettura diversa
al rilevatore. Quello che accade però è che il campo elettrico non subisce
alcuna modifica in quanto i due fasci con lunghezze d'onda diverse
non interferiscono tra di loro:
\begin{gather*}
    E_{TOT_1} = \frac{E_{01}}{2} \cdot  \left(2\cos\left(\vv{k_1} \cdot  \vv{r} - \omega t + k_1(L_1 + L_2) + \frac{\pi}{2}  \right) \cdot  \cos\left(k_1(L_2 - L_1) + \frac{\pi}{2}\right)\right) \\
    E_{TOT_2}= \frac{E_{02}}{2} \cdot  \left(2\cos\left(\vv{k_2} \cdot  \vv{r} - \omega t + k_2(L_1 + L_2) + \frac{\pi}{2}  \right) \cdot  \cos\left(k_2(L_2 - L_1) + \frac{\pi}{2}\right)\right)
\end{gather*}
Ci interessa solamente della dipendenza temporale del campo elettrico. Dato che
alcuni dei termini sono fissati, allora si può utilizzare una certa approssimazione
per evidenziare questa dipendenza temporale ottenendo,
rispettivamente, per i due campi totali:
\begin{gather*}
    A_1\cos(\omega_1 t + \phi_1) \qquad A_2\cos(\omega_2 t + \phi_2)
\end{gather*}
Dunque si ha il seguente campo elettrico sul rilevatore:
\begin{gather*}
      E_{TOT} = A_1\cos(\omega_1 + \phi_1) + A_2 \cos(\omega_2  + \phi_2)
\end{gather*}
Il modulo del campo elettrico totale è dunque:
\begin{gather*}
    \left| \vv{E_{TOT}}  \right|^{2} = A_1^{2}\cos^{2}(\omega_1 t + \phi_1) + A_2^{2}\cos^{2}(\omega_2 t + \phi_2) + 2A_1A_2 \cos(\omega_1 t + \phi_1) \cos(\omega_2 t + \phi_2)      
\end{gather*}
Il terzo termine è esattamente l'interferenza tra le due lunghezze d'onda nel tempo. 
Si può ora dimostrare che quel terzo termine mediato nel tempo è nullo: infatti si può, da prodotto,
trasformalo in una somma di coseni attraverso la formula di Prostaferesi:
\begin{gather*}
     \cos(\omega_1 t + \phi_1) \cos(\omega_2 t + \phi_2)= \frac{1}{2}\left(\cos\left(\frac{\omega_1 + \omega_2}{2}  t + \frac{\phi_1 + \phi_2}{2}\right) + \cos\left(\frac{\omega_1 - \omega_2}{2}t + \frac{\phi_1 -\phi_2}{2}\right)\right)
\end{gather*}
Mediando ora temporalmente questa espressione, si osserva che è zero.
Adesso l'intensità totale del campo mediata nel tempo sarà
\begin{align}
    I_{TOT}(t) = I_1 \sin^{2}\left(\pi\frac{\Delta L (t)}{\frac{\lambda_1}{2}}\right) + I_2\sin^{2}\left(\pi \frac{\Delta L(t)}{\frac{\lambda_2}{2}}\right)  
\end{align}
Dunque non c'è interferenza.

\subsection{Luce non monocromatica: il comportamento di più onde con frequenza diversa dentro l'interferometro}
\begin{wrapfigure}{r}{0.4\textwidth}
    \centering
    \begin{tikzpicture}
        \draw[->](0, 0.2) -- (2.25, 0.2);
        \draw[->](2.25, 0.2) -- (2.25, 2);
        \draw[->](2.5, 2) -- (2.5, 0.75);
        \draw[->](2.5, 0.75) -- (1, 0.75);
        \draw[->](2.5, 0.75) -- (2.7, 0.35) -- (2.7, -2);
        \draw[->](2.25, 0.2) -- (2.55, 0) -- (4.5, 0);
        \draw[->](4.5, -0.25) -- (2.42, -0.25);
        \draw[->](2.42, -0.25) -- (2.42, -2);
        \draw[->](2.42, -0.25) -- (2.15, -0.1) -- (0, -0.1);
        \draw(1.75, -1) -- (2, -1.2) -- (3, 1) -- (2.75, 1.2);
        \draw[very thick](2.75, 1.2) -- (1.75, -1);
        \draw[pattern = north west lines, pattern color = white](1.75, 2) rectangle (3, 2.5);
        \draw[pattern = north west lines, pattern color = white](4.5, 0.75) rectangle (5, -1);
        \filldraw (2.1, -2) rectangle (3, -2.25);
        \draw[|-|, red](2, 1.9) -- (2, 0.3) node[midway, left] {$L_1$};
        \draw[|-|, red](2.5, -0.5) -- (4.4, -0.5) node[midway, below] {$L_2$};
    \end{tikzpicture}
\end{wrapfigure}
Nel caso in cui si facciano entrare nell'interferometro più lunghezze d'onda,
si parla di luce \textbf{non monocromatica}. Si analizza il caso semplice nel quale 
solamente due lunghezze d'onda siano mandate dentro l'interferometro, ma con l'approssimazione
secondo la quale le intensità siano equivalenti per entrambi i
fasci di luce $I_1 = I_2 = I_0$. 
Sfruttando le formule di Prostaferesi e le considerazioni trigonometriche
sul seno quadro per cui $\sin^{2}(\alpha) = \frac{1 - \cos(\alpha)}{2}$, si esprime
l'intensità totale dei due fasci luminosi come 
\begin{gather*}
    I(\Delta L) = \frac{I_0}{2} \left(1 - \cos((k_1 + k_2) \Delta L) \cdot  \cos((k_1 - k_2) \Delta L)\right)
\end{gather*}
Il primo contributo del coseno rappresenta l'oscillazione più lunga mentre il secondo
è una oscillazione più corta (spazialmente) per cui le variazioni di lunghezza d'onda
$\delta l$ per entrambi i fasci sono date dalle seguenti relazioni:
\begin{gather*}
    \delta l_1 = \frac{2\pi}{k_1 + k_2} \\
    \delta l_2 = \frac{2\pi}{k_1 - k_2}
\end{gather*}
L'ampiezza del secondo termine determina la modulazione dell'onda che introduce
una dipendenza spaziale all'oscillazione delle onde. Spostando gli specchi di un certo $\Delta L$ 
nell'interferometro, la luce in ingresso risulta modulata. La rappresentazione grafica di questa interferenza è la seguente
\begin{gather*}
    \begin{tikzpicture}[domain=0:5, samples = 150]
        \draw(0, 0)-- (5, 0) node[at end, below] {$x$};
        \draw(0, 0) -- (0, 3) node[at end, left] {$\left< I \right>$ };
        \draw(-0.1, 1.5) -- (0.1, 1.5) node[at start, left] {$\frac{I_0}{2}$};
        \draw plot (\x, {1.5 * (1 - cos(9 * (\x + 1.5) r) * cos(5 * (\x + 1.5) r))});
    \end{tikzpicture}
\end{gather*}

\clearpage
\subsection{Lo spettro continuo}
\begin{wrapfigure}{r}{0.4\textwidth}
    \centering
    \caption{L'intensità luminosa in funzione della pulsazione}
    \begin{tikzpicture}[domain=0.5:3.5]
        \draw[->](0, 0) -- (4, 0) node[at end, below] {$\omega$};
        \draw[->](0, 0) -- (0, 3) node[at end, left] {$I(\omega)$};
        \draw plot (\x, {3* e^(-(\x - 2)^2)});
        \draw[dashed](2, 3) -- (2, 0) node[at end, below] {$\omega_0$};
        \draw[|-|](2.1, 1.5) -- (2.75, 1.5) node[midway,below] {$\Delta \omega$};
    \end{tikzpicture}    
\end{wrapfigure}
Si studia ora il caso di una distribuzione continua di frequenza, ossia
una sorgente luminosa di luce bianca: questa conterrà tutte le
lunghezze d'onda del visibile in maniera continua. Si può ora andare ad analizzare
cosa accade nell'interferometro al variare di $\Delta L$. 
La luce emessa da un led bianco è centrata intorno ad un valore di $\omega_0$ e $\lambda_0$
secondo una certa distribuzione dell'intensità luminosa. Dalla teoria sulle
onde elettromagnetiche si definisce la pulsazione attorno alla quale è centrata
l'emissione come 
\begin{gather*}
    \omega_0 = ck_0 \ \Longrightarrow \ \omega_0 = c \frac{2\pi}{\lambda_0}
\end{gather*}
Un led bianco ha $\lambda_0 \approx 550$ nm e $\Delta \lambda \approx 150$ nm. Per trovare il corrispettivo intervallo
in termini di $\omega$ si può usare, dato che esiste una relazione lineare tra
$\omega$ e $\lambda$, la seguente relazione:
\begin{gather*}
    \frac{\Delta \lambda}{\lambda_0} = \frac{\Delta \omega}{\omega_0}
\end{gather*}

Conosciuto $\Delta \lambda$, allora è possibile trovare $\Delta \omega$, ossia
l'intervallo della pulsazione della luce. Il $\Delta \lambda$ e $\Delta \omega$ sono
delle stime qualitative della distanza del centro della distribuzione
gaussiana della lunghezza d'onda dal punto di metà altezza. Con
queste considerazioni è possibile determinare l'intensità luminosa che
riceve il rilevatore in funzione dello scostamento delle distanze
dei due specchi $\Delta L$ in maniera discreta:
\begin{gather*}
    I(\Delta L) = \sum I_i \sin^{2}(k_i \Delta L) 
\end{gather*}
È possibile anche passare al caso continuo attraverso l'integrale
nella seguente maniera:
\begin{gather*}
    I(\Delta L) = \int I(\omega)\Delta \omega \sin^{2}(k(\omega) \Delta L) = \int I(\omega) \sin^{2}\left(\frac{\omega}{c}\Delta L\right)d\omega  
\end{gather*}
Il led non ha necessariamente una distribuzione gaussiana delle lunghezze
d'onda centrate intorno ad un certo $\lambda_0$, ma supponiamo che
lo sia entro certe approssimazioni. Si può cercare di risolvere l'integrale 
della funzione dell'intensità luminosa secondo la distribuzione Gaussiana:
\begin{gather*}
    I(\omega) = \frac{I_0}{\sqrt{\pi} \Delta\omega}\exp\left(-\frac{\omega - \omega_0}{\Delta \omega}\right)^{2} 
\end{gather*}
\begin{wrapfigure}{r}{0.4\textwidth}
    \centering
    \caption{Il grafico di $I(\Delta L)$. Il termine esponenziale determina
    quanto velocemente è smorzata mentre il termine nel coseno determina
    l'entità dello "smorzamento" della curva}
    \begin{tikzpicture}[domain=0:5, samples =120]
        \draw[->](0, 0) -- (5, 0) node[at end, below] {$x$};
        \draw[->](0, 0) -- (0, 2) node[at end, left] {$y$};
        \draw plot (\x, { 3 / 2 * (1 - exp(-(3* \x)^2 / (300)) * cos((2 * \x * 400) / (2)))});   
    \end{tikzpicture}    
\end{wrapfigure}
Di conseguenza la risoluzione dell'integrale di $I(\Delta L)$ è:
\begin{align}
    I(\Delta L) =     \frac{I_0}{2}\left(1 - \exp\left(-\frac{(\Delta L \Delta \omega)^{2} }{c}\right)\cos\left(\frac{2\Delta L\omega_0}{2}\right)\right)
\end{align}
Da questo si può vedere cosa succede quando $\Delta L \to 0$: 
\begin{gather*}
    \frac{I_0}{2} \left(1 - \cos\left(2\frac{\Delta L\omega_0}{2}\right)\right) \ \Longrightarrow \ \frac{I_0}{2}\sin^{2}(k_0\Delta L) 
\end{gather*}
Per $\Delta L$ molto piccoli i fasci luminosi oscillano 
normalmente mentre per $\Delta L$ non piccoli c'è da considerare anche il termine
esponenziale. Il coefficiente $\frac{I_0}{\sqrt{\pi}\Delta \omega }$ è tale per cui
si ottiene
\begin{gather*}
    \int I(\omega) d\omega = I_0
\end{gather*}
Si determina il valore minimo di $\Delta L$ tale per cui
si può osservare l'oscillazione dell'intensità luminosa. Questa lunghezza
è detta \textbf{lunghezza di coerenza}, e si indica con $L_c$ e si
definisce come l'ampiezza di oscillazione tale per cui
l'esponente di $e$ diventa 1, dunque
\begin{gather*}
    \frac{L_c \cdot  \Delta \omega}{c} = 1
\end{gather*}
Si ha quindi
\begin{align}
    L_c = \frac{c}{\Delta \omega} \ \Longrightarrow \ L_c = \frac{c}{\omega_0}\frac{\lambda_0}{\Delta \lambda}
\end{align}
Si è sostituito $\Delta \omega$ in funzione di $\lambda_0$ ma si può anche esprimere $\omega_0$ in funzione di $\lambda _0$ come $\omega_0 = c\frac{2\pi }{\lambda_0}$. Dunque
\begin{align}
        L_c = \frac{\lambda_0^{2} }{2\pi \cdot  \Delta \lambda}
\end{align}
Si scopre, facendo i conti, che per un led che emette luce bianca nel
visibile, la lunghezza di coerenza è di 400 nm. Se si volesse dunque
vedere l'oscillazione della luce per un led a luce bianca bisogna utilizzare una
lunghezza di coerenza di questa dimensione. 

\part{Diffrazione}
\chapter{La diffrazione della luce}
\section{Il concetto di diffrazione della luce}
\begin{wrapfigure}{r}{0.4\textwidth}
    \centering
    \caption{La luce in arrivo ad una fenditura}
    \begin{tikzpicture}
        \draw[<->](0, 3.2) -- (0.25, 3.2) node[midway, above] {$\lambda$};
        \draw[->](-1.5, 1.5) -- (-0.5, 1.5) node[at end, above] {$\hat{k}$ };
        \draw(0, 0) -- (0, 3);
        \draw(0.25, 0) -- (0.25, 3);
        \draw(0.5, 0) -- (0.5, 3);
        \draw(0.75, 0) -- (0.75, 3);
        \draw(1, 0) -- (1, 3);
        \draw[very thick](1.25, 0) -- (1.25, 1.25);
        \draw[very thick](1.25, 1.75) -- (1.25, 3);
        \node at (2, 1.5) {fenditura};
    \end{tikzpicture}    
\end{wrapfigure}
L'effetto della diffrazione compare quando l'estensione dei fronti d'onda piana non è più
infinita ma viene limitata spazialmente, per esempio da delle fessure. Dato che i fotoni
oscillano lungo il loro fronte d'onda, ci si aspetta una propagazione ondulatoria
della luce quando passa attraverso la fessura. Avrò dunque una funzione
oscillatoria del tipo $f(x - vt)$, per cui l'intensità di una onda elettromagnetica
è direttamente proporzionale al modulo del suo campo elettrico:
\begin{gather*}
    I \propto E^{2} \ \Longrightarrow \ I(R) \cdot 4\pi R^{2} \ \Longrightarrow \ I \sim \frac{1}{R^{2}}
\end{gather*}
Ossia, per un onda sferica che viene generata da una sorgente puntiforme, l'intensità va come 
uno sul quadrato della distanza rispetto alla sorgente in questione che l'ha generata. 

\begin{wrapfigure}{r}{0.4\textwidth}
    \centering
    \caption{Le sorgenti sulla fenditura}
    \begin{tikzpicture}
        \draw[<->](0, 3.2) -- (0.25, 3.2) node[midway, above] {$\lambda$};
        \draw[->](-1.5, 1.5) -- (-0.5, 1.5) node[at end, above] {$\hat{k}$ };
        \draw(0, 0) -- (0, 3);
        \draw(0.25, 0) -- (0.25, 3);
        \draw(0.5, 0) -- (0.5, 3);
        \draw(0.75, 0) -- (0.75, 3);
        \draw(1, 0) -- (1, 3);
        \draw(1.25, 0) -- (1.25, 1.25);
        \draw(1.25, 1.75) -- (1.25, 3);
        \filldraw[cyan](1.25, 1.35) circle (1pt);
        \filldraw[cyan](1.25, 1.45) circle (1pt);
        \filldraw[cyan](1.25, 1.55) circle (1pt);
        \filldraw[cyan](1.25, 1.65) circle (1pt);
    \end{tikzpicture}    
\end{wrapfigure}
Dunque i fronti d'onda che giungono alla fenditura devono essere pensati come se fossero fronti d'onda di onde sferiche generati da 
sorgenti puntiformi poste sulla fenditura stessa. 
Si sommano allora i loro contributi per determinare l'interferenza tra loro,
quando passano attraverso la fenditura, se i fronte d'onda sono limitati spazialmente: l'onda
si propaga allora nella direzione trasversa. Questo prende il 
nome di \textbf{principio di Huygens}. Dunque, dato $L >> \lambda$, si 
può considerare la propagazione dell'onda attraverso la fenditura come se l'onda generata dalla fenditura
fosse in controfase rispetto all'onda in ingresso e creasse dunque interferenza. A seconda della
differenza di fase (dato che hanno la stessa lunghezza d'onda), si può creare interferenza 
distruttiva se $OO' = \frac{\lambda}{2}$, altrimenti costruttiva e distruttiva, o solo costruttiva se 
$OO' = 0 + k\pi$. Nel caso di $\frac{\lambda}{2}$: 
\begin{gather*}
    \begin{tikzpicture}[domain=-6:0, samples = 50]
        \draw[->](0, 0) -- (-6, 0);
        \draw[->](0, 0) -- (0, 1.57) node[at end, right] {$y$};
        \draw[->](-1.57, 0 ) -- (-1.57, 1.57) node[at end, left] {$y'$};
        \filldraw(0, 0) circle (1pt) node[anchor = west] {$O$};
        \filldraw(-1.57, 0) circle (1pt) node[anchor = north] {$O'$};
        \draw[cyan] plot (\x, {sin(2 * (\x)r)});
        \draw[red] plot (\x, {sin(2 * (\x + pi / 2) r)});
    \end{tikzpicture}
\end{gather*}
Questo accade perché se si considera la somma del campo elettrico dovuta alla sorgente
uno e alla sorgente due, si vede che
\begin{gather*}
    E_0 \cos(ky - \omega t + \phi) + E_0 \cos(ky' - \omega t + \phi) = E_0\cos\left(\left(y - \frac{\lambda}{2}\right) -\omega t + \phi\right)
\end{gather*}
Ossia, diventa che
\begin{gather*}
    -E_0\cos(ky - \omega t + \phi) = 0
\end{gather*}
Che vale per due sorgenti poste a distanza $\frac{\lambda}{2}$, così come per la fenditura,
a patto che si abbia $L >> \lambda$, altrimenti non funzionerebbe. L'intensità 
che passa dunque lungo $y$ è molto debole, ossia
\begin{gather*}
    I = \frac{I_0}{L}\frac{\lambda}{2}
\end{gather*}

\section{Singola fenditura}
\begin{wrapfigure}{r}{0.5\textwidth}
    \centering
    \caption{L'onda che passa attraverso una fenditura}
    \begin{tikzpicture}
        \draw[->](-1, 0) -- (6.25, 0) node[at end, right] {$x$};
        \draw[->](6, -2) -- (6, 2) node[at end, right] {$y$};
        \draw(1, 0.1) circle (0.1);
        \draw(1, 0.3) circle (0.1);
        \draw[<->](0.75, 0) -- (0.75, 0.4) node[midway, left] {$y'$};
        \draw(0.75, 0.4) rectangle (1.25, 2);
        \draw(0.75, -0.4) rectangle (1.25, -2);
        \draw(-0.75, -2) -- (-0.75, 2);
        \draw(-0.5, -2) -- (-0.5, 2);
        \draw(-0.25, -2) -- (-0.25, 2);
        \draw(0, -2) -- (0, 2);
        \draw[cyan](1, 0.1) -- (6, 2) node[midway, below] {$r_0$};
        \draw[cyan](1, 0.3) -- (6, 2) node[midway, above] {$r_y'$};
        \draw[|-|](1.35, -2) -- (5.9, -2) node[midway, below] {$l$};
        \draw[cyan](2, 0.5) arc (30:0:0.9) node[midway, right] {$\theta$};
    \end{tikzpicture}    
\end{wrapfigure}
Considerata adesso una fenditura di larghezza $D$ investita da un'onda piana,
si vuole studiare la distribuzione di luce su di una superficie a grande distanza dalla fenditura ($l >> D$).
In ciano sono rappresentati i contributi delle sorgenti puntiformi poste sulla fenditura stessa,
adesso, poiché $r_0 \approx r_y'$, l'ampiezza delle due onde è circa la stessa
in corrispondenza dell'asse $y$. Di conseguenza $\theta$, ossia l'angolo che si forma tra l'asse che congiunge la
fenditura con la superficie del rilevatore e $r_0$, è piccolo, dunque
si può esprimere la dipendenza del campo elettrico $E_0$ e $E_y'$ in funzione
dell'inverso delle rispettive distanze dal rilevatore:
\begin{gather*}
    E_0 \propto \frac{1}{r_0} \qquad E_y' \propto \frac{1}{r_y'}
\end{gather*}
E, dato che $r_0 >> y'$, in laboratorio si avrà 
\begin{gather*}
    \frac{y'}{r_0} \approx 10^{-4}.
\end{gather*}
 Tuttavia, la piccola differenza tra $r_0$ e $r_y'$ influenza
la fase dell'onda. Il valore di $r_y'$, che corrisponde alla distanza tra
il punto più alto della fenditura con il punto più alto del rilevatore, può
essere ricavato attraverso la trigonometria (e utilizzando gli sviluppi di Taylor in quanto $\theta$ è piccolo):
\begin{gather*}
    x_y'  = \sqrt{y'^{2} + r_0^{2} - 2y'r_0\cos\left(\frac{\pi}{2}-\theta\right)} \ \Longrightarrow \ r_y' \approx r_0\left(1 - \frac{y'}{r_0}\sin\theta\right) = r_0 - y' \sin\theta
\end{gather*}
Il campo generato dall'onda piana sulla fenditura sarà dunque dato dalla somma dei contributi
su tutta la lunghezza della fenditura: si può dunque esprimere il singolo
contributo infinitesimo di ogni piccola sorgente puntiforme e dunque integrare:
\begin{gather*}
    dE(y') \propto E_0 \left(\frac{dy'}{D}\right)\cos(kx_y' - \omega t) 
\end{gather*}
Per cui il campo complessivo sarà dato dal seguente integrale:
\begin{gather*}
    \frac{E_0}{D}\int_{-\frac{D}{2}}^{\frac{D}{2}}dy'\cos(k r_0 - \omega t - ky' \sin\theta) 
\end{gather*}
Dunque 
\begin{align}
        E(\theta) \propto \frac{E_0}{D}\cos(kr_0 - \omega t) \int_{-\frac{D}{2}}^{\frac{D}{2}}\cos(ky'\sin\theta)dy' 
\end{align}
Si risolve l'integrale attraverso il seguente cambio di variabile
\begin{gather*}
    x = ky'\sin\theta \qquad dx = k\sin\theta dy'
\end{gather*}
Dunque si ottiene la seguente espressione
\begin{gather*}
    \frac{E_0}{kD\sin\theta} \cos(kr_0 - \omega t) \int_{-\frac{kD}{2}\sin\theta}^{\frac{kD}{2}\sin\theta}\cos x \ dx = \frac{E_0}{kD\sin\theta}\cos(kr_0 - \omega t)2\sin\left(\frac{kD}{2}\sin\theta\right)  
\end{gather*}
Si può introdurre il \textbf{seno cardinale}, ossia una funzione definita come $sinc = \frac{\sin x}{x}$, la quale
fa 1 per valori molto vicini allo zero, la quale è  possibile sostituirla nell'espressione dell'integrale
risolto per ottenere una soluzione più compatta. 
\begin{align}
        E(\theta) = E_0 \cos(kr_0 - \omega t) sinc\left(\frac{kD}{2}\sin\theta\right)
\end{align}
Si può allora studiare l'intensità che passa attraverso la fenditura attraverso la seguente espressione
\begin{align}
        I(\theta) = \left< E(\theta)^{2} \right>_t c\epsilon_0 = I_0 sinc^{2} \left(\frac{kD}{2}\sin\theta\right) 
\end{align}
Dato che
\begin{gather*}
    \frac{c\epsilon_0E_0^{2}}{2} = I_0 \qquad \left< \cos^{2}(kr_0 - \omega t)\right> = \frac{1}{2} 
\end{gather*}
Allora
\begin{gather*}
    I(\theta) = 0 \ \Longleftrightarrow \ x = \pi
\end{gather*}
La funzione presenta dunque un massimo sia a $x = 0$ che in $x = \frac{3}{2}\pi$ e un
minimo sia in $x = \pi$ che in $x = 2\pi$. Per capire perché c'è un minimo in $\pi$, si studia
il caso delle due sorgenti puntiformi in $y' = 0$ e l'altra in $y' = \frac{D}{2}$. Studiando
la differenza tra i loro cammini ottici per raggiungere il punto
sul rilevatore ad un angolo $\theta = \pi$: la differenza tra i cammini
ottici è proprio 
\begin{gather*}
    \frac{D}{2}\sin\theta \ \Longrightarrow \ kD\frac{1}{2}\sin\theta = \pi
\end{gather*}
Se il ritardo di fase è $\pi$, allora le due onde sono in controfase dunque
le onde si annullano tra di loro. Si è, fino ad ora, considerato solamente due sorgenti
puntiformi, ma questo ragionamento vale per qualsiasi coppia di
sorgenti luminose lungo l'intera fenditura. La diffrazione
è dunque un effetto direttamente proporzionale a $\frac{\lambda}{D}$. 


\section{Doppia fenditura}
\begin{wrapfigure}{r}{0.5\textwidth}
    \centering
    \caption{La doppia fenditura}
    \begin{tikzpicture}
        \draw[->](0.5, 0) -- (6.25, 0) node[at end, right] {$x$};
        \draw[->](6, -2) -- (6, 2) node[at end, right] {$y$};
        \draw(1, -2) -- (1, 2) node[at end, above] {$y'$};
        \filldraw(1, 0) circle (1pt) node[anchor = south west] {$O'$};
        \draw(0.75, 1) rectangle (1.25, 2);
        \draw(0.75, -1) rectangle (1.25, -2);
        \draw(0.75, 0.5) rectangle (1.25, -0.5);
        \draw[cyan](1, 0) -- (6, 2) node[midway, below] {$r_0$};
        \draw[cyan](1, 0.75) -- (6, 2) node[midway, above] {$r_y'$};
        \draw[|-|](1.35, -2) -- (5.9, -2) node[midway, below] {$l$};
        \draw[<->, cyan](0.5, 0.75) -- (0.5, -0.75) node[midway, left] {$a$};
        \draw[<->, red](0.25, 0.5) -- (0.25, 1) node[midway, left] {$D$};
        \draw[<->, red](0.25, -0.5) -- (0.25, -1) node[midway,left] {$D$};
        \draw[<->](-0.25, 0) -- (-0.25, 0.75) node[midway, left] {$y'$};
        \draw[green](2, 0.4) arc (30:0:0.8) node[midway, right] {$\theta$};
    \end{tikzpicture}    
\end{wrapfigure}
Supponendo di avere due fenditure distanziate tra loro di $a$, la quale è necessariamente
maggiore di $D$, e ponendoci sempre nel limite di \textbf{Fraunhofer} 
per cui $l >> D$, la relazione che intercorre
tra il cammino compiuto dalla luce nella fenditura rispetto a $O'$
si esprime come $r_y' = r_0 - y'\sin\theta$, allora il campo elettrico di ogni sorgente sarà dato
dalla seguente:
\begin{gather*}
    E(\theta) \propto \int E_0 \frac{dy'}{2D}\cos(kr_0 - \omega t - ky'\sin\theta)
\end{gather*}
Seguendo il ragionamento per la prima fenditura, si possono determinare gli
estremi di integrazione attraverso la considerazione sulle due sorgenti:
\begin{gather*}
    -\frac{a}{2} -\frac{D}{2} < y' < -\frac{a}{2} + \frac{D}{2} \\
    \frac{a}{2} - \frac{D}{2} < y' < \frac{a}{2}+ \frac{D}{2}
\end{gather*} 
Utilizzando le formule di prostaferesi si ottiene che, dato che il seno è
una funzione dispari, il suo integrale da $-a$ ad $a$ è nullo. Dunque, 
poiché si sta integrando su $y'$ e le sorgenti compaiono su due intervalli
di $y'$, si deve avere che
\begin{gather*}
    E(\theta) \propto \frac{E_0}{2}\left(\int dy' \cos (kr_0 - \omega t) \cos(ky' \sin\theta) + \int dy' \cos(kr_0 - \omega t)\cos(ky'\sin\theta)\right)
\end{gather*}

Per semplificare i calcoli si può eseguire la seguente sostituzione:
\begin{gather*}
    y'' = y' + \frac{a}{2} \qquad y'' = y' - \frac{a}{2}
\end{gather*}
In questo modo si può integrare sempre tra $\pm \frac{D}{2}$. 
\begin{gather*}
    \frac{E_0}{D}\int_{-\frac{D}{2}}^{\frac{D}{2}}\cos(kr_0 - \omega t) \cos(k\sin\theta y'')\cos(k\sin\frac{a}{2}) dy'' 
\end{gather*}
\begin{wrapfigure}{r}{0.4\textwidth}
    \centering
    \caption{Seno cardinale e intensità totale (in rosso)}
    \begin{tikzpicture}[domain=-2:2,samples = 100]
        \draw[->](-2, 0) -- (2, 0) node[at end, below] {$x$};
        \draw[->](0, 0) -- (0, 2) node[at end, left] {$y$};
        \draw plot (\x, {2 * (sin(3 * \x r) / (3 * \x))^2});
        \draw[red] plot (\x, {2 * (sin(3 * \x r) / (3 * \x))^2 * cos(2 * sin(3 * \x r) r)^2});
    \end{tikzpicture}    
\end{wrapfigure}
Risolvendo ora l'integrale si ottiene la seguente espressione per il campo elettrico generato da due fenditure come
\begin{align}
    \frac{2E_0}{kD\sin\theta} \cos(kr_0 - \omega t)\sin\left(\frac{kD\sin\theta}{2}\right)\cos\left(\frac{ka\sin\theta}{2}\right)
\end{align}
Adesso si può valutare l'intensità luminosa attraverso la sorgente secondo la seguente formulazione
\begin{align}
        I(\theta) = I_0 sinc^{2}\left(\frac{kD}{2}\sin\theta\right)\cos^{2}\left(k\frac{a}{2}\sin\theta\right)
\end{align}
Dato che il coseno quadro varia più velocemente del seno cardinale, quando $a  >0$
ci si aspetta che la funzione sia contenuta dal seno cardinale. La funzione
dunque si annulla in $\frac{kD}{2}\sin\theta = \frac{\pi}{2}$, ossia quando $\sin\theta = \frac{\lambda}{2a}$. 

\section{Reticolo di diffrazione: il caso di $n$ fenditure}
    \begin{wrapfigure}{r}{0.37\textwidth}
        \centering
        \caption{Grafico intensità totale: l'intensità totale,
        anche in questo caso, è contenuta nel seno cardinale}
        \begin{tikzpicture}
            \draw(0, 0) -- (4, 0);
            \draw(0, 0) -- (0, 3);
            \draw[domain = 0.01:3, samples = 80, cyan] plot (\x, {2.5 * (sin(3 * \x r) / (3 * \x))^2});
            \draw[domain = 0.01:3, red, samples = 120] plot (\x, {0.005 * (sin(20 * \x r) * sin(20 * \x r)) / (sin(\x r) * sin(\x r))});
        \end{tikzpicture}    
    \end{wrapfigure}
Si vuole ora studiare il caso in cui ci siano $n$ fenditure, ognuna spaziata 
tra loro di una certa distanza $a > D$, sempre nel limite di Fraunhofer $l >> D$.  Per ottenere il campo 
totale passante attraverso le fenditure, si deve semplicemente sommare tutti i contributi
\begin{gather*}
    E^{T} (\theta) = \sum_{i = -\frac{n}{2}}^{\frac{n}{2}} \frac{E_0}{n}\cos(kr_0  - \omega t) sinc\left(\frac{kD}{2} \sin\theta\right) \cos(k\sin\theta ai)
\end{gather*}
Si deve necessariamente avere che il campo in funzione dell'angolo $\theta$ sia in funzione di
\begin{gather*}
    E(\theta) \propto \frac{E_0}{n} \cos(kr_0 - \omega t) sinc\left(\frac{kD}{2}\sin\theta\right)\sum_{i = -\frac{N}{2}}^{\frac{N}{2}}\cos(i ak\sin\theta) 
\end{gather*}
Si può dunque esprimere l'intensità totale uscente dalle fenditure come
\begin{align}
    I(\theta) = I_0 sinc^{2}\left(\frac{kD}{2}\sin\theta\right) \frac{\sin^{2}\left(\frac{n}{2}ak\sin\theta\right)}{n^{2}\sin^{2}\left(\frac{ka}{2}\sin\theta\right)}
\end{align}
Partendo dal seno cardinale, esso si annulla quando l'argomento è $\pi$, ossia
\begin{gather*}
    I(\theta) = 0 \ \Longrightarrow \ \frac{kD}{2} \sin\theta = \pi \ \Longrightarrow \ \sin\theta = \frac{\lambda}{D}
\end{gather*}
Di conseguenza si avranno anche altri minimi per $m \in \mathbb{Z}$, ossia per tutti i multipli di $\pi$, dunque
\begin{gather*}
    \sin\theta = m\frac{\lambda}{D} \qquad m \in \mathbb{Z}
\end{gather*}
Adesso, per trovare i massimi, si impone che il denominatore della frazione a moltiplicare si
annulli, dunque si deve imporre
\begin{gather*}
    I(\theta) = 0 \qquad \frac{ak}{2}\sin\theta = \pi \ \Longrightarrow \ \sin\theta = m\frac{\lambda}{a} \qquad m \in \mathbb{Z}
\end{gather*}
Adesso, il primo minimo si ha quando il seno quadro si annulla, ossia quando 
\begin{gather*}
    \sin\theta = \frac{\lambda}{na}
\end{gather*}
Se l'angolo $\theta << 1$, allora è possibile determinare il numero di fenditure
illuminate da una certa lunghezza d'onda che passa attraverso delle fenditure
spaziate di $a$ secondo la seguente equazione:
\begin{gather*}
    \sin\theta \approx \theta \approx \frac{\lambda}{na}
\end{gather*}
Questo perché la condizione $\sin\theta = \frac{\lambda}{na}$ corrisponde al
primo minimo (zona di buio).



\subsection{Caso tridimensionale}
Nel caso tridimensionale si ha che l'intensità totale
\begin{gather*}
    I(\theta, \phi) = I_0 sinc^{2}\left(\frac{kDx}{2}\sin\phi\right)sinc^{2}\left(\frac{kDy}{2}\sin\theta\right)
\end{gather*}
Nel caso di una fenditura circolare per esempio, si ha una diffrazione maggiore
della luce.

\section{Diffrazione su una lente}
\subsection{Lente singola}
Per schematizzazione supponiamo che la lente sia investita da più
onde piane ad angolazioni diverse. Tenendo conto della diffrazione, l'immagine
non è più un punto: infatti, considerando $\delta y$ come la dimensione dell'immagine
sorgente posta all'infinito che passa attraverso una lente di dimensione $D$, 
si può esprimere $\delta y$ come 
\begin{gather*}
    \delta y = 1.22 \frac{\lambda}{D}f \theta \approx f \frac{\lambda}{D}
\end{gather*}

\subsection{Diffrazione sul microscopio}
\begin{wrapfigure}{r}{0.5\textwidth}
    \centering
    \caption{Schematizzazione di un microscopio}
    \begin{tikzpicture}
        \draw(0, 0) -- (7, 0);
        \draw[<->](1.5, -1) -- (1.5, 1);
        \draw[<->](5, -1.5) -- (5, 1.5);
        \draw[red](0, 0) -- (0, 0.5);
        \draw(0, 0.5) -- (1.5, 0) -- (5, -1.25) -- (6, -0.75);
        \draw(0, 0.5) -- (1.5, 0.5) -- (5, -0.75) -- (6, -0.75);
        \draw(1.5, 0.5) -- (5, -0.5) -- (6, -0.5);
        \draw(4.5, 0) arc (0:-20:1);
        \draw(2, 0) arc (180:160:1);
        \draw[red](6, 0) -- (6, -0.75);
        \draw[<->, red](1.5, -1.6) -- (3, -1.6) node[midway, below] {$f_1$};
        \draw[<->, red](3, -1.6) -- (5, -1.6) node[midway, below] {$f_2$};
        \draw[<->, red](5, -1.6) -- (6, -1.6) node[midway, below] {$f_2$};
        \draw(6, -0.75) -- (1.6, 1.1);
        \draw[cyan](6, -0.6) -- (1.5, 1) node[at end, left] {$\frac{\lambda}{D_1}$};
    \end{tikzpicture}    
\end{wrapfigure}
Il principio di funzionamento di un microscopio è quello di ingrandire 
di molto oggetti molto piccoli, spesso non visibili all'occhio umano.
Per determinare gli ingrandimenti di un microscopio, si utilizza, a partire
dalla relazione del costruttore di lenti, la seguente relazione:
\begin{gather*}
    I = \frac{f_2}{f_1}
\end{gather*}
La diffrazione dopo la prima lente è data da
\begin{gather*}
    \delta y_1 \approx f_2 \frac{\lambda}{D_1}
\end{gather*}
Dove $D_1$ è la dimensione della lente 1,
mentre la diffrazione sulla seconda lente è data da
\begin{gather*}
    \delta y_2 \approx f_2 \frac{\lambda}{D_2}
\end{gather*}
Allora la diffrazione totale che si ha sull'immagine diventa:
\begin{gather*}
    \delta y_t = f_2 \left(\frac{\lambda}{D_1} + \frac{\lambda}{D_2}\right)
\end{gather*}
Si deve avere che $f_1 << f_2$ in modo tale da avere grande magnificazione, 
allora per la legge del costruttore di lenti, si considera una lente biconvessa, la cui
focale totale sarà data da:
\begin{gather*}
    f = \frac{n_1}{n_2 - n_1}\frac{R_1R_2}{R_2- R_1}
\end{gather*}
Dove $R_2 < 0$ e $R_1 > 0$. Si suppone allora che $\left| R_2 \right| = \left| R_1 \right| = R$,
e che $n_1$ sia il coefficiente di rifrazione del vetro e $n_2$ il coefficiente di rifrazione
dell'aria, allora 
\begin{gather*}
    f = 2\frac{-R^{2}}{-2R} = R
\end{gather*}  
Per avere $f \equiv f_1$, si deve avere un $R_1$ molto piccolo, il che porta ad avere
anche un $D_1$ molto piccolo, aumentando notevolmente la diffrazione sulla prima lente.
Questo vuol dire che 
\begin{gather*}
    \frac{\lambda}{D_1} >> \frac{\lambda}{D_2}
\end{gather*}
E dunque la diffrazione totale rispetto alla sorgente terrà conto solo della prima 
lente:
\begin{gather*}
    \delta y_1 \approx I\delta y_s \ \Longrightarrow \ \delta y_s \approx f_1 \frac{\lambda}{D_1}
\end{gather*}

\subsection{Diffrazione per un telescopio}
\begin{gather*}
    \begin{tikzpicture}
        \draw(0, 0) -- (10, 0);
        \draw[->, red](0.5, 0) -- (0.5, 0.75);
        \draw[<->](4.5, -1.5) -- (4.5, 1.5);
        \draw[dashed](6.5, -1.5) -- (6.5, 1.5);
        \draw[<->](7.5, -1.5) -- (7.5, 1.5);
        \draw[->](9, -1.5) -- (9, 1.5) node[at end, right] {$y$};
        \draw[red](0.5, 0.75) -- (6.5, -0.37) -- (9, 0.55);
        \draw[dashed, red](4.5, 0) -- (6.5, -0.75) -- (9, 1.25);
        \draw[cyan](0.5, 0) -- (6.5, 0);
        \draw[cyan, dashed](4.5, 0) -- (6.5, 0.5);
        \draw(8.25, 0.27) arc (18:0:1) node[midway, right] {$\beta$};
        \draw(3.5, 0) arc (180:170:1) node[midway, left] {$\alpha$};
        \draw[|-|](4.5, -1.6) -- (6.45, -1.6) node[midway, below] {$f_1$};
        \draw[|-|](6.55, -1.6) -- (7.5, -1.6) node[midway, below] {$f_2$};
    \end{tikzpicture}
\end{gather*}
Riassumendo il principio di funzionamento di un telescopio, per avere
ingrandimento di una certa sorgente luminosa, dovrei costruire un sistema
di lenti tali che  $f_1 > f_2$. Questo vuol dire che 
\begin{gather*}
    f_1 \tan \alpha = f_2 \tan \beta
\end{gather*}  
Ossia i rapporti tra gli angoli mi permettono di ottenere un ingrandimento maggiore di zero:
\begin{gather*}
    \frac{f_1}{f_2} = \frac{\beta}{\alpha} > 1
\end{gather*}
Dato che le sorgenti che si 
osservano non sono puntiformi ma estese, ci si deve immaginare che ci siano altri raggi luminsoi 
che impattano sulla lente, i quali risultano deflessi di un certo angolo
rispetto ai raggi che arrivano dalla sorgente luminosa. L'immagine in corrispondenza
del fuoco primario della prima lente ha un allargamento (spread), a causa
della diffrazione della prima lente pari a
\begin{gather*}
    \delta y_1 = f_1 \cdot \frac{\lambda}{D_1}
\end{gather*}
I dettagli dell'oggetto osservato non sono più definiti e
l'immagine tende dunque a sfocarsi. La differenza angolare associata a questo 
spread è esprimibile come
\begin{gather*}
    \delta \beta_1 = \frac{\delta y_1}{f_2} = f_1\frac{\lambda}{D_1} \frac{1}{f_2}
\end{gather*}
Anche la seconda lente sarà responsabile di un certo spread dei raggi luminosi, il 
suo contributo angolare è esprimibile dunque come
\begin{gather*}
    \delta \beta_2 = \frac{\lambda}{D_2}
\end{gather*}
L'effetto combinato dello spread di questo sistema di lenti è dunque dato dalla seguente relazione
\begin{gather*}
    \delta \beta = \frac{\lambda}{D_1} \frac{f_1}{f_2} + \frac{\lambda}{D_2}
\end{gather*}
Dove si è usato $\tan \beta_1 \approx \beta_1$ e $\tan \beta_2 \approx \beta_2$
secondo l'approssimazione parassiale. È possibile, inoltre, trascurare il secondo termine
dello spread su $\beta$ in quanto è molto piccolo e considerare solamente il termine 
dovuto dal rapporto delle focali. Ricordando ora le considerazioni fatte per il telescopio,
è possibile trovare il minimo angolo osservabile dal telescopio mettendolo
in relazione con lo spread dei raggi luminosi attraverso la seguente:
\begin{gather*}
    \delta \alpha \frac{f_1}{f_2} = \delta \beta \ \Longrightarrow \ \delta \alpha \approx \frac{\lambda}{D_1}
\end{gather*} 
Così come nel caso del microscopio, tutto dipende dalla prima lente.





\part{Esperienze}
\chapter{Ottica Geometrica}
\section{Introduzione: obiettivi e finalità}
\begin{wrapfigure}{r}{0.4\textwidth}
    \centering
    \caption{Lente convergente con sorgente nel punto focale}
    \begin{tikzpicture}
        \draw(0, 0) -- (5, 0);
        \draw[<->](2.5, -1.5) -- (2.5, 1.5);
        \filldraw(1, 0) circle (1pt);
        \draw(1, 0) -- (2.5, 1) -- (4, 0);
        \draw(1, 0) -- (2.5, 0.5) -- (4, 0);
        \draw(1, 0) -- (2.5, -0.5) -- (4, 0);
        \draw(1, 0) -- (2.5, -1) -- (4, 0);
        \draw[|-|](1, 1.5) -- (2.4, 1.5) node[midway, above] {$p$};
        \draw[|-|](2.6, 1.5) -- (4, 1.5) node[midway, above] {$q$};
    \end{tikzpicture}    
\end{wrapfigure}
L'esperienza dell'ottica geometrica è la verifica della legge delle
lenti sottili e di una misura della focale di una lente convergente incognita. 
Ricordando la legge delle lenti sottili
\begin{gather*}
    \frac{1}{p} + \frac{1}{q} = \frac{1}{f}
\end{gather*}
Si eseguono seguenti procedure in laboratorio
\begin{enumerate}
    \item Ogni sperimentatore misura
    $f \pm \Delta f$ ad un valore diverso $p$ (e di conseguenza di $q$);
    \item Se tutte le misure sono consistenti 
    entro le barre di errore, allora si sarà dimostrata la legge delle lenti sottili. 
    \item Determinare la migliore stima di $f$ mediante la media pesata
    del valore di $f$ trovato da ogni sperimentatore.
\end{enumerate}

\section{L'apparato sperimentale}
\begin{wrapfigure}{r}{0.5\textwidth}
    \centering
    \caption{Rappresentazione apparato Lente di servizio e oculare-occhio}
    \begin{tikzpicture}[scale=1.5]
        \draw(0, 0) -- (4,0);
        \draw[thin](0.5 ,1.25) -- (4, 1.25) node[at end, above] {asse ottico};
        \draw(0, 0) -- (0, 1.5);
        \draw(0.5, 0) -- (0.5, 1.5);
        \draw[thick](0.5, 1) -- (0.5, 1.5); 
        \draw[|-|](0.5, 1.7) -- (1.5, 1.7) node[midway, above] {$d$};
        \draw(1.2, 0) rectangle (1.8, 0.4);
        \draw(1.4, 0.4) rectangle (1.6, 1);
        \draw(1.5, 1.25) ellipse (0.1 and 0.25);
        \draw(0.5 , 1.25) -- (1.5, 1.5) -- (2.5, 1.25);
        \draw(0.5, 1.25) -- (1.5, 1) -- (2.5, 1.25);
        \filldraw(2.5, 1.25) circle (1pt);
        \node at (2.5, 1.75) {sorgente virtuale};
        \draw(2.3, 1) rectangle (3, 1.5);
    \end{tikzpicture}    
\end{wrapfigure}
L'apparato sperimentale consiste in un regolo lungo circa
un metro e mezzo con precisione di un millimetro. Una lente
è posta ad una certa distanza con una scala graduata (nonio)
che mi permette di ottenere una precisione di posizionamento
del decimo di millimetro. La sorgente è realizzata mediante un led che 
illumina una diapositiva in modo tale che il led giaccia esattamente
sull'asse ottico rispetto alla lente e parallelo alla guida.
Per determinare la distanza della diapositiva dalla lente, devo fare in
modo di creare una immagine virtuale utilizzando la lente convergente con $f = 50 \ mm$
nella configurazione $2f - 2f$ (dove $f$ è quella della lente conosciuta). 

\begin{wrapfigure}{r}{0.4\textwidth}
    \centering
    \caption{Schematizzazione del sistema oculare-occhio, in rosso il crocefilo}
    \begin{tikzpicture}
        \draw(0, 0) -- (5, 0);
        \draw[red, thick](0.5, -1) -- (0.5, 1);
        \draw[thin](0.5, 0) -- (1.5, 1) -- (3, 1) -- (4, 0);
        \draw[thin](0.5, 0) -- (1.5, -1) -- (3, -1) -- (4, 0);
        \draw[<->, thick](1.5, -1) -- (1.5, 1);
        \draw[<->, thick](3, -1) -- (3, 1);
    \end{tikzpicture}    
\end{wrapfigure}
La lente che si utilizza prende il nome di \textbf{lente di servizio} per cui:
\begin{gather*}
    \frac{1}{q} = \frac{1}{2f}
\end{gather*}
Per cui $d = 2f$. L'incertezza associata a questa distanza non è fondamentale
in quanto l'immagine nella sorgente virtuale si formerà (più o meno
sfuocata) a prescindere dalla precisione. 
Ad una certa distanza c'è un oculare che mi permette di osservare
l'immagine ottenuta. Sull'oculare c'è un crocefilo che mi permette di osservare
l'allineamento dell'immagine sull'oculare. Il fatto che si utilizza
un oculare con una lente davanti (che è possibile ruotare con una ghiera),
vuol dire che si utilizza l'occhio per osservare l'immagine. 
Si dispone l'oculare in modo tale che il crocefilo sia visibile e anche l'immagine
(questo accade quando il mio occhio li vede nitidi entrambi) e in modo
tale che l'immagine sia sovrapposta esattamente sul crocefilo.
La schematizzazione oculare-occhio comprende il crocefilo con una lente
associata ed il cristallino dell'occhio (che deve essere completamente rilassato
in modo tale da avere il fuoco all'infinito).
Nel disegno il tratto rosso è il crocefilo, le due lenti convergenti sono invece la
lente prima dell'occhio e poi il cristallino. 
Adesso si può spostare l'oculare dal punto dove si forma
l'immagine virtuale e mettere una lente ad una certa distanza
$p$ dall'immagine virtuale ed una certa distanza $q$ dall'oculare nuovo.
\begin{gather*}
     \begin{tikzpicture}[scale=1.2]
        \draw(0, 0) -- (7.5,0);
        \draw[thin](0.5 ,1.25) -- (7.5, 1.25) node[at end, above] {asse ottico};
        \draw(0, 0) -- (0, 1.5);
        \draw(0.5, 0) -- (0.5, 1.5);
        \draw[thick](0.5, 1) -- (0.5, 1.5); 
        \draw[|-|](0.5, 1.7) -- (1.5, 1.7) node[midway, above] {$d$};
        \draw(1.2, 0) rectangle (1.8, 0.4);
        \draw(1.4, 0.4) rectangle (1.6, 1);
        \draw(1.5, 1.25) ellipse (0.1 and 0.25);
        \draw(0.5 , 1.25) -- (1.5, 1.5) -- (2.5, 1.25) -- (4.25, 0.9) -- (6, 1.25);
        \draw(0.5, 1.25) -- (1.5, 1) -- (2.5, 1.25) -- (4.25, 1.6) -- (6, 1.25);
        \filldraw(2.5, 1.25) circle (1pt) node[anchor = north, align = center] {sorgente \\ virtuale};
        \draw(3.95, 0) rectangle (4.55, 0.4);
        \draw(4.15, 0.4) rectangle (4.35, 0.9);
        \draw(4.25, 1.25) ellipse (0.15 and 0.35);
        \draw[|-|](2.5, 1.7) -- (4.2, 1.7) node[midway, above] {$p$};
        \draw[|-|](4.3, 1.7) -- (6, 1.7) node[midway, above] {$q$};
        \draw(5.8, 1) rectangle (6.5, 1.5);
    \end{tikzpicture}  
\end{gather*}

\section{La soluzione di Gauss}
\begin{wrapfigure}{r}{0.4\textwidth}
    \centering
    \caption{Schematizzazione della soluzione di Gauss}
    \begin{tikzpicture}
        \draw(0, 0) -- (6, 0);
        \filldraw(1, 0) circle (1pt) node[anchor = north east] {$S$};
        \draw[<->](2, -1.25) -- (2, 1.25);
        \draw(1, 0) -- (2, 1) -- (5, 0);
        \draw(1, 0) -- (2, -1) -- (5, 0);
        \draw[dashed](1, -2) -- (1, 1.75);
        \draw[|-|](1.1, -2) -- (1.9, -2) node[midway, below] {$p_1$};
        \draw[|-|](2.1, -2) -- (4.9, -2) node[midway, below] {$q_1$};
        \filldraw(5, 0) circle (1pt) node[anchor =north west] {$I$};
        \draw[dashed](5, 1.75) -- (5, -2);
        \draw[|-|](1.1, 1.75) -- (4.9, 1.75) node[midway, above] {$s$};
        \draw[|-|](2.1, 1.25) -- (3.9, 1.25) node[midway, above] {$l$};
        \draw[red](1, 0) -- (4, 1) -- (5, 0);
        \draw[red](1, 0) -- (4, -1) -- (5, 0);
        \draw[red, <->](4, -1.25) -- (4, 1.25);
        \draw[|-|](1.1, -1.8) -- (3.9, -1.8) node[midway, above] {$p_2$};
        \draw[|-|](4.1, -1.8) -- (4.9, -1.8) node[midway, above] {$q_2$}; 
    \end{tikzpicture}    
\end{wrapfigure}
Il problema principale con la configurazione in laboratorio è che il centro di formazione dell'immagine virtuale ha un certo offset rispetto allo zero del
nonio: l'allineamento centro della lente - lettura del nonio introduce
dunque un errore $\epsilon$ per cui bisogna trovare un modo alternativo
per poter misurare $p$ e $q$.  Gauss ha proposto una soluzione a questa
problematica: se esistesse una configurazione per $p$ e $q$ per cui la legge
delle lenti sottili è soddisfatta, allora esiste anche una configurazione 
simmetrica per cui:
\begin{gather*}
    \frac{1}{p_1} + \frac{1}{q_1} = \frac{1}{f} \\
    \frac{1}{p_2} + \frac{1}{q_2} = \frac{1}{f} \\
    \ \Longrightarrow \ p_1 = q_2 \ \wedge \ q_1 = p_2
\end{gather*}
Ossia si invertono le distanze rispetto alla lente convergente considerata. si può 
dunque pensare che se si disponesse la lente ad una certa distanza $p_2 \neq p_1$,
si otterrebbe una situazione in cui la stessa sorgente $S$ possa 
far formare la stessa immagine in $I$. si può 
ora introdurre delle lunghezze (diverse da $p$ e $q$, ma comunque loro funzione) che 
io conosco a priori e che sono definite come
\begin{enumerate}
    \item $s = p_1 + q_1 = p_2 + q_2$: ossia la distanza tra la sorgente e l'immagine;
    \item $l = p_2 - p_1 = q_1 - p_1$: ossia la distanza tra le due lenti nelle due
    differenti configurazioni.  
\end{enumerate} 
Con queste definizioni e con la legge delle lenti sottili si ottiene un sistema
per cui, conosciuto $s$ e $l$, si può ottenere $p_1$ e $q_1$:
\begin{gather*}
    \left\{\begin{array}{l}
        s = p_1 + q_1 \\
        l = q_1 - p_1 \\
        \frac{1}{f} = \frac{1}{p_1} + \frac{1}{q_1}
    \end{array}\right.
\end{gather*}
Dunque, risolvendo, si esprimono $p_1$ e $q_1$:
\begin{gather*}
    \left\{\begin{array}{l}
        q_1 = \frac{s + l}{2} \\
        p_1 = \frac{s - l}{2} \\
    \end{array}\right.
\end{gather*}
Adesso possiamo sostituire queste nella legge delle lenti sottili:
\begin{gather*}
    \frac{1}{f} = \frac{2}{s - l} + \frac{2}{s + l} = \frac{4s}{s^{2} - l^{2}  }\ \Longrightarrow \ f = \frac{s^{2} - l^{2}  }{4s}
\end{gather*}
Ossia l'espressione della focale della lente in funzione di soli $s$ ed $l$. 

\section{Applicazione della soluzione di Gauss all'esperienza geometrica}
\begin{gather*}
    \begin{tikzpicture}
        \draw(0, 0) -- (6, 0);
        \draw(1, -0.2) -- (1, 0.2) node[at start, below] {$I_1^{V}$};
        \draw(2, -0.2) -- (2, 0.2) node[at start, below] {$I_1^{N}$};
        \draw(4, -0.2) -- (4, 0.2) node[at start, below] {$I_2^{V}$};
        \draw(5, -0.2) -- (5, 0.2) node[at start, below] {$I_2^{N}$};
        \draw[|-|](1, 0.5) -- (2, 0.5) node[midway, above] {$\epsilon$};
        \draw[|-|](4, 0.5) -- (5, 0.5) node[midway, above] {$\epsilon$};
    \end{tikzpicture}
\end{gather*}
Per poter trovare $s$ e $l$ per porci nelle condizioni della soluzione di Gauss,
si possono ora indicare le posizioni:
\begin{itemize}
    \item $I_1^{N}$ la posizione misurata con il nonio dell'oculare
    quando la sorgente virtuale coincide con il crocefilo dell'oculare: ossia quando
    per il nostro occhio è a fuoco. Ci possiamo immaginare che a causa della differente posizione
    tra il nonio ed il crocefilo ci sia un certo $\epsilon$ di 
    differenza tra le due posizioni.
    \item $I_1^{V}$ diventa la
    posizione vera della sorgente virtuale rispetto alla scala graduata della guida.  
\end{itemize}
Dato che c'è questo offset rispetto alla formazione dell'immagine virtuale, 
da qualche parte dietro la lente ci sarà un $I_2^{V}$ dove c'è la formazione
dell'immagine vera. Quando l'immagine è a fuoco allora vuol dire che l'immagine vera si
sta generando sul crocefilo.  $I_2^{V}$ è dunque l'immagine reale, ossia la posizione
dell'immagine vera rispetto alla scala graduata, ovviamente in corrispondenza della
$I_2^{V}$ ci sarà la lettura del nonio che è chiamata come $I_2^{N}$, ossia la posizione
misurata dal nonio dell'oculare quando l'immagine è a fuoco.
Dunque si può dire che
\begin{gather*}
    s = I_2^{V} - I_1^{V}  
\end{gather*}
E, per l'offset di $\epsilon$:
\begin{gather*}
    I_1^{N} = I_1^{V} + \epsilon \\
    I_2^{N} = I_2^{V} + \epsilon    
\end{gather*}
Per ricavare le misure vere bisogna fare in modo da ottenere
una espressione per $\epsilon$ attraverso la distanza $s$:
\begin{gather*}
    s = (I_2^{N} - \epsilon) -  (I_1^{N} - \epsilon )
\end{gather*}
Ho dunque legato la distanza tra due posizioni che non conosco,
ma che sono traslate della medesima quantità; dunque, si può determinare
la distanza come
\begin{gather*}
    s = I_2^{N} - I_1^{N}  
\end{gather*}
Ossia semplicemente la distanza tra le due posizioni trovate dall'oculare. 
Bisogna ora trovare la distanza $l$ e determinare se effettivamente esiste
sempre questa distanza per la quale si ottiene la situazione
simmetrica nella soluzione di Gauss. Dato un certo $s$, c'è una condizione
matematica per la quale ci sono solamente certi valori di $s$ per cui
si osserva la formazione dell'immagine. Fissata la focale $f$ e dato $s$,
si può ora ricavare la distanza $l$ in funzione delle altre due:
\begin{gather*}
    l^{2} = s^{2} -4sf 
\end{gather*}
In questo modo si può ottenere che se $s^{2} - 4sf \geq 0$, allora $l^{2}$ ha una
soluzione. si può allora risolvere la disequazione (considerando che $s$ è positivo in quanto
è una distanza):
\begin{gather*}
    s(s - 4f) \geq 0 \ \Longrightarrow \ s \geq 4f
\end{gather*}
Dunque la formazione di questa situazione simmetrica è 
possibile solamente per questa condizione. Dire che $s = 4f$, 
equivale a porci nel caso limite in una configurazione $2f-2f$. 
Si pone il nonio e la lente misurando la prima posizione e poi la seconda in modo tale che
la sorgente virtuale e l'oculare siano fissati e si deve spostare le lenti
per far sì che si veda sempre a fuoco l'immagine


\subsection{Le lenti dell'esperienza}
Si possono determinare le posizioni vere e misurate
dal nonio delle lenti con un altro offset diverso
\begin{itemize}
    \item $L_1^{V}$: posizione vera della prima lente
    \item $L_1^{N}$: posizione dello zero del nonio della prima lente 
    \item $L_2^{V}$: posizione vera della seconda lente
    \item $L_2^{N}$: posizione dello zero del nonio della seconda lente
\end{itemize}
si può allora chiamare $\epsilon'$ l'incertezza tra le due misure
e dunque si può ottenere la relazione
\begin{gather*}
    L_2^{V} = L_2^{N} + \epsilon' \\
    L_1^{V} = L_2^{N} + \epsilon'    
\end{gather*}
Dunque si può ottenere la distanza tra le due lenti $l$ con 
la stessa procedura che si è utilizzato per determinare $s$:
\begin{gather*}
    l = L_2^{V} - L_1^{V} = L_1^{N} - L_1^{N}    
\end{gather*}


\section{Procedure operative in laboratorio}
I banchi ottici che si utilizzano hanno un piano di acciaio che sono 
pesanti e rigidi in modo che le vibrazioni non giungano ai componenti ottici.
Si inizia facendo delle procedure preliminari:
\begin{enumerate}
    \item Si deve indirizzare l'oculare verso una parete bianca e regolare la ghiera fino a che non si vede 
il crocefilo non sia a fuoco per ogni sperimentatore.
    \item Si procede dunque a rendere parallelo l'asse ottico 
    alla guida utilizzando un laser al termine della guida: il laser
    costituisce dunque l'asse ottico.
    \item Si regola dunque il laser in modo che il fascio di luce sia parallelo
    alla guida prendendo prima il montaggio con la lente di servizio e posizionarla 
    in modo che la sorgente laser possa passare attraverso la lente sia quando è vicino che
    quando è lontano.
    \item Si regola ora la posizione della lente incognita secondo la stessa procedura. 
    \item Ci si assicura ora che il fascio laser colpisca la sorgente vera spostando il 
    montaggio della sorgente con la scala graduata
\end{enumerate}
Adesso si procede con la presa delle misure: si procede montando il filtro e 
la lente e l'oculare e aggiustare la posizione dell'oculare (misurandola) 
in modo tale che si veda chiaramente l'immagine. L'incertezza è dovuta all'occhio dello sperimentatore 
e anche nel togliere e riposizionare l'oculare riposizionando l'oculare (metodo più preciso), oppure
trovando l'intervallo di posizioni per le quali si vede a fuoco l'immagine (metodo più
veloce ma meno preciso). Si sfrutta ora il metodo di Gauss posizionando 
l'oculare distante quattro volte la lunghezza focale trovata per la lente incognita. 
Si posiziona dunque la lente nella posizione $L_1$ in modo tale che si formi l'immagine sull'oculare
e misuro con il nonio $L_1$ con una delle due tecniche. Si misura poi $L_2$
con uno dei due metodi di prima. Infine si ripetono tutte le misure per tutti gli 
sperimentatori e si verifica la consistenza delle misure per verificare la legge del 
costruttore di lenti. 


\subsection{Presa delle misure}
Ci sono due metodi di misura: si può riposizionare le lenti tutte le volte (meno errore) 
oppure cercare un intervallo di posizioni in cui si vede a fuoco (più errore). 
Scegliere la posizione per cui un osservatore vede a fuoco 
l'immagine e leggere il nonio (per cui ci sarà l'errore di sens del nonio);
si sposta dunque l'oculare e ripeto questa misura un certo numero di volte
(almeno 5 misure) di $I_1$. Adesso si può ottenere media e 
scarto massimo per la posizione 
\begin{gather*}
    I_1 = \overline{I_1} \pm \Delta I_1  
\end{gather*}
A questo punto si può posizionare l'oculare ad una posizione fissata rispetto 
alla guida (la sua incertezza è solo la sens del nonio). Si sceglie 
quindi $I_2$ in modo tale che $s \geq 4f$ e si ottiene la posizione 
\begin{gather*}
    I_2 = \overline{I_2} \pm \Delta I_2
\end{gather*}
Dove la posizione è misurata una sola volta e l'incertezza è proprio
la sensibilità del nonio. 
Adesso si misura $L_1$ ossia la posizione per la quale si vede a fuoco l'immagine in $I_2$
$5/6/7$ volte in modo che si ottenga
\begin{gather*}
    L_1 = \overline{L_1} \pm \Delta L_1 
\end{gather*}
E la stessa cosa faccio per la posizione della seconda lente
\begin{gather*}
    L_2 = \overline{L_2} \pm \Delta L_2 
\end{gather*}
Si determina ora $s$ come
\begin{gather*}
    \overline{s} = \overline{I_2} - \overline{I_1} \qquad \Delta s = \Delta I_1 + \Delta I_2 \\
    \ \Longrightarrow \ s = \overline{s} + \Delta s   
\end{gather*}
E lo stesso si fa per $l$:
\begin{gather*}
    \overline{l} = \overline{L_2} - \overline{L_1} \qquad \Delta l = \Delta L_1 + \Delta L_2  \\
    \ \Longrightarrow \ l = \overline{l} + \Delta l 
\end{gather*}
Adesso si può riprendere la formulazione che mi permette di trovare $f$:
\begin{gather*}
    f = \frac{s^{2} - l^{2}  }{4s}
\end{gather*}
Dunque per determinare $\overline{f}$ si può determinare
i valori $\overline{l}$ e $\overline{s}$ e sostituirli mentre
per l'incertezza si fa la propagazione degli errori:
\begin{gather*}
    \Delta f = \left|\frac{\partial f}{\partial s}\right| \Delta s + \left| \frac{\partial f}{\partial l}  \right|\Delta l  
\end{gather*} 

\begin{wrapfigure}{r}{0.4\textwidth}
    \centering
    \caption{Queste misure della focale sono consistenti tra di loro}
    \begin{tikzpicture}
        \draw[->](0, 0) -- (4, 0) node[at end, below] {sperimentatore};
        \draw[->](0, 0) -- (0, 4) node[at end, left] {$f$};
        \draw(0, 2) -- (4, 2);
        \draw[|-|](0.8, 1.65) -- (0.8, 2.2);
        \draw[|-|](1.6, 1.78) -- (1.6, 2.4);
        \draw[|-|](2.4, 1.5) -- (2.4, 2.1);
        \draw[|-|](3.2, 1.9) -- (3.2, 2.6);
    \end{tikzpicture}    
\end{wrapfigure}
E dunque si ottiene
\begin{gather*}
    \Delta f = \left| \frac{1}{4} + \frac{l^{2} }{4s^{2} } \right| \Delta s + \left| \frac{l}{2s} \right|\Delta l 
\end{gather*}
Bisogna anche assicurarsi che si riesca sempre a vedere l'immagine anche spostando
indietro l'oculare perché in questo modo si è più suscettibili al cattivo allineamento
dei componenti ottici: si verifica che l'asse ottico sia ben allineato prima
di procedere con la determinazione delle altre misure di $f_i$.
Ogni sperimentatore può ottenere il valore di $f$ e della sua incertezza; si verifica dunque
che tutte le misure di $f$ siano consistenti tra di loro entro le rispettive 
barre di errore.
si può prendere la media pesata e confrontarla con le singole 
incertezze (dovrebbe essere minore).



\chapter{Polarizzazione}
\section{Apparato sperimentale}
\begin{gather*}
    \begin{tikzpicture}[scale=1.2]
        \filldraw[cyan](0, 0) circle (0.1) node[anchor = north] {\tiny sorgente};
        \draw(0.75, 0) circle (0.1);
        \draw[->](0.75, -0.1) -- (0.75, -1) node[at end, right] {$\hat{z} $};
        \filldraw(0.75, 0) circle (1pt) node[anchor = south] {$\hat{y} $};
        \draw(1.5, 0) circle (0.1) node[anchor = north] {$\vv{E_{0y}} $};
        \draw(1.5, 0) circle (1pt);
        \draw[->](1.5, 0.1) -- (1.5, 1) node[at end, right] {$\vv{E_{0z}}$};
        \draw (2.5, -0.5) rectangle (3.5, 0.5);
        \draw(2.5, 0.5) -- (3.5, -0.5);
        \node at (3, 0.6) {\tiny cubo pulizia};
        \draw(4.5, 0) circle (0.1);
        \draw[->](4.5, 0.1) -- (4.5, 1) node[at end, right] {$\vv{E_{0z}}$};
        \draw(5.5, -0.5) rectangle (6, 0.5);
        \node[align = center] at (5.75, 0.6) {\tiny lamina di ritardo};
        \draw[->](7, 0.1) -- (7, 1) node[at end, right] {$\vv{E_{0z}}$};
        \filldraw(7, 0) circle (1pt) node[anchor = north] {$\vv{E_y}$};
        \draw(7, 0) circle (0.1);
        \draw(8, -0.5) rectangle (9, 0.5);
        \node at (8.5, 0.6) {\tiny cubo analisi};
        \draw(8, 0.5) -- (9, -0.5);
        \draw[->](8.5, 0) -- (8.5, -1) node[at end, right] {$\vv{E_y}$};
        \draw[->](8.5, 0) -- (9.25, 0);
        \draw[->](9.5, 0) -- (9.5, 1) node[at end, right] {$\vv{E_z}$}; 
        \draw(8, -1.5) rectangle (9, -2);
        \node at (8.5, -1.75) {\tiny rilevatore};
        \draw[dashed](8.5, -1.5) -- (8.5, 0);
        \draw[dashed](0, 0) -- (10, 0);
        \draw(10, -0.25) rectangle (11, 0.25) node[midway] {\tiny rilevatore};
    \end{tikzpicture}
\end{gather*}
Si studiano le leggi di trasformazione della polarizzazione di una onda
polarizzata linearmente che incide su di una lamina di ritardo con angolo
generico tra polarizzazione incidente e gli assi della lamina. $z$ e $y$ sono
gli assi del cubo polarizzatore mentre gli assi $a$ e $b$ sono gli assi della lamina
rispettivamente dell'asse lento e di quello veloce.  Il campo magnetico uscente
dal cubo polarizzatore è dato da
\begin{gather*}
    \vv{E_{tot}} = \left(E_{0z} \cos^{2}\theta(\psi + \delta \phi) + E_{0z}\sin^{2}\theta \cos\psi\right)\hat{z}  + \left(E_{0z}\cos\theta \sin\theta\cos(\psi + \delta \phi) - E_{0z}\sin\theta \cos\psi\cos\theta \right)\hat{y} 
\end{gather*}
Il fascio di luce passa dopo attraverso una lamina di ritardo e dunque il campo elettrico sarà modificato ed
è possibile esprimerlo attraverso gli assi fast e slow come:
\begin{gather*}
    \vv{E_{out}} = E_{0z}\cos\theta\cos(\psi + \delta \phi) \hat{a} - E_{0z}\sin\theta\cos\psi \hat{b}    
\end{gather*}
Dove
\begin{gather*}
    \hat{a} = \cos\theta \hat{z} + \sin\theta \hat{y} \qquad \hat{b} = -\sin\theta \hat{z} + \cos\theta \hat{y}      
\end{gather*}

\subsection{Lamina $\frac{\lambda}{2}$}
La prima lamina è una lamina $\frac{\lambda}{2}$ e il suo $\delta \phi = \pi$, il campo
elettrico totale uscente dalla lamina può essere espresso come
\begin{gather*}
    \vv{E_{out}} = E_{0z}\left(-\cos^{2}\theta \cos\psi + \sin^{2}\theta\cos\psi\right) \hat{z} - 2E_{0z}\sin\theta\cos\theta\cos\psi\hat{y} = \\
    -E_{0z}\cos\psi\left(\cos 2\theta\right) \hat{z} - E_{0z}\sin 2\theta \cos \psi \hat{y}  
\end{gather*}
Allora l'intensità media rispetto all'asse $z$ in uscita dalla lamina di ritardo sarà
\begin{align}
    I_z = c\epsilon_0 \left< E_z^{2} \right> = c\epsilon_0 \left< E_{0z}^{2} \cos^{2}\psi\right> \cos^{2}2\theta  
\end{align}
Dove il termine $ c\epsilon_0 \left< E_{0z}^{2} \cos^{2}\psi\right>$ indica l'intensità 
luminosa iniziale prima di attraversare la lamina di ritardo.
Si può esprimere ora l'intensità luminosa della luce rispetto 
all'asse $y$ e tracciarne il grafico (per $I_y$ e $I_z$):
\begin{gather*}
    I_y = I_0\sin^{2}2\theta  \\
    \begin{tikzpicture}[domain=0:3.14]
        \draw[->](0, 0) -- (4, 0) node[at end, below] {$\theta$};
        \draw[->](0, 0) -- (0, 4) node[at end, left] {$I_y, I_z$};
        \draw[cyan, samples = 50] plot (\x, {3 * cos(2 * \x r) * cos(2 * \x r)});
        \draw[red, samples = 50] plot (\x, {3 * sin(2 * \x r) * sin(2 * \x r)});
        \filldraw(3.14, 0) node[anchor = north] {$\pi$};
        \filldraw(3.14 / 4, 0) node[anchor = north] {$\frac{\pi}{4}$};
        \filldraw(3.14 / 2, 0) node[anchor = north] {$\frac{\pi}{2}$};
        \filldraw(3.14 * 3 / 4, 0) node[anchor = north] {$\frac{3}{4}\pi$};
    \end{tikzpicture}  
\end{gather*}
Dove il tratto rosso corrisponde all'intensità sull'asse $y$ ed il tratto ciano rappresenta 
l'intensità sull'asse $z$. La lamina ha un asse slow diretto lungo $\hat{a}$ diretto lungo l'asse
$\hat{z}$; dunque rimane una polarizzazione lineare in quanto la lamina ritarda
solamente l'oscillazione dell'onda. la prima lamina con $\theta = 0$ non cambia la
polarizzazione mentre una lamina con angolo $\theta \neq 0$ lo fa.

\subsection{Lamina $\frac{\lambda}{4}$}
Mettendo la lamina a $\frac{\pi}{4}$ rispetto alla posizione iniziale e non
ho più luce polarizzata nel verso $\hat{z}$ ma sarà tutta polarizzata
verso $\hat{y}$ dunque $I_y$ ha un minimo. $\delta \phi = \frac{\pi}{2}$. 
\begin{gather*}
    \begin{tikzpicture}
        \draw[->](0, 0) -- (2, 0) node[at end, below] {$z$};
        \draw[->](0, 0) -- (0, 2) node[at end, left] {$z$};
        \draw(-1, -1) -- (1, 1) node[at end, right] {slow};
        \draw(-1, 1) -- (1, -1) node[at start, above] {fast};
    \end{tikzpicture}
\end{gather*}
Il campo elettrico in uscita dalla lamina ha la seguente espressione (derivata dall'espressione all'inizio della sezione):
\begin{gather*}
    \vv{E_{out}} = E_{0z} \left(\cos^{2}\theta(-\sin\psi) + \sin^{2}\theta \cos\psi\right)\hat{z} + E_{0z}\left(-\sin\psi\cos\theta\sin\theta - \cos\psi \sin\theta \cos\theta\right) \hat{y}   
\end{gather*}
si può ottenere i moduli dei campi elettrici nelle due direzioni elevando al quadrato
e poi, moltiplicando per $\epsilon_0$ e $c$ e mediando nel tempo si ha che
\begin{gather*}
    I_z = c\epsilon_0 \left< E_z^{2} \right> = c\epsilon_0 \left< E_0^{2} (sin^{2}\psi \cos^{4}\theta + \sin^{4} \theta \cos^{2}\psi) \right> = I_0\left(1 - \frac{1}{2}\sin^{2}(2\theta)\right) \\
    I_y 0 \frac{I_0}{2}\sin^{2}(2\theta)
\end{gather*} 
La lamina $\frac{\lambda}{4}$ spancia l'ellisse che descrive il luogo dei punti
che attraversano il campo elettrico.
\begin{gather*}
    \begin{tikzpicture}[samples=50, domain=0:3.14]
        \draw[->](0, 0) -- (4, 0) node[at end, below] {$\theta$};
        \draw[->](0, 0) -- (0, 4) node[at end, left] {$I_y, I_z$};
        \draw[red] plot (\x, {(3 / 2) * sin(2 * \x r) * sin (2 * \x r)});
        \draw[cyan] plot (\x, {3 - (3 / 2) * sin(2 * \x r) * sin(2 * \x r)});
    \end{tikzpicture}
\end{gather*}
Di conseguenza, lungo $y$ non ho mai l'intensità riflessa dal mio cubo, al minimo
ne ho la metà. Se la lamina introduce shift di fase lungo
$z$ non cambia la mia polarizzazione quindi lungo $y$ ho zero luce e
dunque avrò una polarizzazione ellittica.
\begin{gather*}
    \begin{tikzpicture}
        \draw[->](0, 0) -- (1, 0) node[at end, below] {$z$};
        \draw[->](0, 0) -- (0, 2) node[at end, left] {$y$};
        \draw[cyan](0, 0) ellipse (1.25 and 0.5);
    \end{tikzpicture}
\end{gather*}

\section{Scopi e finalità}
\subsection{Il fit sinusoidale}
Nella polarizzazione si deve validare le leggi della polarizzazione
per le lenti $\frac{\lambda}{4}$ e per le lenti $\frac{\lambda}{2}$. 
Si deve realizzare un fit (non lineare) di una funzione sinusoidale:
per realizzarlo dobbiamo fare le seguenti assunzioni (così come per i fit lineari):
\begin{itemize}
    \item l'incertezza $\sigma$ sulla variabile aleatoria $x$ sia trascurabile: questo vuol dire che
    si ha uno strumento che misura ole $x$ con una precisione di diversi ordini di grandezza superiore delle $y$. 
    \item La funzione $g$, ossia la funzione del fit, abbia una distribuzione
    gaussiana e con i vari $\sigma_{y_i}$ costanti per tutte
    le misure $ \ \Longrightarrow \ \sigma_{y_i} \approx \sigma_y$. 
    \item Devo minimizzare il $\chi^{2}$.  
\end{itemize} 

\begin{wrapfigure}{r}{0.4\textwidth}
    \centering
    \caption{La funzione generica}
    \begin{tikzpicture}
        \draw[->](0, 0) -- (4, 0);
        \draw[->](0, 0) -- (0, 4);
        \draw(0.5, 1) ..  controls (1, 1.9) and (1.2, 1.9) .. (1.7, 1.3);
        \draw(1.7, 1.3) .. controls (2.2, 0.7) and (2.7, 1.8) .. (3.5, 1);
        \draw(1, 0.1) -- (1, -0.1) node[at end, below] {$x_1$};
        \draw(2, 0.1) -- (2, -0.1) node[at end, below] {$x_2$};
        \draw(3, 0.1) -- (3, -0.1) node[at end, below] {$x_3$};
    \end{tikzpicture}    
\end{wrapfigure}
Nel nostro caso non si può applicare le regole del fit lineare ma si può supporre di avere una
funzione generica ottenuta con dei certi valori sperimentali
$(x_i, y_i)$ dove i dati sono legati tra di loro mediante una determinata legge
\begin{gather*}
    f(a, b, c, \dots)
\end{gather*}
Tuttavia questa funzione non è completamente arbitraria ma è del tipo
\begin{gather*}
    f = a\cos^{2}(bx + \phi) + \text{offset} 
\end{gather*}
Questa funzione sarà ovviamente una funzione dell'angolo $x$ (ossia la rotazione
della lamina) e descrive l'intensità luminosa del cubo in analisi
in trasmissione e riflessione. Dunque si può determinare il $\chi^{2}$ come
\begin{gather*}
    \chi^{2} = \frac{\sum (y_i - f(a, b, \phi, \text{offset}, x))^{2} }{\sigma_y^{2} } 
\end{gather*}  
si può ottenere la minimizzazione del $\chi^{2}$ mediante l'utilizzo di un calcolatore 
e delle $\overline{a}, \overline{b}, \overline{c}, \overline{d}$. Inoltre è capace di ottenere
anche le incertezze associate alle singole variabili. 

\subsection{Le lamine $\frac{\lambda}{4}$ e $\frac{\lambda}{2}$ nell'esperienza}
\begin{wrapfigure}{r}{0.4\textwidth}
    \centering
    \caption{La lamina $\frac{\lambda}{2}$}
    \begin{tikzpicture}
        \draw[red, thick, ->](0, 0) -- (2, 0) node[at end, right] {asse fast};
        \draw[cyan, thick, ->](0, 0) -- (0, 2) node[at end, below] {asse slow};
        \draw[thick, ->](0, 0) -- (2, 1) node[at end, right] {in};
        \draw[thick, ->](0, 0) -- (-2, 1) node[at end, above] {out};
        \draw(0, 0.5) arc (90:30:0.5) node[midway, above] {$\theta$};
        \draw(0, 0.5) arc (90: 150:0.5) node[midway, above] {$\theta$};
    \end{tikzpicture}    
\end{wrapfigure}
Le lamine possono riflettere il campo elettrico lungo le due direzioni
\begin{gather*}
    E_{z} = E_{z0}\cos(kx - \omega t + \phi_z) \\
    E_{y} = E_{y0}\cos(kx - \omega t + \phi_y)
\end{gather*}
Con la lamina $\frac{\lambda}{4}$ si può cambiare la spanciatura dell'ellisse che 
descrive la polarizzazione ellittica generica senza modificare la direzione degli assi:
per quello si utilizza la $\frac{\lambda}{2}$. 

\subsection{Setup dell'apparato}
Si ha un tavolo con una breadboard: ossia un piano di 
alluminio con dei fori che permette di fissare gli 
strumenti ottici. La polarizzazione in uscita dal laser non la
conosciamo e potrebbe aver qualsiasi tipo di polarizzazione 
ma si conosce la lunghezza d'onda che è di $532 \ nm$. 
\begin{gather*}
        \begin{tikzpicture}
        \draw[->](-1, 0) -- (10, 0) node[at end, below] {$x$};
        \draw(0, -0.25) rectangle (1, 0.25);
        \draw(-1, 0) circle (0.2);
        \filldraw(-1, 0) circle (1pt) node[anchor = south] {$y$}; 
        \draw[->](-1, 0) -- (-1, -2) node[at end, right] {$z$};
        \draw(2, 0) circle (0.2);
        \draw[->](2, 0) -- (2, 1) node[at end, right] {$\vv{E_{\parallel}}$};
        \draw(1.9, 0.1) -- (2.1, -0.1) node[at end, below] {$\vv{E_\perp}$ };
        \draw(1.9, -0.1) -- (2.1, 0.1);
        \draw(3, -0.75) rectangle (3.3, 0.75);
        \draw (4, -0.75) rectangle  (5.5, 0.75);
        \draw[->](6, 0) -- (6, -1) node[at end, below] {$\vv{E_\parallel}$ };
        \draw(6.5, -0.75) rectangle (6.8, 0.75); 
        \draw[->](7.5, 0) -- (7.5, -1) node[at end, right] {$\vv{E_{\parallel}}$};
        \draw(7.4, 0.1) -- (7.6, -0.1) node[at start, above] {$\vv{E_\perp}$ };
        \draw(7.4, -0.1) -- (7.6, 0.1);
        \draw(7.5, 0) circle (0.2);
        \draw(8, -0.75) rectangle (9.5, 0.75);
        \draw(8.75, 0) -- (8.75, -2);
        \draw(10, -0.5) rectangle (11, 0.5);
    \end{tikzpicture}  
\end{gather*}
SI fa passare il campo 
attraverso un filtro variabile che permette di cambiare (variando il
suo angolo di rotazione) l'intensità luminosa che passa. La luce passa poi
attraverso un cubo di pulizia, che permette di filtrare il campo perpendicolare 
e di far passare solo il campo $\vv{E_\parallel}$ che risulta avere
verso contrario a quello di entrata.  \\
Si utilizza un supporto con goniometro che permette di ruotare la lamina; dato che
il laser passa dentro la lamina, si può fare in modo che l'asse slow e quello fast
possano ruotare rispetto alla scala goniometrica. Questo componente ottico sta dopo il cubo
di pulizia e permette di cambiare la polarizzazione in base all'angolo di rotazione $\theta$.
Dopo si pone un altro cubo di pulizia che permette di filtrare nuovamente  il
campo perpendicolare (che prende il nome di cubo di analisi) e porre i due rilevatori. 

\section{Funzioni del fit}
\subsection{Calibrazione dei laser}
Si presentano vari problemi nella raccolta dei dati sperimentali:
\begin{itemize}
    \item Il rilevatore non è perfetto e dunque non riesce a misurare
    perfettamente tutta l'intensità;
    \item Il rilevatore ha una sensibilità diversa da quella dell'altro 
    rilevatore
    \item I rilevatori sono sporchi oppure tarati male
    \item L'allineamento dell'apparato ottico non è perfetto 
\end{itemize}
E' possibile ovviare alla calibrazione dei rilevatori sfruttando la 
lamina $\frac{\lambda}{2}$: dopo aver posizionato i rilevatori si ruota la
lamina $\frac{\lambda}{2}$ e si misura il valore massimo che si ottiene in ognuno
dei rilevatori. 

\subsection{Lamina $\frac{\lambda}{2}$}
Si può ora descrivere le leggi di Malus per i vari assi
\begin{gather*}
    I_\parallel(\theta) = I_0 \cos^{2}(2\theta) \\
    I_\perp(\theta) = I_0 \sin^{2}(2\theta)  
\end{gather*}
SI ottiene allora le seguenti condizioni
\begin{gather*}
    \begin{tikzpicture}
        \draw[red, ->](0, 0) -- (2, 0) node[at end, below] {$E$};
        \draw[cyan,->](0, 0) -- (0, 2);
        \node at (1.5, 1.5) {$\theta = 0$};
        \draw(5, 0) -- (7, 0) node[at end, below] {$E$};
        \draw[cyan, ->](5, 0) -- (5, 2);
        \draw(5, 0) -- (7, 2);
        \draw(5, 0) -- (4, 2); 
        \node at (7, 1.5) {$\theta = \frac{\pi}{4}$};
    \end{tikzpicture}
\end{gather*}
Facendo ora ruotare le polarizzazioni, si ottengono tanti dati sperimentali:
in funzione dell'angolo si può allora disegnare il grafico della legge di Malus
(circa 20 punti per gli angoli da zero a $360$° equispaziati, ossia ogni $\sim 20$°)
Mentre si collezionano i dati per l'intensità trasmessa, si collezionano anche i dati
per l'intensità riflessa e si disegnerà anche il grafico attraverso quei punti.
La funzione del fit sarà ora data da
\begin{gather*}
    f(a, b, \phi, \text{offset}, x) = a\cos^{2}(bx + \phi) + \text{offset} 
\end{gather*}
Supponendo ora di avere questa funzione da farne il fit, dividendo $I_\parallel(\theta)$
e $I_\perp(\theta)$ per i loro valori massimi ci si aspetta che l'ampiezza
dell'oscillazione del coseno e del seno sia $\overline{a} \approx 1$. si può invece dire che $\overline{b} \approx 2$ in quanto 
nella legge di Malus deve risultare $2\theta$ e dove $\text{offset} = 0$. Tra la fase del coseno e del seno ci si
aspetta una differenza di circa $\approx \frac{\pi}{2}$, e dunque, arbitrariamente, si può dire che $\overline{\phi} \approx 0$.  
Riassumendo
\begin{gather*}
    \overline{a} \approx 1 \qquad \overline{b} \approx 2 \qquad \overline{\phi} \approx 0 \qquad \text{offset} \approx 0  
\end{gather*}
Tuttavia in condizioni non ideali l'offset del laser non sarà zero ma 
sarà un valore molto piccolo da dover sottrarre a tutti i valori
trovati in laboratorio. Si può ora vedere che
\begin{gather*}
    \cos^{2}(2\theta) = 1 - \sin^{2}(2\theta) = 1 - \cos^{2} (2\theta + \frac{pi}{2})  
\end{gather*}
Data questa legge, si osserva che si trova un altro set di valori con 
\begin{gather*}
    \overline{a} \approx -1 \qquad \overline{b} \approx 2 \qquad \overline{\phi} \approx = \frac{\pi}{2}  \qquad \text{offset} = 1 
\end{gather*}
In questo caso si vede che $\chi^{2}$ ha due minimi e permette di ottenere due
funzioni dal programma di fit.  

\subsection{La lamina $\frac{\lambda}{4}$}
Con la lamina $\frac{\lambda}{4}$ si per la legge di Malus che
\begin{gather*}
    I_\parallel = I_0\left(1 - \frac{1}{2}\sin^{2}(2\theta) \right) \\
    I_\perp =   \frac{I_0}{2}\left(\sin^{2}(2\theta) \right)
\end{gather*}
Dato che questa lamina mi permette di spanciare l'ellisse, se ruotassi la lamina
di un certo angolo $\theta = \frac{\pi}{2}$, il ritardo di fase è lungo una direzione dove il campo elettrico non
oscilla neanche, allora la polarizzazione rimane tutta $\vv{E_\parallel}$ e dunque
tutta la potenza è trasmessa.
\begin{gather*}
    \begin{tikzpicture}
        \draw[red, ->](0, 0) -- (2, 0) node[at end, below] {E};
        \draw[cyan,->](0, 0) -- (0, 2);
        \node at (1.5, 1.5) {$\theta = 0$};
        \draw[->](4, 0) -- (6, 0);
    \end{tikzpicture}
\end{gather*} 

\begin{wrapfigure}{r}{0.4\textwidth}
    \centering
    \caption{}
    \begin{tikzpicture}[domain=0:4]
        \draw[->](0, 0) -- (4, 0) node[right] {$\theta$};
        \draw[->](0, 0) -- (0, 2) node[right] {$I_0$};
        \draw[samples = 100, red] plot (\x, {sin(2 *\x r) * sin(2 *\x r)}) node[left] {$I_\perp$};
        \draw[samples = 100, cyan] plot (\x, {2 - sin(2 * \x r) * sin(2 * \x r)}) node[left] {$I_\parallel$};
    \end{tikzpicture}    
\end{wrapfigure}
L'ellisse si spancia solamente quando $\theta = \frac{\pi}{4}$. Allora, la polarizzazione
in uscita diventa circolare.
\begin{gather*}
    \theta = \frac{\pi}{4} \qquad I_\parallel = \frac{I_0}{2} \qquad I_\perp = \frac{I_0}{2}
\end{gather*}
Dato che la lamina $\frac{\lambda}{2}$ permette di ruotare il campo di un certo angolo
$\theta$ e, quando la luce passa nuovamente  all'interno della lamina viene riportata alla polarizzazione 
iniziale, tutta la luce che torna sul cubo analisi sarà totalmente riflessa. Invece, una $\frac{\lambda}{4}$
attraversata due volte si comporta come una $\frac{\lambda}{2}$: l'effetto complessivo 
del doppio passaggio è una riflessione rispetto all'asse slow della lamina. 

\section{Procedure operative}
Si inizia bloccando sempre i componenti ottici attraverso delle staffe e bulloni 
che vengono strinti attraverso una chiave a brugola. Le procedure iniziali prima della
raccolta dati sono le seguenti:
\begin{enumerate}
    \item Si regola l'angolo di inclinazione sul cubo polarizzatore attraverso i due cubi polarizzatori: il cubo di analisi è 
utilizzato davanti al cubo di pulizia in modo tale che il fascio di luce colpisca il 
centro della fascia del cubo. In buona approssimazione, il fascio uscente dal cubo è 
polarizzato linearmente: il fascio riflesso deve essere minimizzato regolando l'angolo del cubo 
di analisi. Per fare questa regolazione si fa alla massima potenza del laser in modo da ottenere una polarizzazione 
parallela al tavolo.
    \item a questo punto si può mettere il cubo di analisi e di pulizia 
    nella loro posizione finale in modo che tra di loro si possa inserire la lamina. 
    \item Si regola ora l'angolo del cubo di analisi minimizzando la potenza che incide sul 
    foglio per la polarizzazione perpendicolare (qui, dato che la polarizzazione 
    è lineare e parallela, non trasmette niente) e si fissa con il bullone.
    \item Si inserisce la lamina in modo che sia centrata sul fascio del laser e in modo tale
    che sia perpendicolare al fascio incidente del laser (attraverso un pezzo di carta).
\end{enumerate} 
Il rilevatore è un disco di silicio che sente la luce in ingresso e la legge come voltaggio ed il fotodiodo
al silicio è sensibile a tutte le lunghezze d'onda, dunque vi è posto un filtro che lascia passare
solamente un intervallo di lunghezze d'onda in modo tale che sia vicino alla lunghezza d'onda. I rilevatori 
vanno posizionati senza il filtro e successivamente si controlla che il rilevatore sia posizionato correttamente.
Il funzionamento dei fotodiodi è assicurato essere lineare e inoltre direttamente proporzionale all'intensità 
dell'onda elettromagnetica se e solo se la differenza di potenziale è $ < 10 V$.

\subsection{Presa delle misure}
Adesso si possono prendere le misure sparando il laser:
\begin{itemize}
    \item Le misure si prendono ruotando la lamina di $\approx 18^{\circ}$ 
    per ogni misura per un totale di 20 misure. 
    \item Si misura la differenza di potenziale tra i due rilevatori 
    \item Le misure si fanno velocemente in quanto il laser
    tende a variare intensità nel tempo. 
\end{itemize}

\subsection{Analisi dati}
Una volta presi i dati si procede con l'analisi dati attraverso il programma scritto in Mathematica sul
computer del laboratorio che fornisce il grafico del fit e dei residui con 
i rispettivi valori veri ed incertezze. SI hanno due set di parametri con 
errori in quanto si hanno due funzioni e si confronta la coppia di valori 
alle funzioni teoriche per poter verificare le leggi. Si ha anche il programma per 
il calcolo della variazione di intensità delle lamine $\frac{\lambda}{2}$ e $\frac{\lambda}{4}$
in modo da evidenziare differenze tra i dati sperimentali dai modelli teorici. 
c

\chapter{Diffrazione}
\section{Scopo dell'Esperienza}
Gli scopi dell'esperienza sono i seguenti:
\begin{itemize}
    \item Verifica delle legge di una onda piana da una fenditura 
    nel limite di campo lontano (legge di Fraunhofer);
    \begin{gather*}
        I(\theta) = I_0 sinc^{2}\left(\frac{kD}{2}\sin \theta\right)
    \end{gather*}
    \item Verifica della legge di rifrazione di una onda piana incidente su di un reticolo di diffrazione.
\end{itemize}

\section{L'apparato sperimentale}
\begin{gather*}
    \begin{tikzpicture}
        \draw(0, -0.25) rectangle (1, 0.25);
        \draw(1, 0) -- (3, 0);
        \draw(3, -0.5) rectangle (3.5, 0.5);
        \draw(3.5, 0) -- (10, 0);
        \draw(4.75, 0.2) rectangle (5.25, 1);
        \draw(4.75, -0.2) rectangle (5.25, -1);
        \filldraw(5, 0) circle (1pt);
        \filldraw(10, 0) circle (1pt) node[anchor = west] {$O$};
        \draw[cyan](5, 0) -- (10, 1);
        \draw(7, 0) arc (0:12:2) node[midway, right] {$\theta$};
        \draw[->](10, -1) -- (10, 1) node[at end, right] {$y'$};
        \draw[|-|](5.3, -1) -- (9.95, -1) node[midway, above] {$x$};
        \node at(5, -1.3) {\tiny fenditura};
        \node at(3.25, 0.75) {\tiny filtro};
        \node at(0.5, 0.5) {\tiny laser};
        \node at (11, 0) {\tiny rilevatore};
    \end{tikzpicture}
\end{gather*}
Il laser che si utilizza è un laser a $668,66 \pm 0.01 \ nm$ che punta su
di un filtro in assorbimento variabile. Dopo il filtro c'è una
fenditura larga circa $30 - 40 \ \mu m$ che è posta ad una distanza
$x$ rispetto al rilevatore. Ogni posizione $y'$ sul rilevatore corrisponde ad un certo
angolo $\theta$ per il seno cardinale. Nel limite di Fraunhofer, ossia quando
$x >> D$, in più ci si mette nel limite in cui $x >> y'$ (che  è possibile
in quanto il rilevatore ha una dimensione di circa un centimetro). 
Dunque
\begin{gather*}
    \theta << 1 \ \Longrightarrow \ \sin \theta \approx \tan \theta \approx \theta
\end{gather*}
Il rilevatore è composto da $1032 \times 1032$ pixel ognuno di 
una dimensione di $d \approx 5 \mu m$. Ogni raggio di luce eccita gli elettroni
dei led e dunque il rilevatore restituisce la misura dell'intensità incidente in base alla quantità di elettroni
eccitati. Dato che ogni pixel è in grado di restituire un valore di intensità
al massimo di $2^{8}$, bisogna regolare il filtro in modo tale che
nessun led possa saturare; se i pixel saturassero, non riuscirei a determinare
se l'intensità è effettivamente massima oppure il led arriva a saturazione
troppo presto. \\
Si utilizza un programma in Mathematica per poter fittare il seno cardinale 
in funzione degli indici dei pixel. La coordinata $y'$ di ogni pixel dipende
dalla larghezza del pixel e dal suo indice. Dunque l'intensità
diventerà la seguente funzione:
\begin{gather*}
    I(y') = I_0 sinc^{2}\left(\frac{kD}{2}\sin\theta\right) = I_0 sinc^{2}\left(\frac{kD}{2} \frac{d}{x}y\right)
\end{gather*}
Si introduce inoltre un parametro che mi possa centrare il seno cardinale
proprio sull'asse che congiunge il centro del rilevatore e la sorgente del 
fascio di luce:
\begin{gather*}
    I(y') = I_0 sinc^{2}\left(\frac{kD}{2}\sin\theta\right) = I_0 sinc^{2}\left(\frac{kD}{2} \frac{d}{x}(y - y_0)\right)
\end{gather*}
Dal fit non lineare del profilo di intensità misurato con la CCD ricavo
$B \pm \Delta B$, dove $B = \frac{kD}{2} \frac{d}{x}$. L'idea è ora
confrontare tale misura con il valore teorico nel quale sostituisco i valori 
di $k$ conoscendo $\lambda$ e le loro incertezze. I dati da trovare sono dunque:
\begin{itemize}
    \item $k \pm \Delta k$: si ricava conoscendo $\lambda  \pm \Delta \lambda$;
    \item $D \pm \Delta D$: $\left\{\begin{array}{l}
            40 \pm 2 \ \mu m \\
            30 \pm 2 \ \mu m 
        \end{array}\right.$ a seconda della fenditura in dotazione;
    \item $d = 5.20 \pm 0.01 \ \mu m$;
    \item $x \pm \Delta x$: si misura con il calibro.  
\end{itemize} 
Infine confronto i $B_i \pm \Delta B_i$ sperimentali ottenuti
e li confronto con quello teorico verificando che siano consistenti per poter dimostrare la legge. 
I valori $B_i$ sono tutti diversi ognuno per ogni riga di pixel poiché la fenditura
non è perfetta, inoltre ogni valore dipende dalla posizione rispetto alla fenditura. 
Inoltre, se vi sono $n$ fenditure con larghezza $D$, il seno cardinale
presenterà anche un termine in funzione delle $n$ fenditure spaziate
di $a$:
\begin{gather*}
    I(\theta) = I_0 sinc^{2}\left(\frac{kD}{2}\sin\theta\right) \frac{\sin^{2} \left(\frac{nak}{2}\sin\theta\right)}{n^{2}\sin^{2}\left(\frac{ka}{2}\sin\theta\right)}
\end{gather*}
E dunque
\begin{gather*}
    \frac{ka}{2}\sin\theta = m \pi \ \Longrightarrow \ \sin\theta = \frac{2m\pi}{ka} \ \Longrightarrow \ m\frac{\lambda}{a}
\end{gather*}
In corrispondenza di multipli interi di  questa quantità si osservano
i picchi dell'intensità.

\begin{wrapfigure}{r}{0.4\textwidth}
    \centering
    \caption{Il singolo led}
    \begin{tikzpicture}[scale = 1.5]
        \draw(0, 0) rectangle (2, 2);
        \draw(0.4, 0.4) rectangle (1.6, 1.6);
        \node[align = center] at (1, 1.8) {cornice opaca};
        \node[align = center] at (1, 1) { silicio \\ riflettente};
        \draw[|-|] (0, 2.2) -- (2, 2.2) node[midway, above] {$\approx 5.20 \pm 0.01 \ \mu m$};
    \end{tikzpicture}    
\end{wrapfigure}
Utilizzando dunque la CCD come un reticolo
di diffrazione in riflessione, il singolo pixel è fatto di una parte in silicio
che riflette la luce ed una parte che contorna il pixel che invece assorbe
la luce. La distanza tra i pixel è quindi $a_p$ e la larghezza
della parte riflettente è $D_p$. Si vedrà dunque in riflessione il pattern
dell'intensità sulla parete del laboratorio.  Si può dunque stimare l'angolo di
riflessione del primo ordine rifratto e di quello $-1$. si può prendere
la distanza sulla parete $l$ tra i due ordini uno e meno uno e la distanza
tra la parete e la CCD come $L$: 
\begin{gather*}
    \sin\theta \approx \tan\theta \approx \frac{l}{2L}
\end{gather*}

\section{Seconda parte delle'esperienza: verifica della legge di diffrazione}
Nella seconda parte dell'esperienza si verifica la legge di diffrazione di un'onda
piana incidente su di un reticolo diffrazione. Misuro allora 
$\theta_i$ per valori diversi di $\lambda_i$ ricavando $a_i$. Mi assicuro allora che
i vari valori di $a_i$ trovati siano consistenti tra di loro. Utilizzando una
lampada a gas di Mercurio, si prende una cella di vetro all'interno della quale ci si mette un gas
di interesse: ci si mette un catodo ed un anodo e ci si attacca un generatore ad alta tensione.
Mediante degli impulsi ad alta tensione si eccitano gli elettroni  degli 
gli atomi di Mercurio, i quali saltano da un orbitale all'alto emettendo
dei fotoni che hanno energia
\begin{gather*}
    \Delta E = h \nu 
\end{gather*}
In corrispondenza di un angolo più piccolo si avrà dunque una luce più violacea
e, più cresce l'angolo, più la luce tenderà verso il rosso. Per ogni $\lambda_i$ per lo
spettro goniometro si va ad osservare un $\theta_i^{+}$ ed un $\theta_i^{-}$, per trovare dunque
l'angolo per una data lunghezza d'onda si avrà
\begin{gather*}
    \theta = \frac{\theta_i^{+} - \theta_i^{-}}{2} \ \Longrightarrow \ \sin \theta = m \frac{\lambda}{a}
\end{gather*}
Questa misura $\theta_i^{+}$ e $\theta_i^{-}$ si fa perché, dato che
la lettura è fatta con un crocefilo, vicino allo zero la luce è talmente intensa
che il crocefilo non è visibile, dunque si osserva cosa accade a destra e a sinistra del laser
per ottenere $\theta$. Si ottiene le incertezza sia per $\theta_i^{+}$ che per $\theta_i^{-}$. 

\section{Procedure operative in laboratorio}
Ci si assicura che il laser passi dalla fenditura e dal filtro in modo perfettamente
centrato e vada verso la telecamera. Sul computer si apre la telecamera e 
e si può osservare il profilo a seno cardinale. Si deve dunque
calibrare la fotocamera, la quale sta raccogliendo anche la luce
a $50hz$ della stanza. 



\chapter{Interferometro di Michelson}
\section{Apparato sperimentale}
\begin{wrapfigure}{r}{0.4\textwidth}
    \centering
    \caption{L'interferometro di Michelson}
    \begin{tikzpicture}
        \draw[->](0, 0.2) -- (2.25, 0.2);
        \draw[->](2.25, 0.2) -- (2.25, 2);
        \draw[->](2.5, 2) -- (2.5, 0.75);
        \draw[->](2.5, 0.75) -- (1, 0.75);
        \draw[->](2.5, 0.75) -- (2.7, 0.35) -- (2.7, -2);
        \draw[->](2.25, 0.2) -- (2.55, 0) -- (4.5, 0);
        \draw[->](4.5, -0.25) -- (2.42, -0.25);
        \draw[->](2.42, -0.25) -- (2.42, -2);
        \draw[->](2.42, -0.25) -- (2.15, -0.1) -- (0, -0.1);
        \draw(1.75, -1) -- (2, -1.2) -- (3, 1) -- (2.75, 1.2);
        \draw[very thick](2.75, 1.2) -- (1.75, -1);
        \draw[pattern = north west lines, pattern color = white](1.75, 2) rectangle (3, 2.5);
        \draw[pattern = north west lines, pattern color = white](4.5, 0.75) rectangle (5, -1);
        \filldraw (2.1, -2) rectangle (3, -2.25);
    \end{tikzpicture}
\end{wrapfigure}
L'interferometro di Michelson è un beam splitter e non un polarizer:
dunque non polarizza la luce ma semplicemente divide il fascio 
luminoso in base  al campo incidente. Il laser che si utilizza è un laser
che spara una luce a lunghezza d'onda ignota. Dunque il primo scopo dell'esperienza
è proprio quello di misurare la lunghezza d'onda incognita che viene
sparata dal laser elio-neon. Si pul esprimere l'intensità
totale del campo elettrico in funzione della distanza dei due rilevatori
come
\begin{gather*}
    I = I_0 \sin^{2}\left(\frac{\pi(L_1 - L_2)}{\frac{\lambda}{2}}\right)   
\end{gather*}
La lunghezza d'onda incognita è quella del laser verde utilizzato per l'esperienza
della polarizzazione. Il segnale del fotodiodo sui rilevatori è misurato 
tramite un oscilloscopio digitale che misura una tensione analogica
in Volt a 12 bit. \\
Montando uno dei due specchi su di una ceramica piezoelettrica: ossia 
una ceramica al quale applico una tensione variabile, lo spessore della ceramica
cambia: si dilata dunque in base alla tensione che gli è applicata. Questo fenomeno 
permette di cambiare dunque la lunghezza di uno dei due cammini ottici
attraverso della tensione. Si deve dunque misurare, in funzione della tensione, 
l'intensità del fascio di luce e tracciare un fit sinusoidale attraverso i dati 
raccolti dall'oscilloscopio digitale. Si genera dunque una rampa lineare della tensione applicata
alla ceramica piezoelettrica che dovrebbe, in linea teorica, determinare un allungamento
$\Delta L$ lineare, anche se in realtà è un andamento monotono che non si conosce 
(ma sicuramente non lineare).
\begin{gather*}
    \begin{tikzpicture}
        \draw[->](0, 0) -- (3, 0) node[at end, below] {$t$};
        \draw[->](0, 0) -- (0, 2) node[at end, left] {$V$};
        \draw(0, 0) -- (3, 2);
        \draw[->](5, 0) -- (8, 0) node[at end, below] {$V$};
        \draw[->](5, 0) -- (5, 2) node[at end, left] {$I$};
        \draw[domain=5:8, dashed] plot (\x, {sin(\x * 3 r) * sin(\x * 3 r)});
        \draw[->](10, 0) -- (13, 0) node[at end, below] {$V$};
        \draw[->](10, 0) -- (10, 2) node[at end, left] {$\Delta L$};
        \draw(10, 0) -- (13, 2) node[at end, right] {No};
        \draw[domain=10:13, red] plot (\x, {((\x - 10) * (\x- 10)) /  3});
        \draw[domain=10:13, cyan, samples = 50] plot (\x, {sqrt(\x - 10)});
    \end{tikzpicture}
\end{gather*}
Sul grafico si deve avere sia la tensione in uscita dal rilevatore di luce, che la
tensione alla ceramica piezoelettrica: dunque sullo schermo del computer
si hanno entrambe le tensioni. Il grafico è tratteggiato in quanto la sinusoide è
calcolata ottenendo dati ogni pochi millisecondi. I dati sono 
caricati sul programma scritto in Mathematica in modo tale che si analizzi i punti per trovare 
i massimi locali per determinare la curva $V - \Delta L$.
\begin{gather*}
    \begin{tikzpicture}
        \draw[->](0, 0) -- (3, 0);
        \draw[->](0, 0) -- (0, 2);
        \filldraw[red](0.7, 0.5) circle (1pt);
        \filldraw[green](0.5, 0.5) circle (1pt);
        \filldraw[red](1.4, 0.75) circle (1pt);
        \filldraw[green](1.1, 0.75) circle (1pt);
        \filldraw[red](2.4, 1.25) circle (1pt);
        \filldraw[green](1.9, 1.25) circle(1pt);
    \end{tikzpicture}
\end{gather*}
Si toglie ora il laser
elio-neon e si sostituisce con il laser verde dell'esperienza della polarizzazione
in modo tale che si possa ottenere un grafico in funzione della tensione l'intensità
e si ottiene un grafico simile a quello ricavato con la lampada
elio-neon. I grafici dell'intensità sono leggermente sfasati tra di loro.
L'intensità del seno quadro del laser rosso ha un massimo per
\begin{gather*}
    I_R(\Delta L(v)) = I_0 \sin^{2}\left(\frac{\pi \Delta L(v)}{\frac{\lambda_R}{2}}\right) \ \Longrightarrow \ \max \frac{\pi \Delta L (v)}{\frac{\lambda_R}{2}} = \frac{\pi}{2} + m_R \pi
\end{gather*}
Mentre per il laser verde ha un max per
\begin{gather*}
    I_R(\Delta L(v)) = I_0 \sin^{2}\left(\frac{\pi \Delta L(v)}{\frac{\lambda_R}{2}}\right) \ \Longrightarrow \ \max \frac{\pi \Delta L(v)}{\frac{\lambda_V}{2}} = \frac{\pi}{2} + m_V \pi
\end{gather*}
Si possono dunque ottenere le curve dei massimi delle due sorgenti
rosse e verde.
\begin{gather*}
    \left\{\begin{array}{l}
        m_R = \frac{\Delta L(v)}{\frac{\lambda_R}{2}} - \frac{1}{2} \\
        m_V = \frac{\Delta L(v)}{\frac{\lambda_V}{2}} - \frac{1}{2}
    \end{array}\right.
\end{gather*}
Si deve immaginare ogni volta che ci sia un certo valore di tensione 
della ceramica piezoelettrica in modo tale che si possa far coincidere la curva 
di tensione rossa con quella verde. Matematicamente è possibile farlo attraverso
$m_R \frac{\lambda_R}{\lambda_V}$ per cui la curva rossa viene moltiplicata per il 
rapporto tra le lunghezze d'onda ottenendo
\begin{gather*}
    \frac{\Delta L (v)}{\frac{\lambda_V}{2}} - \frac{1}{2}\frac{\lambda_R}{\lambda_V}
\end{gather*}
Per far coincidere le curve devo aggiungere un certo offset opportuno,
ossia $\frac{1}{2}\frac{\lambda_R}{\lambda_V} - \frac{1}{2}$:
\begin{gather*}
    \frac{\Delta L(v)}{\frac{\lambda_V}{2}} - \frac{1}{2}
\end{gather*}
Si \underbar{misura} il coefficiente $F$, ossia il rapporto tra le lunghezze d'onda: si misura dunque questo 
coefficiente per qualche volta in modo da ricavare anche lo scarto massimo su questo: il $\lambda_R$ è dato 
dalle caratteristiche della lampada a elio neon.
\begin{gather*}
    \left< \lambda_V \right> = \frac{\left< \lambda_R \right> }{\left< F \right> } \ \Longrightarrow \ \frac{\Delta \lambda_V}{\lambda_V} = \frac{\Delta \lambda_R}{\lambda_R} + \frac{\Delta F}{F}
\end{gather*} 
In generale lo spettro è centrato intorno ad una certa lunghezza d'onda con un certo 
intervallo: si stabilisce dunque tramite l'interferometro di Michelson
questa ampiezza: per farlo si deve ricordare la teoria. In presenza di una intensità luminosa
che è distribuita secondo una gaussiana con parametro di larghezza $\sigma = \Delta \omega$: 
se questa è spedita attraverso Michelson, si vede che l'intensità in funzione del $Delta L $ modificherà
la Gaussiana totalmente:
\begin{align}
    I(\Delta L) = \frac{I_0}{2}\left(1 - \exp\left(-\left(\frac{\Delta L - \Delta \omega}{c}\right)^{2}\right) \cos\left(\frac{2\Delta L \omega_0}{c}\right)\right)
\end{align}
Il grafico che si vede facendo questo esperimento è il seguente:
\begin{gather*}
    \begin{tikzpicture}
        \draw[->](0, 0) -- (4, 0) node[at end, below] {$\Delta L$};
        \draw[->](0, 0) -- (0, 3) node[at end, left] {$I(\Delta L$)};
        %\draw[domain = 0.1:4, samples = 50] plot (\x, {2 * sin(3 *\x r) * sin(3 * \x r) / (\x * 0.5)});
    \end{tikzpicture}
\end{gather*}

\section{Sorgente luminosa non monocromatica}
Se si avesse una sorgente di luce non monocromatica, 
l'intensità in uscita dall'interferometro di Michelson
è data dalla seguente (con le formule di bisezione):
\begin{gather*}
    I(\Delta L)  = I_0 \sin^{2}\left(\frac{\Delta L \omega_0}{c}\right)
\end{gather*}
E si riscrive nel modo più canonico:
\begin{gather*}
    I_0\sin^{2}\left(\pi\frac{\Delta L}{\frac{\lambda_0}{2}}\right)
\end{gather*}
L'uscita dipende dal seno al quadrato perché l'onda riflessa acquisisce
un termine di fase negativo, dunque il campo elettrico, e di conseguenza
l'intensità, sono zero. Per $\Delta L << 1$ è come se la sorgente fosse monocromatica.
Quando invece
\begin{gather*}
    \frac{\Delta L\Delta \omega}{c} \approxeq 1
\end{gather*}
Allora questa lunghezza $l_c = \Delta L$ per cui è uguale ad uno prende 
il nome di lunghezza di coerenza: questa fa sì che il coseno sia molto 
piccolo e dunque si ottiene che l'ampiezza dell'oscillazione tende
all'asintoto $\frac{I_0}{2}$. Dunque si sceglie questa lunghezza di coerenza
in modo tale che
\begin{gather*}
    I(l_c) = \frac{I_0}{2}\left(1 + e^{-1}\right)
\end{gather*}
Ossia che si trovi un massimo per la funzione. Per stimare la lunghezza
di coerenza devo anche dare una stima per  $\Delta \omega$: dunque si può relazionare
questa grandezza con la lunghezza d'onda 
\begin{gather*}
    \frac{\Delta \omega}{\omega_0} = \frac{\Delta \lambda}{\lambda_0} \ \Longrightarrow \ \Delta \omega = \frac{\Delta \lambda \omega_0}{\lambda_0} 
\end{gather*}
Allora
\begin{gather*}
    l_c = \frac{c\lambda_0}{\Delta \lambda} \frac{\lambda_0}{2\pi c}
\end{gather*}
Ponendo $\lambda_0 \approx 550 \ nm$ e $\Delta \lambda \approx 150 \ nm$, si ha che
$l_c \approx 350 \ nm$. È fondamentale non toccare lo specchio con la ceramica
piezoelettrica in quanto se si sposta dalla condizione di $\Delta L = 0$ non si vedrà più niente.
Sperimentalmente per trovare $l_c$, si deve misurare $\lambda_0$ e poi si utilizza
il programma in Mathematica. 
\begin{itemize}
    \item Si ricava una traccia per $I(\Delta L)$ attorno a $\Delta L \approx 0$. 
    \item Si determina la lunghezza d'onda $\lambda_0$ 
    \item Si identificano il numero di oscillazioni $n$ affinché si ottenga il valore massimo
    dell'oscillazione sia da $I_0$ a $\frac{I_0}{2}(1 + e^{-1})$ e moltiplico
    per $\frac{\lambda_0}{2}$ per trovare $l_c$.
    \item Si trova $\Delta \lambda_0$ con la formula per la lunghezza di coerenza.  
\end{itemize}
La luce è filtrata da un filtro verde in modo tale che la lunghezza d'onda sia centrata
intorno a $\lambda_0 \approx 552 \pm 2 \ nm$ e che ha un intervallo di 
misura di circa $\Delta \lambda \approx 5 \pm 2 \ nm$. In questo modo il rilevatore
vede una lunghezza d'onda molto stretta e dunque la lunghezza di coerenza aumenta. 

\appendix
\chapter{Approfondimenti dell'ottica}
\section{Interferenza da film sottili}
Consideriamo cosa succede ad un onda piana che attraversa
un sottile strato di materiale dielettrico a facce piane e parallele
con indice di rifrazione $n > 1$. 
\subsection{Legge di Fresnel}
La legge di Fresnel sulla riflessione e trasmissione ad una interfaccia
stabilisce la quantità di luce riflessa e quella trasmessa: 
\begin{align}
    r = \frac{n_1 - n_2}{n_1+ n_2}
\end{align}
Questo coefficiente stabilisce che il campo riflesso e quello trasmesso
è dato dalle seguenti relazioni
\begin{gather*}
    E_r = r E_{in} \qquad E_t = t E_{in}
\end{gather*}
Già da qui si scopre che nel caso tipico Aria-Vetro, si scopre che
$r = -0.2$ e dunque si può quantificare l'intensità e le relazioni tra 
l'intensità riflessa e quella trasmessa:
\begin{gather*}
    I_r = 0.04 I_{in}
\end{gather*} 
Ossia si ha una riflessione dell'ordine del $4\%$. si può dunque 
indicare con $R$ il coefficiente di riflessione dell'intensità luminosa;
per le regole di conservazione dell'energia $T = 1 - R$, dunque
\begin{gather*}
    I_T = TI_{in}
\end{gather*}
Allora si può dire che
\begin{gather*}
    T = 1 - R = 1 - r^{2} \qquad t = \sqrt{1 - r^{2}} 
\end{gather*}
Ottenendo allora la relazione generale che lega il coefficiente di trasmissione 
a quello di riflessione: nel caso specifico, dato $R = 0.04$, allora
si può dire che $t = 0.98$: dunque il campo elettrico trasmesso 
nel dielettrico è ben il $98\%$ del campo totale. 

\subsection{Applicazione della legge di Fresnel}
\begin{wrapfigure}{r}{0.4\textwidth}
    \centering
    \caption{Interfaccia}
    \begin{tikzpicture}
        \draw(0, 0) -- (4, 0);
        \draw(0, -0.5) -- (4, -0.5);
        \node at (-0.25, 0.5) {$n_1$};
        \node at (-0.25, -0.25) {$n_2$};
        \node at (-0.25, -0.75) {$n_1$};
        \draw(1.5, 2) -- (2, 0) -- (2.5, 2);
        \draw[thick, ->](1.5, 2) -- (1.75, 1) node[at end, left] {$\vv{E_0} $};
        \draw[dashed] (2, 0) -- (2, 2);
        \draw(2, 0) -- (2.4, -0.5) -- (2.8, 0) -- (3.3, 2);
        \draw[red, thick](2, 0) -- (2.5, 2);
        \draw[cyan, thick](2.8, 0) -- (3.3, 2);
        \draw[dashed](2, 0) -- (2, -0.5);
        \draw (2, -0.25) arc (-90:-70:0.5) node[at start, left] {$\theta'$};
        \draw(2.4, -0.5) -- (2.65, -1);
        \draw(2.8, 0) -- (3.2, -0.5) -- (3.45, -1);
        \draw(3.2, -0.5) -- (3.6, 0);
        \draw[green, thick](3.6, 0) -- (4.1, 2);
    \end{tikzpicture}    
\end{wrapfigure}
Presa l'interfaccia dielettrica con indice di rifrazione $n_2$
immerso nell'aria, supponendo che i raggi di luce incidano con un angolo
$\theta << 1$, sull'interfaccia aria-vetro si ha una prima riflessione e
trasmissione: per cui si ottiene che una parte del fascio di luce 
viene trasmessa e una parte riflessa, così accade per tutti i fasci riflessi all'interno del 
vetro che provano ad uscire. Dunque, considerato che la luce che viene riflessa 
all'interno del vetro, devo aggiungere un ritardo di fase per tenere conto 
di questo cammino extra che è stato compiuto. Il cammino che compie in più il campo che esce dal vetro 
è dato da
\begin{gather*}
    l = 2\frac{d}{\cos\theta'}
\end{gather*}
Il fascio rosso dunque ha modulo del campo:
\begin{gather*}
    -|r|E_0 \cos(kx - \omega t)
\end{gather*}
Il fascio rosso invece ha come modulo (dato dalla differenza del cammino):
\begin{gather*}
    E_0 t^{2}r\cos\left(kx - \omega t + 2\frac{kn_2 d}{\cos\theta'}\right)
\end{gather*}
Mentre il fascio verde è dato da
\begin{gather*}
    E_0t^{2}r^{3}\cos\left(kx - \omega t + 4\frac{kn_2d}{\cos\theta'}\right)
\end{gather*}
Dunque per studiare riflessione e trasmissione basterà studiare semplicemente i primi 
due fasci in quanto dal terzo in poi i contributi sono molto piccoli e dunque valgono i
risultati che si sono ottenuti nell'interferometro di Michelson. Il motivo per il quale
le bolle di sapone risultano colorate (dei colori dello spettro luminoso) è dato dal
fatto che la luce che esce e rientra nel sapone genera interferenza. Tutte le volte che 
si usa un laser si deve considerare che ogni volta che il fascio di laser passa
attraverso i componenti ottici si perde il $4\%$ dell'intensità. 

\begin{wrapfigure}{r}{0.4\textwidth}
    \centering
    \caption{Il coating delle lenti a bassa riflettività}
    \begin{tikzpicture}
        \draw(0, 0) -- (4, 0);
        \draw(0, -1.5) -- (4, -1.5);
        \filldraw[cyan, opacity = 0.3](0, 0) rectangle (4, -0.25);
        \node at (4.6, -0.15) {$n_1 > 1.5$};
        \node at (2, -0.75) {$n_{vetro} = 1.5$};
        \draw(2, 1) -- (2.5, 0) -- (2.75, -0.25) -- (3, 0) -- (3.5, 1);
        \draw(2.5, 0) -- (3, 1);
    \end{tikzpicture}    
\end{wrapfigure}
Per ovviare al problema
si può depositare sulla superficie delle lenti un materiale dielettrico molto sottile 
con un indice di rifrazione $n_1 > n_2$, ossia maggiore dell'indice di rifrazione
del vetro. Lo spessore del \textbf{coating} fatto da questo materiale dielettrico lo scelgo 
in modo tale che tra i due fasci riflessi ci sia interferenza distruttiva.
Dunque l'intensità di luce che viene riflessa dopo che entra nel dielettrico
(che deve essere in interferenza con quella inizialmente riflessa) è data da
\begin{gather*}
    I_r = I_0 t^{4}r^{6}
\end{gather*}
E dunque si perde solamente lo $0.06\%$ della luce. 

\subsection{Coating ad alta riflettività}
\begin{wrapfigure}{r}{0.4\textwidth}
    \centering
    \caption{Coating ad alta riflettività}
    \begin{tikzpicture}
        \draw(0, 0) -- (4, 0) node[at start, above] {$n_{aria}$};
        \draw (0, -1) -- (4, -1);
        \draw[<->](0, 0) -- (0, -1) node[midway, left] {$\frac{\lambda}{4n_1}$};
        \draw(0, -2.5) -- (4, -2.5);
        \draw[<->](0, -1) -- (0, -2.5) node[midway, left] {$\frac{\lambda}{4n_2}$};
        \node at (4.2, -0.5) {$n_1$};
        \node at (4.8, -1.75) {$1 < n_2 < n_1$};
        \node at (4.2, -3) {$n_1$};
        \draw(0, -3.5) -- (4, -3.5);
        \draw(1.5, 0.5) -- (2, 0) -- (2.5, 0.5) node[at end, right] {1};
        \draw(2, 0) -- (2.5, -1) -- (3, 0) -- (3.5, 0.5) node[at end, right] {2};
        \draw(2.5, -1) -- (3, -2.5) -- (3.5, -1) -- (4, 0) -- (4.5, 0.5) node[at end, right] {3};
    \end{tikzpicture}    
\end{wrapfigure}
Si considera un coating ad alta riflettività: si può sfruttare la legge di Snell, e
di Fresnel e l'interferenza costruttiva in riflessione per creare specchi 
ad elevata riflettività. Si utilizzano film sottili alternati a vetro in modo tale che
si possa sfruttare l'interferenza costruttiva. Si definiscono allora le proprietà dei seguenti 
raggi luminosi (quando ovviamente $\theta << 1$):
\begin{enumerate}
    \item Ha uno shift di $\pi$ dovuto alla riflessione dovuta all'interfaccia 
    aria-mezzo dielettrico con $n_{aria} < n_{d}$. 
    \item Il secondo raggio ha un ritardo di fase di $\pi$ dovuto alla doppia propagazione
    all'interno del dielettrico di spessore di $\frac{\lambda}{4n_1}$.
    \item Il terzo fascio ha un ritardo di fase di $\pi$ rispetto al fascio 2 dovuto alla propagazione
    all'interno del film spesso $\frac{\lambda}{4n_2}$ ed un ritardo di fase di $\pi$ dovuto
    alla riflessione $n_2 \to n_1$. 
\end{enumerate}
Anche questo coating deve necessariamente essere utilizzato con una certa
lunghezza d'onda e con un determinato angolo di incidenza. 

\section{Cavità ottiche}
Supponendo di avere uno specchio ad alta riflettività e di avere 
un infinità di strati sulle superfici di questo specchio; la probabilità
di riflessione $R \approx 1$. Se si ponesse un'altra interfaccia sotto 
allo specchio ad alta riflettività, mi aspetto che lo specchio sotto non 
interferisca con lo specchio sopra. Quello che accade, invece, è che il primo specchio fa
trasmettere tutta la luce sullo specchio posto sotto. 
\begin{wrapfigure}{r}{0.4\textwidth}
    \centering
    \caption{}
    \begin{tikzpicture}
        \draw(0, 0) -- (4, 0);
        \draw(0, -1.5) -- (4, -1.5);
        \draw(0.5, 1) -- (1.75, -2.5) node[at end, left] {(1)};
        \draw(1.75, 0) -- (2.75, -2.5) node[at end, left] {(2)};
        \draw(2.75, 0) -- (3.75, -2.5) node[at end, left] {(3)};
    \end{tikzpicture}    
\end{wrapfigure} 
Si studiano dunque i contributi totali
\begin{enumerate}
    \item $t^{2}E_0\cos(kx' - \omega t)$
    \item $t^{2}r^{2}E_0\cos(kx' - \omega t + 2kn\frac{d}{\cos\theta'})$
    \item $t^{2}r^{4}E_0\cos(kx' - \omega t +4kn \frac{d}{\cos\theta'})$
\end{enumerate}
Complessivamente dunque, il campo elettrico totale in trasmissione diventa
la somma di tutti i contributi trasmessi dopo l'interfaccia:
\begin{gather*}
    E_T = \sum_{j = 0}^{+\infty } t^{2}r^{2j}E_0\cos\left(kx' - \omega t + 2j\frac{knd}{\cos\theta'}\right)
\end{gather*}
Si può dunque chiamare 
\begin{gather*}
    \delta = \frac{2knd}{\cos\theta'} 
\end{gather*}
Allora, utilizzando i complessi, si può ricavare l'intensità totale del campo 
trasmesso sotto l'interfaccia come:
\begin{gather*}
    E_T = (1 - R)E_0 \sum_{j = 0}^{+\infty }R^{j}\text{Re}\left[e^{i(kx' - \omega t + j\delta)}\right] \\
    \ \Longrightarrow \ (1- R)E_0 \text{Re}\left[e^{i(kx' - \omega t)}\sum_{j = 0}^{+\infty } R^{j}e^{ij\delta}\right]
\end{gather*}
Si ricorda dunque la serie notevole
\begin{gather*}
    \sum_{i = 1}^{+\infty }h^{i} = \frac{1}{1 - h} 
\end{gather*}
E dunque
\begin{gather*}
    E_T = (1 - R)E_0 \text{Re}\left[e^{i(kx' -\omega t )} \frac{1}{1 - Re^{i\delta}}\right]
\end{gather*}
si può dunque risolvere prendendo il modulo del secondo numero complesso 
ed il coseno della somma delle fasi complesse:
\begin{gather*}
    E_T = (1 - R)E_0 \left| \frac{1}{1 - Re^{i\delta}} \right| \cos(kx' - \omega t + \phi) 
\end{gather*}
Il fatto che questo campo elettrico oscilli con qualsiasi fase $\phi$ è irrilevante
in quanto poi si medierà l'intensità nel tempo e dunque il termine di fase
non influirà: 
\begin{gather*}
    I_T = I_0(1 - R) \left| \frac{1}{1 - Re^{i\delta}} \right|^{2} = \\
    \frac{I_0(1 - R)^{2}}{\left| 1 - R(\cos\delta + i\sin\delta) \right| } = \\
    \frac{I_0(1-R)^{2}}{1 + R^{2} - 2R\cos\delta} \ \Longrightarrow \ I_0
\end{gather*}
A questo punto, dato che si è preso degli specchi con un alto indice di riflettività $R = 1 - \epsilon$, 
in genere dovrebbe passare un termine $\epsilon$ dallo specchio. Se però c'è un altro specchio 
ad alta riflettività sotto il primo specchio, accade che l'intensità 
trasmessa è proprio  $I_0$ in quanto il coseno fa 1. Se la cavità tra gli specchi
è largo $d$, l'intensità all'interno della cavità è dato da
\begin{gather*}
    \frac{I_0}{1 - R} = \frac{I_0}{1 - \epsilon}
\end{gather*}
Dunque, se si sceglie $\epsilon = 10^{-4}$, allora passerà una luce pari
a $10^{4}I_0$: la potenza dentro la cavità è molto maggiore di quella in entrata.
Il motivo di questo paradosso è che quando il campo elettrico incide sulla prima
superficie riflettente, c'è una piccola probabilità che essa si possa trasmettere. 
Se si accende un laser per poco tempo, è impossibile che dentro la cavità 
ci sia della luce: accendendo allora il laser, si osserva che i pacchetti luminosi
in ingresso arrivano uno dopo l'altro. 
\begin{itemize}
    \item Arriva il pacchetto rosso, che ha una piccola probabilità di 
    riflettere e si trasmette quindi poca intensità.
    \item Arriva il pacchetto blu, che ha anch'esso una piccola probabilità di 
    trasmettere e dunque si trasmessa poca intensità ed entra in controfase con il pacchetto rosso in 
    uscita. 
    \item Arrivando gli altri pacchetti, si sommano sempre di più i contributi in uscita degli altri 
    pacchetti (sono i contributi che si sono trasmessi all'interno e che stanno uscendo dall'interfaccia)
    che sono in controfase con quelli riflessi e dunque più pacchetti arrivano e più si attenua la
    riflessione del primo specchio, annullando, dopo un tempo molto lungo la riflessione.
\end{itemize}
A questo punto si ha solo trasmissione all'interno della cavità ottica dopo un tempo infinito poiché tutti i contributi 
che sono stati trasmessi all'interno hanno completamente annullato qualsiasi riflessione. Dunque aspettando un 
tempo infinito, si arriva alla condizione di intensità luminosa trovata prima e a quel punto 
tutta l'intensità del laser viene trasmessa all'interno della cavità.
Se si vuole studiare la condizione per il quale $\delta$ sia un multiplo di $2\pi$, allora
\begin{gather*}
    2\pi m = \frac{2knd}{\cos\theta} \ \Longrightarrow \ m \frac{nd}{\lambda\cos\theta}
\end{gather*}
La condizione di risonanza della cavità ottica si ha quando 
\begin{gather*}
    \frac{nd}{\cos\theta} = \frac{\lambda}{2}m
\end{gather*}
La condizione fisica dunque per la risonanza è a multipli della metà della lunghezza d'onda e si realizzano 
le condizioni per la trasmissione totale dell'intensità luminosa all'interno della cavità. Si può dunque studiare il
picco di questa funzione (che è una Gaussiana centrata intorno a questi valori ottenuti): 
Si impone allora che il delta larghezza a metà altezza è dato da:
\begin{gather*}
    \frac{I_T}{I_0} = \frac{1}{2}
\end{gather*}
Quando 
\begin{gather*}
    \cos\delta = 1 - \frac{(1- R)^{2}}{2R} \ \Longrightarrow \ \cos\delta \approx 1 - \frac{\epsilon^{2}}{2}
\end{gather*}
Si ha dunque un modo molto preciso per poter confrontare $\lambda$ con distanze macroscopiche di
distanze tra specchi. 

\section*{Utilizzi pratici dell'ottica}
\subsection{L'esperimento di Virgo}
L'esperimento di Virgo utilizza le cavità ottiche in modo tale da aumentare
l'intensità del laser utilizzato per trovare le onde gravitazionali. La potenza 
del laser utilizzato è a $80$ Watt, ma con un \textit{\textbf{finesse}} $\frac{1}{\epsilon} = 5400$, dunque
aumenta la cavità aumenta la potenza del laser. La limitazione di questo
esperimento è sicuramente l'assorbimento degli specchi che iniziano a scaldarsi
quando passa il laser.  

\subsection{Orologi precisi}\
Si utilizzano degli \textbf{orologi} realizzati con le
cavità ottiche che permettono di stabilizzare in maniera molto precisa
la lunghezza d'onda della radiazione e dunque la frequenza della lunghezza
d'onda grazie a questi orologi che hanno una dilatazione termica quasi nulla per 
temperature vicine alla temperatura ambiente. 

\end{document}

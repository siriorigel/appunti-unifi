\documentclass[a4paper, oneside]{article}
\usepackage{graphicx}
\usepackage{amsthm}
\usepackage{amsmath} 
\usepackage{amssymb}
\usepackage[a4paper,
            bindingoffset=0.2in,
            left=2cm,
            right=2cm,
            top=2cm,
            bottom=2cm,
            footskip=.25in]{geometry}
\usepackage[italian]{babel}
\usepackage{pgfplots}
\usepackage{tabularx}
\usepackage{tikz}
\usepackage{wrapfig}
\usepackage{color}
\usepackage[d]{esvect}
\usepackage{chemfig}
\usepackage{mhchem}
\usepackage{svg}
\usepackage{float}
%\definecolor{page}{rgb}{0.129,0.157,0.212}
%\pagecolor{page}
%\color{white}
\graphicspath{ {./images/} }
\usetikzlibrary{shapes.geometric}
\usetikzlibrary{datavisualization}
\usetikzlibrary{datavisualization.formats.functions}
\usetikzlibrary{patterns}
\pgfplotsset{width=10cm,compat=1.18}

\title{Esperienza Diffrazione}
\author{Gruppo 19 \\ Fabbri Marco, Miliani Tommaso, Mongatti Giulio, Tinacci Lorenzo}
\date{10 Dicembre 2025}

\begin{document}
\newtheoremstyle{theoremEnv}
                {}          % Space above
                {}          % Space below
                {\slshape}  % Body font
                {}          % Indent amount
                {\bfseries} % Head font
                {.}         % Punctuation after head
                {\newline}  % Space after theorem head
                {}          % Theorem head spec
\theoremstyle{theoremEnv}

\newtheorem{definition}{Definizione}[section]
\newtheorem{theorem}{Teorema}[section]
\newtheorem{lemma}{Proposizione}[section]
\newtheorem{observation}{Osservazione}[section]
\newtheorem{corollary}{Corollario}[theorem]
\newtheorem{example}{Esempio}[section]
\newtheorem{remark}{Enunciato}[section]

\maketitle

\section{Scopo dell'esperienza ed ipotesi}
Lo scopo generale dell’esperienza è la verifica delle leggi per la distribuzione angolare
dell’intensità di un onda piana in uscita da una singola fenditura e da un reticolo nel
limite di Fraunhofer (grandezza della fenditura molto maggiore della distanza a cui si misura l'intensità). 
La legge che regola la diffrazione da singola fenditura è la seguente:
\begin{align}
    I(\theta) = I_0 \text{sinc}^{2}\left(\frac{kD}{2}\sin \theta\right)
\end{align}
Mentre quella per la diffrazione da un reticolo di fenditura
\begin{align}
    I(\theta) = I_0 \text{sinc}^{2}\left(\frac{kD}{2}\sin\theta\right) \frac{\sin^{2} \left(\frac{nak}{2}\sin\theta\right)}{n^{2}\sin^{2}\left(\frac{ka}{2}\sin\theta\right)}
\end{align}

\section{Diffrazione da singola fenditura}
\subsection{Schema generale di misura}
Gli sperimentatori devono ottenere, attraverso un sistema di acquisizione e un programma
di analisi, delle immagini per poter ricavare una stima del fattore $B$, tramite la seguente relazione:
\begin{align}
    B = \frac{kD}{2} = \frac{\pi D d}{\lambda x}
\end{align}
Dove
\begin{itemize}
    \item $D$ è lo spessore della fenditura attraverso la quale passa il fascio laser;
    \item $\lambda$ è la lunghezza d'onda del laser utilizzato;
    \item $x$ è la distanza tra la fenditura e il rilevatore CCD che permette di acquisire l'immagine;
    \item $d$ è lo spessore dei pixel della fotocamera.
\end{itemize}
Per tale verifica gli sperimentatori misurano, mediante una telecamera, il profilo di intensità
generato da una fenditura su cui incide un fascio laser e lo confrontano con quanto previsto 
dalla teoria.
Il valore misurato di $B$ dal programma va quindi comparato con un valore atteso ottenuto
dalla formula data; se i due valori $B_{\text{atteso}}$ e $B_\text{medio}$ risultano consistenti, la legge può dirsi verificata.
Per misurare $B$ si allinea il CCD con la fenditura in modo tale che si possano vedere il massimo
di intensità centrale e primi due massimi laterali; si acquisisce quindi il profilo di intensità e si 
analizzano più righe del rilevatore per ottenere un profilo unidimensionale di intensità, determinato dalla legge:
\begin{align}
    I(y) = I_0\text{sinc}^2\left(\frac{kDdy}{2x}\right) 
\end{align}
Dove
\begin{gather*}
    \frac{dy}{x} = \theta  \approx \tan\theta \approx \sin\theta
\end{gather*} 
nel limite di Fraunhofer. \\
Per fare un confronto quantitativo è necessario determinare $B_\text{atteso}$ ottenendo le misure di
$x$ e $d$ con relativa incertezza, mentre le altre grandezze sono fornite dal costruttore.
$x$ viene definito dalla somma della distanza $x'$ tra i supporti rispettivamente della fenditura e del CCD,
e la somma delle distanze dei due apparati dai rispettivi supporti $\delta_1 + \delta_2$. $x'$ viene misurato più
volte con un calibro di sensibilità di $0.1$ cm, così da ottenere media e scarto massimo e poter dunque determinare $x$ ed il suo errore secondo la seguente relazione: 
\begin{align}
    x = \delta_1 + x' + \delta_2 \qquad \quad \Delta x = \Delta \delta_1 + \Delta x'.
\end{align}
Per stimare l'errore sulla dimensione dei pixel, si sfrutta il fatto che la riflessione del 
rilevatore suddiviso in pixel quadrati è analoga alla trasmissione di un reticolo bidimensionale con
fenditure di separazione $a=d$.  Per la misura si usufruisce di un laser con lunghezza d'onda nota
($\lambda = 532$ nm), osservando una diffrazione che segue la legge
\begin{gather*}
    \sin\theta = m \frac{\lambda}{d} \quad m \in \mathbb{Z}
\end{gather*}
la quale è approssimabile, nel limite di Fraunhofer, alla seguente: 
\begin{align}
    \frac{l}{2L} = m\frac{\lambda}{d} 
\end{align}
Dove 
\begin{itemize}
    \item $l$: distanza tra ordini di diffrazione 1 e -1;
    \item $L$: distanza fra la CCD e la parete su cui appare la diffrazione, 
    ottenuta trovando la distanza tra telecamera e la parete $L'$ e sommandoci $\delta_2$.
\end{itemize}
Per la stima dell'errore su $d$ si è misurato 5 volte $L$ e $l$ usando il metro in dotazione. 
Calcolati media e scarto massimo, abbiamo ottenuto l'errore su $d$ usando la propagazione dell'errore
secondo le derivate logaritmiche. 
\begin{align}
    \Delta d = D\left( \Delta l\left|-\frac{1}{l}\right| \cdot \Delta \lambda \left|\frac{1}{\lambda}\right| + \Delta L\left| \frac{1}{L}\right|\right)
\end{align}
Dopo aver ottenuto l'errore su $d$ si utilizza nuovamente la propagazione per trovare $\Delta B_\text{atteso}$: 
\begin{align}
    \Delta B_\text{atteso} = B_\text{atteso} \left(\Delta d \left| \frac{1}{d}\right| + \Delta D\left| \frac{1}{D}\right| + \Delta \lambda\left| -\frac{1}{\lambda}\right| + \Delta x\left| -\frac{1}{x} \right|\right)
\end{align}

\subsection{Apparato sperimentale}
\begin{gather*}
    \begin{tikzpicture}
        \draw(0, -0.25) rectangle (1, 0.25);
        \draw(1, 0) -- (3, 0);
        \draw(3, -0.5) rectangle (3.5, 0.5);
        \draw(3.5, 0) -- (10, 0);
        \draw(4.75, 0.2) rectangle (5.25, 1);
        \draw(4.75, -0.2) rectangle (5.25, -1);
        \filldraw(5, 0) circle (1pt);
        \filldraw(10, 0) circle (1pt) node[anchor = west] {$O$};
        \draw(5, 0) -- (10, 1);
        \draw(7, 0) arc (0:12:2) node[midway, right] {$\theta$};
        \draw[->](10, -1) -- (10, 1) node[at end, right] {$y'$};
        \draw[|-|](5.3, -1) -- (9.95, -1) node[midway, above] {$x'$};
        \node at(5, -1.3) {fenditura};
        \node at(3.25, 0.75) {filtro};
        \node at(0.5, 0.5) {laser};
        \node at (11.3, 0) {rilevatore};
    \end{tikzpicture}
\end{gather*}
L'apparato sperimentale per la verifica della legge per una singola fenditura è composto da un laser, una fenditura
e di una telecamera dotata di rilevatore CCD, tutti allineati fra loro. Per prevenire
la saturazione della telecamera, si pone inoltre un filtro aggiuntivo opportunamente regolabile. 
Dell'apparato sperimentale si conoscono le seguenti caratteristiche:
\begin{itemize}
    \item lunghezza d'onda laser $\lambda = (668.66 \pm 0.01)$ nm;
    \item spessore fenditura $D = (30 \pm 2)\  \mu$m;
    \item distanza fenditura-supporto $\delta_1 = (2.8 \pm 0.3)$ mm;
    \item distanza entrata fotocamere-CCD $\delta_2 = 12.536$ mm;
    \item dimensione del singolo pixel della CCD di 5.2 $\mu$m.
\end{itemize}
Si hanno a disposizione, inoltre, i seguenti strumenti:
\begin{itemize}
    \item calibro con sensibilità di $0.05$ mm;
    \item metro con sensibilità di 2 mm. 
\end{itemize}
Il rilevatore CCD è collegato ad un computer sul quale è presente un programma di prelievo ed
analisi delle immagini ottenute mediante il rilevatore, che permette di eseguire il fit su un'area di pixel scelta arbitrariamente dallo sperimentatore.
Per la buona riuscita del fit, è necessario disabilitare il gain e la gamma riducendo al minimo
il segnale in ingresso, diminuire il tempo di esposizione per rendere più visibili i massimi 
ed i minimi di intensità del fascio e settare il framerate a circa 10 fps per eliminare il flickering 
causato dalle lampade a led del laboratorio. 

\subsection{Misure} 
Per trovare $B$ si è utilizzato un programma di analisi dati scritto in Mathematica,
ricavando, dalla seguente foto,
\begin{figure}[H]
    \centering
    \includegraphics[width=0.4\textwidth]{terza.png}
\end{figure} \mbox{} \\
le seguenti misure di $B$, a seconda delle righe di pixel considerate.
\begin{gather*}
    \text{Prima foto} \\
    \begin{tabular}{ c c }
        \hline
        $B$ & Righe \\
        \hline
        $0.00980836$ & 376 -- 401 \\
        $0.00976523$ & 375 -- 610 \\
        $0.00956782$ & 513 -- 617\\
        $0.00989841$ & 374 -- 474 \\
        $0.00970475$ & 465 -- 624\\
        \hline
    \end{tabular}
\end{gather*}
Di seguito si riportano le misure di $x'$ misurate con il calibro e le misure
di $l$ ed $L$ misurate con il metro:
\begin{gather*}
    \begin{tabular}{c c c}
        \hline
        $x'$ (mm) & $l$ (cm) & $L$ (m)\\
        \hline
        61.40 & 93.8  & 4.10 \\
        61.25 & 93.6  & 4.08 \\
        61.15 & 93.6  & 4.08 \\
        61.20 & 94.0  & 4.09 \\
              & 93.8  & 4.09 \\
        \hline
    \end{tabular}
\end{gather*}
Da questi dati si ottengono le seguenti misure con errore associato (ricordando 
di sottrarre l'offset di $10.0$ cm dalle misure di $l$):
\begin{itemize}
    \item $x' = (61.2 \pm 0.2)$ mm;
    \item $l = (83.8 \pm 0.4)$ cm;
    \item $L = (4.09 \pm 0.02)$ m.
    \item $B_\text{medio} = 0.00977 \pm 0.00013$.
\end{itemize}
Da queste misure si possono ottenere le altre misure indirette mediante le relazioni descritte prima con
relativo errore associato:
\begin{itemize}
    \item $x = (76.4 \pm 0.5)$ mm;
    \item $d = (5.20 \pm 0.03) \ \mu$m;
    \item $B_\text{atteso} = 0.0096 \pm 0.0008$.
\end{itemize}

\subsection{Commento qualitativo}
I valori ottenuti di $B_\text{medio}$ e $B_\text{atteso}$ risultano compatibili. Si noti che l'errore
di $B_\text{atteso}$ è di gran lunga maggiore rispetto a quello di $B_\text{medio}$ a causa dell'errore
considerevole sulla larghezza della fenditura ($\frac{\Delta D}{D} = 7\%$). Possiamo dunque dirsi
verificate le leggi sulla diffrazione per una singola fenditura nel limite di Fraunhofer.

\section{Seconda parte dell'esperienza: il reticolo di diffrazione}
\subsection{Scopo dell'esperienza ed ipotesi}
La seconda parte dell’esperienza mira a verificare che la posizione angolare dei massimi
principali di diffrazione segua la regola data dalla seguente relazione:
\begin{align}
        \sin\theta = m\frac{\lambda}{a} \quad m \in \mathbb{Z}
\end{align}
E verificare dunque l'equazione (2). \\
In questa relazione:
\begin{itemize}
    \item $\theta$: Angolo di diffrazione;
    \item $\lambda$: la lunghezza d’onda;
    \item $m$: ordine di diffrazione; 
    \item $a$: il passo del reticolo.
\end{itemize}


\subsection{Schema generale di misura}
Per trovare il passo del reticolo a partire dalla separazione angolare delle bande 
luminose, si utilizza la seguente relazione:
\begin{align}
    a = \frac{m\lambda}{\sin\theta} \qquad m \in \mathbb{Z}
\end{align}
L’errore relativo ad $a$ è dato dalla seguente espressione ottenuta dalla propagazione degli errori:
\begin{align}
    \Delta a = a \left| m \right| \left( \Delta \lambda \left| \frac{1}{\lambda} \right| + \Delta \theta\left| -\frac{\cos\theta}{\sin\theta} \right|   \right)
\end{align}
Di seguito viene descritta la corretta procedura per la calibrazione
dell’apparato sperimentale: inizialmente, si cerca di ottimizzare al
massimo il sistema ottico dello sperimentatore regolando l’oculare
finché il crocefilo non risulta a fuoco. Puntando poi il cannocchiale
verso un oggetto lontano, si muove l’intero sistema crocefilo-oculare
per rendere l’immagine nitida. In questo modo il cannocchiale
risulterà correttamente focalizzato all’infinito.
Si posiziona, in seguito, la fenditura ad una distanza pari alla
lunghezza focale della lente: in questo modo la luce emessa
dalla lampada sarà perfettamente collimata. Quindi, osservando
la fenditura dal cannocchiale mobile appena regolato, si cerca
la distanza ottimale per cui essa apparirà nitida. La luce è ora
assimilabile ad un’onda piana.
Prima di effettuare le misure effettive è necessario posizionare
il reticolo sullo spettrogoniometro e verificare che la direzione di propagazione
dell’onda piana sia ortogonale al piano del reticolo.
Lo sperimentatore può adesso avvicinare l’occhio con cui ha calibrato
l’intero sistema ottico al cannocchiale per osservare l’effetto di
diffrazione prodotto dal reticolo; noterà che, spostando il
cannocchiale, appariranno molteplici fenditure di colore diverso
(fenomeno verificatosi a causa della composizione spettrale della
lampada a Mercurio). Posizionando approssimativamente il crocefilo
sul centro della fenditura interessata, e servendosi della
vite tangenziale, si misurerà l’angolo di diffrazione leggendo la scala dello spettrogoniometro.
Durante questo processo lo sperimentatore noterà che certe lunghezze
d’onda saranno meno vivide di altre; è consigliato prendere più di
una misura per le suddette, farne la media e calcolare lo scarto
massimo.

\subsection{Apparato sperimentale}
\begin{figure}[H]
    \centering
    \includegraphics[width=0.4\textwidth]{goniometrico.png}
\end{figure}
Per la seconda parte dell'esperienza si utilizza uno spettrogoniometro,
strumento di misura che permette di misurare gli angoli
di deviazione o diffrazione della luce con una sensibilità di 30”. Su di esso è posto
un reticolo di diffrazione di 600 linee/mm nominali. Per quanto riguarda
la sorgente luminosa, viene utilizzata una
lampada a Mercurio (Hg) capace di produrre
colori puri a lunghezze d’onda molto precise:
404.656 nm (viola), 407.781 nm (violetto),
435.881 nm (blu), 491.604 nm (verde scuro), 546.074 nm (verde chiaro),
576.959 nm (giallo), 579.065 nm (giallo scuro).
La fenditura è posta tra la lampada e il cannocchiale fisso;
l’onda piana prodotta da essa, incidendo sul reticolo, è 
diffratta ad angoli diversi secondo la legge sopracitata ed è
proiettata sull'oculare del cannocchiale mobile, usato
dallo sperimentatore per misurare gli angoli di
diffrazione.


\subsection{Misure}
Per effettuare la misura dell’angolo di diffrazione,
esistono due approcci: 
\begin{enumerate}
    \item Il metodo più semplice ma meno preciso
    consiste nel misurare la posizione angolare del massimo
    centrale (ordine 0, $\alpha_0$) e la posizione della riga
    spettrale di interesse (ordine 1, $\alpha_1$), per cui si otterrebbe 
    l'angolo di diffrazione come:
    \begin{align}
        \theta_d = \left| \alpha_0 - \alpha_1 \right| 
    \end{align}
    \item Il “metodo della semidifferenza”, invece, sfrutta la
    simmetria del fenomeno della diffrazione. La procedura consiste
    nel rilevare la posizione angolare della stessa riga spettrale
    (medesimo colore) in modo simmetrico rispetto all'asse ottico:
    prima in corrispondenza dell'ordine positivo ($n=+1$, $\alpha_+$)
    e successivamente per l'ordine negativo ($n=-1$, $\alpha_-$).
    L'angolo di diffrazione $\theta$ viene quindi determinato
    calcolando la semidifferenza tra le due coordinate angolari
    registrate
    \begin{align}
        \theta_d = \frac{\left| \alpha_0 - \alpha_1 \right| }{2}
    \end{align}
\end{enumerate}
Non è consigliato usare il primo metodo visto che si potrebbero introdurre
errori sistematici associati alla ricerca del massimo centrale, spesso non
individuabile con alta precisione siccome molto luminoso e largo.
Per effettuare i calcoli abbiamo deciso di usare il secondo al fine di
minimizzare gli errori sistematici. Infatti, in nessun caso è richiesto
di centrare il crocefilo con un colore che abbia un’intensità alta quanto
quella bianca al centro. \\ Per le lunghezze d'onda verde scuro e violetto si sono 
inoltre compiute 5 misure per poter determinare la loro media poiché sono 
le lunghezze d'onda meno visibili e dunque quelle più soggette ad errore. 
L'errore sulla media di queste due lunghezze d'onda è utilizzato
come errore sulle misure per $\alpha_+$ e $\alpha_-$. \\Di seguito si riportano le misure per 
le due lunghezze d'onda:
\begin{gather*}
    \begin{tabular}{l c c c c c}
        \hline
        Colore & Misura 1 & Misura 2 & Misura 3 & Misura 4 & Misura 5\\ 
        \hline
        Verde scuro $\alpha_+$ & 127° 5' & 127° 4' 30'' & 127° 5' & 127° 4' 30 '' & 127° 4' 30'' \\
        Verde scuro $\alpha_-$ & 92° 39' & 92° 38' 30'' & 92° 38' & 92° 38' 30'' & 92° 38' \\
        Violetto $\alpha_+$ & 124° 5' 30'' & 124° 4' 30'' & 124° 5' & 124° 5' 30''& 124° 5' \\
        Violetto $\alpha_-$ & 95° 40' & 95° 39' &95° 38' 30''&95° 39' 30''&95° 40' \\
        \hline
    \end{tabular}
\end{gather*}
Di seguito la tabella riassuntiva per ogni lunghezza d'onda. Per i valori del
verde scuro e del violetto si sono utilizzate le medie delle misure sopraelencate:
\begin{gather*}
    \begin{tabular}{l c c c}
        \hline
        Colore & $\lambda$ $10^{-10}$ m & $\alpha_+$ & $\alpha_-$ \\
        \hline 
        Giallo scuro & 5790.65 & 130° 15' & 89° 24' 30''\\
        Giallo &5769.59 & 130° 10'& 89° 28' 30'' \\
        Verde chiaro & 5460.74 & 129° 2' & 90° 39' \\
        Verde scuro & 4916.04 & 127° 4' 42'' & 92° 38' 12'' \\
        Blu & 4358.81 & 125° 5' 30 '' & 94° 40' \\
        Violetto & 4077.81 & 124° 5' 6''&  95° 39' 24'' \\
        Viola & 4046.56 & 123° 59' 30'' & 95° 46' 30'' \\
        \hline
    \end{tabular}
\end{gather*}
Da queste misure, si sono utilizzate le relazioni per ottenere i seguenti risultati
\begin{gather*}
    \begin{tabular}{l c c}
        \hline
        Colore & $a \pm \Delta a$ $10^{-10}$ m & $N \pm \Delta$N \\
        \hline
        Giallo scuro & $16596 \pm 12$  &  $602.5 \pm 0.4$\\
        Giallo & $16594 \pm 12$ & $602.6 \pm 0.4$ \\
        Verde chiaro & $16611 \pm 13 $ & $602.0 \pm 0.5$ \\
        Verde scuro & $16605 \pm 14 $ & $602.2 \pm 0.5$ \\
        Blu & $16611 \pm 16$  & $602.0 \pm 0.6 $  \\
        Violetto &  $16607 \pm 17$ & $602.1 \pm 0.6$  \\
        Viola & $16661 \pm 17$ & $602.4 \pm 0.6$   \\
        \hline
    \end{tabular}
\end{gather*}

\clearpage
\subsection{Commento qualitativo}
\begin{figure}[H]
    \centering
    \includesvg[width=1\textwidth]{diffrazione-goniometrico.svg}
\end{figure}
Dal grafico sopra si evince che esiste compatibilità tra le misure indirette della spaziatura
tra le fenditure del reticolo di diffrazione utilizzato. Svolgendo la media pesata delle misure
di $a$, si ottiene un valore di $a$ più preciso per il nostro set di dati:
\begin{gather*}
    a = (16604 \pm 5) \cdot 10^{-10} \ \text{m}
\end{gather*}
Anche se questa media pesata differisce di oltre $12\sigma_a$ dal valore 
dato dal costruttore di $16666.66 \ 10^{-10}$ m, possiamo, data la compatibilità
delle nostre misure, dire verificate la legge di diffrazione da reticolo. 

\end{document}
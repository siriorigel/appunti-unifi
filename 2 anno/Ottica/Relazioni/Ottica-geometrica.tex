\documentclass[a4paper, oneside]{article}
\usepackage{graphicx}
\usepackage{amsthm}
\usepackage{amsmath}
\usepackage{amssymb}
\usepackage[a4paper,
            bindingoffset=0.2in,
            left=2cm,
            right=2cm,
            top=2cm,
            bottom=2cm,
            footskip=.25in]{geometry}
\usepackage[italian]{babel}
\usepackage{pgfplots}
\usepackage{tabularx}
\usepackage{tikz}
\usepackage{wrapfig}
\usepackage{color}
\usepackage[d]{esvect}
\usepackage{chemfig}
\usepackage{mhchem}
\usepackage{svg}
\usepackage{float}
%\definecolor{page}{rgb}{0.129,0.157,0.212}
%\pagecolor{page}
%\color{white}
\graphicspath{ {./images/} }
\usetikzlibrary{shapes.geometric}
\usetikzlibrary{datavisualization}
\usetikzlibrary{datavisualization.formats.functions}
\usetikzlibrary{patterns}
\pgfplotsset{width=10cm,compat=1.18}

\title{Esperienza Ottica Geometrica}
\author{Gruppo 19 \\ Fabbri Marco, Miliani Tommaso, Mongatti Giulio, Tinacci Lorenzo}
\date{3 Dicembre 2025}

\begin{document}
\newtheoremstyle{theoremEnv}
                {}          % Space above
                {}          % Space below
                {\slshape}  % Body font
                {}          % Indent amount
                {\bfseries} % Head font
                {.}         % Punctuation after head
                {\newline}  % Space after theorem head
                {}          % Theorem head spec
\theoremstyle{theoremEnv}

\newtheorem{definition}{Definizione}[section]
\newtheorem{theorem}{Teorema}[section]
\newtheorem{lemma}{Proposizione}[section]
\newtheorem{observation}{Osservazione}[section]
\newtheorem{corollary}{Corollario}[theorem]
\newtheorem{example}{Esempio}[section]
\newtheorem{remark}{Enunciato}[section]

\maketitle

\section{Scopo dell'esperienza e ipotesi}
Lo scopo dell’esperienza è la verifica della legge delle lenti sottili 
\begin{align}
    \frac{1}{f} = \frac{1}{p} + \frac{1}{q}
\end{align}
dove
\begin{itemize}
    \item $f$: lunghezza focale della lente;
    \item $p$: distanza della sorgente dalla lente;
    \item $q$: distanza dell'immagine dalla lente.
\end{itemize}
Inoltre si vuole verificare la misura della lunghezza focale incognita di una lente convergente biconvessa.
Per la misura e la verifica sperimentale, vengono poste le seguenti ipotesi di lavoro:
\begin{itemize}
    \item spessore delle lenti trascurabile (molto minore di $p$ e $q$);
    \item asse ottico parallelo al piano di lavoro.
\end{itemize}

\section{Schema generale di misura}
Per la verifica delle leggi delle lenti sottili, ognuno dei quattro sperimentatori che compongono il gruppo di lavoro effettua la misura
della focale della lente per una coppia di valori $p$ e $q$ distinti. Se le quattro misure risultano consistenti
entro l’incertezza, allora la legge risulta verificata. Da una media pesata dei quattro valori, è infine
possibile ricavare una misura più precisa della focale della lente. 
L'esperienza richiede sapere la posizione della sorgente luminosa, tuttavia, poiché non si ha un metodo
convenzionale per capire dove si origina tale sorgente all'interno del supporto che la contiene, si opta,
al fine di semplificare la misura, di usare una lente di focale nota per poter creare una sorgente virtuale
ponendola nella configurazione $2f- 2f$. Nell'esperienza si è usata una lente avente focale $f$ di circa 50 mm, 
quindi, per porla nella configurazione $2f -2f$, l'abbiamo posizionata a circa 10 cm dalla sorgente reale così
da poter creare un immagine virtuale a circa la stessa distanza.  Alla posizione della sorgente virtuale 
si assegna il nome di $I_1$.
Gli sperimentatori utilizzano il metodo di Gauss, che consiste nel ricondurre la misura della focale ignota
alle due grandezze misurabili in laboratorio, ossia la distanza $s$ tra la sorgente virtuale e
l’immagine della sorgente formata dalla lente, e la distanza $l$ tra le due posizioni della lente che
permettono all’immagine della sorgente di formarsi nello stesso punto.  In questo modo la focale
può essere ottenuta mediante la seguente relazione matematica:
\begin{align}
    f = \frac{s^{2} - l^{2}}{4s}
\end{align}

\clearpage
\section{Apparato sperimentale}
\begin{gather*}
     \begin{tikzpicture}[scale=1.2]
        \draw(0, 0) -- (7.5,0);
        \draw[dashed](0 ,1.25) -- (6.5, 1.25) node[at start, above] {\tiny asse ottico};
        \draw[dashed](6.5, 1.25) -- (7.1, 1.25);
        \draw(0.5, 0) -- (0.5, 1.5);
        \draw[thick](0.5, 1) -- (0.5, 1.5); 
        \draw[|-|](0.5, 1.7) -- (1.5, 1.7) node[midway, above] {$d$};
        \draw(1.2, 0) rectangle (1.8, 0.4);
        \draw(1.4, 0.4) rectangle (1.6, 1);
        \draw(1.5, 1.25) ellipse (0.1 and 0.25);
        \draw(0.5 , 1.25) -- (1.5, 1.5) -- (2.5, 1.25) -- (4.25, 0.9) -- (6, 1.25);
        \draw(0.5, 1.25) -- (1.5, 1) -- (2.5, 1.25) -- (4.25, 1.6) -- (6, 1.25);
        \filldraw(2.5, 1.25) circle (1pt) node[anchor = north, align = center] { \tiny sorgente};
        \node at (2.5, 0.9) {\tiny virtuale};
        \draw(3.95, 0) rectangle (4.55, 0.4);
        \draw(4.15, 0.4) rectangle (4.35, 0.9);
        \draw(4.25, 1.25) ellipse (0.15 and 0.35);
        \draw[|-|](2.5, 1.7) -- (4.2, 1.7) node[midway, above] {$p$};
        \draw[|-|](4.3, 1.7) -- (6, 1.7) node[midway, above] {$q$};
        \draw(7.1, 1) rectangle (7.9, 1.5);
        \draw(7.4, 1) rectangle (7.6, 0);
        \node at (7.5, 1.75) {\tiny laser};
    \end{tikzpicture}  
\end{gather*}
L'apparato sperimentale presenta una sorgente luminosa, una lente con focale incognita, una lente di servizio con focale nota e oculare per la misura
della posizione di sorgente ed immagine. Il tutto è posizionato su di una guida rettilinea munita di scala graduata
con sensibilità di un millimetro. Grazie ad un nonio posto sui vari componenti ottici, è possibile determinare la loro
posizione con una precisione di 0.1 mm. Qui di seguito elenchiamo le misure effettuate dopo opportuno
centraggio di tutti i componenti ottici rispetto allo stesso asse, parallelo alla guida, detto asse ottico.
\begin{itemize}
    \item $I_1$: posizione della sorgente virtuale misurata mediante oculare.
    \item $I_2$: posizione dell’immagine misurata mediante oculare.
    \item $L_1$ e $L_2$: le due posizioni della lente di focale incognita che fanno convergere i raggi luminosi 
    emessi da $I_1$ alla posizione $I_2$, dove è posto l'oculare, scelta da ogni sperimentatore. 
    \item $s$: differenza tra $I_2$ e $I_1$.
    \item $l$: differenza tra $L_2$ e $L_1$.
\end{itemize}
Di seguito si riporta uno schema più generale dell'apparato ottico con 
la raffigurazione delle misure effettuate:
    \begin{figure}[H]
        \centering
        \includegraphics[width=0.75\textwidth]{apparato-ottica.png}
    \end{figure}

\clearpage
\section{Misurazioni}
Per la determinazione delle varie posizioni $I_1$, $L_1$, $L_2$, ogni sperimentatore, tranne il primo,
che ne ha prese solo 5, ha eseguito 7 misure di ognuna
di queste grandezze per poi determinarne media e scarto massimo.
Inoltre, per ogni sperimentatore, si è fissata una posizione di $I_2$ (a cui si associa 
come errore quello di sensibilità del nonio di 0.01 cm).  Prima di effettuare le misure,
ogni sperimentatore, guardando nell'oculare con il fuoco dell'occhio all'infinito, ha regolato
l'oculare in modo tale da mettere a fuoco il crocefilo in esso contenuto. Per tutta la durata delle
misurazioni ciascun sperimentatore deve fare attenzione a non modificare la messa a fuoco
dell'oculare toccandolo inavvertitamente.
Nelle tabelle seguenti riportiamo le misure effettuate dai quattro sperimentatori per le grandezze menzionate:
\begin{gather*}
    \text{Mongatti} \qquad \qquad \qquad \qquad \qquad \qquad \quad \text{Fabbri}\\
    \begin{tabular}{ c  c  c }
        \hline
        $I_1$ (cm) & $L_1$ (cm) & $L_2$ (cm)\\
        \hline
        24.96 & 42.83 & 89.96 \\
        24.98 & 42.86 & 90.08\\
        25.14 & 42.85 & 90.09\\
        25.13 & 42.88 & 90.03\\
        24.94 & 42.97 & 89.99 \\
        \hline
    \end{tabular} \qquad \begin{tabular}{ c  c  c }
        \hline
        $I_1$ (cm) & $L_1$ (cm) & $L_2$ (cm)\\
        \hline
        24.83 & 44.68& 72.84 \\
        24.89& 44.93 & 72.89 \\
        24.89& 44.67& 72.83\\
        24.96& 45.12& 72.85 \\
        24.96& 44.70& 72.85 \\
        24.88& 44.99& 72.92 \\
        24.87& 44.95& 72.91 \\
        \hline
    \end{tabular} \\
    \text{Miliani} \qquad \qquad \qquad \qquad \qquad \qquad \text{Tinacci}\\
    \begin{tabular}{c  c  c}
        \hline
        $I_1$ (cm) & $L_1$ (cm) & $L_2$ (cm)\\
        \hline
        24.87	& 43.69	& 78.85 \\
        24.78	&43.88	&78.83 \\
        24.81	&43.96	&78.70 \\
        24.91	&43.89	&78.75 \\
        24.93	&43.96	&78.79 \\
        24.92	&43.84	&78.82 \\
        25.01	&43.92	&78.77 \\
        \hline
    \end{tabular}  \qquad \begin{tabular}{c c  c}
        \hline
        $I_1$ (cm) & $L_1$ (cm) & $L_2$ (cm)\\
        \hline
        24.66&	43.33	&83.53 \\
        24.68&	43.34	&83.52 \\
        24.66&	43.39	&83.54 \\
        24.64&	43.36	&83.49 \\
        24.69&	43.41	&83.53 \\
        24.70 &  43.37	&83.49 \\
        24.71&	43.32	&83.53 \\
        \hline
    \end{tabular}
\end{gather*}
Da questi valori si sono ricavati media ed errore associato, tramite propagazione
degli errori, per le misure indirette $s$ ed $l$. Si riassumono, inoltre, tutte le misure
per ciascun sperimentatore:
\begin{gather*}
    \overline{s} = \overline{I_2} - \overline{I_1}\qquad\quad \Delta s = \Delta I_2 + \Delta I_1  \\
    \overline{l} = \overline{L_2} - \overline{L_1}\qquad\quad \Delta l = \Delta L_2 + \Delta L_1   
\end{gather*}
\begin{gather*}
    \begin{tabular}{c  c  c  c  c  c  c }
        \hline
        Sp. & $I_2$ (cm) & $\overline{s}  \pm \Delta s$ (cm) & $\overline{l} \pm \Delta l$ (cm) & $\overline{I_1}  \pm \Delta I_1$ (cm) & $\overline{L_1}  \pm \Delta L_1$ (cm)& $\overline{L_2}  \pm \Delta L_2$ (cm)\\
        \hline 
        Mongatti & $110.00$ & $84.97 \pm 0.13$ & $47.15 \pm 0.16$ & $25.03 \pm 0.12$ & $42.88 \pm 0.09$ & $90.03 \pm 0.07$ \\
        Fabbri  & $95.00$ & $70.10 \pm 0.08$ & $28.00 \pm 0.35$ & $24.90 \pm 0.07$ & $44.86 \pm 0.30$ & $72.87 \pm 0.05$ \\
        Miliani & $100.00$ & $75.11 \pm 0.13$ & $34.91 \pm 0.28$ & $24.89 \pm 0.12$ & $43.88 \pm 0.19 $ & $78.79 \pm 0.09$ \\
        Tinacci & $104.00$ & $79.32 \pm 0.05$ & $40.16 \pm 0.08$ & $24.68 \pm 0.04$	& $43.36 \pm 0.05$ & $83.52 \pm 0.03$ \\
        \hline
    \end{tabular}
\end{gather*}
Dall'equazione di Gauss, inoltre, si è trovato l'incertezza sulla focale per ogni sperimentatore utilizzando
la propagazione dell'errore con le derivate parziali:
\begin{align}
    \Delta f = \left| \frac{1}{4} + \frac{l^{2}}{4s^{2}} \right|\Delta s + \left| \frac{l}{2s} \right|\Delta l  .
\end{align}
In questo modo si sono ottenute le seguenti stime per la lunghezza focale 
della lente incognita per ogni sperimentatore con relativo errore:
\begin{gather*}
    \begin{tabular}{c  c  c  c  c}
        \hline
        & Mongatti & Fabbri & Miliani & Tinacci \\
        \hline
        $f \pm \Delta f $ (cm)& $14.70 \pm 0.09$ & $14.73 \pm 0.09$ &$14.72 \pm 0.10$ &$14.75 \pm 0.04$ \\
        \hline
    \end{tabular}
\end{gather*}

\clearpage
\section{Commento qualitativo}
Si riportano graficamente le misure della focale incognita per
ogni sperimentatore per verificarne la consistenza:
    \begin{figure}[H]
        \centering
        \includesvg[width=0.75\textwidth]{compatibilita-focali.svg}
    \end{figure}
Dal grafico si osserva che le misure della focale per la lente incognita risultano consistenti tra di loro.
Si può tuttavia notare che una delle misure presenta un errore sensibilmente minore rispetto alle altre, 
cosa che supponiamo esser dovuta ad un utilizzo meno prudente dell'oculare da parte 
degli altri sperimentatori. Meno cura nel maneggiare l'oculare durante le misurazioni
potrebbe aver accidentalmente modificato la messa a fuoco dello strumento, portando ad
una fluttuazione dei valori misurati che potrebbe essere stata fonte di un maggiore errore sulla focale. Appurata la compatibilità delle misure,
si è ricavata la nostra misura per la focale della lente incognita tramite media pesata, con relativo errore:
\begin{gather*}
    f = (14.72 \pm 0.03) \cdot  \text{cm}
\end{gather*}
Si ritenga dunque sperimentalmente verificata la legge delle lenti sottili. 
La nostra misura per la focale della lente incognita dista $\sim2.2 \sigma_f$ dal valore dato dal costruttore di $15.00$ cm, il quale
non risulta dunque essere un valore accurato. 

\end{document}
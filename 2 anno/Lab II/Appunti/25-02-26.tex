\documentclass[a4paper, oneside]{article}
\usepackage{graphicx}
\usepackage{amsthm}
\usepackage{amsmath}
\usepackage{amssymb}
\usepackage[a4paper,
            bindingoffset=0.2in,
            left=2cm,
            right=2cm,
            top=2cm,
            bottom=2cm,
            footskip=.25in]{geometry}
\usepackage[italian]{babel}
\usepackage{pgfplots}
\usepackage{tabularx}
\usepackage{tikz}
\usepackage{wrapfig}
\usepackage{color}
\usepackage[d]{esvect}
\usepackage{circuitikz}
\definecolor{page}{rgb}{0.129,0.157,0.212}
\pagecolor{page}
\color{white}
\graphicspath{ {./images/} }
\usetikzlibrary{shapes.geometric}
\usetikzlibrary{datavisualization}
\usetikzlibrary{datavisualization.formats.functions}
\usetikzlibrary{patterns}
\pgfplotsset{width=10cm,compat=1.18}

\title{Appunti di lab II}
\author{Tommaso Miliani}
\date{25-02-26}

\begin{document}
\newtheoremstyle{theoremEnv}
                {}          % Space above
                {}          % Space below
                {\slshape}  % Body font
                {}          % Indent amount
                {\bfseries} % Head font
                {.}         % Punctuation after head
                {\newline}  % Space after theorem head
                {}          % Theorem head spec
\theoremstyle{theoremEnv}

\newtheorem{definition}{Definizione}[section]
\newtheorem{theorem}{Teorema}[section]
\newtheorem{lemma}{Proposizione}[section]
\newtheorem{observation}{Osservazione}[section]
\newtheorem{corollary}{Corollario}[theorem]
\newtheorem{example}{Esempio}[section]
\newtheorem{remark}{Enunciato}[section]

\maketitle

\section{Legge di Ohm e calcolo della velocità di deriva}
La legge di Gauss in forma locale prende la forma di 
\begin{gather*}
    \vv{J} = \sigma_c \vv{R}  
\end{gather*}
Ossia vale punto per punto nel conduttore. Sia $\vv{J}$ che $\vv{E}$ sono Sia
funzione dello spazio che del tempo. Il termine $\sigma_d$ è chiamato \textbf{conducibilità 
elettrica}. Equivalentemente si può scrivere la legge di Ohm nella 
seguente maniera
\begin{align}
    \vv{E} = \frac{1}{\sigma_c} \vv{J}  = \rho_r \vv{J} 
\end{align}
dove $\rho_r$ prende il nome di \textbf{resistività elettrica}.
\begin{example}[Calcolo della velocità di deriva]
    Ponendo di avere una corrente di un ampere che passa attraverso un filo di raggio
    un millimetro. Dunque la sezione $S = \pi r^{2} = \pi$ mm$^{2}$. Dunque la corrente
    \begin{gather*}
        I = 1 \ A = n q \cdot v_d \cdot S  \ \Longrightarrow \ v_d = 2.5 \cdot 10^{-5} \frac{m}{s}
    \end{gather*} 
\end{example}
Tramite le relazioni descritte si vuole determinare le leggi di Kirchhoff e le leggi di Ohm. Per poter 
arrivare alla legge di Ohm si deve introdurre il concetto di potenziale.
\subsection{Il potenziale elettrico}
La carica in fisica si conserva sempre: sia $S$ una superficie chiusa all'interno della quale 
vi è una carica $Q(t)$ che varia nel tempo di una quantità $-dQ$. Necessariamente questa deve essere uscita 
dalla superficie. Questa variazione è espressa dal flusso attraverso questa superficie che ha 
causato la variazione di carica
\begin{gather*}
    -dQ = \int_{S} \vv{J} \cdot \vv{dS} \ dt   
\end{gather*}
Posso dunque dividere per $dt$ ottenendo la derivata totale 
\begin{gather*}
    -\frac{dQ(t)}{dt} = \int_{S} \vv{J} \cdot \vv{dS}   
\end{gather*}
Non si sa come la densità di carica sia distribuita, ma è possibile ricavare 
\begin{gather*}
    Q(t) = \int_{V} \rho(x, y, z, t) \ dx \ dy \ dz  
\end{gather*}
Derivando ora sia a destra che a sinistra, si ottiene 
\begin{gather*}
    \frac{dQ(t)}{dt} = \int_{V} \frac{\partial\rho(x, y, z, t)}{\partial t} \ dx \ dy \ dz
\end{gather*}
Utilizzando ora il teorema della divergenza, si può uguagliare le espressioni trovate ottenendo 
la seguente espressione 
\begin{gather*}
    -\int_{V} \frac{\partial \rho}{\partial t} \ dV = \int_{S} \vv{J} \cdot \vv{dS} = \int_{V} \vv{\nabla} \cdot \vv{J} \ dV   
\end{gather*}
È possibile definire questa relazione per qualsiasi volume o superficie, dunque in questa situazione 
necessariamente gli integranda devono essere equivalenti:
\begin{gather*}
    - \frac{\partial \rho}{\partial t} = \vv{\nabla} \cdot \vv{J}   
\end{gather*}
Ottenendo l'\textbf{equazione di continuità}:
\begin{align}
    \vv{\nabla} \cdot \vv{J} + \frac{\partial \rho}{\partial t} = 0   
\end{align}
In condizioni stazionarie $\vv{J}$ è solenoidale e dunque la sua divergenza 
è nulla (poiché la derivata parziale rispetto al tempo è nulla nell'equazione), dunque
vuol dire che (usando il teorema di divergenza al contrario):
\begin{gather*}
    \int_{S} \vv{J} \cdot \vv{dS} = 0   
\end{gather*} 
Su una qualsiasi superficie chiusa tanta corrente entra tanta ne deve uscire. 

\section{Leggi di Kirchhoff}
\subsection{Prima legge Kirchhoff}
La prima legge di Kirchhoff afferma che la somma algebrica delle correnti uscenti da un nodo 
è nulla:
\begin{align}
    \int_{S} \vv{J} \cdot \vv{n} dS = 0  
\end{align}
Un \textbf{nodo} è un punto di incontro tra due o più conduttori 
nel quale fluiscono un certo numero di correnti. La normale in questione 
in ogni conduttore è sempre uscente per convenzione. 

\subsection{Seconda legge di Kirchhoff}
In un circuito si definiscono i rami, le maglie e un insieme di maglie prende il 
nome di \textbf{rete}. Un \textbf{ramo} è il conduttore che 
unisce due particolari nodi. Una \textbf{maglia} è una successione chiusa
di rami. Questa legge è legata alla conservatività del campo: ossia il fatto che
e dunque l'integrale su di una linea chiusa 
del campo è nullo.  
\begin{align}
    \vv{\nabla} \times \vv{E} = 0
\end{align}

\section{Il potenziale elettrostatico e la terza legge di Kirchhoff}
Il \textbf{potenziale elettrostatico} si definisce nella seguente maniera. 
Il campo elettrico in un dato punto nello spazio è dato dalla seguente
\begin{gather*}
    \vv{E}(\vv{r} ) = \frac{Q}{4\pi\epsilon_0} \frac{1}{r^{2}}\hat{r}  
\end{gather*}
Presa una carica di prova $Q$ che compie un certo percorso lungo la linea $\gamma$. Il 
lavoro del campo elettrico lungo $\gamma$ è dato dalla seguente
\begin{gather*}
    L_{\gamma} = \int_{\gamma} \vv{F} \cdot \vv{dl} \qquad \vv{dl} = dr\hat{r} +rd\theta \hat{\theta} + r\sin\theta \hat{\phi}d\phi   
\end{gather*} 
Dunque sopravvive solo il primo termine:
\begin{gather*}
    \int q\vv{E} \cdot \vv{dl} = \frac{qQ}{4\pi\epsilon_0} \int_{r_A}^{r_B} \frac{1}{r^{2}} \hat{r} \cdot \hat{r} \ dr  = - \frac{qQ}{4\pi\epsilon_0}\frac{1}{r}   
\end{gather*}
Dunque il lavoro su $\gamma$ è equivalente alla seguente:
\begin{gather*}
    L_{\gamma} = - \frac{qQ}{4\pi\epsilon_0}\left(\frac{1}{r_A} - \frac{1}{r_B}\right)
\end{gather*}
Dunque si vede che c'è una dipendenza dal lavoro di $\frac{1}{r}$. Si definisce il 
potenziale del campo come 
\begin{gather*}
    \lim_{q \to 0} \frac{U}{q} = \boxed{V(r) = \frac{Q}{4\pi\epsilon_0} \frac{1}{r} + \text{const}} 
\end{gather*}
Il potenziale è sempre definito a meno di una costante poiché non esiste un potenziale 
assoluto ma si definiscono le differenze di potenziale. Dunque la differenza di potenziale 
in questo caso sarà dato dalla seguente espressione
\begin{gather*}
    V_A - V_B = \int_{A}^{B} \vv{E} \cdot \vv{dl}   
\end{gather*}
Dunque questo è l'effetto della presenza di un campo nello spazio circostante. Dunque ciò che si misura è
sempre l'effetto dell'integrale. Il potenziale ha come unità di misura 
\begin{gather*}
    [V] = \frac{[U]}{[q]} = \frac{J}{C} = V \ (\text{Volt})
\end{gather*}
Dunque su di una linea chiusa si ha che
\begin{gather*}
    \oint \vv{E} \cdot \vv{dl} = 0  
\end{gather*}
Dunque
\begin{gather*}
    \sum_{k} \Delta V_k = 0 
\end{gather*}
in condizioni stazionarie, ottenendo la \textbf{legge delle maglie}, ossia la terza legge di Kirchhoff. 

\section{La resistenza}
Si è gi definita la resistività a livello locale:
\begin{gather*}
    \vv{E} = \rho_R \cdot \vv{J}  
\end{gather*}
che ha dimensioni 
\begin{gather*}
    [\rho_R] = \left[\frac{N}{C} \frac{m^{2}}{A}\right] = \left[\frac{Jm}{CA}\right] = \left[V\frac{m}{A}\right] = [\Omega \cdot m]
\end{gather*}
La resistività si misura dunque in $\Omega \cdot m$ "Ohm per metro" e cambia radicalmente 
tra materiale e materiale. Si riportano i valori di resistività di alcuni materiali:
\begin{gather*}
    \begin{tabular}{l | c | c | c }
        Materiale & Ferro & Rame & Ceramica \\
        \hline
        Resistività & $10 \cdot 10^{-8}$ & $1.7 \cdot 10^{-8}$ & $10^{16}$
    \end{tabular}
\end{gather*}
Nella corrente ad alta tensione si potrebbe pensare che la corrente passi solamente nei fli, 
tuttavia, dato che i fili sono sorretti in qualche modo, e la corrente sceglie sempre 
il percorso di minor resistenza, anche se il materiale che attacca al nodo è isolante
è possibile che possa comunque passare corrente se si riesce a dare una differenza di potenziale
grande vicino al traliccio.  \\ \noindent
I fulmini cadono poiché si è creata una zona di minore resistenza tra la nuvola e il terreno. Generalmente 
la differenza di potenziale dell'atmosfera è 200 Volt per metro, dunque se la corrente trova un percorso a minor resistenza
allora essa passa indisturbata.  \\ \noindent
Si arriva dunque alla legge di Ohm prendendo un conduttore a sezione
costante tra $A$ e $B$, dunque posso valutare
\begin{gather*}
    \int_{A}^{B} \vv{E} \cdot \vv{dl} = V_A - V_B = \int_{A}^{B} \rho_R \vv{J} \cdot \vv{dl} = \int_{A}^{B} \rho_R \frac{\vv{J}S }{S} \hat{n} \cdot dl 
\end{gather*}
Si era definita la corrente che passa all'interno della sezione come
\begin{gather*}
    \vv{J} \cdot S \hat{n} = I  
\end{gather*}
che è valida solo a livello locale poiché, così come per i fluidi, 
la corrente scorre più velocemente al centro della sezione, mentre ai 
bordi la velocità è nulla.
Dunque si esprime la differenza di potenziale come segue:
\begin{gather*}
    V_A - V_B = I \int_{A}^{B} \frac{\rho_R}{S} \ dl 
\end{gather*}
Il termine integrato si indica con $R$ ed ha come dimensioni $\Omega$, dunque si ottiene la seguente equazione
\begin{align}
     \Delta V = IR
\end{align}
Ottenendo dunque la legge di Ohm. Un oggetto che ha la proprietà di creare resistenza
prende il nome di \textbf{resistore} e si indica così:




\end{document}
\documentclass[a4paper, oneside]{article}
\usepackage{graphicx}
\usepackage{amsthm}
\usepackage{amsmath}
\usepackage{amssymb}
\usepackage[a4paper,
            bindingoffset=0.2in,
            left=2cm,
            right=2cm,
            top=2cm,
            bottom=2cm,
            footskip=.25in]{geometry}
\usepackage[italian]{babel}
\usepackage{pgfplots}
\usepackage{tabularx}
\usepackage{tikz}
\usepackage{wrapfig}
\usepackage{color}
\usepackage[d]{esvect}
\definecolor{page}{rgb}{0.129,0.157,0.212}
\pagecolor{page}
\color{white}
\graphicspath{ {./images/} }
\usetikzlibrary{shapes.geometric}
\usetikzlibrary{datavisualization}
\usetikzlibrary{datavisualization.formats.functions}
\usetikzlibrary{patterns}
\pgfplotsset{width=10cm,compat=1.18}

\title{Appunti di lab II}
\author{Tommaso Miliani}
\date{23-02-26}

\begin{document}
\newtheoremstyle{theoremEnv}
                {}          % Space above
                {}          % Space below
                {\slshape}  % Body font
                {}          % Indent amount
                {\bfseries} % Head font
                {.}         % Punctuation after head
                {\newline}  % Space after theorem head
                {}          % Theorem head spec
\theoremstyle{theoremEnv}

\newtheorem{definition}{Definizione}[section]
\newtheorem{theorem}{Teorema}[section]
\newtheorem{lemma}{Proposizione}[section]
\newtheorem{observation}{Osservazione}[section]
\newtheorem{corollary}{Corollario}[theorem]
\newtheorem{example}{Esempio}[section]
\newtheorem{remark}{Enunciato}[section]

\maketitle

\section{Introduzione}
Informazioni generali:
\begin{itemize}
    \item Prof: Piergiulio Lenzi e Nicola Poli
    \item Libri: "Esperimenti di elettricità e magnetismo", di Poggi; 
    Carlà "Appunti di elettronica per fisici". 
    \item Esame: 4 esperienze in laboratorio:
    \begin{itemize}
        \item Metodo potenziometrico e fem;
        \item Amplificatore operazionale in continua
        \item Caratterizzazione in frequenza di un filtro attivo
        \item Caratterizzazione di linee di trasmissione. 
    \end{itemize}
    Inoltre, la prova finale è un orale che consiste in esercizio su corrente continua o 
    alternata e un colloquio su argomenti teorici e sperimentali affrontati nel corso. 
\end{itemize}
Le esperienze sono organizzate così
\begin{itemize}
    \item Consegna iniziale dell'elaborato entro una settimana dallo svolgimento;
    \begin{itemize}
        \item Consegna ai docenti dell'elaborato
        \item Peer-review con un altro gruppo 
        \item Comunicazione e commenti dell'elaborato di un altro gruppo
    \end{itemize}
    \item Consegna entro due settimane di una seconda versione rivista e corretta, 
    la quale sarà oggetto della revisione e valutazione da parte dei docenti.
\end{itemize}

\section{Definizioni di unità di misura}
Legge di Ohm 
\begin{gather*}
    V = IR
\end{gather*}
i portatori di carica sono gli elettroni, la cui carica è minuscola. Dal 2019 la carica dell'elettrone 
è stata definita come $e = 1.602176634 \cdot 10^{-19} \ C$, fissata come costante nel sistema 
internazionale. A partire da questa si definisce l'ampere, ossia la corrente: 
\begin{gather*}
    1 \ C = 1 \ A \cdot 1 \ s \ \Longrightarrow \ 1 A = \frac{1\ C}{1 \ s } \qquad I = \frac{dq}{dt}
\end{gather*}
In laboratorio non si misurano quantità microscopiche (non si contano i singoli elettroni) anche se 
in alcuni esperimenti esiste una sorta di granularità: il fatto che la corrente è data da piccoli oggetti porta
ad un rumore di fondo chiamato \textbf{Shock noise}, che è misurabile. L'aspetto microscopico è molto importante 
anche se nel corso di Lab II non viene misurato. Tutte le tecniche moderne per misurare la corrente sotto lo shock noise 
sono delle tecniche che permettono di andare oltre le tecniche di statistica classica per contare gli elettroni 
ed i fotoni. \\ \noindent
Supponendo di avere a che fare con un conduttore e di avere degli elettroni di conduzione che sono 
liberi di spostarsi (in genere sono un paio per atomo). La densità del rame è di 9 g per cm$^{3}$, mentre il 
peso molare è 63.5 g per mole. Il numero di elettroni per centimetro cubo per il rame è
\begin{gather*}
    n = \frac{6 \cdot 10^{23} \frac{el}{mol}}{63.5 \frac{g}{mol}} \cdot 9  \frac{g}{cm^{3}} \approx 9 \cdot 10^{22} \frac{el}{cm^{3}}
\end{gather*} 
Il passaggio che si esegue da discreto a continuo in fisica II è utilizzando una funzione di densità continua e ben definita in modo 
tale che contenga un numero sufficientemente grande di elettroni tale per cui sia una funzione costante. La funzione definite in fisica II 
di densità $\rho(x, y, z)$ descrive la densità di carica $\rho = n\cdot l$, dove $l$ è il lato 
considerato dell'ordine del nanometro. \\ \noindent
L'effetto di una forza su di una carica è data dal campo elettrico che risente questa carica dal campo elettrico:
\begin{gather*}
    \vv{F} = q \vv{E}  
\end{gather*}
Dunque, producendo un accelerazione sulla particella, si potrebbe affermare ch
la velocità del flusso possa variare nel tempo. In realtà, il fatto che gli elettroni non sono 
propriamente liberi di muoversi, fa sì che gli elettroni sbattano tra loro e dunque esista
una sorta di frizione o \textbf{resistenza} che impedisce agli elettroni di andare più veloci, facendo sì 
che si instauri una velocità costante e non una accelerazione costante. Ci si può anche chiedere come si confronta la velocità
rispetto alla velocità che deriva dall'agitazione termica del conduttore. Se si considera la nube elettronica
come se fosse una nube di gas costituita da elettroni, la velocità ad una data temperatura si ricava mediante la seguente
\begin{gather*}
    \frac{1}{2}m v_T^{2} = \frac{3}{2}k_B T
\end{gather*}
La velocità termica in un qualsiasi metallo è pari a
\begin{gather*}
    v_T = \sqrt{\frac{3k_BT}{m_e}} \approx 1.2 \cdot 10^{5} \frac{m}{s} 
\end{gather*}
All'equilibrio termodinamico gli elettroni sono effettivamente fermi poiché tutti gli 
elettroni si muovono in tutte le direzioni e dunque il loro moto medio è nullo.
Si vuole arrivare ad una relazione tra il campo elettrico esterno alla velocità di deriva alla quale 
si spostano gli elettroni che danno origine alla corrente descritta dalla legge di Ohm. Se 
si ponesse un campo elettrico al metallo, un qualche elettrone (che si ipotizza parta da fermo)
arriverà ad una certa velocità $\Delta v$ entro un tempo $\Delta t$ fino a che non sbatte 
contro un altra particella tornando a velocità nulla. In questo lasso di tempo il moto risulta 
accelerato e dunque il suo delta velocità 
\begin{gather*}
    \Delta \vv{v} = \vv{a} \Delta t = \frac{\vv{F} }{m}\Delta t = -\frac{e\vv{E} }{m}\Delta t  
\end{gather*} 
Se l'urto è elastico, in media, la velocità di deriva degli elettroni sarà uguale a
\begin{gather*}
    \left< \vv{v_d}  \right>  = - \frac{e\vv{E} }{2m}\Delta T
\end{gather*}
Si osserva che essa è proporzionale al campo. Il $\Delta T$ è il tempo che intercorre 
da una posizione all'altra e si può indicare come 
\begin{gather*}
    \Delta t = \frac{\Delta l}{v_T}
\end{gather*}
Dove $\Delta l$ è il cammino libero medio delle particelle e dunque la velocità di deriva sarà 
\begin{align}
    \vv{v_d} = -\frac{e\vv{E} }{2m} \frac{\Delta l}{v_T} 
\end{align}
Ipotizzando che il cammino libero medio dipenda solamente dalla velocità termica, allora tutte le quantità 
sono costanti (che dipendono dalla temperatura) indipendenti dal campo elettrico, ottenendo la proporzionalità
della velocità di deriva dal campo elettrico. Esiste dunque una dipendenza dalla resistenza alla temperatura:
maggiore è la temperatura e maggiore sarà la resistenza poiché minore è la velocità di deriva degli elettroni.

\subsection{Vettore densità di corrente}
Si definisce ora il \textbf{densità di corrente}: si determina l'equazione di continuità, 
la quale deriva dal principio di conservazione della carica e che, in gergo, dà luogo alla 
prima legge di Kirchhoff e vale in \textbf{condizioni stazionarie}, le quali non sono le stesse 
di fisica II. Per fisica II niente si muove in un conduttore, mentre per questo corso si intende
condizioni di corrente stazionaria. Se in un conduttore il campo è costante, allora anche la corrente 
è costante (solo per una parte del corso). La densità di corrente è dunque definita come il flusso 
di un vettore attraverso una superficie S:
\begin{align}
    I = \int_{S} \vv{J} \cdot \hat{n} d S = \int_{S} \vv{J} \cdot \vv{dS}      
\end{align}
Dove $\vv{J} = n  q \cdot \vv{v_d} $,  che è dunque proporzionale al campo elettrico e, inoltre,
è parallelo anche a $\vv{E}$. Per convenzione si dice che la corrente va da + a meno anche se, a muoversi,
sono gli elettroni. Le unità di misura di $\vv{J}$ sono 
\begin{gather*}
    \left[J \right] = \frac{1}{m^{3}} \cdot C \cdot \frac{m}{s} = \frac{A}{m^{2}}
\end{gather*} 
Ossia densità di corrente di superficie.



\end{document}
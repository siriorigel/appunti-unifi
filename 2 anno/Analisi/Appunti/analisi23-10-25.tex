\documentclass[a4paper, oneside]{article}
\usepackage{graphicx}
\usepackage{amsthm}
\usepackage{amsmath}
\usepackage{amssymb}
\usepackage[a4paper,
            bindingoffset=0.2in,
            left=2cm,
            right=2cm,
            top=2cm,
            bottom=2cm,
            footskip=.25in]{geometry}
\usepackage[italian]{babel}
\usepackage{pgfplots}
\usepackage{tabularx}
\usepackage{tikz}
\usepackage{wrapfig}
\usepackage{color}
\usepackage[d]{esvect}
\definecolor{page}{rgb}{0.129,0.157,0.212}
\pagecolor{page}
\color{white}
\graphicspath{ {./images/} }
\usetikzlibrary{shapes.geometric}
\usetikzlibrary{datavisualization}
\usetikzlibrary{datavisualization.formats.functions}
\usetikzlibrary{patterns}
\pgfplotsset{width=10cm,compat=1.9}

\title{Appunti Anal}
\author{Tommaso Miliani}
\date{23-10-25}

\begin{document}
\newtheoremstyle{theoremEnv}
                {}          % Space above
                {}          % Space below
                {\slshape}  % Body font
                {}          % Indent amount
                {\bfseries} % Head font
                {.}         % Punctuation after head
                {\newline}         % Space after theorem head
                {}          % Theorem head spec
\theoremstyle{theoremEnv}

\newtheorem{definition}{Definizione}[section]
\newtheorem{theorem}{Teorema}[section]
\newtheorem{lemma}{Proposizione}[section]
\newtheorem{observation}{Osservazione}[section]
\newtheorem{corollary}{Corollario}[theorem]
\newtheorem{example}{Esempio}[section]

\maketitle

\section{Curve parametriche}
\begin{definition}
    Si considera la curva 
    \begin{gather*}
        \underline{r}(k) : I \to \mathbb{R}^{3} \qquad I = [a, b] 
    \end{gather*}
    E dunque 
    \begin{gather*}
        \underline{r}(k) = \begin{pmatrix} 
            x(t) \\
            y(t) \\
            z(t)
        \end{pmatrix} \qquad x,y,z : I \to \mathbb{R}  
    \end{gather*}
    Dove $\gamma$ è il sostegno di $\underline{r}$ per cui $\gamma = \underline{r}(I)$.
\end{definition}

\begin{definition}[Oreientazione]
    L'orientazione è il verso di percorrenza del sostegno.
\end{definition}

\begin{definition}[Iniettività]
    Una curva $\underline{r}(t)$ semplice se non ha autointersezioni 
    (ossia non si interseca con sé stessa) $\underline{r}(t)$ è iniettiva.
\end{definition}

\begin{definition}[Semplice e chiusa]
    Una curva $\underline{r}$ è semplice e chiusa se
    $\underline{r}(a) = \underline{r}(b)$. 
\end{definition}

\begin{example}
    La circonferenza piana
    \begin{gather*}
        \underline{x}(\theta) = \begin{pmatrix} 
            R\cos\theta + x_0 \\
            R\sin\theta + y_0
        \end{pmatrix} 
    \end{gather*}
    Dato il centro della circonferenza $C = (x_0, y_0)$ e $R > 0$:
    \begin{gather*}
        \begin{tikzpicture}
            \draw(0, 0) -- (4, 0);
            \draw(0, 0) -- (0, 4);
            \draw(1, 2) circle (1);
            \filldraw(1, 2) circle (1pt);
        \end{tikzpicture}
    \end{gather*}
    Dato che la curva è semplice ed è chiusa (agli estremi si chiude).
\end{example}

\begin{definition}[Regolarità]
    Si dice che una curva è regolare se è semplice e le sue componenti sono 
    derivabili e
    \begin{align}
        |\dot{\underline{r}}(t)| \neq 0 \quad \forall t \in \mathring{I}
    \end{align}
    Dunque posso dire che 
    \begin{align}
        \dot{x}^{2}(t) + \dot{y}^{2}(t) + \dot{z}^{2}(t) \neq 0 \quad \forall t \in (a, b)   
    \end{align}
\end{definition}

\begin{definition}
    Una curva si dice regolare a tratti se
    \begin{gather*}
        \underline{r}(t) = \left\{\begin{array}{l}
            r_1(t) \quad t \in [a, t_1] \\
            \vdots \\
            r_k(t) \quad t \in [t_k, b]
        \end{array}\right.
    \end{gather*}
    Con $\underline{r}_i$ curve regolari
\end{definition}


\begin{definition}[Retta di sostengo]
    
\end{definition}

\begin{definition}[Versore tangente]
    
\end{definition}

\begin{definition}[Derivate della curva]
    \begin{itemize}
        \item $\underline{r}(t)$: è la traiettoria;
        \item $\dot{\underline{r}}(t)$: è la velocità;
        \item $\ddot{\underline{r}}(t)$: è l'accelerazione.
    \end{itemize}
\end{definition}

\begin{definition}
    Due parametri
    \begin{gather*}
        \underline{\phi}[a, b] \to \mathbb{R}^{2} \quad \underline{\psi}[\alpha, \beta] \to \mathbb{R}^{3}  
    \end{gather*}
    Sono equivalenti se e solo se
    \begin{align}
        \exists g : I \to J \quad g \in C^{1} : g'(t) \neq 0 \quad \forall t \in \mathring{I} : \underline{\phi}(t) = \underline{\psi}(g(t)) \quad \forall t \in I 
    \end{align}
\end{definition}

\begin{observation}
    Se $\underline{\psi}$ e $\underline{\phi}$ sono equivalenti allora
\end{observation}

\begin{definition}[Lunghezza di una curva]
    
\end{definition}

\end{document}
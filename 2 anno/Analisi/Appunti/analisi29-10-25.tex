\documentclass[a4paper, oneside]{article}
\usepackage{graphicx}
\usepackage{amsthm}
\usepackage{amsmath}
\usepackage{amssymb}
\usepackage[a4paper,
            bindingoffset=0.2in,
            left=2cm,
            right=2cm,
            top=2cm,
            bottom=2cm,
            footskip=.25in]{geometry}
\usepackage[italian]{babel}
\usepackage{pgfplots}
\usepackage{tabularx}
\usepackage{tikz}
\usepackage{wrapfig}
\usepackage{color}
\usepackage[d]{esvect}
\definecolor{page}{rgb}{0.129,0.157,0.212}
\pagecolor{page}
\color{white}
\graphicspath{ {./images/} }
\usetikzlibrary{shapes.geometric}
\usetikzlibrary{datavisualization}
\usetikzlibrary{datavisualization.formats.functions}
\usetikzlibrary{patterns}
\pgfplotsset{width=10cm,compat=1.18}
\usepgfplotslibrary{colormaps}

\title{Analisi}
\author{Tommaso Miliani}
\date{29-10-25}

\begin{document}
\newtheoremstyle{theoremEnv}
                {}          % Space above
                {}          % Space below
                {\slshape}  % Body font
                {}          % Indent amount
                {\bfseries} % Head font
                {.}         % Punctuation after head
                {\newline}         % Space after theorem head
                {}          % Theorem head spec
\theoremstyle{theoremEnv}

\newtheorem{definition}{Definizione}[section]
\newtheorem{theorem}{Teorema}[section]
\newtheorem{lemma}{Proposizione}[section]
\newtheorem{observation}{Osservazione}[section]
\newtheorem{corollary}{Corollario}[theorem]
\newtheorem{example}{Esempio}[section]

\maketitle

\section{Campi Vettoriali e forme differenziali}
\begin{definition}[Campo Vettoriale]
    Dato un certo insieme $\mathbb{A} \subset \mathbb{R}^{3}$, si definisce
    \textbf{campo vettoriale} su $\mathbb{A}$ è una funzione 
    \begin{gather*}
        F : \mathbb{A} \to \mathbb{R}^{3} 
    \end{gather*} 
    Ogni punto dello spazio associa un elemento di $\mathbb{R}^{3}$
    (che si può pensare come un vettore tridimensionale). Se il punto
    $(x, y, z) \in \mathbb{A}$, a questo punto il campo vettoriale associa 
    il vettore
    \begin{gather*}
        F(x, y, z)  \to p\begin{pmatrix}
                        F_1(x, y, z) \\
                         F_2(x, y, z) \\
                         F_3(x, y, z), 
        \end{pmatrix} 
    \end{gather*} 
    Questa definizione è valida qualunque sia la dimensione di $\mathbb{R}$. 
\end{definition}

\begin{example}[Massa punto materiale]
    Una massa puntiforme che si trova in un punto 
    nello spazio generico $P = (x_0, y_0, z_0)$, e considerato
    il campo gravitazionale che ci agisce, il campo sarà definito
    come
    \begin{gather*}
        F(x, y, z) = -cm \frac{(x - x_0, y - y_0, z - z_0)}{\left((x - x_0)^{2}, (y - y_0)^{2}, (z - z_0)^{2}   \right)^{\frac{3}{2}} }
    \end{gather*}
    Ossia
    \begin{gather*}
        \frac{(P - P_0)}{||P - P_0||^{3} }
    \end{gather*}
    Ossia tutti i vettori convergono verso il punto con massa.
    \begin{gather*}
         \begin{tikzpicture}[scale = 0.6]
 \begin{axis}[
    view = {0}{90},
  domain   = -3:3,
]
\addplot3[cyan, quiver={u=1*x, v=1*y, w =1*z, scale arrows=0.1, colored}, samples=16, -latex] {0};
\end{axis}
\end{tikzpicture}
    \end{gather*}
\end{example}

\begin{example}[Il campo di velocità di un corpo rigido]
    Il campo di velocità di un corpo rigido che ruota intorno all'asse $z$  con velocità
    angolare $\omega$, il campo di velocità della velocità angolare 
    sarà dato da
    \begin{gather*}
        v(x, y, z) = -\omega y \hat{i} + \omega x \hat{j}  
    \end{gather*}
\end{example}

\begin{example}
    Se una funzione è definita come
    \begin{gather*}
        f : \mathbb{A} \to \mathbb{R}
    \end{gather*}
    Ogni punto di $\mathbb{A}$ allora $\nabla f$ 
    \begin{gather*}
        \nabla f(x, y, z) := \left(\frac{\partial f}{\partial x} + \frac{\partial f}{\partial y} + \frac{\partial f}{\partial z}   \right)
    \end{gather*}
    SE 
    \begin{gather*}
        f(x, y, z) = x^{2} + y^{2} + z^{2} \ \Longrightarrow \ \nabla f = (2x, 2y, 2z)   
    \end{gather*}
    E dunque
    \begin{gather*}
                 \begin{tikzpicture}[scale = 0.6]
 \begin{axis}[
    view = {0}{90},
  domain   = -3:3,
]
\addplot3[cyan, quiver={u=2*x, v=2*y, w =2*z, scale arrows=0.1, colored}, samples=16, -latex] {0};
\end{axis}
\end{tikzpicture}
    \end{gather*}
\end{example}

\subsection{Forme differenziali}
    Sia $V$ uno spazio vettoriale, si considera l'insieme delle applicazioni
    lineari da $V \to \mathbb{R}$. Questo insieme è lui stesso 
    uno spazio vettoriale poiché le applicazioni lineari si possono 
    sommare tra di loro e si possono moltiplicare tra di loro. Lo spazio
    della applicazioni lineari di uno spazio $V$ e si indica con $V^{\star}$. 
    Fissata una base canonica si sa che lo spazio duale di $\mathbb{R}^{3}$ è
    lo spazio dei vettori tale per cui
    \begin{gather*}
        L \in \mathbb{R}^{3\star}  \qquad \begin{array}{l}
            a_1 = L(e_1) \\
            a_2 = L(e_2) \\
            a_3 = L(e_3)
        \end{array}
    \end{gather*}  
    E quindi
    \begin{gather*}
        L(v) = a_1v_1 +a_2 v_2 + a_3 v_3 \qquad v \in \mathbb{R}^{3\star}  \qquad v =(v_1, v_2, v_3)
    \end{gather*}
    Si definisce una base di $\mathbb{R}^{3\star}$ come
    \begin{gather*}
        dx_1 : \mathbb{R}^{3} \to \mathbb{R} \qquad dx_1(v) = v_1 \\
        dx_2 : \mathbb{R}^{3} \to \mathbb{R} \qquad dx_2(v) = v_2 \\
        dx_3 : \mathbb{R}^{3} \to \mathbb{R} \qquad dx_3(v) = v_3 \\  
    \end{gather*}
    Allora $L$ lo posso scrivere come combinazione lineare di questi elementi 
    \begin{gather*}
        L = a_1dx_1 +  a_2dx_2 +  a_3dx_3 
    \end{gather*}
    Il simbolo $dx_i$ (che di solito indicano gli incrementi infinitesimi di una certa
    variabile), sono invece elementi del duale di $\mathbb{R}^{3}$ che agiscono
    proiettando ogni componente sul relativo asse. Ogni applicazione lineare
    da $\mathbb{R}^{3} \to \mathbb{R}$ è scrivibile come l'operatore $dx_i$ che
    sfrutta un certo $L$ che può cambiare. Una forma differenziale è dunque una 
    funzione che è definita su di un sottoinsieme di $\mathbb{R}^{3} $ e per ogni 
    elemento di $\mathbb{R}^{3}$ associa un elemento di $\mathbb{R}^{3\star}$.  

\begin{definition}[Forma differenziale]
    Sia $\mathbb{A} \subset \mathbb{R}^{3} $, una forma differenziale  $\omega$ 
    definita su di $\mathbb{A}$ è una applicazione che 
    \begin{align}
        \omega : (x, y, z) \in \mathbb{A} \to L(x, y, z) \in \mathbb{R}^{3\star} 
    \end{align} 
    Di conseguenza, dalla definizione di $L$
    \begin{gather*}
        L = a_1(x, y, z)dx_1 +  a_2(x, y, z)dx_2 +  a_3(x, y, z)dx_3 
    \end{gather*}
    Posso costruire una forma differenziale se possiedo queste
    tre condizioni $a_i$ ma posso anche ricavarmi le condizioni se conosco
    già la forma differenziale.
\end{definition}

\begin{observation}
    Posso associare un campo vettoriale ad una forma differenziale 
    e posso anche fare l'inverso. 
    \begin{gather*}
        \omega(x, y, z) = a_1(x, y, z)dx_1 +  a_2(x, y, z)dx_2 +  a_3(x, y, z)dx_3 \\
        \updownarrow \\
        F(x, y, z) = (a_1(x, y, z), a_2(x, y, z), a_3(x, y, z))
    \end{gather*}
\end{observation}

\begin{example}
    Sia $f : \mathbb{A} \to \mathbb{R}$ è una funzione differenziale e chiamo 
    differenziale di $f$ 
    \begin{gather*}
        df (x, y, z) = \frac{\partial f}{\partial x}(x, y, z)dx_1 + \frac{\partial f}{\partial y}(x, y, z)dx_2 + \frac{\partial f}{\partial z}(x, y, z)dx_3 
    \end{gather*}
\end{example}

\begin{definition}
    Sia $ \omega : \mathbb{A} \subset \mathbb{R}^{3\star} $, $\omega$ si dice
    \textbf{esatta} se $\exists f : \mathbb{A} \to \mathbb{R}$ è 
    differenziabile tale che
    \begin{align}
        \omega = df
    \end{align}
    Cioè se $\omega = a_1(x, y, z)dx_1 +  a_2(x, y, z)dx_2 +  a_3(x, y, z)dx_3$ e se $\exists f$ tale che
    \begin{gather*}
        a_1 =\frac{\partial f}{\partial x_1} \qquad a_2 = \frac{\partial f}{\partial x_2} \qquad \frac{\partial f}{\partial x_3}   
    \end{gather*}
    tale $f$ si chiama \textbf{primitiva}.
\end{definition}

\begin{observation}
    Se $f$ è primitiva di $\omega$ anche $f + c $ lo è (ovviamente).
\end{observation}

\begin{definition}[Campo conservativo]
    Sia $F : \mathbb{A} \to \mathbb{R}^{3}$ un campo
    vettoriale e dunque  $F = (F_1, F_2, F_3)$, $F$ si dice \textbf{conservativo}
    se $\exists f : \mathbb{A} \to \mathbb{R}$ differenziabile
    tale che $f$ è il gradiente di questa funzione differenziabile:
    \begin{align}
        F = \nabla f
    \end{align}
    La funzione $f$ si chiama \textbf{potenziale} di $F$. 
\end{definition}

\begin{example}
    Se $F(x, y, z)$ è il campo gravitazionale (visto prima),
    allora il campo è conservativo e un suo potenziale (primitiva) è 
    \begin{gather*}
        f(x, y, z) = \frac{cm}{\left((x - x_0)^{2} + (y - y_0)^{2} + (z - z_0)^{2}  \right)^{\frac{1}{2}} }
    \end{gather*}
    Infatti, se si facesse
    \begin{gather*}
        \frac{\partial f}{\partial x} =  -cm\frac{x - x_0}{\left((x - x_0)^{2} + (y - y_0)^{2} + (z - z_0)^{2}  \right)^{\frac{1}{2}} }
    \end{gather*}
    Si otterrebbe 
\end{example}

\begin{observation}[Parallelismo tra campi e forme differenziali]
    Essere un campo conservativo equivale ad essere un differenziale esatto
    per una forma differenziale. 
\end{observation}

Non tutti i campi sono conservativi e non tutte le forme differenziali sono esatte.
Alcuni esempi sono
\begin{example}
    Presa su $\mathbb{R}^{2}$  la forma differenziale
    \begin{gather*}
        \omega(x, y) = 3x^{2}dx - xydy 
    \end{gather*} 
    Non è esatta, infatti, se lo fosse, dovremmo trovare una
    funzione $f$ tale che $\omega = df$ e dunque
    \begin{gather*}
        \frac{\partial f}{\partial x}(x, y) = 3x^{2} \qquad \frac{\partial f}{\partial y}(x, y) = -xy   
    \end{gather*}
    Se questo fosse vero, allora potrei dire che
    \begin{gather*}
        \frac{\partial^{2}  f}{\partial y \partial x } = 0 \qquad \frac{\partial^{2 } f}{\partial x \partial y} = -y  
    \end{gather*}
    Dato che le derivate parziali miste devono essere equivalenti (per le
    ipotesi del teorema di Cauchy Schwartz), allora 
    non è possibile che sia esatta. Un altro modo per vederlo è 
    cercare le primitive per la $x $ e per la $y$
    \begin{gather*}
        f(x, y) = x^{3} + c (y)
    \end{gather*}
    Se questa fosse la $f$, allora la 
    \begin{gather*}
        \frac{\partial f}{\partial y}(x, y) = 0 + c'(y) 
    \end{gather*}
    Questa dipenderebbe solo dalla $y$.
\end{example}

\begin{observation}
    Un campo su $\mathbb{R}^{n}$  è continuo in $C^{k}$ se le sue componenti
    sono tutte continue in $C^{k}$. Con $k < n$. 
\end{observation}

\section{Lavoro di un campo vettoriale}
\begin{wrapfigure}{r}{0.4\textwidth}
    \centering
    \caption{Il lavoro su di una curva in un insieme $\mathbb{A}$}
    \begin{tikzpicture}[scale = 1.2]
        \draw(0, 0) .. controls (-1, 1.2) and (-0.5, 2) .. (0, 2.2);
        \draw(0, 2.2) .. controls (1, 2.5) and (1.8, 2.5) .. (2, 1.5);
        \draw(2, 1.5) .. controls (1.5,  1.4) and (2, 0.2) .. (0, 0);
        \filldraw(0.5, 0.5) circle (1pt) node[anchor = north] {$P_1$};
        \filldraw(0, 1.8) circle (1pt) node[anchor = south] {$P_0$};
        \draw(0, 1.8) .. controls (1, 2.2) and (1.5, 2) .. (1.5, 1.7);
        \filldraw(1.5, 1.7) circle (1pt);
        \draw[thick, ->](1.5, 1.7) -- (2.2, 1) node[at end, right] {$F(\phi(t))$}; 
        \draw[thick, ->](1.5, 1.7) -- (1.4, 1.3) node[at end, below] {$\dot{\phi}(t)$};
        \draw(1.5, 1.7) .. controls (1.3, 1.4) ..  (0, 1);
        \draw(0, 1) .. controls (-0.5, 0.8) and (-0.5, 0.5) .. (0.5, 0.5);
        \node at(1.8, 2.5) {$\mathbb{A}$};
    \end{tikzpicture}    
\end{wrapfigure}
Presa una curva orientata $\gamma$ il cui sostegno 
sia contenuto in $\mathbb{A}$, (ossia ha un verso di percorrenza
ed è una curva). Si definisce allora il lavoro di un campo vettoriale
lungo una curva orientata (o anche curvilineo di seconda specie).
\begin{definition}
    Sia $\gamma$ regolare a tratti e sia $F$ un campo continuo
    e sia $\phi[a, b] \to \mathbb{A}$ una parametrizzazione regolare a tratti
    di $\gamma$ e concorde con l'orientazione di $\gamma$. 
    \begin{gather*}
        \phi(a) = P_0 \quad \phi(b) = P_1 \quad \tau(t) = \frac{\dot{\phi}(t)}{||\dot{\phi(t)}||}
    \end{gather*}
    SI definisce allora il \textbf{lavoro} di $F$ su $\gamma$ come
    \begin{align}
        \int_{a}^{b} F(\phi(t)) \cdot  \dot{\phi(t)} \ dt 
    \end{align}
    Si può anche scrivere come
    \begin{align}
        \int_\gamma \left< F, \tau \right> d S
    \end{align}
    Dove in ogni punto della curva $\tau$ è tangente alla curva $\gamma$
    e concorde con l'orientazione della curva stessa. Questo è dovuto al fatto che
    \begin{gather*}
        \int_{\gamma} \left< F, \tau \right> dS = \int_{a}^{b} \left< F(\phi(t)), \tau(\phi(t)) \right>  ||\dot{\phi( t)}|| \ dt
    \end{gather*}
    Dove  
    \begin{gather*}
        \tau(\phi(t)) = \frac{\dot{\phi(t)}}{|| \dot{\phi(t)}||}    
    \end{gather*}
    Si chiama dunque \textbf{integrale curvilineo di seconda specie} 
    oppure \textbf{lavoro di uhn campo su di una curva} che dipende dall'orientazione
    della curva (infatti se si percorresse la curva al contrario il lavoro
    sarebbe lo stesso ma dovrei cambiare il segno).
\end{definition}

\begin{example}
    Se $F(xy, y, 0)$, il lavoro sulla curva
    \begin{gather*}
        \phi_1 = (\cos t, \sin t, 0) \quad t \in [0, 2\pi]
    \end{gather*}
    contenuta nel piano $\{z = 0\}$, allora il lavoro 
    è nullo poiché (dato che la derivata della curva è $(-\sin t, \cos t, 0)$)
    \begin{gather*}
        \int_{0}^{2\pi} (\cos t, \sin t, 0) \cdot  (-\sin t, \cos t, 0) \ dt 
    \end{gather*}
    Dai conti, si vede che l'integrale è esattamente nullo
\end{example}

\end{document}
\documentclass[a4paper, oneside]{article}
\usepackage{graphicx}
\usepackage{amsthm}
\usepackage{amsmath}
\usepackage{amssymb}
\usepackage[a4paper,
            bindingoffset=0.2in,
            left=2cm,
            right=2cm,
            top=2cm,
            bottom=2cm,
            footskip=.25in]{geometry}
\usepackage[italian]{babel}
\usepackage{pgfplots}
\usepackage{tabularx}
\usepackage{tikz}
\usepackage{wrapfig}
\usepackage{color}
\usepackage[d]{esvect}
\usepackage{chemfig}
\usepackage{mhchem}
\definecolor{page}{rgb}{0.129,0.157,0.212}
\pagecolor{page}
\color{white}
\graphicspath{ {./images/} }
\usetikzlibrary{shapes.geometric}
\usetikzlibrary{datavisualization}
\usetikzlibrary{datavisualization.formats.functions}
\usetikzlibrary{patterns}
\pgfplotsset{width=10cm,compat=1.18}

\title{Analisi Bianchi}
\author{Tommaso Miliani}
\date{04-11-25}

\begin{document}
\newtheoremstyle{theoremEnv}
                {}          % Space above
                {}          % Space below
                {\slshape}  % Body font
                {}          % Indent amount
                {\bfseries} % Head font
                {.}         % Punctuation after head
                {\newline}  % Space after theorem head
                {}          % Theorem head spec
\theoremstyle{theoremEnv}

\newtheorem{definition}{Definizione}[section]
\newtheorem{theorem}{Teorema}[section]
\newtheorem{lemma}{Proposizione}[section]
\newtheorem{observation}{Osservazione}[section]
\newtheorem{corollary}{Corollario}[theorem]
\newtheorem{example}{Esempio}[section]
\newtheorem{remark}{Enunciato}[section]

\maketitle

\section{Insiemi semplicemente connessi}
Se $\omega$ è esatta, allora $\omega$ è chiusa, altrimenti,
se $F$ è un campo conservativo, allora $\text{rot}F = 0$.
Vale anche il viceversa? In generale no. Se infatti
$\omega$ è definita su di un certo insieme $\mathbb{A}$ ed è
chiusa, non è detto che esista una funzione $F$ definita su tutto $\mathbb{A}$
tale che $\omega = dF$.
\begin{example}
    Non esiste $F$ definita su $\mathbb{R}^{2} - \{(0, 0)\} : \omega = dF$. \\
    $\exists F$ definita su $\{(x, y) : x> 0\} : \omega = dF$ poiché è un insieme semplicemente
    connesso. 
\end{example}

\begin{observation}
    Se $\mathbb{A}$ è semplicemente connesso, allora si può invertire
    la definizione. 
\end{observation}

\begin{definition}[Essere semplicemente connesso]
    Un insieme aperto $\mathbb{A}$ si dice \textbf{semplicemente connesso}
    se è connesso e inoltre ogni curva chiusa, e interamente contenuta in $\mathbb{A}$, 
    può essere ridotta mediante una deformazione continua ad un unico 
    punto senza mai uscire da $\mathbb{A}$. 
\end{definition}

\begin{example}[Esempi in $\mathbb{R}^{2}$]
    Sono semplicemente connessi cerchi, ellissi, poligoni, semipiani,
    il piano $\mathbb{R}^{2}$ intero ed il piano privato di una semiretta. 
    Non sono semplicemente connessi il piano o un cerchio, o un 
    ellisse, o un poligono privato di un punto, una corona circolare o un insieme che presenta un buco. \\
    In tutti i casi si deve pensare che, per una curva chiusa che gira intorno
    al buco, una deformazione continua non riesce a deformare la curva in un punto
    poiché non riesce a stare dentro l'insieme ma si deve per forza passare per il buco. 
\end{example}

\begin{example}[Esempi in $\mathbb{R}^{3}$]
    Il toro non è un solido semplicemente connesso mentre
    una sfera lo è, così come un guscio sferico in quanto 
    è possibile "scansare" il buco.  Qualsiasi curva nello spazio
    può chiudere un insieme e renderlo semplicemente connesso 
    togliendo un numero finito di punti. 
\end{example}

\begin{definition}
    Un insieme $\mathbb{A}$ si dice \textbf{stellato} se esiste un punto
    $P_0 \in \mathbb{A}$ tale che $\forall P \in \mathbb{A}$, tutto il segmento
    di estremi $P$ e $P_0$ è contenuto in $\mathbb{A}$. Si dice
    stellato poiché sono insiemi concavi. 
\end{definition}

Per la definizione esatta di semplicemente connesso si introduce la def
\begin{definition}[Omotopia tra curve]
    Siano$\gamma_1$ e $\gamma_2$ curve contenute in $\mathbb{A}$ aperto e connesso
    con $\mathbb{A} \subset \mathbb{R}^{2}$ o $\mathbb{R}^{3}$. Supponiamo che
    $\phi_1:(a, b) \to \mathbb{A}$  e $\phi_2:(a, b) \to \mathbb{A}$ siano 
    delle parametrizzazione di $\gamma_1$ e di $\gamma_2$ rispettivamente. 
    In questo modo. 
    \begin{gather*}
        \phi_1(a) = \phi_2(a) = P_a \qquad \phi_1(b) = \phi_2(b) = P_b
    \end{gather*} 
    Allora $\gamma_1$ e $\gamma_2$ si dicono \textbf{Omotope} in $\mathbb{A}$ se
    esiste una funzione continua $\phi(t, \lambda)$, che dipende dal parametro $t \in [a, b]$ e
    $\lambda \in [0, 1]$, tale che siano soddisfatte le seguenti condizioni 
    \begin{enumerate}
        \item $\phi(t, 0) = \phi_1(t) \qquad \phi(t, 1) = \phi_2(t) \qquad \forall t \in [a, b]$;
        \item $\phi(a, \lambda) = P_a \ \qquad \phi(b, \lambda) = P_b \qquad \forall \lambda \in [0, 1]$;
        \item $\forall \lambda \in [0, 1]  \quad\qquad \phi_\lambda := \phi = \phi(t, \lambda) \subset \mathbb{A}$. 
    \end{enumerate}
    L'introduzione del parametro $\lambda$ è un modo per far variare in modo continuo
    una curva generica tra $\lambda_1$ e $\lambda_2$ in modo continuo. Per ogni $\lambda$
    fissato si ha una famiglia di curve con i soliti estremi $P_a$ e $P_b$.  \\
    Se $\gamma_1$ e $\gamma_2$ sono due curve chiuse e 
    \begin{gather*}
        \phi_1(a) = \phi_2(a) = P_a \qquad \phi_1(b) = \phi_2(b) = P_b
    \end{gather*} 
    allora nella definizione precedente queste si dicono omotope se vale la
    definizione precedente con la seconda condizione sostituita dalla seguente:
    \begin{gather*}
        \phi(a, \lambda) = \phi(b, \lambda) \qquad \forall \lambda \in [0, 1]
    \end{gather*}
\end{definition}
Si può ora dare la definizione formale di insieme semplicemente connesso
\begin{definition}[Definizione rigorosa di semplicemente connesso]
    Un insieme aperto di $\mathbb{A}$ di $\mathbb{R}^{2}$ o $\mathbb{R}^{3}$ si dice semplicemente connesso
    se è connesso e due curve qualsiasi contenute in $\mathbb{A}$, ed aventi gli stessi estremi, 
    sono omotope. Questa definizione può essere data anche alle curve chiuse
    in quanto ogni curva chiusa contenuta in $\mathbb{A}$ è omotopa ad una curva costante, ossia
    si riduce ad un solo punto.  
\end{definition}

\begin{theorem}[Teorema $n!$]
    Sia $\omega = a_1(x, y)dx + a_2(x, y)dy$ una forma differenziale
    $C^{1}$ e chiusa in un insieme $\mathbb{A} \subset \mathbb{R}^{2}$ aperto
    e semplicemente connesso. Allora $\omega$ è esatta.  Vale un enucnniato analogo in $\mathbb{R}^{3}$.
\end{theorem}
\begin{proof}
    Prese due curve $\gamma_1$ e $\gamma_2$ con gli stessi estremi $P_a $ e $P_b$, voglio dimostrare allora 
    che l'integrale delle forme differenziali sulle due curve è lo stesso. Poiché $\mathbb{A}$ è semplicemente connesso,
    $\gamma_1$ e $\gamma_2$ sono omotope. Esiste allora una funzione
    \begin{gather*}
        \gamma(t, \lambda) :[a, b] \times [0, 1] \to \mathbb{A}
    \end{gather*}
    che soddisfa le prime due proprietà della definizione di omotopia. 
    Si definisce allora $\forall \lambda \in [0, 1]$ l'integrale
    \begin{gather*}
        I(\lambda) = \int_{\gamma_\lambda}^{} \omega = \int_{a}^{b} a_1(x(t, \lambda), y(t, \lambda)) \frac{\partial x}{\partial t}(t, \lambda) + a_2(x(t, \lambda), y(t, \lambda))\frac{\partial y}{\partial t}(t, \lambda) \ dt   
    \end{gather*}
    Si vuole dimostrare che questo numero non dipenda da $\lambda$ e che dunque $I(0) = I(1)$. 
    Si suppone che $\phi(t, \lambda)$ sia $C^{1}([a, b] \times [0, 1])$ e che 
    le derivate seconde miste 
    \begin{gather*}
        \frac{\partial^{2} x}{\partial \lambda \partial t} \qquad \frac{\partial^{2} y}{\partial \lambda \partial t }  
    \end{gather*}
    Siano entrambe continue. Posso dunque derivare rispetto a $\lambda$
    l'integrale e vedere che sia zero:
    \begin{gather*}
        \frac{d}{d\lambda}I(\lambda) = \frac{d}{d\lambda}\int_{\lambda_\gamma}^{} \omega 
    \end{gather*}
    Fare la derivata dell'integrale è come derivare l'argomento dell'integrale
    (ancora non dimostrata nel corso, ma sarà definita successivamente). A questo punto si 
    ha una funzione composta e quello che si vuole fare è dimostrare che l'ipotesi
    $\omega$ chiusa, implichi che l'integrale tra $a$ e $b$ dell'argomento sia
    \begin{gather*}
        \int_{a}^{b}\frac{d}{d\lambda} \left(\dots\right) dt \underset{dimostrare}{=}\int_{a}^{b} \frac{d}{dt}\left(\dots\right)dt  
    \end{gather*}
    Ogni funzione è primitiva della propria derivata e dunque posso dire che 
    \begin{gather*}
        \int_{a}^{b}\frac{d}{d\lambda} \left(\dots\right) \ dt = \int_{a}^{b}\frac{d}{dt}a_1(x(t, \lambda), y(t, \lambda)) \frac{\partial x}{\partial \lambda}(t, \lambda) + a_2(x(t, \lambda), y(t, \lambda))\frac{\partial y}{\partial \lambda}(t, \lambda) \ dt
    \end{gather*}
    Sfruttando la seconda condizione delle funzioni omotope allora l'integrale
    è esattamente zero. 
    \begin{gather*}
        \frac{d}{d\lambda} \left(a_1(x(t, \lambda), y(t, \lambda)) \frac{\partial x}{\partial t}(t, \lambda) + a_2(x(t, \lambda), y(t, \lambda))\frac{\partial y}{\partial t}(t, \lambda)\right) \\
        \frac{d}{dt} \left(a_1(x(t, \lambda), y(t, \lambda)) \frac{\partial x}{\partial \partial}(t, \lambda) + a_2(x(t, \lambda), y(t, \lambda))\frac{\partial y}{\partial \lambda}(t, \lambda)\right)
    \end{gather*}
    Si vuole ora dimostrare che le due espressioni sono uguali: posso fare la derivata di funzione
    per entrambe le espressioni $I$ e $II$:
    \begin{gather*}
        \frac{d}{d\lambda}(I) = \left(\frac{\partial a_1}{\partial \lambda}\left(\dots\right) \frac{\partial x}{\partial \lambda} + \frac{\partial a_1}{\partial y}\left(\dots\right)\frac{\partial y}{\partial \lambda} \right) \frac{\partial x}{\partial t}     + a_1\left(\dots\right)\cdot \frac{\partial ^{2}x}{\partial \lambda \partial t} \left(\dots\right) + \\
        \left(\frac{\partial a_2}{\partial \lambda}\left(\dots\right) \frac{\partial x}{\partial \lambda} + \frac{\partial a_2}{\partial y}\left(\dots\right)\frac{\partial y}{\partial \lambda} \right) \frac{\partial x}{\partial t}     + a_2\left(\dots\right)\cdot \frac{\partial ^{2}y}{\partial \lambda \partial t} \left(\dots\right) \\
        \frac{d}{d\lambda}(II) = \left(\frac{\partial a_1}{\partial x}\left(\dots\right) \frac{\partial x}{\partial t} + \frac{\partial a_1}{\partial \lambda}\left(\dots\right)\frac{\partial y}{\partial t} \right) \frac{\partial y}{\partial \lambda}     + a_1\left(\dots\right)\cdot \frac{\partial ^{2}x}{\partial t \partial \lambda} \left(\dots\right) + \\
        \left(\frac{\partial a_2}{\partial x}\left(\dots\right) \frac{\partial x}{\partial t} + \frac{\partial a_2}{\partial y}\left(\dots\right)\frac{\partial y}{\partial t} \right) \frac{\partial y}{\partial \lambda}     + a_2\left(\dots\right)\cdot \frac{\partial ^{2}y}{\partial t \partial \lambda} \left(\dots\right)
    \end{gather*}
    Dato che so che è chiusa, posso dire che $\forall (x, y) \in \mathbb{A} \frac{\partial a_1}{\partial y}(x, y) = \frac{\partial a_2}{\partial x}(x, y) $. 
\end{proof}



\end{document}
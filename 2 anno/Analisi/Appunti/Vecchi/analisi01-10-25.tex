\documentclass[a4paper, oneside]{article}
\usepackage{graphicx}
\usepackage{amsthm}
\usepackage{amsmath}
\usepackage{amssymb}
\usepackage[a4paper,
            bindingoffset=0.2in,
            left=2cm,
            right=2cm,
            top=2cm,
            bottom=2cm,
            footskip=.25in]{geometry}
\usepackage[italian]{babel}
\usepackage{pgfplots}
\usepackage{tabularx}
\usepackage{tikz}
\usepackage{wrapfig}
\usepackage{color}
\usepackage[d]{esvect}
\definecolor{page}{rgb}{0.129,0.157,0.212}
\pagecolor{page}
\color{white}
\graphicspath{ {./images/} }
\usetikzlibrary{shapes.geometric}
\usetikzlibrary{datavisualization}
\usetikzlibrary{datavisualization.formats.functions}
\usetikzlibrary{patterns}
\pgfplotsset{width=10cm,compat=1.9}

\title{Appunti di analisi}
\author{Tommaso Miliani}
\date{01-10-25}

\begin{document}
\newtheoremstyle{theoremEnv}
                {}          % Space above
                {}          % Space below
                {\slshape}  % Body font
                {}          % Indent amount
                {\bfseries} % Head font
                {.}         % Punctuation after head
                {\newline}         % Space after theorem head
                {}          % Theorem head spec
\theoremstyle{theoremEnv}

\newtheorem{definition}{Definizione}[section]
\newtheorem{theorem}{Teorema}[section]
\newtheorem{lemma}{Proposizione}[section]
\newtheorem{observation}{Osservazione}[section]
\newtheorem{corollary}{Corollario}[theorem]
\newtheorem{example}{Esempio}[section]

\maketitle

\section{Esempio pisel}

\begin{example}
    \begin{gather*}
        X = C^{0}([-1, 1]) \qquad d = d_{L^{1} } \\
        d(f, g) = \int_{-1}^{1}|f - g| \ dx \ \Longrightarrow \ \{x_k\} \leq X
    \end{gather*}
\end{example}

\begin{definition}
    Ogni successione di Cauchy convergente è detto spazio metrico
    completo.
\end{definition}

\begin{definition}
    Se si ha uno spazio normato che è completo come spazio metrico
    rispetto alla metrica indotta dallo spazio normato è detto \textbf{Spazio di Banach}. 
\end{definition}

\begin{observation}
    $C^{1}([a, b]); d_{C^{1} }$ non è completo. \\
    $C^{0}([a, b]); d_{C^{1} }$ non è completo.  
\end{observation}

Alcune proprietà degli spazi di Banach
\begin{observation}
    Consideriamo uno spazio metrico ed una successione che converge in un
    punto $x_0 \in \mathbb{X}$, allora posso dire che il punto limite
    appartiene alla chiusura dell'insieme $Y$. Un sottoinsieme
    $Y$ di $\mathbb{X}$ è chiuso se e solo se contiene i limiti di tutte le successioni
    convergenti di $Y$. 
\end{observation}

\begin{observation}
    Se $(\mathbb{X}, d)$ è uno spazio metrico completo ed un suo sottoinsieme
    $B$ è chiuso allora $(B, d)$ è uno spazio metrico completo. 
\end{observation}

\begin{definition}
    Prendiamo due spazi metrici ognuno con una sua distanza definita come
    \begin{gather*}
        (\mathbb{X}, d_{X}) \qquad (\mathbb{Y}, d_{Y}) \qquad f: \mathbb{X} \to \mathbb{Y}
    \end{gather*}
    \begin{itemize}
        \item $f$ è continua in un punto $x_0 \in \mathbb{X}$ se $\forall \{x_k\} \subseteq \mathbb{X} : x_n \to x_0$
    si ha che $\{f(x_n)\} \subseteq \mathbb{Y}$ converge a $f(x_0)$ in $(\mathbb{Y}, d_Y)$ .
        \item $f$ è Lipszitchiana se $\exists L > 0$ tale che $d_Y(f(x), f(y)) \leq L d_X(x, y)$.
        \item $f$ si dice \textbf{contrazione} se Lipszitchiana con $L < 1$. 
    \end{itemize} 
\end{definition}

\begin{proof}
    Dimostrazione che se una funzione è Lipszitchiana allora è anche
    continua. \\
    Sia $x_0 \in \mathbb{X}$ considero allora una successione $\{x_k\} \subseteq \mathbb{X}$ tale che
    $x_n \to x_0 \in (\mathbb{X}, d)$ ossia $d(x_n, x_0) \in \mathbb{R} \to 0$. Considero allora 
    \begin{gather*}
        0 \leq d_Y(f(x_n), f(x_0)) \leq L d_X(x_n, x_0) \ \Longrightarrow \ d_Y(f(x_n), f(x_0)) \to 0
    \end{gather*} 
    Ossia
    \begin{gather*}
        f(x_n) \to f(x_0) \in (\mathbb{Y}, d_Y)
    \end{gather*}
    E quindi $f$ è continua in $x_0$. Se e solo se l'insieme è ordinato allora posso dire
    che quest dimostrazione è valida. 
\end{proof}

\begin{theorem}[Teorema di contrazione]
    Il teorema di contrazione (o Teorema di Banach-Caccioppoli) parte da uno spazio
    metrico completo. Considerata $f$ una applicazione da $\mathbb{X}$ in sé stessa,
    allora esiste ed è unico un punto $\overline{x} \in \mathbb{X}$ detto punto
    fisso tale che $f(\overline{x} ) = \overline{x}$.   
\end{theorem}
\begin{proof}
    La dimostrazione costruisce analiticamente il punto $\overline{x}$ andando a costruire
    una successione di punti così definita:
    \begin{gather*}
        x_1 = f(x_0) \qquad x_2 = f(x_1) \dots
    \end{gather*} 
    L'elemento $k + 1$ esimo è l'immagine dell'elemento $k$ esimo. Posso
    allora affermare che la successione $x_k$ è di Cauchy in $(\mathbb{X}, d)$
    secondo la seguente dimostrazione: prima di tutto stimo la distanza tra
    l'elemento $k +1$ e l'elemento $k$ esimo della successione. Per come è
    stata definita la succession $x_{k + 1} = f(x_k) = f(x_{k - 1})$ ossia è come
    scrivere la distanza tra $f(x_k)$ e $f(x_{k - 1})$ . Dato che
    $f$ è una contrazione:
    \begin{gather*}
        d(f(x_k), f(x_{k - 1})) \leq Ld(x_k, x_{k - 1})
    \end{gather*}
    Date la definizione della successione, allora si ha che
    \begin{gather*}
        d(f(x_k), f(x_{k - 1})) \leq Ld(f(x_{k - 1}), f(x_{k - 2})) \leq L^{2}d(x_{k - 1}, x_{k - 2}) 
    \end{gather*}
    Se si procede applicando questo algoritmo recursivamente allora 
    si otterrebbe che $\dots \leq L^{k}d(x_1, x_0) = L^{k} d(x_0, f(x_0))  $. IN questo modo
    la funzione distanza dipende solo ed esclusivamente dal punto $x_0$. Adesso
    voglio dimostrare che questa successione sia di Cauchy. Prendo $m > n$ con $m = n + p$
    e quindi considero la distanza tra
    \begin{gather*}
        d(x_m, x_n) = d(x_{n + p}, x_n)
    \end{gather*} 
    Posso allora valutare la distanza di tutti gli elementi fino a $n + 1$ fino a
    $n + p$; dato che posso valutare solo la distanza tra elementi consecutivi allora utilizzo
    la disuguaglianza triangolare. 
    \begin{gather*}
        d(x_{n + p}, x_n) \leq d(x_{n + p}, x_{n + p -1}) + \dots + d(x_{n + 1}, x_n) \leq \\
        d(x_0, f(x_0)) \left(L^{n + p - 1} + L^{n + p - 2}+ \dots + L^{n}   \right) = d(x_0, f(x_0)) \left(\frac{L^{n} - L^{n + p}  }{1 - L}\right) = \\
        d(x_0, f(x_0)) L^{n}\frac{1 - L^{p} }{1 - L}  < d(x_0 f(x_0)) \frac{L^{n} }{1 - L} 
    \end{gather*}
    L'ultimo passaggio è valido in quanto $L < 1$ e dunque la frazione è minore di $\frac{1}{1 - L}$. 
    Dato che la distanza mi tende a zero 
    \begin{gather*}
        \forall \epsilon > 0 \ \exists N : m, n > 0  \ \Longrightarrow \ d(x_m, x_n) < \epsilon 
    \end{gather*}
    E quindi la succession $\{x_k\}$ è di Cauchy. Non resta che dimostrare
    la tesi del teorema. SI osserva che $f$ è una contrazione e dunque è
    Lipszitchiana e dunque è continua. Allora
    \begin{gather*}
        f(x_k) \to f(\overline{x} )
    \end{gather*}
    Per definizione $f(x_k)$ è l'elemento $x_{k + 1}$ per definizione, ma $x_{k + 1}$ è
    un elemento della successione e dunque converge verso $\overline{x}$. Dato che questa successione
    ha ora due punti limite, per l'unicità del limite allora $\overline{x} = f(\overline{x} )$. Si dimostra ora che
    è unico: ponendo che esistano $\overline{x}$ e $\overline{y}$ che mi validano il teorema.
    Devo dimostrare allora che la distanza tra i due sia zero. Dato che sono
    punti fissi allora devo valutare la distanza tra le loro immagini: dato che
    $f$ è una contrazione, allora con $0 < L < 1$ l'oggetto è sicuramente minore
    della distanza. Se la distanza è diversa da zero, allora posso dire
    che 
    \begin{gather*}
        d(\overline{x}, \overline{y}  ) \leq d(f(\overline{x}, f(\overline{y})  )) \leq L d(\overline{x}, \overline{y}  )
    \end{gather*}     
    Avrei allora che la distanza tra due elemenit è minore o uguale della loro distnza e dunque
    se $d(\overline{x}, \overline{y}  ) \neq 0$ sarebbe assurdo
    $\ \Longrightarrow \ d(\overline{x}, \overline{y}  ) = 0$. 
\end{proof}

\end{document}
\documentclass[a4paper, oneside]{article}
\usepackage{graphicx}
\usepackage{amsthm}
\usepackage{amsmath}
\usepackage{amssymb}
\usepackage[a4paper,
            centering,
            bindingoffset=0.2in,
            left=2cm,
            right=2cm,
            top=2cm,
            bottom=2cm,
            footskip=.25in]{geometry}
\usepackage[italian]{babel}
\usepackage{pgfplots}
\usepackage{tabularx}
\usepackage{tikz}
\usepackage{wrapfig}
\usepackage{color}
\definecolor{page}{rgb}{0.129,0.157,0.212}
\pagecolor{page}
\color{white}
\graphicspath{ {./images/} }
\usetikzlibrary{shapes.geometric}
\usetikzlibrary{datavisualization}
\usetikzlibrary{datavisualization.formats.functions}
\pgfplotsset{width=10cm,compat=1.9}

\title{Appunti Analisi 2}
\author{Tommaso Miliani}
\date{16-09-25}

\begin{document}
\newtheoremstyle{theoremEnv}
                {}          % Space above
                {}          % Space below
                {\slshape}  % Body font
                {}          % Indent amount
                {\bfseries} % Head font
                {.}         % Punctuation after head
                {\newline}         % Space after theorem head
                {}          % Theorem head spec
\theoremstyle{theoremEnv}

\newtheorem{definition}{Definizione}[section]
\newtheorem{theorem}{Teorema}[section]
\newtheorem{lemma}{Proposizione}[section]
\newtheorem{observation}{Osservazione}[section]
\newtheorem{corollary}{Corollario}[theorem]
\newtheorem{example}{Esempio}[section]

\maketitle

\section{Introduzione al corso}
Il corso di analisi 2 si concentra sullo studio di equazioni differenziali, 
calcolo differenziale per funzioni vettoriali (ossia funzioni che
da un vettore restituiscono un vettore), equazioni a più variabili e
infine successioni e serie di funzioni. \\
L'esame di analisi 2 è composto da uno scritto e da un orale:
lo scritto si passa con un voto minimo di 16 e durante l'anno ci saranno
due prove parziali che sarà a Novembre (13) e l'altro parziale i primi giorni
di Gennaio. 
\begin{itemize}
    \item Si può usare un libro di testo ed un foglio A4 (UNO SOLO) nel quale
    c'è tutto quello che si riesce a scriverci.
    \item Si può riprovare un parziale che si è fatto male 
    \item Si può partecipare ai parziali anche se non si è fatto analisi 1
    \item Il libro di testo consigliato è il "Fuscolo, Marcellini Sbordone Lezioni di analisi matematica 2"
    oppure il "Bramanti Pagani Salsa analisi matematica 2" che purtroppo per alcuni argomenti non è completo.
    \item Nessuna dispensa quindi per esercizi svolti rifarsi ai libri di testo "Esercitazioni di analisi due". 
\end{itemize}


\section{Equazioni differenziali}
\begin{definition}[Equazioni differenziali]
    Un'equazione differenziale è una equazione in cui l'incognita è una funzione 
    di una qualche variabile. L'incognita è dunque $y = f(x)$. Si dice che una equazione differenziale
    è ordinaria se dipende solo da una variabile; nel caso di equazioni differenziali
    che dipendono da più variabili prendono il nome di equazioni differenziali a
    \textbf{derivate parziali}.
\end{definition}

\begin{example}[Esempi di equazioni differenziali]
    \begin{gather*}
        y'' - y = 0 \qquad y' = 3e^{y}  \qquad y' = 4x + 8
    \end{gather*}
    La soluzione della prima è $y = \sin x$ in quanto $y'' = -\sin x$ 
    e dunque $\forall x \in \mathbb{R} \ 0 = 0$. 
\end{example}

\begin{definition}[Ordine]
    Si definisce \textbf{ordine} il numero della derivata massima che
    compare all'interno delle equazioni differenziali. 
\end{definition}

\begin{definition}[Forma normale]
    Una equazione si dice in forma \textbf{normale} se la derivata più grande è
    isolata rispetto alle altre
    \begin{align}
        y^{(n)} = f(x, y, y', \dots, y^{(n - 1)} )
    \end{align}
\end{definition}
\begin{example}
    \begin{gather*}
        (y')^{2} + x^{3} = 0   \ \Longrightarrow \ y ' = \pm \sqrt{x^{3} } 
    \end{gather*}
    Non riesco a ricavare una equazione del primo ordine del tipo $F(x, y, y')$. 
    Quando riesco ad isolare da una parte la derivata più grande si riesce a semplificare
    la risoluzione delle equazioni differenziali.
\end{example}

\subsection{Equazioni differenziali del primo ordine}
La forma generale delle equazioni di primo ordine è del tipo
\begin{align}
    y' = f(x, y)
\end{align}
Una soluzione per questa forma generale di equazioni differenziale è una
funzione $y$ della variabile $x$ definita per $x \in \mathbb{A}$ (un qualche insieme)
e tale che per ogni $x \in \mathbb{A}$ risulti che $(x, y(x))$ appartenente al dominio di
$f(x, y)$ e inoltre la derivata di questa funzione calcolata in $x$ è uguale a $y'(x) = f(x, y(x))$.
\begin{example}[Esempio più semplice]
    \begin{gather*}
        y' = g(x) 
    \end{gather*}
    Dove $g(x) $ è una funzione continua appartenente ad un qualche insieme $\mathbb{A}$ 
    e quindi un'equazione di questo tipo equivale a trovare la primitiva di $g$. 
    Allora posso dire che $G(x) + c$ è soluzione dell'equazione differenziale, ossia la
    primitiva di $g(x)$. 
\end{example}
\begin{example}[Esempio di prima con soluzione diversa]
    Nell'equazione differenziale
    \begin{gather*}
        y'' - y = 0
    \end{gather*}
    Anche il coseno è soluzione in quanto si otterrebbe $y = \cos x$. Si ottiene
    allora che la seguente è anch'essa soluzione dell'equazione +
    \begin{gather*}
        C_1 \sin x + C_2 \cos x = 0
    \end{gather*}
    Generalmente l'equazione di $n$ ordine ha $n$ costanti $C_1, \dots, C_n$ che moltiplicano. 
\end{example}
\begin{example}
    \begin{gather*}
        \left\{\begin{array}{l}
            y' = x^{2} \\
            y(1) = 2 
        \end{array}\right.
    \end{gather*}
    Si ha che la primitiva di $y'$ è proprio $y = \frac{x^{3} }{3} + C$, data la seconda condizione
    nel sistema allora si ottiene che
    \begin{gather*}
        \frac{1}{3}  + C = 2 \ \Longrightarrow \ C = \frac{5}{3}
    \end{gather*}
\end{example}
\subsection{Il problema di Cauchy}
Se in una equazione differenziale del primo ordine avessi una situazione del tipo
\begin{gather*}
    \left\{\begin{array}{l}
        y' = f(x, y) \\
        y(x_0) = y_0
    \end{array}\right.
\end{gather*}
Ossia se ho delle condizioni sulle derivate fino alla $n-1$ esima, allora
questa tipologia di esercizi prenderà il nome di \textbf{problema di Cauchy}:
nel caso di una equazione del secondo ordine le condizioni non sono qualsiasi ma 
seguono il seguente schema:
\begin{gather*}
    \left\{\begin{array}{l}
        y'' = f(x, y, y') \\
        y(x_0) = y_0 \\
        y'(x_0) = y_1
    \end{array}\right.
\end{gather*}
Esistono alcuni teoremi che permettono di definire l'insieme delle soluzioni date
le condizioni del secondo membro di ciascuno dei due sistemi. Iniziamo dai
teoremi di esistenza delle soluzioni per il problema di Cauchy del primo ordine
in forma normale 
\begin{gather*}
    \left\{\begin{array}{l}
        y' = f(x, y) \\
        y(x_0) = y_0
    \end{array}\right.
\end{gather*}
SI definiscono alcune ipotesi prima di procedere:
\begin{definition}[Punto interno ad un insieme]
    \begin{gather*}
        Se \ \mathbb{A} \subset \mathbb{R}^{2}, (x_0, y_0) \in \mathbb{A} 
    \end{gather*}
    Si dice che $(x_0, y_0)$ è interno all'insieme $\mathbb{A}$, se
    esiste un cerchio di raggio positivo tale per cui il centro sia $(x_0, y_0)$ e sia contenuto
    dentro all'insieme $\mathbb{A}$. Nel caso in cui $\mathbb{A} \subset \mathbb{R}^{3}$ allora diciamo che un punto
    $(x_0, y_0, z_0)$ è interno all'insieme $\mathbb{A}$ se esiste una sfera di raggio positivo tale per cui
    $(x_0, y_0, z_0)$ sia il centro e sia contenuto all'interno dell'insieme.
\end{definition}
\begin{theorem}[Teorema di Peano]
    Se la funzione $f(x, y)$ è una funzione definita e continua  in un qualche insieme
    $\mathbb{A}$ e $(x_0, y_0) \in \mathbb{A}$, allora il problema di Cauchy ammette almeno una soluzione 
    definita in un intorno di $x_0$.  
\end{theorem}
Da questo possiamo fare due considerazioni:
\begin{enumerate}
    \item La continuità di $f(x, y)$ è necessaria per l'esistenza di una soluzione:
    \item La sola ipotesi di continuità non è garantisce che esista una sola soluzione
\end{enumerate}
\begin{example}[differenziale risolta col problema dii Cauchy]
    \begin{gather*}
        \left\{\begin{array}{l}
            y' = y^{2/3}  \\
            y(0) = 0
        \end{array}\right.
    \end{gather*}
    Allora dato che $f(x, y) = y^{2/3}$ è definita continua sull'intervallo $\mathbb{R} \times \mathbb{R}$ e quindi
    una funzione costante  $f(x) = 0$ ha valore zero in zero e derivata prima uguale a zero. Anche $y(x) = \frac{x^{3} }{27}$ è soluzione
    in quanto la derivata di questa funzione $y' = \frac{x^{2} }{9}$ ed è uguale a zero se calcolata in zero.  Questo problema di Cauchy
    ha allora due soluzioni distinte anche se in realtà ha soluzioni infinite:
    \begin{gather*}
        y(x) = \left\{\begin{array}{l}
            0 \ se \ x \in (-\infty , 0) \\
            \frac{(x - a)^{3} }{27} \ se \ x \in (a, +\infty)
        \end{array}\right.
    \end{gather*}
    Il teorema di Cauchy allora mi garantisce che esitano almeno una soluzione ma 
    non mi da la certezza che sia solo una (potrebbero essere infatti infinite).
\end{example}


\end{document}
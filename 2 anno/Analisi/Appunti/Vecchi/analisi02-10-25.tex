\documentclass[a4paper, oneside]{article}
\usepackage{graphicx}
\usepackage{amsthm}
\usepackage{amsmath}
\usepackage{amssymb}
\usepackage[a4paper,
            bindingoffset=0.2in,
            left=2cm,
            right=2cm,
            top=2cm,
            bottom=2cm,
            footskip=.25in]{geometry}
\usepackage[italian]{babel}
\usepackage{pgfplots}
\usepackage{tabularx}
\usepackage{tikz}
\usepackage{wrapfig}
\usepackage{color}
\usepackage[d]{esvect}
\definecolor{page}{rgb}{0.129,0.157,0.212}
\pagecolor{page}
\color{white}
\graphicspath{ {./images/} }
\usetikzlibrary{shapes.geometric}
\usetikzlibrary{datavisualization}
\usetikzlibrary{datavisualization.formats.functions}
\usetikzlibrary{patterns}
\pgfplotsset{width=10cm,compat=1.9}

\title{Appunti di Analisi}
\author{Tommaso Miliani}
\date{02-10-25}

\begin{document}
\newtheoremstyle{theoremEnv}
                {}          % Space above
                {}          % Space below
                {\slshape}  % Body font
                {}          % Indent amount
                {\bfseries} % Head font
                {.}         % Punctuation after head
                {\newline}         % Space after theorem head
                {}          % Theorem head spec
\theoremstyle{theoremEnv}

\newtheorem{definition}{Definizione}[section]
\newtheorem{theorem}{Teorema}[section]
\newtheorem{lemma}{Proposizione}[section]
\newtheorem{observation}{Osservazione}[section]
\newtheorem{corollary}{Corollario}[theorem]
\newtheorem{example}{Esempio}[section]

\maketitle

\section{Dimostrazione teorema di esistenza ed unicità di Cauchy locale}
Abbiamo già dimostrato il teorema di esistenza ed unicità 
del teorema. Ora si dimostra tutto il teorema basata sulla formulazione
integrale:
\begin{proof}
    Posso definire $M = \max \left| f(x, y) \right|$ nell'insieme $I \times J$ e possiamo
    definire $\delta < \min\{a, \frac{b}{M}, \frac{1}{L}\}$. Se definiamo
    l'insieme $B$ come l'insieme delle funzioni continue $C^{0}$ tali che hanno la norma
    del sup della differenza:
    \begin{gather*}
        B = \{u(x) \in C^{0}(I_\delta) : ||u - y_0||_{C^{0} (I_\delta) } \leq b\}
    \end{gather*} 
    Si osserva che $B$ è sottoinsieme dell'insieme $C^{0}$ e dunque è una palla chiusa e allora
    $(B, || ||_{C^{0} })$ è uno spazio di Banach  perché è un sottoinsieme chiuso di dello
    spazio normato $(C^{0}(I_\delta),||\cdot ||_{C^{0} } )$ che è completo.
    POsso allora definire una funzione
    \begin{gather*}
        F : B \to C^{0} \qquad F(u) = z \qquad z(x) = y_0 + \int_{x_0}^{x} f(t, u(t))dt \forall x \in I_\delta 
    \end{gather*}
    Vogliamo provare ora che  $Im \ F \subseteq B$: infatti se applico la definizione di
    $F$ e sia $u \in B$, allora posso dire che $F(u) \in C^{0}$: infatti:
    \begin{gather*}
        |F(u) - y_0| = |z(x) - y_0| \leq \left| \int_{x_0}^{x} \left| f(t, u(t)) \right| \ dt  \right|  
    \end{gather*}
    Si osserva che $\forall x \in I_\delta$ si ha che $t \in [x_0, x] \subseteq I$:
    \begin{gather*}
        u(t) \in J \ \Longrightarrow \ ||u - y_0||_{C^{0} } \leq b  \ \Longrightarrow \ \left| u(t) - y_0 \right| \leq b \forall x \in I_\delta 
    \end{gather*}
    Allora si ha che
    \begin{gather*}
        (t, u(t)) \in I \times J \ \Longrightarrow \ |f(t, u(t))| \leq M \leq M|x_0 - x| \leq M \delta, x \in I_\delta
    \end{gather*}
    Dato che avevo scelto $\delta$ più piccolo sicuramente di $\frac{b}{M}$, allora posso dire che
    quella espressione è $\leq b$: ho dimostrato allora che $\forall x \in I_\delta$ $f(u, u(t)) \leq b$ anche se non mi basta
    per dire che le immagini sono contenute in $B$. Quindi posso dire che
    \begin{gather*}
        \max  |F(u) - y_0| \leq b \ \Longrightarrow \ ||F(u) - y_0||  \leq b \ \Longrightarrow \ F(u) \subseteq B
    \end{gather*} 
    L'idea è dimostrare ora che questa funzione $F$ ha un punto fisso attraverso il
    teorema delle contrazioni: posso considerare ora il valore assoluto tra le funzioni:
    \begin{gather*}
        |F(y_1) - F(y_2)| \leq \left| \int_{x_0}^{x}\left|f(t, y_1(t))  - f(t, y_2(t))\right| \ dt\right| 
    \end{gather*}
    Posso allora maggiorare l'integrale seconodo la definizione di funzione Lipszitchiana poiché
    valgono le ipotesi del teorema:
    \begin{gather*}
        \leq L|x - x_0| \ ||y_1 - y_2||_{C^{0}(I_\delta) }
    \end{gather*}
    Maggiorato già con la norma: posso allora dire che è
    \begin{gather*}
        \leq L\delta ||y_1 - y_2||_{C^{0}(I_\delta) }
    \end{gather*}
    Quindi posso dire che l'immagine di
    \begin{gather*}
        ||F(y_1) - F(y_2)||_{C^{0}(I_\delta) } = \max \left| F(y_1) - F(y_2) \right| \leq L\delta ||y_1 - y_2||_{C^{0}(I_\delta) } 
    \end{gather*}
    allora per definizione, dato che delta è stato preso più piccolo di $M$, ho dimostrato che
    è Lipszitchiana e quindi anche una contrazione. 
    Allora posso dire che
    \begin{gather*}
        F: B \to B, (B, ||\cdot ||_{C^{0}(I_\delta) })
    \end{gather*}
    UNo spazio completo e con $F$ contrazione rispetto a $||\cdot ||_{C^{0}(I_\delta) }$ allora per il teorema
    delle contrazioni esiste una unica soluzione del problema di Cauchy nella forma integrale:
    \begin{gather*}
        y = y(x) : F(y) = y \ \Longrightarrow \ y(x) = y_0 + \int_{x_0}^{x}f(t, y(t))\ dt
    \end{gather*}
    Cioè $y$ risolve il teorema di Cauchy integrale.
\end{proof}


\section{Le funzioni a 2 variabili}
IN questo corso ci si concentra solo sulle funzioni a due variabili anche
se i procedimenti da applicare nel caso di più variabili sono gli stessi per
le funzioni a due variabili. Nel caso di funzioni a due variabili cambiano
molto le proprietà delle funzioni ed i loro teoremi rispetto alle loro
controparti ad una variabile sola. Lo spazio metrico $\mathbb{R}^{n}$ è uno spazio metrico
molto speciale. Preso un certo insieme
\begin{definition}[Insieme aperto in $\mathbb{R }^{2}$ ]
    \begin{gather*}
    A \subset \mathbb{R}^{2} 
\end{gather*} 
Questo è aperto se e solo se $\forall (x_0, y_0) \in A \exists R > 0$ tale che
\begin{gather*}
    B_R (x_0, y_0) = \{(x, y) \in \mathbb{R}^{2} : ||(x, y) - (x_0, y_0)||_{\epsilon} < R \} \subset  A
\end{gather*}
\end{definition}
\begin{definition}[Insieme chiuso in $\mathbb{R}^{2}$ ]
    \begin{gather*}
        E \subset \mathbb{R}^{2}   
    \end{gather*}
    dove $\overline{E}$ è il più più piccolo chiuso in $\supseteq E$ (ossia la chiusura di $E$)
    e dove $ \dot{E}$ è il più grande aperto $\subseteq E$ ( interno). 
\end{definition}
\begin{definition}[Dominio su $\mathbb{R}^{2}$ ]
    $D$ è un dominio se $D = \overline{A}$ con $A$ insieme aperto e
    \begin{gather*}
        D =\dot{D} \cup \ni D
    \end{gather*} 
\end{definition}
I grafici di una funzione a più variabili si chiama \textbf{superficie}. 
\begin{example}[esercizio tipo in una funzione a due variabili]
    \begin{gather*}
        f(x, y) = \ln(1 - x^{2} - y^{2}  )
    \end{gather*}
    \begin{itemize}
        \item Determinare il dominio naturale
        \item Descrivere le linee di livello 
        \item Trovare $\min/\max $ di $f$ su di un insieme $Q$
        \begin{gather*}
            Q = \{|x| \leq \frac{1}{2}, |y| \leq \frac{1}{2}\}
        \end{gather*}
    \end{itemize}
    Il dominio naturale della funzion sarà dato da:
    \begin{gather*}
        D = \{(x, y) \in \mathbb{R}^{2} : 1 - x^{2} - y^{2} > 0   \} \ \Longrightarrow \  D= \{(x ,y) \in \mathbb{R}^{2} : x^{2} + y^{2} < 1   \}
    \end{gather*}
    Ossia la circonferenza di raggio uno centrata sull'origine.
\end{example}
POsso ora definire le linee di livello come
\begin{gather*}
    f(x, y) = t = \ln(1 - x^{2} - y^{2}  ) \ \Longrightarrow \ e^{t} = 1 - x^{2} - y^{2}.   
\end{gather*}
OSsia
\begin{gather*}
    \left\{\begin{array}{l}
        (x, y) \in D \\
        x^{2} + y^{2} = 1 - e^{t}   
    \end{array}\right.
\end{gather*}
\begin{itemize}
    \item $t > 0 \ \Longrightarrow \ U_t = \emptyset$;
    \item $t = 0 \ \Longrightarrow \ U_t = \{(0, 0)\}$;
    \item $t < 0 \ \Longrightarrow \ U_t$ è una circonferenza di raggio $R = \sqrt{1 - e^{t} }$. 
\end{itemize}
Nel limite in cui $t$ tende a meno infinito allora si ottiene la circonferenza di
raggio $1$ e quindi $U_t$ coinciderà con l'insieme del
$\ni D$. Si possono trovare ora il minimo ed il massimo della funzione nell'insieme $Q$ 

\section{Limiti di funzioni in due variabili}
I limiti in due funzioni funzionano in maniera simile ai limiti in una variaabile 
\begin{align}
    \lim_{(x, y) \to (x_0, y_0)}f(x, y) = L  
\end{align}
Per cui la metrica
\begin{gather*}
    ||(x, y) - (x_0, y_0)|| \to 0 \Longleftrightarrow |f(x_0), L | \to 0
\end{gather*}



\end{document}
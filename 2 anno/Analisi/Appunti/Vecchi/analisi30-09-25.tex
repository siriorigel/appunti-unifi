\documentclass[a4paper, oneside]{article}
\usepackage{graphicx}
\usepackage{amsthm}
\usepackage{amsmath}
\usepackage{amssymb}
\usepackage[a4paper,
            bindingoffset=0.2in,
            left=2cm,
            right=2cm,
            top=2cm,
            bottom=2cm,
            footskip=.25in]{geometry}
\usepackage[italian]{babel}
\usepackage{pgfplots}
\usepackage{tabularx}
\usepackage{tikz}
\usepackage{wrapfig}
\usepackage{color}
\usepackage[d]{esvect}
\definecolor{page}{rgb}{0.129,0.157,0.212}
\pagecolor{page}
\color{white}
\graphicspath{ {./images/} }
\usetikzlibrary{shapes.geometric}
\usetikzlibrary{datavisualization}
\usetikzlibrary{datavisualization.formats.functions}
\usetikzlibrary{patterns}
\pgfplotsset{width=10cm,compat=1.9}

\title{Appunti Analisi}
\author{Tommaso Miliani}
\date{30-09-25}

\begin{document}
\newtheoremstyle{theoremEnv}
                {}          % Space above
                {}          % Space below
                {\slshape}  % Body font
                {}          % Indent amount
                {\bfseries} % Head font
                {.}         % Punctuation after head
                {\newline}         % Space after theorem head
                {}          % Theorem head spec
\theoremstyle{theoremEnv}

\newtheorem{definition}{Definizione}[section]
\newtheorem{theorem}{Teorema}[section]
\newtheorem{lemma}{Proposizione}[section]
\newtheorem{observation}{Osservazione}[section]
\newtheorem{corollary}{Corollario}[theorem]
\newtheorem{example}{Esempio}[section]

\maketitle

\section{Spazi Metrici}
Sul Marcellini Sbordone è buona, mentre sul Pagani Salsa è trattato brevemente. \\
Considerato un insieme $X$, se su questo insieme è definita una
applicazione $d : X X \to d$ si dice che è metrica di $X$ se
\begin{gather*}
    \forall x, y \in X \qquad d(x, y) \geq 0 \Longleftrightarrow x = y \\
    \forall x, y \in X \qquad d(x, y) = d(y, x) \\
    \forall x, y, z \in X \qquad d(x, y) \leq d(x, z) + d(z, y)
\end{gather*}
Uno spazio metrico è una coppia di valori che mi definiscono una
distanza all'interno di un insieme. NOn è necessario che ci sia una struttura algebrica
definita per poter valere questa definizione (ossia non deve necessariamente esserci
una operazione di somma o prodotto per scalari definita).
\begin{example}
    Su $\mathbb{R}$ si può considerare la distanza euclidea:
    \begin{gather*}
        d(x, y) = |x - y| 
    \end{gather*} 
    Un altro esempio
    \begin{gather*}
        d(x, y) = \left\{\begin{array}{l}
            0 \qquad x = y \\
            1 \qquad x \neq y
        \end{array}\right.
    \end{gather*} 
\end{example}
Se l'insieme considerato è uno spazio vettoriale allora è possibile definire una operazione di norma:
\begin{gather*}
    \forall v \in V ||v|| \geq \Longleftrightarrow v = 0 \\
    ||\lambda v || = |\lambda| ||v||, \forall \lambda \in \mathbb{R}, \forall v \in V \\
    \forall u, v \in V \qquad ||u + v|| \leq ||u|| + ||v|| 
\end{gather*}
\begin{definition}
    Si definisce $(V, ||\cdot ||)$ come spazio normato. 
\end{definition}
\begin{example}
    Prendendo come spazio vettoriale $\mathbb{R}^{N}$,
    posso considerare la norma euclidea:
    \begin{gather*}
        ||v|| = \sqrt{\sum v_i^{2} } 
    \end{gather*} 
\end{example}
\begin{lemma}
    UNo spazio normato induce una metrica su $V$. 
    \begin{gather*}
        d(u, v) = ||u - v|| \qquad \forall u, v \in V
    \end{gather*}
\end{lemma}
\begin{definition}
    $(\mathbb{X}, \alpha)$, $(\mathbb{X}, \beta)$ sono due spazi metrici diversi 
    con metrica equivalente: se $\exists k_1, k_2 \in \mathbb{R}$ tale che
    \begin{gather*}
        k_1d_{\beta}(x, y) \leq d_{\alpha}(x, y) \leq k_2 d_{\beta}(x, y)
    \end{gather*}
\end{definition}
\begin{definition}
    Uno spazio vettoriale può ammettere più metriche: la norma  per$\alpha$
    e la norma per $\beta$ sono norme equivalenti se esistono
    $c_1, c_2 \in \mathbb{R}$ tali che sono spazi normati 
    \begin{gather*}
        v \in V \qquad c_1||v||_{\beta} \leq ||v||_{\alpha} \leq c_2 ||v||_{\beta}
    \end{gather*}
\end{definition}
\begin{example}
    Le norme sullo spazio $\mathbb{R}^{N} $  con $x \in \mathbb{R}^{N}$ posso definire la norma $p$ esima, ossia una
    estensione della norma euclidea prendendo le potenze $p$ esime
    della norma euclidea e facendone la radice $p$ esima:
    \begin{gather*}
        ||x ||_{p} = \sqrt{\sum |x_i|^{p} } 
    \end{gather*}
    Con i diversi casi
    \begin{gather*}
        ||x||_1 = \sum |x_1|
    \end{gather*}
\end{example}
\begin{definition}[Palle]
    Da rivedere, copiare disegno. 
    \begin{gather*}
        B_p = \left\{x \in V  : ||x||_p \leq 1\right\}
    \end{gather*}
    Nel caso della norma euclidea la palla unitaria è
    la sfera centrata nell'origine. Nel disegno, quando $ 0 < p \leq 1$
    le sfere disegnate stanno dentro al cerchio mentre
    quelle per cui $p > 1$ stanno fuori dal cerchio e il quadrato è
    il caso limite di $p \to \infty $. 
\end{definition}
\begin{theorem}
    Tutte le norme su $\mathbb{R}^{N}$ sono equivalenti. (ossia 
    se la norma è fatta su di uno spazio finito).
\end{theorem}

\section{Gli spazi metrici infiniti}
\begin{example}
    Dato l'insieme $X = C^{K}([a, b])$.
\end{example}
\begin{gather*}
    X = C^{K}([a, b]) \qquad ||f|| _{C^{0} } = ||f||_{\infty} = \max |f(x) | \qquad f \in C^{0}([a, b]) : f[a, b] \to \mathbb{R} 
\end{gather*}
In questo caso la definizione che si ottiene da Weirestrass è una buona definizione e dunque la norma che induce
\begin{gather*}
    d_{C^{0} }(f, g) = d_{\infty }(f, g) = ||f - g||_{\infty } \\
    d_{L^{1} }(f, g) = \int_{a}^{b} |f - g|\ dx 
\end{gather*}
Sono due metriche non equivalenti sullo stesso spazio. La distanza $L_1$ è la norma dell'integrale
del valore assoluto. Sono entrambe indotte da una norma ma non sono equivalenti. Si possono
infatti trovare delle successioni che mi permettano di dire che non sono equivalenti.
Manca un teorema.

\section{Spazio delle funzioni da $\mathbb{R}^{2}$ a $\mathbb{R}$ }
Un intorno di un punto su di uno spazio è diverso dall'intorno di un punto di $\mathbb{R}$
se io considerassi lo spazio di funzioni allora sarebbe ancora più complicato. Dobbiamo allora
fare alcuni cenni sulla topologia partendo da uno spazio metrico
generico $(X, d)$ e andiamo a definire cosa è uno intorno generico di un punto
in uno spazio metrico generico. 
\begin{definition}
    Per $x_0 \in \mathbb{X}, R > 0$ un intorno aperto di centro $x_0$ di raggio $R$  è
    \begin{gather*}
        B_{R} = \left\{x \in X : d(x_0, x) < R\right\} = U(x_0, R)
    \end{gather*}
    Si definisce invece un intorno chiuso come 
    \begin{gather*}
        B_{R} = \left\{x \in X : d(x_0, x) \leq R\right\}
    \end{gather*}
\end{definition}
\begin{definition}
    $\mathbb{A} \leq \mathbb{X}$ è definito come insieme aperto se
    $\forall a \in \mathbb{A}$ esiste un $R$ tale per cui
    \begin{gather*}
        U(x_0, R) \leq A
    \end{gather*}
\end{definition} 
\begin{definition}
    $\mathbb{D} \leq \mathbb{X}$ è un insieme chiuso se $X \backslash D$ è aperto.
\end{definition}
\begin{theorem}
    SI hanno i seguenti casi a seguito della definizione:
    \begin{itemize}
        \item L'unione di insieme aperti(anche numerabili) è aperta;
        \item L'intersezione finita di insiemi aperti (anc che numerabili) è aperta;
        \item L'unione finita di insiemi chiusi (anche numerabili) è chiusa;
        \item L'intersezione di insiemi chiusi (anche numerabili) è chiusi
    \end{itemize}
\end{theorem}
\begin{definition}
    $x_0 \in \mathbb{X}$ è un punto interno se $\exists R$ tale che
    \begin{gather*}
        U (x_0, R) \leq \mathbb{X} 
    \end{gather*}
\end{definition}
\begin{theorem}[Teorema di Heine-Borel ]
    Se $\mathbb{K} \subset \mathbb{R}^{N}$ è un insieme compatto 
    allora è anche chiuso e limitato. 
\end{theorem}

\section{Le successioni su spazi metrici generali}
\begin{definition}
    Consideriamo una successione $\{x_k\} \leq \mathbb{X}$  è una stringa di numeri
    che è definita all'interno di uno spazio metrico. La successione
    converge ad un valore $x$ se il punto limite appartiene a $\mathbb{X}$ e se
    \begin{gather*}
        \lim_{k \to \infty } d(x_k, x) = 0 
    \end{gather*}
    E' una distanza che ha valore in $\mathbb{R}$ e questo
    limite lo so fare.
\end{definition}
\begin{lemma}
    Se $\exists$ il limite allora è unico.
\end{lemma}
\begin{lemma}
    $d_{\alpha}, d_{\beta}$ sono metriche equivalenti su $\mathbb{X}$ 
    \begin{gather*}
        \{x_k\} \leq \mathbb{X}
    \end{gather*}
    Allora $x_k$ converge in $x$ rispetto alla metrica $d_{\alpha}$ così come
    alla metrica $d_{\beta}$. Questo perché per il teorema visto sulle metriche,
    posso mettere a panino le metriche di beta e alfa per cui se quella fuori va
    a zero allora anche quella nel panino va a zero (così come nel teorema dei
    carabinieri). Per questo su $\mathbb{R}^{N}$ si considera solo la metrica euclidea.  
\end{lemma}

\begin{definition}
    Sia $\{x_k\} \leq \mathbb{X}$ una successione di Cauchy in $(\mathbb{X}, d)$ 
    se $\forall \epsilon > 0$ allora esisterà un $N$ tale per cui
    \begin{gather*}
        d(x_k, x_m) < \epsilon \qquad \forall k, m > N
    \end{gather*}
\end{definition}
\begin{observation}
    Sia $x_k$ una successione di Cauchy, allora $x_k$ converge
    sullo spazio metrico $(\mathbb{X}, d)$ su cui è definita. Non
    è vero che se $x_k$ converge allora sia necessariamente di Cauchy.
\end{observation}
\begin{example}
    Preso l'insieme $\mathbb{X} = C^{1}([a, b]) $ con la distanza $d$ la distanza
    massima 
    \begin{gather*}
        x_k = \sqrt{x^{2} + \frac{1}{k} } x \in [0, 1]
    \end{gather*} 
    Preso questo spazio di funzioni definito in questo modo, dobbiamo
    allora provare che la successione è di Cauchy. Fissato un $\epsilon$ cerco
    un $R$ per determinare una distanza massima:
    \begin{gather*}
        d(x_k, x_m) = \max  |x_k - x_m| 
    \end{gather*}
    Ossia
    \begin{gather*}
        \max \left| \sqrt{x^{2} + \frac{1}{k} } - \sqrt{x^{2} + \frac{1}{m} }  \right| 
    \end{gather*}
    Allora posso dire che se prendessi $k$ e $m$ grandi
    allora, fissato $\epsilon$ si ha la soluzione:
    \begin{gather*}
        \frac{x^{2}+ \frac{1}{k} - x^{2} - \frac{1}{m}  }{\sqrt{x^{2} + \frac{1}{k} } + \sqrt{x^{2} + \frac{1}{m} }  } = \frac{1}{\sqrt{k} } - \frac{1}{\sqrt{m} } < \epsilon
    \end{gather*}
\end{example}



\end{document}
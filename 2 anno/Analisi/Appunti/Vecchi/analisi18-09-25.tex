\documentclass[a4paper, oneside]{article}
\usepackage{graphicx}
\usepackage{amsthm}
\usepackage{amsmath}
\usepackage{amssymb}
\usepackage[a4paper,
            bindingoffset=0.2in,
            left=2cm,
            right=2cm,
            top=2cm,
            bottom=2cm,
            footskip=.25in]{geometry}
\usepackage[italian]{babel}
\usepackage{pgfplots}
\usepackage{tabularx}
\usepackage{tikz}
\usepackage{wrapfig}
\usepackage{color}
\definecolor{page}{rgb}{0.129,0.157,0.212}
\pagecolor{page}
\color{white}
\graphicspath{ {./images/} }
\usetikzlibrary{shapes.geometric}
\usetikzlibrary{datavisualization}
\usetikzlibrary{datavisualization.formats.functions}
\usetikzlibrary{patterns}
\pgfplotsset{width=10cm,compat=1.9}

\title{Appunti analisi 2}
\author{Tommaso Miliani}
\date{18-09-25}

\begin{document}
\newtheoremstyle{theoremEnv}
                {}          % Space above
                {}          % Space below
                {\slshape}  % Body font
                {}          % Indent amount
                {\bfseries} % Head font
                {.}         % Punctuation after head
                {\newline}         % Space after theorem head
                {}          % Theorem head spec
\theoremstyle{theoremEnv}

\newtheorem{definition}{Definizione}[section]
\newtheorem{theorem}{Teorema}[section]
\newtheorem{lemma}{Proposizione}[section]
\newtheorem{observation}{Osservazione}[section]
\newtheorem{corollary}{Corollario}[theorem]
\newtheorem{example}{Esempio}[section]

\maketitle

\section{Equazioni omogenee}
Queste equazioni possono essere riscritte attraverso una opportuna funzione per essere risolte:
\begin{gather*}
    y' = b\left(\frac{y}{x}\right)
\end{gather*}
Ossia posso trasformare la differenziale come funzione di $\frac{y}{x} = t$, quindi cambio varibile
nel seguente modo:
\begin{gather*}
    z(x) = \frac{y(x)}{x} \ \Longrightarrow \ y(x) = xz(x) \ \Longrightarrow \ y'(x) = z + xz'
\end{gather*}
Adesso posso risolvere in funzione di $z$:
\begin{gather*}
    b(z) = z + xz' \ \Longrightarrow \ z' = \frac{b(z) - z}{x}
\end{gather*}
Coni passaggi all'indietro ottengo nuovamente $y(x) = xz(x)$.
\begin{example}
    \begin{gather*}
        y' = \frac{x^{3} + y^{3}  }{xy^{2} }
    \end{gather*}
    Se io definisco $f(x, y)$ questo oggetto è facile
    vedere che è omogenea. Posso ora mettere in evidenza la $x$ sopra
    e dunque ottenere
    \begin{gather*}
        \frac{x^{3}(1 + \frac{y^{3} }{x^{3} }) }{x^{3}(\frac{y^{2} }{x^{2} } ) } = b\left(\frac{y}{x}\right) 
    \end{gather*}
    Ossia si ottiene una funzione in rapporto tra $y$ e $x$, allora posso dire che
    \begin{gather*}
        b(t) = \frac{1 + t^{3} }{t^{2} }
    \end{gather*}
    Allora, data la definizione di $z'$ si ottiene
    \begin{gather*}
        z' = \frac{\frac{1 + z^{3} }{z^{2} } - z}{x} \ \Longrightarrow \ z' = \frac{1}{z^{2} }\frac{1}{x}
    \end{gather*}
    Allora per separazione di variabili possiamo trovare la soluzione come
    \begin{gather*}
        \int z^{2} dz  = \int \frac{dx}{x}
    \end{gather*}
\end{example}

\section{Esistenza locale ed esistenza globale}
\begin{example}
    \begin{gather*}
        \left\{\begin{array}{l}
            y' = -2xy^{2}  \\
            y(0) = -1
        \end{array}\right.
    \end{gather*}
    Essendo questo un problema di Cauchy, possiamo allora risolverlo
    con i metodi che conosciamo; tuttavia si osserva anche che, essendo
    $f(x, y)$ continua, allora anche la differenziale è continua e quindi
    siamo nelle ipotesi del teorema di esistenza ed unicità. Esiste allora
    una sola soluzione per questa equazione e si trova con
    separazione di variabili
    \begin{gather*}
        \int \frac{dy}{y^{2} } = \int -2x \ dx \\
        -\frac{1}{y} = -x^{2} + c \ \Longrightarrow \  c = 1 
    \end{gather*}
    La soluzione è la seguente, con $ x \in \mathbb{R} - \{-1, 1\}$: 
    \begin{gather*}
        y = \frac{1}{x^{2} - 1}
    \end{gather*}
\end{example}
I teoremi di esistenza locale mi dicono che è possibile trovare 
delle soluzioni negli intorni di $x_0$ in quanto sono interessato solo
alle soluzioni vicine a questo intorno. Nei teoremi di esistenza globale invece
io considero l'intero insieme di esistenza della funzione trovando delle soluzioni globali. 
\begin{wrapfigure}{r}{0.4\textwidth}
    \centering
    \caption{Teorema di esistenza globale}
    \begin{tikzpicture}
        \draw[->](-1, 0) -- (4, 0) node[at end, below] {$x$};
        \draw[->](0, -1) -- (0, 3) node[at end, left] {$y$};
        \draw(1, -1) -- (1, 3);
        \draw(3, -1) -- (3, 3);
        \filldraw(1, -1) rectangle (3, 3);
        \draw(-0.1, 1) -- (0.1, 1) node[at start, left] {$y_0$};
        \filldraw(1.6, 0) circle (1pt) node[anchor = north] {$x_0$};
        \filldraw(1, 0) circle (1pt) node[anchor = north east] {$a$};
        \filldraw(3, 0) circle(1pt) node[anchor = north west] {$b$};
        \filldraw(1.6, 1) circle(1pt);
    \end{tikzpicture}    
\end{wrapfigure}
\begin{theorem}[Teorema di esistenza globale]
    Consideriamo il problema di Cauchy del primo ordine
    \begin{gather*}
        \left\{\begin{array}{l}
            y' = f(x, y) \\
            y(x_0) = y_0
        \end{array}\right.
    \end{gather*}
    \begin{enumerate}
        \item $f(x, y)$ e $\frac{\partial f}{\partial y}(x, y)$ sono continue e definite
        $\forall x \in [a, b]$, $y_0 \in \mathbb{R}$;
        \item Esistano due valori positivi $h, k \in \mathbb{R}$ per cui risulta:
        \begin{gather*}
            |f(x, y)| \leq h + k|y| \qquad \forall x \in [a, b], y \in \mathbb{R}
        \end{gather*} 
    \end{enumerate}
    Allora la soluzione è unica e definita su tutto $[a, b]$.
\end{theorem}
\begin{proof}
    Non fatta in classe (si farà? boooh)
\end{proof}

\section{Sistemi di equazioni differenziali}
Un sistema di equazioni differenziali è un sistema i cui membri sono vettori
di funzioni:
\begin{gather*}
    \left\{\begin{array}{l}
        y'_1 = f_1(x, y_1, \dots, y_n) \\
        y'_2 = f_2(x, y_1, \dots, y_n) \\
        \vdots \\
        y'_n = f_n(x, y_1, \dots, y_n)
    \end{array}\right.
\end{gather*}
La soluzione del sistema è allora un vettore di funzioni (chiamata anche \textbf{funzione vettoriale}):
\begin{gather*}
    \begin{pmatrix} y_1(x) \\
    \vdots \\
    y_n(x) \end{pmatrix} \  x \in I, \ I \subset \mathbb{R} : \\
    \left\{\begin{array}{l}
        y'_1 = f_1(x, y_1(x), \dots, y_n(x)) \\
        \vdots  \\
        y'_n = f_n(x, y_1(x), \dots, y_n(x)) 
    \end{array}\right.
\end{gather*}
Allora posso chiamare il vettore
\begin{gather*}
    Y(x) = \begin{pmatrix} y_1(x) \\
    \vdots \\
    y_n(x) \end{pmatrix} \\
\end{gather*}
Tale che $Y'(x)$ contenga tutte le derivate, allora posso chiamare il
vettore funzione come $F(x, Y) = Y'$, allora posso dire che i sistemi di equazioni
differenziali sono tutti risolvibili allo stesso modo. Se si avesse invece delle condizioni
per tutte le derivate fino alla $n - 1$ esima, allora saremmo in presenza di un
sistema di equazioni differenziali di Cauchy. Questo vuol dire che la sua soluzione,
tenuto conto delle considerazioni fatte fino ad ora
\begin{gather*}
    \left\{\begin{array}{l}
        Y'(x) = F(x, Y) \\
        Y(x_0) = y_0
    \end{array}\right.
\end{gather*}

\begin{theorem}[Teorema 1]
    Sia $D \subset \mathbb{R}^{n + 1}$ e sia $F : D \to \mathbb{R}^{n}$,una
    funzione continua e sia $(x_0, y_0)$ un punto interno a $D$, allora 
    il problema di Cauchy (sistema di equazioni differenziali) ha almeno una soluzione definita in un intorno  
    di $x_0$.
\end{theorem}

\begin{theorem}[Teorema 2]
    Nelle ipotesi del teorema precedete supponiamo che
    \begin{gather*}
        \frac{\partial f_i}{\partial y_j} \qquad i, j \in \{1, \dots, n\} 
    \end{gather*}
    Siano continue in $D$ , allora il problema di Cauchy con sistema di equazioni
    differenziali ha una sola soluzione definita nell'intorno di $x_0$.
\end{theorem}

\begin{theorem}[Teorema 3]
    Supponiamo che $f_i(x, y_1, \dots, y_n)$ sia definita per ogni $i = 1,\dots, n$,
    $\forall x \in [a, b]$ e $\forall y_1, \dots, y_n \in \mathbb{R}$ si suppone anche che
    $f_i$ e la derivata parziale rispetto ad ogni $f_i$ rispetto a $y_j$ siano continue
    $\forall y \in [a, b], y_1, \dots, y_n \in \mathbb{R}$. Si suppone anche che $\exists h, k  >0$ tali per cui
    risulti 
    \begin{align}
        ||F(x, Y)|| \leq h + k ||Y||
    \end{align} 
    Con $||Y||$ si indica la norma e ogni soluzione dell'equazione $Y' = F(x , Y)$ 
    è definita su tutto l'intervallo $[a, b]$.
\end{theorem}
\begin{example}[Esempio fondamentale]
    In un sistema di equazioni differenziali di ordine $n$, tale per cui
    \begin{gather*}
        \underset{\heartsuit}{y_1(x) \equiv y(x) \qquad y_2(x) \equiv y'(x)} \qquad y'' = 3xy + 7y' (\star) \\
        \left\{\begin{array}{l}
            y_1' = y_2 \\
            y_2' = 3xy_1 + 7y_2 (\star \star)
        \end{array}\right. 
    \end{gather*}
    Se $y$ risolve $\star$ allora $y_1, y_2$ definite da $\heartsuit$ risolve $\star \star$ viceversa
    se $(y_1, y_2)$ risolvono $\star \star$ allora la funzione $y_1$ ha ordine $\star$.  
\end{example}


\section{Equazioni differenziali omogenee lineari del primo ordine}
\begin{definition}[Equazioni differenziali omogenee lineari]
    Si chiamano equazioni differenziali lineari equazioni del tipo
    \begin{align}
        y^{(n)}  + a(x)_{n - 1}y^{(n - 1)} + \dots + a(x)y = f(x)
    \end{align}
    Se $f(x) $ è zero, allora prendono il nome di \textbf{omogenee}, altrimenti
    \textbf{complete}. Queste equazioni si chiamano lineari perché hanno la 
    caratteristica che riprendono le funzioni lineari negli spazi vettoriali.
\end{definition}
\begin{gather*}
    y' = a(x) y = f(x)   
\end{gather*}
Posso utilizzare la risoluzione vista prima per cui posso eseguire un cambio di variabile
opportuno in modo tale da renderla omogenea
\begin{gather*}
    z' + a(x) z = 0
\end{gather*}
Se $z_1$ e $z_2$ sono soluzioni dell'omogenea, allora anche $\alpha z_1(x)+\beta z_2(x)$
sono soluzioni.  Posso dimostrarlo nel seguente modo:
\begin{gather*}
    (\alpha z_1(x)+\beta z_2(x))' + a(x)(\alpha z_1(x)+\beta z_2(x)) = 0 \\
    \alpha(z_1' + a(x)z_1) + \beta(z_2' + a(x) z_2) = 0
\end{gather*}
Dato che $z_1$ e $z_2$ sono soluzioni, allora è verificata.  \\
Posso ora verificare che esistono soluzioni per l'equazione
generica $z' + a(x)z = 0$. Sia $A(x)$ una primitiva finita senza le costanti 
di $a(x)$ e quindi $A'(x) = a(x)$. Posso eseguire allora la seguente trasformazione: 
\begin{gather*}
    e^{A(x)}z' + a(x)e^{A(x)}z = 0 = (e^{A(x)}z)'     \\
    e^{A(x)}z = c \qquad c \in \mathbb{R} \qquad z = ce^{-A(x)}  
\end{gather*}
Esiste un modo più veloce per poter trovare le soluzioni, questo è dato dalla seguente 
formulazione
\begin{gather*}
    y'e^{A(x)} + a(x)e^{A(x)}y = f(x)e^{A(x)}    \\
    (ye^{A(x)} )' = f(x) e^{A(x)} 
\end{gather*}
Se considero $F(x)$ come primitiva del prodotto a destra, allora
si ottiene che 
\begin{gather*}
    ye^{A(x)} = F(x) + c
\end{gather*}
\begin{align}
    y = ce^{-A(x)} + e^{-A(x)}F(x)   
\end{align}
Si può anche utilizzare un secondo metodo per risolvere le equazioni 
e si chiama \textbf{metodo della variazione delle costanti}.
Penso che $c$ ora sia una funzione e non più una costante, allora dato che
devo considerare una soluzione generale, allora devo considerare $c$ come
funzione. Posso ottenere la derivata
\begin{gather*}
    y' = C'e^{-A(x)} - Ce^{-A(x)}a  \\
    y' + a(x)y = f \ \Longrightarrow \ C' = f(x) e^{A(x)} 
\end{gather*}
Se $C$ ha questa espressione, ossia la primitiva di $C'$, allora 
$C = F(x) + d, d\in \mathbb{R}$, le soluzioni sono allora
\begin{gather*}
    y(x) = C(x)e^{-A(x)} \ \Longrightarrow \ y(x) = (F(x) + d)e^{-A(x)}  
\end{gather*}

\begin{example}[Prima formula]
    \begin{gather*}
        y' + \frac{y}{x}= -\sin x \qquad \left\{\begin{array}{l}
            a(x) = \frac{1}{x} \\
            f(x) = -\sin x\\
            A(x) = \int a(x) \ \Longrightarrow \ \ln |x|
        \end{array}\right. \\
        F(x) = \int f(x) e^{A} \ dx = \int -\sin x e^{A(x)}\ dx = \int -\sin x e^{\ln|x|} \ dx   = \left\{\begin{array}{l}
            x > 0 : \sin x - x\cos x \\
            x < 0 : -\sin x + x\cos x 
        \end{array}\right. 
    \end{gather*}
    Allora si ottiene dalla formula di risoluzione veloce:
    \begin{gather*}
        y(x) = \left\{\begin{array}{l}
            Ce^{-\ln|x|} + e^{-\ln|x|} (\sin x - x\cos x) \qquad x > 0 \\
            Ce^{-\ln|x|} + e^{-\ln|x|} (-\sin x + x\cos x) \qquad x < 0  
        \end{array}\right.
    \end{gather*}
    Si ottengono allora due due soluzioni distinte a seconda di come si
    svolge il valore assoluto.
\end{example}

\section{Spazi di funzioni}
\begin{definition}[Spazi di funzioni]
    Sia $I$ un intervallo su di $\mathbb{R}$ , allora definisco lo spazio delle
    funzioni di da $I \to \mathbb{R}$ lo spazio vettoriale che
    contiene le funzioni con le seguenti proprietà:
    \begin{enumerate}
        \item $\alpha f(x)  + \beta g(x) $ appartiene allo spazio;
        \item $f(x)  \equiv 0$.
    \end{enumerate}
    All'interno di questo spazio vivono anche gli spazi:
    \begin{itemize}
        \item     $C^{0}(I)$, ossia l'insieme dello spazio delle funzioni continue su $I \to \mathbb{R}$;
        \item $C^{1}(I)$ è l'insieme delle funzioni continue derivabili in ogni punto
        la cui derivata è una funzione continua e $\forall x \in I$.
        \item $\vdots$
        \item $C^{(n)}$ è l'insieme delle derivate $n$ esime continue.    
    \end{itemize}
\end{definition}
Si possono ora risolvere le equazioni differenziali omogenee lineari
di ordine $n$-esimo considerando la generica funzione
\begin{gather*}
    y = C^{(n)}(I) 
\end{gather*}
Posso definire l'operatore $E(y) : C^{(n)}(I) \to C^{(0)}(I)  $ come 
l'operatore che trasforma le derivate $n$-esime continue in primitive 
e posso esprimerla (se le soluzioni sono $y_1$ e $y_2$)
\begin{gather*}
    E(\alpha y_1 + \beta y_2) = \alpha E(y_1) + \beta E(y_2)
\end{gather*}
\begin{theorem}
    \begin{enumerate}
        \item L'insieme $V_0$ delle soluzioni delle equazioni differenziali omogenee lineari è uno spazio
        vettoriale di dimensione $n$;
        \item L'insieme delle soluzioni delle equazioni differenziali lineari è 
        l'insieme $\{y(x) + y_f(x) : y \in V_0\}$ con $y_f$ soluzione delle equazioni differenziali lineari.
    \end{enumerate}
\end{theorem}




\end{document}
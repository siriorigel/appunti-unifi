\documentclass[a4paper, oneside]{article}
\usepackage{graphicx}
\usepackage{amsthm}
\usepackage{amsmath}
\usepackage{amssymb}
\usepackage[a4paper,
            bindingoffset=0.2in,
            left=2cm,
            right=2cm,
            top=2cm,
            bottom=2cm,
            footskip=.25in]{geometry}
\usepackage[italian]{babel}
\usepackage{pgfplots}
\usepackage{tabularx}
\usepackage{tikz}
\usepackage{wrapfig}
\usepackage{color}
\usepackage[d]{esvect}
\definecolor{page}{rgb}{0.129,0.157,0.212}
\pagecolor{page}
\color{white}
\graphicspath{ {./images/} }
\usetikzlibrary{shapes.geometric}
\usetikzlibrary{datavisualization}
\usetikzlibrary{datavisualization.formats.functions}
\usetikzlibrary{patterns}
\pgfplotsset{width=10cm,compat=1.9}

\title{Appunti Analisi}
\author{Tommaso Miliani}
\date{21-10-25}

\begin{document}
\newtheoremstyle{theoremEnv}
                {}          % Space above
                {}          % Space below
                {\slshape}  % Body font
                {}          % Indent amount
                {\bfseries} % Head font
                {.}         % Punctuation after head
                {\newline}         % Space after theorem head
                {}          % Theorem head spec
\theoremstyle{theoremEnv}

\newtheorem{definition}{Definizione}[section]
\newtheorem{theorem}{Teorema}[section]
\newtheorem{lemma}{Proposizione}[section]
\newtheorem{observation}{Osservazione}[section]
\newtheorem{corollary}{Corollario}[theorem]
\newtheorem{example}{Esempio}[section]

\maketitle

\section{Dimostrazione teorema volta scorsa}
\begin{theorem}
    Se $(x_0, y_0)$ è un punto di massimo relativo o assoluto per
    $f$ nell'insieme $\mathbb{D}$, allora 
    \begin{itemize}
    \item $(x_0, y_0)$ è interno a $\mathbb{D}$
    e $f$ non è derivabile ;
    \item $(x_0, y_0)$ è interno a $\mathbb{D}$ e $f$ è
     derivabile e $\nabla f(x_0, y_0) = 0$. 
    \item $(x_0 y_0) \in \partial \mathbb{D}$ (ossia è un punto di frontiera). 
    \end{itemize}
\end{theorem}

Da questo teorema deriva il seguente corollario:
\begin{corollary}[Strategia per trovare punti di massimo]
La strategia per trovare dei massimi assoluti di una funzione continua in un insieme
compatto:
\begin{enumerate}
    \item Trovare punti interni critici
    \item Trovare punti interni in cui $f$ non è derivabile
    \item Trovare i punti di massimo di $f$ ristretta alla frontiera
    \item Confrontare tutti i punti trovati ed i valori di $f$ in tali punti.
\end{enumerate}
\end{corollary}

\begin{example}
    Riprendendo lo studio delle seguenti funzioni  
    \begin{gather*}
        f(x, y) = 2xy  \qquad \text{cerchio } x^{2} + y^{2} \leq 4  
    \end{gather*}
    \begin{itemize}
        \item Non esistono punti critici 
        \item L'unico punto in cui il gradiente è nullo è $(0, 0)$.
        \item Studio su $\partial D$ , ossia $\{x^{2} + y^{2} \leq 4  \}$;
    \end{itemize}
    $f$ è iniettiva sulla frontiera solamente se e solo se
    si esprimono i 4 punti del cerchio in cui si interseca la funzione $f(x, y) $:
    I punti possono essere espressi come:
    \begin{gather*}
        2\cos t \sin t \ \Longrightarrow \ g(t) = 2 \cdot  2\cos t \cdot  \sin t
    \end{gather*}
    Per cui i punti di massimo della funzione sono esattamente:
    \begin{gather*}
        g'(t) = 0 \ \Longrightarrow \ t = \frac{\pi}{4}, \frac{3\pi}{4}, \frac{5\pi}{4}, \frac{7\pi}{4}
    \end{gather*}
    Per cui per il primo ed il terzo vale $4$ e per il secondo ed il quarto vale $-4$. Per
    \begin{gather*}
        g(0 ) = 0 \qquad g(2x) = 0
    \end{gather*}
\end{example}


\begin{example}
    Voglio trovare in punti di intersezione tra la funzione 
    \begin{gather*}
        f(x, y) = x^{2}y e^{-(x + y)} \qquad D = \{x \geq 0, y \geq 0 \qquad x + y \leq 4\}  \\
         \begin{tikzpicture}[scale = 0.55]
            \draw[->](0, 0) -- (5, 0);
            \draw[->](0, 0) -- (0, 5);
            \filldraw[opacity = 0.3, color = cyan](0, 0) -- (4, 0) -- (0, 4);
            \node at (1, 1) {$\mathbb{D}$};
            \node at (0, 2) {I};
            \node at (2, 0) {II};
            \node at (2, 2) {III};
        \end{tikzpicture}
    \end{gather*}
    \begin{itemize}
        \item I punti interni in cui $f$ non è derivabile non esistono;
        \item I punti interni in cui $\nabla f = 0$ sono 
        \begin{gather*}
            \{y = 0, x \in [0, 1]\} \cup \{(2, 1)\} \\
            \ \Longrightarrow \ f = 0 \cup f = \frac{4}{e^{3} }
        \end{gather*}
    \end{itemize}
    Adesso voglio studiare i punti sulle tre frontiere del dominio della funzione:
    \begin{enumerate}
        \item In questa frontiera 
        \begin{gather*}
            f(0, y) \qquad f \in [0, 4] 
        \end{gather*}
        Per cui la funzione ristretta a questa frontiera
        \begin{gather*}
            f_{|_{I}} \equiv 0
        \end{gather*}
        \item 
        \begin{gather*}
            f(x, 0) \qquad x \in [0, 4]
        \end{gather*}
        Per cui la funzione ristretta
        \begin{gather*}
            f_{|_{II}} \equiv 0
        \end{gather*}
        \item  In questa frontiera invece considero
        \begin{gather*}
            f(x, 4 - x) \qquad x \in [0, 4]    
        \end{gather*}
        Dove, posto $g(x)  = f(x, 4 - x)$, posso avere che
        \begin{gather*}
            g(x)  = x^{2}(4 - x)e^{-4} \ \Longrightarrow \ g' = 0 \Longleftrightarrow \left\{\begin{array}{l}
                x = 0 \\
                x = \frac{8}{3}
            \end{array}\right.  
        \end{gather*}
        Per cui il punto di massimo è  esattamente $(\frac{8}{3}, \frac{256}{27})$.
    \end{enumerate}
\end{example}

La ricerca di un massimo vincolato, ossia cercando un massimo o di un minimo 
di una funzione: ossia si cerca un massimo di una funzione $f(x, y)$ su di una
funzione vincolo $g(x, y)$  (che è possibile a volte ridurre ad una funzione
ad una variabile) . Tra tutti i punti che soddisfano $g(x, y) = 0$ si trovano quelli
che rendono massima la funzione $f(x, y)$.


\section{Metodo dei moltiplicatori di Lagrange}
\begin{theorem}[Moltiplicatori di Lagrange]
    Sia $G = \{(x, y) : g(x, y) = 0\}$ un vinicolo e sia $(x_0, y_0)$ un
    punto di massimo locale per $f$ ristretta al vincolo $G$. Supponiamo che
    sia $f$ che $g$ siano funzioni $C^{1}$ in un intorno di $(x_0, y_0)$
    e che il $\nabla f(x_0, y_0) \neq 0$. Allora esiste $\lambda_0 \in \mathbb{R}$ 
    tale che il punto $(x_0, y_0, \lambda_0)$ è un punto critico della funzione
    \begin{gather*}
        \mathfrak{L} (x, y, \lambda) := f(x, y) + \lambda g(x, y)
    \end{gather*}
    Si introduce quindi una nuova funzione con una variabile di più in modo
    da cercare i punti candidati ad essere punti di minimo o di massimo vengono
    cercati nei punti critici per questa funzione. Essere punti critici per
    questa funzione significa che 
    \begin{gather*}
        \frac{\partial \mathfrak{L}}{\partial x} = 0 \qquad \qquad \frac{\partial f}{\partial x} (x_0, y_0)  = -\lambda_0 \frac{\partial g}{\partial x}(x_0, y_0) \\
        \frac{\partial \mathfrak{L}}{\partial y} = 0 \qquad \qquad \frac{\partial f}{\partial x} (x_0, y_0)  = -\lambda_0 \frac{\partial g}{\partial y}(x_0, y_0) \\
        \frac{\partial \mathfrak{L}}{\partial \lambda} = 0 \qquad \qquad g(x, y) = 0 
    \end{gather*}
    Se un punto $x_0, y_0$ è un punto di massimo vincolato, allora il gradiente di $g$ è ortogonale
    alle linea di livello e che il gradiente di $f$ deve essere ortogonale alle linee di livello:
    \begin{align}
        \nabla f(x_0, y_0) = - \lambda \nabla g(x_0, y_0)
    \end{align}  
    Il $\nabla g(\overline{x}, \overline{y}  )$ è perpendicolare alla linea
    di livello della funzione $g(x, y)$ che contiene il punto $\overline{x}, \overline{y}$.  
\end{theorem}

\begin{definition}[Definizione di punto di massimo locale vincolato]
    Sia $G = \{(x, y) : g(x, y) = 0\}$ e sia $(x_0, y_0) \in G$. Si dice che
    $(x_0, y_0)$ è un punto di massimo locale vincolato per $f$ sul vincolo $G$ 
    se esiste un intorno $U$ di $(x_0, y_0)$ tale che 
    \begin{align}
        f(x, y) \leq f(x_0, y_0) \quad \forall (x, y) \in U \cap  G
    \end{align} 
\end{definition}


\begin{example}[Punto di minimo vincolato]
    Determinare il punto della curva  $x^{2}y - 16 = 0$
    la cui distanza dal punto $(0, 0)$ è minima: cerco una circonferenza
    $x^{2} + y^{2}$ tangente al vincolo della funzione.  
    \begin{gather*}
        \begin{tikzpicture}
            \draw(-3, 0) -- (3, 0);
            \draw(0, -1) -- (0, 3);
            \draw(-3, 0.2) ..  controls (-1.5, 0.5) and (-0.5, 1.5) .. (-0.2, 3);
            \draw(3, 0.2) ..  controls (1.5, 0.5) and (0.5, 1.5) .. (0.2, 3);
            \filldraw(1.3, 1) circle (1pt); 
            \draw[->](1.3, 1) -- (1.8, 1.6) node[at end, right] {$\nabla g(x_0, y_0)$};
            \draw[->](1.3, 1) -- (0.8, 0.4) node[at end, right] {$\nabla f(x_0, y_0)$};
        \end{tikzpicture}
    \end{gather*}
    I possibili candidati che possono essere punti di massimo e di minimo
    si può costruire una funzione a partire da $f $ e dal vincolo secondo
    il teorema dei moltiplicatori dei Lagrange e con la definizione di massimo
    locale vincolato. 
    \begin{gather*}
        \mathfrak{L} (x, y, \lambda) = x^{2} + y^{2} + \lambda (x^{2}y - 16 )  
    \end{gather*}
    Ossia, secondo il metodo di Lagrange:
    \begin{gather*}
        \frac{\partial \mathfrak{L}}{\partial x} = 0 \qquad 2x + 2\lambda x y = 0 \\
        \frac{\partial \mathfrak{L}}{\partial y} = 0 \qquad 2y + \lambda x^{2} = 0 \\
        \frac{\partial \mathfrak{L}}{\partial \lambda} = 0 \qquad x^{2}y - 16 = 0     
    \end{gather*}
    Da queste condizioni si ricavano le seguenti condizioni:
    \begin{gather*}
        \left\{\begin{array}{l}
            \lambda = -\frac{1}{y} \\
            2y = \frac{1}{y}x^{2} \\
            x^{2}y = 16  
        \end{array}\right. \ \Longrightarrow \ \left\{\begin{array}{l}
            \lambda = -\frac{1}{y} \\
            x = \pm \sqrt{2} y \\
            y = 2
        \end{array}\right. \ \Longrightarrow \ \left\{\begin{array}{l}
            \lambda = -\frac{1}{2} \\
            x = \pm 2\sqrt{2} \\
            y = 2 
        \end{array}\right.
    \end{gather*}
    Se c'è un punto di minimo vincolato allora deve stare su questi due punti: $(2\sqrt{2}, 2 )$ e $(-2\sqrt{2}, 2 )$,
    dunque si ottiene che
    \begin{gather*}
        \min_{G} f = \min_{G \cap C} f
    \end{gather*}
\end{example}

\begin{observation}
    La tecnica si applica anche in dimensioni maggiori di $2$.
\end{observation}

\begin{example}
    Per esempio trovare
    il minimo oppure il massimo di una funzione $f(x, y, z)$ sul
    vincolo $G = \{(x, y, z) : g(x, y, z) = 0\}$. Per esempio trovare il massimo
    o il minimo per 
    \begin{gather*}
        f(x, y, z) = xy^{2}z^{2} \qquad G = \{x^{2} + y^{2} + z^{2} = 1   \}  
    \end{gather*}
    In questi casi il metodo di moltiplicatori di Lagrange è l'unico applicabile. 
    \begin{gather*}
        G = \{x^{2} + y^{2} + z^{2} = 1, x + y + z = \frac{1}{2}    \}
    \end{gather*}
    Con il metodo dei moltiplicatori posso chiamare la prima $g(x, y, z)$ e la
    seconda $h(x, y, z)$, allora la Lagrangiana è
    \begin{gather*}
        \mathfrak{L} (x, y, z, \lambda, \mu) = \lambda g(x, y, z) + \mu h (x, y, z)
    \end{gather*}
\end{example}

\end{document}
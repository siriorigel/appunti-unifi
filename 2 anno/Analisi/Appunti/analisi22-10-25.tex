\documentclass[a4paper, oneside]{article}
\usepackage{graphicx}
\usepackage{amsthm}
\usepackage{amsmath}
\usepackage{amssymb}
\usepackage[a4paper,
            bindingoffset=0.2in,
            left=2cm,
            right=2cm,
            top=2cm,
            bottom=2cm,
            footskip=.25in]{geometry}
\usepackage[italian]{babel}
\usepackage{pgfplots}
\usepackage{tabularx}
\usepackage{tikz}
\usepackage{wrapfig}
\usepackage{color}
\usepackage[d]{esvect}
\definecolor{page}{rgb}{0.129,0.157,0.212}
\pagecolor{page}
\color{white}
\graphicspath{ {./images/} }
\usetikzlibrary{shapes.geometric}
\usetikzlibrary{datavisualization}
\usetikzlibrary{datavisualization.formats.functions}
\usetikzlibrary{patterns}
\pgfplotsset{width=10cm,compat=1.18}

\title{Appunti di Analisi (Bianchini)}
\author{Tommaso Miliani}
\date{22-10-25}

\begin{document}
\newtheoremstyle{theoremEnv}
                {}          % Space above
                {}          % Space below
                {\slshape}  % Body font
                {}          % Indent amount
                {\bfseries} % Head font
                {.}         % Punctuation after head
                {\newline}         % Space after theorem head
                {}          % Theorem head spec
\theoremstyle{theoremEnv}

\newtheorem{definition}{Definizione}[section]
\newtheorem{theorem}{Teorema}[section]
\newtheorem{lemma}{Proposizione}[section]
\newtheorem{observation}{Osservazione}[section]
\newtheorem{corollary}{Corollario}[theorem]
\newtheorem{example}{Esempio}[section]

\maketitle

\section{Funzioni implicite}
Capita di avere la descrizione di una curva come luogo di zeri di una
funzione a due variabili. E' quindi fondamentale capire se questo insieme
definisce una curva o un grafico di funzioni
\begin{gather*}
    \{f(x, y) = 0\}
\end{gather*}
\begin{example}
    \begin{gather*}
        f(x, y) = x^{3} - y +1  \qquad f = 0 \Longleftrightarrow y = x^{3} - 1 
    \end{gather*}
    E' un grafico di funzione
\end{example}
\begin{example}
    \begin{gather*}
        f(x, y) = x^{2} + y^{2} - 1  \qquad x^{2} + y^{2} = 1  
    \end{gather*}
    globalmente non è un grafico di funzione. Però se ragionassi localmente posso dire
    che questo oggetto è un grafico di funzione se scegliessi in modo
    accorto la variabile dipendente e quella indipendente. Ossia per la circonferenza, per ogni
    arco, devo scegliere se è funzione per la $x$ o per la $y$ in modo tale che possa definire la funzione. 
\end{example}

\begin{theorem}[Teorema del Dini]
    Considerato un insieme aperto $\mathbb{A}$ e si considera una funzione
    \begin{gather*}
        F: \mathbb{A} \to \mathbb{R} \qquad f \in C^{1}(\mathbb{A}) 
    \end{gather*}
    Preso un punto $P_0 \equiv (x_0, y_0) \in \mathbb{A}$ in modo tale che $F(x_0, y_0) = 0$ in modo
    tale che il punto sia regolare. Allora $\exists U$ intorno di $x_0$ e $V$ intorno di $y_0$
    tale che l'equazione tra la linea di livello zero, definisce
    un grafico di funzione $y = f(x)$ oppure $x = g(y)$ nell'insieme $U \times V$ intorno di $P_0$.  
    In particolare
    \begin{itemize}
        \item Se $F_y(x_0, y_0) \neq 0$ allora $\exists !y = f(x) : U \to V$ tale che $F(x, f(x)) = 0$.
        Allora $f \in C^{1}(U)$ e 
        \begin{gather*}
            f'(x) = - \frac{F_x(x, f(x) )}{F_y(x, f(x) )} \quad \forall x \in U^{\circ} 
        \end{gather*} 
        \item Se $F_x(x_0, y_0) \neq 0$ allora $\exists !x  = g(y) : V \to U$ tale che $F(g(y), y) = 0$.
        Allora $g \in C^{1}(V)$ e
        \begin{gather*}
            g'(y) = -\frac{F_y(g(y), y)}{F_x(g(y), y)} \quad \forall y \in V^{\circ} 
        \end{gather*} 
    \end{itemize}
\end{theorem}
\begin{proof}
    Dato $P_0$ un punto regolare e si suppone che la  derivata parziale rispetto
    ad $y$ sia 
    \begin{gather*}
        \frac{\partial f}{\partial y} \neq  0 \qquad  F_Y(x_0, y_0) >0
    \end{gather*}
    Supponendo che esista $U$ un intorno di $x_0$ e $V$ un intorno di $y_0$ e
    si dimostra che 
    \begin{gather*}
        F(x_0, y_0 - \delta) < 0 \forall x \in U \\
        F(x_0, y_0 + \delta) > 0 \forall x \in U
    \end{gather*}
    Si deve dunque avere che
    \begin{gather*}
        F_y(x, y) > 0 \forall (x, y) \in \mathbb{R}
    \end{gather*}
    Ossia 
    \begin{gather*}
        F_y \in C^{2} \ \Longrightarrow \ \exists \text{rettangolo} : P_0 \in W \times V    
    \end{gather*}
    Tale che $W$ sia intorno di $x_0$ e $V$ intorno di $y_0$.
    Quindi $\forall x \in W$ fissato $F(x, y)$ come funzione di $y$ è strettamente
    crescente in funzione di $F(x_0, y)$ è strettamente crescente se $F(x_0, y_0) = 0$ 
    Allora si deve avere che 
    \begin{gather*}
        F(x_0, y_0 - \delta)  <0 \quad F(x, y_0 + \delta) > 0 
    \end{gather*}
    Per la permanenza del segno deve esistere $U$ intorno di $x_0, U \subset W$ e in modo
    tale che
    \begin{gather*}
        F(x_0, y_0 - \delta)  <0 \quad F(x, y_0 + \delta) > 0  \ \forall x \in U
    \end{gather*} 
    

    Si dimostra ora la seconda parte del teorema. Si vuole dimostrare che
    \begin{gather*}
        \exists ! f : U \to V : F(c, f(x) ) = 0 \ \forall x \in U
    \end{gather*}
    Si considerare che $\forall x \in U$ si ha che
    \begin{gather*}
        y \to F(x, y) \nearrow
    \end{gather*}
    Con
    \begin{gather*}
        F(x, y _0 -\delta) < 0 \ \forall x \in U \\
        F(x, y_0 + \delta) > 0 \ \forall x \in U \\
        F \in C^{0}  
    \end{gather*}
    Per il teorema dell'esistenza degli zeri
    \begin{gather*}
        \forall x \in U \ \exists y \in V : F(x, y) = 0
    \end{gather*}
    Questo non conclude la dimostrazione in quanto potrebbero esistere più zeri,
    dunque, visto che è strettamente crescente e dunque monotona,
    per la monotonia, esiste un solo zero, ossia $g = f(x) $. \\
    Si dimostra ora il terzo passo del teorema. Dato che si è visto che
    è un grafico, vogliamo dimostrare che $f$ sia $C^{1}$ passando
    dalla dimostrazione per cui $f \in C^{0}$. Fisso $x_1 \in U$ in modo tale
    da considerare $ x \in U$ e considero la funzione ausiliaria 
    \begin{gather*}
        G(t) = F((1 - t)x_1 + tx, (1 - t)f(x_1) + tf(x) ) = F(\psi, \eta)
    \end{gather*}   
    Chiamo la prima variabile $\psi$ e la seconda $\eta$. 
    $\psi$ è la parametrizzazione del segmento $x_0$ fino a $x_1$ 
    mentre $\eta$ la parametrizzazione del segmento tra $f(x)$ e $f(x_1)$.
    Si può dire che
    \begin{gather*}
        G \in C^{1}  
    \end{gather*} 
    per il teorema della funzione implicita ($F \in C^{1} $ ), si osserva che se si calcola
    \begin{gather*}
        G(0) = F(x_1, f(x_1)) = 0 \\
        G(1) = F(x, f(x)) = 0
    \end{gather*} 
    Poiché sono punti che stanno su di  una linea di livello. Per il teorema di Rolle
    esiste allora un punto che prende il nome di $\tau \in (0, 1)$ tale che
    $G'(\tau) = 0$. 
    \begin{gather*}
        G'(\tau)  = F_x(\psi, \eta) \cdot  (x - x_1) + F_y(\psi, \eta) (f(x)  - f(x_1)) = 0
    \end{gather*} 
    Poiché $\tau$ è tale che per $\psi, \eta$, dato che sono funzioni di $t$, la funzione si annulli. 
    \begin{gather*}
        (f(x) - f(x_1)) (F_y((1 - \tau)x_{1} + \tau x, (1 - \tau) f(x_1) + \tau f(x) )) = -(x - x_1)(F_x(\psi, \eta))
    \end{gather*}
    Voglio ora fare il limite per $x \to x_1$ e dimostrare che il primo membro tenda a zero:
    posso farlo maggiorandolo con una quantità infinitesima
    \begin{gather*}
        (f(x) - f(x_1)) \leq |x - x_1| \max_{(x, y) \in \mathbb{R}}\left|\frac{F_x(x, y)}{F_y(x, y)}\right| \leq |x - x_1| \frac{\max_{(x, y) \in \mathbb{R}} |F_x(x, y)|}{\min_{(x, y) \in \mathbb{R}} |F_y(x, y)|}
    \end{gather*}
    Il denominatore è il minimo in quanto $F_y \in C^{0}, F_y \neq  \in \mathbb{R} $ poiché la funzione è $C^{1}$.  
    Il primo oggetto è un infinitesimo per $x \to  x_1$ mentre il secondo oggetto
    è limitato, allora
    \begin{gather*}
        \lim_{x \to x_1} f(x)  = f(x_1) \ \Longrightarrow \ f \in C^{0}(U)  
    \end{gather*} 
    Manca adesso l'ultimo pezzo del teorema, ossia la dimostrazione per
    la formula della derivata. La derivata per un punto $x_1$, se esiste è
    data dal rapporto incrementale 
    \begin{gather*}
        f'(x_1) = \lim_{x \to x_1} \frac{f(x)  - f(x_1)}{x - x_1} = \lim_{x \to x_1} -\frac{F_x(\psi, \eta)}{F_y(\psi, \eta)} 
    \end{gather*}
    Si osserva che le derivate parziali rispetto a $F_x$ e $F_y$ sono continue, infatti per
    $x \to x_1$ si osserva che
    \begin{gather*}
        \psi = (1 - \tau) x_1 + \tau x \to x_1  \\
        \eta = (1 - \tau)f(x_1) + \tau f(x) \to f(x_1)
    \end{gather*}
    Adesso ho entrambe le derivate continue e dunque posso esplicitare il limite del rapporto incrementale
    \begin{gather*}
        - \frac{F_x(x_1, f(x_1))}{F_y(x_1, f(x_1))}
    \end{gather*}
    E' quindi dimostrato
\end{proof}

\begin{observation}
    Il teorema del DIni vale anche se $F_y(x_0, y_0) \neq 0$ e solo $F_y \in C^{0}$
    e anche se $F_x(x_0, y_0)$ e solo se $F_x \in C^{0}$.   
\end{observation}

\begin{theorem}[Teorema del DIni in tre variabili]
    Il teorema del dini ha una formulazione in $n$ variabili, anche se
    in questo corso non ci interessa. In tre variabili si tratta di 
    grafici di superfici e non più di un grafico su di un piano. In $n$
    variabili sarebbe la superficie di un iperpiano in dimensione $n - 1$. 
    Preso una funzione \begin{gather*}
        F(x, y, z) \in C^{(1)}(\mathbb{A}) \quad \mathbb{A} \subset \mathbb{R}^{3} 
    \end{gather*} 
    Con $\mathbb{A}$ un insieme aperto e tae che sia un punto 
    \begin{gather*}
        P_0 \equiv  (x_0, y_0, z_0) \in \mathbb{A} \ \Longrightarrow \ \exists \text{intorno di } P_0 \leq \mathbb{A} 
    \end{gather*}
    Tale che
    \begin{gather*}
        \left\{F(x, y, z) = 0\right\}  \cap  \text{ intorno di } P_0 \text{ è grafico } Z = f(x, y)
    \end{gather*}
    Inoltre $f \in C^{1}$ e 
    \begin{gather*}
        \frac{\partial f}{\partial x}(x, y) = - \frac{F_x(x, y, f(x, y))}{F_z(x, y, f(x, y))} \\
        \frac{\partial f}{\partial y}(x, y) = - \frac{F_y(x, y, f(x, y))}{F_z(x, y, f(x, y))} 
    \end{gather*} 
    localmente vicino a $(x_0, y_0)$. 
\end{theorem}


\begin{observation}
    Si può dire, geometricamente, che una curva non è un grafico di funzione
    \begin{gather*}
        \begin{tikzpicture}[scale = 0.6]
            \draw[->](0, 0) -- (4, 0);
            \draw[red](2, -2) -- (2, 2);
        \end{tikzpicture} \qquad \begin{tikzpicture}
            \draw[->](1, 0) -- (3, 0);
            \draw[red](2, 1) .. controls (1, 0) .. (2, -1);
        \end{tikzpicture} \quad \begin{tikzpicture}
            \draw(-1, 1) -- (1, -1);
            \draw(-1, -1) -- (1, 1);
            \filldraw(0, 0) circle (1pt) node[anchor = east] {$P_0$};
        \end{tikzpicture}
    \end{gather*}
    Non sono grafici di $y = y(x)$ e non sono grafici di $x = x(y)$.
    Oppure se il grafico presenta un bivio in un intorno di $P_0$.
\end{observation}

\begin{observation}
    Se $r(t)$ è una curva regolare per $t \in I$ allora $\forall t_0 \in I^{\circ}$ il \textbf{sostegno}
    è localmente un grafico (ossia il disegno, la curva nel piano) per il teorema del Dini. Provare a dimostrarlo con
    quel teorema.
\end{observation}

\begin{observation}
    Se una funzione è $F \in C^{2}$ e vale il teorema del Dini, allora
    anche le sottofunzioni $f, g$ sono $ \in C^{2}$.  
\end{observation}

Dallo sviluppo di Taylor per una funzione 
\begin{gather*}
    f(x)  = f(x_0) + f'(x_0)(x - x_0) + \frac{f''(x_0)}{2}(x - x_0)^{2} + o((x - x_0)^{2} )
\end{gather*}
Posso trovare con il teorema del Dini la derivata prima
\begin{gather*}
    f(x)  = y_0  - \frac{F_x(x, f(x) )}{F_y(x, f(x) )}
\end{gather*}
Per trovare la derivata seconda devo necessariamente derivare
\begin{gather*}
    f''(x) = \left(- \frac{F_x(x, f(x) )}{F_y(x, f(x) )}\right)' = - \frac{(F_x(x, f(x) ))' F_y(x_0, f(x) ) - F_x(x, f(x) )(F_y(x, f(x) ))'}{(F_y(x, f(x) ))^{2} } = \\
    -\frac{1}{(F_y)^{2} } \left((F_{x, x} + F_{x, y} f')F_y - F_x(F_{y, x} + F_{y, y}f')\right)_{x, f(x) }
\end{gather*}
Dato che $F \in C^{2}$, allora le derivate miste sono uguali. QUesta derivazione
non è applicata negli esercizi ma si deriva esplicitamente la funzione senza passare
da questa definizione.

\begin{example}
    \begin{gather*}
        F(x, y) = x^{2} - x^{4} - y^{2} =    
    \end{gather*}
    Si ricava che 
    \begin{gather*}
        DF = \left(\begin{array}{c}
            2x - 4x^{3} \\
            -2y 
        \end{array}\right) = \left(\begin{array}{c}
            x(1 - 2x) \\
            -y
        \end{array}\right)
    \end{gather*}

    \begin{gather*}
        \begin{tikzpicture}
            \begin{axis}[grid=major, domain=-5:5,
            y domain=-5:5]
            \addplot3[
            surf,
            shader=interp,
            samples=50,
            ] {x^2 - x^4 - y^2   };
            \end{axis}
        \end{tikzpicture}
    \end{gather*}
\end{example}





\end{document}
\documentclass[a4paper, oneside]{article}
\usepackage{graphicx}
\usepackage{amsthm}
\usepackage{amsmath}
\usepackage{amssymb}
\usepackage[a4paper,
            bindingoffset=0.2in,
            left=2cm,
            right=2cm,
            top=2cm,
            bottom=2cm,
            footskip=.25in]{geometry}
\usepackage[italian]{babel}
\usepackage{pgfplots}
\usepackage{tabularx}
\usepackage{tikz}
\usepackage{wrapfig}
\usepackage{color}
\usepackage[d]{esvect}
\definecolor{page}{rgb}{0.129,0.157,0.212}
\pagecolor{page}
\color{white}
\graphicspath{ {./images/} }
\usetikzlibrary{shapes.geometric}
\usetikzlibrary{datavisualization}
\usetikzlibrary{datavisualization.formats.functions}
\usetikzlibrary{patterns}
\pgfplotsset{width=10cm,compat=1.9}

\title{Appunti di analisi}
\author{Tommaso Miliani}
\date{15-10-25}

\begin{document}
\newtheoremstyle{theoremEnv}
                {}          % Space above
                {}          % Space below
                {\slshape}  % Body font
                {}          % Indent amount
                {\bfseries} % Head font
                {.}         % Punctuation after head
                {\newline}         % Space after theorem head
                {}          % Theorem head spec
\theoremstyle{theoremEnv}

\newtheorem{definition}{Definizione}[section]
\newtheorem{theorem}{Teorema}[section]
\newtheorem{lemma}{Proposizione}[section]
\newtheorem{observation}{Osservazione}[section]
\newtheorem{corollary}{Corollario}[theorem]
\newtheorem{example}{Esempio}[section]

\maketitle

\section*{L'Ottimizzazione: ricerca dei valori massimi e minimi delle funzioni}
\section{Ricerca dei valori massimi e minimi liberi}
Data una funzione $f(x, y)$ e voglio cercare il massimo di questa
funzione quando $(x, y) \in \mathbb{A}$. Se $\mathbb{A}$ è un insieme aperto,
(e dunque ogni punto di $\mathbb{A}$ è interno ad esso), allora si parla
di ricerca di \textbf{massimi liberi}. Se invece $\mathbb{A}$ non fosse aperto (ossia
$\mathbb{A}$ fosse una curva o una superficie), allora si parla di
\textbf{massimi vincolati}.

\begin{example}
    Siano $P_1, P_2, P_3$ punti del piano $xy$, si vuole trovare il punto
    del piano che rende minima la somma delle distanze dai tre punti.
    La funzione $f(x, y)$ è la distanza di $(x,y)$ da $P_1$ sommata
    alle distanze degli altri due punti. In questo caso $\mathbb{A} \equiv \mathbb{R}^{2}$:
    \begin{gather*}
        f(x, y) = ||(x, y) - P_1|| +||(x, y) - P_2|| + ||(x, y) - P_3||
    \end{gather*}
\end{example}

Un problema di ottimizzazione molto comune è quello dei problemi
geometrici:
\begin{example}
    Tra tutti i parallelepipedi di superficie 1, determinare il parallelepipedo
    con volume massimo. Posso definire la superficie che caratterizza 
    l'area come
    \begin{gather*}
        S(x, y, z)  =2xy + 2xz + 2yz = 1
    \end{gather*}
    L'intersezione tra la superficie e il primo ottante di $\mathbb{R}^{3}$. 
\end{example}


\begin{definition}[Massimo assoluto]
    Data una funzione $f$ definita in $\mathbb{A}$ con valori in $\mathbb{R}$,
    dato un punto $(x_0, y_0) \in \mathbb{A}$, dico che $x_0$ e $y_0$ è un
    punto di \textbf{massimo assoluto} per $f$ in $\mathbb{A}$ e che $f(x_0, y_0)$ è il massimo assoluto
    se 
    \begin{align}
        f(x_0, y_0) \geq f(x, y) \qquad \forall (x, y) \in \mathbb{A}
    \end{align}
\end{definition}

\begin{definition}[Massimo relativo]
    Data una funzione $f$ definita in $\mathbb{A}$ con valori in $\mathbb{R}$,
    dato un punto $(x_0, y_0) \in \mathbb{A}$, dico che $x_0$ e $y_0$ è un
    punto di \textbf{massimo relativo}, o \textbf{locale},  per $f$ in $\mathbb{A}$ e che $f(x_0, y_0)$ è il massimo relativo
    se $\exists$ un intorno $U$ di $(x_0, y_0)$ tale che
    \begin{align}
        f(x_0, y_0) \geq f(x, y) \qquad \forall (x, y) \in U
    \end{align}
\end{definition}
Le definizioni per il minimo sono del tutto analoghe
\begin{definition}[Minimo assoluto]
    Data una funzione $f$ definita in $\mathbb{A}$ con valori in $\mathbb{R}$,
    dato un punto $(x_0, y_0) \in \mathbb{A}$, dico che $x_0$ e $y_0$ è un
    punto di \textbf{minimo assoluto} per $f$ in $\mathbb{A}$ e che $f(x_0, y_0)$ è il minimo assoluto
    se 
    \begin{align}
        f(x_0, y_0) \leq f(x, y) \qquad \forall (x, y) \in \mathbb{A}
    \end{align}
\end{definition}

\begin{definition}[Minimo relativo]
    Data una funzione $f$ definita in $\mathbb{A}$ con valori in $\mathbb{R}$,
    dato un punto $(x_0, y_0) \in \mathbb{A}$, dico che $x_0$ e $y_0$ è un
    punto di \textbf{minimo relativo}, o \textbf{locale},  per $f$ in $\mathbb{A}$ e che $f(x_0, y_0)$ è il minimo relativo
    se $\exists$ un intorno $U$ di $(x_0, y_0)$ tale che
    \begin{align}
        f(x_0, y_0) \leq f(x, y) \qquad \forall (x, y) \in U
    \end{align}
\end{definition}
Sorgono due problematiche per i punti di massimo e minimo:
\begin{itemize}
    \item Esiste il punto di massimo o minimo?
    \item Come posso trovare questi punti?
\end{itemize}
Posso dimostrare l'esistenza dei punti di massimo e minimo secondo l'analogo
del teorema di Weierstrass per le funzioni a due variabili
\begin{theorem}[Teorema di Weierstrass]
    Se $\mathbb{A}$ è chiuso e limitato e se $f$ è continua in $\mathbb{A}$,
    allora $\exists$ sia il massimo assoluto e anche il minimo assoluto. 
\end{theorem}

Si elaborano adesso delle condizioni analitiche necessarie o sufficienti
per capire se un punto è di minimo o di massimo.  La prima condizione e
l'analoga del teorema di Fermat per le funzioni a due variabili
\begin{theorem}[Teorema di Fermat]
    Se $(x_0, y_0)$ è un punto di massimo relativo e se $(x_0, y_0) \in \mathbb{A}$
    e se $f$ è derivabile in $(x_0, y_0)$, allora 
    \begin{align}
        Df(x_0, y_0) = 0
    \end{align}
\end{theorem}
\begin{proof}
    Considerato $\phi_1(x) = f(x, y_0)$, ossia la funzione che
    determina l'aumento sulle $x$. Dato che $\phi_1$ è derivabile, $x_0$ è un punto
    di massimo relativo per la funzione di una variabile $\phi_1(x)$. $\phi_1$ è
    derivabile in $x_0$ e $x_0$ è interno al dominio di $\phi_1$, valgono 
    allora tutte le condizioni per l'applicazione del teorema di Fermat per
    le funzioni ad una variabile, dunque, applicandolo:
    \begin{gather*}
        \ \Longrightarrow \ \phi_1'(x_0) = 0 \qquad \phi_1'(x_0) = \frac{\partial f}{\partial x}(x_0, y_0) 
    \end{gather*} 
    Analogamente dimostro che 
    \begin{gather*}
        \frac{\partial f}{\partial y}(x_0, y_0) = 0 
    \end{gather*}
\end{proof}

\begin{definition}[Punto critico]
    Un punto $(x_0, y_0)$ in cui $Df(x_0, y_0)$ si annulla, si chiama
\textbf{punto stazionario} o \textbf{punto critico}.
\end{definition}

\begin{example}[Esempio di punto critico]
    Data una funzione
    \begin{gather*}
        f(x, y) = x^{2} + y^{2}  
    \end{gather*}
    Il punto $(0, 0)$ è critico ed è un punto di minimo relativo.
\end{example}

\begin{example}[Un punto critico non sempre è un massimo o un minimo]
    Data la funzione
    \begin{gather*}
        f(x, y) = x^{2} - y^{2}  
    \end{gather*}
    Il punto $(0, 0)$ è critico ma non è né di massimo relativo né di minimo
    relativo. Questo è un prototipo di punto non di minimo e neanche di massimo. 
    Infatti, se restringessi lo studio solamente alla funzione
    \begin{gather*}
        f(x, 0) = x^{2} 
    \end{gather*}
    Sarebbe un punto di minimo, ma se fissassi invece la $x$:
    \begin{gather*}
        f(0,  y) = -y^{2} 
    \end{gather*}
    Avrei un punto di massimo locale. Questo punto prende il nome 
    di \textbf{punto di Sella} poiché è un punto critico che si comporta in modo ambivalente:
    la sua matrice Hessiana è dunque negativa. 
\end{example}

\begin{observation}[Punti non critici]
    Un punto di minimo locale o massimo può anche essere non critico
    nel caso in cui per la generica funzione $f(x, y)$ non esiste
    il gradiente in quel punto specifico.
\end{observation}

\begin{example}
    La funzione norma
    \begin{gather*}
        f(x, y) = ||(x, y)||
    \end{gather*}
    Il punto $(0, 0)$ è un punto di minimo assoluto, ma
    il gradiente non esiste in quel punto e dunque non è un
    punto critico. (E' essenzialmente lo stesso concetto nelle
    funzioni ad una sola variabile).
\end{example}

\begin{observation}
    Supponiamo che $f$ sia $C^{2}$  in un intorno del
punto critico $(x_0, y_0)$. Lo sviluppo di Taylor mi permette di
dire che
\begin{gather*}
    f(x, y) = f(x_0, y_0)+ 0 + \frac{1}{2}\left<D^{2}f(x, y)(x - x_0, y- y_0), (x - x_0, y - y_0)\right> + o\left(||(x - x_0, y - y_0)||\right)^{2} 
\end{gather*}
Vado allora a studiare la matrice Hessiana, per il teorema di Schwartz, se
$f$ è $C^{2}$ questa è simmetrica. Devo allora studiare il segno del prodotto scalare.
Se $A$ è una matrice $n \times n$ simmetrica e $h \in \mathbb{R}^{n} $, si
considera la forma quadratica
\begin{gather*}
    q(h) = \left<Ah, h  \right> 
\end{gather*} 
\begin{itemize}
    \item Definita positiva: se $q(h) > 0 \ \forall h \neq 0$ allora ho un punto di minimo;
    \item Semidefinita positiva: se $q(h) \geq 0  \ \forall h$;
    \item Definita negativa: se $q(h) < 0 \ \forall h \neq 0$ allora ho un punto di massimo;
    \item Semidefinita negativa: se $q(h) \leq 0  \ \forall h$;
    \item Indefinita: se $\exists h_1, h_2 \neq  0$ tale che
    \begin{gather*}
        q(h_1) > 0 \qquad q(h_2) < 0
    \end{gather*}
    E dunque ho un punto di Sella.
\end{itemize}
La funzione nel minimo quindi presenta l'errore (sottoforma di o piccolo),
che non è trascurabile nel caso in cui la matrice $A$ sia semidefinita: dunque devo
ricorrere al teorema seguente per questo caso specifico.
\end{observation}

\begin{theorem}[Condizione necessaria per l'esistenza di un punto di minimo (analoga per un massimo)]
    Se $f : \mathbb{A} \to \mathbb{R}$ è $C^{2}$ in un intorno di $(x_0, y_0)$
    e questo punto $(x_0, y_0)$ è un punto di minimo relativo, ed è
    interno ad $\mathbb{A}$. allora la forma quadratica 
    associata alla matrice
    \begin{align}
        h \to \left< D^{2}f(x_0, y_0)h, h  \right> 
    \end{align}  
    è semidefinita positiva
\end{theorem}
\begin{proof}
    Sia $h \in \mathbb{R}^{2}$ con $h \neq 0$, si considera la funzione
    \begin{gather*}
        \phi(t) =  f(x_0 + th_1, y_0 + th_2) \qquad h = \begin{pmatrix}
            h_1 \\
            h_2
        \end{pmatrix}
    \end{gather*}
    Se $t = 0 \Longleftrightarrow (x_0, y_0)$, scelgo $h \neq 0$ solo
    perché voglio che il vettore identifichi una direzione. 
    \begin{gather*}
        \phi'(0) = 0 \qquad \phi''(0) \geq 0
    \end{gather*} 
    Perché ho imposto che per $t = 0$, la funzione presenti un minimo. Adesso,
    se facessi la deriavata della funzione $\phi(t)$:
    \begin{gather*}
        (\phi(t))' = \left(f(x_0 + th_1, x_0 + th_2)\right)' = \\
        \frac{\partial f}{\partial x}(x_0 + th_1, x_0 + th_2)h_1   + \frac{\partial f}{\partial y}(x_0 + th_1, x_0 + th_2)h_2 
    \end{gather*}
    Posso allora fare la derivata seconda della funzione
    \begin{gather*}
        (\phi(t))'' = \left(\frac{\partial f}{\partial x}(x_0 + th_1, y_0 + th_2)h_1 + \frac{\partial f}{\partial y}(x_0 + th_1, y_0 + th_2)h_2\right)'  
    \end{gather*}
    Ossia
    \begin{gather*}
        \left(\frac{\partial f}{\partial x}(x_0 + th_1, x_0 + th_2) \right)'h_1 + \left(\frac{\partial f}{\partial y}(x_0 + th_1, x_0 + th_2) \right)'h_2 = \\
        \frac{\partial^{2} f }{\partial x^{2} } (x_0 + th_1, x_0 + th_2)h_1^{2} + \frac{\partial^{2}  f}{\partial y\partial x }(x_0 + th_1, x_0 + th_2)h_1h_2 + \\
        \frac{\partial^{2}  f}{\partial y\partial x } (x_0 + th_1, x_0 + th_2)h_1 h_2 + \frac{\partial^{2}  f}{\partial y^{2} }(x_0 + th_1, x_0 + th_2)h_2^{2} 
    \end{gather*}
    Se la calcolassi ora per $t = 0$, si ha che 
    \begin{gather*}
        \left< D^{2}f(x_0, y_0)h, h  \right> 
    \end{gather*}
\end{proof}

\begin{theorem}[Condizione sufficiente]
    Sia $f : \mathbb{A} \to \mathbb{R}$ e sia $(x_0, y_0) \in \mathbb{A}$
    e sia $f$ $C^{2}$ in un intorno di quel punto. 
    Supponiamo che $(x_0, y_0)$ sia un punto critico della funzione. Allora
    \begin{enumerate}
        \item Se $D^{2}f(x_0, y_0)$ è definita positiva $\ \Longrightarrow \ (x_0, y_0)$ è 
        un punto di minimo relativo.
        \item Se $D^{2}f(x_0, y_0)$ è definita negativa $\ \Longrightarrow \ (x_0, y_0)$ è
        un punto di massimo relativo.
        \item Se $D^{2}f(x_0, y_0)$ è indefinita $\ \Longrightarrow \ (x_0, y_0)$ non è né
        un punto di massimo né di minimo relativo, ma è un punto di sella.   
    \end{enumerate} 
\end{theorem}
\begin{proof}
    Si dimostrano le tre:
    \begin{enumerate}
        \item

        \item
        
        \item La dimostrazione segue dal principio di condizione necessaria
        di esistenza del punto di minimo relativo.
    \end{enumerate}
\end{proof}


\end{document}
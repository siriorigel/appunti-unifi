\documentclass[a4paper, oneside]{article}
\usepackage{graphicx}
\usepackage{amsthm}
\usepackage{amsmath}
\usepackage{amssymb}
\usepackage[a4paper,
            bindingoffset=0.2in,
            left=2cm,
            right=2cm,
            top=2cm,
            bottom=2cm,
            footskip=.25in]{geometry}
\usepackage[italian]{babel}
\usepackage{pgfplots}
\usepackage{tabularx}
\usepackage{tikz}
\usepackage{wrapfig}
\usepackage{color}
\usepackage[d]{esvect}
\usepackage{chemfig}
\usepackage{mhchem}
\definecolor{page}{rgb}{0.129,0.157,0.212}
\pagecolor{page}
\color{white}
\graphicspath{ {./images/} }
\usetikzlibrary{shapes.geometric}
\usetikzlibrary{datavisualization}
\usetikzlibrary{datavisualization.formats.functions}
\usetikzlibrary{patterns}
\pgfplotsset{width=10cm,compat=1.18}

\title{Appunti Analisi (Bianchi)}
\author{Tommaso Miliani}
\date{20-11-25}

\begin{document}
\newtheoremstyle{theoremEnv}
                {}          % Space above
                {}          % Space below
                {\slshape}  % Body font
                {}          % Indent amount
                {\bfseries} % Head font
                {.}         % Punctuation after head
                {\newline}  % Space after theorem head
                {}          % Theorem head spec
\theoremstyle{theoremEnv}

\newtheorem{definition}{Definizione}[section]
\newtheorem{theorem}{Teorema}[section]
\newtheorem{lemma}{Proposizione}[section]
\newtheorem{observation}{Osservazione}[section]
\newtheorem{corollary}{Corollario}[theorem]
\newtheorem{example}{Esempio}[section]
\newtheorem{remark}{Enunciato}[section]

\maketitle

\section{}
\begin{proof}
    Le funzioni $h_1$ e $h_2$ sono funzioni continue, dunque 
    posso inscatolare la funzione in un rettangolo $R$. Si definisce 
    $\tilde{f}$ come la funzione che vale 
    \begin{gather*}
        \tilde{f}(x, y) = \left\{\begin{array}{l}
            f(x, y) \quad (x, y) \in \Omega \\
            0 \quad (x, y ) \in R- \{\Omega\}
        \end{array}\right.
    \end{gather*}
    \begin{gather*}
        \begin{tikzpicture}[scale = 1.5]
            \draw[->](0, 0) -- (4, 0) node[at end, below] {$x$};
            \draw[->](0, 0) -- (0, 3) node[at end, left] {$y$};
            \draw(0.5, 0) -- (0.5, 2.5) node[at start, below] {$a$};
            \draw(0, 0.5) -- (3, 0.5) node[at start, left] {$c$};
            \draw(3, 0) -- (3, 2.5) node[at start, below] {$b$};
            \draw(0, 2.5) -- (3, 2.5) node[at start, left] {$d$};
            \draw(0, 1) -- (3, 1) node[at start, left] {$y$};
            \node at (1.75, 2) {$\Omega$};
            \draw[very thick](0.5, 0.5) rectangle (3, 2.5);
            \draw(1, 0.5) .. controls (1.75, 1) and (1.75, 1.75) .. (1, 2.5); 
            \draw(2, 0.5) .. controls (1.5, 1) and (2.75, 2) .. (2.5, 2.5);
            \filldraw(1.45, 1) circle (1pt) node[anchor=south east] {$x = h_1(y)$};
            \filldraw(1.9, 1) circle (1pt) node[anchor = south west] {$x = h_2(y)$};
            \node at (3.2, 2) {$R$};
        \end{tikzpicture}
    \end{gather*}
    Per definizione dunque l'integrale
    \begin{gather*}
        \int_{\Omega}^{} f \ dx dy = \int_{R}^{} \tilde{f} \ dx dy = \int_{c}^{d}\left(\int_{a}^{b} \tilde{f} \ dx \right)dy 
    \end{gather*}
    Il grafico di $\tilde{f}$ come funzione della $x$ con un $y$ fissato, la funzione,
    preso un certo $y$ nel rettangolo e restrinta la funzione proprio al segmento 
    identificato, ci sarà una certa parte compresa tra $h_1(y) \leq f \leq h_2(y)$ 
    ed una parte al di fuori della quale la funzione è nulla poiché si è detto che 
    al di fuori di $\Omega$ la funzione vale zero. Allora l'integrale lo posso 
    calcolare solamente tra i valori che intersecano $h_1$ e $h_2$:
    \begin{gather*}
        \int_{a}^{b}\tilde{f} (x, y) \ dx = \left(\int_{a}^{h_1(y)} + \int_{h_1(y)}^{h_2(y)} + \int_{h_2(y)}^{b} \right)\tilde{f}(x, y) \ dx  = \int_{h_1 (y)}^{h_2(y)}f(x, y) \ dx 
    \end{gather*}
    Dunque posso esprimere l'integrale scritto tra $[a, b]$ e $[c, d]$ come
    \begin{gather*}
        \int_{c}^{d} \left(\int_{h_1(y)}^{h_2(y)}f(x, y) \ dx \right) \ dy  
    \end{gather*}
\end{proof}

\section{Cambiamenti di variabili}
Negli integrali unidimensionali generalmente si faceva un cambio 
di variabile per rendere la funzione più semplice per semplificare 
il calcolo dell'integrale. Nelle funzioni a più variabili i cambi di 
variabili semplificano il dominio della funzione. 
\begin{theorem}[Teorema di cambio variabili in integrali di funzioni a più variabili]
    Esiste una mappa tale che 
    \begin{gather*}
        \left\{\begin{array}{l}
            x = g(u, v) \\
            y = h(u, v)
        \end{array}\right.
    \end{gather*}
    Che mi permetta di dire che
    \begin{align}
        T : \Omega ' \to \Omega \qquad \Omega' \subset u \times v \qquad \Omega \subset x \times y
    \end{align}
    Si suppone che
    \begin{enumerate}
        \item $T$ sua biunivoca su $\Omega' \to \Omega$ 
        \item Le funzioni $g, h $ siano $C^{(1)}$
        \item La matrice Jacobiana abbia determinante diverso da zero per ogni $(u, v) \in \Omega '$. 
    \end{enumerate}
    Allora se $T$ soddisfa queste ipotesi e $\Omega'$ è regolare, allora 
    l'integrale
    \begin{align}
        \int_{\Omega}^{}f(x, y) \ dx dy = \int_{\Omega'}^{} f(g(u, v), h (u, v)) \left| \det J_T(u, v) \right| \ du dv 
    \end{align}

\end{theorem}

\begin{example}
    Sia 
    \begin{gather*}
        \Omega = B(O, 2) \/ B(O, 1)
    \end{gather*}
    Ossia il cerchio a cui tolgo un cerchio. Dunque usare con questa funzione 
    le coordinate polari mi permette di semplificare notevolmente i conti:
    \begin{gather*}
        (x, y) \to (\rho, \theta) \ \Longrightarrow \ \left\{\begin{array}{l}
            x  = \rho \cos\theta \\
            y = \rho \sin\theta 
        \end{array}\right. \qquad \theta \in [0, 2\pi]
    \end{gather*}
    Allora si ottiene l'integrale 
    \begin{gather*}
        \int_{\Omega}^{} x^{2} \ dx dy  \ \Longrightarrow \ \int_{\Omega'}^{} \rho^{2}\cos^{2}\theta \ K
    \end{gather*}
    Dove $K$ è il differenziale calcolato con la Jacobiana:
    \begin{gather*}
        J_T = \begin{pmatrix}
            \cos\theta & -\rho\sin\theta \\
            \sin\theta & \rho\cos\theta
        \end{pmatrix} \ \Longrightarrow \ \det J_T = \rho \ \Longrightarrow \ K = \rho \ d\rho d\theta
    \end{gather*}
    Allora
    \begin{gather*}
        \int_{\Omega '}^{} \rho^{2}\cos^{2}\theta \rho \ d\rho d\theta
    \end{gather*}
    E dunque si risolve l'integrale come
    \begin{gather*}
        \int_{0}^{2\pi}d\theta \int_{1}^{2}\rho^{3}\cos^{2}\theta \ d\rho = \int_{0}^{2\pi} \left(\cos^{2}\theta\frac{15}{4}\right)\ d\theta    = \frac{15}{4}\pi
    \end{gather*}
\end{example}
Quando si va dunque a scrivere la funzione attraverso le variabili ausiliarie vuol dire
scrivere i singoli rettangolini 


\end{document}
\documentclass[a4paper, oneside]{article}
\usepackage{graphicx}
\usepackage{amsthm}
\usepackage{amsmath}
\usepackage{amssymb}
\usepackage[a4paper,
            bindingoffset=0.2in,
            left=2cm,
            right=2cm,
            top=2cm,
            bottom=2cm,
            footskip=.25in]{geometry}
\usepackage[italian]{babel}
\usepackage{pgfplots}
\usepackage{tabularx}
\usepackage{tikz}
\usepackage{wrapfig}
\usepackage{color}
\usepackage[d]{esvect}
\usepackage{chemfig}
\usepackage{mhchem}
\definecolor{page}{rgb}{0.129,0.157,0.212}
\pagecolor{page}
\color{white}
\graphicspath{ {./images/} }
\usetikzlibrary{shapes.geometric}
\usetikzlibrary{datavisualization}
\usetikzlibrary{datavisualization.formats.functions}
\usetikzlibrary{patterns}
\pgfplotsset{width=10cm,compat=1.18}

\title{Appunti di Analisi (Bianchi)}
\author{Tommaso Miliani}
\date{27-11-25}

\begin{document}
\newtheoremstyle{theoremEnv}
                {}          % Space above
                {}          % Space below
                {\slshape}  % Body font
                {}          % Indent amount
                {\bfseries} % Head font
                {.}         % Punctuation after head
                {\newline}  % Space after theorem head
                {}          % Theorem head spec
\theoremstyle{theoremEnv}

\newtheorem{definition}{Definizione}[section]
\newtheorem{theorem}{Teorema}[section]
\newtheorem{lemma}{Proposizione}[section]
\newtheorem{observation}{Osservazione}[section]
\newtheorem{corollary}{Corollario}[theorem]
\newtheorem{example}{Esempio}[section]
\newtheorem{remark}{Enunciato}[section]

\maketitle

\section{Teorema di Stokes e della divergenza}
Si deve prima attribuire cosa sia una superficie su di uno
spazio e come si calcola un integrale su di una superficie 
nello spazio. Questi teoremi possono essere applicati sia nel piano che nello spazio.
Nel caso del piano questi sono casi particolari delle formule di Gauss-Green. 
Se $f : [a, b] \to \mathbb{R}$ sia continua in $C^{(1)}$, se si vuole 
integrare in 
\begin{gather*}
    \int_{a}^{b} \frac{df(x)}{dx}dx = f(b) - f(a)
\end{gather*}
Si pul interpretare questo come se si avesse un insieme in $\mathbb{R}$ che
lega l'integrale di qualche cosa su questo intervallo $[a,b]$ ad una funzione
che coinvolge la funzione solamente agli estremi di questa funzione. Gli
estremi di questo intervallo sono il bordo di questo intervallo che si esprime tramite qualche 
cosa che sta nel bordo di quell'insieme. Se una certa forma differenziale $\omega$ è
esatta e $\omega$ è esattamente il differenziale di una certa funzione $f$, se se $\gamma$ è una curva orientata, allora 
si sa che l'integrale su di $\gamma$ di questa forma differenziale è esattamente
\begin{gather*}
    \int_{\gamma}^{} \frac{\partial f}{\partial x}dx \frac{\partial f}{\partial y}dy = f(P_2) - f(P_1)  
\end{gather*}
Dove $P_1$ e $P_2$ sono, rispettivamente, gli estremi iniziali e finali di $\gamma$ (dunque lei ha
un certo verso di percorrenza). Quello che si sta facendo è che si sta integrando su di una certa curva
e l'integrale di questa cosa è legata solo al valore della funzione nel bordo. Le formule
di Gauss-Green sono una cosa simile ma in $\mathbb{R}^{2}$ il cui bordo è
una curva che delimita un certo insieme. Le formule di permettono di dire che se 
$D \subset \mathbb{R}^{2}$ è un certo insieme che soddisfa certe ipotesi e $\partial D$ è la sua frontiera
orientata in un certo modo allora 
\begin{gather*}
    \int_{D}^{} (f)(x, y) \ dx dy = \int_{\partial D}^{} f
\end{gather*}
Ossia la funzione $(f)$ è una certa funzione che soddisfa le ipotesi e $f$ è 
la funzione della frontiera.

\begin{definition}[Verso di percorrenza di una frontiera]
    Posto $D$ come insieme limitato di $\mathbb{R}^{2}$, si suppone che la $\partial D$
    sia una curva semplice chiusa, regolare a tratti. Si definisce orientazione positiva
    della frontiera di $D$ quella antioraria. Percorrendo la frontiera di $D$ nel verso
    positivo, l'insieme $D$ rimane sempre sulla sinistra. 
\end{definition}

\begin{lemma}
    Sia $F$ un campo vettoriale piano di componenti $F(x, y) = (P(x, y), Q(x, y))$. Sia 
    $D \subset \mathbb{R}^{2}$, supponendo che $F \in C^{(1)}(\overline{D} )$.
    Se $D$ è $y$ semplice, e il suo bordo
    è una curva regolare a tratti chiusa, allora l'integrale su $D$ di 
    \begin{gather*}
        \int_{D}^{} \frac{\partial P}{\partial y}(x, y) \ dx dy  = - \int_{\partial D}^{} P(x, y) \ dx
    \end{gather*} 
    Se $\partial D$ è una curva regolare a tratti, e $D$ è $x$ semplice, allora
    \begin{gather*}
        \int_{D}^{} \frac{\partial Q}{\partial x}(x, y) \ dx dy = \int_{\partial D}^{} Q(x, y)  \ dy
    \end{gather*}
\end{lemma}
\begin{proof}
    (1): Supponendo inizialmente che $D$ sia un insieme $y$ semplice
    \begin{gather*}
        \begin{tikzpicture}
            \draw[->](0, 0) -- (2, 0);
            \draw[->](0, 0) -- (0, 2);
            \draw(0.5, 0.5) -- (0.5, 1);
            \draw(1.5, 0.25) -- (1.5, 1.5);
            \draw(0.5, 1) .. controls (0.75, 1.3) and (1.25, 1.7) .. (1.5, 1.5);
            \draw(0.5, 0.5) .. controls (0.75, 0.5) and (1.25, 0.8).. (1.5, 0.25);
            \node at (1, 1) {$D$};
            \draw[dashed](0.5, 0.5) -- (0.5, 0) node[at end, below] {$a$};
            \draw[dashed](1.5, 0.25) -- (1.5, 0) node[at end, below] {$b$};
        \end{tikzpicture}
    \end{gather*}
    Ossia $D = \{(x, y) : x \in [a, b], \phi_1(x) \leq y \leq \phi_2(x)\}$, con 
    $\phi_1$ e $\phi_2$ che sono entrambe $C^{(1)}$ a tratti. Si può allora calcolare
    l'integrale come
    \begin{gather*}
        \int_{D}^{} \frac{\partial f}{\partial y}(x, y) \ dx dy = \int_{a}^{b}dx \int_{}^{} \frac{\partial P}{\partial y}(x, y) dy    
    \end{gather*}
    Quindi
    \begin{gather*}
        \int_{a}^{b}(P(x, \phi_2(x)) - P(x, \phi_1(x))) dx 
    \end{gather*}
    Posso spezzare la frontiera in quattro parti e chiamare il pezzo più basso 
    come (1)  e procedere in senso antiorario. Richiamando l'integrale su di una curva orientata,
    \begin{gather*}
        \int_{\gamma}^{} g(x, y) dx 
    \end{gather*}
    Dove $\gamma$ è una curva orientata $\gamma : [c,d ] \to \mathbb{R}^{2}$. E dunque
    \begin{gather*}
        \int_{c}^{d}g(\gamma_1(t), \gamma_2(t)) \dot{\gamma_1}(t)\ dt 
    \end{gather*}
    Posso allora calcolare l'integrale (procedendo in senso antiorario) come 
    \begin{gather*}
        \int_{\partial D}^{} P dx = \int_{(1)}^{} + \int_{(2)}^{} + \int_{(3)}^{} + \int_{(4)}^{} 
    \end{gather*}
    Ovviamente
    \begin{gather*}
        \int_{(2)}^{} P\ dx = \int_{(4)}^{} P\ dx = 0
    \end{gather*}
    Poiché, se si facesse la parametrizzazione della curva $(2)$ e $(4)$, si avrebbe che
    \begin{gather*}
        t \to (b, t) \qquad t \in [\phi_1(b) , \phi_2(t)]
    \end{gather*}
    Allora l'integrale è esattamente 
    \begin{gather*}
        \int_{\phi_1(b)}^{\phi_2(b)} P(b, t) \cdot 0 \ dt = 0 
    \end{gather*}
    Per la parametrizzazione della curva (1):
    \begin{gather*}
        t \to (t, \phi_1(t)) \ t \in [a, b]
    \end{gather*}
    E dunque 
    \begin{gather*}
        \int_{(1)}^{} P(x, y) \ dx = \int_{a}^{b}P(t, \phi_1(t)) \cdot  1 \ dt = \int_{a}^{b} P(t, \phi_1(t)) \ dt  
    \end{gather*}
    Rimane da fare la parametrizzazione per (3):
    \begin{gather*}
        t \to (t, \phi_2(t)) \ t\in[a, b]
    \end{gather*}
    Per usare questa parametrizzazione devo mettere un meno in quanto sono nel
    verso discorde:
    \begin{gather*}
        \int_{(3)}^{} P(x, y) \ dx = \int_{a}^{b}P(t, \phi_2(t))\ dt 
    \end{gather*}
    Dato che i due pezzi (1) e (3) sono uguali alle loro controparti nell'integrale
    della tesi (eccetto per il segno meno), allora si ottiene che
    \begin{gather*}
        \int_{a}^{b}P(x, \phi_2(x)) \ dx - \int_{a}^{b}P(x, \phi_1(x)) \ dx = -\int_{(3)}^{} P \ dx - \int_{(1)}^{}  P \ dx
    \end{gather*}
    E dunque è dimostrata la formula (1). 
\end{proof}

\begin{lemma}[Formule di Gauss-Green prime]
    Sia $F$ un campo vettoriale piano di componenti $F(x, y) = (P(x, y), Q(x, y))$. Sia 
    $D \subset \mathbb{R}^{2}$, supponendo che $F \in C^{(1)}(\overline{D} )$.
    Sia $\partial D$ una curva semplice
    e chiusa, regolare a tratti, se $D$ è sia $x$ semplice che $y$ semplice, allora valgono
    entrambe le formule di prima e posso fare 
    \begin{gather*}
        \int_{D}^{}\left(\frac{\partial Q}{\partial x} - \frac{\partial P}{\partial y}  \right) dx dy = \int_{\partial D}^{}Pdx + Qdy  = \int_{\partial D}^{} \left< F, T \right>d?  
    \end{gather*}
    In senso antiorario. 
\end{lemma}

\begin{definition}
    Si dice che un insieme $D \subset \mathbb{R}^{2}$ è un insieme $s$-decomponibile
    se è sia decomponibile in un insieme finito di sottodomini $D_1, \dots, D_s$, semplici
    rispetto ad entrambi gli assi e le cui frontiere sono curve semplici, chiuse
    e regolari a tratti.   
\end{definition}

Se $D$ è $s$-decomponibile, come orientazione positiva su $\partial D$, si intende
sempre quella data alle singole componenti di $\partial D$, in modo che, percorrendole, si
lasci il dominio sempre alla propria sinistra. 

\begin{theorem}
    Se $D$ è $s$-decomponibile, e $F$ è un campo vettoriale continuo in $C^{(1)}(\overline{D} )$,
    allora vale la formula
    \begin{align}
        \int_{D}^{}\left(\frac{\partial Q}{\partial x} - \frac{\partial P}{\partial y}  \right) dx dy = \int_{\partial D}^{}Pdx + Qdy  = \int_{\partial D}^{} \left< F, T \right>d?  
    \end{align}
\end{theorem}
\begin{proof}
    Considerando un dominio qualsiasi $D$ diviso in tanti sottodomini
    $D_i$: 
    \begin{gather*}
        \begin{tikzpicture}
            \draw(0, 0) ellipse (2 and 1);
            \draw(-1, 0) circle (0.5);
            \draw(0.75, -0.5) ellipse (0.75 and 0.25);
            \draw(-1, 0.5) -- (-1, 0.9);
            \draw(-1.5, 0) -- (-2, 0);
            \draw(-0.5, 0) -- (0, -0.5);
            \draw(1.5, -0.5) -- (1.75, -0.5);
            \draw(0.75, -0.75) -- (0.75, -0.9);
            \draw(0.75, -0.25) -- (0.75, 0.9);
            \draw(-1, -0.5) -- (-1, -0.9);
        \end{tikzpicture}
    \end{gather*}
    Si può iniziare dicendo che
    \begin{gather*}
        \int_{D}^{} \left(\frac{\partial Q}{\partial x} - \frac{\partial P}{\partial y}  \right) dx dy
    \end{gather*}
    E' esattamente la somma degli integrali su tutti i domini (che in genere sono finiti per definizione),
    dunque posso dire che
    \begin{gather*}
        \sum \int_{D_i}^{} \left(\frac{\partial Q}{\partial x} - \frac{\partial P}{\partial y}  \right) dx dy
    \end{gather*}
    Ad ognuna di queste parti posso applicare la formula di Gauss-Green e dunque
    \begin{gather*}
        \sum \int_{\partial D_i}^{} P dx + Q dy 
    \end{gather*}
    Per essere valida, vanno percorsi tutti i sottodomini in senso antiorario,
    dunque posso dire che tutti i bordi interni si annullano tra di loro in quanto
    sono percorsi sempre in senso contrario: allora la somma totale dei contributi 
    nell'integrale terrà conto solamente dei bordi esterni e di quelli vicino 
    ai due buchi di $D$. Si ottiene allora
    \begin{gather*}
        \int_{\partial D}^{} Pdx + Q dy 
    \end{gather*}
\end{proof}

\begin{example}
    Usare le formule di Gauss-Green in particolare per 
    \begin{gather*}
        \int_{\gamma}^{}(x - y^{3})dx + (y^{3} + x^{3}) dy 
    \end{gather*}
    Dove $\gamma$ è la curva che descrive la frontiera di
    $D = \{(x, y) : x^{2} + y^{2} \leq 1, x \geq 0, y\geq 0\}$. 
    \begin{gather*}
        \int_{\partial D}^{} idem = \int_{D}^{} \frac{\partial }{\partial x}(y^{3} + x^{3}) - \frac{\partial }{\partial y}(x - y^{3}) \ dx dy    
    \end{gather*}
    Allora
    \begin{gather*}
        \int_{D}^{} (3x^{2} + 3y^{2}) dx dy = \frac{3}{8}\pi  
    \end{gather*}
\end{example}

\section{Teorema di Stokes nel piano}
\begin{definition}[Teorema di Stokes]
    Considerato un campo vettoriale $F = F(F_1, F_2, 0)$, se si calcola il rotore del campo
    \begin{gather*}
        \text{rot} F = k \left(\frac{\partial F_2}{\partial x} - \frac{\partial F_1}{\partial y}\right)  
    \end{gather*}
    Per un campo del genere il rotore ha una componente solo nell'asse $z$. Dunque 
    Se $D$ è $s$-decomponibile, e $F \in C^{(1)}(\overline{D} )$, allora l'integrale
    \begin{gather*}
        \int_{D}^{} \text{rot} F\  dx dy = \int_{\partial D}^{} \left< F, T \right> \ ds   
    \end{gather*} 
    Si è reinterpretato in modo diverso le formule di Gauss-Green. Inoltre, adesso, 
    si può prendere un campo in modo tale che sia 
\end{definition}



\end{document}
\documentclass[a4paper, oneside]{article}
\usepackage{graphicx}
\usepackage{amsthm}
\usepackage{amsmath}
\usepackage{amssymb}
\usepackage[a4paper,
            bindingoffset=0.2in,
            left=2cm,
            right=2cm,
            top=2cm,
            bottom=2cm,
            footskip=.25in]{geometry}
\usepackage[italian]{babel}
\usepackage{pgfplots}
\usepackage{tabularx}
\usepackage{tikz}
\usepackage{wrapfig}
\usepackage{color}
\usepackage[d]{esvect}
\definecolor{page}{rgb}{0.129,0.157,0.212}
\pagecolor{page}
\color{white}
\graphicspath{ {./images/} }
\usetikzlibrary{shapes.geometric}
\usetikzlibrary{datavisualization}
\usetikzlibrary{datavisualization.formats.functions}
\usetikzlibrary{patterns}
\pgfplotsset{width=10cm,compat=1.9}

\title{Appunti di analisi}
\author{Tommaso Miliani}
\date{10-10-25}

\begin{document}
\newtheoremstyle{theoremEnv}
                {}          % Space above
                {}          % Space below
                {\slshape}  % Body font
                {}          % Indent amount
                {\bfseries} % Head font
                {.}         % Punctuation after head
                {\newline}         % Space after theorem head
                {}          % Theorem head spec
\theoremstyle{theoremEnv}

\newtheorem{definition}{Definizione}[section]
\newtheorem{theorem}{Teorema}[section]
\newtheorem{lemma}{Proposizione}[section]
\newtheorem{observation}{Osservazione}[section]
\newtheorem{corollary}{Corollario}[theorem]
\newtheorem{example}{Esempio}[section]

\maketitle

\section{Esercizi vari}
\begin{example}
    Trovare il piano tangente al grafico di
    \begin{gather*}
        f(x, y) = x \sin (y^{2} ) 
    \end{gather*}
    Nel punto $(0; \sqrt{\frac{\pi}{4}}; 0 )$. Per vedere che esiste il piano
    ho bisogno che la funzione sia differenzibile. Utilizzo allora
    il teorema generale della differenziabilità: se $f$ è $C^{1}$ in $\mathbb{R}^{2}$,
    e lo è in quanto è composizione di prodotti di funzioni elementari. Dato che ho
    una funzione che è $C^{1}$ su tutto il piano, allora ammette il piano tangente.
    La tangente ha allora equazione
    \begin{gather*}
        z = <Df(P_0); \underline{x} - P_0> + z_0
    \end{gather*}   
    Ossia il prodotto scalare tra la derivata di $f$ in $P_0$ e il vettore
    $x - P_0$. ALlora la derivata di $f$:
    \begin{gather*}
        f_x(x, y) = \sin(y^{2} ) \\
        f_y(x, y) = 2xy\cos(y^{2} )
    \end{gather*}
    Allora il gradiente nel punto $P_0$ è:
    \begin{gather*}
        D(f(P_0)) = \begin{pmatrix}
            \sin\left(\sqrt{\frac{\pi}{4}} \right)^{2} \\
            0 
        \end{pmatrix} = \begin{pmatrix}
            \frac{1}{\sqrt{2} } \\
            0
        \end{pmatrix}
    \end{gather*}
    Il piano avrà allora espressione:
    \begin{gather*}
        \Pi_{P_0} : z = \begin{pmatrix}
            \frac{1}{\sqrt{2} } \\
            0
        \end{pmatrix} \begin{pmatrix}
            x - 0 \\
            y - \sqrt{\frac{\pi}{4}}
        \end{pmatrix} + 0 \ \Longrightarrow \ z = \frac{x}{\sqrt{2} }
    \end{gather*}
\end{example}

Un'altro esempio:
\begin{wrapfigure}{r}{0.4\textwidth}
    \centering
    \caption{}
    \begin{tikzpicture}
        \draw(0, 0) ellipse (1 and  0.5);
        \draw(1, 0) -- (0, 2) -- (-1, 0);
        \draw(0, 0) -- (1, 0) node[midway, below] {$r$};
        \draw(0, 0) -- (0, 2) node[midway, left] {$h$};
    \end{tikzpicture}    
\end{wrapfigure}
So che l'errore massimo è $\epsilon = 0.1 \ cm$ sulle misure. Stimare
l'errore massimo sul volume dato $r = 10 \ cm$ e $h = 25 \ cm$.
\begin{gather*}
    V(h, r) = \frac{\pi}{3}r^{2}h  
\end{gather*}
Dato che presentano delle incertezze, posso calcolare il differenziale di $V$
come 
\begin{gather*}
    dV = 2\frac{\pi}{3}hr \ dr + \frac{\pi}{3}r^{2} \ dh 
\end{gather*}
Ossia il differenziale di $V$, cioè l'operatore lineaere che mi da
l'approssimazione della funzione. Dato il differenziale, e dato che gli
errori massimi sono uguali, allora l'errore massimo su $V$ è
\begin{align*}
    \epsilon (V) &= \frac{\pi}{3}2rh\epsilon + \frac{\pi}{3}r^{2}\epsilon = \\
    &= \frac{\pi}{3}60 = 20 \pi 
\end{align*}

\begin{wrapfigure}{r}{0.4\textwidth}
    \centering
    \caption{}
    \begin{tikzpicture}[scale = 0.2]
        \draw(0, 0) -- (15, 0);
        \draw(0, 0) -- (0, 15);
        \filldraw(1, 3) circle (3pt);
        \filldraw(3, 3) circle (3pt);
        \filldraw(1, 7) circle (3pt);
        \filldraw(6, 15) circle (3pt);
        \draw[->, cyan] (1, 3) -- (6, 15);
        \draw[->, red] (1, 3) -- (3, 3);
        \draw[->, green](1, 3) -- (1, 7);
    \end{tikzpicture}    
\end{wrapfigure}
\begin{example}
    \begin{gather*}
        f: \mathbb{R}^{2} \to \mathbb{R}^{2}, \qquad f \in C^{1}(\mathbb{R}^{2} )   
    \end{gather*}
    Sapendo che la derivata direzionale:
    \begin{gather*}
        v_1 \quad \vv{AB} \qquad \frac{\partial f}{\partial v_1} (A) = 3 \\
        v_1 \quad \vv{AC} \qquad \frac{\partial f}{\partial v_2} (A) = 26    
    \end{gather*}
    Dove $A = (1, 3), B = (3, 3 ), C = (1, 7), D = (6, 15)$. Si calcoli
    $\frac{\partial f}{\partial w}(A)$ dove $w = \vv{AD}$.   
    I vettori
    \begin{gather*}
        v_1 = \frac{B - A}{|\cdot |} = (1, 0) \\
        v_2 = \frac{C - A}{|\cdot |} = (0, 1) \\
        w = \frac{D - A}{|\cdot |} = (\frac{5}{13}; \frac{12}{13})
    \end{gather*}
    Utilizzando il teorema del gradiente:
    \begin{gather*}
        \frac{\partial f}{\partial w}  = <D(f(A)) ; w> = <(3, 26); (\frac{5}{3}; \frac{12}{13})> = 
    \end{gather*}
\end{example}
L'esempio precedente poteva anche essere risolto in maniera diversa nel caso
in cui $v_1, v_2$ non fossero stati paralleli agli assi. Potevo
porre allora la derivata parziale della funzione rispetto
ai vettori come il prodotto scalare con il gradiente:
\begin{gather*}
    \frac{\partial f}{\partial v_1} = <Df; v_1> = a_1 f_x + b_1 f_y = 3 \\
    \frac{\partial f}{\partial v_2} = <Df; v_2> = a_2 f_x + b_2 f_y = 26  
\end{gather*}
Dove $v_1 = (a_1, b_1), v_2 = (a_2, b_2)$. Posso ora esprimere il vettore
$w$ come combinazione lineare dei due vettori:
\begin{gather*}
    w = \alpha <Df; v_1> + \beta <Df, v_2> = \alpha \frac{\partial f}{\partial v_1} + \beta \frac{\partial f}{\partial v_2}  
\end{gather*}

\begin{lemma}
    Se si ha una funzione definita su di un insieme aperto ed è anche
    differenziabile nell'insieme dei punti del dominio tali per cui
    \begin{gather*}
        f: \mathbb{A} \to \mathbb{R} \qquad P_0 \equiv (x_0, y_0) \in U_k = \{(x, y) \in A : d(x, y) = k\}
    \end{gather*}
    Allora 
    \begin{align}
        D(f(P_0)) \perp U_k \in \mathbb{R}
    \end{align}
\end{lemma}
\begin{proof}
    Vicino a $P_0$ $U_k$ ha al forma
    \begin{gather*}
        \begin{pmatrix}
            x(t) \\
            y(t)
        \end{pmatrix} = \underline{x}(t) \qquad t \in \mathbb{I}
    \end{gather*}
    Chiamata allora la funzione $g(t)  =f(\underline{x}(t))$, devo dimostrare
    ora che il gradiente sia perpendicolare alla linea di livello: ossia
    il prodotto scalare tra 
    \begin{gather*}
        Df(P_0) \perp U_k \in P_0 \ \Longrightarrow \ Df(P_0) \perp \underline{x}(t) \in P_0
    \end{gather*}
    Devo allora verificare che il vettore del differenziale sia peprendicolare
    alla retta tangente: ossia la derivata del vettore $\dot{\underline{x}}(t)$
    dove $\underline{x}(t) = P_0$.  Devo allora ottenere
    \begin{gather*}
        \boxed{f_x(P_0)\dot{x}(t) + f_y (P_0)\dot{y}(t) = 0}
    \end{gather*}
    Se siamo su una linea di livello, allora vale la relazione
    \begin{gather*}
        g(t) = f(\underline{x}(t)) = k \ \Longrightarrow \ g'(t) = 0
    \end{gather*}
    La derivata della funzione $g$ è 
    \begin{gather*}
        g'(t) = <Df(\underline{x}(t)); \dot{x}(t)> = f_x(\underline{x}(t))\dot{x}(t) + f_y(\underline{x}(t))\dot{y}(t)
    \end{gather*}
    Allora mi manca solamente da calcolarla in $t_0$. 
\end{proof}

\begin{theorem}[Teorema delle funzioni con gradiente nullo]
    Se il gradiente di una funzione è nullo in un insieme aperto $A$, allora 
    è costante nell'insieme aperto se $A$ è aperto e connesso.
    Se invece $A= A_1 \cup A_2$ allora esiste sicuramente 
    \begin{align}
        f \equiv k_1 \in A_1 \qquad f \equiv k_2 \in A_2
    \end{align}
\end{theorem}

\end{document}
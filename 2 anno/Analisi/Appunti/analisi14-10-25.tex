\documentclass[a4paper, oneside]{article}
\usepackage{graphicx}
\usepackage{amsthm}
\usepackage{amsmath}
\usepackage{amssymb}
\usepackage[a4paper,
            bindingoffset=0.2in,
            left=2cm,
            right=2cm,
            top=2cm,
            bottom=2cm,
            footskip=.25in]{geometry}
\usepackage[italian]{babel}
\usepackage{pgfplots}
\usepackage{tabularx}
\usepackage{tikz}
\usepackage{wrapfig}
\usepackage{color}
\usepackage[d]{esvect}
\definecolor{page}{rgb}{0.129,0.157,0.212}
\pagecolor{page}
\color{white}
\graphicspath{ {./images/} }
\usetikzlibrary{shapes.geometric}
\usetikzlibrary{datavisualization}
\usetikzlibrary{datavisualization.formats.functions}
\usetikzlibrary{patterns}
\pgfplotsset{width=10cm,compat=1.9}

\title{Appunti di analisi}
\author{Tommaso Miliani}
\date{14-10-25}

\begin{document}
\newtheoremstyle{theoremEnv}
                {}          % Space above
                {}          % Space below
                {\slshape}  % Body font
                {}          % Indent amount
                {\bfseries} % Head font
                {.}         % Punctuation after head
                {\newline}         % Space after theorem head
                {}          % Theorem head spec
\theoremstyle{theoremEnv}

\newtheorem{definition}{Definizione}[section]
\newtheorem{theorem}{Teorema}[section]
\newtheorem{lemma}{Proposizione}[section]
\newtheorem{observation}{Osservazione}[section]
\newtheorem{corollary}{Corollario}[theorem]
\newtheorem{example}{Esempio}[section]

\maketitle

\section{Polinomio di Taylor per funzioni a due variabili}
Nella trattazione del Polinomio di Taylor ci si ferma  al secondo grado. Definita una
funzione $f$ definita su di un insieme aperto $A \subset \mathbb{R}^{2} $, come $f : A \to \mathbb{R}$. 
Se $\underline{x} \in A$ tale che il segmento $[x, y]$ sta in $A$, ossia 
\begin{gather*}
    \underline{x}(t) = \underline{x} + th 
\end{gather*}
Per ogni $t$ questo oggetto sta in $A$. Allora $f(x + h)$ si scrive come 
\begin{gather*}
    f(x + h) = f(x)  + \left< Df(\underline{x}), h \right> + o(\left| \underline{h} \right| ) 
\end{gather*}
Se $f$ è $C^{1}$, allora esiste un teta scalare tra zero ed  tale che $f(\underline{x} + \underline{h}) = f(\underline{x}) + \left< Df(\underline{x} + \theta \underline{h}); h  \right>$. 



\end{document}
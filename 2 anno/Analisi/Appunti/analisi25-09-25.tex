\documentclass[a4paper, oneside]{article}
\usepackage{graphicx}
\usepackage{amsthm}
\usepackage{amsmath}
\usepackage{amssymb}
\usepackage[a4paper,
            bindingoffset=0.2in,
            left=2cm,
            right=2cm,
            top=2cm,
            bottom=2cm,
            footskip=.25in]{geometry}
\usepackage[italian]{babel}
\usepackage{pgfplots}
\usepackage{tabularx}
\usepackage{tikz}
\usepackage{wrapfig}
\usepackage{color}
\usepackage[d]{esvect}
\definecolor{page}{rgb}{0.129,0.157,0.212}
\pagecolor{page}
\color{white}
\graphicspath{ {./images/} }
\usetikzlibrary{shapes.geometric}
\usetikzlibrary{datavisualization}
\usetikzlibrary{datavisualization.formats.functions}
\usetikzlibrary{patterns}
\pgfplotsset{width=10cm,compat=1.9}

\title{Appunti di analisi}
\author{Tommaso Miliani}
\date{25-09-25}

\begin{document}
\newtheoremstyle{theoremEnv}
                {}          % Space above
                {}          % Space below
                {\slshape}  % Body font
                {}          % Indent amount
                {\bfseries} % Head font
                {.}         % Punctuation after head
                {\newline}         % Space after theorem head
                {}          % Theorem head spec
\theoremstyle{theoremEnv}

\newtheorem{definition}{Definizione}[section]
\newtheorem{theorem}{Teorema}[section]
\newtheorem{lemma}{Proposizione}[section]
\newtheorem{observation}{Osservazione}[section]
\newtheorem{corollary}{Corollario}[theorem]
\newtheorem{example}{Esempio}[section]

\maketitle

\section{Studio qualitativo delle differenziali}
\begin{example}
    \begin{gather*}
        \left\{\begin{array}{l}
            y' = 4y(1 - y) \\
            y(0) = a 
        \end{array}\right.
    \end{gather*}
    Si vede che esistono delle soluzioni a variabili costanti
    in quanto mi delimitano regioni del piano nel quale esistono tutte
    le altre soluzioni. In pratica, dato il teorema di unicità delle soluzioni, se
    $ 0 < a < 1$ allora $y(x)$ è anch'essa compresa tra zero e uno
    e lo stesso vale se è minore o maggiore di uno. Questo è un grande aiuto
    perché posso dividere i casi di studio del problema.\\ 
    $0 < a < 1$: allora anche $y(x)$ è limitata tra $0$ e $1$ ma
    prolungabile in $\mathbb{R}$. \\
    $a > 1 \ \Longrightarrow \ y(x) > 1$: è infinita e limitata e quinid il segno della derivata
    è uguale al segno della funzione, allora la funzione in un intervallo
    $I_0$ è decrescente e definito come $I_0 = (-\epsilon, \epsilon)$. Considero allora questo intervallo da $[0, \epsilon)$, allora 
    la funzione è decrescente nell'intervallo e so che $y(x)$ appartiene
    all'intervallo $(a, 1)$. \\
    $a< 0 \ \Longrightarrow \ y(x) < 0$: allora il segno della derivata è uguale al
    segno della funzione e dunque decresce  e quindi per ogni $x$ nell'intervallo $I_0^{-}$.                                 
\end{example}

    \begin{wrapfigure}{r}{0.4\textwidth}
        \centering
        \caption{}
        \begin{tikzpicture}
            \draw(-2, 0) -- (2, 0);
            \draw(0, -2) -- (0, 2);
        \end{tikzpicture}    
    \end{wrapfigure}
\begin{example}
    \begin{gather*}
        \left\{\begin{array}{l}
            y' = x\left(1 + \frac{1}{y}\right) \\
            y(0) =  a
        \end{array}\right.
    \end{gather*}
    Se è $C^1$ allora è una funzione anche Lipstchizana e quindi è definita nell'intervallo
    massimale $I_0$, ossia il più grande intervallo dove può essere definita la funzione.
    \begin{gather*}
        f(x, y) \in x\left(1 + \frac{1}{y}\right) \in C^{1} \  (\mathbb{R} \times \mathbb{R}) 
    \end{gather*}
    E' dunque Lipstchizana in $\mathbb{R}^{\star} = \mathbb{R} - \{0\}$  allora vale che
    esiste una unica $\forall a \neq 0$ definita in un intervallo $I_0$.
    La soluzione costante $y = -1$.
    \begin{gather*}
        a \in (0, 1) \ \Longrightarrow \ y(x) \in (-1, 0) \forall x \in I_0
    \end{gather*}
    Allora dato che esiste una soluzione limitata allora è prolungabile su $\mathbb{R}$. 
    \begin{gather*}
        a > 0 \ \Longrightarrow \ y(x) > 0 \forall x \in I_0 
    \end{gather*}
    Il segno della derivata è maggiore di zero per $x > 0$ e quindi la funzione è crescente per $x > 0$
    allora $0 < \frac{1}{y} < \frac{1}{a}$.
    Ossia $ f(x, y) < xk$, ossia per $x > 0$ la funzione è sub lineare.
    Per il teorema dell'esistenza globale si ha che
    $I_0 = (-\epsilon, + \infty )$ per $x < 0$ è analogo.
    Considero $u(x) = y(-x)$ per dimostrare che sia pari e per risolvere il
    problema di Cauchy considero pure la funzione $u'(x) = -y(-x)$. 
    \begin{gather*}
        x\left(1 + \frac{1}{u}\right) = x\left(1 + \frac{1}{y(-x)}\right) = -\left(-x\left(1 + \frac{1}{y(-x)}\right)\right)
    \end{gather*}
    Con la sostituzione $t = -x$ si ha la seguente
    \begin{gather*}
        -t\left(1 + \frac{1}{y(t)}\right) = -y'(t) = -y'(-x) = u'(x)
    \end{gather*}
    Allora si è dimostrato che se $a > 0$ allora si ha che $I_0 = \mathbb{R}$ 
    \begin{gather*}
        a < - 1 \ \Longrightarrow \ y(x) < -1 \forall x \in I_0 
    \end{gather*}
    Inoltre dato che il segno di $y'$ è o positivo o negativo si ha ch 
    $y(x)$ è limitata E si prolunga su tutto $\mathbb{R}$. 
\end{example}


\end{document}
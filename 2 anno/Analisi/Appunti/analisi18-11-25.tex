\documentclass[a4paper, oneside]{article}
\usepackage{graphicx}
\usepackage{amsthm}
\usepackage{amsmath}
\usepackage{amssymb}
\usepackage[a4paper,
            bindingoffset=0.2in,
            left=2cm,
            right=2cm,
            top=2cm,
            bottom=2cm,
            footskip=.25in]{geometry}
\usepackage[italian]{babel}
\usepackage{pgfplots}
\usepackage{tabularx}
\usepackage{tikz}
\usepackage{wrapfig}
\usepackage{color}
\usepackage[d]{esvect}
\usepackage{chemfig}
\usepackage{mhchem}
\definecolor{page}{rgb}{0.129,0.157,0.212}
\pagecolor{page}
\color{white}
\graphicspath{ {./images/} }
\usetikzlibrary{shapes.geometric}
\usetikzlibrary{datavisualization}
\usetikzlibrary{datavisualization.formats.functions}
\usetikzlibrary{patterns}
\pgfplotsset{width=10cm,compat=1.18}

\title{Appunti di Analisi (Bianchi)}
\author{Tommaso Miliani}
\date{18-11-25}

\begin{document}
\newtheoremstyle{theoremEnv}
                {}          % Space above
                {}          % Space below
                {\slshape}  % Body font
                {}          % Indent amount
                {\bfseries} % Head font
                {.}         % Punctuation after head
                {\newline}  % Space after theorem head
                {}          % Theorem head spec
\theoremstyle{theoremEnv}

\newtheorem{definition}{Definizione}[section]
\newtheorem{theorem}{Teorema}[section]
\newtheorem{lemma}{Proposizione}[section]
\newtheorem{observation}{Osservazione}[section]
\newtheorem{corollary}{Corollario}[theorem]
\newtheorem{example}{Esempio}[section]
\newtheorem{remark}{Enunciato}[section]

\maketitle

\section{Integrali per funzioni a più variabili}
Quando si parla di integrali a più variabili si tratta di andare
a dare un significato al concetto di "area di un insieme piano" o
di misura di un insieme $n$ dimensionale. Bisogna allora 
definire il concetto di area e di volume per gli insiemi perché 
quando si lavora con insiemi che sono "strani". Pensando ad un quadrato in cui
si prende $x = 1$ e $y = 1$ e tali che si prendono solamente i lati razionali,
il concetto di area non è ben definito. Se si considera una funzione solamente 
definita su di un rettangolo $[a, b] \times [c, d]$.
\begin{gather*}
    \begin{tikzpicture}[scale = 1.5]
        \draw[->](0, 0) -- (2, 0) node[at end, below] {$x$};
        \draw[->](0, 0) -- (0, 2) node[at end, left] {$y$};
        \draw(0.5, -0.2) -- (0.5, 1.5) node[at start, below] {$a$};
        \draw(1.5, -0.2) -- (1.5, 1.5) node[at start, below] {$b$};
        \draw(-0.2, 0.5) -- (1.5, 0.5) node[at start, left] {$c$};
        \draw(-0.2, 1.5) -- (1.5, 1.5) node[at start, left] {$d$};
        \draw[->](4, 0) -- (6, 0) node[at end, below] {$x$};
        \draw[->](4, 0) -- (4, 2) node[at end, left] {$y$};
        \draw(4.75, 0.75) rectangle (5.5, 1.5);
        \filldraw(5.25, 1.25) circle (1pt) node[anchor = north] {$P_{h, k}$};
        \node at (5.25, 1.7) {$I_{h, k}$};
        \draw(4.75, 0.1) -- (4.75, -0.1) node[at end, below] {$x_{h - 1}$};
        \draw(3.9, 0.75) -- (4.1, 0.75) node[at start, below] {$x_{k - 1}$};
        \draw(5.5, 0.1) -- (5.5, -0.1) node[at end, below] {$x_h$};
        \draw(3.9, 1.5) -- (4.1, 1.5) node[at start, left] {$y_k$};
    \end{tikzpicture}
\end{gather*}
Preso $n \in \mathbb{N}$ e dividendo l'intervallo $[a, b]$ in $n$ parti uguali,
faccio lo stesso per $[c, d]$, in modo tale che posso definire degli
intervalli $x_{h - 1}$ e $x_h$ sull'asse $x$ e $y_{k - 1}$ e $y_k$ in modo 
tale da ottenere il quadrato fatto di lati $x_{h} - x_{h - 1}$ e $y_k - y_{k - 1}$. 
Si indica allora l'area con il valore assoluto (indica la \textbf{misura}), per l'area
del quadrato considerato è proprio
\begin{gather*}
    I_{h, k} = \left| I_{h, k} \right| = \frac{b - a}{n} \cdot \frac{d - c}{n} 
\end{gather*}
Poiché ogni distanza  $x_{h} - x_{h - 1}$ e $y_k - y_{k - 1}$ è l'$n$-esima parte
del segmento $ab$ e del segmento $cd$ rispettivamente. Posso allora definire le somme di 
Cauchy-Riemann 
\begin{gather*}
    s_n = \sum_{h, k = 1}^{n} f(p_{h, k})\left| I_{h, k} \right|  
\end{gather*}

\begin{definition}
    Si dice che la funzione $f : [a, b] \times [c, d]$, limitata, 
    è integrabile sul rettangolo $R = [a, b] \times [c, d]$ se esiste
    finito 
    \begin{align}
        \lim_{n \to \infty } s_n 
    \end{align} 
    E se tale limite non dipende da come si sono scelti i punti $P_{h, k}$ nei rispettivi
    rettangoli ad ogni passo della costruzione.
\end{definition}

\begin{example}[Funzione non integrabile]
    Presa $f : [0, 1] \times [0, 1] \to   R$ definita come
    \begin{gather*}
        f(x, y) = \left\{\begin{array}{l}
            1 \quad x \in \mathbb{Q} \\
            0 \quad x \not \in  \mathbb{Q}
        \end{array}\right.
    \end{gather*}
    Questa funzione non è integrabile in quanto se si dividesse l'intervallo
    $[0, 1]$ in $n$ parti uguali e si scegliessero i punti come fatto prima, 
    si potrebbero scegliere tutti in modo tale che la coordinata $x$ sia razionale
    per tutti gli intervalli (poiché l'insieme $\mathbb{Q}$ è denso). La funzione 
    vale dunque $1$ per tutti i punti, dunque le somme di Reimann vengono:
    \begin{gather*}
        s_n = \sum 1 \left| I_{h, k} \right| = 1 
    \end{gather*}
    Ottengo allora l'area del rettangolo finita. Se scegliessi invece
    tutti i numeri irrazionali, allora la funzione somme di Reimann vale esattamente
    zero. Questo porta dunque ad una ambiguità: il limite esiste ma non 
    è definito in quanto dipende dalla scelta dei punti e allora la funzione non
    è integrabile. 
\end{example}

\begin{theorem}[Criterio di integrabilità di funzioni]
    Se $f :[a, b] \times [c, d]$ è continua, allora è integrabile.
\end{theorem}
\begin{proof}
    Senza dimostrazione. 
\end{proof}

\subsection{Integrazione di funzioni a due variabili}
\begin{wrapfigure}{r}{0.4\textwidth}
    \centering
    \caption{}
    \begin{tikzpicture}
        \draw[->](0, 0) -- (3, 0) node[at end, below] {$y$};
        \draw[->](0, 0) -- (0, 3) node[at end, left] {$z$};
        \draw[->](0, 0) -- (-1, -2) node[at end, below] {$x$};
        \draw(0.5, 0) -- (-0.25, -1.5);
        \filldraw(0.5, 0) circle (1pt) node[anchor = south] {$c$};
        \draw(2.5, 0) -- (1.75, -1.5);
        \filldraw(2.5, 0) circle (1pt) node[anchor = south] {$d$};
        \draw(-0.15, -0.3) -- (2.35, -0.3);
        \filldraw(-0.15, -0.3) circle (1pt) node[anchor = east] {$a$};
        \draw(-0.75, -1.5) -- (1.75, -1.5);
        \filldraw(-0.75, -1.5) circle (1pt) node[anchor = east] {$b$};
        \draw(0.35, -0.3) -- (0.35, 2);
        \draw(2.35, -0.35) -- (2.35, 1.25);
        \draw(-0.25, -1.5) -- (-0.25, 0.5);
        \draw(1.75, -1.5) -- (1.75, 0.1);
        \draw(0.35, 2) -- (-0.25, 0.5);
        \draw(2.35, 1.25) -- (1.75, 0.1);
    \end{tikzpicture}    
\end{wrapfigure}
Posso considerare l'integrazione di funzioni a due variabili
come il volume del sottografico della superficie descritta dalla
funzione $f(x, y)$. Il processo di integrazione è un processo iterativo
che si svolge in due tempi: in un primo momento fisso la coordinata $y$
e svolgo l'integrale in funzione di $x$ ottenendo l'area del sottografico
quando è fissata $y$. L'integrale risulta dunque essere
\begin{align}
    \int_{[a, b] \times [c, d]}^{} f(x, y) \ dx dy = \int_{c}^{d} \left(\int_{a}^{b} f(x, y) \ dx\right) \ dy 
\end{align}

\begin{example}
    \begin{gather*}
        \int_{[0, 1] \times [0, 2]}^{} xe^{xy} \ dx dy = \int_{0}^{1}\left(\int_{0}^{2}xe^{xy} \ dy \right) \ dx 
    \end{gather*}
    Posso allora risolvere gli integrali secondo i seguenti passaggi:
    \begin{gather*}
        \int_{0}^{1}(e^{2x} - 1) \ dx =  \frac{e^{2}}{2} - 1
    \end{gather*}

\end{example}

\begin{theorem}
    SIa $f: [a, b] \times [c, d] \to \mathbb{R}$ è continua, allora 
    \begin{align}
        \int_{[a, b] \times [c, d]}^{} f(x, y) \ dx dy = \int_{c}^{d}\left(\int_{a}^{b} f(x, y) \ dx \right) \ dy = \int_{a}^{b}\left(\int_{c}^{d} f(x, y) \ dy \right) \ dx   
    \end{align}
\end{theorem}

\begin{lemma}
    La funzione
    \begin{align}
        \phi(x) = \int_{c}^{d}f(x, y) \ dy  
    \end{align}
    è continua in $[a, b]$. Analogamente 
    \begin{align}
        \psi(y) = \int_{a}^{b} f(x, y) \ dx  
    \end{align}
    è continua in $[c, d]$. 
\end{lemma}
\begin{proof}
    $\phi(x)$ è definita per $x \in [a, b]$, dividendolo
    in $n$ parti uguali, per cui ciascuna parte è $x_{h - 1}$ 
    fino a $x_h$. Allora l'integrale 
    \begin{gather*}
        \int_{a}^{b} \phi(x) \ dx = \sum_{h = 1}^{n}\int_{x_{h - 1}}^{x_h} \phi(x) \ dx
    \end{gather*}
    Con il teorema della media integrale, $\forall h$ esiste un qualche
    punto $x_h^{*}$ nell'intervallo $[x_{h - 1}, x_h]$ per cui
    \begin{gather*}
        \int_{a}^{b} \phi(x) \ dx = \sum_{h = 1}^{n}\int_{x_{h - 1}}^{x_h} \phi(x) \ dx = \sum_{h = 1}^{n}\phi(x_h^{*}) \cdot (x_h - x_{h - 1}) 
    \end{gather*}
    Dato che sono tutti uguali gli ultimi termini, posso portarli fuori e
    dire che 
    \begin{gather*}
        \sum_{h = 1}^{n}\phi(x_h^{*}) \cdot (x_h - x_{h - 1})  = \left(\frac{b - a}{n}\right) \sum_{h = 1}^{n} \phi(x_h^{*})
    \end{gather*}
    Dunque posso scriverlo esplicitamente ottenendo:
    \begin{gather*}
        \left(\frac{b - a}{n}\right)\sum_{h = 1}^{n} \int_{c}^{d} f(x_h^{*}, y) \ dy = \left(\frac{b - a}{n}\right)\sum_{h = 1}^{n} \sum_{k = 1}^{n} \int_{y_{k - 1}}^{y_k} f(x_h^{*}, y) \ dy 
    \end{gather*}
    Ossia ho fatto lo stesso procedimento anche per la funzione $\psi(y)$, adesso posso
    esplicitare anche quell'integrale ed esprimerlo attraverso il teorema
    della media integrale:
    \begin{gather*}
        \left(\frac{b - a}{n}\right)\sum_{h = 1}^{n} \sum_{k = 1}^{n} f(x_h^{*}, y_k^{*})(y_k - y_{k - 1}) \ \Longrightarrow \ \left(\frac{d - c}{n}\right)\left(\frac{b - a}{n}\right)\sum_{h = 1}^{n} \sum_{k = 1}^{n} f(x_h^{*}, y_k^{*})
    \end{gather*}
    Ottenendo allora
    \begin{gather*}
        \sum_{h, k = 1}^{n} f(x_h^{*}, y_k^{*})\left| I_{h, k} \right|  
    \end{gather*}
\end{proof}

\begin{definition}
    Sia $f : \Omega \to \mathbb{R}$ con $\Omega \subseteq  \mathbb{R}^{2}$ limitata
    e sia $R = [a, b] \times [c, d]$ un rettangolo contenente $\Omega$ 
    e sia $\tilde{f} : R \to \mathbb{R}$ definita come
    \begin{align}
        \tilde{f}(x, y) = \left\{\begin{array}{l}
            f(x, y) \quad (x, y) \in \Omega  \\
            0 \quad (x, y) \in R - \{\Omega\}
        \end{array}\right.
    \end{align}
    Se $\tilde{f}$ è integrabile su di $R$, allora $f$ è integrabile su $\Omega$
    e il suo integrale è tale che
    \begin{align}
        \int_{\Omega}^{} f(x, y) \ dx dy = \int_{R}^{} \tilde{f} (x, y) \ dx dy
    \end{align}
\end{definition}

\begin{definition}[Insiemi semplici]
    Un insieme $E \subset \mathbb{R}^{2}$ si dice $y$ semplice se è del tipo
    \begin{align}
        E = \{(x, y) : x \in [a, b], g_1(x) \leq y \leq g_2(x)\}
    \end{align}
    Con $g_1 < g_2$ e $g_1, g_2 : [a, b] \to \mathbb{R}$ continue. Questi insiemi
    sono degli insiemi compresi tra due funzioni generiche i cui estremi non sono
    fissi ma variano in base alla $x$. 
    Analogamente si definisce un insieme $x$ semplici se valgono le precedenti ma gli estremi
    variano in funzione delle $y$. 
\end{definition}

\begin{definition}[Insiemi regolari]
    Si definiscono gli insiemi regolari come degli insiemi che sono 
    unioni finite di insiemi semplici. 
\end{definition}
\begin{example}
    La corona circolare non è né $x$ semplice né $y$ semplice ma 
    può essere decomposta in parti $x$ semplici ed $y$ semplici.
    Unendole si ottiene un insieme regolare che è proprio la corona circolare. 
\end{example}


\end{document}
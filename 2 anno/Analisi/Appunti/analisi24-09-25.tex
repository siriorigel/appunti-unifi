\documentclass[a4paper, oneside]{article}
\usepackage{graphicx}
\usepackage{amsthm}
\usepackage{amsmath}
\usepackage{amssymb}
\usepackage[a4paper,
            bindingoffset=0.2in,
            left=2cm,
            right=2cm,
            top=2cm,
            bottom=2cm,
            footskip=.25in]{geometry}
\usepackage[italian]{babel}
\usepackage{pgfplots}
\usepackage{tabularx}
\usepackage{tikz}
\usepackage{wrapfig}
\usepackage{color}
\usepackage[d]{esvect}
\definecolor{page}{rgb}{0.129,0.157,0.212}
\pagecolor{page}
\color{white}
\graphicspath{ {./images/} }
\usetikzlibrary{shapes.geometric}
\usetikzlibrary{datavisualization}
\usetikzlibrary{datavisualization.formats.functions}
\usetikzlibrary{patterns}
\pgfplotsset{width=10cm,compat=1.9}

\title{Appunti analisi}
\author{Tommaso Miliani}
\date{24-09-25}

\begin{document}
\newtheoremstyle{theoremEnv}
                {}          % Space above
                {}          % Space below
                {\slshape}  % Body font
                {}          % Indent amount
                {\bfseries} % Head font
                {.}         % Punctuation after head
                {\newline}         % Space after theorem head
                {}          % Theorem head spec
\theoremstyle{theoremEnv}

\newtheorem{definition}{Definizione}[section]
\newtheorem{theorem}{Teorema}[section]
\newtheorem{lemma}{Proposizione}[section]
\newtheorem{observation}{Osservazione}[section]
\newtheorem{corollary}{Corollario}[theorem]
\newtheorem{example}{Esempio}[section]

\maketitle

\section{Equazioni esempio}

\begin{definition}[Principio di sovrapposizone]
    Il principio di sovrapposizione mi consente di risolvere separatamente due equazioni differenziali
    non omogenee separando i termini termini somma dalla parte di $f(x) $ 
    sfruttando la linearità delle equazioni differenziali. 
\end{definition}
\begin{example}
    Equazion differenziale lineare del secondo ordine a coefficienti costanti
    \begin{gather*}
        y'' -4y'+5y=2x^{2} + 3e^{-x}  
    \end{gather*}
    Si vede allora che il polinomio carattersitico è
    \begin{gather*}
        P_{\lambda} = \lambda^{2} - 4\lambda + 5 = (\lambda - 2 + i)(\lambda - 2 - i)
    \end{gather*}
    Trovo allora le radici di questo polinomio reale ottenendo così due soluzioni dell'equazione
    omogenea associata, dalla combinazione lineare delle due soluzioni:
    \begin{gather*}
        e^{(2 - i)x} \qquad e^{(2 + i)x}  
    \end{gather*}
    Queste soluzioni mi generano tutte le soluzioni complesse, posso trovare allora
    una combinazione lineare di queste soluzioni per ottenere una
    soluzione reale, ossia
    \begin{gather*}
        e^{2x}\cos x \qquad  e^{2x} \sin x  
    \end{gather*}
    Le soluzioni dell'equazione omogenea sono tutte e sole le funzioni
    \begin{gather*}
        Ae^{2x}\cos x + Be^{2x}\sin x \quad \forall A, B \in \mathbb{R}  
    \end{gather*}
    Il fatto che $A, B$ siano reali deriva dal fatto che devono comunque dare delle
    combinazioni lineari reali a partire dalle soluzioni. Si sa che dalla
    teoria lo spazio delle soluzioni dell'equazione completa è dato dallo spazio affine
    \begin{gather*}
        y_T(x) = y_0(x) + y_p(x)
    \end{gather*}
    Dove $y_p(x)$ è la soluzione particolare con cui traslo il mio sottospazio affine
    ma in realtà sono infinite; posso cercarla con il metodo della similarità ed il principio
    di sovrapposizione:
    \begin{gather*}
        \left\{\begin{array}{l}
            y'' -4y'+5y=2x^{2} \\
            y'' -4y'+5y=3e^{-x} 
        \end{array}\right.
    \end{gather*}
    Posso sommare le due soluzioni per ottenere la soluzione dell'equazione "particolare";
    cerco allora un polinomio di secondo grado tale che l'oggetto 
    $y_P(x)$ sia soluzione:
    \begin{gather*}
        y_P(x) = (ax ^{2} + bx + c ) + (ke^{-x} )
    \end{gather*}
    In questo modo posso combinare questa all'interno della differenziale e
    mettere in evidenza ogni termine in modo tale che si ottenga una combinazione
    di $a, b, c, k$ che mi permettano di trovare a destra $2x^{2} + 3e^{-x}$. Derivando e
    risolvendo dentro la differenziale si ottiene
    \begin{gather*}
        5a(x) ^{2} + x(-8a + 5b) + 2a - 4b +5c + e^{-x}(k + 4k + 5k) = 2x^{2} + 3e^{-x} \\
        5a(x) ^{2} + x(-8a + 5b) + 2a - 4b +5c  = 2x^{2} \\
        e^{-x}(k + 4k + 5k) = 3e^{-x} 
    \end{gather*}  
    Trovo allora i coefficienti di $a, b, c, k$ in modo tale che si risolva allora
    l'equazione
    \begin{gather*}
        a = \frac{2}{5} \\
        b = \frac{16}{25} \\
        c = \frac{44}{125} \\
        k = \frac{3}{10}
    \end{gather*}
    L'integrale generale (o soluzioni dell'equazione) si ottiene come 
    lo spazio dell'omogenea più questa soluzione che abbiamo trovato:
    \begin{gather*}
        y_T(x) = Ae^{2x}\cos x + Be^{2x}\sin x + \frac{2}{5}x^{2} + \frac{16}{25}c + \frac{44}{125} + \frac{3}{10}e^{-x} \qquad A, B \in \mathbb{R}    
    \end{gather*}
\end{example}

\begin{example}
    \begin{gather*}
        y'' -4y' + 5y = -e^{2x}\cos x 
    \end{gather*}
    Risolvo l'equazione omogenea associata
    \begin{gather*}
        y'' -4y + 5y = 0
    \end{gather*}
    Abbiamo lo stesso spazio delle soluzioni dell'omogenea di quella
    di prima, allora la soluzione completa dovrò valutare la traslazione del
    sottospazio affine risolvendo la completa. Cerco allora per similarità una
    soluzione per l'equazione 
    \begin{gather*}
        -e^{2x}\cos x = Re(-e^{(2 + i)x}  ) = Re(-e^{(2 - i)x} ) 
    \end{gather*}
    CI sono allora due metodi per risolvere e trovare la soluzione particolare:
    Il primo metodo è per similarità: cerco la radice nel campo $\mathbb{C}$ e trovo allora
    $\tilde{y_P}$ come soluzione complessa e chiamo $y_P = Re (\tilde{y_P})$:
    \begin{gather*}
        y'' - 4y' + 5y = -e^{(2 + i)x} 
    \end{gather*}
    Il secondo metodo è quello di utilizzare la combinazione lineare di seno e
    coseno moltiplicato per $e^{2x}$ (ossia svolgendo il numero complesso e ottenendo allora
    la seguente):
    \begin{gather*}
        y_P(x) = e^{2x}(a\cos x + b\sin x) 
    \end{gather*} 
    La soluzione particolare al grado uno qui sotto genera la stessa soluzione di quella sopra:
    \begin{gather*}
        (\alpha x + \beta)xe^{2x}(a\cos x + b\sin x) 
    \end{gather*}
\end{example}

\begin{theorem}[Teorema di struttura]
    Se una equazione differenziale lineare del secondo ordine è
    data da 
    \begin{gather*}
        ay'' + by' + cy = d(x)
    \end{gather*}
    La cui soluzione del polinomio associato è $\lambda_1 \neq \lambda_2$ e quindi la soluzione
    
\end{theorem}

\end{document}
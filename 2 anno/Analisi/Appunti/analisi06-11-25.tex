\documentclass[a4paper, oneside]{article}
\usepackage{graphicx}
\usepackage{amsthm}
\usepackage{amsmath}
\usepackage{amssymb}
\usepackage[a4paper,
            bindingoffset=0.2in,
            left=2cm,
            right=2cm,
            top=2cm,
            bottom=2cm,
            footskip=.25in]{geometry}
\usepackage[italian]{babel}
\usepackage{pgfplots}
\usepackage{tabularx}
\usepackage{tikz}
\usepackage{wrapfig}
\usepackage{color}
\usepackage[d]{esvect}
\usepackage{chemfig}
\usepackage{mhchem}
\definecolor{page}{rgb}{0.129,0.157,0.212}
\pagecolor{page}
\color{white}
\graphicspath{ {./images/} }
\usetikzlibrary{shapes.geometric}
\usetikzlibrary{datavisualization}
\usetikzlibrary{datavisualization.formats.functions}
\usetikzlibrary{patterns}
\pgfplotsset{width=10cm,compat=1.18}

\title{Analisi (Bianchini)}
\author{Tommaso Miliani}
\date{06-11-25}

\begin{document}
\newtheoremstyle{theoremEnv}
                {}          % Space above
                {}          % Space below
                {\slshape}  % Body font
                {}          % Indent amount
                {\bfseries} % Head font
                {.}         % Punctuation after head
                {\newline}  % Space after theorem head
                {}          % Theorem head spec
\theoremstyle{theoremEnv}

\newtheorem{definition}{Definizione}[section]
\newtheorem{theorem}{Teorema}[section]
\newtheorem{lemma}{Proposizione}[section]
\newtheorem{observation}{Osservazione}[section]
\newtheorem{corollary}{Corollario}[theorem]
\newtheorem{example}{Esempio}[section]
\newtheorem{remark}{Enunciato}[section]

\maketitle

\section{Equazioni differenziali associate alla famiglia di curve}
Dove
\begin{gather*}
    F_c(x, y) = 0
\end{gather*}
Sonon famiglie parametriche di curve con $c \in \mathbb{R}$. 
Se una curva parametrica è regolare questo vuol dire che deve essere
$C^{1}$ e il vettore derivato ha norma diversa da zero.
\begin{gather*}
    F_c \in C^{1}(A) \qquad A \subset \mathbb{R}^{2} 
\end{gather*}
Curve regolari localmente sono grafici di funzioni. Se una curva è regolare 
l'unica cosa che può succedere è che 
\begin{gather*}
    \frac{\partial f}{\partial y} \neq 0 
\end{gather*}
Quindi cerco $y = y(x)$ tale che $F_c(x, y) = 0$ che, per il teorema
del Dini è un grafico di funzione.  Dato che è costante, anche la
sua derivata lo deve essere, dunque
\begin{gather*}
    \frac{\partial }{\partial x}(F_c(x, y(x)))  = 0 \qquad \forall x \in \mathbb{I}
\end{gather*}
Ossia
\begin{gather*}
    \frac{\partial f(x, y(x))}{\partial x} + y'(x) \frac{\partial f(x, y(x))}{\partial y} = 0  
\end{gather*}
$\forall c \in \mathbb{R}$ vale questo. Dato che si hanno due equazioni in due incognite,
è possibile risolvere il sistema per determinare $x$ e $c$
\begin{example}
    \begin{gather*}
        y = cx^{2} \qquad F_c(x, y) = y - cx^{2}
    \end{gather*}
    E dunque la derivata rispetto ad $x$ della funzione parametrica
    \begin{gather*}
        y' - 2cx = 0 \qquad
        c = \frac{y}{x^{2}}
    \end{gather*}
    Dunque
    \begin{gather*}
        y' - 2\frac{y}{x} = 0 \ \Longrightarrow \ xy' -2y = 0
    \end{gather*}
    Imponendo le condizioni $y(0) = 0$ e $y'(0) = 0$ per trovare
    una famiglia di parabole a doppia funzione con vertice coincidente in $V = (0, 0)$.
\end{example}

\subsection{Ricerca di traiettorie ortogonali a $F_c(x, y) = 0$}
\begin{wrapfigure}{r}{0.4\textwidth}
    \centering
    \caption{}
    \begin{tikzpicture}
        
    \end{tikzpicture}    
\end{wrapfigure}
\begin{definition}[Traiettorie ortogonali]
    Le traiettorie ortogonali sono curve che intersecano in un unico punto ogni curva
    della famiglia e in tale punto sono ortogonali con la curva stessa. Ossia i rispettivi 
    vettori tangenti sono perpendicolari tra di loro
\end{definition}
Ossia la curva di partenza e quella derivata sono
\begin{gather*} 
    \begin{pmatrix} x \\
    y(x) \end{pmatrix} \qquad \begin{pmatrix} t \\
    u(t) \end{pmatrix}_{t \in J}
\end{gather*}
Dunque passano per lo stesso punto:
\begin{gather*}
    \begin{pmatrix} x \\
    y(x) \end{pmatrix}  \quad \exists t \in J : \begin{pmatrix} t \\
    u(t) \end{pmatrix} = \begin{pmatrix} x \\
    y(x) \end{pmatrix}  
\end{gather*}
E inoltre sonon perpendicolari i loro vettori derivati (e quindi 
anche loro lo sono):
\begin{gather*}
    \begin{pmatrix} 1 \\
    y'(x)\end{pmatrix} \perp \begin{pmatrix} 1 \\
    u'(t) \end{pmatrix}_{t = x} \ \Longrightarrow \ 1 + y'(x)u'(x) = 0 \ \Longrightarrow \  u'(x) = -\frac{1}{y'(x)}  
\end{gather*}
Posso allora trovare l'equazione differenziale omogenea per le traiettorie
ortogonali come
\begin{gather*}
    f(x, u(x), -\frac{1}{u'(x)}) = 0 
\end{gather*}

\begin{example}
    Cerco le traiettorie ortogonali alla famiglia di parabole localmente 
    ai grafici $(x, u(x))$ che deve soddisfare $f(x, u, -\frac{1}{u'}) = 0$. 
    Riprendendo l'esempio di prima deve risultare che
    \begin{gather*}
        -\frac{1}{u'(x)} - 2\frac{u}{x} = 0 \ \Longrightarrow \ 2uu' - x = 0
    \end{gather*}
    Per cui si ottiene, integrando che
    \begin{gather*}
        u^{2} = -\frac{x^{2}}{2} + k \qquad k \in \mathbb{R}
    \end{gather*}
    Dunque 
    \begin{gather*}
        u^{2} + \frac{x^{2}}{2} = k
    \end{gather*}
\end{example}

\begin{example}
    Trovare le curve regolari localmente grafico di $y(x)$ tale
    che $\forall (x, y) \in \gamma$, la distanza di $(x, y)$ da $Q$ è 
    $d(Q, u(0))$ dove $Q$ è la retta tangente a $\gamma$ del punto $(x, y)$
    e l'asse $y$. 
    \begin{gather*}
        \gamma : \underline{x}(t) = \begin{pmatrix} x(t) \\
        y(t) \end{pmatrix} 
    \end{gather*}
    Per il teorema del Dini è localmente grafico. Allora la retta tangente 
    nel punto $(x, y(x))$ è proprio
    \begin{gather*}
        \begin{pmatrix} x(t)  \\
        y(t)\end{pmatrix}  = \begin{pmatrix} 1 \\
        y'(x) \end{pmatrix} \tau + \begin{pmatrix} x \\
        y(x) \end{pmatrix}  \qquad \tau \in \mathbb{R}
    \end{gather*}
    Allora
    \begin{gather*}
        Y - y = y'(x)(X - x) \ \Longrightarrow \ Q = \begin{pmatrix} 0 \\
        y - xy'(x) \end{pmatrix} 
    \end{gather*}
    Imponendo le distanze al quadrato si ottiene l'equazione differenziale
    \begin{gather*}
        2xyy' = 1 + x^{2} + y^{2}
    \end{gather*} 
    Con $2yy' = (y^{2})'$ si ha che $u(x) = y^{2}(x)$ e dunque:
    \begin{gather*}
        xu' = 1 - x^{2} + u \ \Longrightarrow \ u' - \frac{u}{x} + \frac{x^{2} - 1}{x} = 0
    \end{gather*}
    
\end{example}


\end{document}
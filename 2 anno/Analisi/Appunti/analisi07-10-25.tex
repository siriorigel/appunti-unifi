\documentclass[a4paper, oneside]{article}
\usepackage{graphicx}
\usepackage{amsthm}
\usepackage{amsmath}
\usepackage{amssymb}
\usepackage[a4paper,
            bindingoffset=0.2in,
            left=2cm,
            right=2cm,
            top=2cm,
            bottom=2cm,
            footskip=.25in]{geometry}
\usepackage[italian]{babel}
\usepackage{pgfplots}
\usepackage{tabularx}
\usepackage{tikz}
\usepackage{wrapfig}
\usepackage{color}
\usepackage[d]{esvect}
\definecolor{page}{rgb}{0.129,0.157,0.212}
\pagecolor{page}
\color{white}
\graphicspath{ {./images/} }
\usetikzlibrary{shapes.geometric}
\usetikzlibrary{datavisualization}
\usetikzlibrary{datavisualization.formats.functions}
\usetikzlibrary{patterns}
\pgfplotsset{width=10cm,compat=1.9}

\title{Appunti di analisi}
\author{Tommaso Miliani}
\date{07-10-25}

\begin{document}
\newtheoremstyle{theoremEnv}
                {}          % Space above
                {}          % Space below
                {\slshape}  % Body font
                {}          % Indent amount
                {\bfseries} % Head font
                {.}         % Punctuation after head
                {\newline}         % Space after theorem head
                {}          % Theorem head spec
\theoremstyle{theoremEnv}

\newtheorem{definition}{Definizione}[section]
\newtheorem{theorem}{Teorema}[section]
\newtheorem{lemma}{Proposizione}[section]
\newtheorem{observation}{Osservazione}[section]
\newtheorem{corollary}{Corollario}[theorem]
\newtheorem{example}{Esempio}[section]

\maketitle

\section{Studio di funzioni a due variabili}
\subsection{Le derivate parziali}
\begin{wrapfigure}{r}{0.4\textwidth}
    \centering
    \caption{}
    \begin{tikzpicture}
        \draw(0, 0) -- (4, 0);
        \draw(0, 0) -- (0, 2);
        \draw(0, 0) -- (-1, -2);
        \draw[thick] (2, 0) -- (1, -2) node[at start, above] {$y_0$};
        \draw[thick](-0.5, -1) -- (4, -1) node[at start, left] {$x_0$};
        \filldraw(1.5, -1) circle (1pt) node[anchor = north west] {$P_0$};
        \draw[thin](1.5, -1) circle (0.2);
    \end{tikzpicture}    
\end{wrapfigure}
Preso un insieme aperto $A \subset \mathbb{R}^{2}$ e presa una funzione
\begin{gather*}
    f: A \to \mathbb{R}, \qquad (x_0, y_0) \in A
\end{gather*} 
Posso considerare delle sezioni trasversali con un $y = y_0$ fissato.
Io posso considerare solo le sezioni con un $y$ fissato e posso definire la funzione 
\begin{gather*}
    z = f(x, y_0) = v(x)
\end{gather*} 
Il suo grafico corrisponde alla sezione di quella superficie nel disegno.
Posso ora definire le derivate parziali rispetto a $x$ come il limite
\begin{gather*}
    \lim_{h \to 0} \frac{v(x_0 + h)- v(x_0) }{h} \ \Longrightarrow \ \lim_{h \to 0}\frac{f(x_0 + h, y_0) - f(x_0, y_0)}{h} 
\end{gather*}
Ossia sto valutando l'incremento su $x_0$ fissato $y$. Posso ottenere la stessa cosa
fissando $x_0$: in questo caso si può ottenere il grafico della funzione
come sezione della superficie lungo un $x_0$ fissato:
\begin{gather*}
    \lim_{h \to 0} \frac{u(y_0 + h)- u(y_0) }{h} \ \Longrightarrow \ \lim_{h \to 0}\frac{f(x_0, y_0 + h) - f(x_0, y_0)}{h} 
\end{gather*}
SI possono allora indicare rispettivamente con
\begin{align}
    \frac{\partial f}{\partial x}(x_0, y_0)  = f_X(x_0, y_0)\qquad \qquad \frac{\partial f}{\partial y}(x_0, y_0) = f_Y(x_0, y_0)  
\end{align}
\begin{definition}
    Se esistono i due rapporti incrementali allora $f$ è detta derivabile
    ed esiste
    \begin{align}
        Df(x_0, y_0) = \nabla f(x_0, y_0) = \begin{pmatrix}
            f_X(x_0, y_0) & f_Y(x_0, y_0)
        \end{pmatrix}
    \end{align}
\end{definition}

\begin{observation}
    Le funzioni elementari sono derivabili in un insieme aperto all'interno
    del loro dominio. Non si possono fare su un insieme chiuso in quanto
    devo potermi muovere con $\pm h$. Valgono inoltre
    le regole di derivazione dell'algebra sia rispetto ad $x$ che $y$.
\end{observation}
\begin{example}
    Prendiamo una funzione
    \begin{gather*}
        f(x, y) = xe^{x^{2} + y^{2}  } 
    \end{gather*}
    E vogliamo calcolare la derivata nel punto $(0, 0)$. Si
    può applicare la definizione:
    \begin{gather*}
        \frac{\partial f}{\partial x}(0, 0) = \lim_{h \to 0} \frac{f(h, 0) - f(0, 0)}{h} = \lim_{h \to 0} \frac{he^{h^{2} } }{h} = 1 \\
        \frac{\partial f}{\partial y}(0, 0) = \lim_{k \to 0} \frac{f(0, k) - f(0, 0)}{k} = 0
    \end{gather*}
    Questo vuol dire che $f$ è derivabile in $(0, 0)$ e $D(f(0, 0)) = (1, 0)$. 
    Da notare come le derivate possono essere diverse. Se volessi derivarlo in altri punti $(x, y)$ 
    potrei utilizzare tutta la teoria delle derivate ad una variabile e osservare
    come questa funzione sia composizione di funzioni derivabili che è ancora
    derivabile. 
    \begin{gather*}
        \frac{\partial f}{\partial x}(x, y) = e^{x^{2} + y^{2}  }  + 2x^{2}e^{x^{2} + y^{2}  }  \\
        \frac{\partial f}{\partial y}(x, y) = 2yxe^{x^{2} + y^{2}  }  
    \end{gather*}
    Se volessi esprimere il gradiente allora
    \begin{gather*}
        Dg(x, y) = \begin{pmatrix}
            (1 + 2x^{2} )e^{x^{2} + y^{2}  } & 2xye^{x^{2} + y^{2}  } 
        \end{pmatrix}
    \end{gather*}
\end{example}

\begin{observation}
    La composizione di funzioni derivabili in due variabili è anch'essa
    derivabile e si applicano le regole di derivazione.
\end{observation}

\begin{example}
    \begin{gather*}
        f(x, y) = \sin(xy)
    \end{gather*}
    Stabilire se esiste $(x_0, y_0)\in \mathbb{R}^{2}$ il gradiente
    della funzione sia parallelo al vettore $(1, 0)$. Posso ora applicare la derivata rispetto
    alle funzioni composte:
    \begin{gather*}
        \frac{\partial f}{\partial x}(x, y) = y\cos(xy) \\
        \frac{\partial f}{\partial y}(x, y) = x\cos(xy)  
    \end{gather*} 
    ALlora posso esprimere il gradiente come il vettore
    riga
    \begin{gather*}
        Df(x, y) = \begin{pmatrix}
            y\cos(x, y) & x\cos(xy)
        \end{pmatrix}
    \end{gather*}
    Allora esiste il gradiente se e solo se $\exists k \in \mathbb{R}, k \neq 0$ in modo
    tale che
    \begin{gather*}
        Df(x, y) = k(1, 0) \Longleftrightarrow \left\{\begin{array}{l}
            y\cos(xy) = k \\
            x\cos(xy) = 0
        \end{array}\right.
    \end{gather*}
    Si osserva che $x = 0$ e dunque $y =k$ per cui tutti i punti
    dell'asse verticale soddisfano la condizione e dunque il gradiente
    della funzione è $(y, 0) , y \in \mathbb{R}$ che soddisfano le condizioni
    di parallelismo del gradiente.  
\end{example}

\begin{example}
    \begin{gather*}
        f(x, y) = (x - y)\sqrt{|y - x^{2}| } 
    \end{gather*}
    Questa funzione è definita $D = \{(x, y) \in \mathbb{R}^{2} \} = \mathbb{R}^{2}$; voglio
    vedere se è derivabile nel punto $(1, 1)$. La radice non sempre è
    derivabile: infatti è derivabile se e solo se l'argomento è 
    diverso da zero, altrimenti non è derivabile. Nel punto 
    $(1, 1)$ l'argomento fa zero e dunque non ho informazioni sulla derivabilità
    della funzione. Non posso applicare allora la derivabilità delle funzioni
    elementari: devo allora procedere ad eseguire il limite
    del rapporto incrementale. 
    \begin{gather*}
        \frac{\partial f}{\partial x}(1, 1) = \lim_{h \to 0} \frac{f(1 + h, 1) - f(1, 1)}{h} = 0
    \end{gather*} 
    Rispetto ad $x$ è allora derivabile e in quel punto fa esattamente zero. La derivata
    parziale rispetto ad $y$, se esiste, è:
    \begin{gather*}
        \frac{\partial f}{\partial y}(1, 1) = \lim_{h \to 0} \frac{f(1, 1 + h) - f(1, 1)}{h} = 0  
    \end{gather*}
    Ho dimostrato allora che $f$ è derivabile nel punto $(1, 1)$ e che il
    gradiente in quel punto è il vettore nullo. 
\end{example}

\begin{example}[Il gradiente non esiste]
    \begin{gather*}
        f(x, y) = \sqrt{x^{2} + y^{2}  } 
    \end{gather*}
    Dato che la funzione della radice non è derivabile nel punto $(0, 0)$, allora
    devo vedere con il rapporto incrementale se esistono i limiti
    \begin{gather*}
        \lim_{h \to 0} \frac{f(h, 0) - f(0, 0)}{h} = \lim_{h \to 0} \frac{\sqrt{h^{2} } }{h}   
    \end{gather*}
    Questo limite non esiste in quanto potrebbe essere $\pm 1$. Non è derivabile
    invece rispetto ad $y$ in quanto questa funzione è pari. Allora il gradiente nel
    punto $(0, 0)$ non esiste. 
\end{example}

\begin{definition}
    Si definisce una funzione \textbf{pari} una funzione a due variabili se
    \begin{align}
        f(x, y) = f(y, x)
    \end{align} 
\end{definition}

\begin{example}[La derivabilità non implica sempre la continuità]
    \begin{gather*}
        f(x, y) = \left\{\begin{array}{l}
            \frac{xy}{x^{2} + y^{2}  } \quad (x, y) \neq (0, 0) \\
            0 \qquad (x, y) = (0, 0)
        \end{array}\right.
    \end{gather*}
    Non potendo utilizzare la derivazione delle funzioni semplici poiché non so
    se questa funzione è continua, devo andare a studiare la derivabilità 
    della funzione attraverso il rapporto incrementale:
    \begin{gather*}
        \frac{\partial f}{\partial x}(0, 0) = \lim_{h \to 0} \frac{f(h, 0) - f(0, 0)}{h} = \lim_{h \to 0} \frac{0}{h} = 0 \\
        \frac{\partial f}{\partial y}(0, 0) = \lim_{h \to 0} \frac{f(0, h) - f(0, 0)}{h} = \lim_{h \to 0} \frac{0}{h} = 0      
    \end{gather*}
    La funzione è allora derivabile. Si può verificare ora se è continua:
    \begin{gather*}
        \lim_{(x, y) \to (0, 0)} f(x, y) - f(0, 0) =  \lim_{(x, y) \to (0, 0)} \frac{xy}{x^{2} + y^{2}} - 0 
    \end{gather*}
    Se si passasse in coordinate polari si osserverebbe che questo limite non esisterebbe,
    allora non è continua poiché dipende dal percorso scelto. 
\end{example}

\begin{observation}
    Se $f$ è una funzione a due variabili ed è derivabile in
    un punto $(x_0, y_0)$, non sempre $f$ è continua in quel punto. 
    Non esiste il piano tangente al grafico. Il concetto di differenziabilità
    e derivabilità sono due cose molto diverse: la differenziabilità ci dà 
    le proprietà delle derivate nelle funzioni ad una variabile. 
\end{observation}

\subsection{Derivate di ordine successivo}
Si può definire la derivata rispetto ad una delle due variabili
di una funzione a due variabili come
\begin{align}
    \frac{\partial^{2} f }{\partial x^{2} }(x, y) = \frac{\partial}{\partial x}\left(\frac{\partial f}{\partial x} (x, y) \right)  \\
    \frac{\partial^{2} f}{\partial x \partial y}(x, y) = \frac{\partial }{\partial y} \left(\frac{\partial f}{\partial x}(x,y) \right)  \\
    \frac{\partial^{2} f }{\partial y^{2} }(x, y) = \frac{\partial }{\partial y}\left(\frac{\partial f}{\partial y} (x, y) \right)  \\
    \frac{\partial^{2} f}{\partial y \partial x}(x, y) = \frac{\partial }{\partial x} \left(\frac{\partial f}{\partial y}(x,y) \right)  
\end{align}
Le derivate miste sono quell per cui prima si deriva ad una variabile e poi
con l'altra nella derivata seconda.
Si indica il gradiente al quadrato è dato dato dalla matrice delle derivate
parziali che prende il nome di \textbf{matrice Hessiana}
\begin{align}
    D^{2}f(x, y) = \begin{pmatrix}
        f_{x, x}(x, y) & f_{xy}(x, y) \\
        f_{y, x}(x, y) & f_{yy}(x, y)
    \end{pmatrix}
\end{align}

\begin{definition}
    Si definisce la traccia del gradiente al quadrato come l'operatore \textbf{Laplaciano}
\end{definition}

\begin{example}
    \begin{gather*}
        f(x, y) = \sin(x^{3} + y^{3}  )
    \end{gather*}
    Allora la derivata doppia è rispetto a $x$:
    \begin{gather*}
        \frac{\partial^{2} f}{\partial x^{2} }(x, y) = \frac{\partial f}{\partial x} \left(3x^{2}\cos(x^{3} + y^{3}  ) \right) =  \\
        \frac{\partial^{2} f}{\partial y^{2} }(x, y) = \frac{\partial f}{\partial y} \left(3y^{2}\cos(x^{3} + y^{3} ) \right) =     
    \end{gather*}
\end{example}


\begin{theorem}[Teorema di Schwarz]
    Preso un insieme aperto $A \subset \mathbb{R}^{2} $ con un punro qualsiasi
    $(x_0, y_0) \in A$ tale che la funzione $f: A \to \mathbb{R}$ sia
    derivabile due volte, allora se le derivate miste
    $f_{xy}, f_{yx}$ sono continue in $(x_0, y_0)$ allora le
    loro derivate sono uguali: si ottiene allora che la matrice Hessiana è simmetrica.
\end{theorem}
\begin{proof}
    Si farà
\end{proof}
\begin{observation}
    L'ipotesi di continuità delle derivate miste non può essere rimossa. 
\end{observation}
\begin{example}
    \begin{gather*}
        f(x, y) = \left\{\begin{array}{l}
            \frac{x^{3}y - xy^{3}  }{x^{2} + y^{2}  } \qquad (x, y) \neq (0, 0) \\
            0 \qquad \qquad\quad (x, y) = (0, 0)
        \end{array}\right.
    \end{gather*}
    Posso allora cercare, se $f \in C^{0}( \mathbb{R }^{2} )$, posso dire, facendo le derivate
    parziali:
\end{example}


\end{document}
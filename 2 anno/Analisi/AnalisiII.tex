\documentclass[a4paper, oneside]{book}
\usepackage{graphicx}
\usepackage{amsthm}
\usepackage{amsmath}
\usepackage{amssymb}
\usepackage[a4paper,
            bindingoffset=0.2in,
            left=2cm,
            right=2cm,
            top=2cm,
            bottom=2cm,
            footskip=.25in]{geometry}
\usepackage[italian]{babel}
\usepackage{pgfplots}
\usepackage{tabularx}
\usepackage{tikz}
\usepackage{wrapfig}
\usepackage{color}
\definecolor{page}{rgb}{0.129,0.157,0.212}
\pagecolor{page}
\color{white}
\graphicspath{ {./images/} }
\usetikzlibrary{shapes.geometric}
\usetikzlibrary{datavisualization}
\usetikzlibrary{datavisualization.formats.functions}
\usetikzlibrary{patterns}
\pgfplotsset{width=10cm,compat=1.9}

\title{Appunti di Analisi 2}
\author{Tommaso Miliani}
\date{2025/2026}

\begin{document}
\newtheoremstyle{theoremEnv}
                {}          % Space above
                {}          % Space below
                {\slshape}  % Body font
                {}          % Indent amount
                {\bfseries} % Head font
                {.}         % Punctuation after head
                {\newline}         % Space after theorem head
                {}          % Theorem head spec
\theoremstyle{theoremEnv}

\newtheorem{definition}{Definizione}[chapter]
\newtheorem{theorem}{Teorema}[chapter]
\newtheorem{lemma}{Proposizione}[chapter]
\newtheorem{observation}{Osservazione}[chapter]
\newtheorem{corollary}{Corollario}[theorem]
\newtheorem{example}{Esempio}[chapter]

\maketitle

\tableofcontents

\chapter*{Introduzione al corso}
Il corso di analisi 2 si concentra sullo studio di equazioni differenziali, 
calcolo differenziale per funzioni vettoriali (ossia funzioni che
da un vettore restituiscono un vettore), equazioni a più variabili e
infine successioni e serie di funzioni. \\
L'esame di analisi 2 è composto da uno scritto e da un orale:
lo scritto si passa con un voto minimo di 16 e durante l'anno ci saranno
due prove parziali che sarà a Novembre (13) e l'altro parziale i primi giorni
di Gennaio. 
\begin{itemize}
    \item Si può usare un libro di testo ed un foglio A4 (UNO SOLO) nel quale
    c'è tutto quello che si riesce a scriverci.
    \item Si può riprovare un parziale che si è fatto male 
    \item Si può partecipare ai parziali anche se non si è fatto analisi 1
    \item Il libro di testo consigliato è il "Fuscolo, Marcellini Sbordone Lezioni di analisi matematica 2"
    oppure il "Bramanti Pagani Salsa analisi matematica 2" che purtroppo per alcuni argomenti non è completo.
    \item Nessuna dispensa quindi per esercizi svolti rifarsi ai libri di testo "Esercitazioni di analisi due". 
\end{itemize}


\chapter{Equazioni differenziali}
\begin{definition}[Equazioni differenziali]
    Un'equazione differenziale è una equazione in cui l'incognita è una funzione 
    di una qualche variabile. L'incognita è dunque $y = f(x)$. Si dice che una equazione differenziale
    è ordinaria se dipende solo da una variabile; nel caso di equazioni differenziali
    che dipendono da più variabili prendono il nome di equazioni differenziali a
    \textbf{derivate parziali}.
\end{definition}

\begin{example}[Esempi di equazioni differenziali]
    \begin{gather*}
        y'' - y = 0 \qquad y' = 3e^{y}  \qquad y' = 4x + 8
    \end{gather*}
    La soluzione della prima è $y = \sin x$ in quanto $y'' = -\sin x$ 
    e dunque $\forall x \in \mathbb{R} \ 0 = 0$. 
\end{example}

\begin{definition}[Ordine]
    Si definisce \textbf{ordine} il numero della derivata massima che
    compare all'interno delle equazioni differenziali. 
\end{definition}

\begin{definition}[Forma normale]
    Una equazione si dice in forma \textbf{normale} se la derivata più grande è
    isolata rispetto alle altre
    \begin{align}
        y^{(n)} = f(x, y, y', \dots, y^{(n - 1)} )
    \end{align}
\end{definition}
\begin{example}
    \begin{gather*}
        (y')^{2} + x^{3} = 0   \ \Longrightarrow \ y ' = \pm \sqrt{x^{3} } 
    \end{gather*}
    Non riesco a ricavare una equazione del primo ordine del tipo $F(x, y, y')$. 
    Quando riesco ad isolare da una parte la derivata più grande si riesce a semplificare
    la risoluzione delle equazioni differenziali.
\end{example}

\section{Equazioni differenziali del primo ordine}
La forma generale delle equazioni di primo ordine è del tipo
\begin{align}
    y' = f(x, y)
\end{align}
Una soluzione per questa forma generale di equazioni differenziale è una
funzione $y$ della variabile $x$ definita per $x \in \mathbb{A}$ (un qualche insieme)
e tale che per ogni $x \in \mathbb{A}$ risulti che $(x, y(x))$ appartenente al dominio di
$f(x, y)$ e inoltre la derivata di questa funzione calcolata in $x$ è uguale a $y'(x) = f(x, y(x))$.
\begin{example}[Esempio più semplice]
    \begin{gather*}
        y' = g(x) 
    \end{gather*}
    Dove $g(x) $ è una funzione continua appartenente ad un qualche insieme $\mathbb{A}$ 
    e quindi un'equazione di questo tipo equivale a trovare la primitiva di $g$. 
    Allora posso dire che $G(x) + c$ è soluzione dell'equazione differenziale, ossia la
    primitiva di $g(x)$. 
\end{example}
\begin{example}[Esempio di prima con soluzione diversa]
    Nell'equazione differenziale
    \begin{gather*}
        y'' - y = 0
    \end{gather*}
    Anche il coseno è soluzione in quanto si otterrebbe $y = \cos x$. Si ottiene
    allora che la seguente è anch'essa soluzione dell'equazione:
    \begin{gather*}
        C_1 \sin x + C_2 \cos x = 0
    \end{gather*}
    Generalmente l'equazione di $n$ ordine ha $n$ costanti $C_1, \dots, C_n$ che moltiplicano. 
\end{example}
\begin{example}
    \begin{gather*}
        \left\{\begin{array}{l}
            y' = x^{2} \\
            y(1) = 2 
        \end{array}\right.
    \end{gather*}
    Si ha che la primitiva di $y'$ è proprio $y = \frac{x^{3} }{3} + C$, data la seconda condizione
    nel sistema allora si ottiene che
    \begin{gather*}
        \frac{1}{3}  + C = 2 \ \Longrightarrow \ C = \frac{5}{3}
    \end{gather*}
\end{example}
\section{Il problema di Cauchy}
Se in una equazione differenziale del primo ordine avessi una situazione del tipo
\begin{gather*}
    \left\{\begin{array}{l}
        y' = f(x, y) \\
        y(x_0) = y_0
    \end{array}\right.
\end{gather*}
Ossia se ho delle condizioni sulle derivate fino alla $n-1$ esima, allora
questa tipologia di esercizi prenderà il nome di \textbf{problema di Cauchy}:
nel caso di una equazione del secondo ordine le condizioni non sono qualsiasi ma 
seguono il seguente schema:
\begin{gather*}
    \left\{\begin{array}{l}
        y'' = f(x, y, y') \\
        y(x_0) = y_0 \\
        y'(x_0) = y_1
    \end{array}\right.
\end{gather*}
Esistono alcuni teoremi che permettono di definire l'insieme delle soluzioni date
le condizioni del secondo membro di ciascuno dei due sistemi. Iniziamo dai
teoremi di esistenza delle soluzioni per il problema di Cauchy del primo ordine
in forma normale 
\begin{gather*}
    \left\{\begin{array}{l}
        y' = f(x, y) \\
        y(x_0) = y_0
    \end{array}\right.
\end{gather*}
SI definiscono alcune ipotesi prima di procedere:
\begin{definition}[Punto interno ad un insieme]
    \begin{gather*}
        Se \ \mathbb{A} \subset \mathbb{R}^{2}, (x_0, y_0) \in \mathbb{A} 
    \end{gather*}
    Si dice che $(x_0, y_0)$ è interno all'insieme $\mathbb{A}$, se
    esiste un cerchio di raggio positivo tale per cui il centro sia $(x_0, y_0)$ e sia contenuto
    dentro all'insieme $\mathbb{A}$. Nel caso in cui $\mathbb{A} \subset \mathbb{R}^{3}$ allora diciamo che un punto
    $(x_0, y_0, z_0)$ è interno all'insieme $\mathbb{A}$ se esiste una sfera di raggio positivo tale per cui
    $(x_0, y_0, z_0)$ sia il centro e sia contenuto all'interno dell'insieme.
\end{definition}
\begin{theorem}[Teorema di Peano]
    Se la funzione $f(x, y)$ è una funzione definita e continua  in un qualche insieme
    $\mathbb{A}$ e $(x_0, y_0) \in \mathbb{A}$, allora il problema di Cauchy ammette almeno una soluzione 
    definita in un intorno di $x_0$.  
\end{theorem}
Da questo possiamo fare due considerazioni:
\begin{enumerate}
    \item La continuità di $f(x, y)$ è necessaria per l'esistenza di una soluzione:
    \item La sola ipotesi di continuità non è garantisce che esista una sola soluzione
\end{enumerate}
\begin{example}[differenziale risolta col problema dii Cauchy]
    \begin{gather*}
        \left\{\begin{array}{l}
            y' = y^{2/3}  \\
            y(0) = 0
        \end{array}\right.
    \end{gather*}
    Allora dato che $f(x, y) = y^{2/3}$ è definita continua sull'intervallo $\mathbb{R} \times \mathbb{R}$ e quindi
    una funzione costante  $f(x) = 0$ ha valore zero in zero e derivata prima uguale a zero. Anche $y(x) = \frac{x^{3} }{27}$ è soluzione
    in quanto la derivata di questa funzione $y' = \frac{x^{2} }{9}$ ed è uguale a zero se calcolata in zero.  Questo problema di Cauchy
    ha allora due soluzioni distinte anche se in realtà ha soluzioni infinite:
    \begin{gather*}
        y(x) = \left\{\begin{array}{l}
            0 \ se \ x \in (-\infty , 0) \\
            \frac{(x - a)^{3} }{27} \ se \ x \in (a, +\infty)
        \end{array}\right.
    \end{gather*}
    Il teorema di Cauchy allora mi garantisce che esitano almeno una soluzione ma 
    non mi da la certezza che sia solo una (potrebbero essere infatti infinite).
\end{example}


\section{I teoremi di esistenza di Cauchy}
\begin{wrapfigure}{r}{0.35\textwidth}
    \centering
    \caption{}
    \begin{tikzpicture}
        \draw[->](-1, 0) -- (4, 0) node[at end, below] {$z$};
        \draw[->](0, -1) -- (0, 3) node[at end, left] {$g(z)$};
        \draw(0, 0) .. controls (1, 2) and (1.5, 1) .. (2, 1);
        \draw(2, 1) .. controls (2.5, 1.5) and (3.5, 1.2) .. (4, 1.25) node[at end, right] {$g(z)$};
        \draw(0.7, 1) -- (2, 1);
        \filldraw(0.7, 1) circle (1pt);
        \filldraw(2, 1) circle (1pt);
        \draw[|-|](0.3, 0.2) -- (0.3, -0.2) node[at end, below] {$a$};
        \draw[|-|](2.3, 0.2) -- (2.3, -0.2) node[at end, below] {$b$};
        \draw[dashed](0.7, 0) -- (0.7, 1) node[at start, below] {$z_1$};
        \draw[dashed](2, 0) -- (2, 1) node[at start, below] {$z_2$};
        \draw[<->](0.2, -0.7) -- (2.8, -0.7) node[midway, below] {$I$}; 
    \end{tikzpicture}    
\end{wrapfigure}
Per spiegare il teorema di esistenza e di unicità bisogna chiarire innanzitutto
cosa è una funzione Lipschitziana.
\begin{definition}[Funzione Lipschitziana]
    Presa una funzione definita come $g : I \to \mathbb{R}$, dove $I \subset \mathbb{R}$; posto
    $[a, b] \subset I$, si dice allora che la funzione $g$ è Lipschitziana nell'intervallo $[a, b]$ se
    $\exists L \in \mathbb{R} > 0$ tale che comunque si scelgano $a, b$ risulta che,
    per ogni coppia di numeri $z_1, z_2 \in [a, b]$, 
    \begin{align}
        g(z_1) - g(z_2) \leq L|z_1 - z_2|
    \end{align}
    Che è equivalente alla seguente formulazione:
    \begin{align}
        -L \leq \frac{g(z_1) - g(z_2)}{z_1 - z_2} \leq L
    \end{align}
    Ossia ogni funzione ha un certo valore massimo per la sua derivata
    prima. Una funzione non è Lipschitziana se presenta dei flessi a tangenti verticali o dei
    punti di non derivabilità.
\end{definition}
Si può dare un esempio di una funzione che è Lipschitziana solo per un certo intervallo:
\begin{example}[Esempio di una funzione potenza]
    \begin{gather*}
        g(z) = z^{\alpha}, \alpha \in (0,  1) 
    \end{gather*}
    Se si prende un intervallo che non contiene lo zero allora è Lipschitziana
    in quanto le rette avranno sempre una pendenza ben definita. Se invece si prendesse un intervallo che contiene anche lo
    zero, allora si vede che non è una funzione Lipschitziana perché il rapporto
    incrementale diventa grande quanto si vuole. 
\end{example}
\begin{definition}[Derivate parziali]
    Presa $f(x, y)$ definita nel punto $(x_0, y_0)$ all'interno del
    dominio di $F$. Si ottiene allora che la derivata parziale rispetto 
    alla $y$ 
    \begin{align}
        \frac{\partial f}{\partial y}(x_0, y_0) = \lim_{h \to 0}   \frac{F(x_0, y_0 + h) - F(x_0, y_0)}{h}
    \end{align}
    Ossia eseguo la derivata rispetto alla variabile considerata, considerando
    come costanti tutte le variabili per cui non derivo.
\end{definition}
\begin{example}[Esempio pratico]
    \begin{gather*}
        f(x, y) = xy^{3} + \sin y \\
        \frac{\partial f}{\partial y}(3, y_0) = 9y^{2} + \cos y    
    \end{gather*}
    Nel punto $(3, y_0)$ la derivata della funzione rispetto a $y$ diventa $3y^{3} + \sin y$. 
\end{example}



\begin{wrapfigure}{r}{0.4\textwidth}
    \centering
    \caption{}
    \begin{tikzpicture}
        \draw[->](0, 0) -- (4, 0);
        \draw[->](0, 0) -- (0, 3);
        \draw[|-|](2, 0.2) -- (2, -0.2) node[at end, below] {$x_0$};
        \draw[|-|](2.5, 0.2) -- (2.5, -0.2);
        \draw[|-|](1.5, 0.2) -- (1.5, -0.2);
        \draw[|-|](0.2, 2) -- (-0.2, 2) node[at end, left] {$y_0$}; 
        \draw[|-|](0.2, 2.5) -- (-0.2, 2.5);
        \draw[|-|](0.2, 1.5) -- (-0.2, 1.5);
        \filldraw (2, -0.5) node[anchor = north] {$I$};
        \filldraw (-0.7, 2) node[anchor = east] {$J$};
        \draw[dashed](0.2, 2.5) -- (2.5, 2.5);
        \draw[dashed](0.2, 1.5) -- (2.5, 1.5);
        \draw[dashed](1.5, 0.2) -- (1.5, 2.5);
        \draw[dashed](2.5, 0.2) -- (2.5, 2.5);
    \end{tikzpicture}    
\end{wrapfigure}
Possiamo ora introdurre il seguente teorema
\begin{theorem}[Teorema di esistenza di Cauchy]
    Si considera il problema di Cauchy per una differenziale di primo
    ordine 
    \begin{gather*}
        \left\{\begin{array}{l}
            y' = f(x, y) \\
            y(x_0) = y_0
        \end{array}\right.
    \end{gather*}
    Possiamo supporre allora che ogni $f(x, y)$ sia definita $\forall (x, y) \in I \times J$, ossia su di un
    prodotto cartesiano tra due sottoinsiemi qualsiasi di $\mathbb{R}:I = (x - x_0, x + x_0)$ e $J = (y - y_0, y + y_0)$. 
    Supponiamo allora le seguenti ipotesi:
    \begin{enumerate}
        \item $f(x, y)$ è continua in $I \times J$;
        \item $\exists$ costante $L > 0$ tale che la funzione $f(x, y)$ rispetta la definizione
        di funzione Lipschitziana nell'intervallo $J \ \forall x \in I$, $\forall y_1, y_2 \in J$. 
        \end{enumerate}
    Allora esisterà $\delta > 0$ e esisterà una funzione $y(x)$ che risolve il sistema in una funzione definita
    nell'insieme $(x_0 - \delta, x_0 + \delta)$. 
\end{theorem}
Per poter dimostrare questo teorema dovremmo allora introdurre prima dei lemmi
che ci consentano di dimostrare il teorema
\begin{proof}
    \begin{lemma}[Equivalenza tra il teorema del calcolo integrale e la derivata della differenziale]
        Sia $\delta > 0$ e supponiamo che siano valide le
        ipotesi del teorema, allora le seguenti sono equivalenti
        \begin{enumerate}
            \item $\exists y(x)$ derivabile in $[x_0 - \delta, x_0 + \delta]$ tale che 
            $y'(x) = f(x, y(x)) \ \forall x \in [x_0 - \delta, x_0 + \delta], y(x_0) = y_0$.
            \item $\exists y(x)$ funzione continua in $[x_0 - \delta, x_0 + \delta]$ tale che
            \begin{gather*}
                y(x) = y_0 + \int_{x_0}^{x}f(t, y(t)) \ dt, \qquad \forall x \in (x_0 - \delta, x_0 + \delta). 
            \end{gather*} 
        \end{enumerate}
    \end{lemma}
    \begin{proof}
        Posso dimostrare questa attraverso il teorema del calcolo integrale:
        se $g(z)$ è continua, allora posso scrivere che $g(z) = g(z_0) + \int_{z_0}^{z}g'(t) \ dt$. \\
        $ 1 \ \Longrightarrow \ 2$ So che $y(x)$ è una funzione che soddisfa la prima definizione e allora $y'$ è continua e quindi dal teorema
        fondamentale del calcolo integrale io so che
        \begin{gather*}
            y(x) = y(x_0) + \int_{x_0}^{x}y'(t) \ dt
        \end{gather*}
        Allora posso dire che, dato $y'(t) = f(t, y(t))$ , posso
        sostituire nell'integrale e ottenere la tesi. \\
        $2 \ \Longrightarrow \ 1$: Sia $y$ una funzione che soddisfi la
        seconda proposizione, allora posso utilizzare un'altra forma del teorema del
        calcolo integrale per cui se $h$ è continua, allora la funzione $h(x)  \to z_0 + \int_{x_0}^{x} h(t) \ dt$ e quindi
        se $h$ è continua e derivabile, la derivata di questa funzione è esattamente $h(x)$. Si ottiene che la funzione
        all'interno della tesi dell'integrale è continua e quindi l'integrale è una funzione derivabile e la sua derivata
        è esattamente (con l'implicita formulazione che $y(x_0) = y_0$)
        \begin{gather*}
            y'(x) = f(x, y(x))
        \end{gather*}
        Proprio come si voleva ottenere.
    \end{proof}
    \begin{lemma}[Se valgono queste considerazioni le ipotesi del teorema sono verificate]
        Se $f$ è continua in un insieme $\mathbb{A}$, $(x_0, y_0)$ è interno all'insieme
        e inoltre si ha che
        \begin{enumerate}
            \item $                \exists \frac{\partial f}{\partial y}(x, y)$.
            \item $\frac{\partial f}{\partial y}(x, y)$ è continua 
        \end{enumerate}
        $\forall (x, y) \in \mathbb{A}$ e allora le ipotesi di questo teorema sono
        verificate. 
    \end{lemma}
    La dimostrazione è da finire nelle prossime lezioni.
\end{proof}
\begin{example}[Funzione non Lipschitziana]
    \begin{gather*}
        \left\{\begin{array}{l}
            y' = y^{\frac{2}{3}} \\
            y(0) = 0 
        \end{array}\right.
    \end{gather*}
    Quindi dico che il mio $x_0$ e $y_0$ contengano l'origine, se io facessi il
    grafico di $y^{\frac{2}{3}}$, allora si ottiene che questo grafico è una radice simmetrica rispetto
    all'asse $y$ e per questo non è Lipschitziana nell'intorno di $0$: le immagini non comprendono
    le ordinate negative. 
\end{example}

\section{Equazioni a variabile separata}
Le equazioni di funzioni a più variabili prendono il nome di funzioni a variabili separabili, ossia
della forma 
\begin{align}
    y' = a(x) b(y) \ \Longrightarrow \ y'(x) = a(x)b(y(x))
\end{align}
Dove $a(x)$ è continua in $I \subset \mathbb{R}$ e $b(y)$ è continua
in $J \subset \mathbb{R}$. Posso trovare delle soluzioni per queste equazioni nella
seguente maniera: se $\overline{y}$ è univoco tale che $b(\overline{y}) = 0$ allora
la funzione $y(x) \equiv \overline{y}$ è soluzione. Se si suppone che
$b(y) \neq 0$ allora posso dire che se due funzioni sono uguali il loro integrale sarà uguale:
\begin{gather*}
    \frac{y'}{b(y)} = a(x) \ \Longrightarrow \ \frac{y'(x)}{b(y(x))} = a(x) \ \Longrightarrow \ \int\frac{y'(x)}{b(y(x))} \ dy = \int a(x) \ dx
\end{gather*}
Se ponessi ora $y = y(x)$ che soddisfano questa equazione
\begin{gather*}
    \int \frac{dy}{b(y)} = \int a(x) \ dx
\end{gather*}
Se $B(y)$ è una primitiva, allora posso dire che 
\begin{gather*}
    B(y) = A(x) + c
\end{gather*}
Se si riuscisse a invertire invece la funzione $B$ si ottiene che esiste una soluzione,
la quale è data proprio da:
\begin{gather*}
    y(x) = B^{-1} (A(x) + c)
\end{gather*}
\begin{example}[Esempio di separazione di variabili]
    \begin{gather*}
        \left\{\begin{array}{l}
            y' = \frac{1}{y} \\
            y(0) = 2
        \end{array}\right.
    \end{gather*}
    Dato che $ a(x) \equiv 1$ e $b(y) = \frac{1}{y}$, si può integrare ed ottenere che
    \begin{gather*}
        \frac{dy}{dx} = \frac{1}{y} \ \Longrightarrow \ \int y \ dy = \int 1 \ dx \ \Longrightarrow \ \frac{y^{2} }{2} = x + c
    \end{gather*}
    Dato che il quadrato non è sempre invertibile, devo stare attento alle soluzioni che
    ottengo, in questo caso però mi va bene qualsiasi funzione inversa ed entrambre sono soluzioni
    ma scelgo quella positiva
    \begin{gather*}
        y = \sqrt{2x + 2c} \ \Longrightarrow \ y = \sqrt{2x + 4} 
    \end{gather*}
\end{example}
\begin{example}[Altro Esempio]
    \begin{gather*}
        y' = ay(1 - by), a, b > 0
    \end{gather*}
    In questo caso ho delle soluzioni costanti che mi permettono di dire che
    $y(x) \equiv 0$, $y(x) \equiv \frac{1}{b}$, allora posso passare all'integrale
    separando le variabili e ottenere:
    \begin{gather*}
        \int \frac{dy}{y(1 - by)} = \int a \ dx \ \Longrightarrow \ \ln\left| \frac{y}{1 - by} \right| = ax + c \\
        \ \Longrightarrow \ \left| \frac{y}{1 - by} \right| = e^{ax} \cdot  e^{c}    = e^{ax}c_0, c_0 \in \mathbb{R}  
    \end{gather*}
    Dato che $c_0$ è una costante sempre positiva, posso sostituirla con $c_1 \in \mathbb{R} > 0$: 
    \begin{gather*}
        y = \frac{c_1e^{ax}  }{1 + bc_1e^{ax}  }
    \end{gather*}
\end{example}

\section{Equazioni omogenee}
Sono delle equazioni del tipo
\begin{gather*}
    y' = f(x, y)
\end{gather*}
dove $f(x, y)$ è omogenea di grado $0$, ossia $\forall \lambda \in \mathbb{R} \ f(\lambda x, \lambda y) = f(x, y)$.
\begin{example}[Questa funzione differenziale è omogenea]
    \begin{gather*}
        y' = \frac{2xy}{x^{2} + y^{2}  } \qquad f(\lambda x, \lambda y) = \frac{2(\lambda x)(\lambda y)}{(\lambda x)^{2} + (\lambda y)^{2}  } = f(x, y)
    \end{gather*}
\end{example}

Queste equazioni possono essere riscritte attraverso una opportuna funzione per essere risolte:
\begin{gather*}
    y' = b\left(\frac{y}{x}\right)
\end{gather*}
Ossia posso trasformare la differenziale come funzione di $\frac{y}{x} = t$, quindi cambio varibile
nel seguente modo:
\begin{gather*}
    z(x) = \frac{y(x)}{x} \ \Longrightarrow \ y(x) = xz(x) \ \Longrightarrow \ y'(x) = z + xz'
\end{gather*}
Adesso posso risolvere in funzione di $z$:
\begin{gather*}
    b(z) = z + xz' \ \Longrightarrow \ z' = \frac{b(z) - z}{x}
\end{gather*}
Coni passaggi all'indietro ottengo nuovamente $y(x) = xz(x)$.
\begin{example}
    \begin{gather*}
        y' = \frac{x^{3} + y^{3}  }{xy^{2} }
    \end{gather*}
    Se io definisco $f(x, y)$ questo oggetto è facile
    vedere che è omogenea. Posso ora mettere in evidenza la $x$ sopra
    e dunque ottenere
    \begin{gather*}
        \frac{x^{3}(1 + \frac{y^{3} }{x^{3} }) }{x^{3}(\frac{y^{2} }{x^{2} } ) } = b\left(\frac{y}{x}\right) 
    \end{gather*}
    Ossia si ottiene una funzione in rapporto tra $y$ e $x$, allora posso dire che
    \begin{gather*}
        b(t) = \frac{1 + t^{3} }{t^{2} }
    \end{gather*}
    Allora, data la definizione di $z'$ si ottiene
    \begin{gather*}
        z' = \frac{\frac{1 + z^{3} }{z^{2} } - z}{x} \ \Longrightarrow \ z' = \frac{1}{z^{2} }\frac{1}{x}
    \end{gather*}
    Allora per separazione di variabili possiamo trovare la soluzione come
    \begin{gather*}
        \int z^{2} dz  = \int \frac{dx}{x}
    \end{gather*}
\end{example}

\section{Esistenza locale ed esistenza globale}
\begin{example}
    \begin{gather*}
        \left\{\begin{array}{l}
            y' = -2xy^{2}  \\
            y(0) = -1
        \end{array}\right.
    \end{gather*}
    Essendo questo un problema di Cauchy, possiamo allora risolverlo
    con i metodi che conosciamo; tuttavia si osserva anche che, essendo
    $f(x, y)$ continua, allora anche la differenziale è continua e quindi
    siamo nelle ipotesi del teorema di esistenza ed unicità. Esiste allora
    una sola soluzione per questa equazione e si trova con
    separazione di variabili
    \begin{gather*}
        \int \frac{dy}{y^{2} } = \int -2x \ dx \\
        -\frac{1}{y} = -x^{2} + c \ \Longrightarrow \  c = 1 
    \end{gather*}
    La soluzione è la seguente, con $ x \in \mathbb{R} - \{-1, 1\}$: 
    \begin{gather*}
        y = \frac{1}{x^{2} - 1}
    \end{gather*}
\end{example}
I teoremi di esistenza locale mi dicono che è possibile trovare 
delle soluzioni negli intorni di $x_0$ in quanto sono interessato solo
alle soluzioni vicine a questo intorno. Nei teoremi di esistenza globale invece
io considero l'intero insieme di esistenza della funzione trovando delle soluzioni globali. 
\begin{wrapfigure}{r}{0.4\textwidth}
    \centering
    \caption{Teorema di esistenza globale}
    \begin{tikzpicture}
        \draw[->](-1, 0) -- (4, 0) node[at end, below] {$x$};
        \draw[->](0, -1) -- (0, 3) node[at end, left] {$y$};
        \draw(1, -1) -- (1, 3);
        \draw(3, -1) -- (3, 3);
        \draw(-0.1, 1) -- (0.1, 1) node[at start, left] {$y_0$};
        \filldraw(1.6, 0) circle (1pt) node[anchor = north] {$x_0$};
        \filldraw(1, 0) circle (1pt) node[anchor = north east] {$a$};
        \filldraw(3, 0) circle(1pt) node[anchor = north west] {$b$};
        \filldraw(1.6, 1) circle(1pt);
    \end{tikzpicture}    
\end{wrapfigure}
\begin{theorem}[Teorema di esistenza globale]
    Consideriamo il problema di Cauchy del primo ordine
    \begin{gather*}
        \left\{\begin{array}{l}
            y' = f(x, y) \\
            y(x_0) = y_0
        \end{array}\right.
    \end{gather*}
    \begin{enumerate}
        \item $f(x, y)$ e $\frac{\partial f}{\partial y}(x, y)$ sono continue e definite
        $\forall x \in [a, b]$, $y_0 \in \mathbb{R}$;
        \item Esistano due valori positivi $h, k \in \mathbb{R}$ per cui risulta:
        \begin{gather*}
            |f(x, y)| \leq h + k|y| \qquad \forall x \in [a, b], y \in \mathbb{R}
        \end{gather*} 
    \end{enumerate}
    Allora la soluzione è unica e definita su tutto $[a, b]$.
\end{theorem}
\begin{proof}
    Non fatta in classe (si farà? boooh)
\end{proof}

\section{Sistemi di equazioni differenziali}
Un sistema di equazioni differenziali è un sistema i cui membri sono vettori
di funzioni:
\begin{gather*}
    \left\{\begin{array}{l}
        y'_1 = f_1(x, y_1, \dots, y_n) \\
        y'_2 = f_2(x, y_1, \dots, y_n) \\
        \vdots \\
        y'_n = f_n(x, y_1, \dots, y_n)
    \end{array}\right.
\end{gather*}
La soluzione del sistema è allora un vettore di funzioni (chiamata anche \textbf{funzione vettoriale}):
\begin{gather*}
    \begin{pmatrix} y_1(x) \\
    \vdots \\
    y_n(x) \end{pmatrix} \  x \in I, \ I \subset \mathbb{R} : \\
    \left\{\begin{array}{l}
        y'_1 = f_1(x, y_1(x), \dots, y_n(x)) \\
        \vdots  \\
        y'_n = f_n(x, y_1(x), \dots, y_n(x)) 
    \end{array}\right.
\end{gather*}
Allora posso chiamare il vettore
\begin{gather*}
    Y(x) = \begin{pmatrix} y_1(x) \\
    \vdots \\
    y_n(x) \end{pmatrix} \\
\end{gather*}
Tale che $Y'(x)$ contenga tutte le derivate, allora posso chiamare il
vettore funzione come $F(x, Y) = Y'$, allora posso dire che i sistemi di equazioni
differenziali sono tutti risolvibili allo stesso modo. Se si avesse invece delle condizioni
per tutte le derivate fino alla $n - 1$ esima, allora saremmo in presenza di un
sistema di equazioni differenziali di Cauchy. Questo vuol dire che la sua soluzione,
tenuto conto delle considerazioni fatte fino ad ora
\begin{gather*}
    \left\{\begin{array}{l}
        Y'(x) = F(x, Y) \\
        Y(x_0) = y_0
    \end{array}\right.
\end{gather*}

\begin{theorem}[Teorema 1]
    Sia $D \subset \mathbb{R}^{n + 1}$ e sia $F : D \to \mathbb{R}^{n}$,una
    funzione continua e sia $(x_0, y_0)$ un punto interno a $D$, allora 
    il problema di Cauchy (sistema di equazioni differenziali) ha almeno una soluzione definita in un intorno  
    di $x_0$.
\end{theorem}

\begin{theorem}[Teorema 2]
    Nelle ipotesi del teorema precedete supponiamo che
    \begin{gather*}
        \frac{\partial f_i}{\partial y_j} \qquad i, j \in \{1, \dots, n\} 
    \end{gather*}
    Siano continue in $D$ , allora il problema di Cauchy con sistema di equazioni
    differenziali ha una sola soluzione definita nell'intorno di $x_0$.
\end{theorem}

\begin{theorem}[Teorema 3]
    Supponiamo che $f_i(x, y_1, \dots, y_n)$ sia definita per ogni $i = 1,\dots, n$,
    $\forall x \in [a, b]$ e $\forall y_1, \dots, y_n \in \mathbb{R}$ si suppone anche che
    $f_i$ e la derivata parziale rispetto ad ogni $f_i$ rispetto a $y_j$ siano continue
    $\forall y \in [a, b], y_1, \dots, y_n \in \mathbb{R}$. Si suppone anche che $\exists h, k  >0$ tali per cui
    risulti 
    \begin{align}
        ||F(x, Y)|| \leq h + k ||Y||
    \end{align} 
    Con $||Y||$ si indica la norma e ogni soluzione dell'equazione $Y' = F(x , Y)$ 
    è definita su tutto l'intervallo $[a, b]$.
\end{theorem}
\begin{example}[Esempio fondamentale]
    In un sistema di equazioni differenziali di ordine $n$, tale per cui
    \begin{gather*}
        \underset{\heartsuit}{y_1(x) \equiv y(x) \qquad y_2(x) \equiv y'(x)} \qquad y'' = 3xy + 7y' (\star) \\
        \left\{\begin{array}{l}
            y_1' = y_2 \\
            y_2' = 3xy_1 + 7y_2 (\star \star)
        \end{array}\right. 
    \end{gather*}
    Se $y$ risolve $\star$ allora $y_1, y_2$ definite da $\heartsuit$ risolve $\star \star$ viceversa
    se $(y_1, y_2)$ risolvono $\star \star$ allora la funzione $y_1$ ha ordine $\star$.  
\end{example}


\section{Equazioni differenziali omogenee lineari del primo ordine}
\begin{definition}[Equazioni differenziali omogenee lineari]
    Si chiamano equazioni differenziali lineari equazioni del tipo
    \begin{align}
        y^{(n)}  + a(x)_{n - 1}y^{(n - 1)} + \dots + a(x)y = f(x)
    \end{align}
    Se $f(x) $ è zero, allora prendono il nome di \textbf{omogenee}, altrimenti
    \textbf{complete}. Queste equazioni si chiamano lineari perché hanno la 
    caratteristica che riprendono le funzioni lineari negli spazi vettoriali.
\end{definition}
\begin{gather*}
    y' = a(x) y = f(x)   
\end{gather*}
Posso utilizzare la risoluzione vista prima per cui posso eseguire un cambio di variabile
opportuno in modo tale da renderla omogenea
\begin{gather*}
    z' + a(x) z = 0
\end{gather*}
Se $z_1$ e $z_2$ sono soluzioni dell'omogenea, allora anche $\alpha z_1(x)+\beta z_2(x)$
sono soluzioni.  Posso dimostrarlo nel seguente modo:
\begin{gather*}
    (\alpha z_1(x)+\beta z_2(x))' + a(x)(\alpha z_1(x)+\beta z_2(x)) = 0 \\
    \alpha(z_1' + a(x)z_1) + \beta(z_2' + a(x) z_2) = 0
\end{gather*}
Dato che $z_1$ e $z_2$ sono soluzioni, allora è verificata.  \\
Posso ora verificare che esistono soluzioni per l'equazione
generica $z' + a(x)z = 0$. Sia $A(x)$ una primitiva finita senza le costanti 
di $a(x)$ e quindi $A'(x) = a(x)$. Posso eseguire allora la seguente trasformazione: 
\begin{gather*}
    e^{A(x)}z' + a(x)e^{A(x)}z = 0 = (e^{A(x)}z)'     \\
    e^{A(x)}z = c \qquad c \in \mathbb{R} \qquad z = ce^{-A(x)}  
\end{gather*}
Esiste un modo più veloce per poter trovare le soluzioni, questo è dato dalla seguente 
formulazione
\begin{gather*}
    y'e^{A(x)} + a(x)e^{A(x)}y = f(x)e^{A(x)}    \\
    (ye^{A(x)} )' = f(x) e^{A(x)} 
\end{gather*}
Se considero $F(x)$ come primitiva del prodotto a destra, allora
si ottiene che 
\begin{gather*}
    ye^{A(x)} = F(x) + c
\end{gather*}
\begin{align}
    y = ce^{-A(x)} + e^{-A(x)}F(x)   
\end{align}
Si può anche utilizzare un secondo metodo per risolvere le equazioni 
e si chiama \textbf{metodo della variazione delle costanti}.
Penso che $c$ ora sia una funzione e non più una costante, allora dato che
devo considerare una soluzione generale, allora devo considerare $c$ come
funzione. Posso ottenere la derivata
\begin{gather*}
    y' = C'e^{-A(x)} - Ce^{-A(x)}a  \\
    y' + a(x)y = f \ \Longrightarrow \ C' = f(x) e^{A(x)} 
\end{gather*}
Se $C$ ha questa espressione, ossia la primitiva di $C'$, allora 
$C = F(x) + d, d\in \mathbb{R}$, le soluzioni sono allora
\begin{gather*}
    y(x) = C(x)e^{-A(x)} \ \Longrightarrow \ y(x) = (F(x) + d)e^{-A(x)}  
\end{gather*}

\begin{example}[Prima formula]
    \begin{gather*}
        y' + \frac{y}{x}= -\sin x \qquad \left\{\begin{array}{l}
            a(x) = \frac{1}{x} \\
            f(x) = -\sin x\\
            A(x) = \int a(x) \ \Longrightarrow \ \ln |x|
        \end{array}\right. \\
        F(x) = \int f(x) e^{A} \ dx = \int -\sin x e^{A(x)}\ dx = \int -\sin x e^{\ln|x|} \ dx   = \left\{\begin{array}{l}
            x > 0 : \sin x - x\cos x \\
            x < 0 : -\sin x + x\cos x 
        \end{array}\right. 
    \end{gather*}
    Allora si ottiene dalla formula di risoluzione veloce:
    \begin{gather*}
        y(x) = \left\{\begin{array}{l}
            Ce^{-\ln|x|} + e^{-\ln|x|} (\sin x - x\cos x) \qquad x > 0 \\
            Ce^{-\ln|x|} + e^{-\ln|x|} (-\sin x + x\cos x) \qquad x < 0  
        \end{array}\right.
    \end{gather*}
    Si ottengono allora due due soluzioni distinte a seconda di come si
    svolge il valore assoluto.
\end{example}

\section{Spazi di funzioni}
\begin{definition}[Spazi di funzioni]
    Sia $I$ un intervallo su di $\mathbb{R}$ , allora definisco lo spazio delle
    funzioni di da $I \to \mathbb{R}$ lo spazio vettoriale che
    contiene le funzioni con le seguenti proprietà:
    \begin{enumerate}
        \item $\alpha f(x)  + \beta g(x) $ appartiene allo spazio;
        \item $f(x)  \equiv 0$.
    \end{enumerate}
    All'interno di questo spazio vivono anche gli spazi:
    \begin{itemize}
        \item     $C^{0}(I)$, ossia l'insieme dello spazio delle funzioni continue su $I \to \mathbb{R}$;
        \item $C^{1}(I)$ è l'insieme delle funzioni continue derivabili in ogni punto
        la cui derivata è una funzione continua e $\forall x \in I$.
        \item $\vdots$
        \item $C^{(n)}$ è l'insieme delle derivate $n$ esime continue.    
    \end{itemize}
\end{definition}
Si possono ora risolvere le equazioni differenziali omogenee lineari
di ordine $n$-esimo considerando la generica funzione
\begin{gather*}
    y = C^{(n)}(I) 
\end{gather*}
Posso definire l'operatore $E(y) : C^{(n)}(I) \to C^{(0)}(I)  $ come 
l'operatore che trasforma le derivate $n$-esime continue in primitive 
e posso esprimerla (se le soluzioni sono $y_1$ e $y_2$)
\begin{gather*}
    E(\alpha y_1 + \beta y_2) = \alpha E(y_1) + \beta E(y_2)
\end{gather*}
\begin{theorem}
    \begin{enumerate}
        \item L'insieme $V_0$ delle soluzioni delle equazioni differenziali omogenee lineari è uno spazio
        vettoriale di dimensione $n$;
        \item L'insieme delle soluzioni delle equazioni differenziali lineari è 
        l'insieme $\{y(x) + y_f(x) : y \in V_0\}$ con $y_f$ soluzione delle equazioni differenziali lineari.
    \end{enumerate}
\end{theorem}




\appendix
\chapter{Risoluzione degli esercizi}
\section{Equazioni differenziali di primo ordine}
\subsection{Immediate}
Le equazioni differenziali immediate sono quelle della tipologia
\begin{gather*}
    y'(x) = f(x)  \ \Longrightarrow \ y \int f(x) \ dx
\end{gather*}

\subsection{Variabili separate}
Le equazioni differenziali a variabili separate sono quelle che possono essere risolte 
separando $y$ da una parte e $x$ dall'altra e integrando:
\begin{gather*}
    y' = a(x) b(y) \ \Longrightarrow \ \int\frac{dy}{b(y)} = \int a(x) \ dx 
\end{gather*}

\subsection{Lineari}
Le EDO lineari sono del tipo 
\begin{gather*}
    y' + a(x) y = f(x) 
\end{gather*}
Si risolvono nel seguente modo, posto $A(x) = \int a(x)  \ dx$: 
\begin{gather*}
    e^{A(x)}y' + a(x)e^{A(x)}y = e^{A(x)}f(x)  \\
     \left(ye^{A(x)}\right)' = e^{A(x)}f(x) 
\end{gather*}
Si integra adesso da entrambe le parti 
\begin{gather*}
    y(x) = e^{-A(x)}\int e^{A(x)}f(x)  \ dx  
\end{gather*}

\subsection{Sostituzione}
Se non fosse immediata, o presentasse dei termini $y^{n}$, si possono comunque 
risolvere secondo i metodi illustrati semplicemente operando un opportuna sostituzione:
\begin{gather*}
    t = yf(x) 
\end{gather*}
Dunque 
\begin{gather*}
    y' = \left(tf(x) ^{-1}\right)' 
\end{gather*}
Si risolve ora la differenziale in funzione di $t$ e si sostituisce una 
volta trovata la soluzione per $t(x)$ secondo i metodi visti prima. 

\section{Equazioni differenziali di secondo ordine}
\subsection{Soluzione generale di EDO II ordine}
Un'equazione del tipo 
\begin{gather*}
    ay''(x) + by'(x) + cy(x) = 0 
\end{gather*}
Si risolve col polinomio caratteristico:
\begin{gather*}
    a\lambda^{2} + b\lambda + c
\end{gather*}
Adesso si hanno tre casi distinti
\begin{gather*}
    \begin{array}{l}
        \text{Tipo di soluzioni} \\
        \text{2 sol reali distinte} \\
        \text{2 sol reali coincidenti} \\
        \text{2 soluzioni complesse} 
    \end{array}  \ \Longrightarrow \ 
    \begin{array}{l}
        \text{Soluzioni} \\
        y(x) = c_1 e^{\lambda_1 x} + c_2e^{\lambda_2 x} \\
        y(x) = c_1 e^{\lambda_1 x} + c_2xe^{\lambda_2 x} \\
        y(x) = c_1e^{\alpha x}\cos(\beta x) + c_2e^{\alpha x}\sin(\beta x) 
    \end{array} \ \Longrightarrow \ 
    \begin{array}{l}
        \text{Base} \\
        \left< e^{\lambda_1 x}, e^{\lambda_2x} \right> \\
        \left< e^{\lambda_1 x}, xe^{\lambda_2 x} \right> \\
        \left< e^{\alpha x}\cos(\beta x), e^{\alpha x}\sin(\beta x) \right> 
    \end{array}
\end{gather*}
Dunque le soluzioni generali sono la combinazione lineare della base 
costituita dalle soluzioni.


\subsection{Variabili costanti}
Un'equazione del tipo
\begin{gather*}
    ay''(x) + by'(x) + c y(x) = f(x) 
\end{gather*}
Si cerca la soluzione omogenea $y_0(x)$ con il metodo precedente, dunque si calcola
la soluzione particolare $y_p(x)$ della forma $c_1(x)y_1(x) + c_2y_2(x)$, in pratica la soluzione 
totale risulterà traslata rispetto all'omogenea associata. Si risolve il sistema
\begin{gather*}
    \left\{\begin{array}{l}
        c_1'(x) y_1(x) + c_2'(x) y_2(x) = 0 \\
        c_1'(x)y_1'(x) + c_2'(x)y_2'(x) = f(x) 
    \end{array}\right.
\end{gather*}
Si integra poi per trovare $c_1(x)$ e $c_2(x)$. Dunque si ottiene la soluzione generale come 
\begin{gather*}
    y(x) = y_0(x) + y_P(x)
\end{gather*}
Dove 
\begin{gather*}
    y_P(x) = c_1(x)y_1(x) + c_2(x)y_2(x)
\end{gather*}
Ossia la soluzione particolare trasla il sottospazio individuato dallo spazio
delle soluzioni $y_0(x)$. 

\subsection{Equazioni differenziali di secondo ordine per somiglianza}
Permettono di trovare la soluzione generale alle equazioni differenziali di 
secondo ordine. La soluzione si articola nella seguente maniera: data una 
equazione del tipo
\begin{gather*}
    ay''(x) + by'(x) + cy(x) = P(x) \vee e^{\lambda x} \vee \sin(\mathcal{B}x) \vee \cos(\mathcal{B}x) \vee \sin^{m}(\mathcal{B}x) \cdot \cos^{n}(\mathcal{B}x)
\end{gather*}
Si risolve l'omogenea associata secondo il metodo generale. Una volta trovata la soluzione generale, 
si considerano $P_1(x)$ e $P_2(x)$ due polinomi generici dello stesso grado di $P(x)$ 
con coefficienti da determinare (nel caso in cui a destra ci sia un polinomio). 
\begin{gather*}
    \left\{\begin{array}{l}
        \lambda_{1, 2} \neq \Lambda \wedge  \beta = 0 \ \Longrightarrow \ y_P(x) = \overline{P}(x) \vee e^{\Lambda x} \\
        \text{soluzioni distinte} \ \Longrightarrow \ y_P(x) = x\overline{P}(x) \wedge  xe^{\Lambda x} \\
        \text{Soluzioni coincidenti} \ \Longrightarrow \ \lambda_1 = \lambda_2  = \Lambda \wedge  \beta = 0 \ \Longrightarrow \ y_P(x) = x^{2} \overline{P}(x) \vee x^{2}e^{\Lambda x} \\
        \alpha \neq \Lambda \vee \beta \neq \mathcal{B} \ \Longrightarrow \ y_P(x) = e^{\Lambda x} \vee \left(\overline{P_1}(x) \cos(\mathcal{B}x) + \overline{P_2}(x) \sin(\mathcal{B}x) \right)  \\
        \alpha = \Lambda \wedge \beta = \mathcal{B} \ \Longrightarrow \ y_P(x) = xe^{\Lambda x} \vee \left(\overline{P_1}(x) \cos(\mathcal{B}x) + \overline{P_2}(x) \sin(\mathcal{B}x) \right)
    \end{array}\right.
\end{gather*}
Per determinare i polinomi $\overline{P_1}(x)$ e $\overline{P_2}(x)$ si determinano calcolando 
$y'_P(x)$ e $y_P''(x)$ e si sostituiscono nell'equazione differenziale di 
partenza. Si raccoglie i termini nei coseni e nei seni e si uguagliano ai coefficienti 
del polinomio che moltiplica il seno o il coseno.   


\end{document}
\documentclass[a4paper, oneside]{book}
\usepackage{graphicx}
\usepackage{amsthm}
\usepackage{amsmath}
\usepackage{amssymb}
\usepackage[a4paper,
            bindingoffset=0.2in,
            left=2cm,
            right=2cm,
            top=2cm,
            bottom=2cm,
            footskip=.25in]{geometry}
\usepackage[italian]{babel}
\usepackage{pgfplots}
\usepackage{tabularx}
\usepackage{tikz}
\usepackage{wrapfig}
\usepackage{color}
\usepackage[d]{esvect}
\usepackage{svg}
\usepackage{chemfig}
\usepackage{mhchem}
\definecolor{page}{rgb}{0.129,0.157,0.212}
\pagecolor{page}
\color{white}
\graphicspath{ {./images/} }
\usetikzlibrary{shapes.geometric}
\usetikzlibrary{datavisualization}
\usetikzlibrary{patterns}
\usetikzlibrary{datavisualization.formats.functions}
\usetikzlibrary{external}
\tikzexternalize[prefix=tikz/]
\pgfplotsset{width=10cm,compat=1.9}

\title{Appunti di Astronomia}
\author{Tommaso Miliani}
\date{2025/2026}

\begin{document}
\newtheoremstyle{theoremEnv}
                {}          % Space above
                {}          % Space below
                {\slshape}  % Body font
                {}          % Indent amount
                {\bfseries} % Head font
                {.}         % Punctuation after head
                {\newline}  % Space after theorem head
                {}          % Theorem head spec
\theoremstyle{theoremEnv}

\newtheorem{definition}{Definizione}[chapter]
\newtheorem{theorem}{Teorema}[chapter]
\newtheorem{lemma}{Proposizione}[chapter]
\newtheorem{observation}{Osservazione}[chapter]
\newtheorem{corollary}{Corollario}[theorem]
\newtheorem{example}{Esempio}[chapter]

% Useful definitions for Tikz
\def\proton(#1,#2){%
    \fill[color=red] (#1,#2) circle (5pt);
    \node at (#1,#2) {\texttt{+}};
}
\def\neutron(#1,#2){%
    \fill[color=gray] (#1,#2) circle (5pt);
}
\def\electron(#1, #2){%
    \fill[color=cyan] (#1, #2) circle (3pt);
    \node at (#1, #2) {\texttt{-}};
}
\def\orbit(#1, #2, #3){
    \draw (#1, #2) circle (#3);
}

\maketitle

\tableofcontents

\chapter*{Introduzione al corso}
Il corso si divide in tre sezioni:
\begin{enumerate}
    \item Astronomia sferica: ossia l'astronomia celeste, cenni storici,
    trigonometria sferica;
    \item L'indagine astronomica: ossia l'utilizzo di strumenti come telescopi
    per compiere misurazioni ed osservazioni sulle stelle; la determinazione dell'età
    delle stelle e l'osservazione stellare in generale;
    \item Gravitazione e meccanica celeste: campo e potenziale celeste con
    legge di Gauss, problema dei due e tre corpi, fluidi auto gravitanti (fluidi + gravità),
    struttura stellare.
\end{enumerate}
L'esame consiste in un colloquio orale diviso in tre domande,
una per parte del corso.


\part{Introduzione ai concetti dell'astronomia}
\chapter{Sfera celeste}
\section{Il cielo come proiezione}
Gli astri delle costellazioni e gli altri corpi hanno tutti distanze diverse ma
a noi appaiono proiettati su di una superficie: dalle tre coordinate per individuare un astro ne bastano
quindi due. La Terra, ruotando sul proprio asse, fa si che ruoti il cielo intorno
e quindi le stelle visibili di notte: tutti i pianeti e le stelle ruotano attorno al polo
Nord celeste, che non è altro che il prolungamento del polo nord terrestre sulla sfera celeste.
Dato che si tratta di una superficie sferica, otteniamo le coordinate degli oggetti
attraverso gli angoli: servono allora due cerchi per determinare da dove considerare gli
angoli.\\
Tutti gli astri sorgono da Est e tramontano ad Ovest: tuttavia con la latitudine può cambiare
l'osservazione dei corpi celesti: in particolar modo cambia l'altezza del Sole. Si può tuttavia utilizzare come riferimento
la stella Polare in quanto il polo Nord celeste è molto vicino a quella stella; si utilizza anche
la proiezione dell'equatore sulla sfera celeste chiamato \textbf{equatore celeste}. \\
Si hanno dunque due tipi di sistemi di coordinate astronomici: sistemi locali e assoluti.
Per passare ad un sistema di coordinate assoluti io utilizzo l'equatore celeste o l'eclittica in quanto non 
dipende dalla latitudine; si utilizza anche il sistema delle stelle fisse che ci permette
di ottenere le coordinate di un astro rispetto ad una stella fissa nel cielo (ossia una stella il cui moto proprio
è veramente molto piccolo). 

\section{Sistemi di coordinate}
\subsubsection{Sistema orizzontale}
Il sistema di coordinate più immediato e quello più naturale è il sistema di coordinate locale
che ci consente di ottenere le coordinate di un oggetto attraverso due coordinate:
\begin{itemize}
    \item Azimut: coordinate rispetto all'asse Sud-Nord: ossia la differenza
    angolare rispetto al Nord ($0 \sim 360$°)
    \item Altezza: coordinata rispetto all'orizzonte ($-90 \sim 90$°).
\end{itemize}

\subsubsection{Sistema equatoriale relativo}
Si utilizza l'inclinazione dell'asse terrestre costante rispetto alle
stelle fisse, utilizzando il meridiano locale. E' un sistema di coordinate
solidale con la Terra e ci permette di definire due coordinate (è ancora locale perché dipende da dove
sono sulla Terra):
\begin{itemize}
    \item Declinazione $\delta$: Ossia l'angolo tra l'oggetto e l'equatore celeste che diventa il
    cerchio fondamentale della sfera celeste;
    \item Angolo orario $h$: si misura in ore e ci indica proprio lo spostamento rispetto al
     meridiano  locale celeste (ottenuto con il prolungamento del meridiano sulla sfera celeste) .
\end{itemize}

\subsubsection{Sistema equatoriale assoluto}
Questo sistema di coordinate è definito a partire da due coordinate
ma a differenza di quello relativo, questo è fisso rispetto alle stelle fisse e
permette di avere le stesse coordinate per un oggetto a prescindere
dalla posizione sulla superficie terrestre.
\begin{itemize}
    \item Declinazione $\delta$: Ossia la differenza angolare rispetto al cerchio fondamentale
    (ossia l'equatore celeste);
    \item Ascensione retta $\alpha$: ossia la declinazione angolare rispetto al meridiano di Greenwich 
    sulla sfera celeste. 
\end{itemize}
Si può ottenere il \textbf{tempo siderale locale}: ossia il il tempo siderale è il tempo che
impiega la Terra a compiere un giro completo rispetto alle stelle (che è circa 3 min e 56 secondi più veloce
del giorno solare a cui siamo abituati).
\begin{align}
    \Theta = h_{\gamma} = h + \alpha
\end{align}
Con questo, possiamo ottenere le coordinate equatoriali assolute
conoscendo il nostro angolo orario e viceversa. 

\subsubsection{Sistema eclitticale}
In questo sistema di coordinate il cerchio fondamentale è l'Eclittica, ossia
il piano dell'orbita terrestre ed il cerchio di riferimento è il cerchio 
meridiano passante per i poli e i punti equinoziali. Come coordinate si utilizzano
le seguenti:
\begin{itemize}
  \item \textbf{Latitudine eclittica} $\beta$: ossia l'angolo rispetto all'eclittica;
  \item \textbf{Longitudine eclittica} $\gamma$: ossia la longitudine rispetto al punto di contatto tra l'equatore
  celeste e l'eclittica (ossia rispetto all'equinozio di primavera). 
\end{itemize}

\subsubsection{Il sistema galattico}
Per studiare corpi nella galassia si utilizzano le coordinate locali rispetto 
al Sole e come piano fondamentale quello galattico e il piano rispetto al quale
determino la longitudine galattica utilizzo la congiungente Sole Centro della Galassia.


\chapter{Trigonometria sferica}
\section{Triangoli sferici e prime definizioni della trigonometria sferica}
\begin{wrapfigure}{c}{0.8\textwidth}
    \centering
    \includegraphics[width=0.8\textwidth]{grossa.png}
    \caption{I triangoli sferici}
\end{wrapfigure}
Un triangolo sferico $ABC$ è formato da tre archi appartenenti a
tre cerchi massimi. Se $r$ è il raggio della sfera, allora si ha che 
l'arco $|AB| = r\theta$.  Possiamo trovare l'area del triangolo sferico 
che è proporzionale al suo eccesso sferico che è sempre diverso da zero
ed è diverso per ogni triangolo. L'\textbf{eccesso sferico} è la differenza tra
la somma dei tre angoli interni del triangolo e l'angolo piatto. Se le
ampiezze degli angoli sono $\alpha, \beta, \gamma$, allora l'eccesso sferico sarà
\begin{align}
    E = \alpha + \beta + \gamma - \pi
\end{align}
L'area di un triangolo sferico è allora
\begin{align}
    A = Er^{2} 
\end{align}

\section{Coordinate sferiche e trasformazioni}
\begin{wrapfigure}{r}{0.32\textwidth}
    \centering
    \caption{Le coordinate sferiche}
    \includegraphics[width=0.32\textwidth]{coordinate sferiche.png}
\end{wrapfigure}
Se ruotiamo due sistemi cartesiani di un certo angolo $\chi$, mantenendo invariata
la $x$ in modo tale che $x = x'$.  
\begin{gather*}
    \left\{\begin{array}{l}
        x' = x \\
        y' = y\cos \chi + z\sin\chi \\
        z' = -y\sin\chi + z\cos\chi
    \end{array}\right.
\end{gather*}
Possiamo allora passare alle coordinate sferiche nel seguente modo:
\begin{align*}
    x &= \cos\psi\cos\theta  &x' &= \cos\psi'\cos\theta' \\
    y &= \sin\psi\cos\theta  &y' &= \sin\psi'\cos\theta' \\
    z &= \sin\theta &z' &= \sin\theta'
\end{align*}
A questo punto si possono ottenere le seguenti relazioni risolvendo utilizzando le
relazioni che si sono viste fino ad ora per i sistemi di coordinate sferici:

\begin{align*}
    \cos\phi'\cos\theta &= \cos\psi\cos\theta \\
    \sin\psi'\cos\phi' &= \sin\psi\cos\theta\cos\chi + \sin\theta\sin\chi \\
    \sin\theta' &= -\sin\psi\cos\theta\sin\chi + \sin\theta\cos\psi
\end{align*}


\begin{wrapfigure}{r}{0.35\textwidth}
    \centering
    \caption{Applicazioni al triangolo sferico}
    \includegraphics[width=0.35\textwidth]{applicazioni al triangolo sferico.png}
\end{wrapfigure}
Si possono allora applicare le coordinate nelle espressioni
del triangolo sferico. Considerando il seguente sistema di riferimento 
nell'immagine, posso trovare le seguenti relazioni per gli angoli
$\psi, \theta, \chi$:
\begin{gather*}
    \psi = \alpha - 90\text{°} \qquad \theta = 90\text{°} - b  \qquad \chi = c \\
    \psi' = 90\text{°} - \beta \qquad \theta ' = 90 \text{°} - a
\end{gather*}
Si ottiene allora dalla trigonometria (nell'assunzione secondo la quale il raggio della sfera sia molto 
grande):
\begin{align*}
    \sin \beta \sin a &= \sin \alpha \sin b \\
    \cos \beta \sin a &= -\cos \alpha \sin b \cos c + \cos b \sin c \\
    \cos a &= \cos \alpha \sin b \sin c + \cos b \cos c 
\end{align*}
Da qui posso utilizzare la regola sei seni per ottenere le seguenti relazioni
\begin{gather*}
    \frac{\sin a}{\sin \alpha} = \frac{\sin b}{\sin B} = \frac{\sin c}{\sin \gamma}
\end{gather*}

\section{Il triangolo nautico : da equatoriale relativo a orizzontale}
\begin{wrapfigure}{r}{0.35\textwidth}
    \centering
    \caption{Il triangolo nautico}
    \includegraphics[width=0.35\textwidth]{Triangolo nautico.png}
\end{wrapfigure}
Il triangolo nautico serve per trasformare le coordinate
equatoriali relative nel sistema orizzontale. Oppure si possono utilizzare le
trasformazioni inclinando di un angolo negativo : 
qui $S$ equatoriale, $S$' è il sistema orizzontale. 
\begin{gather*}
    \psi = 90\text{°} - h  \qquad \theta = \delta \\
    \psi' = 90\text{°} - \alpha \qquad \theta' = a \\
    \chi = -(90\text{°} - \phi)
\end{gather*}
Impostati ora gli angoli , posso ricavare mediante il sistema di riferimento
delle coordinate sferiche mediante le relazioni:
\begin{align*}
    \sin\alpha\cos a &= \sin h \cos \delta \\
    \cos\alpha \cos a &= \cos h \cos \delta \sin \phi - \sin \delta \cos \phi \\
    \sin a &= \cos h \cos \delta \cos \phi + \sin \delta \sin \phi
\end{align*}
Possiamo osservare che la trasformazione da equatoriale relativo ad orizzontale
ci permette di ottenere le coordinate azimutali e di altezza (le quali per un osservatore
poco esperto sono concettualmente più facili da capire ed utilizzare) di un certo oggetto
se qualcuno ha trovato le coordinate equatoriali relative.


\section{Il tempo siderale: da equatoriale assoluto a relativo}
Per trasformare le coordinate equatoriali assolute, ossia la posizione
di un astro dal catalogo o dalle effemeridi. La posizione da catalogo indica la
posizione di uh astro secondo i cataloghi stellari ad una certa ora
e giorno dell'anno mentre la posizione di un astro dalle effemeridi è la posizione
di un astro tramite il giorno giuliano e il tempo siderale alla mezzanotte di
Greenwich. Nel sistema orizzontale serve il tempo siderale
(TS locale) ovvero l'angolo orario del punto $\gamma$:
\begin{align}
    \Theta = h_{\gamma} = h + \alpha
\end{align}
Puntando verso un astro di riferimento con un telescopio
dotato di montatura equatoriale posso leggere $h$ sul disco orario e ricavare
allora $TS$ in quell'istante. Altrimenti servirebbe un orologio specifico più rapido di 
circa 3min e 56 secondi rispetto a quelli standard. 

\section {Culminazione, levata e tramonto degli astri}
Ci poniamo al meridiano di Greenwich (per cui $h = \alpha = 0$). L'altezza massima 
raggiunta da un astro si troverà come
\begin{align}
    a_{max} = 90\text{°} - \phi + \delta \ \Longrightarrow \ z_{min} = \phi - \delta
\end{align}
L'altezza massima è anche chiamate culminazione superiore o transito di un astro; un
astro risulta \textbf{circumpolare} (ossia rimane sempre sopra l'orizzonte) per
una data latitudine se
\begin{gather*}
    \delta > 90\text{°} - \phi
\end{gather*}
Sempre nella terza relazione possiamo porre $a = 0$ l'angolo orario
di levata e tramonto di un astro si ricava da:
\begin{align}
    \cos h = -\tan\delta \tan\phi
\end{align}

\section{Variazioni di coordinate}
\begin{wrapfigure}{r}{0.3\textwidth}
    \centering
    \caption{Precessione}
    \includegraphics[width=0.3\textwidth]{Screenshot 2025-09-22 195611.png}
\end{wrapfigure}
Esistono molti motivi per cui le stelle
 possono cambiare nel tempo le loro
 coordinate. Il primo è il loro moto
 proprio, dato che su tempi lunghi non
 esistono vere stelle fisse. Scomponendo
 la velocità in tangenziale e radiale, solo
 la velocità tangenziale porta a variazioni
 di coordinate celesti, mentre la componente
 radiale è misurabile tramite effetto
 Doppler (più preciso). Il satellite Gaia
 sta mappando un miliardo di stelle con
 precisione di 20 micro secondi d'arco
 (distanze e moti propri). \\
 L'asse terrestre in realtà non è fisso: a
 causa della non-sfericità della Terra si ha
 un moto di precessione con periodo di 26
 mila anni. Il punto d'Ariete si sposta in
 modo retrogrado sull'eclittica, con
 incremento delle longitudini galattiche di
 50''/anno (precessione degli equinozi).
 Differenziando le trasformazioni tra i due
 sistemi eclitticale ed equatoriale assoluto:
 \begin{gather*}
     d\delta = d\gamma \sin\epsilon\cos\alpha \\
     d\alpha = d\gamma(\sin\epsilon\sin\alpha\tan\delta + \cos\epsilon)
 \end{gather*}
 Ulteriori correzioni che coinvolgono pure
 l'angolo di inclinazione sono dovute alla
 Luna (nutazione, periodo di 18.6 anni), di
 entità minore e spesso trascurate.
 Il moto di rivoluzione terrestre
 provoca l'aberrazione (velocità
 finita della luce, max 21'', solo
 0.3'' per la rotazione) e per stelle
 vicine la parallasse annua (se la
 distanza è un parsec vale 1'').
 C'è poi la rifrazione (per $a\lessapprox10$°).


\chapter{Misura del tempo}
\section{Gnomoni meridiane e orologi solari}
Il moto diurno del Sole è lo strumento più ovvio per poter misurare il tempo.
Si utilizzava ai tempi lo \textbf{gnomone}, ossia la punta di un palo che proietta la sua 
ombra su di una \textbf{meridiana}, la quale, al mezzogiorno, individua esattamente
l'asse Nord-Sud; le ombre che proietta sono diverse per ogni stagione. Si usa invece
l'orologio solare che è una meridiana che presenta anche le ore sopra di essa. \\
Il giorno è definito come due passaggi successivi in meridiana da parte del Sole; tuttavia
via mentre la terra gira la Terra sta anche girando intorno al Sole: essendo il giorno definito come
un moto di 24h, il giorno siderale è sempre minore del giorno solare. Se definisco
$P$ il tempo assoluto di un anno di $P = 365.2564$ e il giorno normale come $\tau = 1$ e
$\tau_s$ la durata del giorno siderale rispetto a quello normale, ottengo il numero di giorni
in più come
\begin{gather*}
    \frac{P}{\tau_s} - \frac{P}{\tau} = I
\end{gather*}
Nell'uso civile ci si rifà alle 24h anche se il giorno in realtà ha
durata variabile durante i mesi dell'anno. Si può trovare la durata del giorno effettivo
come la differenza tra l'ora solare vera e l'ora solare media:
\begin{gather*}
    E.T. = T - T_M
\end{gather*}
Se ci cercasse di scattare delle foto del Sole alla stessa ora solare media della culminazione del
Sole con una macchina fissa, sovrapponendo le immagini di tutto l'anno si ottiene una visualizzazione
delle differenze della durata del giorno ed il vero moto del Sole nel cielo chiamato \textbf{Analemma solare}.

\section{Tempo universale e fusi orari}
Dato che il Sole non può sorgere in tutti i punti della Terra
allo stesso momento, si definisce un tempo universale
rispetto al meridiano di Greenwich e si utilizzano i fusi orari
(essenzialmente degli sfasamenti rispetto all'orario standard) 
per ogni zona della terra. Esistono ben 24 fusi orari e ogni paese decide quale 
utilizzare per determinare l'ora. \\
Essendo che la rotazione della Terra non è uniforme, non posso definire il secondo come
$1/86400$ del giorno solare, come era definito una volta, ma si è passati a
definire il secondo secondo degli orologi atomici che correggono gli
effetti della relatività generale basandosi sulla variazione energetica degli
elettroni nell'atomo di Cesio-133. Anche la definizione di anno non è universale in quanto
nel corso della storia si sono susseguiti diversi calendari, ognuno ottenuto attraverso
metodi di calcolo dei giorni diversi tra di loro. Giulio Cesare impose che
l'anno sarebbe durato $365.25$ giorni, ma rispetto al Calendario Gregoriano si è arrivati
ad uno sfasamento di 10 giorni nel 1582; per ovviare al problema, i bisestili non erano stati
più contati ogni 4 anni ma in modo tale da avere 97 bisestili ogni 400
anni invece che 100.
Si può stimare allora il tempo siderale conoscendo i giorni che mancano all'equinozio di primavera
e aggiungendo dodici ore in quanto si misurano da SUD
\begin{align}
    \Theta \approx T + 12h + n \cdot 4min
\end{align}

\part{Osservazione e studio della sfera celeste}
\chapter{I telescopi}
\section{I parametri fondamentali dei telescopi}
Per poter compiere osservazioni sulla radiazione
luminosa proveniente dagli oggetti celesti si fa utilizzo di uno strumento
ottico che prende il nome di \textbf{telescopio}. Esistono diverse tipologie di
telescopi che vengono utilizzati 
ognuno nel loro campo di applicazione.  Galileo Galilei fu il primo ad
utilizzare e realizzare telescopi per l'osservazione della sfera celeste. Qualunque
sia la tipologia di telescopio, ognuno di essi presenta i seguenti parametri:
\begin{itemize}
    \item \textbf{Apertura} ($D$): ossia il diametro dell'obbiettivo (anche chiamato 
    specchio primario nei riflettori), che determina il potere di raccolta di luce (proporzionale a $D^{2}$).
    \item \textbf{Lunghezza focale} ($f$): la distanza tra l'obbiettivo ed il piano focale
    che definisce, insieme all'oculare, l'ingrandimento del telescopio.
    \item \textbf{Rapporto focale} ($\frac{f}{\#}= \frac{f}{D}$): È una quantità adimensionale che descrive 
    la velocità o luminosità del sistema: un telescopio "veloce" ha $\frac{f}{D}$ basso ($\frac{f}{1}$  fino a $\frac{f}{6}$), 
    apre a campi molto ampi (con tempi di esposizione veloci). Un telescopio "lento" ha 
    un $\frac{f}{D}$ alto, ossia $\frac{f}{10}$ fino a $\frac{f}{20}$, fornendo maggiori ingrandimenti 
    (e dunque aprendo a campi più stretti), ma con tempi di esposizione lunghi. Il significato di $\frac{f}{\#}$ vuol dire che
    la lunghezza focale è $\#$ volte l'apertura. Il reciproco di $\frac{f}{D}$ è anche chiamato f number ($F$). 
    \item \textbf{Tempo di esposizione}: Non è propriamente una caratteristica
    dei telescopi; il tempo di esposizione è il tempo per cui si mantiene aperto l'obbiettivo
    sulla sorgente in modo da poter raccogliere quanta più luce possibile
    per avere immagini più luminose.
    \item \textbf{Oculare}: E' una lente che sta dopo il punto focale della lente primaria
    e ha la focale sull'occhio dell'osservatore. Oculari diversi possiedono ingrandimenti diversi. 
    \item \textbf{Campo di vista} (FOV): porzione di cielo che riesce a vedere il telescopio.
    È inversamente proporzionale alla lunghezza focale a
    parità di oculare. Spesso è il rilevatore al piano focale a determinare il FOV finale. 
    \item \textbf{Ingrandimento} (M): è dato dal rapporto tra $f$ e $f_{\text{oculare}}$, l'ultima dipende dall'oculare che si sceglie. 
    E' logico quindi pensare che più è "lungo" l'oculare e più ingrandimenti ottengo.
    \item \textbf{Scala spaziale}: Lega le dimensioni angolari a misure spaziali al piano focale. La \textbf{Plate scale} è data come
    segue:
    \begin{gather*}
        Ps [arcsec/mm] = \frac{206265}{f} \ (mm)
    \end{gather*}
    Infatti, quando si osserva il cielo, si osserva una proiezione di una porzione
    del cielo sul piano dell'immagine. Questa grandezza lega le dimensioni degli angoli reali a quella
    del piano dell'immagine e si ricava nella seguente maniera: 
    \begin{gather*}
        \tan \theta = \frac{D}{f} \ \Longrightarrow \ \theta \approx \frac{D}{f}
    \end{gather*} 
    Si può convertire questa a arcosecondi definiti come
    $1 \ rad = 180 \cdot  3600 \ arcsec$. Si può dunque esprimere l'angolo in radiante e quindi il plate scale come
    \begin{gather*}
        \theta \approx \frac{D}{f} \cdot  180 \cdot  3600 \ \Longrightarrow \ \theta = \frac{180\cdot 3600}{F}
    \end{gather*}
    \item \textbf{Potere risolutivo}: E' l'angolo minimo oltre al quale il telescopio non riesce a distinguere 
    tra due sorgenti puntiformi ad una data lunghezza d'onda. Questo si può ricavare dal limite di diffrazione
    fatto nel corso di ottica, secondo l'equazione di Rayleigh, legata all'apertura del telescopio:
    \begin{gather*}
        \theta_r \approx 1.22 \  \frac{\lambda}{D}\text{°} \quad \quad \theta_r \approx 2.5 \cdot  10^{5} \frac{\lambda}{D} '' 
    \end{gather*}
\end{itemize}

\section{Telescopi rifrattori}
\subsection{Il telescopio Galileiano}
Il telescopio Galileiano presenta una lente primaria convergente ed un oculare con lente
divergente in modo tale che l'oculare viene a trovarsi prima del fuoco dell'obiettivo,
ad una distanza da questo pari alla distanza focale dell'oculare. Questo telescopio produce immagini dritte 
ma ha un campo molto ristretto ($\frac{f}{10}$ e $\frac{f}{15}$):
\begin{gather*}
    \begin{tikzpicture}
        \draw(0, 0) -- (7, 0);
        \draw[<->](2, -1.5) -- (2, 1.5);
        \draw[>-<](5, -1) -- (5, 1);
        \draw[cyan](0.1, 1.7) arc (160:180:0.5) node[midway, left] {$\alpha$};
        \draw(0, 1.5) -- (2, 1.5) -- (5, 0.75) -- (6, 0.75);
        \draw[dashed](6, 0.75) -- (6.5, 0.75);
        \draw(0, -1.5) -- (2, -1.5) -- (5, -0.75) -- (6, -0.75);
        \draw[dashed](6, -0.75) -- (6.5, -0.75);
        \draw[cyan](0, 1.75) -- (2, 1.5) -- (5, -0.25) -- (6, -0.65);
        \draw[cyan](0, 0.25) -- (2, 0) -- (5, -0.65) -- (6, -1);
        \draw[cyan](0, -1.25) -- (2, -1.5) -- (5, -1) -- (6, -1.35);
    \end{tikzpicture}
\end{gather*}



\subsection{Il telescopio Kepleriano}
\begin{wrapfigure}{r}{0.4\textwidth}
    \centering
    \begin{tikzpicture}[scale=1.2]
        \draw(0, 0) -- (5, 0);
        \draw[<->](2, -1) -- (2, 1);
        \draw[<->](4, -1) -- (4, 1);
        \draw[red](0, 0.75) -- (2, 0.75) -- (4, -0.25) -- (4.75, -0.25);
        \draw[cyan](0, 0.75) -- (2, 0) -- (3.5, -0.75) -- (4, -0.75) -- (5, 0.75);
        \draw[cyan](0, 1.5) -- (2, 0.75) -- (3.5, -0.75) -- (4.5, 0.75);
        \draw[very thick, cyan](3.5, 0) -- (3.5, -0.75) node[midway, left] {$h$};
        \draw[cyan](0.5, 0.55) arc (135:180:0.75) node[midway, left] {$\alpha$}; 
        \draw[cyan](0.5, 1.3) arc (135:180:0.75) node[midway, left] {$\alpha$}; 
        \filldraw(4.5, 0) circle (1pt) node[anchor = north west] {$F_2$};
        \filldraw(3.5, 0) circle (1pt)node[anchor = south] {$F_1$};
        \draw[cyan](4.75, 0.38) arc (62:0:0.4) node[midway, right] {$\beta$};
        \draw[cyan](4.25, 0.38) arc (62:0:0.4) node[midway, right] {$\beta$};
        \draw[very thick] (2, 0) -- (4, 0);
        \draw[very thick, cyan](2, 0) -- (3.5, -0.75);
        \draw[very thick, cyan](3.5, -0.75) -- (4, 0);
    \end{tikzpicture}    
\end{wrapfigure}
Il telescopio Kepleriano è il tipo di telescopio più semplice da realizzare. È un telescopio 
costituito da due lenti convergenti poste ad una distanza che è la somma delle loro lunghezze focali, in modo tale che
il fuoco primario cada all'interno del tubo e l'oculare intercetta i raggi convergenti.
Questa tipologia di telescopio presenta dunque un percorso ottico più lungo di quello
Galileiano ma produce una immagine capovolta dell'oggetto osservato pur presentando un campo visivo
più ampio e luminoso ($\frac{f}{5}$ e $\frac{f}{15}$). Si vede come non sia possibile ridurre maggiormente lo spessore delle lenti
per poter ottenere $F$ number più grossi 
poiché quest'ultime diventerebbero estremamente fragili. Questo perché il vetro è un fluido estremamente viscoso. La lente più grande
mai realizzata è l'elemento L-1 progettato per la fotocamera del Vera C. Rubin Observatory con un diametro
di $1.57$ metri.

\section{Telescopi riflettori}
I telescopi riflettori utilizzano la riflessione invece della rifrazione per poter 
concentrare i raggi luminosi. La luce di un dato oggetto celeste colpisce uno specchio primario
concavo, viene riflessa e poi focalizzata verso un punto o verso un secondo specchio 
che ne devia il fascio verso l'oculare o il sensore.
\subsection{Il telescopio newtoniano}
Il telescopio newtoniano è il telescopio riflettore a configurazione più semplice. Il 
suo specchio primario è uno specchio paraboloidale mentre il suo specchio secondario è
uno specchio piano diagonale. In questo modo arriva un'immagine specchiata
e capovolta. Questo telescopio ha un $F$ number basso
di circa $\frac{f}{3} \sim \frac{f}{10} $. 
\begin{gather*}
    \begin{tikzpicture}
        \draw[dashed](0, 0) -- (7, 0);
        \draw[very thick](0, 1.25) -- (6, 1.25) -- (6, -1.25) -- (2, -1.25) -- (2, -2);
        \draw[very thick](0, -1.25) -- (1, -1.25) -- (1, -2);
        \draw(5.5, 1.1) -- (5.85, 1.1) -- (5.85, -1.1) -- (5.5, - 1.1);
        \draw(5.5, -1.1) .. controls (5.8, -0.55) and (5.8, 0.55) .. (5.5, 1.1);
        \draw[cyan](0, 1) -- (5.55, 1) -- (1.8, 0.3) -- (1.5, -2);
        \draw[cyan](0, -1) -- (5.55, -1) -- (1.2, -0.3) -- (1.5, -2);
        \draw[thick] (1.1, -0.4) -- (1.9, 0.4);
    \end{tikzpicture}
\end{gather*}


\subsection{Configurazioni moderne}
Tra le varie configurazioni moderne di telescopi riflettori si hanno 
\begin{itemize}
    \item Telescopio Cassegrain: classico telescopio con $F$ number alto, uno specchio
    primario parabolico e uno specchio secondario iperbolico, l'immagine è capovolta.
    \begin{gather*}
        \begin{tikzpicture}
            \draw[very thick](0, 1.25) -- (6, 1.25) -- (6, 0.25);
            \draw[very thick](6, -0.25) -- (6, -1.25) -- (0, -1.25);
            \draw(5.5, 1.1) -- (5.85, 1.1) -- (5.85, 0.25) -- (5.7, 0.25);
            \draw(5.5, -1.1) -- (5.85, -1.1) -- (5.85, -0.25) -- (5.7, -0.25);
            \draw(5.5, 1.1) .. controls (5.75, 0.7) and (5.67, 0.4) .. (5.7, 0.25);
            \draw(5.5, -1.1) .. controls (5.75, -0.7) and (5.67, -0.4) .. (5.7, -0.25);
            \draw[cyan](0, 1) -- (5.55, 1) -- (1.8, 0.3) -- (6.75, 0);
            \draw[cyan](0, -1) -- (5.55, -1) -- (1.8, -0.3) -- (6.75, 0);
            \draw[thick] (1.75, 0.4) arc (10:-10:2.5);
            \filldraw[align = center] (6.75, 0) circle (1pt) node[anchor = north west] {Fuoco del \\ Cassegrain};
            \draw[dashed] (1.8, 0.3) -- (0.6, 0) -- (1.8, -0.3);
            \filldraw[align = center](0.6, 0) circle (1pt) node[anchor = north] {Fuoco \\ Primario};
        \end{tikzpicture}
    \end{gather*}
    \item Ritchey-Chrétien: telescopio con $F$ number alto simile al Cassegrain ma con due specchi iperbolici.
    Permette la correzione di "coma" e dell'astigmatismo e produce un'
    immagine capovolta. 
    \begin{gather*}
        \begin{tikzpicture}
            \draw[very thick](0, 1.25) -- (6, 1.25) -- (6, 0.25);
            \draw[very thick](6, -0.25) -- (6, -1.25) -- (0, -1.25);
            \draw(5.5, 1.1) -- (5.85, 1.1) -- (5.85, 0.25) -- (5.6, 0.25);
            \draw(5.5, -1.1) -- (5.85, -1.1) -- (5.85, -0.25) -- (5.6, -0.25);
            \draw(5.5, 1.1) .. controls (5.6, 0.7) and (5.6, 0.4) .. (5.6, 0.25);
            \draw(5.5, -1.1) .. controls (5.6, -0.7) and (5.6, -0.4) .. (5.6, -0.25);
            \draw[cyan](0, 1) -- (5.55, 1) -- (1.8, 0.3) -- (6.75, 0);
            \draw[cyan](0, -1) -- (5.55, -1) -- (1.8, -0.3) -- (6.75, 0);
            \draw[thick] (1.75, 0.4) arc (10:-10:2.5);
            \filldraw[align = center] (6.75, 0) circle (1pt) node[anchor = north west] {Fuoco del \\ R-C};
            \draw[dashed] (1.8, 0.3) -- (0.6, 0) -- (1.8, -0.3);
            \filldraw[align = center](0.6, 0) circle (1pt) node[anchor = north] {Fuoco \\ Primario};
        \end{tikzpicture}
    \end{gather*}
    \item Gregoriano: presenta un tubo più lungo degli altri due tipi di telescopi con uno specchio primario 
    parabolico ed uno specchio secondario ellittico positivo. Dunque produce un immagine reale dritta.
    \begin{gather*}
        \begin{tikzpicture}
            \draw[very thick](0, 1.25) -- (6, 1.25) -- (6, 0.25);
            \draw[very thick](6, -0.25) -- (6, -1.25) -- (0, -1.25);
            \draw(5.5, 1.1) -- (5.85, 1.1) -- (5.85, 0.25) -- (5.7, 0.25);
            \draw(5.5, -1.1) -- (5.85, -1.1) -- (5.85, -0.25) -- (5.7, -0.25);
            \draw(5.5, 1.1) .. controls (5.75, 0.7) and (5.67, 0.4) .. (5.7, 0.25);
            \draw(5.5, -1.1) .. controls (5.75, -0.7) and (5.67, -0.4) .. (5.7, -0.25);
            \draw[cyan](0, 1) -- (5.55, 1) -- (1.8, 0.3) -- (6.75, 0);
            \draw[cyan](0, -1) -- (5.55, -1) -- (1.8, -0.3) -- (6.75, 0);
            \draw[thick] (1.75, 0.4) arc (170:190:2.5);
            \filldraw[align = center] (6.75, 0) circle (1pt) node[anchor = north west] {Fuoco del \\ Gregoriano};
            \draw[dashed] (1.8, 0.3) -- (0.6, 0) -- (1.8, -0.3);
            \filldraw[align = center](0.6, 0) circle (1pt) node[anchor = north] {Fuoco \\ Primario};
        \end{tikzpicture}
    \end{gather*}
\end{itemize}

\section{Montatura dei telescopi}
I telescopi si montano su delle basi che ne guidano il movimento ed il puntamento. Le montature
si dividono in due categorie principali:
\begin{itemize}
    \item Altazimutali: muovono il telescopio sull'asse verticale e asse orizzontale (secondo altitudine e azimut).
    Sono intuitive da usare per puntare oggetti luminosi ad occhio, ma richiedono delle correzioni simultanee
    su entrambi gli assi per seguire le stelle.
    \item Equatoriali: possiedono un asse parallelo all'asse terrestre chiamato
    \textbf{asse polare}. Un solo motore o sistema di inseguimento mantiene l'oggetto al centro del
    campo visivo, compensando la rotazione della Terra. 
\end{itemize}

\subsection{Montature Coudé e Nasmyth}
Si analizzano ora altre due tipologie di montature 
nel caso in cui si necessiti di focali più lunghe del telescopio stesso. Queste montature
sono utilizzate nei telescopi riflettori.
\paragraph{Montatura Coudé} \mbox{} \\
\noindent La montatura Coudé utilizza una serie di specchi terziari e quaternari che indirizzano la luce lungo l'asse di montatura
verso un piano focale esterno. In questo modo lo strumento rimane fermo quando il telescopio si muove. Questa
montatura permette di collocare strumenti pesanti in posizione fissa evitando dunque carichi sull'ottica mobile, 
come per esempio per spettrografi ad alta risoluzione, garantendo un'elevata stabilità termica 
e meccanica degli strumenti utili nella spettroscopia ad alta risoluzione. Ha anche il vantaggio di alimentare
in modo semplice sistemi di calibrazione e fibre ottiche verso strumenti remoti. Lo svantaggio di questa montatura è 
che per un telescopio a montatura già altazimutale si deve anche utilizzare un
\textbf{derotatore} per compensare la rotazione degli astri in quanto il piano dell'immagine ruota mentre si insegue l'astro. 
Inoltre, dato che si aggiungono più specchi, c'è una perdita considerevole di luce.

\paragraph{Montatura Nasmyth} \mbox{} \\ \noindent
La montatura Nasmyth utilizza uno specchio terziario che indirizza la luce orizzontalmente insieme all'asse verticale (altitudine) 
verso un lato del telescopio. Quindi la montatura Nasmyth si muove con il telescopio lungo l'asse azimutale ma il fascio di luce
rimane orizzontale rispetto al terreno. Come nella montatura Coudé, il campo visivo ruota mente si mette a fuoco un particolare oggetto,
che può essere compensato utilizzando un derotatore o montando lo strumento su di una piattaforma ruotante che ruota con il campo visivo.

\section{Aberrazioni}
Le aberrazioni sono delle deviazioni del sistema ottico dal comportamento previsto dal limite parassiale.
Queste sono un difetto di un sistema ottico (lente o specchio) che causa la deviazione dei raggi di luce
dalla loro traiettoria ideale, impedendo la formazione di una immagine perfetta e puntiforme di un oggetto puntiforme. 
Esse sono delle sfocature o distorsioni dell'immagine finale. Le aberrazioni sono una conseguenza intrinseca della natura ondulatoria e
del comportamento della luce (riflessione o rifrazione) quando interagisce con superfici non ideali. Esistono
due principali tipologie di aberrazioni:
\begin{itemize}
    \item Monocromatiche: sono aberrazioni che si verificano anche con singole lunghezza d'onda.
    \item Cromatiche: si verificano con luce composta da più lunghezze d'onda.
\end{itemize}

\subsection{Monocromatiche}
Sono dovute alla geometria del sistema ottico e al fatto che i raggi non parassiali vengono 
focalizzati in punti diversi. La teoria delle aberrazioni di Seidel le descrive matematicamente al terzo ordine, estendendo 
anche l'approssimazione parassiale (non in programma). Esistono diverse tipologie di aberrazione
\begin{itemize}
    \item \textbf{Aberrazione sferica}: i raggi paralleli all'asse ottico che passano 
    attraverso le zone periferiche di una lente sferica convergono in un punto focale diverso rispetto a quelli che passano vicino al centro. 
    \item \textbf{Coma}: Si manifesta per oggetti posti fuori dall'asse ottico. L'immagine di un
    oggetto puntiforme non è un punto vero e proprio ma forma una sorta di scia che si allunga radialmente causata
    dalle differenze di ingrandimento dei raggi che attraversano diverse zone della lente. 
    \item \textbf{Astigmatismo}: Anch'esso per oggetti fuori dall'asse ottico. L'immagine 
    di un punto si forma come se fossero due linee focali di cui una radiale (tangenziale) ed una circonferenziale (sagittale),
    a distanze diverse dalla lente. Questo porta a immagini che sono nitide solo in una direzione. 
    \item \textbf{Curvatura di campo}: Un sistema ottico con curvatura di campo non proietta un oggetto piano su di un immagine 
    piana ma su di una superficie curva. Di conseguenza, non è possibile mettere a fuoco per intero tutto l'oggetto su di un sensore piano.
    \item \textbf{Distorsione}: non altera la nitidezza ma è deforma geometricamente l'immagine. Le linee rette, specialmente quelle lontane dall'asse 
    ottico, appaino curve. Esistono due tipologie di distorsioni: a \textbf{barilotto} (con linee curve verso l'esterno) o a \textbf{cuscinetto} 
    (con linee curve verso l'interno). 
\end{itemize}

\subsection{Cromatiche}
Si verificano con luce composta da diverse lunghezze d'onda (luce bianca) a causa della \textbf{dispersione}, ossia la dipendenza 
dell'indice di rifrazione di un materiale dalla lunghezza d'onda. 

\subsection{Aberrazioni nei rifrattori}
Nel telescopio Galileiano non è presente nessuna aberrazione cromatica aggiuntiva all'oculare ma la nitidezza dell'immagine è
limitata dal piccolo obiettivo. Mentre per il Kepleriano si ha una aberrazione cromatica dell'obiettivo visibile senza
vetri correttivi, necessitando di oculari di qualità per mantenere il contrasto. 

\section{Strumentazione ausiliaria}
Il telescopio è solo uno strumento per ingrandire oggetti molto lontani anche se, da solo,
non è sufficiente per osservazioni scientifiche. Al telescopio si aggiunge dunque della strumentazione,
che prende il nome di strumentazione piano focale, che utilizza l'immagine prodotta dal telescopio stesso
per analizzare la radiazione che vi arriva. I principali strumenti che analizzano le immagini prodotte
dai telescopi sono i seguenti:
\begin{itemize}
    \item \textbf{Spettrometri}: analizzatori di spettro.
    \item \textbf{Fotometri}: analizzatori di flusso.
    \item \textbf{Polarimetri}: analizzatori di polarizzazione.
\end{itemize}

\section{Location dei telescopi}
I telescopi professionali sono spesso situati lontani dall'inquinamento luminoso (e atmosferico)
delle città, oppure nello spazio. Per condurre osservazioni molto precise è anche possibile combinare
diversi telescopi in tutta la Terra, come l'Event Horizon Telescope che ha ricavato la prima
immagine di un buco nero. Per ovviare al problema dell'assorbimento atmosferico di 
determinate lunghezze d'onda e dell'inquinamento luminoso delle città, è possibile utilizzare i \textbf{radiotelescopi},
i quali, permettono di compiere osservazioni nelle onde radio. Questa tipologia di telescopi sono
costituiti da delle enormi parabole che possono anche essere combinate in array come il \textbf{Very large array} in Messico.
Alternativamente è possibile utilizzare dei telescopi nello spazio per ovviare a tutti i problemi connessi 
all'atmosfera. Alcuni di questi telescopi sono il \textbf{James Webb Space Telescope}, un telescopio Cassegrain ad infrarossi, 
o l'\textbf{Euclid} (un telescopio ottico e ad infrarossi). Questi telescopi sono situati nel punto $L_2$, anche chiamato \textbf{Secondo punto Lagrangiano}, 
dove le forze orbitali e gravitazionali si bilanciano, permettendo ai telescopi di rimanere fermi (relativamente) 
rispetto alla Terra, orbitando con lo stesso periodo orbitale. 


\chapter{Fotometria}
La maggior parte delle osservazioni astronomiche si fonda sull'acquisizione di radiazione elettromagnetica. La \textbf{fotometria}
si occupa di misurare il flusso
di una sorgente in una banda di lunghezze d'onda per ottenere informazioni sulla sorgente stessa. Dalle
misure fotometriche si ottengono:
\begin{itemize}
    \item \textbf{Intensità}: detta anche \textbf{Radianza} o \textbf{Brillanza}.
    \item \textbf{Flusso}: anche chiamato densità di flusso o \textbf{Emittanza} se emesso da una sorgente di radiazione o \textbf{Irradianza} se raccolto da un rilevatore.
    \item \textbf{Luminosità}: quantità di "erg" ricevuta per unità di tempo.
    \item \textbf{Densità di energia}: ossia la radiazione contenuta in un dato volume.
    \item \textbf{Magnitudine}: misura logaritmica della luminosità apparente.
    \item \textbf{Colori}: differenza di magnitudine tra due bande, utile per stimare la temperatura di un corpo celeste, la composizione, distanza, \dots.
\end{itemize}

\section{Sistemi fotometrici}
Un sistema fotometrico è un insieme di \textbf{filtri} e \textbf{procedure di calibrazione} usati per misurare 
la luminosità di un oggetto celeste in specifiche bande di lunghezza d'onda, confrontando le misure fatte con strumenti diversi in tempi diversi. Fornisce informazioni sul colore e quindi sulla temperatura, composizioni 
e stato evolutivo degli oggetti. Le componenti principali di un sistema ottico sono le seguenti:
\begin{itemize}
    \item \textbf{Filtri}: selezionano intervalli di lunghezze d'onda (come l'ultravioletto, il blu, 
    verde, rosso e infrarosso, onde gamma, onde radio e così via). 
    \item \textbf{Curve di trasmissione}: descrivono l'efficienza del sistema filtro-ottica-rilevatore. 
    \item \textbf{Stelle standard}: Sono utilizzate per calibrare le misurazioni effettuate dai sistemi fotometrici. 
\end{itemize}
Alcuni sistemi fotometrici diffusi sono il Johnson-Cousins UBVRI, ossia il più famoso sistema fotometrico. È un sistema fotometrico
storico che utilizza filtri per gli ultravioletti, il blu, il verde, il rosso e gli infrarossi,così definito
\begin{align*}
    \begin{tabular}{c c c}
        & $\lambda_0(nm)$ & $\Delta \lambda(nm)$\\
        \textcolor{purple}{U} & 365 & 70 \\
        \textcolor{cyan}{B} & 440 & 100 \\
        \textcolor{green}{V} & 550 & 90 \\
        \textcolor{orange}{R} & 700 & 220 \\
        \textcolor{red}{I} & 880 & 240 
    \end{tabular}
\end{align*}

\clearpage
\section{Intensità}
\begin{wrapfigure}{r}{0.45\textwidth}
    \centering
    \caption{Schematizzazione dell'Intensità attraverso una definita superficie $dA$}
    \begin{tikzpicture}
        \draw(0, 0) ellipse (0.5 and 1);
        \draw[dashed](0, -1) -- (0, 1);
        \filldraw(0, 0) circle (1pt);
        \node at (-0.2, 0) {$P$};
        \node at (-0.5, -1.2) {$dA\cos\theta$};
        \draw[rotate around={55:(0, 0)}, red] (0, 0) ellipse (0.5 and 1); 
        \node[red] at (-0.75, 1.25) {$dA$};
        \draw[->, red](0, 0) -- (1, 1.42) node[at end, right] {$\vv{n}$ };
        \draw[->](0, 0) -- (3, 0) node[at end, below] {$I_{\nu}$};
        \draw(0.5, 0.71) arc (55:0:0.85) node[midway, right] {$\theta$};
        \draw[cyan] (0, 0) -- (2.5, 1);
        \draw[cyan] (0, 0) -- (2.5, -1);
        \draw[cyan](2.5, 0) ellipse (0.5 and 1);
        \node[cyan] at (2.2, -1.1) {$d\Omega$};
    \end{tikzpicture}    
\end{wrapfigure}
Assumendo che una certa quantità di radiazione fluisca attraverso una superficie $dA$. 
Una parte di tale radiazione attraverserà $dA$, entro l'angolo solido $d\Omega$ centrato lungo 
la direzione indicata dalla freccia in figura, formante un angolo di $\theta$ con la normale alla 
superficie $dA$. Si può determinare la quantità di energia caratterizzata da una frequenza tra
$\nu$ e $\nu + d\nu$ entrante l'angolo solido $d\Omega$ in un certo intervallo di tempo $dt$ come
\begin{align}
    dE_\nu = I_\nu \cos\theta \ dA \ dv \ d\Omega \ dt
\end{align}
In questa formula si identificano le seguenti:
\begin{itemize}
    \item $I_\nu$ prende il nome di \textbf{Intensità specifica} della radiazione 
considerata ad una data frequenza $\nu$ nella direzione dell'angolo solido $d\Omega$.
Infatti, la sua unità di misura è Wm$^{-2}$Hz$^{-1}$sr$^{-1}$, dove
$sr$ è lo \textbf{steradiante}. 
    \item $dA\cos\theta$: si deve al fatto che l'elemento di superficie considerato $dA$  è visto dalla
    direzione $\theta$. 
\end{itemize}
Adesso è possibile definire l'Intensità luminosa totale a tutte le frequenze come l'integrale su 
tutto il dominio delle frequenze:
\begin{align}
    I = \int_{0}^{+\infty } I_\nu d\nu 
\end{align}
L'Intensità può assumere diversi nomi:
\begin{itemize}
    \item \textbf{omogenea}: se $\frac{dI_\nu}{dP} = 0$;
    \item \textbf{isotropa}: se $\frac{dI_\nu}{d\Omega} = 0$;
    \item \textbf{stazionaria}: se $\frac{dI_\nu}{dt} = 0$. 
\end{itemize}


\subsection{Costanza dell'Intensità luminosa trasportata da un raggio luminoso}
È possibile dimostrare come l'Intensità luminosa non dipenda dalla distanza.
Date due superfici $dS_1$, $dS_2$, con vettori normali $\vv{n_1}$ e $\vv{n_2}$, angoli solidi
$d\Omega_1$ e $d\Omega_2$ e angoli rispetto ai vettori normali di $\theta_1$ e $\theta_2$:
\begin{gather*}
    dE_1 = I_1 \ dS_1 \cos\theta_1 \ dt_1 \ d\Omega_1 \qquad dE_2 = I_2 \ dS_2 \cos\theta_2 \ dt_2 \ d\Omega_2 
\end{gather*}
Se le due superfici sono spaziate tra loro di una certa distanza $D$, allora è possibile
esprimere gli angoli sferici nella seguente maniera: 
\begin{gather*}
    d\Omega_1 = \frac{dS_2\cos\theta_2}{D^{2}} \qquad d\Omega_2 = \frac{dS_1 \cos\theta_1}{D^{2}}
\end{gather*}
Dunque
\begin{gather*}
    I_1 \ dS_1\cos\theta_1 \ dt_1 \frac{dS_2 \cos\theta_2}{D^{2}} = I_2 \ dS_2 \cos\theta_2\ dt_2 \frac{dS_1\cos\theta_1}{D^{2}}
\end{gather*}
Ossia
\begin{gather*}
    I_1 dt_1 = I_2 dt_2
\end{gather*}
per far sì che l'energia che attraversa le due superfici sia uguale, si deve porre $dt_1 = dt_2 = dt$. 
Se così fosse si avrebbe che $dE_1 = dE_2$, dunque $I_1 = I_2$. 

\subsection{Brillanza superficiale del Sole}
Si assuma 



\clearpage
\section{Flusso}
\begin{wrapfigure}{r}{0.45\textwidth}
    \centering
    \caption{Rappresentazione del flusso}
    \begin{tikzpicture}[scale=1.4]
        \draw(0, 0) -- (3, 0);
        \draw(0, 0) -- (0, 3);
        \draw(0, 0) -- (2.5, -0.5);
        \draw(0, 3) arc (90:0:3);
        \draw(0, 3) arc (71:0:3.72);
        \draw(3, 0) arc (-37:-51:3);
        \draw[<-](1.4, 0) arc (-37:-51:1.4) node[at start, above] {$\phi$};
        \draw[green](1.7, 0) arc (-37:-51:1.7) node[midway, right] {$d\phi$};
        \draw(0, 0) -- (0.75, 2.65);
        \draw[->](0, 1.5) arc (71:58:1.85) node[midway, above] {$\theta$};
        \draw[thick, green](0.75, 2.65) arc (58:50:3.72);
        \draw[thick, green](1, 2.83) arc (70:60:3) ;
        \draw[thick, green](1, 2.83) -- (0.75, 2.65);
        \draw[thick, green](1.17, 2.33) -- (1.47, 2.62);
        \node[green] at (0.9, 2.3)  {$d\theta$};
        \node[green] at (1.75, 2.3) {$d\phi\sin\theta$};
    \end{tikzpicture}    
\end{wrapfigure}
Il flusso, o densità di flusso, esprime la potenza della radiazione per unità di area, avendo 
dunque grandezza fisica Wm$^{-2}$Hz$^{-1}$, per il flusso rispetto ad una certa frequenza $\nu$. 
\begin{align}
        F_{\nu} = I_{\nu} \oint _{4\pi} \cos\theta \ d\Omega
\end{align}
Dato che il flusso è un'unità di misura molto grande, si utilizza il Jansky, che vale $10^{-26}$ Wm$^{-2}$ Hz$^{-1}$.
È possibile determinare il flusso attraverso una superficie sferica attraversata da un'intensità isotropa, 
partendo dalla definizione di un piccolo angolo solido:
\begin{gather*}
    d\Omega = \frac{dA\cos\theta}{r^{2} }
\end{gather*}
Si ottiene allora che il flusso (rispetto ad una data lunghezza d'onda $\lambda$), è
dato dalla seguente:
\begin{align*}
    F_\lambda = \int_{0}^{2\pi}d\phi\int_{0}^{\pi} d\theta I_\lambda \cos\theta\sin\theta 
    = 2\pi I_\lambda \int_{0}^{\pi}\cos\theta\sin\theta \ d\theta 
    = 2\pi I_\lambda \left[\frac{\sin^{2}\theta}{2}\right]^{\pi}_{0} = 0
\end{align*}
Data la simmetria della sfera, se si potesse osservare contemporaneamente sia la metà rivolta verso 
l'osservatore, che quella non rivolta verso l'osservatore, il flusso totale sarebbe nullo, e dunque la sfera
apparirebbe completamente buia. Se invece il flusso emergesse da solo metà 
della superficie sferica con Intensità isotropa, si otterrebbe un integrale
diverso da zero:
\begin{align*}
    F_\lambda &= \int_{0}^{2\pi}d\phi\int_{0}^{\frac{\pi}{2}} d\theta I_\lambda \cos\theta\sin\theta = \\
    &= 2\pi I_\lambda \int_{0}^{\frac{\pi}{2}} d\theta \cos\theta \sin\theta = 2\pi I_\lambda \left[\frac{\sin^{2}\theta}{2}\right]^{\frac{\pi}{2}}_{0} = \pi I_\lambda
\end{align*}

\subsection{Flusso in un punto $P$ a distanza $D$ prodotto da una stella: il flusso specifico.}
\begin{wrapfigure}{r}{0.4\textwidth}
    \centering
    \begin{tikzpicture}
        \draw(0, 0) circle (1);
        \draw(0, 0) -- (0, 1) node[midway, left] {$R_{\star}$} -- (5, 0) node[midway, above] {$r$};
        \draw(0, 0) -- (5, 0) node[midway, below] {$D$};
        \draw(2, 0) arc (180:169:3) node[midway, left] {$\theta_r$};
        \filldraw(5, 0) circle (1pt) node[anchor = south]{$P$};
    \end{tikzpicture}    
\end{wrapfigure}
Si analizza ora il flusso della radiazione luminosa in un punto $P$ a grande distanza $D$ prodotto da una stella.
La stella è schematizzabile come se fosse un corpo sferico ad emissione isotropa. 
L'angolo $\theta$ su cui si integra arriva fino a $\theta_r$, definito dalla trigonometria come
\begin{gather*}
    \sin\theta_r = \frac{R_{\star}}{D}
\end{gather*}
nel caso in cui l'angolo tra $R_\star$ e $D$ sia $\frac{\pi}{2}$. Ogni elemento della superficie
avrà un flusso infinitesimo, definito come prima, che contribuisce al flusso totale emesso
dalla stella. Si può adesso quantificare il flusso specifico di questa stella $f_\lambda$ ad una certa
distanza come il flusso semisferico per unità di angolo solido. Dunque 
\begin{align*}
    f_\lambda &= \int_{0}^{\theta_r} d\theta \int_{0}^{2\pi} d\phi I_\lambda \cos\theta \sin\theta  = 2\pi \int_{0}^{\arcsin \frac{R_\star}{D}} I_\lambda \cos\theta\sin\theta \ d\theta  \\
    &= 2\pi I_\lambda \int_{0}^{\frac{R_\star}{D}}\sin\theta d\sin\theta  = 2\pi \int_{0}^{\frac{R_\star}{D}}y \ dy = 2\pi I_\lambda \left[\frac{y^{2}}{2}\right]^{\frac{R_\star}{D}}_{0} = \pi \frac{R_\star^{2}}{D^{2}}I_\lambda 
\end{align*} 
Il flusso è proporzionale a $\frac{1}{D^{2}}$. 

\clearpage
\section{Luminosità}
\begin{wrapfigure}{r}{0.3\textwidth}
    \centering
    \begin{tikzpicture}[scale=1.2]
        \draw[dashed](0, 0) -- (3, 0) node[midway, above] {$D$};
        \filldraw[orange] (0, 0) circle (0.4);
        \draw[red] (3, 0) circle (0.15);
        \node[orange] at (0, 0.75) {Stella};
        \node[red] at (3, 0.5) {$f_\lambda$};
    \end{tikzpicture}    
\end{wrapfigure}
La \textbf{luminosità specifica} corrisponde al flusso totale che passa attraverso 
la superficie che contiene la sorgente. La luminosità specifica avrà dunque
dimensioni WHz$^{-1}$.
\begin{align}
    L_\nu = \int_{A} F_\nu \ dA
\end{align}
Per una stella si integra il flusso sulla sua superficie $A$, se $F_\lambda = \pi I_\lambda$ è costante.
Dunque, dato che $A$ corrisponde esattamente alla superficie della stella, 
se questa avesse raggio $R_\star$ si otterrebbe
\begin{align}
    L_\lambda = 4\pi R^{2}_\star F_\lambda 
\end{align}
Dove $F_\lambda$ è il flusso emesso dalla superficie della stella. Se l'osservatore si ponesse ad una 
certa distanza $D$ da questa stella, la quale si assume emetta isotropicamente con $F_\lambda = \pi I_\lambda$ e che
non ci sia assorbimento lungo $D$, allora si ha che
\begin{gather*}
    f_\lambda = \frac{L_\lambda}{4\pi D^{2}} = \frac{4\pi R_\star^{2}F_\lambda}{4\pi D^{2}} = \frac{R_\star^{2}}{D^{2}}\pi I_\lambda
\end{gather*} 

\section{Densità di energia}
\begin{wrapfigure}{r}{0.4\textwidth}
    \centering
    \caption{}
    \begin{tikzpicture}
        \draw(0, 0) -- (0.75, 0.38);
        \draw[dashed](1, 0.5) -- (3, 1.5);
        \draw[rotate around={30:(3, 1.5)}](3, 1.5) ellipse (0.4 and 0.7);
        \draw[->](3, 1.5) -- (4, 2) node[at end, right] {$I$};  
        \draw[rotate around={30:(1, 0.5)}](1, 0.5) ellipse (0.4 and 0.7);
        \draw(0.75, 1.15) -- (2.7, 2.13);
        \draw(1.3, -0.13) -- (3.35, 0.9);
        \draw[<->](1.3, -0.33) -- (3.35, 0.7) node[midway, below] {$c \ dt$};
        \node at (2.9, 1.8) {$dA$};
    \end{tikzpicture}    
\end{wrapfigure}
Si può calcolare la densità di energia di un campo di radiazione contenuta in un certo volume 
\begin{gather*}
    dV = dA \cdot c \ dt
\end{gather*}
Supponendo che della radiazione con intensità $I$ attraversi una superficie 
$dA$ perpendicolarmente, e che questa radiazione percorra un certo spazio $c \cdot dt$
riempiendo dunque un volume di $dV = c \ dt \ dA$. L'energia attraverso questo volume è 
data dalla seguente relazione
\begin{gather*}
    dE = I \ dA \ d\Omega \ dt = \frac{I}{c}\ d\Omega \ dV
\end{gather*}
Pertanto è possibile ricavare la densità di energia $u$ come la derivata rispetto 
al volume dell'energia:
\begin{gather*}
    \left.\frac{dE}{dV}\right|_{\Omega} = \left. u \right|_{\Omega} = \frac{I}{c} \ d\Omega 
\end{gather*}
Se l'emissione di $I$ è isotropa, allora è possibile integrare per ottenere la densità di energia come
\begin{align}
    u = \oint_{4\pi} \left. u \right|_{\Omega} d\Omega = \frac{4\pi}{c}I 
\end{align}

\subsection{Relazione tra radiazione elettromagnetica e unità radiometrica}
Si utilizza il \textbf{Vettore di Poynting} per descrivere il flusso di energia associato alla propagazione del campo elettromagnetico:
\begin{align}
    \vv{S} = \frac{1}{\mu_0}\vv{E} \times \vv{B}   
\end{align}
Il modulo di questo vettore si trova secondo le seguenti relazioni
\begin{gather*}
    \frac{E}{B} = c \ \Longrightarrow \ I = \frac{1}{T}\int_{T} \left| \vv{S}  \right| dt = \frac{1}{c\mu_0} \frac{E_{rms}^{2}}{2} 
\end{gather*}

\clearpage
\section{Fotometria nei telescopi}
\subsection{Sorgente puntiforme}
\begin{wrapfigure}{r}{0.4\textwidth}
    \centering
    \begin{tikzpicture}
        \draw[->](0, 0) -- (3.5, 0) node[at end, below] {$\lambda$};
        \draw[->](0, 0) -- (0, 3) node[at end, left] {$I_\lambda$};
        \draw[cyan](0, 0.05) -- (0.75, 0.05) node[at end, below] {380};
        \draw[cyan](2.25, 0.05) -- (3, 0.05) node[at start, below] {750};
        \draw[cyan](0.75, 0.05) .. controls (0.85, 0.1) and (0.9, 0.2) .. (1, 1.5) -- (2, 1.5) .. controls (2.1, 0.2) and (2.15, 0.1) .. (2.25, 0.05);
        \draw(0, 0) .. controls (0.75, 0.1) and (1.3, 2.1) .. (1.5, 2) .. controls (2.25, 0.75) and (2.75, 0.5) .. (3.5, 0.2);
        \draw[red](1.4, 0.05) .. controls (1.65, 0) and (1.7, 1.7) .. (1.75, 1.75) .. controls (1.8, 1.7) and (1.85, 0) .. (2.1, 0.05);
        \node[red] at (1.9, 1.9) {$\epsilon_\lambda$};
        \node[cyan] at (0.75, 1.75) {$\eta_\lambda$};
        \node at (1.5, 2.4) {$I_\lambda$};
    \end{tikzpicture}    
\end{wrapfigure}
Le immagini delle galassia o dei corpi celesti sono la combinazione di 
immagini prese con filtri diversi e combinate insieme per ottenere le immagini
"belle" che si vedono su internet. L'occhio umano, durante il processo evolutivo, è stato
tarato per osservare meglio la radiazione luminosa dello spettro vicino al colore verde.
Ricordando i parametri del telescopio, si può determinare l'energia 
raccolta dal telescopio da una sorgente puntiforme secondo la seguente relazione:
\begin{align}
    E = \pi \frac{R_{\star}^{2} }{d^{2} }A \int_{0}^{+\infty }I_{\lambda}\eta_\lambda\epsilon_\lambda \  \Delta t  \ d\lambda
\end{align}
Dove 
\begin{itemize}
    \item $I_{\lambda}$: intensità luminosa in funzione della lunghezza d'onda
    \item $\eta_\lambda$: estinzione atmosferica e interstellare;
    \item $\epsilon_\lambda$: efficienza del filtro, ossia l'assorbimento del filtro.
    \item $A$: apertura della lente del telescopio;
    \item $\Delta t$: tempo di esposizione.
\end{itemize}
Per una sorgente luminosa puntiforme dunque $E \propto A$. 

\subsection{Sorgente estesa}
\begin{wrapfigure}{r}{0.4\textwidth}
    \centering
    \begin{tikzpicture}
        \draw(0, 0) ellipse (0.25 and  0.75);
        \node at (0, 1.3) {$A$};
        \draw[<->](0.1, 0.75) -- (0.8, 0.75) node[midway, above] {$f$};
        \draw[<->](-0.1, -0.75) -- (-2.4, -0.75) node[midway, below] {$D$};
        \draw[<->](-2.9, -0.5) -- (-2.9, 0.5) node[midway, left] {$S$};
        \draw(-2.8, -0.5) -- (-2.25, -0.25) -- (-2.25, 0.75) -- (-2.8, 0.5) -- (-2.8, -0.5);
        \filldraw(-2.5, 0) circle (1pt) node[anchor = south] {$P$};
        \draw[->](-2.5, 0) -- (-2, 0) node[at end, right] {$I$};
        \draw[green](0, -0.75) -- (-2.5, 0) -- (0, 0.75);
        \draw[green](-1, 0) ellipse (0.15 and 0.45);
        \node[green] at (-1.1, -0.6) {$\Omega$};
        \draw(0.75, -0.2) -- (1.05, -0) -- (1.05, 0.3) -- (0.75, 0.1) -- (0.75, -0.2);
        \draw[<->](1.15, -0) -- (1.15, 0.3) node[midway, right] {$S'$};
        \draw[red](-2.25, 0.75) -- (0.75, -0.2);
        \draw[red](-2.25, -0.25) -- (0.75, 0.1);
    \end{tikzpicture}    
\end{wrapfigure}
Nel caso di una sorgente estesa si deve anche considerare la dimensione dell'elemento sensibile 
che definisce la porzione di sorgente estesa rilevata. Se si indica con $S$ l'estensione della sorgente,
al piano focale questa sorgente non sarà più puntiforme ma avrà una estensione pari a $S'$.
Chiamando $A$ l'apertura del telescopio, $f$ la sua lunghezza focale e $D$ la distanza dell'oggetto
osservato rispetto alla lente, si ottiene l'angolo solido per la sorgente in $P$ ed il rapporto tra
la dimensione reale dell'oggetto e la dimensione sul led della CCD (sensore) come: 
\begin{gather*}
    \Omega = \frac{A}{D^{2} } \qquad \frac{S^{2} }{D^{2} } = \frac{S'^{2} }{f^{2} }
\end{gather*}
Allora il flusso in funzione della lunghezza d'onda sarà
\begin{gather*}
    f_\lambda = \frac{S^{2} }{D^{2} }I_\lambda \qquad f_\lambda = \frac{I_\lambda}{D^{2} } \pi R_{\star}^{2} 
\end{gather*}
Dunque l'energia che riceve il rilevatore ad una certa lunghezza d'onda sarà data, per una certa esposizione $\Delta t$, da
\begin{gather*}
    E_\lambda = I_\lambda \Omega S^{2}\eta_\lambda \epsilon_\lambda \Delta t 
\end{gather*}
Per ottenere l'energia totale ricevuta per tutte le lunghezze d'onda si deve integrare 
\begin{align}
    E = \Delta t A \frac{S'^{2} }{f^{2}}\int I_\lambda\eta_\lambda\epsilon_\lambda \ d\lambda 
\end{align}
Dove gli estremi di integrazione non sono definiti in quanto dipendono dalle
lunghezze d'onda determinate dal filtro utilizzato. Per una sorgente estesa dunque
$E \propto \frac{A}{f^{2}} = \left(\frac{1}{F^{\#}}\right)^2$.  
Il parametro $\epsilon_\lambda$ determina la \textbf{Magnitudine monocromatica} 
nel sistema fotometrico di Johnson-Morgan-Cousins e dipende da svariati fattori 
come il rilevatore, il filtro, il telescopio ma anche le condizioni atmosferiche. Infatti, il cielo ha tantissime variabili che possono interferire
    sulla qualità delle osservazioni. Parametri come l'umidità, la luce zodiacale,
    il vento e l'inquinamento possono modificare sostanzialmente la quantità di luce raccolta. 
Quando vengono soddisfatte queste condizioni si parla di fotometria assolute. 


\chapter{Magnitudini}
\section{La scala della magnitudine e i tipi di magnitudine}
L'idea di classificare le stelle in base alla loro luminosità è nata
già nell'antica Grecia con Ipparco, il quale divise le stelle
in 6 classi: nella prima classe c'erano le stelle molto luminose mentre
nella classe 6 ci stavano le stelle appena percettibili. Si introducono
adesso i concetti principali della \textbf{magnitudine}.

\subsection{Magnitudine apparente}
\begin{wrapfigure}{r}{0.4\textwidth}
    \centering
    \begin{tikzpicture}
        \draw[->](-1, 0) -- (4.5, 0) node[at end, right] {$\log F$};
        \draw[->](0, -1) -- (0, 4.5) node[at end, left] {$m$};
        \draw(0.5, 4) -- (4, 0.5);
        \draw[dashed](0.5, 0) -- (0.5, 4) node[at start, below] {$F$};
        \draw[dashed](0, 0.5) -- (4, 0.5);
        \draw[dashed](4, 0) -- (4, 0.5) node[at start, below] {$100F$};
        \draw[dashed](0, 4) -- (0.5, 4);
    \end{tikzpicture}    
\end{wrapfigure}
Norman R. Pogson nel 1856 si accorse che la risposta dell'occhio umano alla 
differente luminosità dei corpi celesti non era lineare, bensì logaritmica. Egli 
diede dunque forma matematica alla classificazione di Ipparco secondo il grafico a fianco.
Definì che il rapporto tra le luminosità apparenti delle stelle di classe $n$ e $n+1$ era pari a
$100^{\frac{1}{5}}$, ossia circa $\approx 2.512$.  
Si può ottenere la \textbf{magnitudine apparente} di un corpo come
\begin{align}
    m = -100^{\frac{1}{5}}\log\frac{F}{F_0} \ \approx \ -2.5\log\frac{F}{F_0}
\end{align}
La magnitudine apparente è dunque definita in funzione della densità di flusso osservata, $F$,
stabilendo, per esempio, che alla magnitudine zero corrisponda un valore noto e standard di $F_0$. 
La magnitudine è tarata in modo tale che $F_0$ corrisponda al flusso della stella Vega; in altre parole,
la stella Vega ha magnitudine zero. Per ricavare la differenza tra la magnitudine tra due oggetti celesti 
con magnitudine $m_1$ e $m_2$, si potrebbe esprimere considerando la scala di  Ipparco ed
esprimo le magnitudini $ 1$ e $6$ come
\begin{gather*}
    1 = a \log F + b \\
    6 = a \log F + b
\end{gather*}
Facendo la sottrazione membro a membro si ha
\begin{gather*}
    -5 = a \log \frac{100F}{F} \ \Longrightarrow \ a \approx -2.5
\end{gather*}
Allora la differenza di magnitudini tra due oggetti è data da:
\begin{gather*}
    m_1 = -2.5\log F_1 + b \\
    m_2 = -2.5\log F_2 + b
\end{gather*}
Dunque si ottiene
\begin{gather*}
    m_1 - m_2 = -2.5\log \frac{E_1}{E_2} = -2.5\log\frac{F_1}{F_2}
\end{gather*}
Tutti gli oggetti con magnitudine più grande di zero sono meno luminosi
di Vega mentre tutti gli oggetti con magnitudine più piccola di zero sono
più luminosi di Vega. Questa tipologia di magnitudine prende il nome di \textbf{apparente}
perché dipende dallo strumento con cui si compiono le osservazioni, dalla lunghezza
d'onda osservata e dalla distanza dell'oggetto dall'osservatore. La magnitudine fino ad ora discussa non fornisce dunque alcun tipo di 
informazione sulla luminosità propria di un oggetto.

\subsection{Magnitudine monocromatica}
La magnitudine monocromatica si definisce con $\epsilon_\lambda$ secondo il 
sistema fotometrico Johnson-Morgan-Cousins e non è facile da definire. Infatti essa dipende
dalle seguenti condizioni 
\begin{itemize}
    \item Rilevatore 
    \item Filtri
    \item Telescopio
    \item Osservatorio
    \item Cielo fotometrico
\end{itemize}
Nel caso in cui queste condizioni siano perfette, si parla di fotometria assoluta.

\subsection{Magnitudine bolometrica}
La magnitudine \textbf{bolometrica} è la magnitudine che si otterrebbe se
si riuscisse a misurare la radiazione emessa da un corpo celeste in tutto lo spettro. 
In pratica è impossibile poiché la maggior parte della radiazione di un corpo celeste è 
assorbita dall'atmosfera e dal mezzo interstellare, oltre che è impossibile,
con un singolo rilevatore, osservare tutta la radiazione emessa da un corpo celeste. Si può derivare
la magnitudine bolometrica applicando la \textbf{correzione bolometrica}. 
\begin{align}
    m_{bol} = m_V - BC
\end{align}
Sempre che si assuma $\eta_\lambda = \epsilon_\lambda = 1$. Per le stelle di tipo 
spettrale F5 (come il Sole), $BC = 0$. 

\section{Lunghezza d'onda efficace}
Assumiamo $F(\lambda)$  sia variabile all'interno della banda spettrale monocromatica,
si calcola la sua variazione totale come
\begin{gather*}
    F(\lambda) \approx F(\lambda_0) + \left.\frac{dF(\lambda)}{d\lambda}\right|_{\lambda = \lambda_0} (\lambda - \lambda_0 )
\end{gather*}
Di conseguenza è possibile integrare da entrambe le parti per ottenere l'energia che giunge
al telescopio da una certa distanza $D$
\begin{align*}
    E  &\approx A\frac{R_{\star}^{2} }{D^{2} }\Delta t \int_{0}^{\infty } \left(F(\lambda_0) + \left.\frac{dF(\lambda)}{d\lambda}\right|_{\lambda_0}(\lambda - \lambda_0)\right) \eta_\lambda \epsilon_\lambda \ d\lambda = \\
    &= A \frac{R_{\star}^{2} }{D^{2} } \Delta t \left(F(\lambda_0)\int_{0}^{\infty }\eta_\lambda \epsilon_\lambda \ d\lambda + \left.\frac{dF(\lambda)}{d\lambda}\right|_{\lambda_0}\int_{0}^{\infty } (\lambda - \lambda_0)\eta_\lambda \epsilon_\lambda \ d\lambda\right) = \\
    &= A \frac{R_\star^{2}}{D^{2}}\Delta t F(\lambda_0)\int_{0}^{+\infty } \eta_\lambda \epsilon_\lambda \ d\lambda  
\end{align*}
Il secondo integrale è nullo poiché il flusso di una variazione infinitesima è
anch'esso infinitesimo. Dunque si può ottenere il valore di $\lambda_0$, anche chiamato \textbf{lunghezza d'onda efficace}, 
come il valore medio di $\lambda$ pesato su $\eta_\lambda$ e $\epsilon_\lambda$. 
\begin{gather*}
    \lambda_0 = \frac{\int_{0}^{\infty }\lambda\eta_\lambda \epsilon_\lambda \ d\lambda}{\int_{0}^{\infty }\eta_\lambda \epsilon_\lambda \ d\lambda}
\end{gather*}

\section{Magnitudine assoluta}
La \textbf{magnitudine assoluta} è una grandezza che descrive la vera luminosità di un oggetto
celeste e non dipende dalla sua posizione rispetto all'osservatore. Per rendersi indipendenti 
dalla posizione, si fa in modo di definire la magnitudine assoluta come la magnitudine apparente dell'oggetto
come se fosse posto a $10$ parsec di distanza dall'osservatore. Dunque
la magnitudine assoluta si ricava mediante la differenza tra 
\begin{align}
    m - M \approx -2.5\log \frac{F(r)}{F(10 \ pc)} = -2.5 \log\frac{E(r)}{E(10 \ pc)}=  -2.5 \log \left(\frac{10 \ pc}{r}\right)^{2}
\end{align} 
Dove $r$ è la distanza effettiva dell'oggetto dall'osservatore. Con questa formulazione è 
dunque possibile sia ricavare la magnitudine assoluta che la distanza, se si
ha una della due.  Svolgendo i conti e semplificando, la differenza prende il nome di \textbf{modulo di distanza}:
\begin{align}
    m - M = 5\log r - 5 
\end{align}
Dove $r$ è dato in parsec. 

\section{Indice di colore}
L'\textbf{indice di colore} fornisce informazioni sul colore della stella e dunque della sua
temperatura e, in prima approssimazione, sembrerebbe che non dipenda dalla distanza (anche se si dimostrerà che
non è così). Si ricava l'indice di 
colore per una data stella facendo la differenza tra le magnitudini 
corrispondenti a 2 filtri di colore diversi. Ad esempio, si utilizza la
differenza tra la banda blu e quella verde:
\begin{gather*}
    B - V = m_B - m_V = -2.5 \log \frac{F_B}{F_V} + const
\end{gather*}
La costante dipende dal sistema fotometrico utilizzato (Infatti sistemi fotometrici 
differenti potrebbero avere scale differenti):
\begin{gather*}
    m_B = m_{B_0} - 2.5 \log F_B + 2.5 \log F_{B_0} \\
    m_V = m_{V_0} - 2.5 \log F_V + 2.5 \log F_{V_0}
\end{gather*}
La costante sarà allora
\begin{gather*}
    c = m_{B_0} - m_{V_0} + 2.5 \log \frac{F_{B_0}}{F_{V_0}}
\end{gather*}
Nel sistema JMK si ha che $m_{B_0} = m_{V_0} = 0$ e quindi i flussi 
$F_{B_0}$ e $F_{V_0}$ sono dati tali per cui $B  - V = 0$ per
la stella Vega: questa si chiama \textbf{convenzione di Vega}. La costante
prende il nome di \textbf{zero point} che cambia in base al sistema fotometrico
scelto: in questo modo se so quali sistemi utilizzo posso passare
dalle magnitudini delle stelle ai flussi.  
In questo modo si può correlare direttamente l'emissione di luce
colorata con la temperatura: nel caso dell stella Vega, l'emissione tra il blu
ed il verde è zero  allora $B - V$. In una stella blu
$V > B$, allora la banda $B$ è più luminosa in quanto è proporzionale
al logaritmo inverso: si ha prevalenza di flusso blu rispetto al flusso verde:
\begin{align*}
    B - V < 0 \ &\Longrightarrow \ \text{Stella blu} \\
    B - V > 0 \ &\Longrightarrow \ \text{Stella giallo-verde}
\end{align*}
Come si vedrà poi in dettaglio nella sezione dedicata alle stelle, più una stella emette 
in luce blu, e più è calda rispetto ad una stella che ha una emissione maggiore sulla banda verde.


\chapter{Effetti dell'atmosfera}
L'atmosfera terrestre presenta sia vantaggi che svantaggi per l'osservazione e lo
studio della sfera celeste.  I problemi principali derivati dall'atmosfera per
l'astronomia sono i seguenti:
\begin{itemize}
    \item Solo la finestra ottica e radio sono (quasi) completamente libere;
    \item Seeing: è la misura della qualità della risoluzione angolare di un 
    osservazione astronomica da Terra, limitata dalla \textbf{turbolenza atmosferica};
    \item Airglow: effetto naturale e artificiale dovuto alla luce
    \item Rifrazione: causa angoli zenitali grandi;
    \item Estinzione;
\end{itemize}
Altri effetti dovuti all'atmosfera sono la presenza di nuvole e vento che ostacolano le osservazioni 
mentre l'umidità ed il particolato atmosferico deteriorano i telescopi. 

\section{Estinzione atmosferica}
\begin{wrapfigure}{r}{0.4\textwidth}
    \centering
    \begin{tikzpicture}[scale=1.2]
        \draw[<->](0.5, 0) -- (0.5, 2) node[midway, left] {$H$};
        \draw(1, 0) -- (4, 0);
        \draw[red, ->](2, 0) -- (2, 2) node[at end, left] {$x$};
        \draw[orange, ->](2, 0)-- (3, 2) node[at end, right] {$x'$};
        \draw[->](3, 2) -- (2.75, 1.5) node[at end, right] {$I_{0\lambda}$}; 
        \draw(2, 1) arc (90:65:1) node[midway, above] {$\theta$}; 
    \end{tikzpicture}    
\end{wrapfigure}
È possibile schematizzare l'atmosfera come se fossero una serie
di piani paralleli tra loro con indici di rifrazione diversi. Nel caso in cui le distanze
zenitali non siano troppo elevate, dunque $\theta < 60$°, si possono fare le seguenti considerazioni. 
Una stella che è posta ad una distanza zenitale $\theta$ attraversa una quantità di 
atmosfera pari a
\begin{align}
    x = \frac{H}{\cos\theta}
\end{align}
$H$ indica l'altezza totale dell'atmosfera allo zenith e $I_{0\lambda}$ l'intensità proveniente dal corpo 
celeste osservato.   \\ \noindent
Si può definire adesso $k_\lambda(x)$ come il \textbf{coefficiente di estinzione} atmosferica (che dipende dunque
solo dalla quota e non dall'angolo). Secondo la legge di Lambert-Beer si può esprimere come 
si attenua l'intensità luminosa in funzione del coefficiente di estinzione. 
\begin{align}
    dI_\lambda = I_\lambda k_\lambda(x) \ dx'
\end{align}
Se la direzione di una stella è come quella in figura, si può definire il cammino
$dx'$ infinitesimo come
\begin{gather*}
    dx' = \frac{dx}{\cos \theta} = \sec \theta \ dx \ \Longrightarrow \ dI_\lambda = I_\lambda k_\lambda(x) \sec\theta \ dx 
\end{gather*}
Si integra ora per separazione delle variabili ottenendo 
\begin{gather*}
    \int_{I_{0\lambda}}^{I_\lambda} \frac{dI_\lambda '}{I_\lambda'} = \int_{H}^{0} k_\lambda(x) \sec\theta \ dx \ \Longrightarrow \ \left[\ln I_\lambda'\right]^{I_0}_{I_{0\lambda}}  = - \int_{0}^{H}k_\lambda (x) \sec \theta \ dx 
\end{gather*}
Ottenendo allora 
\begin{gather*}
    I_\lambda = I_{0\lambda}\exp\left(- \int_{0}^{H}k_\lambda (x) \sec \theta \ dx\right) = I_{0\lambda}\exp\left(-\sec\theta \int_{0}^{H} d\tau_\lambda \right)
\end{gather*}
Dove si è definito
\begin{gather*}
    d\tau_\lambda = k_\lambda(x) \ dx 
\end{gather*}
come lo \textbf{spessore ottico}. In termini di magnitudini si può relazionare il passaggio attraverso una
atmosfera "più fitta" secondo le seguenti espressioni:
\begin{gather*}
    m_\lambda - m_{\lambda0} = -2.5\log \frac{I_\lambda}{I_{0\lambda}} = -2.5 \frac{-sec\theta \int_{0}^{H}d\tau_\lambda }{\ln 10}
\end{gather*}
Ottenendo dunque 
\begin{align}
    m_\lambda = m_{0\lambda} + \frac{2.5}{\ln 10} \sec\theta \int_{0}^{H}d\tau_\lambda 
\end{align}
Chiamato allora l'ultimo integrale come $\tau_{1\lambda}$, ossia la \textbf{massa d'aria}, si può definire
\begin{gather*}
    x_\lambda = 2.5 \log e \tau_{1\lambda} \approx 1.086 \tau_{1\lambda}
\end{gather*}
Dunque si ottiene 
\begin{align}
    m_\lambda = m_{0\lambda} + x_\lambda \sec \theta
\end{align}
Nel caso limite di $\theta = 60$° si ha che $\sec \theta = 2$. Per determinare il coefficiente di estinzione si 
deve osservare il moto di una stella nel suo cammino in cielo per
diverse masse d'aria si può ricavare la magnitudine $m_\lambda$ per diversi 
valori della secante di $\theta$ per determinare la magnitudine reale che si 
avrebbe senza atmosfera tramite un fit lineare quando $\theta \to 0$. 

\section{Seeing}
Si definisce \textbf{seeing} il parametro che quantifica la sfocatura dell'immagine
dovuta all'atmosfera. Infatti l'atmosfera presenta le seguenti caratteristiche
\begin{itemize}
    \item L'atmosfera è composta da strati d'aria con temperature e densità diverse. 
    \item Ogni strato ha un indice di rifrazione diverso
\end{itemize}
L'immagine di ogni stella appare come un disco tremolante e sfocato. Un seeing buono è tra 
$0.5'' \sim 1.5''$, limitando la \textbf{risoluzione effettiva} dei telescopi a Terra, influenzando 
la qualità delle immagini e la precisione fotometrica e astrometrica.
Inoltre, l'atmosfera produce anche \textbf{diffrazione},
ossia il fenomeno per il quale le onde si propagano dietro a degli ostacoli. In particolare 
se un onda luminosa attraversa una fenditura circolare produce una figura di diffrazione 
che assume la forma della \textbf{Figura di Airy}: ossia un disco centrale con tanti dischi intorno.
La distanza tra il primo minimo ed il picco (dove $D$ è il diametro della fenditura):
\begin{gather*}
    \Delta \theta_D = 1.22 \frac{\lambda}{D}
\end{gather*}
Questo allargamento angolare è dovuto al fenomeno della diffrazione ed è difficilmente 
eliminabile. Dato che l'allargamento è un
fenomeno casuale, si potrebbe dire che abbia una distribuzione Gaussiana; posso 
allora determinare (attraverso il principio di FWHM) che
 $\Delta \theta_S$, ossia l'angolo di allargamento dovuto alla
\textbf{rifrazione} dell'atmosfera, si metta in relazione con $\Delta \theta_D$:
\begin{gather*}
    \Delta \theta_S \geq \Delta \theta_D
\end{gather*}
In buoni siti, si ha che $\Delta \theta_S$ è dell'ordine di circa un arcosecondo. Per telescopi
piccoli con aperture dell'ordine di $0.1 \ m$:
\begin{gather*}
    \Delta \theta_S = \Delta\theta_D = 1.22 \frac{\lambda}{D}
\end{gather*}
Si definisce il \textbf{parametro di Fried} che definisce, in funzione
della lunghezza d'onda, l'apertura del telescopio al limite dell'angolo di diffrazione:
\begin{align}
    1.22 \frac{\lambda}{\Delta \theta_S} = r_0
\end{align}
Quindi $\Delta\theta_D$ mi indica il limite teorico al di sotto del quale
non si riesce più a distinguere due sorgenti puntiformi. Generalmente, per telescopi con
grande potere risolutivo, si può utilizzare solamente $\Delta \theta_S$ in quanto
il $\Delta \theta_D$ sarà sempre molto minore.

\section{Estinzione interstellare e intergalattica}
Se tra la sorgente e l'osservatore non c'è il vuoto la radiazione viene diffusa
o assorbita dal mezzo interstellare e da quello intergalattico (che ha un effetto minore). Sebbene
molto rarefatto, le distanze stellari sono talmente tanto elevate che non è possibile ignorare il 
suo contributo. Si può comparare il numero di particelle per centimetro cubo:
\begin{align*}
    &\text{Media Via Lattea}& \ &\Longrightarrow \ &1 \ \text{cm}^{-3} \\
    &\text{Mezzo intergalattico}& \ &\Longrightarrow \ &10^{-3}  \ \text{cm}^{-3} \\
    &\text{Aria al L.D.M.}& \ &\Longrightarrow \ &2.4\cdot 10^{23}\ \text{cm}^{-3}
\end{align*}
che vengono fuori dalla relazione
\begin{gather*}
    n = \frac{P}{kT} \qquad T = 300 \ K, P = 10^{5} \ Pa
\end{gather*}
Per una stella con luminosità $L_0$ ad una certa distanza $r$, la luminosità scenderà 
ad un valore $L_\lambda$ e a distanza $r + dr$ ad un valore $L_\lambda - dL_\lambda$ a
causa dell'assorbimento (isotropo). Si ha allora che
\begin{gather*}
    dL_\lambda = - \alpha_\lambda L_\lambda \ dr
\end{gather*}
Dove $\alpha_\lambda$ prende il nome di \textbf{opacità}. Si definisce
\begin{gather*}
    d\tau_\lambda = \alpha_\lambda \ dr
\end{gather*}
come \textbf{spessore ottico}. Dunque si esprime la variazione di luminosità della
stella in funzione dello spessore ottico come 
\begin{align}
    dL_\lambda = -L_\lambda \ d \tau_\lambda
\end{align}
Integrando ora la luminosità dalla sorgente all'osservatore si avrà che
\begin{align}
    \int_{L_{0}}^{L_\lambda} \frac{dL_\lambda'}{L_\lambda'} = - \int_{0}^{\tau_\lambda} d \tau_\lambda ' \ \Longrightarrow \ L_\lambda = L_{0} e^{-\tau_\lambda}  
\end{align}
Il flusso della superficie della stella è dato da
\begin{gather*}
    F_{\lambda_0} = \frac{L_{\lambda_0}}{4\pi R_\star^{2}}
\end{gather*}
Mentre il flusso ad una profondità ottica 
\begin{gather*}
    f_\lambda = \frac{L_\lambda}{4\pi r^{2}} \ \Longrightarrow \ f_\lambda = F_0 \frac{R_\star^{2}}{r^{2}}e^{-\tau_\lambda}
\end{gather*}
Per determinare la magnitudine assoluta, si ricava il flusso a 10 pc senza estinzione interstellare:
\begin{gather*}
    f_\lambda (10 pc) = F_0 \frac{R_\star^{2}}{(10 pc)^{2}}
\end{gather*}
Si determina ora la magnitudine della stella 
\begin{gather*}
    m - M = -2.5 \log \frac{f_\lambda(r)}{f_\lambda(10 pc)} = -2.5\log\frac{F_0 \frac{R_\star^{2}}{r^{2}}e^{-\tau_\lambda}}{F_0 \frac{R_\star^{2}}{10^{2}}} = 5 \log \frac{r}{10 pc} + 2.5 \tau_\lambda \log e
\end{gather*}
Dunque, definendo $A_\lambda = 2.5 \tau_\lambda \log e$, come il valore dell'estinzione 
del mezzo interstellare, si ottiene la seguente espressione
per il modulo di distanza:
\begin{align}
    m - M = 5\log \frac{r}{10 pc } + A_\lambda
\end{align} 
Dove $m$ è la magnitudine apparente estinta e $M$ la magnitudine assoluta corretta per l'estinzione 
interstellare.


\section{Arrossamento interstellare}
Come avviene nell'atmosfera terrestre, la luce blu viene diffusa di più della
luce rossa, anche per le osservazioni nell'universo si assiste al fenomeno 
dell'\textbf{arrossamento interstellare}. L'indice di colore 
$B-V$ aumenta. La magnitudine $m_V$ di una stella risulta essere
\begin{align*}
    m_V &= M_V + 5\log \frac{r}{10 pc} + A_V \\
    m_B &= M_B + 5 \log \frac{r}{10 pc} + A_B
\end{align*} 
Assumendo che $M_V$ e $M_B$ siano non estinte, si può determinare l'indice $B-V$ come
\begin{gather*}
    B - V = m_B - m_V = M_B - M_V + A_B - A_V 
\end{gather*}
Equivalentemente
\begin{align}
    B - V = (B - V)_0 + E_{B - V}
\end{align}
Dove $(B- V)_0= M_B - M_V$ e prende il nome di \textbf{indice di colore intrinseco}
di una stella mentre $E_{B - V} = A_B - A_V$ viene chiamato \textbf{eccesso di colore}.
Esiste una relazione che lega l'estinzione nella banda $V(A_V)$ e l'eccesso di colore $E_{B - V}$. Dunque 
la magnitudine associata ha un valore all'incirca costante per tutte le stelle.  
\begin{align}
    R = \frac{A_V}{E_{B - V}} \approx 3.0
\end{align}
Grazie a tale relazione si ottiene il valore di $A_V$ da misure dell'eccesso di colore:
\begin{align}
    A_V \approx 3 E_{B - V}
\end{align}

\section{Ottica adattiva contro ottica attiva}
L'ottica attiva: 
\begin{itemize}
    \item Corregge lentamente deformazioni statiche o a lungo termine dallo specchio primario
    dovute a gravità, temperatura o flessioni meccaniche: lo specchio primario è mosso 
    da degli attuatori che ne modificano la forma e sensori che misurano la forma dello specchio
    correggendolo ogni minuti / ore. 
    \item Mantiene lo specchio nella form ottimale.
    \item Frequenza di correzione molto bassa. 
    \item Alcuni telescopi di grande diametro come VLT e LBT e Keck. 
\end{itemize}
L'ottica adattiva: 
\begin{itemize}
    \item Corregge in tempo reale le distorsioni rapide introdotte dalla turbolenza 
    atmosferica utilizzando un sensore di fronte d'onda che misura le distorsioni della luce stellare 
    centinaia o migliaia di volte al secondo mentre uno specchio secondario cambia forma 
    in tempo reale. 
    \item Ripristina un fronte d'onda piatto per ottenere immagini quasi al limite di diffrazione. 
    \item Frequenza di correzione altissima (centinaia di hertz). 
    \item Sistemi AO, VLT, Subaru. 
\end{itemize}



\part{Le Stelle}
\chapter{Radiazione e spettroscopia}
\section{Meccanismi di radiazione}
\begin{wrapfigure}{r}{0.4\textwidth}
    \centering
    \caption{L'atomo di Elio secondo Bohr}
    \begin{tikzpicture}
        \orbit(0, 0, 1);
        \neutron(0.1, -0.1);
        \neutron(-0.1, 0.1);
        \proton(0.1, 0.1);
        \proton(-0.1, -0.1);
        \electron(1, 0);
        \electron(-0.5, 0.845);
        \draw[thick, ->](-1.5, 1.2) -- (-0.6, 0.84) node[at end, above] {$\gamma$};
        \draw[thick, ->](1, 0.1) -- (1, 1) node[at end, right] {$\vv{v}$};
    \end{tikzpicture}
\end{wrapfigure}
Senza scendere troppo nei dettagli, e ricordando quanto già fatto al corso di Chimica,
la radiazione elettromagnetica è emessa o assorbita da atomi e molecole quando queste passano
da un livello di energia all'altro. Gli elettroni, quando passano da un livello energetico 
all'altro secondo la seguente relazione
\begin{align}
    E_f - E_i = \Delta E = h\Delta\nu 
\end{align}
Dove $h = 6.626 \cdot 10^{-34} Js$ e prende il nome di \textbf{costante di Planck}. I livelli 
energetici concessi per gli elettroni non sono continui ma \textbf{quantizzati}: questo 
vuol dire che agli elettroni, in un atomo, sono consentiti solamente determinati livelli energetici.  
Se si bombardasse un atomo con dei fotoni con una certa energia, accadrebbe che 
gli elettroni dello \textbf{stato fondamentale} possano saltare ad un livello energetico superiore 
che prende il nome di \textbf{stato eccitato}. Se
l'energia del fotone $\gamma$ è molto maggiore dell'energia $\Delta E$ necessaria per
poter far saltare l'elettrone allo stato eccitato, allora l'elettrone viene strappato 
all'atomo e quest'ultimo prende il nome di \textbf{ionizzato}.
A meno che non ci sia un flusso di fotoni che tenga stabile gli elettroni
allo stato eccitato, dopo un tempo $\delta \tau \approx 10^{-8} \ s$, gli elettroni decadono
rilasciando indietro lo stesso pacchetto di energia $\gamma$ che li ha
investiti, tornando dunque allo stato fondamentale. \\ \noindent
Secondo il modello di Bohr, gli elettroni negli atomi orbitano il nucleo secondo orbite
circolari a determinate distanze $r$. Il primo postulato di Bohr asserisce che il momento angolare
dell'elettrone è un multiplo di $\hbar$:
\begin{align}
    m_e v r = n\hbar
\end{align}
Dove $m_e$ è la massa dell'elettrone, $v$ la sua velocità orbitale, $r$ il suo 
raggio orbitale e $\hbar = \frac{h}{2\pi}$. L'interpretazione del primo postulato di Bohr
discende dalla natura ondulatoria dell'elettrone: dato che esso è un onda, esso è 
rappresentato come un onda stazionaria, dunque la lunghezza della sua orbita deve essere un multiplo 
della lunghezza di de Broglie
\begin{gather*}
    \lambda = \frac{\hbar}{q_e} = \frac{\hbar}{mv}
\end{gather*}
Dove $q_e = m_e v$. Tuttavia, secondo le leggi dell'elettrodinamica, una particella carica in orbita circolare
dovrebbe emettere radiazione elettromagnetica, perdendo dunque energia e dunque facendo decadere
la sua orbita. Dunque è necessario introdurre il secondo postulato di Bohr per poter spiegare come gli elettroni riescano 
ad orbitare l'atomo di idrogeno senza "cadere" nel nucleo. Il secondo postulato di Bohr afferma che 
un elettrone non emette alcuna radiazione elettromagnetica se si muove lungo un orbita permessa, l'emissione 
avviene dunque solamente quando vi è un salto energetico. L'energia da fornire per far saltare un elettrone 
ad un livello più alto (o quella emessa se scende ad un livello più basso) è data dalla seguente relazione
\begin{gather*}
    h\nu = E_{n_2} - E_{n_1}
\end{gather*} 
Secondo la legge di Coulomb, la forza di attrazione esercitata dal protone
sull'elettrone è la seguente
\begin{gather*}
    F = \frac{1}{4\pi\epsilon_0} \frac{e^{2}}{r_n^{2}}
\end{gather*}
Dove
\begin{itemize}
    \item $\epsilon_0 = 8.82 \cdot 10^{-12} C^{2}N^{-1}m^{-2}$: permeabilità del vuoto
    \item $e = 1.6 \cdot 10^{-19} C$: carica dell'elettrone
    \item $r_n$: distanza dell'elettrone dal nucleo.  
\end{itemize}
L'accelerazione della particella che si muove lungo l'orbita circolare è dunque
\begin{gather*}
    a = \frac{v_n^{2}}{r_n}
\end{gather*}
applicando la seconda legge di Newton si ottiene che
\begin{gather*}
    m_e \frac{v_n^{2}}{r_n^{2}} = \frac{1}{4\pi\epsilon_0} \frac{e^{2}}{r_n^{2}}
\end{gather*}
Si ottiene allora 
\begin{gather*}
    v_n = \frac{e^{2}}{4\pi\epsilon_0 \hbar n} \qquad r_n = n^2 \frac{4\pi\epsilon_0\hbar^{2}}{me^{2}}
\end{gather*}
L'energia totale di un elettrone che si trovi nell'orbita $n$ è data dalla somma della 
sua energia cinetica $T$ e potenziale $V$ e risulta dunque pari a 
\begin{gather*}
    E_n = T + V = \frac{1}{2}m_ev_n^{2}- \frac{e^{2}}{4\pi\epsilon_0r_n} = -\frac{me^{4}}{32\pi^{2}\epsilon_0^{2}\hbar^{2}n^{2}} = -k\frac{1}{n^{2}}
\end{gather*}
Dunque si esprime, con $k$ una costante, la variazione di energia per un certo livello energetico
\begin{gather*}
    h\nu = \Delta E = k\left(\frac{1}{n_1^{2}} - \frac{1}{n_2^{2}}\right)
\end{gather*} 
In termini della lunghezza d'onda
\begin{gather*}
    \frac{1}{\lambda} = \frac{\nu}{c} = \frac{k}{hc} \left(\frac{1}{n_1^{2}} - \frac{1}{n_2^{2}}\right)
\end{gather*}
$\frac{k}{hc}$ prende il nome di \textbf{costante di Rydberg} e si indica con $R = 1.097 \cdot 10^{7} m^{-1}$, facendo diventare
così
\begin{align}
    h\nu = R\left(\frac{1}{n_1^{2}} - \frac{1}{n_2^{2}}\right)
\end{align} 

\section{Leggi di Kirchhoff}
Kirchhoff formulò tre leggi che descrivono come si formano i diversi tipi di spettri in base
alle condizioni fisiche della sorgente e del mezzo attraversato dalla radiazione. 
\begin{enumerate}
    \item \textbf{Legge dello spettro continuo}: Un corpo caldo e denso emette uno spettro continuo: tutte le lunghezze
    d'onda sono presenti senza interruzioni;
    \item \textbf{Legge dello spettro a righe di emissione}: un gas caldo rarefatto emette
    radiazione solo a lunghezze d'onda caratteristiche degli elementi presenti. Questo 
    prende il nome di spettro a righe brillanti. Ogni elemento ha dunque uno spettro di emissione 
    unico. 
    \item \textbf{Legge dello spettro a righe di assorbimento}: se la radiazione di uno spettro continuo attraversa 
    un gas più freddo e rarefatto, questo assorbe fotoni a lunghezze d'onda specifiche, generando 
    dunque uno spettro continuo con righe scure. Le righe di assorbimento coincidono con quelle di emissione 
    dello stesso elemento. 
\end{enumerate}

\clearpage
\section{Allargamenti di riga}
Lo spettro di assorbimento è suscettibile a tre tipologie
di allargamento (ossia quando il gas non è perfetto le bande
di assorbimento e di emissione sono allargate):
\begin{itemize}
    \item \textbf{Allargamento naturale}
    \item \textbf{Allargamento collisionale}
    \item \textbf{Allargamento termico (o Doppler termico)}
\end{itemize}

\subsection{Allargamento naturale}
L'allargamento di riga naturale è dovuto alla meccanica quantistica e al principio di indeterminazione 
di Heisenberg. La durata finita dello stato eccitato introduce un'incertezza in energia, 
dunque quando un'elettrone eccitato decade verso uno stato fondamentale emette una radiazione 
con un profilo allargato (l'onda che viene emessa non è certa). 

\subsection{Allargamento collisionale}
È una tipologia di allargamento dovuta all'eccitazione delle molecole di un gas
non rarefatto: più è alta la sua temperatura e la sua pressione e maggiormente le
bande di assorbimento e di emissione risultano allargate a causa degli urti delle particelle. 
Gli urti, infatti, perturbano i livelli energetici ed è dunque tipico in atmosfere stellari 
dense, poiché i gas ad alte temperature aumentano la probabilità di collisione
delle loro particelle.

\subsection{Allargamento termico (o Doppler termico)}
L'origine dell'allargamento Doppler è il moto termico casuale causato dalle particelle. Gli atomi che
si muovono verso l'osservatore risulteranno avere una frequenza maggiore (spostamento verso il blu), 
mentre quelli che si avvicinano risulteranno avere una frequenza minore (spostamento verso il rosso)
L'effetto redshift si quantifica come
\begin{gather*}
    z = \frac{\lambda_{o} - \lambda_{em}}{\lambda_{em}}
\end{gather*}

\section{Spettro continuo}
Gli spettri di emissione continua si originano per effetto delle ricombinazioni 
e delle transizioni free-free. Nelle ricombinazioni gli atomi catturano gli elettroni liberi, 
le cui energie non sono quantizzate, e dunque le righe di emissione che si originano 
da queste transizioni possono avere qualsiasi lunghezza d'onda, il risultato è che si 
origina uno spettro continuo. Altrimenti si possono anche formare per gas otticamente spessi
o ad alta pressione, poiché all'aumentare della pressione gli atomi si avvicinano 
da loro e disturbano l'uno i livelli energetici dell'altro. Se la pressione è troppo alta,
le righe si sovrappongono e dunque c'è emissione di spettro continuo. Anche i solidi e liquidi
gli atomi sono ancora più vicini tra di loro rispetto ai gas e si perturbano a vicenda. 

\subsection{Corpo nero}
Un \textbf{corpo nero} è un oggetto che non riflette o diffonde la luce che vi incide, ma
la assorbe e la riemette completamente. Un corpo nero è, dunque, una sorta di radiatore perfetto
anche se, nella realtà, un radiatore perfetto non esiste, alcuni corpi hanno uno spettro di 
emissione assimilabile a quello di un corpo nero, come il Sole, le Stelle...  \\ \noindent
La radiazione dipende solamente dalla temperatura $T$ e non dal materiale di cui 
è costituito, né dalla sua forma. La distribuzione della sua radiazione segue la \textbf{legge di Planck}:
\begin{align}
    B_\nu(T) = \frac{2h\nu^{3}}{c^{2}\exp\left(\frac{h\nu}{kT}\right) - 1} \ \text{Wm}^{-2}\text{sr}^{-1}\text{Hz}^{-1}
\end{align}
Chiamato anche radiazione termica che ha la caratteristica di essere un'emissione isotropa. La funzione di Planck
si può esprimere anche in funzione di $\lambda$ 
\begin{gather*}
    -B_\nu d\lambda = B_\nu d\nu \ \Longrightarrow \ B_\lambda = -B_\nu \frac{d\nu}{d\lambda} = B_\nu \frac{c}{\lambda^{2}}
\end{gather*}
Dunque si ottiene la funzione di Planck come 
\begin{align}
    B_\lambda(T) = \frac{2hc^{2}}{\lambda^{5} \exp\left(\frac{hc}{k\lambda T}\right) - 1} \ \text{Wm}^{-2}\text{sr}^{-1}\mu\text{m}^{-1}
\end{align}
Si ottiene l'intensità totale della radiazione emessa da un corpo nero integrando su tutto il dominio delle
frequenze o delle lunghezza d'onda secondo la seguente:
\begin{align}
    B(T) = \int_{0}^{\infty } B_\nu(T) \ d\nu = \int_{0}^{\infty }B_\lambda (T) \ d\lambda 
\end{align}
Risolvendo il primo integrale
\begin{gather*}
    \int_{0}^{\infty }B_\nu(T) \ d\nu = \frac{2h}{c}\int_{0}^{\infty }\frac{\nu^{3}}{\exp\left(\frac{h\nu}{kT}\right) - 1} d\nu \quad x = \frac{h\nu}{kT} \ \Longrightarrow \ B(T) = \frac{2hk^{4}T^{4}}{c^{2}h^{4}} \int_{0}^{\infty } \frac{x^{3}}{e^{x} - 1} \ dx  = \frac{2k^{4}}{c^{2}h^{3}} \frac{\pi^{4}}{15}T^{4}
\end{gather*}
Dunque, chiamando 
\begin{gather*}
    A = \frac{2k^{4}\pi}{15c^{2}h^{3}}
\end{gather*}
Si ottiene 
\begin{align}
    B(T) = AT^{4}
\end{align}
Poiché la radiazione di emissione di un corpo nero è isotropa, allora si può esprimere la densità di flusso $F$ come
\begin{align}
    F = \pi B = \sigma T^{4}
\end{align}
che prende il nome di \textbf{legge di Stefan-Boltzmann}, dove
$\sigma = \pi A$, che prende il nome di \textbf{costante di Stefan-Boltzmann}.
\begin{figure}[h]
    \centering
    \caption{Distribuzioni di intensità di corpi neri aventi temperature di 12000K, 9000K e 6000K,
    le lunghezze d'onda si ottengono con la relazione di Wien, dato che le temperature sono in relazione
    tra loro come 4:3:2, le lunghezze d'onda sono in relazione 1:4, 1:3, 1:2 e le intensità sono 
    proporzionali a $4^{4}$, $3^{4}$, $2^{4}$.}
   \includesvg[width=1\textwidth]{temperatura.svg}
\end{figure}
Da questa si trova la relazione tra luminosità di una stella e la sua temperatura superficiale
a partire dalla luminosità della stella espressa come il flusso:
\begin{align}
    L = 4\pi R_\star^{2}F = 4\pi R_\star^{2}\sigma T^{4}
\end{align}
La temperatura che si ottiene da questa relazione prende il nome di
\textbf{temperatura efficace}. La luminosità è legata alla magnitudine bolometrica
secondo la seguente relazione
\begin{gather*}
    M_{\text{bol}} - M_{\text{bol}, \odot} = - 2.5 \log \frac{L}{L_\odot} 
\end{gather*}
Dunque, secondo la relazione sulla luminosità
\begin{align}
    M_{\text{bol}} - M_{\text{bol}, \odot} = -5\log\frac{R}{R_\odot} - 10 \log \frac{T}{T_\odot}
\end{align}
La lunghezza d'onda del massimo si può trovare uguagliando a zero la derivata di
$B_\lambda(T)$ rispetto a $\lambda$. Il risultato 
\begin{align}
    \lambda_{\text{max}} T = b
\end{align}
è nota come \textbf{Legge dello spostamento di Wien}, mentre $b$ prende il nome di
\textbf{costante dello spostamento di Wien} e ha come valore $b = 0.28978 \cdot 10^{-2}$Km.
La funzione di Planck può essere approssimata in due casi. Se $\lambda \approx \lambda_{\text{max}}$,
ossia quando
\begin{gather*}
    \frac{hc}{\lambda k T} >> 1 \ \Longrightarrow \ \exp\left(\frac{hc}{\lambda k T}\right) >> 1
\end{gather*}
e la planckiana può essere scritta in quella che viene anche chiamata \textbf{approssimazione di Wien}:
\begin{align}
    B_\lambda(T) \approx \frac{2hc^{2}}{\lambda^{5}} \exp\left(-\frac{hc}{k\lambda T}\right)
\end{align}


(TODO) (rifare il grafico)
\begin{wrapfigure}{r}{0.4\textwidth}
    \centering
    \includegraphics[width=0.35\textwidth]{PlanckWienRayleigh.png}
\end{wrapfigure}
\noindent Se invece si avesse $\lambda >> \lambda_{\text{max}}$, allora si avrebbe che 
\begin{gather*}
    \frac{hc}{\lambda k T} << 1 \ \Longrightarrow \ \exp\left(\frac{hc}{\lambda k T}\right) \approx 1 + \frac{hc}{\lambda k T}
\end{gather*}
e la planckiana può essere scritta nella forma definita come \textbf{approssimazione di Rayleigh-Jeans}:
\begin{align}
    B_\lambda(T) \approx \frac{2c^{2}kT}{\lambda^{4}}
\end{align}
L'approssimazione di Rayleigh, tuttavia, è l'unica prevista dalla fisica classica. Il problema di 
questa approssimazione è che, quando $\lambda$ tende a zero, la radiazione luminosa 
emessa aumenterebbe senza alcun limite. Talea accadimento non trovava alcun riscontro nelle osservazioni 
ed era noto come \textbf{catastrofe ultravioletta}. L'approssimazione di Wien, invece, 
risulta conforme coi dati sperimentali. 


\section{Temperature}
La temperatura in astrofisica variano da 0K alle decine di milioni di gradi. Si possono definire differenti temperature
a seconda del processo di definizione: relazioni diverse producono temperature con definizioni diverse. La temperatura
di una stella, per esempio, è ottenibile attraverso una relazione con un corpo nero ma, anche se le
stelle non sono veri e propri corpi neri, il loro spettro è approssimabile a quello di un corpo nero una volta
che le linee spettrali sono eliminate. In condizioni di \textbf{equilibrio termodinamico} tutte le temperature
hanno lo stesso valore. 

\subsection{Temperatura efficace $T_{\text{e}}$}
La \textbf{temperatura efficace} è la temperatura corrispondente, per definizione,
alla temperatura del corpo nero che irradia una quantità di energia uguale a quella della stella considerata.
Per trovare il valore di $T_{e}$, sarà dunque necessario utilizzare la legge di Stefan-Boltzmann. Se la densità di flusso
alla superficie della stella sarà pari ad $F$ dalla legge di Stefan-Boltzmann
\begin{align}
    T_e = \sqrt[4]{\frac{F}{\sigma}} 
\end{align}
Se il raggio della stella è $R_\star$ allora 
\begin{gather*}
    B(T_e) \approx B_\star(T)
\end{gather*}
Dunque, se la luminosità della stella è $L = 4\pi R_\star^{2}$ e la densità di flusso $F'$ ad 
una distanza $r$ sarà 
\begin{gather*}
    F' = \frac{L}{4\pi r^{2}} = \left(\frac{\alpha}{2}\right)^{2}\sigma T_e^{4} 
\end{gather*}
Dunque per misurare $T_e$ si devono conoscere le dimensioni della stella ed il flusso 
emesso da essa. 

\subsection{Temperatura di brillanza $T_b$}
Se si assume che ad una determinata lunghezza d'onda $\lambda$ la densità di flusso sulla superficie 
della stella $F_\lambda$ si possa ottenere dalla legge di Planck, si può derivare la \textbf{temperatura di brillanza}
$T_b$. Nel caso di emissione isotropa $F_\lambda = \pi B_\lambda(T_b)$. Se la stella ha raggio
$R_\star$ e distanza pari a $r$, allora si osserverà una densità di flusso pari a
\begin{align}
    F_\lambda' = \frac{R_\star^{2}}{r^{2}}F_\lambda
\end{align}
Anche in questo caso si deve conoscere la dimensione angolare della stella $\alpha$. Si ottiene dunque
la temperatura di brillanza prima attraverso la luminosità della stella
\begin{gather*}
    B_\lambda = \frac{2ck}{\lambda^{4}}T_b
\end{gather*}
Dunque
\begin{align}
    T_b = \left(\frac{2}{\alpha}\right)^{2} \frac{F_\lambda' \lambda^{4}}{2\pi ck}  
\end{align}

\subsection{Temperatura di colore $T_c$}
La \textbf{temperatura di colore} $T_c$ è legata alla temperatura di picco del corpo 
nero, dunque non serve conoscere le dimensioni della stella. Si misura dunque lo spettro a due 
lunghezze d'onda diverse o in due bande diverse. Assumendo ora che la distribuzione di intensità 
segue la legge di Planck, si uguaglia il rapporto tra le due densità di flusso al rapporto
delle corrispondenti relazioni di Planck:
\begin{gather*}
    \frac{F_{\lambda_1}'(T)}{F_{\lambda_2}'(T)} = \frac{B_{\lambda_1}(T)}{B_{\lambda_2}(T)} = \frac{\lambda_2^{5}}{\lambda_1^{5}} \frac{\exp\left(\frac{hc}{\lambda_2 kT}\right) - 1}{\exp\left(\frac{hc}{\lambda_1 kT}\right) - 1}
\end{gather*}
Il valore della temperatura, ossia la temperatura di colore, definisce la temperatura del colore 
nell'intervallo $\lambda_1$ e $\lambda_2$ considerato. In magnitudini
\begin{gather*}
    m_{\lambda_1} - m_{\lambda_2} = -2.5 \log \frac{F_{\lambda_1}'}{F_{\lambda_2}} + c = -2.5\log \frac{B_{\lambda_1}}{B_{\lambda_2}} + c
\end{gather*}
Se la temperatura non è troppo alta, si può utilizzare l'approssimazione di Wien:
\begin{gather*}
    m_{\lambda_1} - m_{\lambda_2} = -2.5 \log \left(\frac{\lambda_2}{\lambda_1}\right)^{5} + 2.5 \frac{hc}{kT_c}\left(\frac{1}{\lambda_1} - \frac{1}{\lambda_2}\right) \log e + c 
\end{gather*}
Dunque
\begin{align}
    m_{\lambda_1} - m_{\lambda_2} = a + \frac{b}{T_c}
\end{align}
Si evince che l'indice di colore è $\propto \frac{1}{T_c}$. 

\subsection{Temperatura cinetica $T_k$}
La \textbf{temperatura cinetica} $T_k$ è legata all'energia cinetica delle molecole
di nel gas. In particolare, nel caso di un gas ideale, la teoria cinetica dei gas consente di derivare una relazione
semplice tra l'energia cinetica del gas e la sua temperatura
\begin{gather*}
    E_k = \frac{1}{2}m\overline{v}^{2} = \frac{3}{2}kT_k
\end{gather*}
La relazione assume dunque che le particelle abbiano tutte massa $m$ e che
la velocità media sia $\overline{v}$, mentre $k$ è la costante di Boltzmann. Dunque
\begin{align}
    T_k = \frac{mv^{2}}{3k}
\end{align} 

\subsection{Temperatura di eccitazione $T_{\text{exc}}$}
La \textbf{temperatura di eccitazione} $T_{\text{exc}}$ è la temperatura ricavata 
dalla distribuzione di Boltzmann. 
\begin{align}
    \frac{n_i}{n_0} = \frac{g_i}{g_0} \exp\left(-\frac{\Delta E}{kT_\text{exc}}\right)
\end{align}
Se la distribuzione degli atomi nei diversi livelli sia dovuta esclusivamente alle
loro collisioni, allora la temperatura di eccitazione uguaglia quella cinetica.

\subsection{Temperatura di ionizzazione $T_i$}
La \textbf{temperatura di ionizzazione} $T_i$ si ottiene confrontando il numero di atomi
nei diversi stati di ionizzazione. Poiché le stelle non si comportano come veri corpi neri, i valori che
si ottengono per le temperature di eccitazione e di ionizzazione dipendono dalle righe spettrali
degli che vengono utilizzate. 


\chapter{Stelle}
\section{La classificazione stellare}
Una \textbf{stella} è un oggetto celeste costituito da gas e plasma nel quale si è raggiunto l'equilibrio tra la forza
gravitazionale e la pressione interna dovuta alle reazioni termonucleari. 
A seconda dell'emissione spettrale di una stella, è possibile classificare
le stelle in \textbf{classi spettrali}, le quali dipendono dal colore apparente
(appunto lo spettro di emissione). La seguente classificazione prende il nome
di \textbf{classificazione Harvard}.
\begin{figure}[h]
    \centering
    \includegraphics[width=0.9\textwidth]{classificazione-spettrale.png}
\end{figure}


\begin{wrapfigure}{r}{0.45\textwidth}
    \centering
    \includesvg[width=0.45\textwidth]{H-R_diagram.svg}
\end{wrapfigure}
\noindent 
Oltre alle classi O-M, esistono anche altre due classi oltre la M, ossia le classi L e T, alle quali 
appartengono le \textbf{nane brune}, che sono delle stelle deboli di bassa temperatura superficiale 
e con un colore rosso. Inoltre esistono anche le classi C e S, le quali sono stelle con temperature simili a quelle 
G-K-M ma con composizione chimica differente. \\
\noindent Oltre alla classificazione Harvard esiste anche un'altra classificazione per le stelle, 
la quale è nata per integrare il sistema Harvard aggiungendo come informazione
la \textbf{luminosità intrinseca}. Rispetto al sistema Harvard si aggiungono due dettagli fondamentali
\begin{itemize}
    \item \textbf{Tipo spettrale}: oltre alla classificazione da O a M, sono state
    introdotte delle sottoclassi numeriche che indicano la temperatura superficiale;
    \item \textbf{Classe di luminosità}: espressa in numeri romani, indica la gravità superficiale e dunque
    la fase evolutiva. Essa va da $\mathbf{I}_a$ (supergiganti brillanti), $\mathbf{I}_b$ (supergiganti), fino a $\mathbf{V}$,  ossia le stelle nane. 
\end{itemize}

\clearpage
\section{Popolazioni stellari}
Le stelle sono classificate in gruppi, chiamate \textbf{popolazioni}, basate principalmente sulla composizione chimica 
(metallicità) e sull'età. Inizialmente introdotta da Walter Bade negli anni 40, le classi principali sono ora
\begin{itemize}
    \item \textbf{Popolazione I}
    \begin{itemize}
        \item \textbf{Metallicità}: alta ($Z \geq 0.01$), ricche di elementi più pesanti dell'elio.
        \item \textbf{Età}: giovani (milioni di anni a pochi miliardi di anni).
        \item \textbf{Localizzazione}: dischi galattici, bracci a spirale e ammassi aperti.
        \item \textbf{Esempi}: Sole, stelle O e B nei bracci a spirale.
        \item \textbf{Origine}: formate da has arricchito da generazioni precedenti di stelle
    \end{itemize}
    \item \textbf{Popolazione II}
    \begin{itemize}
        \item \textbf{Metallicità}: bassa ($Z \leq 0.001$), povere di metalli (poiché si sono formate agli albori dell'universo)
        \item \textbf{Età}: molto antiche (diversi miliardi di anni)
        \item \textbf{Localizzazione}: alone galattico, ammassi globulari e bulge centrale.
        \item \textbf{Esempi}: giganti rosse negli ammassi globulari, cefeidi a corto periodo.
        \item \textbf{Origine}: nate quando l'universo era molto giovane.
    \end{itemize}
    \item \textbf{Popolazione III}: sono l'ultimo tipo di popolazione (aggiunta dopo le altre due) e per questo
    sono solo una popolazione teorica: sono state teorizzate poiché non si spiegava la metallicità delle stelle 
    di popolazione II. 
    \begin{itemize}
        \item \textbf{Metallicità}: nulla ($Z \approx 0$).
        \item \textbf{Età}: prime stelle dopo il Big Bang
        \item \textbf{Caratteristiche}: masse molto elevate, vita breve, responsabili della prima nucleosintesi di elementi pesanti
    \end{itemize}
\end{itemize}


\section{Diagramma HR}
\begin{wrapfigure}{r}{0.54\textwidth}
    \centering
    \includegraphics[width=0.54\textwidth]{HRDiagram.png}
\end{wrapfigure}
Il \textbf{diagramma Hertzsprung-Russell}, anche chiamato \textbf{diagramma HR}, è un diagramma che mette in relazione
la magnitudine assoluta delle stelle ed i tipi spettrali delle stelle. Le stelle nel diagramma non sono distribuite in modo uniforme
ma si trovano concentrate in due gruppi distinti:
\begin{itemize}
    \item \textbf{Main sequence}
    \item \textbf{Giants}
\end{itemize} 
Il ramo delle giganti si dispone in diversi gruppi: 
il ramo orizzontale, il ramo delle giganti rosse ed il ramo asintotico. Il 
ramo orizzontale è caratterizzato da stelle giganti con magnitudine 
assoluta vicino allo zero. Il ramo delle giganti rosse parte tra le stelle di 
tipo K e M, mentre il ramo asintotico parte dal ramo orizzontale per raggiungere la sommità 
del ramo delle giganti rosse. Le stelle giganti del ramo orizzontale sono circa
100 volte più luminose del Sole e, dal momento che le stelle nane e giganti dello stesso tipo
spettrale sono caratterizzate dalla stessa temperatura superficiale, la differenza di 
luminosità si ha nel raggio. Le stelle più luminose sono le \textbf{supergiganti}, alle quali 
appartiene \textbf{Betelgeuse},una stella nella costellazione di Orione che è 20000 volte più luminosa 
del Sole e avente 400 volte il raggio della nostra stella. Le nane bianche, invece, sono delle stelle 
molto deboli ma molto numerosi. \\ \noindent
Il diagramma HR è stato notevolmente ampliato grazie al lavoro del satellite Gaia,
il quale, tra il 2014 e il 2025, ha compiuto oltre 3 trilioni di osservazioni, creando uno dei cataloghi stellari
più forniti di tutti i tempi (con 580 milioni di oggetti). 

\section{Come le stelle si mantengono in vita e sprigionano energia: la catena protone-protone}
Le stelle mantengono continuamente un equilibrio tra forza di gravità e la pressione
sprigionata dal nucleo della stella. La pressione che permette alle stelle di non collassare
su sé stesse è data dalla \textbf{fusione nucleare}: ossia il processo nel quale
due atomi di idrogeno collidono tra loro. Il processo di fusione nucleare nelle stelle 
di sequenza principale prende il nome di \textbf{catena protone-protone}, che è costituita
dai seguenti passaggi:
\begin{align*}
    &(1)  &\ce{^{1}H} + \ce{^{1}H} &\to \ce{^{2}H} + \ce{e+} + \nu_e \\
    &  &\ce{^{1}H} + \ce{^{1}H} + \ce{e-} &\to \ce{^{2}H} + \nu_e \\
    &(2)  &\ce{^{2}H} + \ce{^{1}H} &\to \ce{^{3}He} + \gamma \\
    &(3)  &\ce{^{3}He} + \ce{^{3}He} &\to \ce{^{4}H} + 2\ce{^{1}H}
\end{align*}
Per ciascuna delle reazioni (3), le reazioni (1) e (2) devono compiersi due volte. 
La reazione (1) ha una probabilità molto bassa, il suo risultato è che il neutrino 
scappa velocemente dalla stella, mentre il positrone ($\ce{e+}$), incontra un elettrone e si 
annichiliscono formando due raggi gamma. La reazione sotto la (1) (che è molto simile), 
è invece più probabile e porta alla reazione (2), la quale è molto veloce. L'ultimo passo della catena
si può svolgere in tre modi, ma quello più probabile è quello riportato ed è anche
responsabile del 91\% dell'energia prodotta da una stella della sequenza principale.  

\section{Cenni di evoluzione stellare: i tempi scala}
Le stelle non sono degli oggetti fissi ed eterni nella vastità del cosmo. Esse sono infatti soggette ad una evoluzione continua 
al loro interno che prende il nome di \textbf{evoluzione stellare}. A seconda della massa della stella, la vita di essa può essere più 
o meno lunga: stelle che bruciano idrogeno molto velocemente emettendo grandi quantità di energia tendono a vivere molto meno 
di stelle che emettono poca radiazione. Esistono tre principali tempi scala evolutivi nelle stelle. 
\begin{itemize}
    \item \textbf{Tempo scala nucleare} ($t_n$);
    \item \textbf{Tempo scala termico} ($t_t$);
    \item \textbf{Tempo scala dinamico (o freefall)} ($t_d$).
\end{itemize}
Poiché si è già detto come le stelle della sequenza principale (ossia stelle simili al 
nostro Sole) producano grandi quantitativi di energia, tuttavia è possibile classificare le 
energie prodotte dalle stelle secondo tre categorie principali
\begin{itemize}
    \item \textbf{Energia gravitazionale (e termica)}: 
    \begin{gather*}
        E_{gra} \approxeq E_{term} \approxeq \frac{1}{2}\frac{GM_\star^{2}}{R_\star} \approxeq 1.9 \cdot 10^{41} \ J
    \end{gather*}
    \item \textbf{Energia chimica}: nell'ottocento si pensava che il Sole fosso composto per metà
    di carbonio e ossigeno, dunque si può calcolare l'energia chimica associata come 
    \begin{gather*}
        E_{chi} = \frac{M_\star}{28m_H} E_{CO} = \approxeq 6.4 \cdot 10^{37} \ J
    \end{gather*}
    Dove $E_{CO} = 10 \ eV$, con $eV = 1.6 \cdot 10^{-19} \ J$ e $m_H = 1.67 \cdot 10^{-27}$ kg.
\end{itemize}
Dato che il Sole è in vita da $t_\circledcirc \approxeq 4.5 \cdot 10^{9}$ anni, se si assume che abbia sempre 
avuto una luminosità costante e pari a  $L_\circledcirc = 3.8 \cdot 10^{26}$ W, si ricava che 
il sole abbia speso un energia pari a 
\begin{gather*}
    E_{TOT} = L_\circledcirc \cdot t_\circledcirc \approxeq 5.5 \cdot 10^{43} \ J
\end{gather*}
Si vanno ora ad analizzare i tempi scala dell'evoluzione stellare. \\ \noindent
\subsection{Tempo scala nucleare}
Il \textbf{tempo scala nucleare} è il tempo che occorre alla stella per emettere tutta l'energia che
è stata prodotta per mezzo delle reazioni nucleari. Il suo valore si può stimare calcolando il tempo che impiega 
la stella a trasformare tutto l'Idrogeno in Elio. Dal momento che solamente $\alpha = 0.79 \%$ della frazione di massa 
a riposo dell'idrogeno viene convertita in energia e che venga consumato almeno il 10 \% dell'idrogeno disponibile prima 
che si inneschino altre reazioni nucleari all'interno della stella. Dunque 
\begin{align}
    t_n \approx \frac{0.0079 \cdot 0.1 \cdot M_\star c^{2}}{L_\star}
\end{align}
Dove $M_\star$ è la massa della stella e $L_\star$ la sua luminosità. Il tempo 
caratteristico del Sole è di $t_{n_\circledcirc} \approx 10^{10}$ anni. Dunque il tempo scala di una 
qualsiasi stella della sequenza principale si ottiene secondo la seguente
\begin{align}
    t_n \approx \frac{ML_\circledcirc}{M_\circledcirc L} \cdot 10^{10} \ \text{anni}
\end{align}
\subsection{Tempo scala termico}
Il \textbf{tempo scala termico} è il tempo scala che occorrerebbe ad una stella per irradiare tutta la 
sua energia se la sorgente di produzione (ossia le reazioni nucleari) venisse improvvisamente spenta. Il 
teorema del Viriale stabilisce che l'energia cinetica dovuta al moto termico delle particelle è pari a 
\begin{align}
    2E_K + U = 0 \ \Longrightarrow \ t_t = \frac{\frac{1}{2}G \frac{M_\star^{2}}{R_\star}}{L_\star}
\end{align}
Dato che, per il Sole, è $t_t \approx 2 \cdot 10^{7}$ anni, si può ricavare per una stella qualsiasi secondo 
\begin{align}
    t_t = \frac{\left(\frac{M_\star}{M_\circledcirc}\right)^{2}}{\frac{R_\star}{R_\circledcirc}\cdot \frac{L_\star}{L_\circledcirc}} \cdot  2\cdot 10^{7} \ \text{anni}
\end{align}
\subsection{Tempo scala dinamico (o di freefall)}
Il \textbf{tempo scala dinamico (o di freefall)} è il tempo che impiegherebbe la stelal a collassare su
sé stessa se la pressione che la sostiene cessasse del tutto. Si può stimare calcolando il tempo che una particella 
impiegherebbe a raggiungere il centro dalla superficie e corrisponde a metà del periodo 
ottenibile secondo la terza legge di Keplero se al semiasse maggiore dell'orbita si sostituisse
mezzo raggio della stella:
\begin{gather*}
    \frac{\left(\frac{R_\star}{2}\right)^{3}}{(2 t_d)^{2}} = \frac{GM_\star}{4\pi^{2}}
\end{gather*}
Dunque si ottiene che
\begin{gather*}
    t_d = \frac{\pi}{2}\sqrt{\frac{R_\star^{3}}{2GM_\star}}  
\end{gather*}
Per il Sole, il tempo dinamico è circa 30 minuti. Dato che la massa di una stella si esprime come
\begin{gather*}
    M_\star = \frac{4\pi}{3} \rho_\star R_\star^{3}
\end{gather*}
Si può esprimere il tempo dinamico
\begin{align}
    t_d = \sqrt{\frac{3\pi}{32G\rho_\star}} 
\end{align} 

\section{Parametri che si ricavano dalle osservazioni}
Dall'osservazione di una stella si riescono a ricavare diversi parametri 
\begin{itemize}
    \item \textbf{Distanza}: metodo diretto con misura della parallasse geometrica 
    \item \textbf{Luminosità}: Nota la distanza, è possibile misurare il flusso $f$ il quale,
    dopo essere stato opportunamente corretto per l'assorbimento interstellare, permette di ricavare
    la luminosità di una stella.
    \begin{gather*}
        L  = 4\pi d^{2}f
    \end{gather*} 
    \item \textbf{Temperatura}: Determinando l'indice di colore, è possibile derivare la temperatura
    di colore secondo la seguente:
    \begin{gather*}
        B - V = a + \frac{b}{T_c}
    \end{gather*}
    Se si assume che la stella emetta come un corpo nero, allora la sua temperatura
    efficace sarà uguale a quella di colore. 
    \item \textbf{Raggio}: Con metodi interferometrici si ricava il diametro delle stelle, così 
    come con le occultazioni lunari o con le binarie ad eclisse. Nota la luminosità e la temperatura
    efficace, invece, si ricava il raggio tramite
    \begin{gather*}
        R_\star = \frac{L}{4\pi\sigma T_e^{4}}
    \end{gather*}
    \item \textbf{Massa}: la massa delle stelle si può determinare solo studiando il moto di stelle doppie interagenti
    gravitazionalmente. 
    \item \textbf{Rotazione}: la rotazione delle stelle si misura tramite l'effetto Doppler studiando 
    l'allargamento dello spettro: se una stella gira in senso antiorario, allora la parte
    destra della stella avrà uno spostamento verso il rosso mentre la parte sinistra verso il blu. 
\end{itemize}







\chapter{Distanze cosmiche}
\section{Scala delle distanze cosmiche}
La distanza degli oggetti astronomici è fra i parametri più importanti
e difficili da misurare (forse il parametro più difficile). Non esiste, inoltre, 
un unico metodo per la misura delle distanze, ma si ricorre ad una successione di tecniche
calibrate l'una sull'altra, nota come \textbf{scala delle distanze cosmiche}. Si suddividono 
i metodi per misurare le distanze in due categorie principali: 
\begin{enumerate}
    \item \textbf{Metodi diretti}: si basano sulle misure geometriche che non richiedono 
    dunque stime sulla luminosità intrinseca. Il metodo diretto per eccellenza è la 
    \textbf{parallasse trigonometrica} (o parallasse annua). Le incertezze di questi metodi 
    dipendono principalmente dalla precisione strumentale. 
    \item \textbf{Metodi indiretti}: si basano su oggetti astrofisici o relazioni calibrate. Un esempio
    sono le \textbf{candele standard}, ossia le \textbf{Cefeidi} e le \textbf{Supernovae $I_a$} e le \textbf{relazioni dinamiche}
    (Tully - Fisher, Faber-Jackson). La distanza è derivata dunque da una misura osservabile e 
    da un modello empirico o teorico che lega tale osservazione alla luminosità intrinseca. 
\end{enumerate}
Si riassumono le caratteristiche delle due metodologie nella seguente tabella
\begin{gather*}
    \begin{tabular}{l | l | l}
        & Misure dirette & Misure indirette \\
        \hline
        Tipologia di misura & geometriche & fotometriche o dinamiche \\
        Dipendenza dai modelli & minima & elevata \\
        Campo di applicazione & migliaia di parsec & intero universo \\
        Incertezze principali & errore strumentale & sistematico, legato alla calibrazione
    \end{tabular}
\end{gather*}

\section{I metodi diretti: la parallasse annua}
\begin{wrapfigure}{r}{0.5\textwidth}
    \centering
    \caption{Rappresentazione della parallasse annua}
    \begin{tikzpicture}
        \draw(0, 0) circle (1);
        \draw(0, 0) -- (0, -1) node[midway, left] {$AU$};
        \draw(0, 0) -- (5, 0) node[midway, above] {$d$};
        \filldraw[yellow](0, 0) circle (3pt);
        \draw(0, -1) -- (5, 0);
        \draw(0, 1) -- (5, 0);
        \draw[dashed](5, 0) -- (6, 0.25);
        \draw[dashed](5, 0) -- (6, -0.25);
        \filldraw[cyan] (5, 0) circle (2pt) node[anchor = south] {Stella};
        \draw(3.5, 0.3) arc (165:180:1) node[midway, left] {$p$};
        \filldraw[align = center](6, 0.25)  circle (2pt) node[anchor = south] {Posizione \\ apparente};
        \filldraw(6, -0.25) circle (2pt);
        \filldraw[cyan](0, -1) circle (1pt) node[anchor = north] {Terra};  
    \end{tikzpicture}    
\end{wrapfigure}
Il metodo della parallasse annua si basa sullo spostamento delle stelle nella volta celeste 
in riflesso allo spostamento terrestre intorno al Sole. Se le misure sono effettuate da Terra, questo 
metodo è affidabile fino a circa 30 parsec (ossia circa 100 anni luce). Se le misure sono effettuate dallo spazio, 
invece, è affidabile fino a circa 9 kiloparsec. Adesso si definisce propriamente cosa si intende 
con \textbf{parallasse}: essa è l'angolo sotteso $p$ che si riferisce alla posizione apparente di 
una stella in due punti della Terra separati distanza nota (come l'orbita
terrestre). 
\begin{gather*}
    \tan p = \frac{1 \ \text{AU}}{d} \ \Longrightarrow \ p = \arctan\left(\frac{1 \ \text{AU}}{d}\right)
\end{gather*} 
Per stelle che approcciano la distanza massima di validità di questo metodo, è possibile 
utilizzare la seguente approssimazione 
\begin{gather*}
    p \approx \frac{1 \ \text{AU}}{d}
\end{gather*}
Ottenendo dunque la distanza in unità astronomiche
\begin{align}
    d(AU) = \frac{206265}{p}
\end{align}
Dove $d$ è la distanza in unità astronomiche e $p$ è misurato in arco secondi. Si può
definire il \textbf{Parsec} come la distanza alla quale un oggetto celeste
mostra una parallasse annua di un arco secondo quando osservato
con la parallasse annua:
\begin{gather*}
    d(pc) = \frac{1}{p}
\end{gather*}
Di conseguenza il parsec è un unità di misura spaziale che vale esattamente 206265 AU e,
inoltre, misura la distanza percorsa dalla luce in $3.26$ anni.



\section{I metodi indiretti}
\subsection{Candele standard: Cefeidi}
Le cefeidi sono delle stelle pulsanti, ossia delle stelle la cui luminosità varia in modo periodico. 
Occupano una posizione specifica nel diagramma HR fuori dalla sequenza principale in una fascia che
prende il nome di \textbf{fascia di instabilità}. In questa fascia si trovano stelle 
che, a causa delle loro caratteristiche fisiche, sono soggette a pulsazioni periodiche (ossia 
variazioni periodiche della loro magnitudine). la distanza delle cefeidi si misura dal loro periodo di pulsazione ($P$) e 
dalla loro magnitudine apparente ($m_\lambda$). \\ \noindent
Inizialmente si conosce, dalle cefeidi più vicine, delle relazioni  PL calibrate che legano la magnitudine 
alla pulsazione della stella utilizzando le bande V e I (in quanto sono quelle meno 
sensibili alla polvere interstellare, così come le bande J, H, K). 
\begin{gather*}
    M_\lambda = a_\lambda \log P + b_\lambda
\end{gather*}
Dove $a_\lambda$ e $b_l\lambda$ sono i parametri ricavati dalle Cefeidi locali tramite 
satelliti come Gaia o LMC. Attraverso la misurazione di $P$ si ottiene allora $M_\lambda$. Per
trovare sia $P$ che $m_\lambda$ si seguono i seguenti passaggi
\begin{itemize}
    \item  Si acquisiscono dei dati fotometrici da cui si misura $m_\lambda$ in 
    funzione del tempo (si osserva la stella per circa 20 o 30 settimane).
    \item Si costruisce dunque la \textbf{curva di luce}, ossia $m_\lambda$ in funzione
    del tempo a partire dai dati di osservazione (cercando sempre un picco ed un minimo).
    \item Si esegue un fit dei dati per trovare il periodo e la magnitudine apparente media
    $\left< m_\lambda \right>$ (e relativa incertezza). 
    \item Si corregge l'estinzione $E(B - V)$, ottenendo una magnitudine "corretta":
    \begin{gather*}
        \left< m_\lambda\right>^{\circ}  = \left< m_\lambda \right> - R_\lambda E(B - V)
    \end{gather*}
\end{itemize} 
Adesso, da $\left< m_\lambda \right>^{\circ}$ e da $M_\lambda$ si ottiene il \textbf{modulo di distanza}:
\begin{align}
    DM = \left< m_\lambda \right>^{\circ} - M_\lambda
\end{align}
E dunque si ottiene la distanza in parsec secondo la seguente relazione
\begin{align}
    d = 10^{\frac{DM + 5}{5}}
\end{align}
Di seguito si elencano le distanze minime raggiungibili da diversi strumenti con questa metodologia:
\begin{gather*}
    \begin{tabular}{l | c | c | c}
        Strumento & $m_{\text{lim}}$ & $M_I$ tipica \\
        \hline
        Telescopio da Terra 2 - 4 m & 21-22 & -5 & $\sim$ 3 Mpc \\
        Hubble ST (ottico) & 26-27 & -5 & $\sim$ 30 Mpc \\
        JW ST (Infrarosso) & 28-29 & -5 & $\sim$ 100 Mpc
    \end{tabular}
\end{gather*}
In futuro l'Extremely Large Telescope, dovrebbe riuscire ad arrivare a magnitudini apparenti 
limiti di 30-31, con distanza massima di $\sim$ 200 Mpc. 


\subsection{Candele standard: Supernovae $I_a$ }
Le \textbf{supernovae} di tipo $I_a$ sono esplosioni termonucleari di nane bianche 
che si trovano in sistemi binari. Sono delle ottime candele standard in quanto la loro 
luminosità a picco è molto uniforme. Questa tipologia di supernovae si originano quando 
una nana bianca si accresce con i gas della stella compagna, arrivando oltre la massa limite
di $M_{\max} \approx 1.4 M_{\circ}$ (massa di Chandrasekhar). Quando la nana bianca esplode, 
libera un energia di $\approx 10^{51}$ erg. Questa tipologia di oggetti celesti
è molto più luminosa delle Cefeidi e sono visibili in galassie per cui esiste la misure di distanza
con le Cefeidi (per eseguire la calibrazione). \\ \noindent
Per poter eseguire il calcolo della distanza con questa tipologia di oggetti si utilizzano
\begin{itemize}
    \item Immagini fotometriche: si acquisiscono molte immagini (per esempio in banda B), per circa
    un mese e si costruisce una curva di luce da cui ci si calcola la magnitudine
    apparente in banda al picco ($m_B(t_\text{max})$) e il \textbf{parametro di declino} $\Delta m_B$, 
    definito come 
    \begin{gather*}
        \Delta m_B = m_B (t_\text{max} + 15 d) - m_B(t_{\max})
    \end{gather*}
    Dunque $\Delta m_B$ misura di quanto scende la magnitudine nei 15 giorni dopo al picco.
    \item Legge di Philips: una relazione empirica che lega la magnitudine assoluta al picco $M_B$ (in una data banda), 
    al parametro di declino $\Delta m_B$. Applicando questa legge si ottiene 
    \begin{gather*}
        M_B = M_B' + \gamma(\Delta m_B - 1.1)
    \end{gather*}
    Dove $M_B'$ è il \textbf{valore di calibrazione} ottenuto dalle cefeidi (zero point a circa -19.3) e $\gamma$
    indica la pendenza empirica ($\sim 1.1$ magnitudine per unità di $\Delta m_B$). Questi due parametri sono ricavati 
    da supernovae locali a distanza nota (con metodo della parallasse o cefeidi) per definire le calibrazioni. 
    La magnitudine apparente al picco $m_B(t_{\max})$ si corregge, inoltre, per l'estinzione ($E(B - V)$) e dal confronto 
    con $M_B$ si trova il $DM$:
    \begin{align}
        DM = m_B^{\text{cor}} (t_{\max}) - M_B
    \end{align}
    E da qui la distanza secondo la relazione vista prima.
\end{itemize}


\section{L'espansione dell'universo: la legge di Hubble}
La legge di Hubble è una relazione lineare fra la distanza di un dato oggetto 
celeste ed il suo redshift ($z$). Il redshift, determinato come 
\begin{gather*}
    z = \frac{\Delta \lambda}{\lambda_{em}} = \frac{\lambda_{o} - \lambda{em}}{\lambda_{em}} = \frac{\lambda_{o}}{\lambda_{em}} - 1
\end{gather*}
misura lo spostamento verso il rosso, ossia di quanto la radiazione osservata risulta spostata verso la 
banda rossa (se l'oggetto di cui stiamo facendo l'osservazione si allontana), oppure verso la banda blu
(se l'oggetto si sta avvicinando). Ricorrendo alla relazione
\begin{gather*}
    \lambda = \frac{c}{\nu} \ \Longrightarrow \ 1 + z = \frac{\nu_{em}}{\nu_{o}}
\end{gather*}
Da qui si ottiene la \textbf{Legge di Hubble}:
\begin{align}
    \nu = H_0 d
\end{align}
Dunque l'effetto Doppler 
\begin{align}
    \frac{\Delta \lambda}{\lambda_{em}} = \frac{v}{c} = z
\end{align}
Adesso, dividendo per la velocità della luce i due membri della legge di Hubble si ottiene la relazione
che lega la distanza al redshift:
\begin{align}
    d = \frac{c}{H_0}z
\end{align}
Dove $H_0 = 70$ km s$^{-1}$ Mpc$^{-1}$. 


\chapter{Stelle binarie}
\section{Stelle binarie visuali e stelle binari ottiche}
Dal momento che ciò che si vede nella volta celeste è una proiezione degli oggetti celesti, 
si potrebbe pensare che, quando due stelle sono vicine tra loro, esse siano 
delle stelle binarie, tuttavia questo non è il caso nella maggior parte delle volte. Si deve
dunque operare una distinzione tra le stelle binarie \textbf{visuali} (ossia che sono 
gravitazionalmente legate tra loro) e le binarie \textbf{ottiche}, le quali appaiano essere
delle stelle binarie, ma non sono assolutamente legate tra loro gravitazionalmente. Esistono 
anche sistemi multipli di stelle che sono gravitazionalmente legati tra loro. Le stelle singole, 
come il Sole, sono meno del 50 \% del totale dei sistemi di stelle mentre i sistemi costituiti
da almeno due stelle costituiscono la maggioranza e, solitamente, hanno una struttura gerarchica: 
una stella attorno alla quale orbita una coppia di stelle oppure due coppie di stelle. Un esempio 
è Castore e Polluce, Castore è un sistema multiplo di almeno 6 stelle e Polluce è una stella singola, 
così come Mizar e Alcor. 

\section{Le binari visuali}
Le stelle binarie visuali sono quelle stelle che presentano una separazione angolare maggiore 
di un decimo di arcosecondo (si possono dunque osservare solo con il telescopio). Il problema delle 
binarie visuali è determinare l'orbita delle stelle in quanto l'orbita del sistema è 
proiettata nel piano del cielo. Serve dunque la distanza del sistema binario e si deriva la 
massa totale del sistema secondo la terza legge di Keplero.  Si distinguono dunque due tipi di orbite
\begin{itemize}
    \item \textbf{Orbita apparente}: è l'orbita che si vede e misura dalla Terra: è dunque
    la proiezione dell'orbita vera sul fondo del cielo. Questo piano bidimensionale sul quale si ha 
    l'orbita visuale (e che è perpendicolare alla nostra linea di vista) prende il nome di \textbf{Piano del Cielo}.
    \item \textbf{Orbita reale}: è l'ellisse che la stella secondaria $M_2$ percorre fisicamente attorno alla primaria $M_1$, 
    o, più correttamente, attorno al centro di massa. Essa giace su un piano nello spazio chiamato \textbf{Piano Orbitale}. 
\end{itemize}
Il problema principale è che un'ellisse (orbita vera) proiettata su un piano inclinato 
appare comunque come un'ellisse, ma con forma e orientamento diversi da quella reale. Come 
possiamo ricostruire l'ellisse tridimensionale reale partendo dall'ellisse bidimensionale 
che si osserva? 
Per tracciare questa orbita, si raccolgono una coppia di dati $(\theta, \rho)$, che descrivono gli angoli
di posizione del sistema e le separazioni angolari della coppia, determinando $a''$ come il semiasse 
dell'orbita apparente e periodo $P$. Si può convertire in unità reali calcolando la distanza $d$ della stella nel sistema
ed il semiasse maggiore diventerà dunque 
\begin{gather*}
    [AU]a = a'' \cdot d[pc]
\end{gather*}
Se si considera $s = d \cdot \theta$, come la \textbf{distanza lineare} in unità astronomiche, 
si può determinare l'angolo in radianti a partire dall'angolo in arcosecondi come 
\begin{gather*}
    \theta(rad) = \text{arcosecondi} \cdot  \frac{\pi}{648000} 
\end{gather*}
da cui si ricava la parallasse. \\ \noindent
Per quanto riguarda la determinazione dell'orbita vera, innanzitutto si trova la massa
totale del sistema come 
\begin{align}
    M_1 + M_2 = \frac{a^{3}}{P^{2}}[M_{\circledcirc}]
\end{align}
Dove $a$ è in unità astronomiche e $P$ è in anni. Se poi si riesce a ricostruire le orbite
delle singole stelle attorno al loro centro di massa, allora
si ha che
\begin{align}
    \frac{M_1}{M_2} = \frac{a_2}{a_1}
\end{align}

\noindent
\begin{wrapfigure}{r}{0.4\textwidth}
    \centering
    \caption{Il centro di massa di stelle binarie}
    \begin{tikzpicture}
        \draw[->](0, 0) -- (2, 0) node[at end, below] {$x$};
        \draw[->](0, 0) -- (0, 2) node[at end, left] {$y$};
        \draw[->](0, 0) -- (-1 ,-1) node[at end, below]{$z$};
        \draw[->](0, 0) -- (1, 1) node[midway, above] {$\vv{r_1}$ };
        \draw[->](0, 0) -- (2, 1.2) node[midway, below] {$\vv{r_2}$ };
        \filldraw(1, 1) circle (1pt) node[anchor = south] {$M_1$};
        \filldraw(2, 1.2) circle (1pt) node[anchor = south] {$M_2$};
        \filldraw(1.5, 1.1) circle (1pt) node[anchor = south] {$P$};
    \end{tikzpicture}    
\end{wrapfigure}
Dove $a_1$ e $a_2$ sono i semiassi maggiori delle due orbite ellittiche attorno ai centri di massa. 
La definizione del centro ci massa del sistema è dato dalla seguente 
\begin{align}
    \vv{r_{CM}} = \frac{M_1 \vv{r_1} + M_2 \vv{r_2}  }{M_1 + M_2}
\end{align}
dove $\vv{r_1}$ e $\vv{r_2}$ sono i vettori posizione delle due masse e $P$ è il punto di 
equilibrio delle due masse. Spostando il centro del sistema di riferimento in $P$ si vede come 
\begin{gather*}
    M_1 r_1 = M_2 r_2 
\end{gather*}
entrambe le ellissi delle orbite delle due stelle condividono un fuoco comune (ossia il centro di massa).
Dato che la relazione sopra vale sempre, essa vale anche quando si trovano ai periapsi e all'apoapsi  (ossia 
punto più vicino e lontano, rispettivamente, di un orbita ellittica, analogo al perielio e afelio o perigeo e apogeo).
Dato che il semiasse maggiore è la somma dei semiasse $a_1$ e $a_2$, e che vale
\begin{gather*}
    M_1 r_{1, apo} = M_2 r_{2, apo} \qquad M_1 r_{1, per} = M_2 r_{2, per}
\end{gather*}
Da qui si ottiene 
\begin{gather*}
    \frac{M_1}{M_2} = \frac{a_2}{a_1}
\end{gather*}

\subsection{I nodi}
L'orbita della stella secondaria è un'ellisse chiusa. Muovendosi lungo la sua orbita, essa attraversa il piano 
del cielo in due punti specifici che, per definizione, si trovano sulla linea dei Nodi e dunque prendono il nome 
di \textbf{Nodi}:
\begin{itemize}
    \item \textbf{Nodo Ascendente} ($\Omega$): è il punto in cui la stella secondaria attraverso 
    il piano allontanandosi all'osservatore. La sua velocità radiale (componente
    $z$ della velocità) passa da negativa (verso di noi) a positiva (via da noi).
    \item \textbf{Nodo Discendente} ($\omega$): è il punto in cui la stella secondaria attraversa
    il piano del cielo avvicinandosi all'osservatore. La sua velocità radiale passa 
    da positiva a negativa. 
\end{itemize}
Si definisce allora \textbf{linea dei nodi} come l'intersezione tra il piano orbitale di un corpo ed 
il piano di riferimento che, per le binarie visuali, è il piano del cielo. Questo segmento congiunge i
nodi ascendente e discendente. La linea dei nodi è la linea di riferimento fondamentale per due dei sei elementi che 
descrivono l'orientamento dell'orbita:
\begin{enumerate}
    \item \textbf{Inclinazione}: ossia l'angolo tra il piano orbitale e il piano del cielo. Ossia 
    l'angolo tra la linea di vista e la normale al piano orbitale: 
    \begin{itemize}
        \item $i = 0$°: orbita "face-on": l'orbita vera e apparente coincidono e la linea dei nodi è indefinita
        \item $i = 90$°: orbita "edge-on": le stelle si muovono avanti e indietro. Questo è il caso limite che
        dà origine alle \textbf{binarie ad eclissi}. 
    \end{itemize}
    \item \textbf{Longitudine del Nodo Ascendente} ($\Omega$): definisce
    l'orientamento della Linea dei Nodi sul piano del cielo. È un angolo misurato sul piano del cielo a partire da
    una direzione di riferimento (il Nord celeste) e, ruotando verso Est, fino a incontrare 
    la direzione del Nodo Ascendente. 
\end{enumerate}

\section{Binarie spettroscopiche}
\begin{wrapfigure}{r}{0.5\textwidth}
    \centering
    \caption{}
    \includegraphics[width=0.5\textwidth]{binarie-spettroscopiche.png}
\end{wrapfigure}
Quando le binarie visuali sono troppo vicine per poter essere distinte, anche dai telescopi 
ottici più potenti del mondo, l'unico modo per poter distinguere tali stelle è 
utilizzare uno spettroscopio. Nel caso in cui un oggetto osservato presenti uno spettro dalle 
variazioni periodiche dovuto all'effetto Doppler, si è in presenza di un sistema binario: 
la separazione delle righe spettrali è massima quando una delle due stelle si avvicina 
all'osservatore mentre l'altra è lontana. In questi sistemi binari, il periodo di variazione 
delle righe è legato al periodo orbitale delle stelle. Non si ricavano le orbite ma la 
velocità delle due stelle in funzione del tempo. \\ \noindent
Se si assumesse delle orbite circolari, 
\begin{gather*}
    m_1 a_1 = m_2 a_2 \qquad a = a_1 + a_2 \ \Longrightarrow \ a_1 = \frac{am_2}{m_1 + m_2}
\end{gather*}
Dunque si ricava la ???????????


\section{Binarie fotometriche}
\begin{wrapfigure}{r}{0.5\textwidth}
    \centering
    \caption{I differenti tipi di binarie fotometriche}
    \includegraphics[width=0.5\textwidth]{stelle-fotometriche.png}
\end{wrapfigure}
Sono delle stelle che mostrano variazioni periodiche della loro magnitudine, 
dovute al fatto che le due stelle passano l'una di fronte all'altra note come 
\textbf{variabili ad eclisse}, dunque l'inclinazione del sistema è prossima 90°
e sono dunque le uniche stelle binarie di cui si è in grado di ricavare con certezza 
l'inclinazione e dunque misurare con precisione la massa. Si misura la 
\textbf{curva di luce}, ossia una variazione della magnitudine in funzione del tempo. 
Esistono tre tipologie di curve di luce:
\begin{itemize}
    \item \textbf{Algol} (classiche): eclissi nette, si vedono dunque due minimi 
    chiari nella curva di luce (uno più pronunciato rispetto all'altro).
    \item \textbf{$\beta$ Lyrae}: tipologia di curve di luce in cui vi è un contatto 
    leggero tra le due stelle, la forma delle due stelle diventa dunque quasi ellissoidale
    e la luce varia quasi continuamente. 
    \item \textbf{$W$ Ursa Majoris}: sistema binario a contatto in cui le curve di luce 
    sono quasi sinusoidali. 
\end{itemize}
I parametri che si possono ricavare dalla curva di luce sono i seguenti: 
\begin{itemize}
    \item \textbf{periodo orbitale}: distanza tra ripetizioni identiche nella curva;
    \item \textbf{profondità dell'ellisse primaria e secondaria}: ossia 
    $\Delta m_1$ e $\Delta m_2$ determino i raggi stellari relativi $\frac{r_1}{a}$ e $\frac{r_2}{a}$ 
    e l'inclinazione $i$. 
\end{itemize}

\chapter{Esopianeti}
\section{Sistemi planetari}
La nostra Galassia contiene $10^{11}$ stelle e circa $10^{9} \sim 10^{10}$ stelle 
potrebbero far parte di un sistema planetario.
Un sistema planetario consiste di diversi oggetti di natura non stellare in orbita attorno ad una stella o ad un sistema stellare.
Dato che la \textbf{fascia abitabile} (ossia il range di orbite che 
supporta acqua liquida e atmosfera a temperature simili a quelle della Terra)
è molto piccola, la ricerca di esopianeti (e dunque la vita) è una missione 
difficile poiché dipende dalla luminosità della stella e dal tipo spettrale della stella: 
stelle meno luminose avranno una zona abitabile molto vicina alla stella, 
mentre stelle più luminose avranno una zona abitabile molto più lontana dalla stella stessa.

\section{Esopianeti}
I pianeti che non appartengono al sistema solare, ossia che orbitano attorno a una stella diversa dal Sole,
prendono il nome di \textbf{esopianeti} o \textbf{pianeti extrasolari}. Esistono diverse tipologie 
di esopianeti a seconda delle loro caratteristiche superficiali e a seconda delle loro dimensioni:
\begin{itemize}
    \item \textbf{Hot Jupiter}: giganti gassosi molto caldi e vicini alla stella, 
    con periodo orbitali minore di 10 giorni.
    \item \textbf{Super-Terre}: pianeti rocciosi con massa compresa tra le 1 e le 10
    masse terrestri, spesso più grandi della Terra ma più piccoli di Nettuno.
    \item \textbf{Mini-Nettuni}: corpi con strati gassosi spessi attorno al nucleo roccioso o ghiacciato.
    \item \textbf{Pianeti terrestri}: analoghi alla Terra o a Marte, target primari per la ricerca di abitabilità.
    \item \textbf{Pianeti freddi e ghiacciati}: pianeti con orbite lontane, analoghi a Urano e Nettuno.
\end{itemize}
Al momento si sono riusciti ad osservare ben 5528 esopianeti confermati nella nostra galassia, di cui 
solamente 7 non sono classificati.

\section{Tecniche di rilevamento degli esopianeti}
Esistono 4 tipologie principali di rilevamento per gli esopianeti:
\begin{itemize}
    \item \textbf{Astrometria}: rilevamento dello spostamento della stella sul piano del cielo causato dal moto orbitale
    indotto dal pianeta, efficiente per sistemi in cui il pianeta è grande quasi quanto la stella in orbite 
    molto grandi. Il contro è in questi sistemi, il pianeta richiede un tempo molto lungo per completare la sua orbita.
    \item \textbf{Doppler shift}: questo metodo utilizza l'effetto Doppler nella stella ospite indotta dall'attrazione
    gravitazionale del pianeta, ossia dipende dalle velocità radiali: dunque l'entità
    dell'effetto Doppler permette di definire la massa del pianeta. Dunque questo metodo 
    è efficiente se i pianeti sono massicci ed orbitano vicino alla stella. 
    \item \textbf{Transit}: tecnica basata sul monitoraggio della diminuzione di luminosità quando 
    il pianeta passa davanti alla stella formando dunque una curva di luce. La diminuzione della luminosità della stella
    permette la determinazione del diametro del pianeta.  Quando il pianeta transita davanti alla stella,
    alcuna della luce passa attraverso l'atmosfera del pianeta, rendendo possibile
    studiare la composizione dell'atmosfera del pianeta. Quando il pianeta
    si torna dietro la stella, la sua emissione di infrarossi è completamente oscurata. Studiando 
    la riduzione della radiazione infrarossa si può determinare la temperatura del pianeta.  Questa tecnica
    è possibile applicarla solo per meno dell'1 \% dei pianeti in quanto richiede che 
    l'inclinazione dell'orbita sia quasi 90°. 
    \item \textbf{Immagine diretta}:  isolamento della luce del pianeta separandola da quella della stella 
    con tecniche di coronografia o ottica adattiva (si cerca di fare una foto al pianeta eliminando la radiazione
    della stella e cercando di isolare il pianeta).
\end{itemize}


\end{document}
\documentclass[a4paper, oneside]{article}
\usepackage{graphicx}
\usepackage{amsthm}
\usepackage{amsmath}
\usepackage{amssymb}
\usepackage[a4paper,
            bindingoffset=0.2in,
            left=2cm,
            right=2cm,
            top=2cm,
            bottom=2cm,
            footskip=.25in]{geometry}
\usepackage[italian]{babel}
\usepackage{pgfplots}
\usepackage{tabularx}
\usepackage{tikz}
\usepackage{wrapfig}
\usepackage{color}
\usepackage[d]{esvect}
\definecolor{myyellow}{RGB}{254,241,24}
\definecolor{myorange}{RGB}{234,125,1}
\definecolor{page}{rgb}{0.129,0.157,0.212}
\pagecolor{page}
\color{white}
\graphicspath{ {./images/} }
\usetikzlibrary{shapes.geometric}
\usetikzlibrary{datavisualization}
\usetikzlibrary{datavisualization.formats.functions}
\usetikzlibrary{decorations.markings} 
\usetikzlibrary{patterns}
\pgfplotsset{width=10cm,compat=1.9}

\title{Appunti}
\author{Tommaso Miliani}
\date{15-10-25}

\begin{document}
\newtheoremstyle{theoremEnv}
                {}          % Space above
                {}          % Space below
                {\slshape}  % Body font
                {}          % Indent amount
                {\bfseries} % Head font
                {.}         % Punctuation after head
                {\newline}         % Space after theorem head
                {}          % Theorem head spec
\theoremstyle{theoremEnv}

\newtheorem{definition}{Definizione}[section]
\newtheorem{theorem}{Teorema}[section]
\newtheorem{lemma}{Proposizione}[section]
\newtheorem{observation}{Osservazione}[section]
\newtheorem{corollary}{Corollario}[theorem]
\newtheorem{example}{Esempio}[section]

\maketitle

\section*{Cenni di spettroscopia}
\section{Introduzione alla spettroscopia}
Si può introdurre il concetto di spettroscopia in astronomia
senza scendere troppo nel dettaglio (in quanto richiederebbe
la fisica quantistica) a partire dall'energia degli elettroni di un
atomo:
\begin{align}
    E_f - E_o = \Delta E = h\nu_{i, f}
\end{align}
\begin{wrapfigure}{r}{0.4\textwidth}
    \centering
    \caption{}
    \begin{tikzpicture}
        \def\proton(#1,#2){%
    \fill[color=myyellow] (#1,#2) circle (5pt);
    \node at (#1,#2) {\texttt{+}};
}
\def\neutron(#1,#2){%
    \fill[color=myorange] (#1,#2) circle (5pt);
}
\def\electron(#1, #2){%
    \fill[color=cyan] (#1, #2) circle (3pt);
    \node at (#1, #2) {\texttt{-}};
}
\def\orbit(#1, #2, #3){
    \draw (#1, #2) circle (#3);
}
        \orbit(0, 0, 1);
        \orbit(0, 0, 1.5);
        \orbit(0, 0, 2);
        \neutron(0.1, -0.1);
        \neutron(-0.1, 0.1);
        \proton(0.1, 0.1);
        \proton(-0.1, -0.1);
        \electron(1, 0);
        \electron(-0.5, 0.84);
        \draw[thick, ->](-1.5, 1.2) -- (-0.6, 0.84) node[at end, above] {$\gamma$};
    \end{tikzpicture}
\end{wrapfigure}
Ossia 
\begin{gather*}
    \frac{1}{\lambda} = R\left(\frac{1}{N^{2} } - \frac{1}{h^{2} }\right)
\end{gather*}
Se si bombarda un atomo con dei fotoni con una certa energia, accade che 
gli elettroni dello \textbf{stato fondamentale} possano saltare ad un livello energetico superiore 
che prende il nome di \textbf{stato eccitato}. Se
l'energia del fotone $\gamma$ è molto maggiore dell'energia $\Delta E$ necessaria per
poter far saltare l'elettrone allo stato eccitato, allora l'elettrone viene strappato 
all'atomo e l'atomo prende il nome di \textbf{ionizzato}.
A meno che non ci sia un flusso di fotoni che tenga stabile gli elettroni
allo stato eccitato, dopo un tempo $\delta \tau = 10^{-8} \ s$, gli elettroni decadono
rilasciando indietro lo stesso pacchetto di energia $\gamma$ che li ha
investiti tornando allo stato fondamentale. 

\section{Leggi di Kirchoff}
\subsection{Prima legge}
Un gas rarefatto caldo emette uno spettro di righe
in emissione: questo vuol dire che il gas emette delle
radiazione e faccio passare questa emissione attraverso un prisma,
si osserva che sullo spettro nero si sovrappongono delle bande colorate
non continue. 

\subsection{Seconda legge di Kirchoff}
Un gas freddo posto davanti ad una sorgente emette
uno spettro di righe di assorbimento. In pratica
un gas freddo assorbe determinate lunghezze d'onda e lascia
passare le altre lunghezze d'onda.

\subsection{Terza legge di Kirchoff}
La terza legge mi dice che una sorgente ideale di luce emette uno
spettro continuo di radiazione luminosa. 

\subsection{Allargamenti di riga}
Lo spettro di assorbimento è suscettibile a tre tipologie
di allargamento (ossia quando il gas non è perfetto le bande
di assorbimento e di emissione sono allargate):
\begin{itemize}
    \item Allargamento naturale: l'allargamento di riga naturale è quello
    dovuto alla meccanica quantistica;
    \item Allargamento collisionale: è la tipologia di allargamento dovuta
    all'eccitazione del gas: più è alta la sua temperatura e la sua pressione e
    maggiormente le bande di assorbimento e di emissione risultano allargate;
    \item Allargamento termico o Doppler termico: Se la stella ruota verso
    l'osservatore, allora le bande si spostano verso il blu, mentre quando ruota
    allontanandosi dalla stella si spostano verso il rosso
    (molto similmente all'effetto doppler). Questo effetto è tanto maggiore quanto
    è vicina la stella e tanto più veloce ruota. 
\end{itemize}

\end{document}
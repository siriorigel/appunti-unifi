\documentclass[a4paper, oneside]{article}
\usepackage{graphicx}
\usepackage{amsthm}
\usepackage{amsmath}
\usepackage{amssymb}
\usepackage[a4paper,
            bindingoffset=0.2in,
            left=2cm,
            right=2cm,
            top=2cm,
            bottom=2cm,
            footskip=.25in]{geometry}
\usepackage[italian]{babel}
\usepackage{pgfplots}
\usepackage{tabularx}
\usepackage{tikz}
\usepackage{wrapfig}
\usepackage{color}
\usepackage[d]{esvect}
\definecolor{page}{rgb}{0.129,0.157,0.212}
\pagecolor{page}
\color{white}
\graphicspath{ {./images/} }
\usetikzlibrary{shapes.geometric}
\usetikzlibrary{datavisualization}
\usetikzlibrary{datavisualization.formats.functions}
\usetikzlibrary{patterns}
\pgfplotsset{width=10cm,compat=1.9}

\title{APpunti di astronomia}
\author{Tommaso Miliani}
\date{06-10-25}

\begin{document}
\newtheoremstyle{theoremEnv}
                {}          % Space above
                {}          % Space below
                {\slshape}  % Body font
                {}          % Indent amount
                {\bfseries} % Head font
                {.}         % Punctuation after head
                {\newline}         % Space after theorem head
                {}          % Theorem head spec
\theoremstyle{theoremEnv}

\newtheorem{definition}{Definizione}[section]
\newtheorem{theorem}{Teorema}[section]
\newtheorem{lemma}{Proposizione}[section]
\newtheorem{observation}{Osservazione}[section]
\newtheorem{corollary}{Corollario}[theorem]
\newtheorem{example}{Esempio}[section]

\maketitle

\section{Indice di colore}
E' una quantità generica che può essere definita a partire dalla 
differenza di qualsiasi magnitudine. In particolar modo posso utilizzare la
differenza tra la banda blu e quella verde come la 
differenza tra le magnitudini :
\begin{gather*}
    B - V = m_B - m_V = -2.5 \log \frac{F_B}{F_V} + c
\end{gather*}
Dove $c$ è una costante che dipende dal sistema fotometrico utilizzato. 
\begin{gather*}
    m_B = m_{B0} - 2.5 \log F_B + 2.5 \log F_{B0} \\
    m_V = m_{V0} - 2.5 \log F_V + 2.5 \log F_{V0}
\end{gather*}
La costante sarà allora
\begin{gather*}
    c = m_{B0} - m_{V0} + 2.5 \log \frac{F_{B0}}{F_{V0}}
\end{gather*}
Nel sistema JMK Si ha che $m_{B0} = m_{V0} = 0$ e quindi i flussi 
$F_{B0}$ e $F_{V0}$ sono dati tali per cui $B  - V = 0$ per
la stella Vega. Questa si chiama \textbf{convenzione di Vega}. La costante
prende il nome di \textbf{zero point} che cambia in base al sistema fotometrico
scelto: in questo modo se so quali sistemi utilizzo posso passare
dalle magnitudini delle stelle ai flussi.  
In questo modo si può correlare direttamente l'emissione di luce
colorata con la temperatura: nel caso dell stella Vega, l'emissione tra il blu
ed il verde è zero  allora $B - V$. In una stella blu
$V > B$, allora la banda $B$ è più luminosa in quanto è proporzionale
al logaritmo inverso: ho prevalenza di flusso blu rispetto al flusso verde:
\begin{gather*}
    B - V < 0 \ \Longrightarrow \ \text{Stella blu} \\
    B - V > 0 \ \Longrightarrow \ \text{Stella giallo-verde}
\end{gather*}
Si può allora introdurre il concetto di temperatura: più una stella emette 
in luce blu è più calda di una stella che ha emissione maggiore sulla banda verdastra.

\section{$\eta_\lambda$: L'estinzione atmosferica}
\begin{wrapfigure}{r}{0.4\textwidth}
    \centering
    \caption{La figura di Airy di profilo sul rivelatore}
    \begin{tikzpicture}
        \draw(0, -2) -- (0, 2);
        \draw(0.1, 0.5) ..controls (0.2, 0.2)  and (5.1, 0)  .. (0.1, -0.5);
    \end{tikzpicture}    
\end{wrapfigure}
Quando la luce passa nel mezzo interstellare e nell'atmosfera, la luce viene deviata
oppure perde energia; siccome la descrizione del mezzo interstellare è molto complicato,
qui ci si concentra solo sugli effetti dell'atmosfera. L'atmosfera
ci permette di osservare la luce ottica e le onde radio; tutto il resto è assorbito
o fortemente mitigato. L'atmosfera si presenta allora come un filtro che
riduce il flusso che viene dalle stelle e si comporta anche come una lente
che quindi modifica l'angolazione dei raggi luminosi. Inoltre il vento,
le nuvole etc possono distorcere le immagini astronomiche. Si
definisce allora \textbf{seeing} il parametro che quantifica la sfocatura dell'immagine
dovuta all'atmosfera; per parlare di seeing si introduce il concetto di \textbf{diffrazione}:
ossia il fenomeno per il quale le onde si propagano dietro a degli ostacoli; se la
sorgente è puntiforme allora si parla di \textbf{Figura di Airy}.
Si può immaginare la figura di Airy come una immagine tridimensionale sulle $x, y$
e sull'asse dell'intensità.  \\
La distanza tra il primo minimo ed il picco è (dove $D$ è il diametro della fenditura):
\begin{gather*}
    \Delta \theta_D = 1.22 \frac{\lambda}{D}
\end{gather*}
Ossia un allargamento angolare che è dovuto al fatto che
ho diffrazione e che ho sempre al rilevatore: il telescopio con queste
caratteristiche è l'allargamento angolare che produce il mio
telescopio per una sorgente puntiforme. Dato che l'allargamento è un
fenomeno casuale, posso determinare che abbia una distribuzione Gaussiana; posso 
allora determinare (attraverso il principio di FWHM) che
 $\Delta \theta_S$, ossia l'angolo di allargamento dovuto alla
\textbf{rifrazione} dell'atmosfera, si mette in relazione con $\Delta \theta_D$:
\begin{gather*}
    \Delta \theta_S \geq \Delta \theta_D
\end{gather*}
In buoni siti, il mio $\Delta \theta_S$ è dell'ordine di circa un arcosecondo. Per telescopi
piccoli con aperture dell'ordine di $0.1 \ m$  posso ottenere che
\begin{gather*}
    \Delta \theta_S = \Delta\theta_D = 1.22 \frac{\lambda}{D}
\end{gather*}
Posso allora definire il \textbf{parametro di Fried} che mi definisce dalla lunghezza d'onda
l'apertura del telescopio al limite dell'angolo di diffrazione:
\begin{align}
    1.22 \frac{\lambda}{\Delta \theta_S} = r_0
\end{align}
Quindi $\Delta\theta_D$ mi indica il limite teorico oltre sotto al quale io
non riesco più a distinguere due sorgenti puntiformi. Generalmente, per telescopi con
grande potere risolutivo, si può utilizzare solamente $\Delta \theta_S$ in quanto
il $\Delta \theta_D$ sarà sempre molto minore. 

\subsection{L'assorbimento}
\begin{wrapfigure}{r}{0.4\textwidth}
    \centering
    \caption{}
    \begin{tikzpicture}
        \draw(0, 0) -- (3, 0);
        \draw[|-|](-0.2, 0) -- (-0.2, 2) node[midway, left] {$H$};     
    \end{tikzpicture}    
\end{wrapfigure}
L'altro effetto dell'atmosfera è l'assorbimento dell'atmosfera. Una certa
sorgente che proviene dallo zenith giunge al rilevatore con la massima intensità
in quanto c'è molta meno atmosfera da attraversare. Per una
sorgente ad un certo angolo $z$ rispetto allo zenith,
si ha che seguono la legge di \textbf{Beer-Lambert}: 
\begin{align}
    dI_\lambda = I_\lambda K_\lambda(x)dx'
\end{align}
Dove $K_\lambda(x)$ è il coefficiente di assorbimento e dipende
dalla quota dell'atmosfera e non dall'angolo $z$. SI definisce inoltre
$dx$ la distanza tra uno strato e l'altro dell'atmosfera degli strati provenienti
dallo zenith e $dx'$ come la distanza tra uno strato e l'altro di una sorgente
proveniente da una sorgente con un certo angolo $z$ rispetto allo zenith. 
\begin{gather*}
    \int_{I_{0\lambda}}^{I_\lambda} \frac{dI_\lambda'}{I_\lambda'} = - \int_{H}^{0} K_\lambda(x)\sec z \  dx = -\int_{0}^{H}K_\lambda (x) \sec z \ dx 
\end{gather*}
Con questo integrale io posso determinare quanto si attenua il flusso
con l'atmosfera:

\end{document}
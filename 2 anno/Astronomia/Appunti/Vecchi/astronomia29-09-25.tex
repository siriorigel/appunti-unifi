\documentclass[a4paper, oneside]{article}
\usepackage{graphicx}
\usepackage{amsthm}
\usepackage{amsmath}
\usepackage{amssymb}
\usepackage[a4paper,
            bindingoffset=0.2in,
            left=2cm,
            right=2cm,
            top=2cm,
            bottom=2cm,
            footskip=.25in]{geometry}
\usepackage[italian]{babel}
\usepackage{pgfplots}
\usepackage{tabularx}
\usepackage{tikz}
\usepackage{wrapfig}
\usepackage{color}
\usepackage[d]{esvect}
\definecolor{page}{rgb}{0.129,0.157,0.212}
\pagecolor{page}
\color{white}
\graphicspath{ {./images/} }
\usetikzlibrary{shapes.geometric}
\usetikzlibrary{datavisualization}
\usetikzlibrary{datavisualization.formats.functions}
\usetikzlibrary{patterns}
\pgfplotsset{width=10cm,compat=1.9}

\title{Appunti di astronomia}
\author{Tommaso Miliani}
\date{29-09-25}

\begin{document}
\newtheoremstyle{theoremEnv}
                {}          % Space above
                {}          % Space below
                {\slshape}  % Body font
                {}          % Indent amount
                {\bfseries} % Head font
                {.}         % Punctuation after head
                {\newline}         % Space after theorem head
                {}          % Theorem head spec
\theoremstyle{theoremEnv}

\newtheorem{definition}{Definizione}[section]
\newtheorem{theorem}{Teorema}[section]
\newtheorem{lemma}{Proposizione}[section]
\newtheorem{observation}{Osservazione}[section]
\newtheorem{corollary}{Corollario}[theorem]
\newtheorem{example}{Esempio}[section]

\maketitle

\section{Esempio superficie attraversata da una intensità isotropa (DA RIVEDERE)}
\begin{wrapfigure}{r}{0.4\textwidth}
    \centering
    \caption{}
    \begin{tikzpicture}
        \draw (0, 0) circle (2);
        \draw[->](0, 0) -- (2.5, -0.5) node[at end, below] {$x$};
        \draw[->](0, 0) -- (0, 2.5) node[at end, left] {$z$};
        \draw[->](0, 0) -- (-2, -2) node[at end, below] {$y$};
        \draw(-2, 0) -- (2, 0);
        \draw[thin, dashed](0, 0) -- (0.75, 1);
        \draw[thin, dashed](0, 0) -- (1, 1.05);
    \end{tikzpicture}    
\end{wrapfigure}
Si può esprimere il flusso attraverso la sfera come:
\begin{gather*}
    F_{\nu} = I_{\nu} \oint _{4\pi} \cos\theta \ d\Omega
\end{gather*}
Possiamo considerare una superficie sferica, possiamo allora considerare
il flusso attraverso una superficie sferica molto piccola. Si considera allora il piccolo
angolo sferico 
\begin{gather*}
    d\Omega = \frac{dA\cos\theta}{r^{2} }
\end{gather*} 
Possiamo utilizzare un integrale doppio per determinare l'angolo
solido che sottende alla superficie sarà allora determinato nel seguente modo:



Dato la simmetria della sfera, se si potesse vedere tutta la sfera luminosa nella sua
interità allo stesso momento si osserverebbe che la sfera apparirebbe buia.

\section{Determinare il flusso luminoso da una stella distante da uno osservatore}
\begin{wrapfigure}{r}{0.4\textwidth}
    \centering
    \caption{}
    \begin{tikzpicture}
        \draw(0, 0) circle (1);
        \draw(0, 0) -- (0.5, 0.85) node[midway, left] {$R_{\star}$} -- (4, 0) node[midway, above] {$r$};
        \draw(0, 0) -- (4, 0) node[midway, below] {$D$};
        \draw(2, 0.4) arc (168:180:2) node[midway, left] {$\theta_r$};
    \end{tikzpicture}    
\end{wrapfigure}
Posso determinare il flusso specifico di un piccolo elemento sulla stella
come
\begin{gather*}
    d F_{\nu} =  I_{\nu} \oint_{0}^{\theta_r} \cos\theta \sin\theta \ d\theta d \phi
\end{gather*}
L'angolo $\theta$ va da $0$  a $\theta_r$, ossia l'angolo che si ha tra l'asse ottico e la congiungente
tra la parte superiore della stella e l'occhio (che è molto molto piccolo). Quindi ogni elemento
di superficie avrà un certo flusso infinitesimo, dato che $\sin\theta_r = \frac{R_{\star}}{D}$ poiché
mi sono messo nelle condizioni nelle quali l'angolo tra $R_{\star}$ e $D$ è $90$°,  data
la simmetria della sfera, posso dire che
\begin{gather*}
    F_{\nu} = 2\pi I_{\nu} \oint_{0}^{\theta_r} \cos\theta \sin\theta \ d\theta
\end{gather*}
Risolvendo questo integrale si ottiene la seguente conclusione:
\begin{gather*}
    F_{\nu} = 2I_{\nu}\pi \int_{0}^{\frac{R_{\star}}{D}}\sin\theta d\sin\theta = \pi I_{\nu}\left(\frac{R_{\star}}{D}\right)^{2} 
\end{gather*}

\section{Luminosità}
La Luminosità nel sistema internazionale ha come unità di grandezza $W \cdot hz^{-1}$ rispetto ad
una certa superficie scelta $A$, misura il l'integrale di flusso:
\begin{align}
    L_{\nu} = \int_{A} F_{\nu} \ dA 
\end{align}
Se la superficie è sferica si ottiene la Luminosità emessa da una sfera
sulla superficie della sfera stessa di raggio $R_{\star}$:
\begin{align}
    L_{\nu} = 4\pi R_{\star}^{2}F_{\nu} 
\end{align}
Se volessi trovare il flusso ad una certa distanza $D$ dalla superficie della sfera come
\begin{align}
    f_{\nu} = \pi I_{\nu}\left(\frac{R_{\star}}{D}\right)^{2} 
\end{align}


\end{document}
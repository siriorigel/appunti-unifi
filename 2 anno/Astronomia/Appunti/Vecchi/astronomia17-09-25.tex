\documentclass[a4paper, oneside]{article}
\usepackage{graphicx}
\usepackage{amsthm}
\usepackage{amsmath}
\usepackage{amssymb}
\usepackage[a4paper,
            bindingoffset=0.2in,
            left=2cm,
            right=2cm,
            top=2cm,
            bottom=2cm,
            footskip=.25in]{geometry}
\usepackage[italian]{babel}
\usepackage{pgfplots}
\usepackage{tabularx}
\usepackage{tikz}
\usepackage{wrapfig}
\usepackage{color}
\definecolor{page}{rgb}{0.129,0.157,0.212}
\pagecolor{page}
\color{white}
\graphicspath{ {./images/} }
\usetikzlibrary{shapes.geometric}
\usetikzlibrary{datavisualization}
\usetikzlibrary{datavisualization.formats.functions}
\usetikzlibrary{calc}
\pgfplotsset{width=10cm,compat=1.9}

\title{Appunti astronomia}
\author{Tommaso Miliani}
\date{17-09-25}

\begin{document}
\newtheoremstyle{theoremEnv}
                {}          % Space above
                {}          % Space below
                {\slshape}  % Body font
                {}          % Indent amount
                {\bfseries} % Head font
                {.}         % Punctuation after head
                {\newline}         % Space after theorem head
                {}          % Theorem head spec
\theoremstyle{theoremEnv}

\newtheorem{definition}{Definizione}[section]
\newtheorem{theorem}{Teorema}[section]
\newtheorem{lemma}{Proposizione}[section]
\newtheorem{observation}{Osservazione}[section]
\newtheorem{corollary}{Corollario}[theorem]
\newtheorem{example}{Esempio}[section]

\maketitle

\section{Sfera celeste}
\subsection{Il cielo come proiezione}
Gli astri delle costellazioni e gli altri corpi hanno tutti distanze diverse ma
a noi appaiono proiettati su di una superficie: dalle tre coordinate per individuare un astro ne bastano
quindi due. La Terra, ruotando sul proprio asse, fa si che ruoti il cielo intorno
e quindi le stelle visibili di notte: tutti i pianeti e le stelle ruotano attorno al polo
Nord celeste, che non è altro che il prolungamento del polo nord terrestre sulla sfera celeste.
Dato che si tratta di una superficie sferica, otteniamo le coordinate degli oggetti
attraverso gli angoli: servono allora due cerchi per determinare da dove considerare gli
angoli.\\
Tutti gli astri sorgono da Est e tramontano ad Ovest: tuttavia con la latitudine può cambiare
l'osservazione dei corpi celesti: in particolar modo cambia l'altezza del Sole. Si può tuttavia utilizzare come riferimento
la stella Polare in quanto il polo Nord celeste è molto vicino a quella stella; si utilizza anche
la proiezione dell'equatore sulla sfera celeste chiamato \textbf{equatore celeste}. \\
Si hanno dunque due tipi di sistemi di coordinate astronomici: sistemi locali e assoluti.
Per passare ad un sistema di coordinate assoluti io utilizzo l'equatore celeste o l'eclittica in quanto non 
dipende dalla latitudine; si utilizza anche il sistema delle stelle fisse che ci permette
di ottenere le coordinate di un astro rispetto ad una stella fissa nel cielo (ossia una stella il cui moto proprio
è veramente molto piccolo). 

\subsection{Sistemi di coordinate}
\subsubsection{Sistema orizzontale}
Il sistema di coordinate più immediato e quello più naturale è il sistema di coordinate locale
che ci consente di ottenere le coordinate di un oggetto attraverso due coordinate:
\begin{itemize}
    \item Azimut: coordinate rispetto all'asse Sud-Nord: ossia la differenza
    angolare rispetto al Nord ($0 \sim 360$°)
    \item Altezza: coordinata rispetto all'orizzonte ($-90 \sim 90$°).
\end{itemize}

\subsubsection{Sistema equatoriale relativo}
Si utilizza l'inclinazione dell'asse terrestre costante rispetto alle
stelle fisse, utilizzando il meridiano locale. E' un sistema di coordinate
solidale con la Terra e ci permette di definire due coordinate (è ancora locale perché dipende da dove
sono sulla Terra):
\begin{itemize}
    \item Declinazione $\delta$: Ossia l'angolo tra l'oggetto e l'equatore celeste che diventa il
    cerchio fondamentale della sfera celeste;
    \item Angolo orario $h$: si misura in ore e ci indica proprio lo spostamento rispetto al
     meridiano  locale celeste (ottenuto con il prolungamento del meridiano sulla sfera celeste) .
\end{itemize}

\subsubsection{Sistema equatoriale assoluto}
Questo sistema di coordinate è definito a partire da due coordinate
ma a differenza di quello relativo, questo è fisso rispetto alle stelle fisse e
permette di avere le stesse coordinate per un oggetto a prescindere
dalla posizione sulla superficie terrestre.
\begin{itemize}
    \item Declinazione $\delta$: Ossia la differenza angolare rispetto al cerchio fondamentale
    (ossia l'equatore celeste);
    \item Ascensione retta $\alpha$: ossia la declinazione angolare rispetto al meridiano di Greenwich 
    sulla sfera celeste. 
\end{itemize}
Si può ottenere il \textbf{tempo siderale locale}: ossia il il tempo siderale è il tempo che
impiega la Terra a compiere un giro completo rispetto alle stelle (che è circa 3 min e 56 secondi più veloce
del giorno solare a cui siamo abituati).
\begin{align}
    \Theta = h_{\gamma} = h + \alpha
\end{align}
Con questo, possiamo ottenere le coordinate equatoriali assolute
conoscendo il nostro angolo orario e viceversa. 

\subsubsection{Sistema eclitticale}
In questo sistema di coordinate il cerchio fondamentale è l'Eclittica, ossia
il piano dell'orbita terrestre ed il cerchio di riferimento è il cerchio 
meridiano passante per i poli e i punti equinoziali. Come coordinate si utilizzano
le seguenti:
\begin{itemize}
  \item \textbf{Latitudine eclittica} $\beta$: ossia l'angolo rispetto all'eclittica;
  \item \textbf{Longitudine eclittica} $\gamma$: ossia la longitudine rispetto al punto di contatto tra l'equatore
  celeste e l'eclittica (ossia rispetto all'equinozio di primavera). 
\end{itemize}

\subsubsection{Il sistema galattico}
Per studiare corpi nella galassia si utilizzano le coordinate locali rispetto 
al Sole e come piano fondamentale quello galattico e il piano rispetto al quale
determino la longitudine galattica utilizzo la congiungente Sole Centro della Galassia.


\section{Trigonometria sferica}
\subsection{Triangoli sferici e prime definizioni della trigonometria sferica}
IMMGINE GROSSA
Un triangolo sferico $ABC$ è formato da tre archi appartenenti a
tre cerchi massimi. Se $r$ è il raggio della sfera, allora si ha che 
l'arco $|AB| = r\theta$.  Possiamo trovare l'area del triangolo sferico 
che è proporzionale al suo eccesso sferico che è sempre diverso da zero
ed è diverso per ogni triangolo. L'\textbf{eccesso sferico} è la differenza tra
la somma dei tre angoli interni del triangolo e l'angolo piatto. Se le
ampiezze degli angoli sono $\alpha, \beta, \gamma$, allora l'eccesso sferico sarà
\begin{align}
    E = \alpha + \beta + \gamma - \pi
\end{align}
L'area di iun triangolo sferico è allora
\begin{align}
    A = Er^{2} 
\end{align}

\subsection{Coordinate sferiche e trasformazioni}
IMMAGINE
Se ruotiamo due sistemi cartesiani di un certo angolo $\chi$, mantenendo invariata
la $x$ in modo tale che $x = x'$.  
\begin{gather*}
    \left\{\begin{array}{l}
        x' = x \\
        y' = y\cos \chi + z\sin\chi \\
        z' = -y\sin\chi + z\cos\chi
    \end{array}\right.
\end{gather*}
Possiamo allora passare alle coordinate sferiche nel seguente modo:
\begin{align*}
    x &= \cos\psi\cos\theta  &x' &= \cos\psi'\cos\theta' \\
    y &= \sin\psi\cos\theta  &y' &= \sin\psi'\cos\theta' \\
    z &= \sin\theta &z' &= \sin\theta'
\end{align*}
A questo punto si possono ottenere le seguenti relazioni risolvendo utilizzando le
relazioni che si sono viste fino ad ora
\begin{align*}
    \cos\phi'\cos\theta &= \cos\psi\cos\theta \\
    \sin\psi'\cos\phi' &= \sin\psi\cos\theta\cos\chi + \sin\theta\sin\chi \\
    \sin\theta' &= -\sin\psi\cos\theta\sin\chi + \sin\theta\cos\psi
\end{align*}


IMMAGINE
Si possono allora applicare le coordinate nelle espressioni
del triangolo sferico. Considerando il seguente sistema di riferimento 
nell'immagine, posso trovare le seguenti relazioni per gli angoli
$\psi, \theta, \chi$:
\begin{gather*}
    \psi = \alpha - 90\text{°} \qquad \theta = 90\text{°} - b  \qquad \chi = c \\
    \psi' = 90\text{°} - \beta \qquad \theta ' = 90 \text{°} - a
\end{gather*}
Si ottiene allora dalla trigonometria (nell'assunzione secondo la quale il raggio della sfera sia molto 
grande):
\begin{align*}
    \sin \beta \sin a &= \sin \alpha \sin b \\
    \cos \beta \sin a &= -\cos \alpha \sin b \cos c + \cos b \sin c \\
    \cos a &= \cos \alpha \sin b \sin c + \cos b \cos c 
\end{align*}
Da qui posso utilizzare la regola sei seni per ottenere le seguenti relazioni
\begin{gather*}
    \frac{\sin a}{\sin \alpha} = \frac{\sin b}{\sin B} = \frac{\sin c}{\sin \gamma}
\end{gather*}

\subsection{Il triangolo nautico : da equatoriale relativo a orizzontale}
IMMAGINE

Il triangolo nautico serve per trasformare le coordinate
equatoriali relative nel sistema orizzontale. Oppure si possono utilizzare lòe
trasformazioni inclinando di un angolo negativo : 
qui $S$ equatoriale, $S$' è il sistema orizzontale. 
\begin{gather*}
    \psi = 90\text{°} - h  \qquad \theta = \delta \\
    \psi' = 90\text{°} - \alpha \qquad \theta' = a \\
    \chi = -(90\text{°} - \phi)
\end{gather*}
Impostati ora gli angoli , posso ricavare mediante il sistema di riferimento
delle coordinate sferiche mediante le seguenti relazioni:
\begin{align*}
    \sin\alpha\cos a &= \sin h \cos \delta \\
    \cos\alpha \cos a &= \cos h \cos \delta \sin \phi - \sin \delta \cos \phi \\
    \sin a &= \cos h \cos \delta \cos \phi + \sin \delta \sin \phi
\end{align*}


\subsection{Il tempo siderale: da equatoriale assoluto a relativo}
IMMAGINE
Per trasformare le coordinate equatoriali assolute, ossia la posizione
di un astro dal catalogo o dalle effemeridi. La posizione da catalogo indica la
posizione di uh astro secondo i cataloghi stellari ad una certa ora
e giorno dell'anno mentre la posizione di un astro dalle effemeridi è la posizione
di un astro tramite il giorno giuliano e il tempo siderale alla mezzanotte di
Greenwich. Nel sistema orizzontale serve il tempo siderale
(TS locale) ovvero l'angolo orario del punto $\gamma$:
\begin{align}
    \Theta = h_{\gamma} = h + \alpha
\end{align}
Puntando verso un astro di riferimento con un telescopio
dotato di montatura equatoriale posso leggere $h$ sul disco orario e ricavare
allora $TS$ in quell'istante. Altrimenti servirebbe un orologio specifico più rapido di 
circa 3min e 56 secondi rispetto a quelli standard. 

\subsection {Culminazione, levata e tramonto degli astri}
Ci poniamo al meridiano di Greenwich (per cui $h = \alpha = 0$). L'altezza massima 
raggiunta da un astro si troverà come
\begin{align}
    a_{max} = 90\text{°} - \phi + \delta \ \Longrightarrow \ z_{min} = \phi - \delta
\end{align}
L'altezza massima è anche chiamate culminazione superiore o transito di un astro; un
astro risulta \textbf{circumpolare} (ossia rimane sempre sopra l'orizzonte) per
una data latitudine se
\begin{gather*}
    \delta > 90\text{°} - \phi
\end{gather*}
Sempre nella terza relazione possiamo porre $a = 0$ l'angolo orario
di levata e tramonto di un astro si ricava da:
\begin{align}
    \cos h = -\tan\delta \tan\phi
\end{align}

\subsection{Variazioni di coordinate}
Esistono molti motivi per cui le stelle
 possono cambiare nel tempo le loro
 coordinate. Il primo è il loro moto
 proprio, dato che su tempi lunghi non
 esistono vere stelle fisse. Scomponendo
 la velocità in tangenziale e radiale, solo
 la velocità tangenziale porta a variazioni
 di coordinate celesti, mentre la componente
 radiale è misurabile tramite effetto
 Doppler (più preciso). Il satellite Gaia
 sta mappando un miliardo di stelle con
 precisione di 20 micro secondi d'arco
 (distanze e moti propri). \\
 L'asse terrestre in realtà non è fisso: a
 causa della non-sfericità della Terra si ha
 un moto di precessione con periodo di 26
 mila anni. Il punto d'Ariete si sposta in
 modo retrogrado sull'eclittica, con
 incremento delle longitudini galattiche di
 50''/anno (precessione degli equinozi).
 Differenziando le trasformazioni tra i due
 sistemi eclitticale ed equatoriale assoluto:
 \begin{gather*}
     d\delta = d\gamma \sin\epsilon\cos\alpha \\
     d\alpha = d\gamma(\sin\epsilon\sin\alpha\tan\delta + \cos\epsilon)
 \end{gather*}
 Ulteriori correzioni che coinvolgono pure
 l'angolo di inclinazione sono dovute alla
 Luna (nutazione, periodo di 18.6 anni), di
 entità minore e spesso trascurate.
 Il moto di rivoluzione terrestre
 provoca l'aberrazione (velocità
 finita della luce, max 21'', solo
 0.3'' per la rotazione) e per stelle
 vicine la parallasse annua (se la
 distanza è un parsec vale 1'').
 C'è poi la rifrazione (per $a\lessapprox10$°).




\section{Misura del tempo}
\subsection{Gnomoni meridiane e orologi solari}
Il moto diurno del Sole è lo strumento più ovvio per poter misurare il tempo.
Si utilizzava ai tempi lo \textbf{gnomone}, ossia la punta di un palo che proietta la sua 
ombra su di una \textbf{meridiana}, la quale, al mezzogiorno, individua esattamente
l'asse Nord-Sud; le ombre che proietta sono diverse per ogni stagione. Si usa invece
l'orologio solare che è una meridiana che presenta anche le ore sopra di essa. \\
Il giorno è definito come due passaggi successivi in meridiana da parte del Sole; tuttavia
via mentre la terra gira la Terra sta anche girando intorno al Sole: essendo il giorno definito come
un moto di 24h, il giorno siderale è sempre minore del giorno solare. Se definisco
$P$ il tempo assoluto di un anno di $P = 365.2564$ e il giorno normale come $\tau = 1$ e
$\tau_s$ la durata del giorno siderale rispetto a quello normale, ottengo il numero di giorni
in più come
\begin{gather*}
    \frac{P}{\tau_s} - \frac{P}{\tau} = I
\end{gather*}
Nell'uso civile ci si rifà alle 24h anche se il giorno in realtà ha
durata variabile durante i mesi dell'anno. Si può trovare la durata del giorno effettivo
come la differenza tra l'ora solare vera e l'ora solare media:
\begin{gather*}
    E.T. = T - T_M
\end{gather*}
Se ci cercasse di scattare delle foto del Sole alla stessa ora solare media della culminazione del
Sole con una macchina fissa, sovrapponendo le immagini di tutto l'anno si ottiene una visualizzazione
delle differenze della durata del giorno ed il vero moto del Sole nel cielo chiamato \textbf{Analemma solare}.

\subsection{Tempo universale e fusi orari}
Dato che il Sole non può sorgere in tutti i punti della Terra
allo stesso momento, si definisce un tempo universale
rispetto al meridiano di Greenwich e si utilizzano i fusi orari
(essenzialmente degli sfasamenti rispetto all'orario standard) 
per ogni zona della terra. Esistono ben 24 fusi orari e ogni paese decide quale 
utilizzare per determinare l'ora. \\
Essendo che la rotazione della Terra non è uniforme, non posso definire il secondo come
$1/86400$ del giorno solare, come era definito una volta, ma si è passati a
definire il secondo secondo degli orologi atomici che correggono gli
effetti della relatività generale basandosi sulla variazione energetica degli
elettroni nell'atomo di Cesio-133. Anche la definizione di anno non è universale in quanto
nel corso della storia si sono susseguiti diversi calendari, ognuno ottenuto attraverso
metodi di calcolo dei giorni diversi tra di loro. Giulio Cesare impose che
l'anno sarebbe durato $365.25$ giorni, ma rispetto al Calendario Gregoriano si è arrivati
ad uno sfasamento di 10 giorni nel 1582; per ovviare al problema, i bisestili non erano stati
più contati ogni 4 anni ma in modo tale da avere 97 bisestili ogni 400
anni invece che 100.
Si può stimare allora il tempo siderale conoscendo i giorni che mancano all'equinozio di primavera
e aggiungendo dodici ore in quanto si misurano da SUD
\begin{align}
    \Theta \approx T + 12h + n \cdot 4min
\end{align}

\end{document}
\documentclass[a4paper, oneside]{article}
\usepackage{graphicx}
\usepackage{amsthm}
\usepackage{amsmath}
\usepackage{amssymb}
\usepackage[a4paper,
            bindingoffset=0.2in,
            left=2cm,
            right=2cm,
            top=2cm,
            bottom=2cm,
            footskip=.25in]{geometry}
\usepackage[italian]{babel}
\usepackage{pgfplots}
\usepackage{tabularx}
\usepackage{tikz}
\usepackage{wrapfig}
\usepackage{color}
\definecolor{page}{rgb}{0.129,0.157,0.212}
\pagecolor{page}
\color{white}
\graphicspath{ {./images/} }
\usetikzlibrary{shapes.geometric}
\usetikzlibrary{datavisualization}
\usetikzlibrary{datavisualization.formats.functions}
\pgfplotsset{width=10cm,compat=1.9}

\title{Appunti astronomia}
\author{Tommaso Miliani}
\date{15-09-25}

\begin{document}
\newtheoremstyle{theoremEnv}
                {}          % Space above
                {}          % Space below
                {\slshape}  % Body font
                {}          % Indent amount
                {\bfseries} % Head font
                {.}         % Punctuation after head
                {\newline}         % Space after theorem head
                {}          % Theorem head spec
\theoremstyle{theoremEnv}

\newtheorem{definition}{Definizione}[section]
\newtheorem{theorem}{Teorema}[section]
\newtheorem{lemma}{Proposizione}[section]
\newtheorem{observation}{Osservazione}[section]
\newtheorem{corollary}{Corollario}[theorem]
\newtheorem{example}{Esempio}[section]

\maketitle
Il corso si divide in tre sezioni:
\begin{enumerate}
    \item Astronomia sferica: ossia l'astronomia celeste, cenni storici,
    trigonometria sferica;
    \item L'indagine astronomica: ossia l'utilizzo di strumenti come telescopi
    per compiere misurazioni ed osservazioni sulle stelle; la determinazione dell'età
    delle stelle e l'osservazione stellare in generale;
    \item Gravitazione e meccanica celeste: campo e potenziale celeste con
    legge di Gauss, problema dei due e tre corpi, fluidi auto gravitanti (fluidi + gravità),
    struttura stellare.
\end{enumerate}
L'esame consiste in un colloquio orale diviso in tre domande,
una per parte del corso.

\section{Introduzione al corso}



\end{document}
\documentclass[a4paper, oneside]{article}
\usepackage{graphicx}
\usepackage{amsthm}
\usepackage{amsmath}
\usepackage{amssymb}
\usepackage[a4paper,
            bindingoffset=0.2in,
            left=2cm,
            right=2cm,
            top=2cm,
            bottom=2cm,
            footskip=.25in]{geometry}
\usepackage[italian]{babel}
\usepackage{pgfplots}
\usepackage{tabularx}
\usepackage{tikz}
\usepackage{wrapfig}
\usepackage{color}
\usepackage[d]{esvect}
\definecolor{page}{rgb}{0.129,0.157,0.212}
\pagecolor{page}
\color{white}
\graphicspath{ {./images/} }
\usetikzlibrary{shapes.geometric}
\usetikzlibrary{datavisualization}
\usetikzlibrary{datavisualization.formats.functions}
\usetikzlibrary{patterns}
\pgfplotsset{width=10cm,compat=1.9}

\title{Astro}
\author{Tommaso Miliani}
\date{22-10-25}

\begin{document}
\newtheoremstyle{theoremEnv}
                {}          % Space above
                {}          % Space below
                {\slshape}  % Body font
                {}          % Indent amount
                {\bfseries} % Head font
                {.}         % Punctuation after head
                {\newline}         % Space after theorem head
                {}          % Theorem head spec
\theoremstyle{theoremEnv}

\newtheorem{definition}{Definizione}[section]
\newtheorem{theorem}{Teorema}[section]
\newtheorem{lemma}{Proposizione}[section]
\newtheorem{observation}{Osservazione}[section]
\newtheorem{corollary}{Corollario}[theorem]
\newtheorem{example}{Esempio}[section]

\maketitle

\section{Diagramma HR e tipologia di stelle}
\subsection{Gaia}
\subsection{catena protone protone nelle stelle main sequence}
\subsection{esercizio per casa}
Se ho 10000 stelle con massa stellare metà di qeulla del sole, 
in questo ammasso c'è una sola stella che è 
\begin{gather*}
    M_{\star} = 2 M_{\circ}
\end{gather*}
Quale è il colore della stella che sto osservando ?
Di che colore è la stella con massa 10 volte la massa
solare nello stesso ammasso?

\section{Distanze stellari}
\subsection{Scala delle distanze cosmiche}
\subsection{Metodi diretti: metodi geometrici, la parallasse annua}
\begin{wrapfigure}{r}{0.45\textwidth}
    \centering
    \caption{}
    \begin{tikzpicture}
        \draw(0, 0) circle (1);
        \draw(0, 0) -- (0, -1) node[midway, left] {$AU$};
        \draw(0, 0) -- (5, 0) node[midway, above] {$d$};
        \filldraw[yellow](0, 0) circle (3pt);
        \draw(0, -1) -- (5, 0);
        \draw(0, 1) -- (5, 0);
        \draw[dashed](5, 0) -- (6, 0.25);
        \draw[dashed](5, 0) -- (6, -0.25);
        \filldraw[cyan] (5, 0) circle (2pt);
        \draw(3.5, 0.3) arc (165:180:1) node[midway, left] {$p$};
        \filldraw(6, 0.25)  circle (2pt);
        \filldraw(6, -0.25) circle (2pt);
    \end{tikzpicture}    
\end{wrapfigure}
La parallasse trigonometrica è il primo metodo geometrico e si basa
sullo spostamento angolare di un oggetto celeste sulla sfera celeste quando 
questo oggetto è osservato a distanza di 6 mesi. Si può allora ottenere la distanza
$d$ di un corpo celeste in funzione dell'angolo dell'osservazione $p$ e della distanza
Terra-Sole (ossia l'unità astronomica). L'angolo $p$ è misurato in arco secondi.
\begin{gather*}
    \tan p = \frac{1 \ AU}{d} \ \Longrightarrow \ p \approx \frac{1 \ AU}{d}
\end{gather*} 
Allora si ottiene la distanza come
\begin{align}
    d(AU) = \frac{206265}{p}
\end{align}
Dove $d$ è la distanza in unità astronomiche e $p$ è misurato in arco secondi. Si può
definire il \textbf{Parsec} come la distanza alla quale un oggetto celeste
mostra una parallasse annua di un arco seconodo quando osservato
con la parallasse annua:
\begin{gather*}
    d(pc) = \frac{1}{p}
\end{gather*}
La distanza di un Parsec è esattamente
\begin{gather*}
    1 \ pc = 206265 \ AU
\end{gather*}
In questo modo un parsec è la distanza percorsa dalla luce in $3.26$ anni.  \\
Più la stella è vicina e più gli angoli della parallasse annua sono grandi ma
più che la stella si allontana più che si incontra un limite di precisione fisico
dello strumento utilizzato per l'osservazione della parallasse.
Il limite della parallasse è 
\begin{gather*}
    \text{Terra } \to 30 \ pc \\
    \text{Gaia } \to \ \sim 9 kpc
\end{gather*}
I metodi diretti dunque non sono più utilizzabili per determinare le
distanze dei corpi celesti con distanze superiori a 9 kiloparsec.

\subsection{Metodi indiretti}
\subsection{Cefeidi}

\end{document}
\documentclass[a4paper, oneside]{article}
\usepackage{graphicx}
\usepackage{amsthm}
\usepackage{amsmath}
\usepackage{amssymb}
\usepackage[a4paper,
            bindingoffset=0.2in,
            left=2cm,
            right=2cm,
            top=2cm,
            bottom=2cm,
            footskip=.25in]{geometry}
\usepackage[italian]{babel}
\usepackage{pgfplots}
\usepackage{tabularx}
\usepackage{tikz}
\usepackage{wrapfig}
\usepackage{color}
\definecolor{page}{rgb}{0.129,0.157,0.212}
\pagecolor{page}
\color{white}
\graphicspath{ {./images/} }
\usetikzlibrary{shapes.geometric}
\usetikzlibrary{datavisualization}
\usetikzlibrary{datavisualization.formats.functions}
\pgfplotsset{width=10cm,compat=1.9}

\title{Appunti di astronomia}
\author{Tommaso Miliani}
\date{22-09-25}

\begin{document}
\newtheoremstyle{theoremEnv}
                {}          % Space above
                {}          % Space below
                {\slshape}  % Body font
                {}          % Indent amount
                {\bfseries} % Head font
                {.}         % Punctuation after head
                {\newline}         % Space after theorem head
                {}          % Theorem head spec
\theoremstyle{theoremEnv}

\newtheorem{definition}{Definizione}[section]
\newtheorem{theorem}{Teorema}[section]
\newtheorem{lemma}{Proposizione}[section]
\newtheorem{observation}{Osservazione}[section]
\newtheorem{corollary}{Corollario}[theorem]
\newtheorem{example}{Esempio}[section]

\maketitle

\section{I telescopi}
L'astrofisico utilizza il telescopio per poter osservare la radiazione
luminosa proveniente dagli oggetti celesti. Ci sono diverse tipologie di
telescopi (rifrattori e riflettori) che vengono utilizzati 
ognuno nel loro campo di applicazione.  Galileo Galilei fu il primo ad
utilizzare e realizzare telescopi per l'osservazione della sfera celeste. 

\subsection{I parametri fondamentali dei telescopi}
\begin{itemize}
    \item \textbf{Apertura} ($D$): ossia il diametro dell'obbiettivo (anche chiamato 
    specchio primario nei riflettori), che determina la raccolta di luce proporzionale a $D^{2}$.
    \item \textbf{Lunghezza focale} ($f$): la distanza tra l'obbiettivo ed il piano focale
    e definisce insieme all'oculare l'ingrandimento del telescopio.
    \item \textbf{Rapporto focale} ($F = \frac{f}{D}$): determina la velocità di raccogliere i fotoni,
    ossia se ha un $f$ number (ossia il reciproco di $F = \frac{f}{D}$) molto alto allora 
    è un telescopio che raccoglie meno luce ma con ingrandimenti maggiori. Se invece si ha un $F$ number
    basso allora avrà campi più ampi e tempi di esposizione più brevi ma con meno ingrandimenti. 
    \item \textbf{Tempo di esposizione}: Non è propriamente una caratteristica
    dei telescopi; il tempo di esposizione è il tempo per cui si mantiene aperto l'obbiettivo
    sulla sorgente in modo da poter raccogliere quanta più luce possibile
    per avere immagini più luminose.
    \item \textbf{Oculare}: E' una lente che sta dopo il punto focale della lente primaria
    e ha la focale sull'occhio dell'osservatore. Oculari diversi hanno ingrandimenti diversi. 
    \item \textbf{Campo di vista} (FOV): e' inversamente proporzionale alla lunghezza focale a
    parità di oculare.
    \item \textbf{Ingrandimento} (M): è dato dal rapporto tra $f$ e $f_{oculare}$; dove
    l'$f$ dell'oculare non è una grandezza propria del telescopio ma dell'oculare che si sceglie. 
    E' logico quindi pensare che più è "lungo" l'oculare e più ingrandimenti ottengo.
    \item \textbf{Scala spaziale}: Quando si osserva il cielo si osserva una proiezione di una porzione
    del cielo sul piano dell'immagine. Questa grandezza mi lega le dimensioni degli angoli reali a quella
    del piano dell'immagine. 
    \begin{gather*}
        \tan \theta = \frac{D}{f} \ \Longrightarrow \ \theta \approx \frac{D}{f}
    \end{gather*} 
    Si può convertire questa a arcosecondi  (ossia secondi di arco ossia $''$) definiti come
    $1 \ rad = 180 \cdot  3600 \ arcsec$. Posso quindi esprimere l'angolo in radiante e quindi il plate scale come
    \begin{gather*}
        \theta \approx \frac{D}{f} \cdot  180 \cdot  3600 \ \Longrightarrow \ \theta = \frac{180\cdot 3600}{F}
    \end{gather*}
    \item \textbf{Potere risolutivo}: E' l'angolo che mi dice ad una data lunghezza d'onda
    e ad una data apertura quanta è la dimensione angolare di una sorgente puntiforme: più è grande l'angolo
    di risoluzione e più risulta grande una sorgente puntiforme nell'immagine. Con la seguente si
    riesce a determinare il limite di Rayleigh, ossia il limite oltre al quale due sorgenti puntiformi 
    sono indistinguibili l'una dall'altra:
    \begin{gather*}
        \theta_r \approx 1.22 \frac{\lambda}{D}\text{°} \ \vee \ \theta_r \approx 2.5 \cdot  10^{5} \frac{\lambda}{D} '' 
    \end{gather*}.
\end{itemize}

\subsection{Aberrazioni}
Esistono vari tipi di aberrazioni (ossia dei difetti dei un sistema ottico) che causa la deviazione
dei raggi luminosi impedendo la formazione di una immagine perfetta e puntiforme di un oggetto puntiforme,
sfocatura o distorsione dell'immagine finale. Le aberrazioni sono una conseguenza intrinseca della natura ondukatoria
e del comportamento della luce. 


\section{I vari tipi di telescopi}
\subsection{Il telescopio Galileiano}
Il telescopio Galileiano è un telescopio semplice che combina degli 
obbiettivi convergenti ed un oculare divergente ed un percorso ottico
corto ed una immagine dritta. Si vede che non è utilizzabile in quanto il problema delle lenti
è che sono estremamente fragili e dato che il vetro è un fluido estremamente viscoso
allora è molto difficile anche produrre lenti con lunghezze focali molto piccole e diametri grandi.



\end{document}
\documentclass[a4paper, oneside]{article}
\usepackage{graphicx}
\usepackage{amsthm}
\usepackage{amsmath}
\usepackage{amssymb}
\usepackage[a4paper,
            bindingoffset=0.2in,
            left=2cm,
            right=2cm,
            top=2cm,
            bottom=2cm,
            footskip=.25in]{geometry}
\usepackage[italian]{babel}
\usepackage{pgfplots}
\usepackage{tabularx}
\usepackage{tikz}
\usepackage{wrapfig}
\usepackage{color}
\usepackage[d]{esvect}
\definecolor{page}{rgb}{0.129,0.157,0.212}
\pagecolor{page}
\color{white}
\graphicspath{ {./images/} }
\usetikzlibrary{shapes.geometric}
\usetikzlibrary{datavisualization}
\usetikzlibrary{datavisualization.formats.functions}
\usetikzlibrary{patterns}
\pgfplotsset{width=10cm,compat=1.9}

\title{Appunti di astronomia}
\author{Tommaso Miliani}
\date{24-09-25}

\begin{document}
\newtheoremstyle{theoremEnv}
                {}          % Space above
                {}          % Space below
                {\slshape}  % Body font
                {}          % Indent amount
                {\bfseries} % Head font
                {.}         % Punctuation after head
                {\newline}         % Space after theorem head
                {}          % Theorem head spec
\theoremstyle{theoremEnv}

\newtheorem{definition}{Definizione}[section]
\newtheorem{theorem}{Teorema}[section]
\newtheorem{lemma}{Proposizione}[section]
\newtheorem{observation}{Osservazione}[section]
\newtheorem{corollary}{Corollario}[theorem]
\newtheorem{example}{Esempio}[section]

\maketitle

\section{Unità fondamentali: gli angoli}
\subsection{Rifrattori: il telescopio Kepleriano}
Il telescopio Kepleriano è un telescopio rifrattore: ha un obbiettivo convergente
con lente positiva ed un oculare convergente ed un percorso ottico più lungo con
il fuoco primario all'interno del tuvo oculare e produce un immagine capovolta anche se ha un campo visivo più ampio e luminoso
con un rapporto focale tra $\frac{f}{5}$ e $ \frac{f}{15}$.

\section{Aberrazioni}
Le aberrazioni sono le distorsioni dell'immagine (come sfuocature) 
che sono dovuti a difetti della lente oppure dell'intero apparato. 
Le aberrazioni monocromatiche sono dovute alla geometria del sistema ottico
e al fatto che i raggi non sono parassiali (cioè quelli lontani dall'asse ottico)  e vengono quindi
focalizzati in punti diversi. Questo tipo di aberrazione è studiata dalla teoria di Seidel che descrive matematicamente
con equazioni differenziali di terzo ordine estenendo l'approssimaziona parassiale.

\subsection{Aberrazione sferica}
I raggi paralleli all'asse ottico che passano attraverso le zone periferiche 
di una lente sferica convergono in un punto focale diverso da quello sull'asse
ottico: questa aberrazione fa si che il puntino di luce non appare più come un puntino
di luce ma una palla di luce sfuocata. 

\subsection{Coma}
La coma è una aberrazione che si manifesta per oggetti fuori dall'asse ottico: l'immagine dei queste sorgenti 
non vengono messi a fuoco nello stesso punto e qunque si ha la formazione
di una coda. E' causata dalla differenza di ingrandimento per i raggi che attraversano zone diverse
della lente apparendo dunque sfuocata. 

\end{document}
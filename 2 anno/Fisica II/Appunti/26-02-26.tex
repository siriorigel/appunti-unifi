\documentclass[a4paper, oneside]{article}
\usepackage{graphicx}
\usepackage{amsthm}
\usepackage{amsmath}
\usepackage{amssymb}
\usepackage[a4paper,
            bindingoffset=0.2in,
            left=2cm,
            right=2cm,
            top=2cm,
            bottom=2cm,
            footskip=.25in]{geometry}
\usepackage[italian]{babel}
\usepackage{pgfplots}
\usepackage{tabularx}
\usepackage{tikz}
\usepackage{wrapfig}
\usepackage{color}
\usepackage[d]{esvect}
\definecolor{page}{rgb}{0.129,0.157,0.212}
\pagecolor{page}
\color{white}
\graphicspath{ {./images/} }
\usetikzlibrary{shapes.geometric}
\usetikzlibrary{datavisualization}
\usetikzlibrary{datavisualization.formats.functions}
\usetikzlibrary{patterns}
\pgfplotsset{width=10cm,compat=1.18}

\title{Appunti di Fisica II}
\author{Tommaso Miliani}
\date{26-02-26}

\begin{document}
\newtheoremstyle{theoremEnv}
                {}          % Space above
                {}          % Space below
                {\slshape}  % Body font
                {}          % Indent amount
                {\bfseries} % Head font
                {.}         % Punctuation after head
                {\newline}  % Space after theorem head
                {}          % Theorem head spec
\theoremstyle{theoremEnv}

\newtheorem{definition}{Definizione}[section]
\newtheorem{theorem}{Teorema}[section]
\newtheorem{lemma}{Proposizione}[section]
\newtheorem{observation}{Osservazione}[section]
\newtheorem{corollary}{Corollario}[theorem]
\newtheorem{example}{\underline{Esempio}}[section]
\newtheorem{remark}{Enunciato}[section]

\maketitle

\section{La prima legge di Maxwell}
Data una superficie $S$ chiusa, si fa l'integrale 
\begin{gather*}
    \int_{S}^{}  \vv{E} \cdot \hat{n} \ dS = \frac{Q_{int}}{\epsilon_0} 
\end{gather*}
Si integra su tutto il dominio dei punti che appartengono alla superficie $S$,
una superficie però definisce anche un volume $\tau$, si può dunque 
riscrivere l'integrale come 
\begin{gather*}
    \int_{\tau} \frac{\rho(\vv{r} )}{\epsilon_0} \ d\tau = \frac{Q_{int}}{\epsilon_0}
\end{gather*}
Si assume che all'interno del volume si possa definire una densità per volume 
di carica (per ogni volumetto $d\tau$). Se c'è carica, allora
\begin{gather*}
    dQ = \rho\ d\tau
\end{gather*}
Se invece non c'è carica allora è nulla poiché la densità di carica è nulla (ovviamente).
Secondo il teorema della divergenza (le condizioni valgono sempre), si ha che
\begin{gather*}
    \int_{S} \vv{E} \cdot \hat{n} \ dS = \int_{\tau} \vv{\nabla} \cdot \vv{E}  = \int_{\tau} \frac{\rho (\vv{r} )}{\epsilon_0} \ d\tau
\end{gather*}
SI può dire che $S$ è al bordo del volume $\tau$. Queste due uguaglianze dicono che si hanno due 
integrali che si eguagliano (gli ultimi due) e queste relazioni valgono per qualunque volumetto $d\tau$ scelto 
(perché vale per qualsiasi $dS$ scelto), questo porta a dire che la funzione integranda è 
uguale:
\begin{align}
        \vv{\nabla} \cdot \vv{E} = \frac{\rho}{\epsilon_0}  
\end{align}
Ossia la \textbf{Prima equazione di Maxwell}. Comunque sono distribuite le cariche nell'universo,
ola divergenza vale solo per la carica considerata localmente (in quanto è stata ricavata da considerazioni 
locali).

\subsection{Coordinate cilindriche e notazione di Einstein}
In coordinate cartesiane l'equazione di Maxwell diventa
\begin{gather*}
    \frac{\partial E_x}{\partial x} + \frac{\partial E_y}{\partial y} + \frac{\partial E_z}{\partial z} = \frac{\rho}{\epsilon_0}   
\end{gather*}
Si utilizza in questo corso la notazione di Einstein per indicare la divergenza, i prodotti scalari e 
prodotti vettoriali associando le coordinate spaziali a numeri: $x = 1$, $y = 2$, $z = 3$:
\begin{gather*}
    \sum_{i = 1}^{n} \frac{\partial E_i}{\partial i} = \frac{\rho}{\epsilon_0} \ \Longrightarrow \ \frac{\partial E_i}{\partial i} = \frac{\rho}{\epsilon_0}   
\end{gather*}
La somma è omessa per semplicità e velocità di scrittura (l'indice ripetuto per due quantità vettoriali diverse sottintende 
la sommatoria).

\begin{wrapfigure}{r}{0.4\textwidth}
    \centering
    \caption{}
    \begin{tikzpicture}
        \draw[->](0, 0) -- (2, 0) node[at end, below] {$\hat{y} $};
        \draw[->](0, 0) -- (-1, -0.75) node[at end, left] {$\hat{x}$ };
        \draw[->](0, 0) -- (0, 2.5) node[at end, left] {$\hat{z}$ };
        \draw(0, 0) -- (1, -1);
        \draw(1, -1) -- (1, 1);
        \filldraw(1, 1) circle(1pt) node[anchor = west] {$P$};
        \draw(1, 1) -- (0, 2);
        \draw(-0.5, -0.375) arc (-130:-67:1) node[midway, below] {$\phi$};
    \end{tikzpicture}    
\end{wrapfigure}
\noindent
Le coordinate cilindriche sono ricavate a partire da quelle cartesiane 
nella seguente maniera: dato un punto $P$ sul piano cartesiano $P:(x, y)$, 
si indica l'angolo che il vettore $P$ ha rispetto all'asse $x$ con $\phi$,
dunque 
\begin{gather*}
    \begin{array}{l}
        x = r\cos\phi \\
        y = r\sin\phi
    \end{array} \quad 
    \begin{array}{l}
        r = \sqrt{x^{2} + y^{2}} \\
        \tan\phi = \frac{y}{x} 
    \end{array}
\end{gather*}
In un sistema tridimensionale si associano comunque le coordinate $\phi$ ed $r$ nella 
medesima maniera, tuttavia si utilizza la coordinata $z$ per ottenere l'altezza (come nel disegno):
\begin{gather*}
    \begin{array}{l}
        x = r\cos\phi \\
        y = r\sin\phi \\
        z = z
    \end{array}
\end{gather*}
Si devono introdurre ora i \textbf{versori curvilinei}:
$\hat{r} $ e $\hat{\phi}$ descrive l'incremento angolare ed è normale a
$\hat{r}$. Si ha allora che
\begin{gather*}
    \begin{array}{l}
        \hat{r} = \hat{x}\cos\phi  + \hat{y}\sin\phi \\
        \hat{\phi} = -\hat{x}\sin\phi + \hat{y}\cos\phi   \\
        \hat{z} = \hat{z}  
    \end{array}
\end{gather*}
Si può provare che rispettano le regole del prodotto scalare e
del prodotto vettoriale. I versori polari, quando si sposta 
il punto, cambiano direzione e dunque dipendono dal tempo. Questa è 
una conseguenza dell'operatore nabla: le coordinate $x, y, z$ hanno le
stesse unità di misura mentre le coordinate polari no, dunque si deve 
considerare l'incremento di lunghezza quando ci si muove lungo lungo $\hat{\phi}$.
In coordinate cartesiane si sarebbe scritto
\begin{gather*}
    \vv{dl} = dx \hat{x} + dy \hat{y}   
\end{gather*} 
In coordinate cilindriche, lo spostamento di un certo $\vv{dl}$ si esprime 
come 
\begin{gather*}
    \vv{dl} = dr \hat{r} + rd\theta \hat{\theta}   
\end{gather*}
Quando si fa la divergenza si deriva per $\theta$ e si deve tenere conto
del fatto che cambia anche la direzione di $\hat{\theta}$ quando ci si sposta. 
\begin{example}[L'esempio del filo con le coordinate cilindriche]
    L'esempio del filo infinito può essere espresso mediante le coordinate cilindriche:
    \begin{gather*}
        \vv{E}(r, \phi, z) = E_r(r, \phi, z)\hat{r} + E_\phi(r, \phi, z) \hat{\phi} + E_z(r, \phi, z) \hat{z}   
    \end{gather*}
    Dunque potrei, data l'invarianza traslazionale rispetto alla $z$ e 
    la simmetria cilindrica, il campo non potrà dipendere né da $z$, né 
    dall'angolo $\phi$ rispetto all'asse $\hat{x}$ dal quale 
    lo vedo (invarianza rotazionale). Il campo non può essere 
    diretto lungo l'asse $z$ poiché non può dipendere né se ci si 
    sposta verso l'alto né se ci si sposta verso il basso: dunque
    è impossibile avere una dipendenza da $z$ se c'è invarianza traslazionale, 
    poiché altrimenti l'alto del filo ed il basso del filo sarebbero diversi tra loro,
    anche se, per le ipotesi del problema, il filo è infinito e ha distribuzione omogenea e 
    lineare di carica. Inoltre non può dipendere da $\phi$ poiché comunque lo si guardi 
    il filo, per lo stesso ragionamento fatto per l'asse $z$, non cambia se $\phi = 0$°, 
    $\phi = 90$°. Dunque il campo si esprime come 
    \begin{gather*}
        \vv{E}(x, \phi, z) = \frac{\lambda}{2\pi\epsilon_0r} \hat{r}  
    \end{gather*}
\end{example}

\begin{example}
    Supponendo di avere un filo infinito di spessore trascurabile e una densità lineare 
    di carica ben definita $\lambda$. Si può utilizzare il teorema di Gauss 
    per determinare il campo elettrico locale. Si può applicare il teorema di Gauss solo 
    se si riesce a trovare un campo elettrico costante o nullo su quella superficie e, sfruttando
    le simmetrie, si riduce la complessità del problema ad un solo grado di libertà. Se l'esercizio
    è a simmetria cilindrica, l'unica variabile che deve trovare è solo la componente radiale del campo 
    sapendo che dipende solo dalla distanza. A quel punto si deve trovare una superficie 
    per cui si ha una relazione locale (le superfici sono dei cilindri concentrici al 
    filo con altezza $h$ e raggio $r$). Il flusso attraverso questo cilindro è
    \begin{gather*}
        \Phi_{S}(\vv{E} ) = \Psi_{S_1}(\vv{E} ) + \Psi_{S_2}(\vv{E} ) +\Psi_{S_L}(\vv{E} )
    \end{gather*}
    Il primo contributo è zero poiché è ortogonale al campo, così come il secondo poiché le loro normali
    sono lungo $\hat{r}$ e $\hat{r}$ mentre il terzo contributo ha la normale lungo $\hat{z}$. Dunque 
    \begin{gather*}
        \Phi_{S}(\vv{E}  ) = \int_{S_L} \vv{E} \cdot \hat{r} \ dS   
    \end{gather*}   
    QUesto integrale, per calcolarlo, si prende un elemento di superficie piccolo in modo tale che 
    si possa risolvere l'integrale doppio:
    \begin{gather*}
        \int_{S} E_r(r) \ dS = \int_{z}^{z + h}\int_{0}^{2\pi}   E_r r\ d\theta \ dz = 2E(r)r\pi  h
    \end{gather*}
    Siccome $E(r)$ è costante, allora si poteva anche non fare l'integrale. 
    L'altro pezzo del teorema di Gauss è 
    \begin{gather*}
        \frac{Q_{int}}{\epsilon_0} = \int_{z}^{z + h} \lambda \ dz' = \frac{\lambda h}{\epsilon_0}
    \end{gather*}
    La carica è ovviamente definita dalla densità di superficie del filo. Per il teorema di Gauss si equagliano 
    \begin{gather*}
        \frac{\lambda h}{\epsilon_0} = 2E(r)r\pi h
    \end{gather*}
    Dunque si ottiene 
    \begin{gather*}
        \vv{E} = \frac{\lambda}{2\pi\epsilon_0 r}\hat{r}  
    \end{gather*}
\end{example}

\begin{example}[Il filo con sezione non nulla]
    Nel caso in cui si avesse un filo infinito con sezione non 
    nulla con raggio $R$ e densità di carica $\rho$, si deve anche determinare
    i contributi dentro al filo. Invece di usare il principio di sovrapposizione 
    direttamente porterebbe a grandi conti, dunque si utilizza Gauss in quanto si sfrutta 
    la simmetria (anche in questo caso è invariante lungo la traslazione rispetto a $z$ e
    rispetto alla rotazione intorno al filo). Dunque anche in questo caso il campo elettrico 
    dipenderà solamente da $\hat{r}$:
    \begin{gather*}
        \vv{E}(r, \phi, z) = E_r(r)\hat{r}  
    \end{gather*} 
    Si deve però fare attenzione ad utilizzare il teorema di Gauss dentro e fuori dal filo
    poiché hanno due contributi differenti e distinti. Utilizzando il teorema di Gauss:
    \begin{gather*}
        \int_{S} \vv{E} \cdot \hat{n} \ dS = \frac{1}{\epsilon_0} \int_{\tau} \rho(r) \ d\tau    
    \end{gather*}
    Nel teorema di Gauss $S = 2\tau$, nei fini pratici in questo problema, se il cilindro è più grande del filo
    considerato, la $\rho$ è dentro al filo, dunque operativamente i due integrali sono diversi. Se invece il 
    cilindro è esattamente quello del filo, allora la $\rho$ è su tutto il cilindro considerato. Per $r < R$
    allora si ha che
    \begin{gather*}
        \int_{S} \vv{E} \cdot \hat{n} \ dS =   2E_r\pi r h = \frac{1}{\epsilon_0} \int_{\tau} \rho(r) \ d\tau = \frac{1}{\epsilon_0} \rho \pi r^{2} h
    \end{gather*}
    Ottenendo dunque 
    \begin{gather*}
        \vv{E} = \frac{\rho}{2\epsilon_0} r \hat{r}  
    \end{gather*}
    Dentro il filo il campo cresce linearmente fino al bordo del filo. $\rho$ non dipende 
    più dal raggio del filo. \\
    Nel caso in cui $r > R$ il primo integrale è uguale a quello prima, ma il 
    secondo integrale si ferma a $R$, dunque conta solo il volume in cui c'è la carica dentro.
    \begin{gather*}
        2E_r\pi r h = \frac{1}{\epsilon_0}  \rho\pi R^{2} h
    \end{gather*}
    Ottenendo allora 
    \begin{gather*}
        \vv{E} = \frac{\rho R^{2}}{2\epsilon_0 r} \hat{r}  
    \end{gather*}
    Evidenziando dunque una dipendenza da $\frac{1}{r}$ quando si esce dal filo. 
    Questo ragionamento è applicabile anche se $\rho(r)$ è una funzione del raggio
    (ma non dipende da $\theta$ e dunque ha sempre simmetria cilindrica): va dunque 
    fatto l'integrale. Un esempio è il caso in cui il filo abbia un foro concentrico 
    (sempre cilindrico) al suo interno. In quel caso, se il raggio è $R_2$ e il raggio del buco $R_1$,
    nel caso in cui $r < R_1$ il campo è nullo. \\ \noindent
    Invece, se si considera $ R_1 < r < R_2$, la carica contenuta 
    in questo volume è dato dalla seguente:
    \begin{gather*}
        2E_r\pi r h = \frac{\rho}{\epsilon_0} (\pi r^{2} - \pi R_1^{2}) h 
    \end{gather*} 
    Ossia si elimina il contributo del volume del buco. \\ \noindent
    Se si prende ora $r > R_2$:
    \begin{gather*}
        2E_r\pi r h = \frac{\rho}{\epsilon_0} (\pi R_2^{2} - \pi R_1^{2})h
    \end{gather*}
    Dunque il grafico è zero fino a $R_1$, dopo va come un termine 
    che cresce meno di un termine lineare e poi decade come $\frac{1}{r}$. 
    Con il principio di sovrapposizione si deve considerare il segno 
    della densità di carica, ottenendo che per il cilindro più interno si avrebbe 
    sia un termine $\rho$ e $-\rho$, la soluzione che si è ricavato è dunque 
    la somma delle soluzioni fino a $R_1$ poiché entrambi i campi crescono linearmente. Tra
    $R_1$ e $R_2$ il termine negativo decresce come $\frac{1}{r}$ mentre l'altro cresce
    come $r$. Quando si esce dal cilindro, invece, entrambi vanno come $\frac{1}{r}$.  
\end{example} 

\begin{example}[Filo pieno con densità $\rho$ risolto con Maxwell]
    Nel caso di un filo con densità di carica $\rho$, si può ricavare con la divergenza 
    il campo elettrico. Dato che $\rho(r)$, allora il problema è facilmente risolubile se 
    si utilizza la divergenza in coordinate cilindriche. 
    \begin{gather*}
        \vv{\nabla} \cdot \vv{E} = \frac{\rho(r)}{\epsilon_0} \ \Longrightarrow \ \frac{1}{r} \frac{\partial rE_r}{\partial r} + \frac{1}{r} \frac{\partial E_\phi}{\partial \phi} + \frac{\partial E_z}{\partial z}     
    \end{gather*}
    Con questa formulazione, si è scomposto in coordinate cilindriche. DUnque non c'è componente $\phi$ e $z$
    per via della simmetria del problema, dunque 
    \begin{gather*}
        \vv{\nabla} \cdot \vv{E} = \frac{1}{r} \frac{\partial rE_r}{\partial r}   
    \end{gather*} 
    Si divide dunque una regione dello spazio in cui $r < R$ e una regione in cui $r > R$. $r < R$:
    \begin{gather*}
        \int_{0}^{r} \frac{\partial rE_r}{\partial r} = \int_{0}^{r} \frac{\rho r}{r} \ dr ??? = \left.rE_r\right|_{0}^{r}  = \left.\frac{\rho}{\epsilon_0} \frac{r^{2}}{2}    \right|_{0}^{r} = rE_r(r) = \frac{\rho}{2\epsilon_0}r^{2}
    \end{gather*}
    Si ottiene allora che, dentro al filo
    \begin{gather*}
        \vv{E}(r) = \frac{\rho}{2\epsilon_0} r\hat{r}  
    \end{gather*}
    Per $r > R$:
    \begin{gather*}
        \frac{1}{r}\frac{\partial rE_r}{\partial r} = 0 
    \end{gather*}
    poiché fuori dal filo $\rho(r)  =0$. Integrando da entrambe le parti:
    \begin{gather*}
        rE_r = \text{const} \ \Longrightarrow \ E_r = \frac{c}{r}
    \end{gather*}
    Ossia una costante che non si conosce. Quello che si è scritto è che il campo è continuo 
    rispetto ad una superficie data (ma non è sempre così: se sulla superficie esiste una densità
    superficiale di carica allora non è continuo). Nel caso in cui $r = R$  si determina la costante come
    \begin{gather*}
        E(r) = \frac{\rho R}{2\epsilon_0} \qquad E_r(r) = \frac{c}{R} \ \Longrightarrow \ c = \frac{\rho R^{2}}{2\epsilon_0}
    \end{gather*}
    Se la funzione è continua vale l'ultima uguaglianza. Dunque si ottiene che, per $r > R$:
    \begin{gather*}
        \vv{E} = \frac{\rho R^{2}}{2\epsilon_0} \frac{1}{r}\hat{r}  
    \end{gather*}
    Se il filo perde spessore si può moltiplicare e divide per $\pi $ e si ottiene 
    $\lambda = \rho \pi R^{2}$. 
\end{example}

\section{La densità superficiale di carica}
Supponiamo di avere una lastra di materiale carica con densità di carica
\begin{gather*}
    \rho(x, y, z) = g(x, y) + f(z) 
\end{gather*}
Matematicamente la dipendenza tra $x, y$ fattorizza: la dipendenza 
da $z$ cambia ma la forma funzionale rimane la stessa. Si suppone adesso
che 
\begin{gather*}
    f(z) = \begin{array}{l}
        \frac{1}{h} \qquad \left| z \right| < \frac{h}{2} \\
        0 \qquad \left| z \right| > \frac{h}{2} 
    \end{array}
\end{gather*}
La carica lungo $z$ è distribuita in modo uniforme dentro 
la lastra. Dividendo la funzione per $\frac{h}{h}$, si chiama
\begin{gather*}
    \rho(x, y, z) = \sigma(x, y) \cdot \Theta(z) \qquad \sigma(x, y) = g(x, y) h
\end{gather*}
Se $h \to 0$ questa funzione tende verso l'infinito. Questa funzione $\Theta(z)$ è
una funzione ben definita e dunque nell'intervallo $-\frac{h}{2}$ e $\frac{h}{2}$ 
si può definire il seguente integrale:
\begin{gather*}
    \int_{-\frac{h}{2}}^{\frac{h}{2}} \Theta(z) \ dz = 1 
\end{gather*}
Questo integrale fa sempre uni qualunque sia $h$ perché si integra in un intervallo in cui la 
funzione è sempre $\frac{1}{h}$, dunque è l'area del rettangolo $h \cdot \frac{1}{h}$. 
Questo integrale include il caso in cui $h \to 0$. 
Si definisce una distribuzione chiamata Delta di Dirak:
\begin{gather*}
    \delta(z) = \lim_{h \to 0}\Theta(z) 
\end{gather*}
È una funzione che è zero ovunque eccetto in un punto in cui vale 
$\infty$. Si definisce questa funzione senza sapere come si costruisce. 
Più in generale si può dare una definizione di funzione (o distribuzione)
tale che
\begin{gather*}
    \delta(z) = \left\{\begin{array}{l}
        \infty \qquad z = 0 \\
        0 \qquad z \neq 0
    \end{array}\right.
\end{gather*}
Questo infinito è un tipo di infinito integrabile tale per cui:
\begin{gather*}
    \int_{-1}^{1} \delta (z) \ dz = 1 
\end{gather*}
Questo perché gli infiniti hanno gerarchie e questo particolare infinito 
è tale per cui è integrabile e si definisce un'altra proprietà:
\begin{gather*}
    \int_{-\infty }^{\infty } \eta(z) \delta (z - z_0)  \ dz 
\end{gather*}
Questa ha la proprietà tale per cui è uguale a $\eta(z_0)$. Queste tre proprietà
definiscono in modo più generale la distribuzione delta di Dirak. Detto questo si studia
la lastra sottile per vedere che, entro una certa approssimazione lo studio di 
questa lastra non dipende dallo spessore della lastra. Se la carica è definita 
su di un volume nullo, per far sì che la carica sia finita, l'integranda deve andare 
all'infinito:
\begin{gather*}
    \rho = \sigma \Theta \ \Longrightarrow \ \sigma(x, y) \delta(z)
\end{gather*} 
Data
\begin{gather*}
    \vv{E} = \frac{1}{4\pi\epsilon_0} \int_{\tau} \frac{\rho(\vv{r'} )}{\left| r - r' \right|^{3} }(\vv{r} - \vv{r'}  ) \ d\tau  
\end{gather*}
Si trasforma in un integrale triplo (perché di volume):
\begin{gather*}
    \frac{1}{4\pi\epsilon_0} \int_{\tau} \frac{\sigma(x', y') \delta(z')(\vv{r} - \vv{r'}  )}{\left| \vv{r} -\vv{r'}   \right|^{3} } dx' dy' dz' 
\end{gather*}
L'integrale in $dz$ contiene una $\delta(z)$, per, per la proprietà della $\delta(z)$ 
\begin{gather*}
    = \frac{1}{4\pi\epsilon_0} \int_{S} \frac{\sigma(x', y')( \vv{r} - \vv{r'}  )}{\left| r - r' \right|^{3} }dx \ dy
\end{gather*}
Dunque, dato che
\begin{gather*}
    \left| \vv{r} - \vv{r'}   \right| = \sqrt{(x - x')^{2} + (y - y')^{2} + (z - z') ^{2}}  
\end{gather*}
Nell'ultimo integrale siamo solamente su di una superficie, per cui $z' = 0$. Dunque nell'integrale su $S$
la stessa funzione diventa 
\begin{gather*}
        \left| \vv{r} - \vv{r'}   \right| = \sqrt{(x - x')^{2} + (y - y')^{2} + (z) ^{2}}  
\end{gather*}
Dunque è la distanza del punto del campo da un qualsiasi punto della sorgente che sta 
sul piano $z = 0$. La lastra è dunque schiacciata a carica costante e dunque, 
se l'integrale di volume rimane costante, aumenta la densità di carica, dunque se tende a zero 
si utilizza il concetto di densità di superficie (il volume nullo mi dice che $\rho \to 0$, 
e dunque si utilizza $\sigma$, ossia la densità di carica superficiale). La $\sigma$ può tendere a
$\lambda = \rho \cdot \delta(z) \delta(x)$ (ossia si ottiene la densità di carica lineare per un filo schiacciando
la terza dimensione). Il filo carico visto come $\rho$ è la stessa cosa della lastra schiacciata non più sun una sola dimensione,
ma anche sull'altra:
\begin{gather*}
    \rho = \sigma(\delta z) \ \Longrightarrow \ \sigma = \lambda \delta (x) \qquad \rho = \lambda \delta(z) \delta(x)
\end{gather*} 
Se si schiacciasse anche la terza dimensione, si ottiene la carica puntiforme, ossia
\begin{gather*}
    \rho = q \delta(x) \delta (y)\delta(z) \equiv q\delta(\vv{r} )
\end{gather*}



\end{document}
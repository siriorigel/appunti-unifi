\documentclass[a4paper, oneside]{article}
\usepackage{graphicx}
\usepackage{amsthm}
\usepackage{amsmath}
\usepackage{amssymb}
\usepackage[a4paper,
            bindingoffset=0.2in,
            left=2cm,
            right=2cm,
            top=2cm,
            bottom=2cm,
            footskip=.25in]{geometry}
\usepackage[italian]{babel}
\usepackage{pgfplots}
\usepackage{tabularx}
\usepackage{tikz}
\usepackage{wrapfig}
\usepackage{color}
\usepackage[d]{esvect}
\definecolor{page}{rgb}{0.129,0.157,0.212}
\pagecolor{page}
\color{white}
\graphicspath{ {./images/} }
\usetikzlibrary{shapes.geometric}
\usetikzlibrary{datavisualization}
\usetikzlibrary{datavisualization.formats.functions}
\usetikzlibrary{patterns}
\pgfplotsset{width=10cm,compat=1.18}

\title{Appunti di Fisica II}
\author{Tommaso Miliani}
\date{23-02-26}

\begin{document}
\newtheoremstyle{theoremEnv}
                {}          % Space above
                {}          % Space below
                {\slshape}  % Body font
                {}          % Indent amount
                {\bfseries} % Head font
                {.}         % Punctuation after head
                {\newline}  % Space after theorem head
                {}          % Theorem head spec
\theoremstyle{theoremEnv}

\newtheorem{definition}{Definizione}[section]
\newtheorem{theorem}{Teorema}[section]
\newtheorem{lemma}{Proposizione}[section]
\newtheorem{observation}{Osservazione}[section]
\newtheorem{corollary}{Corollario}[theorem]
\newtheorem{example}{Esempio}[section]
\newtheorem{remark}{Enunciato}[section]

\maketitle

\section{Intro}
Informazioni generali sul corso
\begin{itemize}
    \item Prof: Massimo Gurioli
    \item Esercitazioni: Giovanni Ferioli, Francesco Biccari
    \item Libri: Mencuccini Silvestri Fisica II (facile) o Jackson Elettrodinamica classica e Griffiths Introduction to Electrodynamics (difficili);
    \item Esame: scritto + orale, il voto dello scritto vale 1 ANNO e si entra
    all'orale con 18 (si può entrare a fare l'orale anche con 17 ma ci sarà un esercizio
    all'orale). Lo scritto ha 3 esercizi con tre domande ciascuno, di cui si deve anche
    trovare i valori numerici. La valutazione totale deriva in LARGA parte dal voto dell'orale (il voto dello scritto è solo per 
    entrare, dunque anche chi entra con 18 può arrivare a 30).
\end{itemize}
Il corso si divide in due parti
\begin{itemize}
    \item Elettrostatica;
    \item Elettrodinamica
\end{itemize}

\section{Cenni storici e introduzione all'elettrostatica}
Il primo fenomeno studiato è quello dell'elettrizzazione (per strofinio)  come 
quando si strofina la plastica o l'ambra. Un'altro fenomeno studiato nell'antica Grecia 
è sicuramente la magnetizzazione, studiando la magnetite, iniziarono a scoprire che questi materiali
si attraevano tra loro. Con il procedere degli anni, si è iniziato a capire come alcuni 
materiali attraessero altri materiali mentre altri si respingevano tra loro, alcuni naturalmente, 
altri mediante strofinio, come se dovessero essere caricati. Nel settecento si iniziò 
a studiare le correnti, i fulmini, fino ad arrivare a Maxwell.

\section{Sviluppo storico}
Per misurare i fenomeni elettrici con precisione si doveva utilizzare
\begin{enumerate}
    \item MOlta carica
    \item Saperla conservare
    \item Saperla misurare
    \item Inventare uno strumento che poteva misurare con precisione la forza al variare della distanza
    \item Trovare un modello o legge matematica
    \item Infine definire delle leggi che unificassero tutti i fenomeni distinti.
\end{enumerate}

\subsection{Generatori di carica}
I generatori di carica utilizzavano lo strofinio 

\subsection{Accumulo di carica}
Dai generatori di carica si potevano (tramite la bottiglia di Leida), estrarre questa carica. La
bottiglia di Leida era una sorta di condensatore (costituito da una bottiglia di vetro ricoperta da un foglio di alluminio).

\subsection{Misuratori}
Per poter studiare la carica era necessario misurarla utilizzando degli strumenti che avevano
delle lamelle che divergevano per effetto delle forze elettromagnetiche

\subsection{Misuratori di forza}
Utilizzando dei misuratori di forza, ora era possibile introdurre il concetto di forza anche 
ai fenomeni elettrostatici attraverso un pendolo di torsione ideato da Coulomb. 

\subsection{Modelli}
Coulomb introdusse ora un modello magnetico attraverso le masse delle particelle (che tuttavia
non funzionò molto bene), mentre il modello per le cariche elettriche funziona ancora oggi interpolando 
i dati sperimentali. Coulomb, nonostante i grandi errori, pensò che anche questa legge dovesse andare come 
il quadrato della distanza (così come aveva fatto Newton) poiché secondo lui fittava meglio i dati. Ad oggi si 
sa che quel due è preciso fino a $10^{-16}$. 
\begin{gather*}
    F = k_e \frac{q_1 q_2}{R^{2}}
\end{gather*}
Per misurare questa forza si definisce il Coulomb come 
\begin{gather*}
    \vv{F} = \frac{1}{4\pi\epsilon_0} \frac{q_1 q_2}{r^{2}}\vv{r}  
\end{gather*}
Mentre nel sistema CGS si utilizza lo statC, ossia $1 \ \text{statC} \approx 3.33564 \cdot 10^{-10} \ C$. 

\section{Storica dell'elettrodinamica}
Alessandro Volta fu il primo fisico a capire che era possibile utilizzare una differenza
di potenziale per poter far fluire cariche nei conduttori per utilizzarla per svariati usi. Grazie ad
Ampere si è creato uno strumento per misurare questa corrente, mentre grazie a Ohm si è arrivati a delle leggi sulla 
corrente mentre Oersted è riuscito ha capire che con le correnti si muovevano anche 
i campi magnetici, poi è stato il turno di Faraday a scoprire le linee di forza di un conduttore. \\ 
Solo Maxwell è riuscito a dare forma alle leggi e ai modelli elettrodinamici aggiungendo un termine che 
ha chiamato \textbf{corrente di spostamento}:
\begin{gather*}
    \vv{\nabla} \times \vv{B} = \mu_0 \vv{J} + \boxed{ \mu_0 \epsilon_0 \frac{\partial \vv{E} }{\partial t}}
\end{gather*}
Dopo Maxwell, Poynting (chiede all'esame sennò scopa culo di tutti), che ha verificato l'esistenza dell'onda elettromagnetica. 


\section{Definizione della forza elettrostatica}
\begin{wrapfigure}{r}{0.4\textwidth}
    \centering
    \caption{}
    \begin{tikzpicture}
        \filldraw(0, 0) circle (1pt) node[anchor = north] {$Q$};
        \filldraw(2, 1) circle (1pt) node[anchor = north] {$q$};
        \draw[dashed](0, 0) -- (2, 1);
        \draw[->](2, 1) -- (3, 1.5) node[midway, above] {$\vv{r}$ };
        \draw[->](0, 0) -- (1, 0) node[at end, below] {$x$};
        \draw[->](0, 0) -- (-0.7, -0.3) node[at end, below] {$y$};
        \draw[->](0, 0) -- (0, 1) node[at end, left] {$z$};
    \end{tikzpicture}    
\end{wrapfigure}
Date due cariche puntiforme (molto piccole data la grande distanza che le separa),
l'approssimazione di cariche puntiforme è molto utile dal punto di vista matematico è molto utile in quanto 
non c'è bisogno di alcun modello per descriverle. La forza tra loro è data da
\begin{gather*}
    \vv{F_q}  = \frac{1}{4\pi\epsilon_0}\frac{qQ}{r^{2}}\hat{r} 
\end{gather*}
Il segno della forza descrive l'attrattività o repulsività della forza. 
Si definisce il campo elettrico della forza 
\begin{gather*}
    \vv{E}(\vv{r} ) =  \lim_{q \to 0} \frac{\vv{F_q} }{q} 
\end{gather*}
Ci vuole l'accortezza, da parte dello sperimentatore, di verificare che la carica di prova 
non perturbi la distribuzione di carica che genera il campo elettrico. Per poter misurare dunque
il campo elettrico si deve utilizzare una carica di prova sempre più vicina a zero di intensità. La 
successione di cariche che tende verso zero deve convergere verso un certo valore, come nel seguente grafico:
\begin{gather*}
    \begin{tikzpicture}
        \draw[->](0, 0) -- (3, 0) node[at end, below] {$q$};
        \draw[->](0, 0) -- (0, 3) node[at end, left] {$F_q$};
        \filldraw(0.5, 0.5) circle (1pt); 
        \filldraw(1, 1) circle (1pt);
        \filldraw(1.5, 1.5) circle (1pt);
        \filldraw(2, 2) circle (1pt);
        \filldraw(2.5, 2.25) circle (1pt);
        \filldraw(3, 2.5) circle (1pt);
    \end{tikzpicture}
\end{gather*}
Solamente vicino a zero 
il campo elettrico è definito come lineare, per cariche lontane dallo zero, invece, la forza non è più lineare. 
Nel limite di carica puntiforme, il campo elettrico per la carica puntiforme è equivalente a
\begin{gather*}
    \vv{E}(\vv{r}) = \frac{1}{4\pi\epsilon_0}\frac{Q}{r^{2}}\hat{r}  = \frac{1}{4\pi\epsilon_0} \frac{Q}{r^{3}}\hat{r} 
\end{gather*}

\noindent
\begin{wrapfigure}{r}{0.4\textwidth}
    \centering
    \begin{tikzpicture}
        \draw[->](0, 0) -- (1, 0) node[at end, below] {$x$};
        \draw[->](0, 0) -- (0, 1) node[at end, left] {$z$};
        \draw[->](0, 0) -- (-0.7, -0.5) node[at end, below] {$y$};
        \draw[->](0, 0) -- (1, 1) node[midway, above] {$\vv{r'}$ };
        \draw[->](0, 0) -- (3, 1) node[midway, below] {$\vv{r}$ };
    \end{tikzpicture}    
\end{wrapfigure}
Quando un corpo ha carica, lo spazio intorno a lui viene modificato e non è più vuoto
e prende il nome di \textbf{campo elettrico} (ossia un campo vettoriale) e descrive una forma di energia nell'universo 
misurabile e concreta. Un campo vettoriale è un vettore associato ad ogni punto dello spazio, e dunque 
funzione del punto dello spazio ed è univocamente definito se si conosce la divergenza ed il rotore, oltre che alle
sue boundary conditions in un certo volume. Dato che il campo è derivato dalla forza, e la forza totale 
su di un corpo è la somma vettoriale delle forze, il campo in un punto dello spazio è la somma 
dei campi su quel punto; il fatto che valga per le forze il prinicpio di sovrapposizione, vale anche per il campo. 
Se si avessero due cariche, prima si dovrebbe trovare una relazione che valga anche quando la carica 
non è nell'origine. La carica $\vv{r'}$ è la \textbf{carica sorgente}, mentre la carica di prova $q$ è quella 
che prende il nome di \textbf{carica di campo}. Il campo elettrico è dunque definito come
\begin{gather*}
    \vv{E}(\vv{r} ) = \frac{1}{4\pi\epsilon_0}\frac{Q}{\left| \vv{r}- \vv{r'}   \right|^{3} } (\vv{r} - \vv{r'}  )
\end{gather*} 
Presa un insieme numerabile di cariche, $\vv{r'}$ diventa scomodo e dunque la coordinata sorgente è $\vv{r_i}$. Dunque la 
formula di prima diventerà 
\begin{gather*}
    \vv{E} = \vv{E}_1 + \vv{E}_2   
\end{gather*}  
sfruttando il principio di sovrapposizione, per cui se si calcolasse il campo in un punto $\vv{r}$ qualsiasi, esso è dato 
dalla somma dei due campi che generano le due sorgenti. Il risultato è
\begin{gather*}
    \vv{E}(\vv{r} ) = \frac{1}{4\pi\epsilon_0}\left(\frac{Q_1}{\left| r - r_1 \right|^{3} } (\vv{r} - \vv{r_1}  ) + \frac{Q_2}{\left| r - r_2 \right|^{3} }(\vv{r} - \vv{r_2}  )\right)  = \frac{1}{4\pi\epsilon_0} \sum_{i = 1}^{2}\frac{Q_i}{\left| r - r_i \right|^{3} }(\vv{r} - \vv{r_i}  ) 
\end{gather*}

\noindent
Questa equazione definisce in maniera definitiva il problema elettrostatico delle cariche. Si vuole ora determinare la distribuzione di carica su di un volume.
Si introduce dunque il concetto di densità di carica eseguendo $N$ partizioni $d\tau_i$.  In ogni partizione del volume considerato
c'è una carica
\begin{gather*}
    dQ = \rho (\vv{r} )d\tau
\end{gather*}
Se si volesse calcolare il campo rispetto ad un punto $\vv{r}$, allora, per il principio di sovrapposizione, si deve eseguire la somma
di tutti i volumetti e dunque la somma diventa un integrale di volume. 
\begin{gather*}
    \begin{array}{l}
        d\tau \to 0 \\
        N \to \infty 
    \end{array} \ \Longrightarrow \ \vv{E}(\vv{r} ) = \frac{1}{4\pi\epsilon_0} \int_{\tau} \frac{dQ(\vv{r'} )  }{\left| r - r' \right|^{3} } (\vv{r} - \vv{r'}  )
\end{gather*} 
Ottenendo dunque
\begin{align}
    \vv{E}(\vv{r} ) =  \frac{1}{4\pi\epsilon_0}\int_{\tau} \frac{\rho(\vv{r} )}{\left| r - r' \right|^{3} }(\vv{r} - \vv{r'}  )d\tau 
\end{align}

\begin{example}[Risoluzione di un esercizio di distribuzione di carica e filo infinito]
    Consideriamo un cilindro lungo $2L$ e che abbia una sezione di raggio $d$, su questo
    è distribuita, in modo uniforma, una carica totale $Q$ e si vuole calcolare il campo sull'asse
    ad una certa distanza $x$. Si vuole determinare questa distribuzione secondo le seguenti condizioni:
    $d << L$ e $d << x$. Posso eseguire dunque la seguente approssimazione: si può ignorare le
    dimensioni fisiche del cilindro e si può allora definire la arica sul piccolo volumetto $dz$:
    \begin{gather*}
        dQ = \rho dz S
    \end{gather*}
    Dove $S = \pi d^{2}$. Adesso si può riscrivere
    \begin{gather*}
        dQ = \rho dz S = \lambda dz \qquad \lambda = \rho S
    \end{gather*}
    A questo punto si può trattare il problema come se fosse un filo: la densità non è più 
    carica su metro cubo ma carica su metro (infatti $\lambda$ è la densità lineare). 
    \begin{gather*}
        E(x, 0) = \frac{1}{4\pi\epsilon_0} \int_{-L}^{L} \frac{\lambda dz}{\left| x^{2} + z^{2} \right|^{\frac{3}{2}} }(x\hat{x} - z\hat{z}  )  \qquad 
        \begin{array}{l}
            \vv{r} = x\hat{x} \\
            \vv{r'} = z\hat{z}    
        \end{array} \ \Longrightarrow \  \vv{r} -\vv{r'} = x\hat{x} - z\hat{z}    
    \end{gather*}
    Si può ora procedere a risolvere i due integrali: il secondo integrale è 
    nullo in quanto i contributi in $z$ sono sempre simmetrici, mentre l'altro integrale
    \begin{gather*}
        \frac{\lambda x\hat{x} }{4\pi\epsilon_0}\int_{-L}^{L} = \frac{1}{(x^{2} + z^{2})^{\frac{3}{2}}} dz 
    \end{gather*}
    La soluzione di questo integrale non è immediata e non è da sapere per i compiti, ma la sua soluzione è la seguente:
    \begin{gather*}
        \frac{\lambda x \hat{x} }{4\pi \epsilon_0} \frac{1}{x^{2}} \left.\frac{z}{\sqrt{x^{2} + z^{2}} }\right|_{-L}^{L} = \frac{\lambda}{4\pi\epsilon_0}\frac{\hat{x} }{x}\left(\frac{2L}{\sqrt{x^{2} + z^{2}} }\right)
    \end{gather*}
    $\lambda 2L$, data la definizione di $\lambda$, dà tutta la carica contenuta nel filo. Dunque, sull'asse $x$, quello
    che conta è la carica totale. Quando ci si allontana dal filo, ossia quando $x >> L$, la soluzione diventa 
    \begin{gather*}
        \frac{\lambda 2L\hat{x} }{4\pi \epsilon_0x^{2}} = \frac{Q}{4\pi\epsilon_0 } \frac{\hat{x} }{x^{2}}
    \end{gather*}
    E dunque, se si è sufficientemente lontani, la carica diventa puntiforme. Che succede invece se, per un filo 
    finito, si vuole determinare il campo in un punto molto vicino? Negli esercizi si può dire che esiste un filo indefinito carico
    con $Q$ ben definito, ci si pone nel caso $L >> x >> d$, dunque il campo:
    \begin{gather*}
        \vv{E}(x) = \frac{1}{2\pi\epsilon_0}\frac{\lambda }{x} \hat{x}  
    \end{gather*}
    In questo caso il campo non va come $\frac{1}{r^{2}}$ ma come $\frac{1}{r}$, così come si era visto per la
    definizione di campo quando la carica è sufficientemente piccola. Il motivo fisico è perché il filo è infinito: se è 
    distribuita su di un filo infinito, si ottengono risultati molto strani. 
\end{example}

\section{Teorema di Gauss}
Il teorema di Gauss mi dice che, scelta una superficie chiusa e orientata, il flusso del 
campo elettrico attraverso questa superficie chiusa è uguale alla carica contenuta diviso 
$\epsilon_0$, unendo dunque il flusso ed il campo magnetico. 
Se si prende la sezione $S$ di un tubo, la quantità di massa che passa attraverso a questa superficie
è data da 
\begin{gather*}
    \Delta M = \rho v \ dt S
\end{gather*}
Si divide ora per il tempo considerato $\Delta t$, ottenendo:
\begin{gather*}
    \frac{\Delta M}{\Delta t} = \rho v S \ \Longrightarrow \ J \cdot \hat{n} S  
\end{gather*}
Dove $\hat{n}$ è la normale alla superficie $S$ scelta. La quantità di materia che 
fluisce in un tubo in un certo tempo è data da questa espressione prendendo il nome di 
\textbf{flusso di $\vv{J} $ attraverso $S$}  indicato con $\Phi_{\vv{J}}(S)$. Se la superficie
fosse inclinata, dovrei anche considerare l'angolo tra $\hat{n}$ e $\vv{v}$: dunque 
introduco il coseno:
\begin{gather*}
    \frac{\Delta M}{\Delta t} = JvS_2\cos\theta
\end{gather*}   
Secondo il teorema di \textbf{Cavalieri}, $S = S_2\cos\theta$, dunque il flusso 
non dipende dalla superficie scelta all'interno del tubo. Se poi $\vv{J}$ dipendesse dalla 
superficie in qualche modo, allora dovrei anche fare l'integrale di superficie
\begin{gather*}
    \Phi_{\vv{J} }(S) = \int_{S} \vv{J}(r') \cdot \hat{u} \ dS   
\end{gather*}
Fuori dal tubo il flusso è ovviamente zero. Detto questo si può utilizzare il concetto di flusso per ogni 
campo vettoriale; è chiaro che, quando si utilizza per $\vv{E}$ in elettrostatica non fluisce nulla, ma 
a noi interessa l'integrale sulla superficie chiusa considerata che chiamiamo flusso. L'angolo solido è 
il concetto di angolo elevato allo spazio: l'angolo solido si definisce data una calotta sferica per cui
\begin{gather*}
    \Omega = \frac{S}{r^{2}}
\end{gather*}
Sia data una carica $Q$ e la pongo nell'origine, supponendo che questa carica sia 
dentro una superficie a caso $S$. Si sceglie ora di calcolare il flusso attraverso un elemento
infinitesimo di superficie chiamato $dS$:
\begin{gather*}
    d\Phi = \vv{E} \cdot \hat{n} \ dS  = \frac{1}{4\pi\epsilon_0}\frac{Q}{r^{2}}\hat{r} \cdot \hat{n} \ dS = \frac{Q}{4\pi\epsilon_0} \frac{dS\cos\theta}{r^{2}} = \frac{Q}{4\pi\epsilon_0} \frac{dS_n}{r^{2}}
\end{gather*}
secondo la definizione di flusso. Quando la superficie è infinitesima, il campo è sempre 
uniforme sulla superficie data. Il campo adesso è conosciuto e cambia verso a seconda della carica,
ma il fatto che il campo entra oppure esca è contenuta dentro $Q$ e dunque il campo è sempre uscente 
ed il segno è determinato dalla carica. Il flusso è quindi entrante se è negativa e uscente se invece è 
una carica positiva (anche se convenzionalmente $\hat{n}$ è uscente così come $\hat{r}$  ). Dato che c'è 
$\cos\theta$, il campo elettrico non è parallela alla normale, col coseno riesco a proiettarla in 
modo tale che siano parallele e che chiamo $dS_n$ e che considero essere un pezzetto di sfera. Dato che è 
infinitesimo è possibile farla questa approssimazione e dunque 
\begin{gather*}
    \frac{dS_n}{r^{2}} = d\Omega 
\end{gather*}
Adesso il flusso non dipende più dalla superficie scelta ma dal cono di spazio che
arriva all'elemento considerato. Adesso, per arrivare al teorema di Gauss, si deve integrare lungo tutta la superficie
considerata. Ma, dato che ogni elemento infinitesimo è un elemento di calotta sferica, allora l'integrale totale 
sarà un integrale su di una sfera:
\begin{gather*}
    \Phi = \int_{S}^{} d\Phi = \frac{Q}{4\pi\epsilon_0}\int_{S} d\Omega = \frac{Q}{\epsilon_0} 
\end{gather*}
Si è assunto, fino ad ora, che il coseno sia sempre positivo, questo perché il raggio versore è sempre uscente
dalla superficie e dunque l'angolo rispetto alla direzione uscente è sempre compreso tra
-90° e 90°. L'unica possibilità  per cui il coseno non sia positivo è che ci sia una superficie lungo la quale la linea di forza esce, poi rientra e poi riesce. Data
una superficie chiusa con carica interna, la linea di forza considerata può attraversare la superficie solo un numero dispari di volte (in quanto alla
fine DEVE sempre uscire), questo vuol dire che $\cos\theta$ è sempre positivo. Il campo sulla superficie è sempre la somma
di tutti i campi delle cariche interne ed il risultato rimane sempre lo stesso poiché il campo gode della sovrapposizione e 
secondo perché l'integrale è un operatore lineare, giungendo dunque alla conclusione
\begin{align}
    \Phi_S  = \frac{Q_{int}}{\epsilon_0}
\end{align}
Se si considerasse ora una carica esterna, il ragionamento che si utilizza è considerare un angolo
solido infinitesimo che parta adesso dalla carica. Senza fare dimostrazioni precise, l'angolo solido o non 
intercetta mai la superficie o la intercetta sempre un numero pari di volte: se entra deve anche 
uscire (anche se fosse arzigogolata, essa incontrerà sempre la superficie un numero pari di volte). 
Si considera ora che questa carica attraversi la superficie due volte. Si considerano le due superfici
$dS_1$ e $dS_2$ e considero il flusso attraverso la somma di queste due superfici:
\begin{gather*}
    d\Phi = \vv{E_1} \cdot \hat{n_1} \ dS_1 + \vv{E_2} \cdot \hat{n_2} \ dS_2 = \frac{Q}{4\pi\epsilon_0}\left(\frac{\cos\theta_1\  dS_1}{r_1^{2}} + \frac{\cos\theta_2\  dS_2}{r_2^{2}}\right) = 0
\end{gather*}
Dato che $\vv{r}$ è sempre uscente, quando si arriva sulla superficie, la normale è sempre uscente, tra i due versori l'angolo
è dunque compreso tra $\frac{\pi}{2}$ e $\pi$ per la prima superficie, mentre la seconda superficie è compreso 
tra 0 e $\frac{\pi}{2}$. Dunque succede che, le due quantità sono sempre l'angolo solido, ma i due segno sono sempre opposti: il primo
è sempre negativo ed il secondo può essere anche negativo. Sono dunque opposti $\cos\theta_1 \ dS_1$ e $\cos\theta_2 \ dS_2$.  
Detto questo, il flusso fa zero perché i due moduli sono equivalenti ma hanno segno opposto. Il campo elettrico 
di una carica esterna ad una superficie non contribuisce al flusso del campo elettrico della superficie. Riassumendo, 
per una carica puntiforme si è trovato che il flusso 
\begin{gather*}
    \Phi = \frac{Q}{\epsilon_0}
\end{gather*}
Se la carica è interna. Se invece la carica è estena
\begin{gather*}
    \Phi = 0
\end{gather*}
Se si avessero più cariche, ovviamente si utilizza il prinicpio di sovrapposizione e la 
linearità dell'operatore integrale, ottenendo che
\begin{gather*}
    \Phi = \int_{S} \sum_{i = 1}^{n} \vv{E_i}\cdot \hat{n} \ dS     
\end{gather*}
Il teorema non dice che il flusso totale è uguale a $\frac{Q_{int}}{\epsilon_0}$, ma lo fa per una sola carica e poi utilizzando 
la linearità dell'operatore integrale e la sovrapposizione. Il campo $\vv{E}$ è il campo di tutto l'universo ed 
è indipendente dalla carica che lo genera.  

\begin{example}
    Supponendo che alcune di cariche ($M$) siano interne alla superficie, mentre le altre sono 
    interne, allora il flusso totale è
    \begin{gather*}
        \sum_{i = 1}^{M} \int \vv{E_i} \cdot \hat{n} \ dS = \sum_{i = 1}^{M} \frac{Q_i}{\epsilon_0} 
    \end{gather*}
    Mentre la somma delle altre è sempre zero.
\end{example}




\end{document}
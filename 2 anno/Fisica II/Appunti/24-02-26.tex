\documentclass[a4paper, oneside]{article}
\usepackage{graphicx}
\usepackage{amsthm}
\usepackage{amsmath}
\usepackage{amssymb}
\usepackage[a4paper,
            bindingoffset=0.2in,
            left=2cm,
            right=2cm,
            top=2cm,
            bottom=2cm,
            footskip=.25in]{geometry}
\usepackage[italian]{babel}
\usepackage{pgfplots}
\usepackage{tabularx}
\usepackage{tikz}
\usepackage{wrapfig}
\usepackage{color}
\usepackage[d]{esvect}
\definecolor{page}{rgb}{0.129,0.157,0.212}
\pagecolor{page}
\color{white}
\graphicspath{ {./images/} }
\usetikzlibrary{shapes.geometric}
\usetikzlibrary{datavisualization}
\usetikzlibrary{datavisualization.formats.functions}
\usetikzlibrary{patterns}
\pgfplotsset{width=10cm,compat=1.18}

\title{Esercizi di fisica II}
\author{Tommaso Miliani}
\date{24-02-26}

\begin{document}
\newtheoremstyle{theoremEnv}
                {}          % Space above
                {}          % Space below
                {\slshape}  % Body font
                {}          % Indent amount
                {\bfseries} % Head font
                {.}         % Punctuation after head
                {\newline}  % Space after theorem head
                {}          % Theorem head spec
\theoremstyle{theoremEnv}

\newtheorem{definition}{Definizione}[section]
\newtheorem{theorem}{Teorema}[section]
\newtheorem{lemma}{Proposizione}[section]
\newtheorem{observation}{Osservazione}[section]
\newtheorem{corollary}{Corollario}[theorem]
\newtheorem{example}{Esempio}[section]
\newtheorem{remark}{Enunciato}[section]

\maketitle

\section{Esercizi elettrostatica}
\begin{example}[Esercizio 1 del compito dell'8/01/2007]
    Quattro cariche che formano un quadrato di lato $L = 0.12$ m, le cui 
    cariche sono $Q = 10^{-9}$ Coulomb. Calcolare il campo elettrico elettrico nel punto
    $O$ e nel punto $M$. 
    \begin{gather*}
        \begin{tikzpicture}
            \filldraw(-1, 1) circle (1pt) node[anchor = south] {D};
            \filldraw(1, 1) circle (1pt) node[anchor = south] {C};
            \filldraw(-1, -1) circle (1pt) node[anchor = north] {A};
            \filldraw(1, -1) circle (1pt) node[anchor =north] {B};
            \filldraw(0, 0) circle (1pt) node[anchor = south] {$O$};
            \draw[dashed](-1, 1) -- (1, 1) -- (1, -1) -- (-1, -1) -- (-1, 1);
            \draw[dashed](-1, 1) -- (1, -1);
            \draw[dashed](1, 1) -- (-1, -1);
            \filldraw(1, 0) circle (1pt) node[anchor = west] {M};
            \draw[->, thick](0, 0) -- (0.5, 0.5);
            \draw[->, thick](0, 0) -- (-0.5, 0.5);
            \draw[->, thick](0, 0) -- (0.5, -0.5);
            \draw[->, thick](0, 0) -- (-0.5, -0.5);
        \end{tikzpicture}
    \end{gather*}
    Il campo elettrico nel punto O è ovviamente zero in quanto c'è una simmetria
    intrinseca del sistema, infatti i campi
    \begin{gather*}
        E_A = - E_C \qquad E_B = - E_D \qquad E_A = E_B = E_C = E_D \ \Longrightarrow \ E(O) = 0
    \end{gather*}
    Adesso, nel punto $M$ il campo può essere schematizzato secondo il seguente sistema di riferimento:
    \begin{gather*}
        \begin{tikzpicture}
            \draw[->](0, -1.5) -- (0, 1.5) node[at end, left] {$y$};
            \draw[->](0, 0) -- (3, 0) node[at end, below] {$x$};
            \filldraw(0, 1) circle (1pt) node[anchor = east] {D};
            \filldraw(2, 1) circle (1pt) node[anchor = south] {C};
            \filldraw(0, -1) circle (1pt) node[anchor = east] {A};
            \filldraw(2, -1) circle (1pt) node[anchor =north] {B};
        \end{tikzpicture}
    \end{gather*}
    Così si ottengono i seguenti vettori posizione
    \begin{gather*}
        \vv{r_A} = \begin{pmatrix} 0\\
        -\frac{L}{2} \end{pmatrix}  \qquad \vv{r_B} = \begin{pmatrix} L \\
        -\frac{L}{2} \end{pmatrix} \qquad \vv{r_C} = \begin{pmatrix} L \\
        \frac{L}{2} \end{pmatrix} \qquad \vv{r_D} = \begin{pmatrix} 0 \\
        \frac{L}{2} \end{pmatrix} \qquad \vv{r_M} = \begin{pmatrix} L \\
        0 \end{pmatrix}   
    \end{gather*}
    Dunque il vettore posizione in M è solo lungo l'asse $x$ per costruzione. Utilizzando 
    il principio di sovrapposizione, si può determinare il campo elettrico nel punto M:
    \begin{gather*}
        \vv{E} (r_M) = E_A(r_A) +  E_B(r_B) + E_C(r_C) + E_D(r_D) 
    \end{gather*}
    Si trova
    \begin{gather*}
        \vv{E_A} = \frac{Q}{4\pi\epsilon_0} \frac{1}{\left| \vv{r_A} - \vv{r_M}   \right|^{2} } \frac{\vv{r_M} - \vv{r_A}  }{\left| \vv{r_M} - \vv{r_A}   \right| }  \qquad \vv{r_M} - \vv{r_A} = L\hat{x} + \frac{L}{2}\hat{y}    
    \end{gather*}
    Ottenendo
    \begin{gather*}
        \vv{E_A} = \frac{Q}{4\pi\epsilon_0} \frac{1}{\left(L\sqrt{\frac{5}{4}} \right)^{3}} \left(L \hat{x} + \frac{L}{2}\hat{y}  \right)
    \end{gather*}
    Si calcola ora il campo $\vv{E_B} $:
    \begin{gather*}
        \vv{E_B} = \frac{Q}{4\pi\epsilon_0} \frac{1}{\left| \vv{r_B} - \vv{r_M}   \right|^{2} } \frac{\vv{r_M} - \vv{r_B}  }{\left| \vv{r_M}  - \vv{r_B} \right| }  \qquad \vv{r_M} - \vv{r_B} = \frac{L}{2} \hat{y}   
    \end{gather*}
    Dunque 
    \begin{gather*}
        \vv{E_B} =  \frac{Q}{4\pi\epsilon_0} \frac{1}{\left(\frac{L}{2}\right)^{3}} 
    \end{gather*}
    Manca solamente i due campi elettrici in $D$ e $C$, data la simmetria del problema, le componenti
    lungo l'asse $y$ saranno uguali in modulo ma con il segno contrario:
    \begin{gather*}
        \vv{E_C}  = - \vv{ E_B} \qquad \vv{E_D}  = \frac{Q}{4\pi\epsilon_0} \frac{1}{\left(L\sqrt{\frac{5}{4}} \right)^{3}} \left(L \hat{x} - \frac{L}{2}\hat{y}  \right)
    \end{gather*}
    Dunque il campo totale nel punto $M$ sarà data dalla seguente espressione
    \begin{gather*}
        \vv{E_M} =  \frac{Q}{2\pi\epsilon_0} \frac{1}{\left(L\sqrt{\frac{5}{4}} \right)^{3}} \left(L\hat{x}  \right)
    \end{gather*}
\end{example}

    Prendendo un volume qualsiasi, è possibile applicare il principio di sovrapposizione su 
    una distribuzione di carica continua:
    \begin{gather*}
        \begin{tikzpicture}
            \draw[->](0, 0) -- (1, 0) node[at end, below] {$y$};
            \draw[->](0, 0) -- (0, 1) node[at end, left] {$z$};
            \draw[->](0, 0) -- (-0.7, -0.5) node[at end, below] {$x$};
        \end{tikzpicture}
    \end{gather*}
    Dunque, per una qualsiasi carica $dq = \rho(r' )dx$, dove $\rho$ è la densità di carica del volume. In generale
    il campo elettrico sarà uguale a 
    \begin{gather*}
        \vv{E} = \int_{r'} \frac{dq(r')}{\left| \vv{r} - \vv{r'}   \right|^{3} }  (\vv{r} - \vv{r'}  ) \frac{1}{4\pi\epsilon_0}
    \end{gather*}
    Dove $\vv{r}$ è un generico punto nello spazio nel quale si vuole 
    determinare il campo elettrico, mentre $\vv{r'}$ è il vettore posizione dell'infinitesimo di carica $dq$ 
    rispetto al sistema di riferimento considerato 

\begin{example}[]
    Dato un filo infinito con distribuzione di carica lineare (e uniforme) $\rho_l$, si vuole determinare il 
    campo elettrico in un punto generico dello spazio.
    \begin{gather*}
        \begin{tikzpicture}
            \draw[very thick, ->](0, -2) -- (0, 2) node[at end, left] {$z$};
            \filldraw(2, 0) circle (1pt) node[anchor = south] {P};
            \draw[->](0, 0) -- (3, 0) node[at end, above] {$ x$};
            \draw(2, 0) -- (0, -1);
        \end{tikzpicture}
    \end{gather*}
    Per motivi di simmetria, il campo elettrico in ogni punto $z$ è sempre uguale, 
    inoltre, il problema è anche dotato di simmetria cilindrica secondo la quale 
    non si perde la generalità del problema anche quando si ruota il filo. Si considera 
    ora la generica carica alla posizione $z$ nello spazio che genera un campo elettrico 
    rispetto al punto di riferimento $P$ giacente sull'asse $x$. 
    \begin{gather*}
        dq = \rho_l \ dz 
    \end{gather*}
    Data $R$ la distanza del punto $P$ dal filo, si ottiene l'ipotenusa nel disegno come 
    \begin{gather*}
        \left| r \right| = \frac{R}{\cos\phi} \qquad \phi : \quad z = R\tan\phi 
    \end{gather*}
    Si può ora determinare il campo elettrico 
    \begin{gather*}
        \vv{E} = \begin{pmatrix} E_x \\
        E_z \end{pmatrix}   \qquad 
        \begin{array}{l}
            E_x = \left| E \right| \cos\phi \\
            E_z = \left| E \right| \sin\phi   
        \end{array}
    \end{gather*}
    Dove
    \begin{gather*}
        \left| E \right| = \frac{1}{4\pi\epsilon_0} \frac{\rho_l \ dz}{\left(\frac{R}{\cos\phi }\right)^{2}} =  \frac{1}{4\pi\epsilon_0} \frac{\rho_l\cos^{2}\phi \ dz }{R^{2}}
    \end{gather*}
    Dunque se si volesse determinare il campo lungo $z$ si dovrebbe integrare lungo tutto il filo:
    \begin{gather*}
        E_z^{TOT} = \int_{-\infty }^{\infty } \frac{\rho_l}{4\pi\epsilon_0} \frac{\cos^{2}\phi(z)}{R^{2}} \sin\phi(z) \ dz = \int_{-\frac{\pi}{2}}^{\frac{\pi}{2}} = \frac{\rho_l}{4\pi\epsilon_0} \frac{1}{R}\sin\phi \ d\phi = 0 
    \end{gather*} 
    Per trovare la soluzione si esegue un cambio di variabile secondo la definizione di $z$:
    \begin{gather*}
        z = R\tan\phi \ \Longrightarrow \ dz = \frac{R}{\cos^{2}\phi} \ d\phi
    \end{gather*}
    Dunque il campo elettrico totale sul punto $P$ generico sarà solamente lungo l'asse $x$:
    \begin{gather*}
        E_x^{TOT} = \int_{-\frac{\pi}{2}}^{\frac{\pi}{2}} \frac{\rho_l}{4\pi\epsilon_0} \frac{\cos^{3}\phi}{R^{2}} \frac{R}{r\cos^{2}\phi}  = \frac{\rho_l}{2\pi\epsilon_0 R}
    \end{gather*}
\end{example}

\begin{example}[Campo elettrico generata da una spira circolare]
    Si determina il campo elettrico generato da una spira di raggio $R$
    \begin{gather*}
        \begin{tikzpicture}
            \draw(0, 0) ellipse (1.25 and 0.5);
            \draw[->](0, -1) -- (0, 3) node[at end, left] {$x$};
            \draw(0, 0) -- (1.25, 0) node[midway, above] {$R$};
            \draw[dashed](1.25, 0) -- (0, 2.5) -- (-1.25, 0);
            \draw(0, 1.5) arc(270:295:1) node[midway, below] {$\alpha$};
        \end{tikzpicture}
    \end{gather*}
    Dato che la spira ha una distribuzione lineare di carica $\rho_l$, si può osservare che
    il campo elettrico sarà distribuito solamente lungo l'asse $x$. A questo punto si può determinare 
    il campo elettrico lungo $x$:
    \begin{gather*}
        dE_x = \frac{\rho_l dl}{4\pi\epsilon_0} \frac{1}{x^{2} + R^{2}} \cos\alpha 
    \end{gather*}
    Dunque il campo elettrico totale lungo l'asse $x$ sarà dato da
    \begin{gather*}
        E_x^{TOT} = \int_{SPIRA} dE_x = \frac{\rho_l \cos\alpha}{4\pi\epsilon_0 (x^{2} + R^{2})}\int_{SPIRA}  dl = \frac{\rho_l \cos\alpha}{2\epsilon_0} \frac{R}{x^{2} + R^{2}}
    \end{gather*}
    Per come è costruita la spira, il coseno dell'angolo 
    \begin{gather*}
        \cos\alpha = \frac{x}{\sqrt{x^{2} + R^{2}} }
    \end{gather*}
    Con questa sostituzione, ad una generica distanza $x$ dal centro della spira, il campo elettrico è dato 
    dalla seguente espressione:
    \begin{gather*}
        E_x = \frac{\rho_l}{2\epsilon_0} \frac{xR}{(x^{2} + R^{2})^{\frac{3}{2}}}
    \end{gather*}
    Se $x >> R$ va come $x^{2}$, avendo un massimo a $\frac{R}{\sqrt{2} }$ ed un minimo
    a $\frac{-R}{\sqrt{2} }$, con dipendenza lineare tra questi due punti. Inoltre, in zero, 
    come ci si sarebbe aspettati, è zero. 
\end{example}


\end{document}
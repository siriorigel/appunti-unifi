\documentclass[a4paper, oneside]{article}
\usepackage{graphicx}
\usepackage{amsthm}
\usepackage{amsmath}
\usepackage{amssymb}
\usepackage[a4paper,
            bindingoffset=0.2in,
            left=2cm,
            right=2cm,
            top=2cm,
            bottom=2cm,
            footskip=.25in]{geometry}
\usepackage[italian]{babel}
\usepackage{pgfplots}
\usepackage{tabularx}
\usepackage{tikz}
\usepackage{wrapfig}
\usepackage{color}
\usepackage[d]{esvect}
\definecolor{page}{rgb}{0.129,0.157,0.212}
\pagecolor{page}
\color{white}
\graphicspath{ {./images/} }
\usetikzlibrary{shapes.geometric}
\usetikzlibrary{datavisualization}
\usetikzlibrary{datavisualization.formats.functions}
\usetikzlibrary{patterns}
\pgfplotsset{width=10cm,compat=1.18}

\title{Esercizi di Lab II}
\author{Tommaso Miliani}
\date{27-02-26}

\begin{document}
\newtheoremstyle{theoremEnv}
                {}          % Space above
                {}          % Space below
                {\slshape}  % Body font
                {}          % Indent amount
                {\bfseries} % Head font
                {.}         % Punctuation after head
                {\newline}  % Space after theorem head
                {}          % Theorem head spec
\theoremstyle{theoremEnv}

\newtheorem{definition}{Definizione}[section]
\newtheorem{theorem}{Teorema}[section]
\newtheorem{lemma}{Proposizione}[section]
\newtheorem{observation}{Osservazione}[section]
\newtheorem{corollary}{Corollario}[theorem]
\newtheorem{example}{Esempio}[section]
\newtheorem{remark}{Enunciato}[section]

\maketitle

\section{Esercizi sul teorema di Gauss}
Se la simmetria del problema è tale per cui 
riduce la complessità del problema (identificando una superficie
che ha lo stesso campo elettrico), il teorema di Gauss è molto 
comodo per determinare il campo elettrico uscente; altrimenti
si dovrebbe utilizzare la sovrapposizione.
\begin{example}
    Si considera una sfera di raggio $R$ con distribuzione uniforme e costante
    di carica $\rho_0$: 
    \begin{gather*}
        \rho(x, y, z,) = \left\{\begin{array}{l}
            \rho_0 \qquad r \leq R \\
            0 \qquad r > R
        \end{array}\right.
    \end{gather*}
    Questo problema, la cui raffigurazione è la seguente:
    \begin{gather*}
        \begin{tikzpicture}
            \draw (0, 0) circle (1);
            \draw[->](0, 0 ) -- (2, 0) node[at end, below] {$\hat{x} $};
            \draw[->](0, 0) -- (0, 2) node[at end, left] {$\hat{y}$ };
            \draw[->](0, 0) -- (-1.5, -1 ) node[at end, left] {$\hat{z}$ };
            \draw[dashed, red](0, 0) circle (0.7);
            \draw[red](0, 0) -- (-0.5, 0.5) node[midway, below] {$r$};
        \end{tikzpicture}
    \end{gather*}
    Ha delle simmetrie evidenti. Il campo elettrico, ci si aspetta, che 
    sia una funzione di $r, \theta, \phi$:
    \begin{gather*}
        \vv{E}(r, \theta, \phi) = E_r(r, \theta, \phi)\hat{r} + E_\theta(r, \theta, \phi)\hat{\theta} + E_\phi(r, \theta, \phi)\hat{\phi}    
    \end{gather*}
    Data la simmetria ci si aspetta che il campo possa solo dipendere dalla distanza dal centro. 
    Dal punto di vista vettoriale ci si aspetta dunque che l'unica componente del campo
    sia solo 
    \begin{gather*}
        \vv{E}(r) = E_r(r) \hat{r}  
    \end{gather*}
    Dunque il campo in ogni punto della sfera di raggio piccolo $r$ con $r\leq R$ ci si aspetta
    che sia lo stesso da ogni direzione $\theta$ e $\phi$ lo si guardi. SI può allora determinare 
    il flusso del campo come 
    \begin{gather*}
        \Phi(\vv{E} ) = \int \vv{E}(r) \cdot \hat{n} \ d\Omega = E(r) \int d\Omega = E(r) 4\pi r^{2}  
    \end{gather*}
    Il flusso, per il teorema di Gauss, è anche equivalente a
    \begin{gather*}
        \frac{1}{\epsilon_0} \int_{V} \rho(R) dV = \frac{\rho_0}{\epsilon_0} \int_{V} dV = \frac{\rho_0}{\epsilon_0} \frac{4}{3}\pi r^{3}
    \end{gather*}
    Ossia il volume della superficie rossa quando $r = R$. Dunque 
    per il teorema di Gauss si possono eguagliare e ottenere:
    \begin{gather*}
        \frac{\rho_0}{\epsilon_0} \frac{4}{3}\pi r^{3} = E(r) 4\pi r^{2}
    \end{gather*}
    Risolvendo ora per $E(r)$ si ottiene 
    \begin{gather*}
        E(r) = \frac{1}{3}\frac{\rho_0}{\epsilon_0}r
    \end{gather*}
    Adesso, per il caso in cui $r > R$, vale il teorema di Gauss, per cui si ha che il 
    flusso totale della sfera deve valere
    \begin{gather*}
        \Phi(E ) = E(r)4\pi r^{2} 
    \end{gather*}
    Quello che non cambia è la carica interna: infatti la carica rimane confinata dentro
    la sfera di raggio $R$, dunque la carica totale in questo caso è esattamente funzione di $R$ (dunque
    costante):
    \begin{gather*}
        \frac{Q_{int}}{\epsilon_0} = \int_{V} \frac{\rho(r)}{\epsilon_0} \ dV = \frac{1}{\epsilon_0} \int_{0}^{r} \rho(r) 4 \pi r^{2} \ dr' = \frac{1}{\epsilon_0} \int_{0}^{r} \rho(r')4\pi r^{'2} \ dr' = \frac{\rho_0}{\epsilon_0} \frac{4\pi}{3} R^{3}
    \end{gather*}
    Dunque il campo elettrico sarà dato da 
    \begin{gather*}
        E(r) 4\pi r^{2} = \frac{\rho_0}{\epsilon_0} \frac{4\pi}{3} R^{3} \ \Longrightarrow \ E(r) = \frac{1}{3} \frac{\rho_0}{\epsilon_0} \frac{R^{3}}{r^{2}}
    \end{gather*}
    Se si volesse plottare questa funzione 
    \begin{gather*}
        \begin{tikzpicture}
            \draw[->](0, 0) -- (5, 0) node[at end, below] {$\hat{x}$ };
            \draw[->](0, 0) -- (0, 3) node[at end, left] {$\hat{y}$ };
            \draw[domain=2:5] plot (\x, {(5 * 3 * 3 * 3) / (3 * 5.57 * \x * \x)});
            \draw[domain=0:2] plot (\x, {\x});
            \draw[dashed, cyan](2, 0) -- (2, 2) node[at start, below] {$R$};
        \end{tikzpicture}
    \end{gather*}
\end{example}

\begin{example}[Campo elettrico di un filo infinito]
    Si considera un filo infinito con distribuzione di carica $\rho_L$ lineare
    utilizzando il teorema di Gauss. 
    \begin{gather*}
        \begin{tikzpicture}
            \draw[thick, ->](0, 0) -- (0, 4) node[at end, above] {$\hat{z}$ };
            \draw[red](-1, 0) -- (-1, 4);
            \draw[red](1, 0) -- (1, 4);
            \draw[red](0, 0) ellipse (1 and 0.3);
            \draw[red](0, 4) ellipse (1 and 0.3);
            \draw[red, <->](-1.25, 0) -- (-1.25, 4) node[midway, left] {$L$};
            \draw[->](0, 2) -- (3, 0) node[at end, below] {$\hat{x}$ };
        \end{tikzpicture}
    \end{gather*}
    Si cercano innanzitutto le simmetrie del problema: ossia la simmetria cilindrica in 
    questo caso, per cui l'espressione del campo elettrico dipenderà solamente dal versore 
    radiale:
    \begin{gather*}
        \vv{E} = E_r(r, \theta, z) = E_r(r, \theta, z) \hat{r} + E_\theta(r, \theta, z) \hat{\theta} + E_z(r, \theta, z) \hat{z} = E_r(r)    
    \end{gather*}
    Questo è verificabile con il principio di sovrapposizione (per $\theta$) e per
    l'invarianza traslazionale (per $z$ in quanto il filo è infinito). Dunque si può trovare
    il flusso del campo come 
    \begin{gather*}
        \Phi_r = \int_{\tau}^{} E_r(r) \hat{r} \cdot \hat{n} \ d\sigma   
    \end{gather*}
    Il contributo delle basi del cilindro è nullo poiché $\hat{n}$ e $\hat{r}$ sono 
    perpendicolari, mentre il contributo sul lato non è nullo ed è dato da
    \begin{gather*}
        E(r)\int d\sigma = E(r) 2 \pi r L 
    \end{gather*}
    Dove $L$ è la lunghezza del cilindro e $r$ il raggio del cilindro. Per il teorema di Gauss deve equagliare a
    \begin{gather*}
        \frac{Q_{int}}{\epsilon_0} = \frac{L \rho_L}{\epsilon_0} = \Phi_r
    \end{gather*}
    Dunque si ottiene il campo elettrico come 
    \begin{gather*}
        E_r = \frac{1}{2\pi \epsilon_0} \frac{\rho_L}{r}
    \end{gather*}
\end{example}

\begin{example}[Piano uniformemente carico]
    Dato un piano uniformemente carico (infinito), con densità superficiale 
    di carica $\sigma$, si vuole determinare il campo elettrico di questa superficie piana. 


    Data la simmetria del problema, ci si aspetta che il campo elettrico 
    sia diretto lungo l'asse $\hat{x}$ (verificabile dal fatto che il piano 
    è infinito e dal principio di sovrapposizione). Consideriamo comunque il campo 
    in funzione delle tre variabili cartesiane. Si deve determinare ora il flusso 
    attraverso una superficie finita del piano, come un cilindro 
    perpendicolare uscente dal piano. Le normali sul lato del cilindro sono 
    espresse in coordinate polari, mentre quelle della base sono dirette lungo $x$.
    \begin{gather*}
        \Phi = \int_{\text{Base } 1}\vv{E} \cdot \hat{n} \ d\Sigma + \int_{\text{Base }2} + \int_{\text{Lato}}       
    \end{gather*}  
    Si sa che il lato ha dunque contributo zero, mentre il versore normale $\hat{n} = \hat{x}$ per 
    entrambe le basi. DUnque 
    \begin{gather*}
        (1)  \quad \hat{n} = \hat{x} \quad \vv{E} = E\hat{x} \\
        (2) \ \ \hat{n} = -\hat{x}  \ \ \vv{E}= -E\hat{x}       
    \end{gather*}  
    Il prodotto scalare dei due ha lo stesso segno e dunque i contributi si
    sommano. Da qui si può considerare solo l'integrale sulla prima base:
    \begin{gather*}
        \Phi = 2\int d\Sigma = 2 E_x \ d\Sigma 
    \end{gather*}
    Dove $\Sigma$ è la superficie della base del cilindro considerato. Utilizzando il teorema di Gauss, si trova 
    la carica 
    \begin{gather*}
        \frac{Q_{int}}{\epsilon_0} = d\Sigma \sigma \ \Longrightarrow \ E_x = \frac{\sigma}{2\epsilon_0}
    \end{gather*}
    Si otterrebbe lo stesso risultato se si avesse utilizzato il principio di sovrapposizione. 
\end{example}

\begin{example}[Esercizio 1 parziale 4/4/2014]
    Data una sfera con densità di carica $\rho(r)$ è così definita:
    \begin{gather*}
        \rho(r) = \left\{\begin{array}{l}
            \frac{2\rho_0 r}{R} - \rho_0 \quad r \leq R \\
            0 \qquad \qquad \ \ r > R
        \end{array}\right.
    \end{gather*}
    La carica è dunque negativa fino a $\frac{R}{2}$ e poi positiva fino a $R$. La carica 
    totale tuttavia non è nulla: anche se la distribuzione di carica è lineare, bisogna considerare
    la simmetria sferica del problema:


    Calcolando la carica interna, 
    \begin{gather*}
        Q_{in} = \int_{0}^{R}\rho(r) \ dV = \int_{0}^{\infty } \rho(r) 4\pi r^2 \ dr 
    \end{gather*}
    Dunque si ottiene 
    \begin{gather*}
        \int_{0}^{R} \left(\frac{2\rho_0 r}{R} - \rho_0\right) 4\pi r^{2}\ dr = \int_{0}^{R} 4\pi r^{3} \frac{2\rho_0}{R} \ dr - 4\pi\rho_0 \int_{0}^{R} r^{2} \ dr 
    \end{gather*}
    Dunque si risolvono:
    \begin{gather*}
        Q_{in} = 2\pi\rho_0 R^{3} - \frac{4}{3} \pi R^{3} \rho_0  = \frac{2}{3}\pi\rho_0 R^{3}
    \end{gather*}
    Che è, come ci si aspettava, maggiore di zero. Se $r > R$:
    \begin{gather*}
        4\pi R^{2} E(r) = \frac{1}{\epsilon_0} \frac{2}{3}\pi\rho_0 R^{3} \ \Longrightarrow \ E(r) = \frac{1}{6\epsilon_0} \rho_0\frac{R^{3}}{r^{2}}
    \end{gather*}
    Se $r \leq R$, la carica interna va integrata da $0$ a $r$:
    \begin{gather*}
        4\pi r^{2} E(r) = \frac{1}{\epsilon_0} \int_{0}^{r} \rho_0(r) 4 \pi r^{'2} \ dr' = \frac{1}{\epsilon_0} \int_{0}^{r} \left(\frac{2\rho_0 r'}{R} - \rho_0\right) 4 \pi r^{'2} dr'  
    \end{gather*}
    Ora si risolve:
    \begin{gather*}
        \int_{0}^{r} \left(\frac{2\rho_0 r'}{R} - \rho_0\right) 4 \pi r^{'2} dr'  = \frac{2\pi \rho_0}{R} \int_{0}^{r} 4\pi r^{3} dr' - 4\pi \rho_0 \int_{0}^{r} r^{'2} \ dr' = \frac{2\rho_0}{R}\pi R^{4} - \frac{4}{3}\pi \rho_0 R^{3}  = 2\pi \rho_0  \left(\frac{r^{4}}{R} - \frac{2}{3}r^{3}\right)
    \end{gather*}
    Dunque si ottiene il campo elettrico come
    \begin{gather*}
        E(r) = \frac{\rho_0}{2\epsilon_0} \left(\frac{r^{2}}{R} - \frac{2}{3}r\right)
    \end{gather*}
\end{example}

\begin{example}[Esercizio 1 totale 2 / 2 / 2014]
    Si determini il campo elettrico di una sfera con distribuzione di carica 
    \begin{gather*}
        \rho_0 = \left\{\begin{array}{l}
            0 \qquad r \leq R_1 \\
            \rho_0 \quad R_1 < r \leq R_2 \\
            0 \qquad r > R_2
        \end{array}\right.
    \end{gather*}
    In questo esercizio il metodo più veloce è utilizzare il principio di sovrapposizione
    in quanto questo è il tipico problema del guscio sferico: si esegue dunque 
    la somma algebrica di una sfera carica positivamente ($E_1$) e il guscio interno carico
    negativamente ($E_2$). 
    \begin{gather*}
        E = E_1 + E_2
    \end{gather*}
\end{example}

\begin{example}[Sfera con guscio buco non concentrico]
    Se la sfera avesse un buco non concentric, allora non si potrebbe 
    applicare il principio di Gauss a causa dell'assenza di simmetrie. 
    

    Il campo elettrico nel generico punto $P$ interno alla sfera è 
    indicato dalle coordinate
    \begin{gather*}
        \vv{P} = \vv{r} \qquad \left| \vv{r}  \right|   = \sqrt{a^{2} + r^{'2}} 
    \end{gather*}
    Il campo della sfera carica totale (cioè senza buco) è uguale a
    \begin{gather*}
        \vv{E_1} (r) = \frac{r}{3} \frac{\rho_0}{\epsilon_0}\hat{r} 
    \end{gather*}
    Per il buco, la carica è $-\frac{4}{3}\pi r^{'3} \rho_0$, dunque
    \begin{gather*}
        \vv{E_2}(r') = -\frac{\rho_0}{3\epsilon_0} r' \hat{r'} 
    \end{gather*}
    Si considera adesso $\vv{r} = \vv{r'} + \vv{a}$.    
    Si può dunque scrivere che 
    \begin{gather*}
        \vv{E_1}(r)  = \frac{\rho_0}{3\epsilon_0} (\vv{r'} + \vv{a}  ) 
    \end{gather*}
    Facendo ora la somma 
    \begin{gather*}
        \vv{E} = \frac{\rho_0}{3\epsilon_0} \vv{a}  
    \end{gather*}
    Il campo è orientato nella direzione tra il centro della sfera grande e 
    quello della sfera piccola. 
\end{example}

\end{document}
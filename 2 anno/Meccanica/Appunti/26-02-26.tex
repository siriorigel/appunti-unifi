\documentclass[a4paper, oneside]{article}
\usepackage{graphicx}
\usepackage{amsthm}
\usepackage{amsmath}
\usepackage{amssymb}
\usepackage[a4paper,
            bindingoffset=0.2in,
            left=2cm,
            right=2cm,
            top=2cm,
            bottom=2cm,
            footskip=.25in]{geometry}
\usepackage[italian]{babel}
\usepackage{pgfplots}
\usepackage{tabularx}
\usepackage{tikz}
\usepackage{wrapfig}
\usepackage{color}
\usepackage[d]{esvect}
\definecolor{page}{rgb}{0.129,0.157,0.212}
\pagecolor{page}
\color{white}
\graphicspath{ {./images/} }
\usetikzlibrary{shapes.geometric}
\usetikzlibrary{datavisualization}
\usetikzlibrary{datavisualization.formats.functions}
\usetikzlibrary{patterns}
\pgfplotsset{width=10cm,compat=1.18}

\title{Appunti di Meccanica}
\author{Tommaso Miliani}
\date{26-02-26}

\begin{document}
\newtheoremstyle{theoremEnv}
                {}          % Space above
                {}          % Space below
                {\slshape}  % Body font
                {}          % Indent amount
                {\bfseries} % Head font
                {.}         % Punctuation after head
                {\newline}  % Space after theorem head
                {}          % Theorem head spec
\theoremstyle{theoremEnv}

\newtheorem{definition}{Definizione}[section]
\newtheorem{theorem}{Teorema}[section]
\newtheorem{lemma}{Proposizione}[section]
\newtheorem{observation}{Osservazione}[section]
\newtheorem{corollary}{Corollario}[theorem]
\newtheorem{example}{Esempio}[section]
\newtheorem{remark}{Enunciato}[section]

\maketitle

\section{Parametrizzazione e curve regolari}
L'idea è sceglier  un tempo $t$ ed un istante successivo $\Delta t$ per ottenere da una
qualsiasi curva $\vv{r}(t)$ il vettore differenza tra i vettori posizioni $\vv{r}(t + \Delta t)$
e $\vv{r}(t)$:  
\begin{gather*}
    \vv{r}(t + \Delta t) - \vv{r}(t)   
\end{gather*} 
Dunque si può ottenere il rapporto incrementale tra le due posizioni:
\begin{gather*}
    \frac{d\vv{r} }{dt} = \lim_{\Delta t \to 0} \frac{\vv{r}(t + \Delta t) - \vv{r}(t) }{\Delta t}
\end{gather*}
Quando questo limite esiste (se si da sufficiente regolarità alle funzioni 
basta che siano $C^{1}$). Si nota che 
\begin{gather*}
    \lim_{\Delta t \to 0} = \frac{\hat{i}\left(r_x( t+ \Delta t) - r_x (t)\right) + \hat{j} \left(r_y (t + \Delta t) - r_y (t)\right) + \hat{k} \left(r_z(\Delta t + t) - r_z (t)\right)   }{\Delta t}
\end{gather*}
Se le funzioni sono regolari, allora si può ottenere 
\begin{gather*}
    \frac{d\vv{r} }{dt} = \hat{i}\frac{dx}{dt} + \hat{j} \frac{dy}{dt} + \hat{k} \frac{dz}{dt}   
\end{gather*}
Per ricordare che è una derivata rispetto al tempo si usa $\dot{x}, \dots$. L'altra grande protagonista 
della fisica è l'accelerazione:
\begin{gather*}
    \vv{a} =  \frac{d^{2}\vv{r} }{dt^{2}} = \hat{x}\ddot{x} + \hat{j}\ddot{y} + \hat{k}\ddot{z}   
\end{gather*}
Si dice che una curva $\vv{r}(t)$ si dice \textbf{regolare} se è regolare 
come funzione matematica ossia se $\vv{r}(t) \in (C^{1}(\mathbb{R}))^{3}$. Inoltre 
si vuole anche che 
\begin{gather*}
    \left| \frac{d\vv{r} }{dt} \right|  \neq 0
\end{gather*}  
Esiste dunque una differenza tra regolarità geometrica e regolarità analitica. Ovviamente
dire che il modulo si annulla vuol dire che tutte le componenti si annullano. Una proprietà che 
tornerà comoda è la seguente:
\begin{gather*}
    \hat{u}(t) \text{ versore} 
\end{gather*}
Che può variare nel tempo e nella direzione, ha modulo costante, ma se $t$ cambia si orienta 
in direzione diversa. Vale dunque la seguente semplice proprietà: 
\begin{gather*}
    \left< \frac{d\hat{u} }{dt}, \hat{u}  \right> = 0 
\end{gather*}
Inoltre
\begin{gather*}
    \left| \hat{u}  \right|^2 = 1 
\end{gather*}
Si ricorda poi il legame tra modulo e prodotto scalare
\begin{gather*}
    \left< \hat{u}, \hat{u}   \right> = \left| \hat{u}  \right|  ^{2} = 1
\end{gather*}
Se si deriva questa espressione:
\begin{gather*}
    0 = \frac{d}{dt}\left< \hat{u}, \hat{u}   \right> = \left< \frac{d\hat{u} }{dt}, \hat{u}  \right> + \left< \hat{u}, \frac{d\hat{u} }{t}  \right> = 2\left< \frac{d\hat{u} }{dt}, \hat{u}  \right>    
\end{gather*}
poiché vale la regola di Leibniz,
\begin{example}[Esempio veloce]

\end{example}
Data una curva regolare 
\begin{gather*}
    \vv{r}(t) = \hat{i}t^{2} + t^{3}\hat{y}   
\end{gather*}
La parabola viene definita a partire da questo parametro:
\begin{gather*}
    \begin{array}{l}
        x(t) = t^{2} \ \Longrightarrow \ t = \pm \sqrt{x}  \\
        y(t) = t^{3} = x^{\frac{3}{2}} = \pm x^{\frac{3}{2}}
    \end{array}
\end{gather*}
Per vedere se è regolare devo valutare la derivata:
\begin{gather*}
    \frac{dr}{dt} = 2t\hat{i} + 3t^{2}\hat{j}  
\end{gather*}
Devo vedere se il vettore che definisce la curva non si annulla:
\begin{gather*}
    \left| \frac{d\vv{r} }{dt} \right|^2 = 4t^{2} + 9t^{4} 
\end{gather*}
Dunque costituisce una mappa lineare $y = x^{2}  +x^{3}$.  Andando a 
creare una cuspide in $x = 0$. 

\begin{theorem}
    La regolarità di una curva non dipende dalla parametrizzazione. 
\end{theorem}
\begin{proof}
    Presa una curva con parametrizzazione $r(t)$:
    \begin{gather*}
        t(t_1) \to r_1(t_1) = r(t(t_1))
    \end{gather*}
    Applicando la derivazine della funzione composta:
    \begin{gather*}
        \frac{d\vv{r_1} }{dt_1} = \left| \frac{d\vv{r} }{dt} \right|  \left| \frac{dt}{dt_1} \right| 
    \end{gather*}
    Se la parametrizzazione è regolare, allora 
    \begin{gather*}
        \left| \vv{t_1}  \right|  = \left| \vv{t}  \right| 
    \end{gather*}
    Dunque per definizione di regolarità mi permette di definire che 
    la regolarità della curva non dipende dalla parametrizzazione ed è 
    una proprietà intrinseca della curva perché la parametrizzazione è solo
    un modo per descrivere la curva.
\end{proof}

\section{Lunghezza di una curva}
Si può procedere adesso a definire la lunghezza di una curva (regolare).
Il metodo intuitivo per costruire una curva è spezzettare la curva in 
piccoli pezzi dritti (poiché so determinarne la lunghezza con la 
norma euclidea). Dunque la lunghezza della spezzata è la somma dei pezzi: 
\begin{gather*}
    l(r(t_A), r(t_B)) =  \int_{t_A}^{t_B}\left| \frac{dr}{dt} \right|   \ dt 
\end{gather*}
Ossia integrando il modulo del versore tangente: dunque si ottiene 
la seguente espressione



\end{document}
\documentclass[a4paper, oneside]{article}
\usepackage{graphicx}
\usepackage{amsthm}
\usepackage{amsmath}
\usepackage{amssymb}
\usepackage[a4paper,
            bindingoffset=0.2in,
            left=2cm,
            right=2cm,
            top=2cm,
            bottom=2cm,
            footskip=.25in]{geometry}
\usepackage[italian]{babel}
\usepackage{pgfplots}
\usepackage{tabularx}
\usepackage{tikz}
\usepackage{wrapfig}
\usepackage{color}
\usepackage[d]{esvect}
\definecolor{page}{rgb}{0.129,0.157,0.212}
\pagecolor{page}
\color{white}
\graphicspath{ {./images/} }
\usetikzlibrary{shapes.geometric}
\usetikzlibrary{datavisualization}
\usetikzlibrary{datavisualization.formats.functions}
\usetikzlibrary{patterns}
\pgfplotsset{width=10cm,compat=1.18}

\title{Appunti di Meccanica}
\author{Tommaso Miliani}
\date{24-02-26}

\begin{document}
\newtheoremstyle{theoremEnv}
                {}          % Space above
                {}          % Space below
                {\slshape}  % Body font
                {}          % Indent amount
                {\bfseries} % Head font
                {.}         % Punctuation after head
                {\newline}  % Space after theorem head
                {}          % Theorem head spec
\theoremstyle{theoremEnv}

\newtheorem{definition}{Definizione}[section]
\newtheorem{theorem}{Teorema}[section]
\newtheorem{lemma}{Proposizione}[section]
\newtheorem{observation}{Osservazione}[section]
\newtheorem{corollary}{Corollario}[theorem]
\newtheorem{example}{Esempio}[section]
\newtheorem{remark}{Enunciato}[section]

\maketitle

\section{Introduzione al corso}
\begin{itemize}
    \item Prof: Omar Morandi
    \item Esame: 2 parziali con possibilità di recuperarne uno, 
    altrimenti si sostiene un esame scritto totale + orale. Si possono
    portare libri e appunti (i compitini valgono fino a settembre). L'orale è 
    un colloquio (più teorico) anche se ci possono essere esercizi. Si può accedere
    all'orale con media dei compiti anche con 14-17.
    \item Testi: "Meccanica Analitica" di Marmi, Fasano
    \item 
\end{itemize}


\section{Introduzione}
L'idea è capire il moto di sistemi meccanici complessi come strutture articolate 
rigide. La descrizione tramite metodi matematici è il motivo per il quale prende il nome 
di meccanica analitica (che è simile alla meccanica razionale). Il termine analitica deriva 
dal fatto che si utilizzano funzioni lagrangiane e hamiltoniane.
Ci si concentra su oggetti meccanici tipo molle, aste, ruote, riformulato 
in una formulazione più generale, elegante e completa rispetto al corso di fisica I. 

\subsection{Richiami da fisica I e geometria}
Si utilizzeranno per determinate cose anche notazioni diverse, il motivo è 
che ogni notazione ha una sua definizione e un suo utilizzo. 
Vettori
\begin{gather*}
    \vv{r}  = \begin{pmatrix} x\\
    y\\
    z \end{pmatrix}  = (A - O)  \qquad \left| \vv{r}  \right| \doteq \sqrt{x^{2} + y^{2} + z^{2}}  
\end{gather*}
Per cui si definisce il prodotto scalare come 
\begin{gather*}
    \alpha \in \mathbb{R} \qquad \alpha \vv{v} = \alpha v_1 \hat{1} + \dots + \alpha v_k \hat{k}   
\end{gather*}
Si ottiene il versore di un vettore dividendo il vettore per il suo modulo:
\begin{gather*}
    \hat{v} = \frac{\vv{v} }{\left| v \right| } 
\end{gather*}
Il suo prodotto scalare è definito come la mappa lineare che 
\begin{gather*}
    \left< \vv{v}, \vv{w}   \right>  = \vv{v} \cdot \vv{w} = \left| v \right| \cdot \left| w \right| \cos(\vv{v} \angle \vv{w} )  \in \mathbb{R}  
\end{gather*}
L'angolo tra due vettori non è mai definito in quanto due vettori formano sempre un piano e dunque esistono
due angoli tra loro: uno interno ed uno esterno, anche se, per il coseno, la definizione risulta non ambigua. 
\begin{gather*}
    \left| \vv{v}  \right| = \sqrt{\left< \vv{v}, \vv{v}   \right> }  
\end{gather*}
Bla bla bla calcolo vettoriale

\begin{example}
    Presa un asta di lunghezza $l$ vincolata ad avere i suoi due estremi appartenenti 
    all'asse $x$ e all'asse $y$:
    \begin{gather*}
        \begin{tikzpicture}
            \draw[->](0, 0) -- (3, 0) node[at end, below] {$x$};
            \draw[->](0, 0) -- (0, 3) node[at end, left] {$y$};
            \draw[->](0, 0) -- (1, 1);
            \filldraw(1, 1) circle (1pt) node[anchor = south west ] {$C$};
            \draw(0, 2) -- (2, 0) node[midway, above] {$l$};
        \end{tikzpicture}
    \end{gather*}
    Conoscendo l'angolo $\alpha$, quanto valgono le coordinate del centro della scala? \\ \noindent
    Si determina adesso il vettore posizione come 
    \begin{gather*}
        (C - O) = (A - O)  + (C - A) = \hat{j}l\cos\alpha + \frac{l}{2}\sin\alpha \hat{i} -  \frac{l}{2}\sin\alpha \hat{i} 
    \end{gather*}
\end{example}

\begin{example}
    Si ha un asta che è vincolata ad un perno, come una biella attaccata ad un 
    albero motore:
    \begin{gather*}
        \begin{tikzpicture}
        \draw[->](0, 0) -- (5, 0) node[at end, below] {$x$};
        \draw(0.5, 0) -- (2.5, 0.85) node[midway, above] {$l$};
        \draw(2.5, 0.85) -- (3, 0);
        \draw[dashed](3, 0) circle (1);
        \draw(1.7, 0.5) arc (30:0:1) node[midway, left] {$\theta$};
        \draw(3.4, 0) arc (0:125:0.4) node[midway, above] {$\alpha$};
        \filldraw(0.5, 0) circle (1pt) node[anchor = north] {A};
        \filldraw(2.5, 0.85) circle (1pt) node[anchor = south] {B};
        \filldraw(3, 0) circle (1pt) node[anchor = north] {C};
        \draw(3, 0) -- (3.82, -0.5) node[midway, below] {$R$};
        \filldraw(0, 0) circle(1pt) node[anchor = north] {O};
    \end{tikzpicture} 
    \end{gather*}
    Si deve determinare la posizione di A in funzione dell'angolo $\alpha$ in figura. 
    Si ricava dunque 
    \begin{gather*}
        (A  - O) = (B - O) + (A - B)
    \end{gather*}
    Si determinano ora i vettori che descrivono la posizione di A:
    \begin{gather*}
        (A - B) = -l(\cos\theta \hat{i} + \sin\theta \hat{j}  )\\
        (B - O) = (B - C) - (C - O) =  R(\hat{i}\cos\alpha + \hat{j}\sin\alpha  ) +d\hat{i} 
    \end{gather*}
    Dato che $\theta$ ed $\alpha$ non sono indipendenti (ancora non si sa che 
    cosa vuol dire però è intuibile facendo variare uno e bloccando l'altro: se l'oggetto 
    si rompe, allora erano dipendenti), devo adesso ricondurre tutto in funzione
    di $\alpha$. Utilizzando ora il teorema dei seni, si può ricavare il legame che 
    intercorre tra i due angoli:
    \begin{gather*}
        \frac{R}{\sin\theta} \frac{1}{\sin(\pi -\alpha)}
    \end{gather*} 
    Teorema dei seni 
    \begin{gather*}
        \frac{a}{\sin\alpha} = \frac{b}{\sin\beta} = \frac{c}{\sin\gamma}
    \end{gather*}
    E teorema dei coseni:
    \begin{gather*}
        \left| a \right|^{2} = \left| b \right|^{2} + \left| c \right|^{2} - 2\left| b \right| \left| c \right|\cos\alpha     
    \end{gather*}
\end{example}

\subsection{Introduzione alla meccanica analitica vera e propria}
Richiamo del teorema di derivazione di composte
\begin{theorem}
    $f(x) : U \subset R \to \mathbb{R}^{n}$, pensando $x$ come funizione di $y$, il quale 
    è definito su $V \subset R$,
    $x = x(y)$ funzione $R \to R$ , si definisce una funzione composta $h(y)$ definita come 
    \begin{gather*}
        h(y) \doteq f(x(y))
    \end{gather*}
    Si può determinare la derivata
    \begin{gather*}
        \frac{dh}{dy} = \left. \frac{df}{dx} \right| _{x(y)} \frac{dx}{dy}
    \end{gather*}
    Chiaramente una funzione composta. 
\end{theorem}
Si richiama anche il teorema del calcolo integrale:
\begin{theorem}
    \begin{gather*}
        \int_{y_1}^{y_2} h(y) \ dy = \int_{y_1}^{y_2} f(x(y)) \ dy = \int_{x(y_1)}^{x(y_2)} f(x) \left(\frac{dy}{dx}\right)  \ dx
    \end{gather*}

\end{theorem}

\section{Curve}
\subsection{Perché interessano}
Le curve permettono di descrivere le traiettorie (parametrizzandole rispetto al tempo)
di qualsiasi oggetti. L'insieme dei punti spaziali occupati dal mio oggetto rispetto 
al tempo costituiscono una curva. Concettualmente le traiettorie che 
si considerano sono sempre reali e possono essere curve in $\mathbb{R}^{3}$ o $\mathbb{R}^{2}$. 
Si vede anche che, una traiettoria in uno spazio $\mathbb{R}^{n}$ descrive uno spazio ?????.

\subsection{Curva in $\mathbb{R}^{3}$}
Una curva in $\mathbb{R}^{3}$ è una applicazione continua definita 
su di un intervallo aperto continuo e regolare:
\begin{align}
    \vv{r} (t) : U \subset \mathbb{R} \to \mathbb{R}^{3}
\end{align}
Dunque si può definire 
\begin{gather*}
    r(t) = \begin{pmatrix} x(t) \\
    y(t) \\
    z(t)\end{pmatrix} = x(t) \hat{i} + y(t) \hat{j} + z(t) \hat{k}   
\end{gather*}
Si considera adesos il seguente esempio 
\begin{example}[Esempio di curva e parametrizzazioni diverse]
    Cosideriamo una curva di raggio 1:
    \begin{gather*}
        \begin{tikzpicture}
            \draw[->](-2, 0) -- (2, 0) node[at end, below] {$x$};
            \draw(-1, 0) arc (180:0:1);
            \draw[->](0, 0) -- (0, 2) node[at end, left] {$y$};
            \draw(0, 0) -- (0.85, 0.5);
            \draw(0.6, 0) arc (0:30:0.6) node[midway, right] {$\theta$};
            \filldraw(0, 0) circle (1pt) node[anchor = north] {O};
        \end{tikzpicture}
    \end{gather*}
    Devo adesso parametrizzare la curva secondo un parametro:
    potrei scegliere $\theta$, per cui posso esprimere
    \begin{gather*}
        \vv{r} = \cos\theta \hat{i} + \sin\theta \hat{j} \qquad 0 < \theta < \pi   
    \end{gather*}
    Si può anche utilizzare un parametro $t$ per definire:
    \begin{gather*}
        \vv{r}(t) = t\hat{i} + \sqrt{1 - t^{2}}\hat{j} \qquad -1 < t < 1   
    \end{gather*}
    A livello formale queste due curve sono diverse. Questo perché le curve sono univocamente definite
    da un parametro: cambiando parametro si cambia curva anche se, 
    ovviamente, hanno il solito tracciato. Dal punto di vista cinematico conviene 
    considerarle come se fossero uguali. 
\end{example}
\noindent
Due parametrizzazioni diverse (due curve diverse), $r_1(t_1)$ e $r_2(t_2)$ sono
equivalenti se $\exists$ una relazione biuinivoca tra l'insieme di definizione, 
regolare, derivabile e con inversa derivabile e tale che
\begin{gather*}
    t_2(t_1) : U_1 \to U_2 \ | \ r_2(t_2(t_1)) = r_1(t_1) 
\end{gather*}
Da un punto di vista matematico sono equivalenti ed esprimono la stessa quantità con 
parametrizzazioni differenti. Nell'esempio di prima si definisce una relazione tra le due
parametrizzazioni nella seguente maniera: 
\begin{gather*}
    t = \cos\theta \ \text{per } \hat{i} \qquad \sqrt{1 - t^{2}} = \sin\theta \ \text{per } \hat{j}   
\end{gather*}
Dunque queste sono equivalenti. Devo adesso verificare che il segno della derivata 
sia uguale.


\end{document}
\documentclass[a4paper, oneside]{article}
\usepackage{graphicx}
\usepackage{amsthm}
\usepackage{amsmath}
\usepackage{amssymb}
\usepackage[a4paper,
            bindingoffset=0.2in,
            left=2cm,
            right=2cm,
            top=2cm,
            bottom=2cm,
            footskip=.25in]{geometry}
\usepackage[italian]{babel}
\usepackage{pgfplots}
\usepackage{tabularx}
\usepackage{tikz}
\usepackage{wrapfig}
\usepackage{color}
\usepackage[d]{esvect}
\definecolor{page}{rgb}{0.129,0.157,0.212}
\pagecolor{page}
\color{white}
\graphicspath{ {./images/} }
\usetikzlibrary{shapes.geometric}
\usetikzlibrary{datavisualization}
\usetikzlibrary{datavisualization.formats.functions}
\usetikzlibrary{patterns}
\pgfplotsset{width=10cm,compat=1.18}

\title{Appunti di Metodi}
\author{Tommaso Miliani}
\date{25-02-26}

\begin{document}
\newtheoremstyle{theoremEnv}
                {}          % Space above
                {}          % Space below
                {\slshape}  % Body font
                {}          % Indent amount
                {\bfseries} % Head font
                {.}         % Punctuation after head
                {\newline}  % Space after theorem head
                {}          % Theorem head spec
\theoremstyle{theoremEnv}

\newtheorem{definition}{Definizione}[section]
\newtheorem{theorem}{Teorema}[section]
\newtheorem{lemma}{Proposizione}[section]
\newtheorem{observation}{Osservazione}[section]
\newtheorem{corollary}{Corollario}[theorem]
\newtheorem{example}{Esempio}[section]
\newtheorem{remark}{Enunciato}[section]

\maketitle

\section{Introduzione al corso}
\begin{itemize}
    \item Prof: Panico 
    \item Risorse per lo studio: Dispense del corso (complete, vengono aggiornate periodicamente, quelle con * non sono necessarie
    per passare e non sono richieste), la grande maggioranza delle dimostrazioni 
    non saranno richieste all'orale. Eserciziario del corso con tutti i compiti degli anni passati.
    \item Esame: scritto (3 ore con 3 esercizi, uno per parte del corso)
    + orale che consiste in una discussione di argomenti (in modo più pratico) e 
    di alcune dimostrazioni. La validità dello scritto è illimitata. 
\end{itemize}
Si studia un estensione di geometria negli spazi vettoriali 
infiniti ed è necessaria per studiare meccanica quantistica. 
Si divide in tre parti 
\begin{enumerate}
    \item Analisi complessa: estensione del concetto di funzione, serie e integrale sul piano complesso
    \item Analisi armonica: sviluppo delle funzioni complesse in serie di seni e coseni
    e trasformate integrali (Fourier e Laplace), ossia l'estensione continua dei seni e coseni.
    \item Spazi di Hilbert: generalizzazione degli spazi vettoriali a dimensione infinita. 
\end{enumerate}



\section{Introduzione all'analisi complessa}
\subsection{Richiami dei numeri complessi e notazioni}
Tutta l'analisi complessa si basa sull'utilizzo dei numeri complessi. I numeri complessi
sono un estensione del campo reale: questa estensione è ottenuta mediante il seguente numero 
\begin{align}
    i^{2} = -1 
\end{align}
che prende il nome di \textbf{unità immaginaria}, che serve per estendere i numeri reali al
campo dei complessi. Un generico numero complesso $z$ si scrive in questa maniera
\begin{align}
    z = x + iy \qquad x, y \in \mathbb{R}
\end{align}
\begin{wrapfigure}{r}{0.4\textwidth}
    \centering
    \caption{La rappresentazione di un generico numero complesso $z = x + iy$}
    \begin{tikzpicture}[scale=1.2]
        \draw[->](-1, 0) -- (2, 0) node[at end, below] {$Re(z)$};
        \draw[->](0, -2) -- (0, 2) node[at end, left] {$Im(z)$};
        \draw[dashed](1, 0) -- (1, 1) node[at start, below] {$x$};
        \draw[dashed](0, 1) -- (1, 1) node[at start, left] {$y$};
        \draw[->, thick](0, 0) -- (1, 1) node[midway, above] {$\left| z \right|$ };
        \filldraw(1, 1) circle (1pt) node[anchor = west] {$z = x + iy$};
    \end{tikzpicture}    
\end{wrapfigure}
Generalmente valgono anche le stesse proprietà dei numeri reali, ricordando sempre 
che $i^{2} = -1$. All'interno dei numeri complessi esiste un sottogruppo che corrisponde esattamente 
ai numeri reali (se $y = 0$), dunque $\mathbb{R} \subset \mathbb{C}$. La rappresentazione 
dei numeri complessi avviene sul \textbf{piano complesso}, dunque $\mathbb{C} \cong \mathbb{R} \times \mathbb{R}$,
dunque ad ogni numero complesso corrisponde un punto sul piano complesso con ascissa $x$ e
ordinata $y$ per un complesso generico $z = z + iy$. \\ \noindent
Ai numeri complessi si associa il \textbf{norma, modulo, o valore assoluto}, che non è nient'altro 
che la lunghezza in figura 
\begin{align}
    \left| z \right| = \sqrt{x^{2} + y^{2}} \in \mathbb{R}_+  
\end{align}
Si definisce l'operazione di \textbf{coniugazione complessa}, come l'operazione 
che manda 
\begin{align}
    z \to z^{\star} \ (o \ \overline{z} ) = x - iy
\end{align}
Dunque la coniugazione complessa cambia il segno della parte immaginaria, mantenendo 
inalterata la parte reale che ha le seguenti proprietà:
\begin{itemize}
    \item $\left| z \right|^{2} = z \cdot z^{\star} = z^{\star}z = \left| z^{\star} \right|^{2}$
    \item $(z^{\star})^{\star} = z$.
    \item $\frac{1}{z} = \frac{z^{\star}}{z^{\star}z} = \frac{z^{\star}}{\left| z \right|^{2} }$ 
\end{itemize}
Un'altra rappresentazione molto utile dei complessi sono le \textbf{coordinate polari}: esse permettono 
di rappresentare un numero complesso identificandolo mediante il suo modulo e l'angolo rispetto 
all'asse reale.
\begin{align}
    z = x + iy = r(\cos\phi + i\sin\phi)
\end{align}
Se $r = 0$, l'angolo non è ben definito. Tuttavia, scritto in questa maniera, 
si può prendere qualsiasi valore di $\phi$, così si limita ad una sola volta la copertura 
del piano complesso con $\phi \in [0, \alpha + 2\pi)$. Dove $r = \left| z \right|$ e l'angolo 
è anche chiamato \textbf{argomento di $z$}. La rappresentazione polare permette di 
esprimere in modo semplice il prodotto di numeri complessi, dunque 
l'insieme dei complessi contiene naturalmente il prodotto tra numeri complessi:
\begin{gather*}
    z_1 \cdot z_2 = r_1 r_2 (\cos(\phi_1 + \phi_2) + i\sin(\phi_1 + \phi_2)) \quad z_1, z_2 \in \mathbb{C}
\end{gather*}
Si definiscono anche le potenze di numeri complessi:
\begin{gather*}
    z^{n} = r^{n}(\cos (n\phi) + i \sin(n\phi)) \qquad n \in \mathbb{Z}
\end{gather*}
L'estrazione di radici permette di ricavare tutte le soluzioni della seguente equazione
\begin{gather*}
    z^{\frac{1}{n}} \ \Longrightarrow \ w^{n} = z
\end{gather*}
Ossia tutti gli $w$ che risolvono tale equazione. Dunque si utilizza la notazione polare:
\begin{gather*}
    w = \rho(cos\theta + i\sin\theta) \qquad z =r^{n}(\cos(n\phi) + i\sin(n\phi))
\end{gather*}
La radice quadrata del modulo è ovvio, dunque $\rho = r^{\frac{1}{n}}$. Per identificare 
gli angoli dei vari numeri complessi, si potrebbe pensare a $n\theta = \phi$, tuttavia
sto perdendo la periodicità dell'angolo, per tenere conto di questo si ottiene che
\begin{gather*}
    n\theta = \phi + 2\pi \ \Longrightarrow \ \theta = \frac{\phi}{n} + \frac{2\pi k}{n}
\end{gather*}
Le soluzioni distinte per teta sono infinite, tuttavia le devo identificare e dunque 
devo scegliere $k \in [0, n - 1]$. Le radici di un numero complesso descrivono 
poligoni inscritti in un certo di raggio $\rho$.
Le radici di un numero complesso qualsiasi si ottengono moltiplicando una delle radici 
per l'insieme delle radici dell'unità. Questo perché le radici di un complesso 
qualsiasi sono la rotazione e la dilatazione di un dato poligono formato dalle radici 
dell'unità (facilmente dimostrabile). 

\subsection{Topologia e metrica di $\mathbb{C}$}
Una \textbf{topologia} è un insieme di intorni di un punto che si considera. Si introduce il
concetto di \textbf{convergenza} su $\mathbb{C}$, estendendo il concetto di 
convergenza su $\mathbb{R}$. Visto che gli elementi dei complessi sono elementi di 
$\mathbb{R}^{2}$, dunque si considerano i limiti della parte reale e della parte immaginaria,
se questi esistono, allora esistono anche su $\mathbb{C}$. Una successione di numeri complessi 
$\{z_n\}$ convergono se converge sia la parte reale che quella immaginaria:
\begin{gather*}
    \left\{\begin{array}{l}
        \lim_{n \to \infty } Re(z_n) = Re(z) \\
        \lim_{n \to \infty } Im(z_n) = Im(z)   
    \end{array}\right.
\end{gather*}
Equivalentemente 
\begin{gather*}
    \lim_{n \to \infty } \left| z_n - z \right| = 0  
\end{gather*}
Si possono vedere con due topologie differenti: la prima utilizza topologie 
rettangolari mentre la seconda utilizza le palle aperte e dunque sono equivalenti.
Dunque $\mathbb{C}$ è uno spazio metrico con la \textbf{distanza} definita dal modulo della 
differenza tra due numeri complesso $z_1$ e $z_2$:
\begin{gather*}
    d(z_1, z_2) = \left| z_1 - z_2 \right| 
\end{gather*}
Data questa distanza si definiscono gli insieme degli intorni di quel punto, 
ossia le palle aperte di quel punto. Dunque gli intorni di $z$ sono le collezioni di 
palle aperte con raggio $r = \left| z \right|$:
\begin{gather*}
    z : B(z, r)
\end{gather*} 
Se $z_n$ è di Cauchy, anche $x_n$ e $y_n$ sono di Cauchy.
Dunque $\mathbb{C}$ è uno \textbf{spazio completo}, ossia ogni successione di Cauchy ha limite.
Si può introdurre la \textbf{norma} per $\mathbb{C}$, in quanto sussiste l'isomorfismo
con $\mathbb{R}^{2}$, e dunque è uno spazio normato con 
\begin{align}
    \left\lVert z \right\rVert  = \left| z \right| 
\end{align}
Ossia la norma naturale su $\mathbb{R}^{2}$. Dunque anche $\mathbb{C}$ è uno spazio di Banach. 
Con $\mathbb{C}$ non esiste un ordinamento naturale dei numeri: l'isomorfismo con $\mathbb{R}^{2}$
non permette di mantenere questo ordinamento. Le uniche disuguaglianze che si possono fare
tra complessi coinvolgono il modulo. 

\section{Serie su $\mathbb{C}$}
Su $\mathbb{R}$ una serie di potenze converge in un dato intervallo. Si definiscono 
le successioni su $\mathbb{C}$ così come si è fatto per $\mathbb{R}$; si dice che una 
serie converge quando la successione delle somme parziali converge. 
\begin{definition}[Serie di potenze]
    Data una successione $\{z_n\}_{n \in \mathbb{N}} \in \mathbb{C}$, e si considerano 
    le \textbf{ridotte parziali} 
    \begin{gather*}
        \sum_{n = 0}^{N} z_n = S_N  
    \end{gather*}
    Se converge la successione delle ridotte parziali, allora si dice che la
    serie è convergente:
    \begin{gather*}
        \lim_{n \to \infty }S_N = S = \lim_{N \to \infty } \sum_{n = 0}^{N} z_n = \sum_{n = 0}^{\infty } z_n   
    \end{gather*}
\end{definition} \noindent
Così come su $\mathbb{R}$, per essere convergente esiste una condizione 
necessaria su $z_n$. Su $\mathbb{R}$ $z_n$ deve essere infinitesimo, dunque 
la successione dei moduli deve essere infinitesima.
\begin{theorem}[Condizione necessaria per la convergenza]
        La condizione necessaria ma non sufficiente per la convergenza è che
        $\left| z_n \right|$ sia una successione infinitesima:
        \begin{gather*}
            \lim_{n \to \infty } \left| z_n \right|   = 0
        \end{gather*} 
\end{theorem} \noindent
Si possono utilizzare dei criteri molto simili a quelli che si utilizzano
su $\mathbb{R}$, che si applicano alla successione dei moduli per ottenere 
la convergenza della serie. 
\begin{theorem}[criterio del rapporto]
    La serie $\sum_{n} z_n$ è convergente se 
    \begin{gather*}
        \exists \ p\  : \ 0 \leq p < 1, \ \overline{n} \in \mathbb{N}  \ : \ \left| \frac{z_{n + 1}}{z_{n}} \right| \leq p \quad \forall n \geq \overline{n} 
    \end{gather*} 
\end{theorem}
Il secondo criterio di convergenza è 
\begin{theorem}[Criterio delle radici ]
        La serie $\sum_{n} z_n$ è convergente se 
    \begin{gather*}
        \exists \ p\  : \ 0 \leq p < 1, \ \overline{n} \in \mathbb{N}  \ : \ \left| z_n \right|^{\frac{1}{n}}  \leq p \quad \forall n \geq \overline{n} 
    \end{gather*} 
\end{theorem}

\begin{definition}[Convergenza assoluta]
    La serie $\sum_{n} z_n$ converge assolutamente se 
    $\sum_{n} \left| z_n \right|$ converge. 
\end{definition}

\begin{theorem}[Conv assoluta $\ \Longrightarrow \ $ con standard]
    La convergenza assoluta implica la convergenza standard
\end{theorem} \noindent

\begin{theorem}[Criterio del confronto]
    Se una successione $\{z_n\}_{n \in \mathbb{N}}$ complessa è maggiorata in \textbf{modulo} 
    da una successione $\{a_n\}_{n \in \mathbb{N}}$ di numeri reali non negativi, la cui 
    serie associata converge, allora anche la successione complessa converge, e dunque 
    converge anche assolutamente. 
\end{theorem}

\section{Serie di potenze}
Sono delle potenze, così come sono state definite in $\mathbb{R}$, centrata 
in un qualsiasi punto di $\mathbb{C}$ :
\begin{align}
    \sum_{n = 0}^{\infty } a_n(z - z_0)^{n} \qquad
    \begin{array}{l}
        a_n \in \mathbb{C} \\
        z, z_0 \in \mathbb{C}
    \end{array} 
\end{align}
La potenza 0 è sempre interpretata come l'identità in una serie (ossia il caso banale).
Il caso in cui $z_0 = 0$ è il caso più semplice in quanto le conclusioni che si hanno per questo caso 
valgono anche per il caso in cui si trasli di un qualsiasi numero complesso $z_0$. Dunque
\begin{gather*}
    \sum_{n = 0}^{n} a_n z^{n} 
\end{gather*} 
Si definisce ora l'\textbf{insieme di convergenza}:
\begin{definition}
    L'insieme di convergenza è dato dall'insieme degli $\mathbb{C}_a \equiv \{z\} \in \mathbb{C}$
    tali per cui la somma $\sum_{n } a_n z^{n}$ converge. La notazione $\mathbb{C}_a$
    ci ricorda che ogni serie ha un insieme di convergenza ben definito e diverso da ogni altra 
    serie. 
\end{definition}
Si dimostra adesso il lemma di Abel che caratterizza l'insieme di convergenza di una serie.
\begin{lemma}[Lemma di Abel]
    Sia $w \in \mathbb{C}$ tale che $a_nw^{n}$ sia limitata. Allora $\forall z \in \mathbb{C}$ tale 
    che $\left| z \right| < \left| w \right|$, allora la serie $\sum_{n} a_nz^{n}$ converge 
    assolutamente   
\end{lemma}
\begin{proof}
    Si utilizza il teorema del confronto costruendo una serie convergente a partire 
    dalla serie $a_nw^{n}$. Si ha che
    \begin{gather*}
        \left| a_n w^{n} \right| \leq M  
    \end{gather*}
    si può vedere
    \begin{align*}
        \left| a_nz^{n} \right| &= \left| a_n w^{n} \frac{z^{n}}{w^{n}} \right|  \\
        &= \left| a_nw^{n} \right| \left| \frac{z}{w} \right|^{n} \leq M \left| \frac{z}{w} \right|^{n}    
    \end{align*}
    Dato che la serie sulla destra è convergente, allora anche quella a sinistra della 
    disuguaglianza è convergente assolutamente poiché si è maggiorato i moduli. Dunque converge in tutto il 
    cerchio aperto costruito a partire dal modulo di $w$, ossia sulla palla aperta $B(0, \left| w \right| )$. 
\end{proof} \noindent
Dato che si è parlato di convergenza su di un cerchio, si definisce il \textbf{raggio di convergenza}
\begin{definition}
    In una serie, il raggio di convergenza è definito come $\sup\{\left| z \right| : z \in \mathbb{C}_a \}$,
    ossia l'insieme dei moduli tali per cui si ha convergenza. Ovviamente è un numero reale e non negativo (che può essere
    anche zero in determinati casi).
\end{definition} \noindent
L'insieme di convergenza sarà un cerchio con raggio definito 
    il raggio di convergenza. Dal lemma di Abel si ha che l'interno del cerchio si ha convergenza assoluta, dunque
    l'interno del cerchio fa parte dei punti di convergenza. Sul bordo del cerchio dipende dalla serie (tipicamente
    se è presente convergenza, essa non è assoluta). In generale la palla aperta di raggio $r_a$ è nel raggio di convergenza e esso 
    è incluso nella palla chiusa di raggio $r_a$.  Si vede, dal Lemma di Abbel, che 
    ciò che sta fuori dal cerchio i termini divergono. 
Se si trova un punto in cui i moduli sono limitati, quello è il bordo del cerchio. Il risultato 
caratterizza gli insiemi di convergenza:
\begin{theorem}
    L'insieme di convergenza $\mathbb{C}_a$ contiene il cerchio aperto $B(0, R_a)$ ed è 
    contenuto nel cerchio chiuso $B(0, R_a)$. Sulla frontiera non si sa esattamente 
    cosa accade. 
\end{theorem} \noindent
Questo risultato ci dice che se si volesse trovare il raggio di convergenza, si potrebbe studiare 
la serie dei moduli (ossia una serie sui reali). Si estrae dunque il raggio di convergenza che
è applicabile su $\mathbb{C}$. 



\end{document}
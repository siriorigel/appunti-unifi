\documentclass[a4paper, oneside]{article}
\usepackage{wrapfig}
\usepackage{graphicx}
\usepackage{amsthm}
\usepackage{amsmath}
\usepackage{amssymb}
\usepackage[a4paper,
            bindingoffset=0.2in,
            left=2cm,
            right=2cm,
            top=2cm,
            bottom=2cm,
            footskip=.25in]{geometry}
\usepackage[italian]{babel}
\usepackage{pgfplots}
\usepackage{tabularx}
\usepackage{tikz-3dplot}
\usepackage{wrapfig}
\usepackage{color}
\usepackage{multicol}
\usepackage{arydshln}
\usepackage{mathtools}
\usepackage{enumerate}
\usepackage{graphicx}
\usepackage{svg}
\usepackage{cancel}
\usepackage[d]{esvect}
\usepackage[dvipsnames]{xcolor}
\usepackage{pgfplots}
\usepackage{pifont}
\usetikzlibrary{patterns}
%\usepackage{animate}
%\usepackage{xfp} % utile se vuoi fare calcoli aggiuntivi
\pgfplotsset{compat=1.18}
\usetikzlibrary{tikzmark}
\newcommand{\TikzNCbar}[4][10pt]{
\tikz[overlay,remember picture]{\draw[#2] (#3) --++(0,-#1) -| (#4);}}

\graphicspath{ {images/} }

\definecolor{redish}{rgb}{255, 0, 30}
\definecolor{page}{rgb}{0.129,0.157,0.212}
\pagecolor{page}
\color{white}   
\graphicspath{ {./images/} }
\usetikzlibrary{shapes.geometric}
\usetikzlibrary{datavisualization}
\usetikzlibrary{datavisualization.formats.functions}
\pgfplotsset{width=10cm,compat=1.9}

\setlength\dashlinedash{0.2pt}
\setlength\dashlinegap{1.5pt}
\setlength\arrayrulewidth{0.3pt}

\newcommand\eqq{\stackrel{\mathclap{\normalfont\mbox{?}}}{=}}
\newcommand\bulletout  {\labelitemfont \textbullet}
\newcommand{\tab}{\hspace*{2em}}
\newcommand{\xmark}{
\tikz[scale=0.23] {
    \draw[line width=0.7,line cap=round] (0,0) to [bend left=6] (1,1);
    \draw[line width=0.7,line cap=round] (0.2,0.95) to [bend right=3] (0.8,0.05);
}}
\newcommand{\cmark}{
\tikz[scale=0.23] {
    \draw[line width=0.7,line cap=round] (0.25,0) to [bend left=10] (1,1);
    \draw[line width=0.8,line cap=round] (0,0.35) to [bend right=1] (0.23,0);
}}
% Comando:
%   \potato[opzioni]{(x,y)}{scala}
%
% Opzioni = facoltative (es. fill=red!20, draw=black, thick)
% (x,y)   = centro della patata
% scala   = fattore di scala
%
\def\potatoshape{
  (1,0) (2,1.5) (1.6,3) (0.3,2.7) (-0.4,1.2)
}
\newcommand{\potato}[3][draw=white]{
  \begin{scope}[shift={#2}, scale=#3]
    \draw[#1]
      plot [smooth cycle, tension=1]
      coordinates {\potatoshape};
  \end{scope}
}
 \newcommand{\hookbox}[1]{
\begin{center}
\hfill\break
\begin{tikzpicture}
\node[inner sep=0pt,outer sep=0pt,anchor=base] (A) {
\begin{minipage}{\dimexpr\linewidth-5em}
\centering
#1
\end{minipage}
};
% Draw the left bracket
\draw ([xshift=0pt]A.north west) -- ++(0, 0.5) -- ++(0.4, 0);
% Draw the right bracket
\draw ([xshift=0pt]A.south east) -- ++(0, -0.5) -- ++(-0.4, 0);
\end{tikzpicture}
\end{center}} 
\title{Complementi di statistica}
\author{Gariboldi Alessandro}
\date{ }


\begin{document}

\newtheoremstyle{theoremEnv}
                {}          % Space above
                {}          % Space below
                {\slshape}  % Body font
                {}          % Indent amount
                {\bfseries} % Head font
                {.}         % Punctuation after head
                {\newline}         % Space after theorem head
                {}          % Theorem head spec
\theoremstyle{theoremEnv}

\newtheorem{definition}{Definizione}[section]
\newtheorem{theorem}{Teorema}[section]
\newtheorem{lemma}{Lemma}[section]
\newtheorem{observation}{Oss.}[section]
\newtheorem{corollary}{Corollario}[theorem]
\newtheorem{example}{Esempio}[section]
\newtheorem{problem}{Problema}[section]
\newtheorem{solution}{Soluzione}[section]
\newtheorem{proposition}{Proposizione}[section]


\maketitle

\section{28/11/25}
\section{Conti altra volta}
La formula dell'altra volta è sbagliata perché la dipendenza dal volume
è sbagliata: infatti, dato che $V$ è lo stato che ci interessa e $V_0$ è
lo stato di riferimento (è una costante assegnata ) che non dipende
dal numero di particelle ma ha le stesse dimensioni di un volume. Quando
$N$ cresce, l'argomento del logaritmo è proporzionale ad $N$ e dunque
$S \propto N \ln N$. Questo non è solo un problema di dipendenza da $N$,
supponendo infatti che sia vera, allora, dato che è l'unico termine dipendente da
$N$, il resto può essere trascurato, il che porta a dire che l'entropia non dipenda
dalla temperatura per $N$ grandi, il che è inconsistente.  \\
\begin{wrapfigure}{r}{0.4\textwidth}
    \centering
    \caption{Esperimento che prova l'invalidità della formula}
    \begin{tikzpicture}
        \filldraw[cyan, opacity = 0.3](0, 0) rectangle (2, 2);
        \draw[very thick](0, 0) rectangle (4, 2);
        \draw[very thick](2, 0) -- (2, 2);
        \node at (1, 1.6) {$V$};
        \node at (1, 1.1) {$N$};
        \node at (1, 0.6) {$T$};
        \node at (3, 1.6) {$V$};
        \node at (3, 1.1) {$N$};
        \node at (3, 0.6) {$T$};
        \draw[very thick](0, -1) rectangle (4, -3);
        \filldraw[cyan, opacity = 0.3](0, -1) rectangle (4, -3);
        \node at (2, -1.4) {$2V$};
        \node at (2, -1.9) {$2N$};
        \node at (2, -2.4) {$T$};
    \end{tikzpicture}    
\end{wrapfigure}
Per capire che è inconsistente dal punto di vista fisico, si può provare
che predice una cosa sbagliata. Preso un contenitore adiabatico e perfettamente
isolato e si mette una parete ideale perfettamente diatermica tra i due volumi. 
Si pul applicare la formula utilizzata per entrambi gli ambienti :
\begin{gather*}
    S_1^{(i)} = Nk_B \left(\ln \frac{V}{V_0} + f(T)\right) \\
    S_2^{(i)} = Nk_b \left(\ln \frac{V}{V_0} + f(T)\right)
\end{gather*}
Dove
\begin{gather*}
    f(T) = \frac{3}{2}\ln\frac{T}{T_0} + \frac{3}{2}
\end{gather*}
L'entropia totale è dunque la somma delle entropie. Se si rimuove
adesso la parete diatermica, ci si trova nella situazione in cui 
il volume totale è il doppio e così come il doppio delle particelle ma il gas
non ha cambiato la sua temperatura:
\begin{gather*}
    S^{(f)} = 2Nk_B\left(\ln \frac{2V}{V_0} + f(T)\right)
\end{gather*} 
Allora la differenza di entropia è data dalla seguente:
\begin{gather*}
    \Delta S = 2Nk_B\left(\ln \frac{2V}{V_0} - \ln \frac{V}{V_0}\right) = 2Nk_B\ln 2 
\end{gather*}
Questo risultato è diverso da zero e proporzionale ad $N$. Se fosse dunque vera quella formula,
allora semplicemente togliendo la parete l'entropia del sistema è aumentata estensivamente. In realtà
questa cosa non ha senso in quanto l'unica differenza era che si era separato il gas con 
una parete diatermica e con spessore infinitesimo. Le proprietà macroscopiche sono le stesse per 
i due stati mentre con quella formula non è . Perché a livello termodinamico questi stati non possono 
essere diversi? Questo perché se si avesse messo due gas diversi si avrebbe un mescolamento 
(che è diverso dall'aver utilizzato due volumi dello stesso identico gas). Questa formula sarebbe
giusta se i due gas fossero diverse, questo ci porta a dire che c'è un problema se si assume che
i due sottosistemi siano composti dallo stesso gas.  

\subsection{Soluzione per gas omogenei}
Questo problema deriva dal fatto che definire il numero di stati come
\begin{gather*}
    \frac{d^{3N} \cdot d^{3N}v}{\Delta \Gamma}
\end{gather*}
Questo vuol dire che si considerano distinguibili le particelle tra di loro: due particelle che si possono 
invertire le velocità sono contati come due stati diversi (che per un gas le cui particelle sono 
tutte uguali) non è vero! Non si è quindi in grado di seguire la dinamica esatta delle particelle e
l'unico modo per distinguere le particelle di un gas omogeneo e "etichettarle". Nel caso di particelle identiche,
devo dividere per il numero di stati che si ottengono per lo scambio tra di loro,
ossia $N!$: il corretto conteggio degli stati è dunque
\begin{gather*}
   \frac{d^{3N} \cdot d^{3N}v}{N!\Delta \Gamma}
\end{gather*}
Questo risolve il problema? Posso determinare il numero di stati come
\begin{gather*}
    \mathcal{Z}_C = \frac{\mathcal{Z}}{N!}
\end{gather*}
Il singolo stato dunque
\begin{gather*}
    \mathcal{z}_C 0 \left(\frac{\mathcal{Z}}{N1}\right)^{\frac{1}{N}} = \frac{z}{(N!)^{\frac{1}{N}}}
\end{gather*}
Dunque
\begin{gather*}
    \ln \mathcal{z}_C = \ln \mathcal{z} - \frac{1}{N}\ln(N!)
\end{gather*}
Utilizzando la correzione di Stirling:
\begin{gather*}
    \ln(N!) \approx N \ln N - N \ \Longrightarrow \ \ln \mathcal{z}_C = \ln \mathcal{z} - \ln N + 1
\end{gather*}
Allora si può esprimere l'entropia come 
\begin{gather*}
    S = Nk_B \left(\ln \frac{V}{NV_0} + \frac{3}{2} \ln \frac{T}{T_0} + \frac{5}{2}\right)
\end{gather*}
A questo punto questa espressione è correttamente estensiva. 
La risoluzione dello spazio delle fasi non può però essere scelta a piacere
ma la meccanica quantistica ci dice che questo non è possibile, infatti si ha che
\begin{gather*}
    \Delta x \Delta p \geq h \qquad \Delta x \Delta v \geq \frac{h}{m}
\end{gather*}
Non si può dunque scegliere oggetti arbitrari, ma inevitabilmente si avrà 
che $(\Delta x \Delta v)^{3} \geq \frac{h^{3}}{m^{3}}$. Dato che la scala
di risoluzione che si sceglie non può essere più piccola di questo limite, imponiamo
questo limite di risoluzione come
\begin{gather*}
    S = Nk_B \left(\ln\left(\frac{V}{N} \left(\frac{2\pi k_B T}{h^{2}}\right)^{\frac{3}{2}}\right) + \frac{5}{2}\right)
\end{gather*}

\section*{Fluidodinamica}
\section{Fluidi ideali}
\begin{wrapfigure}{r}{0.4\textwidth}
    \centering
    \caption{spostamento Lagrangiano}
    \begin{tikzpicture}
        \filldraw(1.5, 2.1) circle (1pt);
        \filldraw(1, 2) circle(1pt);
        \draw[->](0, 0) -- (3, 0) node[at end, below] {$x$};
        \draw[->](0, 0) -- (0, 3) node[at end, left] {$y$};
        \draw[->](0, 0) -- (-1, -1) node[at end, left] {$z$};
        \draw[thick, ->] (0, 0) -- (1, 2) node[midway, left] {$\vv{r_0}$ };
        \draw[->](0, 0) -- (1.5, 2.1) node[midway, right] {$\vv{v} + d\vv{r}$ };
        \draw[->](1, 2) -- (1.5, 2.1);
    \end{tikzpicture}    
\end{wrapfigure}
In un sistema di riferimento  si pul prendere un tempo di riferimento
$t = 0$ ed un elemento fluido che si trova in una posizione $\vv{r_0} $ 
e descrivere il moto come lo spostamento 
\begin{gather*}
    \zeta (\vv{r_0}, t )
\end{gather*}
Posso scegliere di seguire il moto di un punto del fluido e studiarne il moto,
questo prende il nome di \textbf{spostamento Lagrangiano}.
Se invece decidessi di fissare il mio sguardo su di un punto fisso dello spazio
e valuto cosa accade a quell'elemento fluido cosa accade nel tempo; questa è 
una \textbf{descrizione Euleriana} del mio fluido. Supponendo di analizzare il
comportamento dell'elemento fluido a $dt$ rispetto al tempo di riferimento secondo
la descrizione Euleriana. Per poter impostare le derivate, devo ricordarmi che da quel
punto il fluido si muove. Non mi interessa dunque la storia del fluido ma 
solo quello che accadrà dopo 
l'elemento fluido che sto analizzando si sarà spostato rispetto alla posizione iniziale.
A questo punto posso definire la velocità del fluido come se fosse un punto materiale:
\begin{gather*}
    \vv{u}(\vv{r}, t ) = \frac{d\vv{r} }{dt} = \lim_{\Delta t \to 0} \frac{\vv{r}(t + \Delta t) - \vv{r}(t) }{\Delta t}
\end{gather*}
Ossia lo si definisce come nella meccanica. Si può definire la densità 
e la pressione come
\begin{gather*}
    \left\{\begin{array}{l}
        p(\vv{r}, t ) \\
        \rho(\vv{r}, t ) \\
        \vv{u}(\vv{r}, t ) 
    \end{array}\right.
\end{gather*}
Definisce un campo vettoriale. Si definisce ora l'accelerazione ora nella fluidodinamica.\\
\begin{gather*}
    \vv{a} =  \frac{d\vv{u} }{dt} = \lim_{\Delta t \to 0} \frac{u(t + \Delta t, \vv{r}( t + \Delta t) ) - \vv{u}(t, \vv{r}(t) )}{\Delta t}
\end{gather*}
Tenendo a mente ora che $\Delta \vv{r} = \vv{u}(t,\vv{r}) \Delta t$ che riscritto per le singole componenti diventa:
\begin{gather*}
  \Delta \vv{r} = \Delta x \ \underline{i} + \Delta y \ \underline{j} + \Delta z \ \underline{k} = \sum_{i=1}^{3} \Delta x_i \underline{x}_i 
\end{gather*}
Posso dire che la velocità di una certa particella avrà velocità
\begin{gather*}
    u_i(t + \Delta t , \vv{r} (t + \Delta t)) \\
    = u_i(t, \vv{r} (t) ) + \frac{\partial u_i}{\partial t}\Delta t + \sum_{k=1}^{3} \Delta r_k\frac{\partial u_i}{\partial x_k} + o(\Delta t^2) \\
    = u_i(t, \vv{r} ) + \frac{\partial u_i}{\partial t}\Delta t + \sum_{k=1}^{3} u_k \Delta t \frac{\partial u_i}{\partial x_k} + o(\Delta t^{2})  
\end{gather*}
Ossia lo sviluppo in serie di Taylor per una funzione a più variabili. Posso dunque derivare l'espressione
di $u_i$ secondo la relazione utilizzata prima :
\begin{gather*}
    \frac{\partial u_i}{\partial t} = \lim_{\Delta t \to 0} \frac{\left(\cancel{u_i(t, \vv{r} )} + \cancel{\Delta t}\left(\frac{\partial u_i}{\partial t} + \sum_{k=1}^{3} u_i \frac{\partial u_i}{\partial x_k} \right) \cancel{- u_i(t, \vv{r} )}\right) + \cancel{o(\Delta t^2)} }{\cancel{\Delta t}}
\end{gather*}
Dunque si ottiene:
\begin{gather*}
  \boxed{\frac{\partial u_i}{\partial t} = \frac{\partial u_i}{\partial t} + \sum_{k=1}^{3} u_i \frac{\partial u_i}{\partial x_k}  }
\end{gather*}
e inoltre vettorialmente:
\begin{gather*}
  \boxed{\vv{a} = \frac{\partial \vv{u}}{\partial t} + \sum_{k=1}^{3} u_i \frac{\partial \vv{u}}{\partial x_k} } 
\end{gather*}
L'accelerazione non corrisponde dunque solo alla derivata parziale di $u$ rispetto al tempo,
ma dipende anche dalle componenti spaziali. Si può anche scrivere
\begin{gather*}
    \frac{\partial u_i}{\partial t}  =\frac{\partial u_i}{\partial t} + (\vv{u} \cdot \vv{\nabla}  )u_i  
\end{gather*}
Dove $u_i$ è una delle tre componenti della velocità. Questa cosa non vale solamente per la 
velocità ma vale comunque per qualunque campo scalare $f(x, y, z, t)$ che descrive il moto del fluido:
\begin{gather*}
    f(\vv{r}, t ) \ \Longrightarrow \ f'(\vv{r}, t ) = \lim_{\Delta t \to 0} \frac{f(\vv{r}(t + \Delta t), t + \Delta t ) - f(\vv{r}, t )}{\Delta t} 
\end{gather*}
Ossia
\begin{gather*}
    \frac{\partial f_i}{\partial t} = \frac{\partial f}{\partial t} + (\vv{u} \cdot \vv{\nabla}  )f  
\end{gather*}
Nel sistema Euleriano mi permette di dire come varia il moto dell'elemento fluido al variare del tempo
e prende il nome di \textbf{derivata sostanziale totale}. La prima componente è il termine che
descrive cosa accade ad un punto a tempi diversi. 
\begin{gather*}
    df = \frac{\partial f}{\partial dt}dt + \frac{\partial f}{\partial x}dx +\frac{\partial f}{\partial y}dy + \frac{\partial f}{\partial z}dz \\
    \ \Longrightarrow \ \frac{df}{dt} = \frac{\partial f}{\partial t} + u_x \frac{\partial f}{\partial x} + u_y \frac{\partial f}{\partial y} + u_z \frac{\partial f}{\partial z}    
\end{gather*}
Allora l'accelerazione rispetto alla $x$
\begin{gather*}
    a_x = \frac{\partial u_x}{\partial t} + u_y \frac{\partial u_x}{\partial t} + u_z \frac{\partial u_x}{\partial t}   
\end{gather*}
E lo stesso per le altre componenti.

\section{Linee di campo (o di flusso)}
\begin{wrapfigure}{r}{0.4\textwidth}
    \centering
    \caption{La costruzione della linea di flusso}
    \begin{tikzpicture}[scale=1.5]
        \draw[->](0, 0) -- (1, 1) node[at end, right] {$\vv{u}(\vv{r}, t )$};
        \draw(-0.1, -0.05) -- (0.1, 0.05) node[midway, below] {$\vv{dl}$ };
    \end{tikzpicture}    
\end{wrapfigure}
Si definisce una \textbf{linea di campo} o \textbf{linea di flusso} come una linea di spazio parallelo al campo di velocità $\vv{u}$:
\begin{align}
    \vv{dl} = c \vv{u}  
\end{align} 
Dunque posso costruire la linea di campo in modo tale che
\begin{gather*}
    \left\{\begin{array}{l}
        dx = cu_x \\
    dy = cu_y \\
    dz = cu_z
    \end{array}\right.
\end{gather*}
Dato che la $c$ è la stessa, questo vuol dire che deve valere 
\begin{gather*}
    \frac{dx}{u_x} = \frac{dy}{u_y} = \frac{dz}{u_z}
\end{gather*}
Le linee di flusso hanno senso si è in \textbf{condizioni stazionarie}:
ossia le condizioni per cui $\frac{\partial }{\partial t} = \emptyset$.  
In quel caso le linee di flusso rappresentano il moto dell'elemento di fluido
scelto. Se si avesse una linea di flusso arbitraria, un qualsiasi elemento di fluido 
su questa linea di flusso seguira il flusso stesso ma solo nelle condizioni
stazionarie. 

\begin{wrapfigure}{r}{0.4\textwidth}
    \centering
    \caption{Tubo di flusso}
    \begin{tikzpicture}
        \draw[->](0, 0) .. controls (1, 0.25) and (2, -0.5) .. (3, -0.5);
        \draw[->](0, -1) -- (3, -1);
        \draw[->](0, -2) .. controls (1, -2.25) and (2, -1.5) .. (3, -1.5);
        \draw(0, -1) ellipse (0.25 and 1);
        \draw(3, -1) ellipse (0.2 and 0.5);
        \filldraw(0, -2) circle(1pt) node[anchor = north] {$C$};
        \filldraw(3, -1.5) circle(1pt) node[anchor = north] {$C'$};
        \node at (2, 0) {$\vv{u}$ };
    \end{tikzpicture}    
\end{wrapfigure}
La linea di flusso mi da la direzione ed il verso della velocità ma non il modulo,
considerando di avere delle linee di flusso, si considera la superficie fatta da tutte
le linee di flusso tangenti a $C$ e $C'$, e come basi si appoggiano a $C$ e $C'$, 
prende il nome di \textbf{tubo di flusso}. 
Si può dunque prendere una linea di flusso piccola a piacere in modo tale
che le variabili (pressione, densità, velocità, ..) siano quasi costanti. Posso dunque
considerare un piccolo tubo di flusso in modo tale che in questa sezione i valori sono
costanti e che prende il nome di \textbf{tubo di flusso sottile}. 

\section{Aspetto dinamico della fluidodinamica}
\subsection{Equazione fondamentale della fluidodinamica}
Le forze che venivano applicate ad un fluido statico, sono delle forze
di pressione che si possono esprimere come 
\begin{gather*}
    \vv{F_\Sigma} = - \vv{\nabla}p\ dV  
\end{gather*}
Ci sono anche le forze di volume
\begin{gather*}
    \vv{F_V} = dm \vv{g} = \rho \ dV \vv{g} 
\end{gather*}
Applicata alla gravità (che può non necessariamente essere lei), se
si considera adesso un volumetto infinitesimo e si identifica un
volumetto infinitesimo che contiene una massa $dm = \rho dV$, al tempo
$t$ voglio conoscere le forze che agiscono su questa massa. Posso dunque definire, dalla
prima cardinale
\begin{gather*}
    \vv{F_\Sigma} + \vv{F_V} = -\vv{\nabla}p \ dV + \rho g \ dV  =  \rho \ dV \frac{d\vv{u} }{dt}
\end{gather*}
Si trova dunque l'\textbf{equazione di Eulero}:
\begin{gather*}
    \rho\frac{d\vv{u} }{dt} = -\vv{\nabla}p + \rho\vv{g}  
\end{gather*}
Ossia
\begin{gather*}
    \rho\left(\frac{\partial \vv{u} }{\partial t} + (\vv{u} \cdot \vv{\nabla}  )\vv{u} \right) = -\vv{\nabla}p + \rho \vv{g}  
\end{gather*}
Dato che è una equazione vettoriale, bisogna trovare tre equazioni in quanto non ci basta questa cosa. Dato che la massa si 
deve conservare, per un volume finito di fluido, posso determinare la massa 
\begin{gather*}
    M = \int_{V}^{} \rho \ dV \ \Longrightarrow \ \frac{dM}{dt} 
\end{gather*}
Ossia definisco il \textbf{flusso di massa} entrante e uscente dal fluido. Posso dunque 
definirla meglio
\begin{gather*}
    \frac{dM}{dt} = \dot{M} = -\int_{\Sigma(V)}^{} \rho \vv{u} \cdot \hat{n} \ d\sigma  
\end{gather*}
Dove $\Sigma(V)$ è il volume totale e $\hat{n}$ il vettore normale alla superficie del fluido
e $d\sigma$ rappresenta esattamente l'elemento infinitesimo di superficie $\sigma$. Posso però 
esprimere anche 
\begin{gather*}
    M = \int_{V}^{} \frac{\partial \rho}{\partial dt} \ dV = - \int_{\Sigma(V)}^{}  \rho  \vv{u} \cdot \hat{n} \ d\sigma  
\end{gather*} 
E dunque si è trovata l'equazione di conservazione della massa. L'unico 
problema è che $V$ è in forma integrale e va trasformata in forma differenziale. 

\newpage
\section{25/11/25}
Dalla lezione scorsa:
\begin{gather*}
  S = K_b \ln \mathcal{N} (U,V)\\
  \mathcal{N}(U,V) = \int \delta\left[ U-E(x,v) \frac{dx^{3N} \ dv^{3N}}{\Delta \Gamma} \right] 
\end{gather*}
Senza cambiare il valore dell'energia che voglio selezionare posso scrivere:
\begin{gather*}
  \mathcal{N}'(U,V) = \int \delta\left[ U-E(x,v) \frac{dx^{3N} \ dv^{3N}}{\Delta \Gamma \ a} \right] 
\end{gather*}
Abbiamo che l'unica differenza tra le due è una costante $a$.\\
\begin{gather*}
  \mathcal{N}' (U,V) = \frac{\mathcal{N}(U,V)}{a}\\
  S' = K_b \ln \left[\frac{\mathcal{N}'(U,V)}{a}\right] = \underbrace{K_b \ln \mathcal{N}(U,V)}_{S} - K_b \ln a = S - K_b \ln a
\end{gather*}
Quindi 
\begin{gather*}
  S' = S - K_b \ln a\\
  s' = S - \text{cost.}
\end{gather*}
Prendiamo ora un gas ideale, diviso in volumetti di stati elementari di volume $V_0$
\begin{center}
  \begin{tikzpicture}
    
  \draw(0,0) rectangle (0.4,0.4);
  \draw(0.4,0) rectangle (0.8,0.4);
  \draw(0.8,0) rectangle (1.2,0.4);
  \draw(1.2,0) rectangle (1.6,0.4);
  \draw(1.6,0) rectangle (2,0.4);
  \node at (0.2,0.2){$V_0$};
  \node at (0.2,0.2){$V_0$};
  \end{tikzpicture}
\end{center}
Se Ho il volume totale che è $V$ ho quindi $V/V_0$ volumetti di stato elementari.\\
Nel caso di un atomo avrò $\#_1 = \frac{V}{V_0}$\\
Nel caso di due atomi avrò $\#_2 = \left(\frac{V}{V_0}\right)^2$\\
Nel caso di $N$ atomi avrò:
\begin{gather*}
  \#_N = \left(\frac{V}{V_0}\right)^N
\end{gather*}
Se riconsidero che $W_N (U,V)$ e mi interessa calcolare il numero di modi in modo che abbiano volume ed energia assegnata. (con sostanzialmente $W_N = \mathcal{N}$)\\
\begin{gather*}
  S = K_B \ln W_N (U,V) = K_B \ln \left[ \text{cost.} \left(\frac{V}{V_0}\right)^N \right] =\\
  = K_b \ln \left[ \left(\frac{V}{V_0}\right)^N  \right] + c\\
  = N K_b \ln\left(\frac{V}{V_0}\right) + c 
\end{gather*}
\begin{gather*}
  S(U,U_f) - S(U,U_i) = \Delta S = N K_b \ln \left(\frac{V_f}{V_0}\right) - N K_b \ln \left(\frac{V_i}{V_0}\right) =\\
  = N_A K_b \ln \left(\frac{V_f}{V_i}\right) = R \ln \left(\frac{V_f}{V_i}\right)
\end{gather*}
Consideriamo ora $S = K_b \ln W$ con $W$ probabilità termodinamica\\
\begin{gather*}
  \frac{W}{M} \\
  \frac{P(V_f)}{P(V_i)} = \frac{W(V_f)}{W(V_i)} = \frac{V_f}{V_i}
\end{gather*}

\newpage
\section{2/12/25}
\begin{gather*}
  \rho \left( \underbrace{\frac{\partial \vv{u}}{\partial t} + (\vv{u} \vv{\nabla})}_{\frac{d \vv{u}}{dt} \quad \frac{\partial u_i}{\partial t}+\sum_k v_k \frac{\partial}{\partial x_k}v_i} \right) = -\vv{\nabla} p + \rho \vv{g}\\
  M = \int_v \rho \ dv
\end{gather*}
L'unico modo per vriare la massa è che questa entra o esca dal volume, quindi posso dire che la variazione della massa è il flusso di questa:
\begin{gather*}
  M = - \int_{\sum(v)} \rho \vv{u} \ \widehat{n} \ d\sigma
\end{gather*}
\begin{center}
  \begin{tikzpicture}
    %assi
    \draw[->] (0,0,0) -- (3,0,0) node[at end, above]{$x$};
    \draw[->] (0,0,0) -- (0,3,0) node[at end, above]{$z$};
    \draw[->] (0,0,0) -- (0,0,3) node[at end, above]{$y$};
    %cubo
    \draw(1,0,0) -- (1,1,0) -- (0,1,0);
    \draw(1,0,0) -- (1,0,1);
    \draw(1,1,0) -- (1,1,1);
    \draw(0,1,0) -- (0,1,1);
    \draw(1,0,1) -- (1,1,1) -- (0,1,1) -- (0,0,1) -- cycle;
  \end{tikzpicture}
\end{center}
Abbiamo un volumetto (a volume fissato) la massa al suo interno sarà:
\begin{gather*}
  \rho(x,y,z t) \ dx \ dy \ dz\\
  M = \frac{\partial \rho}{\partial t} \ dx \ dy \ dz
\end{gather*}
PArtiamo dalla componente x e la componente $\phi_x$ che sarà il flusso che esce o entra sulll'asse $x$:
\begin{gather*}
  \phi_x = \left[(\rho \ u_x) \left( x + dx , y , z \right) - \left(\rho u_x\right)(x,y,z)\right] \ dy \ dz\\
  \phi_y = \left[(\rho \ u_y) \left( x , y + dy , z \right) - \left(\rho u_y\right)(x,y,z)\right] \ dx \ dz\\
  \phi_z = \left[(\rho \ u_z) \left( x , y , z + dz \right) - \left(\rho u_z\right)(x,y,z)\right] \ dx \ dy
\end{gather*}
Adesso mettiamo insieme eguagliando la massa alla somma di queste tre componenti:
\begin{gather*}
  \frac{\partial \rho}{\partial t} \ dx \ dy \ dz = \\
  -\left[\frac{(\rho \ u_x) \left( x + dx , y , z \right) - \left(\rho u_x\right)(x,y,z)}{dx}\right] \ \cancel{dy \ dz} +\\
  -\left[\frac{(\rho \ u_y) \left( x , y + dy , z \right) - \left(\rho u_y\right)(x,y,z)}{dy}\right] \ \cancel{dx \ dz} +\\
  -\left[\frac{(\rho \ u_z) \left( x , y , z + dz \right) - \left(\rho u_z\right)(x,y,z)}{dz}\right] \ \cancel{dx \ dy}
\end{gather*}
Mettiamo a sistema loperatore divergenza con il flusso di massa che sono rispettivamente:
\begin{gather*}
  \frac{\partial \rho}{\partial t} = - \frac{\partial }{\partial x} (\rho u_x) - \frac{\partial }{\partial y} (\rho u_y) - \frac{\partial }{\partial z} (\rho u_z)\\
  \vv{\nabla} \vv{f} = \sum_i \frac{\partial f_i}{\partial x_i} \quad \vv{f} = -\rho \vv{u}
\end{gather*}
Procediamo:
\begin{gather*}
  \begin{cases}
    \rho \left( \frac{\partial \vv{u}}{\partial t} + (\vv{u}\cdot \vv{\nabla}) \vv{u} \right) = -\vv{\nabla} p + \rho \vv{g}\\
    \frac{\partial \rho}{\partial t} + \vv{\nabla} (\rho \vv{u}) = 0
  \end{cases}
\end{gather*}



\newpage
\section{4/12/25}
Abbiamo visto nel caso incomprimibile che un fluido che si muove è descritto da:
\begin{gather*}
  \frac{v^2}{2} + \frac{p}{\rho} + gh = \text{costante} \label{eq:bernoulli}
\end{gather*}
Mentre nel caso comprimibile:
\begin{gather*}
  \frac{v^2}{2} + \int_{p_0}^{p} \frac{dp}{\rho} + yh = \text{costante} \label{eq:bernoulli_compr}
\end{gather*}
Se abbiamo un caso in cui se le quote sono uguali si ha:
\begin{gather*}
  \frac{u_1^2}{2} + \frac{p_1}{\rho_0} = \frac{u_2^2}{2} + \frac{p_2}{\rho_0} = \frac{u_1^2}{2}\left( \frac{\sum_1}{\sum_2}  \right)^2+ \frac{p_2}{\rho_0} \\
  \frac{p_1-p_2}{\rho_0} = \frac{u_1^2}{2}\left[ \left(\frac{\sum_1}{\sum_2}\right)^2 -1 \right] 
\end{gather*}
Questo è detto come \textbf{effetto venturi}\\

\subsection{Fuoriuscita di un fluido da un foro}
\begin{gather*}
  Q_r = \sum_1 u_1 = \sum_1 \sqrt{\frac{2(p_1-P_2)}{P_0} \frac{1}{\left( \frac{\sum_1}{\sum_2}\right)^2-1 }}
\end{gather*}
\begin{wrapfigure}{r}{0.4\textwidth}
  \centering
  \begin{tikzpicture}
    \draw(0,0) rectangle (2,3);
    \draw(2,0.5) rectangle (2.5,1) node[at end, right] {$\Sigma \ P_2=P_0$};
    \node at(3,0.2) {$\boxed{2} \quad U_2 $};
    \node at(1,3.5) {$\boxed{1} \quad P_1 = P_0 \quad \Sigma_1 $};
    \draw[->](-1,2) -- (-1, 0) node[midway, left] {$g$};
  \end{tikzpicture}
\end{wrapfigure}
\begin{gather*}
  \cancel{\frac{u_1^2}{2}} + \frac{\cancel{p_1}}{\rho_0} + y h_1 = \frac{u_2^2}{2} + \frac{\cancel{p_2}}{\rho_0} + 0\\
  \boxed{u_2 = \sqrt{2gh_1}}
\end{gather*}
Questa è l'esperienza di \textbf{torricielli}.\\
\subsection{Misurare la posizione di un aereo rispetto all'aria}
Ovviamente non si puo usare il gps perchè quello è rispetto al suolo.
\begin{center}
  \begin{tikzpicture}
    %\draw(0,0) ellipse (3 and 1);
    %\draw(1.4,-1) ellipse (1 and 2);
    %\fill[page](1.5,-0.2) circle (1);
    \draw(3,-0.5) ..controls(0,0).. (4,0.5);
  \end{tikzpicture}
  \hfill\\
  Questo è il tubo di \textbf{pitot}
\end{center}
\begin{gather*}
  \frac{p_1}{\rho_0} = \frac{u_2^2}{2} + \frac{p_2}{\rho_0}
\end{gather*}
\subsection{es rubinetto}
Da un rubinetto da cui gocciola l'acqua si nota che la goccia che sis tacca si stringe man mano che scende.
\begin{center}
  \begin{tikzpicture}
    \draw(-1,0) -- (1,0) node[midway, above]{$\Sigma_1$} node[at end, right]{$p_1 = p_0$};
    \draw(-0.5,-3) -- (0.5, -3) node[midway, below]{$\Sigma_2$} node[at end, right]{$p_2 = p_0$};
    \draw(-0.5,-3) .. controls (-0.5,-1.5) .. (-1,0);
    \draw(0.5,-3) .. controls (0.5,-1.5) .. (1,0);
    \node at(1,-1.5) {$z$};
  \end{tikzpicture}
\end{center}
\begin{gather*}
  \frac{u_1^2}{2} + \frac{\cancel{p_1}}{\rho_0} = \frac{u_2^2}{2} + \frac{\cancel{p_2}}{\rho_0}\\
  \Sigma_1 u_1 = \Sigma_2 u_2\\
  u_2^2 = u_1^2 + 2yz\\
  u_2 = \frac{\Sigma_1}{\Sigma_2} u_1\\
  \left( \frac{\Sigma_1}{\Sigma_2} \right)^2 u_1^2 = u_1^2 + 2yz\\
  \left( \frac{\Sigma_1}{\Sigma_2} \right)^2 = 1 + \frac{2 yz}{u_1^2}\\
  \frac{\Sigma_1}{\Sigma_2} = \sqrt{1 + \frac{2 yz}{u_1^2}}
\end{gather*}
Quindi vediamo che la sezione della goccia diminuisce al crescere di $z$.\\
Adesso complichiamo leggerlmente il caso e facciamo che la goccia cada in un'altro liquido.\\
Abbiamo quindi un fattore di densità $\rho_e$ che fa si che:
\begin{gather*}
  p_e = p_0 + \rho_e g z\\
  p_2 = p_{e1} + \rho_e g z 
\end{gather*}
Continuando ho:
\begin{gather*}
  \frac{u_2^2}{2} = \frac{u_1^2}{2} + yz - \frac{p_e}{\rho_0} g z\\
  u_2^2 = u_1^2 + 2 yz + (1 - \frac{p_e}{\rho_0})\\
  \left(\frac{\Sigma_1}{\Sigma_2}\right)^2 u_1^2 = u_1^2 + 2g(1- \frac{\rho_e}{\rho_0}) z \Rightarrow \left( \frac{\Sigma_1}{\Sigma_2} \right)^2 = 1 + \frac{2g}{u_1^2} \left( 1- \frac{\rho_e}{\rho_0} \right)z 
\end{gather*}

\subsection{caso della mano fuori dal finestrino}
Le due equazioni che si usano sempre sono:
\begin{gather*}
  \frac{\partial \rho}{\partial t} + \vv{\nabla} \cdot (\rho \vv{u}) = \psi \qquad \boxed{1} \\
  \rho \left( \frac{\partial \vv{u}}{\partial t} + (\vv{u} \cdot \vv{\nabla}) \vv{u} \right) = -\vv{\nabla} p + \rho \vv{g} \qquad \boxed{2}
\end{gather*}
Seguendo alcuni passaggi:
\begin{gather*}
  \rho \frac{\partial \vv{u}}{\partial t} = \frac{\partial \rho \vv{u}}{\partial t} - \vv{u} \frac{\partial \rho}{\partial t}\\
  \rho ( \vv{u} \cdot \vv{\nabla}) u_k = \rho \sum_i u_i \frac{\partial u_k}{\partial x_i}
\end{gather*}
Si pensa $\nabla$ come $\partial_x \widehat{x} + \partial_y \widehat{y} + \partial_z \widehat{z}$
\begin{gather*}
  = \sum_i \rho u_i \frac{\partial u_k}{\partial x_i} = \sum_i \frac{\partial}{\partial x_i} \left( \rho u_i u_k \right) - u_k \sum_i \frac{\partial}{\partial x_i} (\rho u_i)\\
  = \sum_i \frac{\partial}{\partial x_i} (\rho u_i u_k) - u_k \vv{\nabla} \cdot (\rho \vv{u}) \quad \text{\tiny(frs u non ha vec)}
\end{gather*}
Abbiamo un oggetto a due indici, questo è un tensore, per i nostri scopi lo tratteremo simile ad una matrice e generalmente si indica con:
\begin{gather*}
  \vv{\vv{S}} = \{ S_{i k} \} = \rho u_i u_k
\end{gather*}
COntinuando possiamo scrivere che l'equazione \fbox{2} diventa:
\begin{gather*}
  \frac{\partial}{\partial t} ( \rho \vv{u} ) - \vv{u} \frac{\partial \rho}{\partial t} + \vv{\nabla} \cdot \vv{\vv{S}} - \vv{u} \vv{\nabla} \cdot (\rho \vv{u}) = -\vv{\nabla} p + \rho \vv{g}
\end{gather*}
Continuando...
\begin{gather*}
  \frac{\partial}{\partial t} (\rho \vv{u}) + \vv{\nabla} \cdot \vv{\vv{S}} = \\
  \int_v \frac{\partial}{\partial t} (\rho \vv{u}) + \int_v \vv{\nabla} \cdot \vv{\vv{S}}\ dv = -\int_v \vv{\nabla} p + \int_v \rho \vv{g} \ dv\\
  \frac{d}{dt} \int_v \rho \vv{u} \ dv + \int_{\sum(v)} \vv{\vv{S}} \cdot \widehat{n} \ d\sigma = -\int_{\sum(v)} p \widehat{n} \ d\sigma + \int_v \rho \vv{g} \ dv\\
  \rho u_i u_j \widehat{n} = (\rho \vv{u}) \cdot u_n\\
  \frac{\partial }{\partial t} (\rho \vv{u}) + \vv{\nabla} \cdot \vv{\vv{S}} - \vv{u}\left( \cancel{ \frac{\partial \rho}{\partial t} + \vv{\nabla} \cdot (\rho \vv{u}) } \right) = -\vv{\nabla} p + \rho \vv{g}
\end{gather*}
Se considero $\vv{\vv{S}} = \left(\rho \vv{u} \vv{u}\right)  \widehat{n}$ lo esprimo come:
\begin{gather*}
  \begin{pmatrix}
    \rho u_x u_x & \rho u_x u_y & \rho u_x u_z \\
    \rho u_y u_x & \rho u_y u_y & \rho u_y u_z \\
    \rho u_z u_x & \rho u_z u_y & \rho u_z u_z
  \end{pmatrix} \cdot \begin{pmatrix}
    1 \\ 0 \\ 0
  \end{pmatrix} = \begin{pmatrix}
    \rho u_x u_x \\
    \rho u_x u_y \\
    \rho u_x u_z
  \end{pmatrix} = \left( \rho u_x \widehat{x} + \rho u_y \widehat{y} + \rho u_z \widehat{z} \right)  u_x = \rho \vv{u} u_x
\end{gather*}


\newpage
\section{5/12/25}
Dalla scorsa lezione eravamo partiti da:
\begin{gather*}
  \frac{\partial \rho}{\partial t} + \vv{\nabla} \cdot (\rho \vv{u}) = 0\\
  \rho \left( \frac{\partial \vv{u}}{\partial t} + (\vv{u} \cdot \vv{\nabla}) \vv{u} \right) = -\vv{\nabla} p + \rho \vv{g}
\end{gather*}
E siamo giunti a:
\begin{gather*}
  \frac{\partial (\rho \vv{u})}{\partial t} + \vv{\nabla} \cdot (\rho \vv{u} \vv{u}) = -\vv{\nabla} p + \rho \vv{g}
\end{gather*}
E abbiamo interpretato il termine $(\rho \vv{u} \vv{u})$ come un tensore.\\
\begin{gather*}
  \vv{S} = \rho \vv{u} \vv{u} \\
  \vv{\vv{S}} \cdot \widehat{n} = \rho \vv{u} u_n
\end{gather*}
Identifichiamo questo oggetto $\vv{S} \cdot \vv{\nabla}$ come \textbf{pressione ram} o pressione dinamica.\\
\begin{example}
  Supponiamo di avere un tubo di sezione $\Sigma$ con un flusso entrante e uscente.
  \begin{center}
    \begin{tikzpicture}
      \draw[->](2,2) -- (3,2) node[midway, above]{$F=2Q u$};
      \draw[dashed, gray](-3, 1.9) -- (-3, -1.9);
      \draw(-2,0) ellipse (5 and 2);
      \draw(-2,0) ellipse (3 and 1);
      \fill[page] (-8,-3) rectangle (-3,3);
      \node at(-3,-1.5) {$\Sigma$};
      \node at(-3,1.5) {$\Sigma$};
      \draw[->](-2,-1.5) -- (-1.5, -1.5) node[at end, right] {$\vv{u_1}$};
      \draw[->](-3.5,1.5) -- (-4, 1.5) node[at end, left] {$\vv{u_2}$};
    \end{tikzpicture}
  \end{center}
  \begin{gather*}
    Q_M = \rho_0 u_1 \Sigma\\
    F_1 = \rho_0 \vv{u_1} \cdot u \Sigma\\
    F_2 = \rho \vv{u_2} \cdot u \Sigma = -\rho_0 \vv{u}_1 u_1 \Sigma\\
    F_{tot} = F_1 + F_2 = -Q_M \vv{u}_1 - Q_M \vv{u}_1 = -2 Q_M \vv{u}_1
  \end{gather*}  
  \begin{gather*}
    \int_v \frac{\partial \rho \vv{u}}{dt} dv + \int_v \vv{\nabla} \cdot (\rho \vv{u} \vv{u}) dv = -\int_v \vv{\nabla} p \ dv + \int_v \rho \vv{g} \ dv\\
    \frac{d}{dt} \mathcal{G} + \int_{\Sigma(v)} (\rho \vv{u} \vv{u}) \cdot \widehat{n} \ d\sigma = -\int_{\Sigma(v)} p \widehat{n} \ d\sigma + \int_v \rho \vv{g} \ dv\\
    = -\vv{\nabla} p + \rho \vv{y}
  \end{gather*}
\end{example}

\begin{example}
  Prendiamo un'altro esempio, una pala eolica.\\
  \begin{center}
    \begin{tikzpicture}
      \draw (0,1.2) ellipse (0.1 and 1);
      \draw (0,-1.2) ellipse (0.1 and 1);
      \draw(0,0) circle (0.2);
      \draw[->](-2,0) -- (-1,0) node[midway, above]{$\vv{u_1}$};
      \draw[->](1,0) -- (2,0) node[midway, above]{$\vv{u_2}$};
      \draw[<->](-0.5,0.2) -- (-0.5,2) node[midway, left]{$\Sigma$};
    \end{tikzpicture}
  \end{center}
  Supponiamo di conoscere la sezione della pala e la velocità in ingresso $u_1$ e anche la potenza erogata dalla pala (in $KW$).\\
  Da questo possiamo conoscere
  \begin{itemize}
    \item il rendimento della pala
    \item la velocità in uscita $u_2$
    \item la spinta che il flusso di massa esercita sulla pala
  \end{itemize}

  \begin{gather*}
    Q_M = \rho u_1 \Sigma
  \end{gather*}
  Ora posso ricavarmi la potenza, che il flusso di massa esercita sulla pala:
  \begin{gather*}
    P_1 = Q_M \frac{u_1^2}{2}\\
    \eta = \frac{P_\varepsilon}{P_1} \rightarrow P_\varepsilon = \eta P_1
  \end{gather*}
  La potenza che esce dalla pala sarà:
  \begin{gather*}
    P_2 = P_1 - P_\varepsilon = P_1 (1-\eta) \\
    = \cancel{Q_M} \frac{u_2^2}{2} = \cancel{Q_M} \frac{u_1^2}{2} (1-\eta) \\
    F = Q_M (\vv{u_1} - \vv{u_2}) \\
    u_2 = u_1 \sqrt{1-\eta} = -Q_M \vv{u_1} (1-\sqrt{1-\eta})
  \end{gather*}
\end{example}

\subsection{Cerchiamo bernulli tramite }
Partiamo da:
\begin{gather*}
  \frac{\partial \rho }{\partial t} + \vv{\nabla} \cdot (\rho \vv{u}) = 0 \quad \boxed{1}\\
  \rho \left( \frac{\partial \vv{u}}{\partial t} + (\vv{u} \cdot \vv{\nabla}) \vv{u} \right) = -\vv{\nabla} p + \rho \vv{\nabla} \ \phi_\mathcal{G} \quad \boxed{2}\\
  \vv{u} \left[\rho \left( \frac{\partial \vv{u}}{\partial t} + (\vv{u} \cdot \vv{\nabla}) \vv{u} \right)\right]  = \left[-\vv{\nabla} p + \rho \vv{\nabla} \ \phi_\mathcal{G}\right] \vv{u} \\
  = -\vv{\nabla} p \cdot \vv{u} - \rho \vv{\nabla} \phi_\mathcal{G} \cdot \vv{u} \qquad \boxed{*}\\
\end{gather*}
Considero che:
\begin{gather*}
  \vv{\nabla} (a \cdot \vv{B}) = \vv{B} \cdot \vv{\nabla} a + a \cdot (\vv{\nabla} \cdot \vv{B})\\
  \vv{\nabla} (p \vv{u}) = \vv{u} \cdot \vv{\nabla} p + p \cdot (\vv{\nabla} \cdot \vv{u})
\end{gather*}
Da quì posso scrivere l'ultimo passaggio \fbox{*} come:
\begin{gather*}
  = \vv{-\nabla} (p \vv{u}) + p (\vv{\nabla} \cdot \vv{u}) - \vv{\nabla} (\rho \phi_\mathcal{G} \vv{u}) + \phi_\mathcal{G} \vv{\nabla} (\rho \cdot \vv{u})\\
  = -\vv{\nabla} (p \vv{u}) - \vv{\nabla} (\rho \phi_\mathcal{G} \vv{u}) - \frac{\partial }{\partial t} (\rho \phi_\mathcal{G})
\end{gather*}
Infine tornando all'equazione iniziale \fbox{2} posso scrivere che il termine di sinistra diventa:
\begin{gather*}
  \rho \mathcal{G} \frac{\partial \vv{u}}{\partial t} = \rho \frac{\partial }{\partial t} \left( \frac{u^2}{2} \right)\\
  = \frac{\partial }{\partial t} \left( \rho \frac{u^2}{2} \right) - \frac{u^2}{2} \frac{\partial \rho}{\partial t}
\end{gather*}
Per quanto riguarda il termine a destra:
\begin{gather*}
  \vv{\nabla} (\rho \frac{u^2}{2}) = \sum_k \frac{\partial}{\partial x} \rho \frac{u^2}{2} u_k = \frac{u^2}{2} \sum_k  \frac{\partial \rho}{\partial x_k} \rho u_k + \rho \sum_k u_k \frac{\partial }{\partial x_k} \left( \frac{u^2}{2} \right) \\
  = \frac{u^2}{2} \vv{\nabla} \cdot (\rho \vv{u}) + \rho \sum_k u_k \sum_i \frac{\partial }{\partial x_k} \frac{u_i^2}{2}\\
  = \frac{u^2}{2} \vv{\nabla} \cdot (\rho \vv{u}) + \rho \sum_k \rho u_k \sum_i u_i \frac{\partial}{\partial x_k} u_i\\
  = \frac{u^2}{2} \vv{\nabla} \cdot (\rho \vv{u}) + \sum_i \rho u_i \sum_k u_k \frac{\partial }{\partial x_k} u_i\\
  \vv{\nabla} (\frac{\rho}{2} u^2 \vv{u}) = \frac{u^2}{2} \vv{\nabla} \cdot (\rho \vv{u}) + \rho \vv{u} \left[(\vv{u} \cdot \vv{\nabla}) \vv{u}\right] 
\end{gather*}
Ho scritto la divergenza per rho^2/2 per u che è proprio : $\sum_i \rho u_i \sum_k u_k \frac{\partial }{\partial x_k} u_i$
\begin{gather*}
  \frac{\partial}{\partial t} (\rho \vv{u}) + \vv{\nabla} \cdot (\rho \vv{u} \vv{u}) = -\vv{\nabla} p + \rho \vv{\nabla} \phi_\mathcal{G}\\
\end{gather*}
\end{document}
\documentclass[a4paper, oneside]{article}
\usepackage{wrapfig}
\usepackage{graphicx}
\usepackage{amsthm}
\usepackage{amsmath}
\usepackage{amssymb}
\usepackage[a4paper,
            bindingoffset=0.2in,
            left=2cm,
            right=2cm,
            top=2cm,
            bottom=2cm,
            footskip=.25in]{geometry}
\usepackage[italian]{babel}
\usepackage{pgfplots}
\usepackage{tabularx}
\usepackage{tikz-3dplot}
\usepackage{wrapfig}
\usepackage{color}
\usepackage{multicol}
\usepackage{arydshln}
\usepackage{mathtools}
\usepackage{enumerate}
\usepackage{graphicx}
\usepackage{svg}
\usepackage{cancel}
\usepackage[d]{esvect}
\usepackage[dvipsnames]{xcolor}
\usepackage{pgfplots}
\usepackage{pifont}
\usetikzlibrary{patterns}
%\usepackage{animate}
%\usepackage{xfp} % utile se vuoi fare calcoli aggiuntivi
\pgfplotsset{compat=1.18}
\usetikzlibrary{tikzmark}

\newcommand{\TikzNCbar}[4][10pt]{
\tikz[overlay,remember picture]{\draw[#2] (#3) --++(0,-#1) -| (#4);}}

\graphicspath{ {images/} }

\definecolor{redish}{rgb}{255, 0, 30}
\definecolor{page}{rgb}{0.129,0.157,0.212}
\pagecolor{page}
\color{white}   
\graphicspath{ {./images/} }
\usetikzlibrary{shapes.geometric}
\usetikzlibrary{datavisualization}
\usetikzlibrary{datavisualization.formats.functions}
\pgfplotsset{width=10cm,compat=1.9}

\setlength\dashlinedash{0.2pt}
\setlength\dashlinegap{1.5pt}
\setlength\arrayrulewidth{0.3pt}

\newcommand\eqq{\stackrel{\mathclap{\normalfont\mbox{?}}}{=}}
\newcommand\bulletout  {\labelitemfont \textbullet}
\newcommand{\tab}{\hspace*{2em}}
\newcommand{\xmark}{
\tikz[scale=0.23] {
    \draw[line width=0.7,line cap=round] (0,0) to [bend left=6] (1,1);
    \draw[line width=0.7,line cap=round] (0.2,0.95) to [bend right=3] (0.8,0.05);
}}
\newcommand{\cmark}{
\tikz[scale=0.23] {
    \draw[line width=0.7,line cap=round] (0.25,0) to [bend left=10] (1,1);
    \draw[line width=0.8,line cap=round] (0,0.35) to [bend right=1] (0.23,0);
}}
% Comando:
%   \potato[opzioni]{(x,y)}{scala}
%
% Opzioni = facoltative (es. fill=red!20, draw=black, thick)
% (x,y)   = centro della patata
% scala   = fattore di scala
%
\def\potatoshape{
  (1,0) (2,1.5) (1.6,3) (0.3,2.7) (-0.4,1.2)
}
\newcommand{\potato}[3][draw=white]{
  \begin{scope}[shift={#2}, scale=#3]
    \draw[#1]
      plot [smooth cycle, tension=1]
      coordinates {\potatoshape};
  \end{scope}
}
 \newcommand{\hookbox}[1]{
\begin{center}
\hfill\break
\begin{tikzpicture}
\node[inner sep=0pt,outer sep=0pt,anchor=base] (A) {
\begin{minipage}{\dimexpr\linewidth-5em}
\centering
#1
\end{minipage}
};
% Draw the left bracket
\draw ([xshift=0pt]A.north west) -- ++(0, 0.5) -- ++(0.4, 0);
% Draw the right bracket
\draw ([xshift=0pt]A.south east) -- ++(0, -0.5) -- ++(-0.4, 0);
\end{tikzpicture}
\end{center}} 
\title{Fluidi-Termodinamica-Statistica}
\author{Gariboldi Alessandro}
\date{ }


\begin{document}

\newtheoremstyle{theoremEnv}
                {}          % Space above
                {}          % Space below
                {\slshape}  % Body font
                {}          % Indent amount
                {\bfseries} % Head font
                {.}         % Punctuation after head
                {\newline}         % Space after theorem head
                {}          % Theorem head spec
\theoremstyle{theoremEnv}

\newtheorem{definition}{Definizione}[section]
\newtheorem{theorem}{Teorema}[section]
\newtheorem{lemma}{Lemma}[section]
\newtheorem{observation}{Oss.}[section]
\newtheorem{corollary}{Corollario}[theorem]
\newtheorem{example}{Esempio}[section]
\newtheorem{problem}{Problema}[section]
\newtheorem{solution}{Soluzione}[section]
\newtheorem{proposition}{Proposizione}[section]


\maketitle

\section{Introduzione al corso}
Il corso di Fluidi si divide in tre parti 
\begin{enumerate}
    \item Termodinamica: studia processi non di carattere meccanico che si manifestano a livello
    macroscopico
    \item Statistica: la fisica statistica coniuga la meccanica con la statistica che si usa nelle applicazioni
    che vanno al di là della fisica classica: da un fondamento microscopico agli effetti della termodinamica.
    \item Meccanica dei fluidi: sistemi materiali macroscopici per lo studio dei fluidi.
\end{enumerate}
Prima di tutto si inizia con un appello che collega le tre parti;
poi si inizia con statica dei fluidi, dopo termodinamica, poi statistica e infine
dinamica dei fluidi. Questo perché alcuni strumenti di analisi due saranno
utilizzati per la parte di dinamica dei fluidi e statistica. \\
L'esame è diviso in tre domande orali (una per sezione) (niente prova scritta) con esercizi
applicati alla realtà e di effetto pratico (che sono fatti a lezione).
Il libro consigliato è "Elementi di meccanica di fluidi" di Egidio Landi, per la parte di termodinamica
"Termodinamica" di Enrico Fermi e per la parte di fisica statistica le dispense del prof
in quanto non ci sono libri che facciano al caso del corso. \\
Il vantaggio di studiare tre discipline in un corso è che si riesce ad avere una idea più
completa dei sistemi che si studiano.

\section{Statica dei fluidi}
\subsection{Introduzione alla statica dei fluidi}
Il caso più semplice della fisica è quello dell'approssimazione del
punto materiale; cosa succede allora se io aumentassi il numero di punti materiali
e li mettessi tutti insieme? Se si mettono insieme in modo da poter costituire un
corpo rigido allora avrei da risolvere diverse equazioni per poter determinare il comportamento
del corpo; nessun materiale reale gode di questa proprietà ma è solo una idealizzazione
che vale entro certe approssimazioni. Il modello del corpo rigido funziona allora molto
bene per i corpi solidi che in prima approssimazione sono ben descrivibili con questo modello mentre
nel caso dei fluidi questa approssimazione non è valida. \\
Nei sistemi fluidi si considera allora il movimento degli atomi neutri
all'interno del fluido (non di plasmi) e quindi si può, utilizzando le conoscenze
di Fisica I come questi si muovono.

\subsubsection{L'atomo di Feyman}
In questo studio dei fluidi utilizziamo la definizione di atomo di Feyman, ossia
atomi che si attraggono se sono sufficientemente vicini ma se troppo vicini si respingono.
Essenzialmente a livello macroscopico l'atomo è visto come una pallina che costituisce il corpo. Possiamo
allora definire diverse proprietà della materia che utilizziamo per la descrizione dei fluidi:
\begin{itemize}
    \item Gli atomi sono dei punti materiali che hanno una certa massa $m$ che obbediscono alle leggi della meccanica
    di Newton;
    \item Un campione di materia è un insieme di $N$ atomi (ossia un numero molto grande $N>>>>1$) che
    possono essere chiusi in un contenitore ideale (ossia con pareti rigide e fisse).  
    \item Gli atomi interagiscono tra di loro solo tramite forze conservative, ossia che possono
    essere derivate e che queste forze dipendono esclusivamente dalla distanza tra
    gli atomi stessi; questo vuol dire che compiono solo urti completamente elastici con il contenitore.
    \\(in teoria dipendono anche in base a come sono orientate le molecole ma questo non è rilevante in questo caso).
\end{itemize}
Posso allora definire le forze intermolecolari. Possiamo definire $r_0$ come distanza caratteristica, sulla quale mi baso per distinguere i diversi casi (come si comportano gli atomi fra di loro)
\begin{itemize}
    \item Se la distanza tra i due atomi è maggiore ma dello stesso ordine di $r_0$,
    allora le forze sono attrattive.
    \item Se la distanza è minore di $r_0$, allora la forza è repulsiva. 
    \item Le interazioni decadono molto velocemente con l'aumentare della distanza. Quindi se ho distanze molto maggiori di $r_0$ anche di pochi ordini di grandezza sopra avrò forze trascurabili (attenzione a questo punto perchè è completamente vero su una coppia di atomi ma quando considero $N>>>>1$ atomi la cosa cambia).\\
    Questa distanza è indicata con $a$ ed è definita come:
\end{itemize}
\begin{align}
    \exists \ a \ t.c. \ r > a r_o \to 0 \Rightarrow F \to 0
\end{align}
Quindi se allontano due atimi di $a$ volte la distanza $r_0$ (caratteristica fissa) la forza tra i due tenderà a $0$.

\subsubsection{Il potenziale di interazione}
\begin{wrapfigure}{r}{0.4\textwidth}
    \centering
    \caption{Il potenziale di interazione}
    \begin{tikzpicture}
        \draw[->](-1, 0) -- (4, 0) node[at end, below] {$r$};
        \draw[->](0, -2) -- (0, 2) node[at end, left] {$V(r)$};
        \draw[red](0.5, 2) .. controls (1.2, -2) and (1.7, -2) ..  (3, 0);
        \draw[red](3, 0) -- (4, 0);
        \draw[dashed, red](4, 0) -- (5, 0);
        \draw[dashed](3, 0) -- (3, -1)node[at end, right] {$a \cdot r_0$};
        \draw[dashed](1.7, 3) -- (1.7, -2) node[at end, right] {$r_0$};
    \end{tikzpicture}    
\end{wrapfigure}
Posso allora definire il \textbf{potenziale di interazione} tra due atomi come $V(r)$ e 
quindi posso relazionare le tre condizioni precedenti con il potenziale e devo ottenere diverse condizioni:\\
intanto questa $V(r)$ deve diventare 
costante quando i due oggetti si trovano a grandi distanze questo perchè la forza è
trascurabile e riflette il fatto che siano molto lontani. 
Posso fissare un sistema di riferimento fisso e inerziale con il centro di massa e dunque
posso dire che l'energia totale del sistema sarà data da
\begin{align}
    E = K + V\ \Longrightarrow \ \frac{1}{2}m\sum |\vec{v}_i|^{2}  + U(\vec{r}_1, \dots, \vec{r}_n  )
\end{align}
Ora ho una parte di energia cinetica $K$ che riscrivo con la sommatoria e una parte di energia potenziale $E$ che voglio descrivere proprio con la mia funzione $V(r)$.\\
Infatti data l'ipotesi che ci siano solo delle forze conservative all'interno del sistema
che studiamo, allora posso riscrivere la componente potenziale (la quale dipende da $n$ variabili),
è scrivibile come la somma di tutte le possibili coppie e dunque
\begin{align}
    U(\vec{r}_1, \dots, \vec{r}_n  ) = \sum_{i = 0}^{n} \sum_{j = i + 1}^{n}  V(\left| \vec{r}_i - \vec{r}_j   \right| )
\end{align}
Da notare che ho riscritto la parte di energia potenziale proprio come la somma delle mie $V(r)$ con la distanza tra ogni atomo e quindi sono partito da $j=i+1$ per evitare di contare un atomo che interagisce con se stesso.
Posso anche esprimere anche l'energia potenziale attraverso la sommatoria rispetto a tutte le coppie
escludendo l'elemento già considerato (quindi si dimezza la sommatoria poiché il numero di coppie da considerare è $\frac{N(N - 1)}{2} \approx Nn_s$):
\begin{align}
    U(\vec{r}_1, \dots, \vec{r}_n  ) = \frac{1}{2}\sum_{i = 1}^{n} \sum_{j = 1 (i \neq j)}^{n} V(\left| \vec{r}_i - \vec{r}_j   \right| )  
\end{align}
I termini contano solo se la distanza è minore $a$ volte di $r_0$: fissato allora un istante di tempo
posso scegliere un atomo e dire quali coppie possono effettivamente contribuire
all'energia potenziale e quindi ogni atomo interagisce con gli atomi vicini che si trovano
nella sfera di raggio $a$ volte $r_0$ (e quindi posso esprimerlo come $a \cdot  r_0$), posso allora approssimare con un pochino meno termini
rispetto a $\frac{N(N -1 )}{2}$(che è proprio il numero delle coppie di $N$ atomi) in quanto ogni atomo interagisce solo con gli altri
all'interno della sfera di interazione.
\begin{gather*}
    U(\vec{r}_1, \dots, \vec{r}_n  ) = \frac{1}{2}\sum_{i = 0}^{n}\sum_{j \in S(i)}^{n}V(\left| \vec{r}_i - \vec{r}_j   \right| )  
\end{gather*}
Quanti sono allora gli atomi massimi consentiti all'interno della sfera di interazione
$S(i)$ rispetto ad un dato atomo $j$? Se prendessi allora questa sfera di raggio 
$a \cdot r_0$, si ottiene una stima che dipende da $r_0$ e lo divido per il volume della singola sferetta, ovvero $r_0$ 
stesso poichè sappiamo che oltre quella soglia gli atomi si respingono e quindi considero ogni atomo come una sferetta di raggio $r_0/2$
(il quale è sempre una approssimazione per eccesso perché non considero che siano rigide, ovvero sto calcolando quanto volume di una sferetta di raggio $r_0/2$ è contenuta in una sfera di raggio $r_0 \cdot a$).\\
La sfera di interazione ha raggio $a \cdot r_0$ e il singolo atomo ha raggio $\frac{r_0}{2}$
e quindi devo fare il rapporto tra i volumi e ottengo $n_s \approx (2a)^{d}$ dove $d$ è la dimensione (piano = 2, spazio = 3).\\
Se infatti considero per d=3 quindi il volume della sfera di interazione sul volume della sfera degli atomi ho:
\begin{gather*}
    n_s = \frac{\cancel{\frac{4}{3}\pi} (a \cancel{r_o})^3}{\cancel{\frac{4}{3}\pi} (\frac{\cancel{r_0}}{2})^3} = (2 a)^3
\end{gather*}
\hfil\\
Si ottiene allora che per $a = 2.5$, $d = 2$ si ha $n_s \approx 25$, per $d = 3$ e $a = 5$ si ha $n_s \approx 125$.
La dimensione è legata a partire dal volume della sfera, infatti il raggio è elevato alla dimensione
dello spazio considerato. \\
L'energia potenziale degli atomi nella materia è direttamente proporzionale al numero di atomi
infatti si ottiene che $N \propto n_s$, in questo caso quando una grandezza soddisfa questa proprietà allora
è una grandezza \textbf{estensiva}; la maggioranza dei termini non conta e dunque l'energia di un sistema è
una grandezza estensiva: se così non fosse allora la materia non si comporterebbe come si comporta.

\subsubsection{L'interazione tra due parti di uno stesso contenitore}
\begin{wrapfigure}{r}{0.3\textwidth}
    \centering
    \caption{L'interazione tra due recipienti uniti}
    \begin{tikzpicture}
        \draw(0, 0) rectangle (3, 2);
        \filldraw (0.5, 1) circle (0pt) node[anchor = south] {$N_1$};
        \filldraw (2, 1) circle (0pt) node[anchor = south] {$N_2$};
        \draw(1, 2) -- (1, 0);
        \draw[dashed](1.2, 2) -- (1.2, 0);
        \draw[dashed](0.8, 2) -- (0.8, 0);
        \filldraw(4, 1.5) circle (0pt) node[anchor = south] {$N_1, N_2 \propto L^{d} $ };
        \filldraw(4, 1) circle (0pt) node[anchor = south] {$N_{1 \ 2} \propto L^{d - 1} $ };
    \end{tikzpicture}    
\end{wrapfigure}
Immaginando di suddividere il contenitore in due parti allora un po' del fluido sta da una parte e un
po' sta dall'altra: questa divisione va fatta in modo tale che le due parti debbano essere macroscopiche
(non tre atomi da una parte e il resto dall'altra). Il quesito è ora come posso determinare l'energia di questo sistema?
Dato allora il sistema $N_1 + N_2 = N$, ossia il numero di atomi del primo più il secondo 
mi dà gli atomi totali e che siano comunque molto grandi rispetto a $1$.  Devo dividere il contenitore
in modo tale che il "divisore" sia di una dimensione in meno dello spazio considerato (es nello spazio un piano)
in modo tale che sia perfettamente dritto. \\
Posso allora esprimere l'energia totale come
\begin{gather*}
    E = E_1 + E_2 + E_{1,2} \\
    E_1 =  \frac{1}{2}\sum_{i = 1}^{N_1}\left| \vec{v^{1} }_i  \right|^{2} + U (\vec{r}_1^{(1)} , \dots \vec{r}_N^{(1)}   )  \propto N_1  \\
    E_2 =  \frac{1}{2}\sum_{i = 1}^{N_2}\left| \vec{v^{2} }_i  \right|^{2} + U (\vec{r}_1^{(2)} , \dots \vec{r}_N^{(2)}   )  \propto N_2
\end{gather*}
Posso esprimere il contributo energetico dovuto all'interazione tra le due parti del contenitore
come la seguente formula:
\begin{gather*}
    E_{1, 2} = \frac{1}{2}\sum_{i = 1}^{N_1}\sum_{\substack{j \neq i \\ j = 1}^{N_2}} V\left(\left| \vec{r^{N_1} }_i   - \vec{r^{N_2} }_j   \right| \right)  
\end{gather*}
Il contributo di questa interazione è molto più piccola dell'interazione $E_1$ e $E_2$ in quanto solamente gli atomi
molto vicini al bordo che divide i due sistemi interagiscono tra di loro e dunque
più cresce $N$ allora più diminuisce questo contributo. Inoltre si osserva che sono proporzionali ad $L$ (la lunghezza della parete attorno alla quale stiamo considerando gli atomi) che è dato
dai limiti fisici del contenitori (chiamata scala libera) e infatti,
ricordando che $\ N_{1,2} \propto L^{(d-1)}$ perché si considera proporzionale al separatore quindi se siamo in un cubo il separatore è un piano se siamo in un quadrato il separatore è una linea ecc..., si ha che:
\begin{gather*}
    \begin{cases}
        N_1,N_2 \propto L^d\\
        N_{1,2} \propto L^{d-1}
    \end{cases} 
    \xRightarrow{\text{elevo alla} \ \frac{1}{d}}  
    \begin{cases}
        ({L^{\cancel{d}}})^{\cancel{\frac{1}{d}}} \propto (N_1)^{\frac{1}{d}}\\
        (N_{1,2})^{\frac{1}{d}} \propto (L^{d-1})^{\frac{1}{d}}
    \end{cases}
    \ \Longrightarrow \ N_{1, 2} \propto N_1^{\frac{d- 1}{d}}\\
    \text{es. per un volume a 3 dimensioni } N_{1,2} \propto N_1^{\frac{2}{3}} \to \sqrt[3]{N_1^2}
\end{gather*}
Se $N_1$ cresce, allora $N_{1, 2}$ diventa molto piccolo.  Si ha allora che l'energia complessiva di un sistema
che ho arbitrariamente diviso in due pezzi allora l'energia che io ho sarà la somma dei due contributi ignorando il
termine di interazione tra le due parti purché i due pezzi siano macroscopici (infatti non sarebbe vero
se una delle due componenti avesse pochi atomi).
\begin{align}
    E \approx E_1 + E_2 \qquad \qquad per \  N >>>> 1 \ \Longrightarrow \ E = E_1 + E_2
\end{align}
Allora ottengo che l'energia è una \textbf{quantità additiva} (infatti è vero per $m$ divisioni fino a che
le componenti sono macroscopiche). Tutte queste considerazioni sono valide \underbar{se e solo se}
io continuo a ignorare i contributi all'esterno della sfera di interazione e quindi quei contributi convergano a zero in modo
sufficientemente rapido: infatti l'analisi ci insegna che somme di piccole quantità molto grandi non sempre
convergono a numeri finiti. Si può dimostrare che devono convergere a zero più velocemente di 
$r^{-d} $ (per l'interazione gravitazionale questo infatti non è vero). 
\\\\
(Quello che ho scritto io)
\\La premessa che ho fatto inizialmente che pone un problema è proprio che quando una coppia è molto distante io considero energia nulla, ma gli atomi sono un numero \textbf{MOLTO} grande quindi devo far si che l'energia tenda rapidamente a $0$, più rapidamente di $r^{-d}$.\\
Queste cose valgono solo per oggetti (ad es.) quindi $d = 3$ cehe fanno decadere l'energia rapidamente, ad es. corpi che interagiscono con la gravità NON lo sono.


\newpage
\section{09/09/25}
Prendiamo una coppia di molecole voglio quantificare le forze repulsive e attrattive tra di loro:
\begin{multicols}{2}
    \noindent

    \begin{theorem}
   \tab Th. di Lennard-Jones\\
   Questo teorema ci afferma che:\\
   \begin{gather*}
    E^{\text{(total)}}_{PE} = E^{\text{(repulsive)}}_{PE} + E^{\text{(attractive)}}_{PE} = +\frac{B}{r^{12}} - \frac{A}{r^6}
   \end{gather*}
   Con $A$ e $B$ caratteristiche intrinseche delle molecole, da notare che le forze repulsive sono sempre positive mentre quelle attrattive sono sempre negative.\\
   Quando queste forze ($E^{\text{(repulsive)}}_{PE}$ e $E^{\text{(attractive)}}_{PE}$) sono uguali la distanza $r$ è $0$ e questa distanza la identifichiamo con $\sigma$.\\
   Ora passiamo a scrivere la formula in una maniera più rappresentativa dove esprimiamo $A$ e $B$ in funzione di una nuova quantità $\epsilon$.
    \begin{gather*}
        \epsilon \ r_0^{12} = B \ \text{e} \ \epsilon \ 2r_0^6 = A\\ \text{Quindi $\epsilon$ è proprio una caratteristica della coppia di} \\ \text{atomi che sto considerando\dots}\\
        V(r) = \epsilon[(\frac{r_0}{r})^{12}-2(\frac{r_0}{r})^6]\\
        V(r) = 4\epsilon[(\frac{\sigma}{r})^{12}-2(\frac{\sigma}{r})^6]\\
    \end{gather*}
\end{theorem}
Ad esempio con $r_0 = 2^{\frac{1}{6}} \sigma$ avrò che $r_0 \simeq 1.12 \sigma$
\columnbreak

\begin{tikzpicture}
    \draw[->] (-1, 0) -- (4.5, 0) node[at end, below]{$r/r_0$};
        \draw[->](0, -1) -- (0, 2.5) node[at end, left]{$V/\epsilon$};
        \draw[red](0.75, 3) ..controls (1, -2.5) and (1.35, -0.25) .. (2, -0.05) node[at start, red, left]{$E^{\text{(total)}}_{PE}$};
        \draw[red](2, -0.05) -- (3.5, -0.05);
        \draw[red, dashed](3.5, -0.05) -- (4, -0.05);
        \draw[cyan, dashed](1.27, -1.5) ..controls(2, -0.1)and(3,-0.2) ..  (4, -0.05) node[midway, cyan, below]{$- \frac{A}{r^6}$};
        \draw[green, dashed](0.75, 3.2) ..controls (1, 1.5) and (1.35, 0.25) .. (2, 0.05);
        \draw[green, dashed](2, 0.05) -- (4, 0.05) node[midway, green, above]{$+\frac{B}{r^{12}}$};
        \draw[magenta, <->](1.28, 0) -- (1.28, -0.67) node[midway, left, magenta] {$\epsilon$} node[white, at start, above]{$r_0$};
        \draw[<->](0.05, 0.1) -- (0.85,0.1) node[midway, above]{$\sigma$};
        \filldraw[white] (1.28,0) circle(0.5 pt);
    \end{tikzpicture}

Se facciamo l'esempio di energia potenziale di una molla questa oscillera ed avrà ung rafico simile a:
\begin{tikzpicture}
    \begin{axis}[
    axis lines=middle, % Puts axes in the middle of the plot
    xlabel={$x$},      % Label for x-axis
    ylabel={$V(x) = \frac{1}{2} k x^2$},      % Label for y-axis
    xmin=-7, xmax=7,   % X-axis limits
    ymin=-2, ymax=10,  % Y-axis limits
    xtick=\empty,
    ytick=\empty,
    width = 6cm,
    height = 6cm
]
\addplot[
    domain=-4:4,      % Domain for the function
    samples=100,      % Number of points to plot for smoothness
    smooth,           % Makes the line smooth
    blue              % Color of the line
] {x^2 * 1/2}; % The function to plot (y = x^2)

\node at (250,430) {$g(z) = z^\alpha \quad \alpha\in(0,1)$};
\node at (120,200) {[};
\node at (120,170) {a};
\node at (220,200) {]};
\node at (220,170) {b};
\end{axis}

\end{tikzpicture}
\\ con $k$ cost. elastica del materiale
\end{multicols}
\hfil\\
Valori tipici di $\epsilon$ variano:
      \begin{gather*}
        \text{da } 10^{-3} \text{ a } 10^{-2} eV\\
    \end{gather*}
Valori tipici di $\sigma$ variano:
    \begin{gather*}
        \text{da } 0.2 \text{ a } 0.4 \ nm\\
    \end{gather*}
Passiamo ora a calcolare effettivamente l'interazione tra più atomi
ricordando sempre che:
    \begin{gather*}
    E = \frac{1}{2} m \sum_{i = 1}^{N}|\vec{v_i}|^2+U(\vec{r_1}, ... \ ,\vv{r_N})\\
    U = \frac{1}{2}\sum_{i = 1}^{N}\sum_{j = 1}^{N}V(|\vec{r_1}, ... \ ,\vv{r_j}|)
    \end{gather*}



Prendo tre termini e li fo interagire tra loro\dots
\begin{multicols}{2}
    \noindent
\begin{gather*}
    \text{per ognuno avrò le mie coordinate: }(x,y)\\
    \text{Prendiamo un esempio specifico:}\\
    (x_1,y_1) = (0,0)\\
    (x_2, y_2) = (r_0, 0)\\
    (x_3, y_3) = (\frac{1}{2}r_0, \frac{\sqrt{3}}{2r_0})\\
    \text{Posso calcolare l'energia totale facendo:}\\
    U = V(r_{12}) + V(r_{13}) + V(r_{23})
\end{gather*}

    \columnbreak

    Disposizione degli atomi: a triangolo equilatero\\
    \begin{wrapfigure}{i}{0.5\textwidth}
    \begin{center}
            
    \begin{tikzpicture}
        \filldraw [blue] (0,0) circle (2pt);
        \filldraw [blue] (1,0) circle (2pt);
        \filldraw [blue] (0.5, 0.866) circle (2pt);
        \draw[dashed](0, 0) -- (1, 0);
        \draw[dashed](0, 0) -- (0.5, 0.866);
        \draw[dashed](1, 0) -- (0.5, 0.866);
        \node at (-0.7,0) {$(0,0)$};
        \node at (1.7,0) {$(r_0,0)$};
        \node at (0.5,1.2) {$(\frac{1}{2}r_0, \frac{\sqrt{3}}{2r_0})$};
    \end{tikzpicture}

    \end{center}
    \end{wrapfigure}

\end{multicols}


\hfil\\
Questa cosa la posso fare fino a che $N$ è relativamente piccolo più $N$ cresce più il problema di calcolarlo aumenta esponenzialmente.
\\Per $N$ grandi si può considerare che l'energia che stiamo calcolando è proprio l'energia potenziale del sistema, e quindi si procede in questo modo: 

\begin{multicols}{2}
    \noindent
Si definiscono due vettori che useremo per descrivere il reicolo a esagono degli atomi:
\begin{gather*}
    \vec{a} = r_0 \widehat{i}\\
    \vec{b} = \frac{r_0}{2}(-\widehat{i} + \sqrt{3} \widehat{j})\\
\end{gather*}
ogni atomo avrà coordinate ottenute come combinazione di $\vv{a}$ e $\vv{b}$\\
\begin{gather*}
        \vv{r_{n,m}} = n \vec{a} + m \vec{b}  \qquad n,m \in \mathbb{Z}\\
\end{gather*}
Questa è semplicemente la definizione di un \textbf{reticolo esagonale}, tipico di un solido 2D (tipo il grafene).
    \columnbreak
    \begin{wrapfigure}{i}{0.5\textwidth}
        \vspace*{-10 mm}
        \begin{center}
        \includegraphics[width=0.3\textwidth]{atoms.png}
        \end{center}  
    \end{wrapfigure}

\end{multicols}

Si trova che l’energia totale del sistema (molti atomi) ha la forma:
\begin{gather*}
    E = E_{min} +\Delta E\\
\end{gather*} 
Dove:
\begin{itemize}
    \item     $E_{min}$ è l’energia minima (quando gli atomi sono nella configurazione “perfetta” a distanza $r_0$) per calcolarla si considera un atomo e i suoi $6$ vicini, sappiamo che l'energia di una coppi di atomi a a distanza $r_0$ è di $-\epsilon$, se dividiamo questa energia per due otteniamo l'energia fornita da un atomo, a quel punto ci basta moltiplicarla per $N$.
\begin{gather*}
    E_{min} = -\frac{6}{2}N\epsilon = -3N\epsilon
\end{gather*}
    \item $\Delta E$ è l’energia in più del sistema rispetto al minimo, dovuta alle oscillazioni degli atomi attorno a $r_0$.\\
    Può anche essere pensata come la somma di due contributi:
    \begin{itemize}
        \item \textbf{energia cinetica} delle particelle (perché non stanno ferme, vibrano)
        \item \textbf{energia potenziale extra} (perché si spostano da $r_0$).
    \end{itemize}
\end{itemize}

\hfil\\
Ora, se la vibrazione è piccola (quindi $\Delta E << |E_{min}|$), in media l’energia cinetica e quella potenziale extra si dividono più o meno a metà (questa è la base del teorema di equipartizione).\\
Quindi se si considera $\Delta E$ come tutta l'energia "in eccesso" del sistema e si divide per il numero di atomi $N$, si ottiene proprio l'energia cinetica media di una particiella:
\begin{gather*}
    \frac{1}{2}m |\vec{v_i}|^2 \approx \frac{\Delta E}{N}
\end{gather*}
\hfil\\
Tenendo in conto tutte quese considerazioni posso riscrivere l'energia totale del sistema come:
\begin{gather*}
    E_{tot} = E_{n,m} \approx \frac{1}{2}\sum_{i = 1}^{N}\sum_{j = 1}^{z}V(r_0) = -3N\epsilon\\
    \Delta E << |E_{min}|
\end{gather*}
dove $z$ è il numero di atomi adiacenti all'atomo $i-esimo$ che stiamo considerando, nel reticolo 2D si ha $z=6$,
ricordiamo inoltre che abbiamo usato l'ipotesi che $\Delta E << |E_{min}|$, quindi stiamo considerando che il mio sistema ha vibrazioni minime e quindi trascurabili (non ha molta energia).

\hfil\\
Come possiamo approssimare l'energia di una generica coppia di atomi?\\
Se vogliamo studiare il comportamento degli atomi vicino alla distanza di equilibrio $r_0$, possiamo fare uno sviluppo in serie di Taylor attorno a $r_0$
\begin{gather*}
    V(r_{ij}) \approx V(r_0) +\frac{1}{2} V''(r_0)(r_{ij}-r_0)^2 = -\epsilon +\frac{1}{2} V''(r_0)(r_{ij}-r_0)^2
\end{gather*}


\hfil\\
\begin{itemize}
    \item Se si da tanta energia al sistema l'energia potenziale dei singoli atomi sara grande relativamente a $\epsilon$ e quindi l'energia attrattiva sarà poco rilevante infatti il mio modello si comporta come un gas.
    \item Se ho un'energia piccola rispetto ad $\epsilon$ avrò una forma solida poichè avrò anche che $\Delta E$ è trascurabile quindi abbiamo proprio uno schema vicino ad un solido perfetto.
    \item Se ho un'energia tendenzialmente piccola rispetto ad $\epsilon$ ma comparabile ad essa avrò un liquido.
\end{itemize}
\[yapping \ shit:\]Per il liquido al livello macroscopico le cose si comlicano. Al livello di compressione un liquido e un solido sono uguali (un liquido è addirittura più isotropo), ma ad altri tipi di forze si comporta in maniera completamente diversa, in più si adatta alla forma del suo contenitore.\\

\section{19/09/25}
\begin{multicols}{2}
    \noindent
\begin{tikzpicture}   
        \filldraw [blue] (0,0) circle (2pt);
        \filldraw [blue] (1,0) circle (2pt);
        \filldraw [blue] (0.5, 0.8) circle (2pt);
        \filldraw [blue] (0.5 , -0.8) circle (2pt);
        \filldraw [blue] (-1 , 0) circle (2pt);
        \filldraw [blue] (-0.5 , 0.8) circle (2pt);
        \filldraw [blue] (-0.5 , -0.8) circle (2pt);
        \draw[dashed](0, 0) -- (1, 0);
        \draw[dashed](0, 0) -- (0.5, 0.8);
        \draw[dashed](0, 0) -- (0.5, -0.8);
        \draw[dashed](0, 0) -- (-1, 0);
        \draw[dashed](0, 0) -- (-0.5, 0.8);
        \draw[dashed](0, 0) -- (-0.5, -0.8);
        
\end{tikzpicture}
    \columnbreak
    \hfil\\
    Questa disposizione per gli atomi rappresenta una gabbia ed hanno tutti la stessa distanza dall'atomo centrale inoltre sono posizionati come i vertici di un esagono regolare,quindi l'atomo centrale, anche se gli venisse fornita un energia molto grande, non è in grado di muoversi.
\end{multicols}


\begin{gather*}
    U(x,y) = \sum_{i=1}^{6}V(|\vec{r}-\vec{r_i}|)\\
    E = \frac{1}{2} m ({v_x}^2 + {v_y}^2) + U(x,y)\\
    A = (x,y) \in \mathbb{R}^2 \ | \ U(x,y) \leq E\\
    \end{gather*}
Dove A è proprio linsieme dei punti del piano accessibili al moto
\hfil\\\hfil\\\hfil\\\hfil\\\hfil\\\hfil
\begin{multicols}{2}
    \noindent
    Prima guardiamo un caso ideale in cui le particielle attorno all'atomo centrale supponiamo restano ferme
    \columnbreak
    \begin{tikzpicture}   
        \draw[->](-1.5, -2) -> (-1.5, 2);
        \draw[->](-2, -1.5) -> (2, -1.5);
        \filldraw [blue] (0,0) circle (2pt);
        \filldraw [blue] (1,0) circle (2pt);
        \filldraw [blue] (0.5, 0.8) circle (2pt);
        \filldraw [blue] (0.5 , -0.8) circle (2pt);
        \filldraw [blue] (-1 , 0) circle (2pt);
        \filldraw [blue] (-0.5 , 0.8) circle (2pt);
        \filldraw [blue] (-0.5 , -0.8) circle (2pt);
        \draw (1, 0) (0.5, -0.8);
        \draw(0.5, -0.8) -- (-0.5, -0.8);
        \draw(-1, 0) -- (-0.5, -0.8);
        \draw(-1, 0) -- (-0.5, 0.8);
        \draw(-0.5, 0.8) -- (0.5, 0.8);
        \draw(0.5, 0.8) -- (1, 0);
        \draw(1,0) -- (0.5, -0.8);

        
\end{tikzpicture}

\end{multicols}
\begin{multicols}{2}
    \noindent
    Se consideriamo una situazione piu realistica avremo che i nodi distanti verranno allontanati daò nodo centrale e avremo una situazione di questo tipo
    \columnbreak
    \begin{tikzpicture}   
        \draw[->](-1.5, -2) -> (-1.5, 2);
        \draw[->](-2, -1.5) -> (2, -1.5);
        \filldraw [blue] (0,0) circle (2pt);
        \filldraw [blue] (1,0) circle (2pt);
        \filldraw [blue] (0.5, 0.8) circle (2pt);
        \filldraw [blue] (0.5 , -0.8) circle (2pt);
        \filldraw [blue] (-1 , 0) circle (2pt);
        \filldraw [blue] (-0.5 , 0.8) circle (2pt);
        \filldraw [blue] (-0.5 , -0.8) circle (2pt);
        \draw(1, 0) ..controls(0.6, -0.4).. (0.5, -0.8);
        \draw(0.5, -0.8) ..controls(0, -0.6).. (-0.5, -0.8);
        \draw(-1, 0) ..controls(-0.8, -0.2).. (-0.5, -0.8);
        \draw(-1, 0) ..controls(-0.5, 0.4).. (-0.5, 0.8);
        \draw(-0.5, 0.8) ..controls(0, 0.7).. (0.5, 0.8);
        \draw(0.5, 0.8) ..controls(0.5, 0.5).. (1, 0);

        
\end{tikzpicture}

\end{multicols}
\begin{multicols}{2}
    \noindent
    Se ora consideriamo la situazione in cui i due atomi superiori si allontanano della stessa distanza avremo una situazione in cui si crea uno spazio in cui l'atomo carico di energia può scappare verso l'alto
    \columnbreak
    \begin{tikzpicture}   
        \draw[->](-1.5, -2) -> (-1.5, 2);
        \draw[->](-2, -1.5) -> (2, -1.5);
        \filldraw [blue] (0,2) circle (2pt);
        \filldraw [blue] (1,0) circle (2pt);
        \filldraw [blue] (1, 0.8) circle (2pt);
        \filldraw [blue] (0.5 , -0.8) circle (2pt);
        \filldraw [blue] (-1 , 0) circle (2pt);
        \filldraw [blue] (-1 , 0.8) circle (2pt);
        \filldraw [blue] (-0.5 , -0.8) circle (2pt);
        \draw(1, 0) ..controls(0.6, -0.4).. (0.5, -0.8);
        \draw(0.5, -0.8) ..controls(0, -0.6).. (-0.5, -0.8);
        \draw(-1, 0) ..controls(-0.8, -0.2).. (-0.5, -0.8);
        \draw(-1, 0) ..controls(-0.8, 0.4).. (-1, 0.8);
        \draw(1, 0.8) ..controls(0.8, 0.5).. (1, 0);
        \draw(-0.3, 2) ..controls(-0.3, 1.2).. (-1, 0.8);
        \draw(0.3, 2) ..controls(0.3, 1.2).. (1, 0.8);

        
\end{tikzpicture}

\end{multicols}

  \begin{enumerate}[$i)$]
    \item solido $\to \frac{\Delta E}{N} << \epsilon$
    \item liquido $\to \frac{\Delta E}{N} \approx \epsilon$
    \item gas $\to \frac{\Delta E}{N} >\approx 3\epsilon$
    \end{enumerate}


    \paragraph{Calcolo di un sistema di particielle}
    \hfil\\Cosa dovrei fare per calcolare con $F=ma$ in un sistema di molecole di 1 bicchiere d'acqua?
    \begin{gather*}
        \vec{F_i} = m \vec{a_i} \quad 1, ... \ , N \quad N >>>>>>> 1\\
        \text{1 bicchiere di acqua}, \quad 13.89 mol \quad N = 8363735156400000000000000 \qquad N \approxeq 8.4 10^24\\
    \end{gather*}
    Sono $10^{25} \ Byte, 10^{16} \ GB, 10^{13} \ TB$ dunque non è qualcosa di risolvibile in maniera esatta ne ora ne tra molto tempo\\
    \\Quindi posso pensare di cercare di semplificare il problema di considerare meno molecole e cercare di calcolarlo sfruttando calcolatori, i primi a pensarlo furnono:
    \begin{enumerate}[ ]
        \item Enrico Fermi
        \item Jhon Pasta
        \item May Tsingou
        \item Stanislav Ulan
    \end{enumerate}
    Fermi la ha carriata il progetto fu chiamato FPU perchè l'altro aveva fatto il solo codice e lo aveva preso in culo , ora purtroppo siccome siamo in una società in cui LGBTQ puo parlare lo abbiamo dovuto rimettere e si chiama FPUT.\\

    \paragraph{Calcolo del sistema di $10^6$ particielle}
    \hfil\\
    Chiamo $L$ la lunghezza del lato del sistema che stiamo considerando, come un cubo ad es, e $V$ il suo volume\\
    \begin{gather*}
        N \propto V = L^{3} \rightarrow L \propto N^{1/3}\\
        N_{bordo} \propto L^2 \propto N^{2/3}\\
        \frac{N_{bordo}}{N} \propto \frac{N^{2/3}}{N} = N^{-1/3} \approx 10^{-2}
    \end{gather*}
    
    \section{23/09/25}
    \begin{proposition}
        INTRODUZIONE ALLA TERMODINAMICA\\
        è un modo di lavorare legato al concetto di campo in fisica (questo è il punto 3 gli altri 2 non li so). Saranno introdotti alcuni strumenti di calcolo con i campi e che serviranno a prepararsi alla seconda parte del corso e anche per altri corsi come astronomia.\\
    \end{proposition}
    \begin{definition}
        Fluidodinamica significa dare una definizione di fluido:\\
        Consideriamo le forze che agiscono su di esso:
        \begin{enumerate}[$i)$]
            \item Forze di volume: per ciascun elemento di massa $dm= \rho dV$, sono le forze proporzionali al volume dell'elemento, come la forza peso $P = gdm = g \ \rho dV$\\
            \item Forze di superficie: avvengono sulla superficie dell'elemento preso in considerzione e sono proporzionali alla superficie.\\
                Le forze di superficie possono essere scomposte in due forze:
                \begin{itemize}
                    \item forze parallele alla superficie $F_\parallel$, dette anche forze di taglio $F_\tau$
                    \item forze perpendicolari alla superficie $F_\perp$, dette anche forze normali $F_n$
                \end{itemize}
        \end{enumerate}
        Per la descrizione del fluido si considerano le \textbf{FORZE PER UNITÀ DI SUPERFICIE} dato che la superficie, come elemento del fluido, è arbitraria.
    \end{definition}

\begin{multicols}{2}
    \noindent

    \columnbreak
\end{multicols}
\begin{tikzpicture}   
        \draw(1, 0) ..controls(0.6, -0.4).. (0.5, -0.8);
        \draw(0.5, -0.8) ..controls(0, -0.6).. (-0.5, -0.8);
\end{tikzpicture}

\begin{gather*}
    \vec{F} = F_{\parallel} \widehat{n} + \vec{F_{\perp}}\\
    |F_{\perp}| << |F_{\parallel}| \qquad \text{\fbox{$F_{\parallel} > 0$}}\\
    F_{\parallel} = \vec{F} \widehat{n}\\
    \vec{F}_{BA} = -\vec{F}_{AB}
\end{gather*}
$F_{\parallel}$ è sempre positivo perchè se ho una situazione di questo tipo e il mio corpo è un fluido, questo tenderà sempre ad espandersi quindi la forza parallela è sempre positiva.\\
Mentre si ha $|F_{\perp}| << |F_{\parallel}|$ perchè se si ha una situazione di questo tipo si ha che le forze perpendicolari \\\\
Se invece consideriamo $|F_{\perp}| << F_{\parallel}$ e $F_{\perp} \equiv 0 \rightarrow$ si ha un FLUIDO IDEALE.\\\\
Ragioniamo su quest'ultimo.\\
Prendo un sistema di riferimento collegato al mio corpo\\
\begin{figure}[tbh]
    \centering
    \includesvg[scale=1.2]{potato}
    \caption{schema patata}\label{fig:testsvg}
\end{figure}
\begin{gather*}
    \vec{r} = x \widehat{x} + y \widehat{y} + z \widehat{z}\\
    -\vv{\Delta R} = \overline{P} \Delta \Sigma \widehat{n}\\
    \Delta R = \overline{P} \Delta \Sigma\\
    \overline{P} = \frac{\Delta R}{\Delta \Sigma}\\
    p(x,y,z) = \lim_{\Delta \Sigma \to 0} \frac{\Delta R}{\Delta \Sigma} \Rightarrow \text{dimensionalmente: } \frac{[F]}{[L]^2} = \frac{[E]}{[L]^2} \\
\end{gather*}
quest'ultima funzione, oltre ad essere la PRESSIONE, è un CAMPO SCALARE, dove ad ogni punto posso associare un pezzo di superficie\\
La misura nel S.I. è $N/m^2 \equiv$ Pascal 

\begin{proposition}
    Definisco ora la mia DENSITA per ogni punto, e definisco quindi un secondo CAMPO SCALARE
    \begin{gather*}
        \overline{\rho} = \frac{\Delta m }{\Delta V}\\
        \rho(x,y,z) = \lim_{\Delta V \to 0} \frac{\Delta m }{\Delta V}
    \end{gather*}
    Un fluido occupa tutto lo spazio disponibile se areiforme, mentre in generale se liquido sarà incomprimibile (tutto questo al livello ideale).\\
    Di fatto non avrò mai un liquido del tutto incomprimibilee quindi avrò che se raggiungo una certa pressione avrò una certa compressione.\\
    \begin{gather*}
        \frac{\Delta P}{\epsilon} = -\frac{\Delta V}{V_0}
        \begin{cases}
            \epsilon_{aria} \approxeq 10^5 Pascal\\
            \epsilon_{H_2O} \approxeq 10^9 Pascal\\
        \end{cases}
    \end{gather*}
\end{proposition}

\begin{tikzpicture}

        \draw[->](-1.5, -2) -> (-1.5, -0.8);
        \draw[->](-1.5, -2) -> (0, -2);
        \draw[->](-1.5, -2) -> (-2.3, -2.8);
        \draw(0, 0.5) ..controls(2.5, 0.7) and (0.5, 0.8).. (3, 0);
        \draw(0.5, -0.8) ..controls(0, -0.6).. (-0.5, -0.8);
        \node at (2.5,2) {$d\vec{F}_\Sigma = p \widehat{n} d\sigma$};
\end{tikzpicture}
\begin{gather*}
    -p \widehat{u} d\sigma\\
    \vec{R_\Sigma} = \oint_{\Sigma (V)} p \widehat{u} \ d\sigma\\ 
\end{gather*}
Dove $-p \widehat{u} \ d\sigma$ è proprio la forza di un punto materiale che agisce per cercare di
\\l'area del mio corpo che sto considerando è proprio $\Sigma$ e per questo fo l'integrale su $\Sigma(V)$ cioè sull'area del volume (idk tbh)\\
Adesso si prende un volume infinitesimo all'interno del corpo. Ne risulterà\dots
\begin{gather*}
    \overrightarrow{dF_V} = \vec{f}(x,y,z) dV\\
\end{gather*} 
Con questultimo creiamo invece un campo vettoriale.
\begin{gather*}
    \overrightarrow{dF_V} = d m \vec{g} = \rho \vec{g} \ dV\\
    \overrightarrow{R_V} = \int_V \rho \vec{g} \ dV
\end{gather*}
Ora sommo 
\begin{gather*}
    -\oint_{\Sigma V} p \widehat{n} \ d\sigma + \int_{V} \rho \vec{g} \ dV = 0\\
\end{gather*}
Ci mettiamo nelle condizioni in cui lr forze sul volume risultino trascurabili.\\
\begin{gather*}
    \overrightarrow{R_V} = \oint \rho \vec{g} \ dV\\
    \oint_{\Sigma (V)} \vec{g} \ dV \rightarrow p \oint  \vec{g} \ d\sigma \equiv 0\\
\end{gather*}

\begin{proof}
    \begin{gather*}
        dV = d\sigma \ \vec{\delta r_0} \widehat{n}\\
        \Delta V = \int_{\Sigma (V)} d\sigma \ \vec{\delta r_0} \widehat{n} = \delta r_0 \oint_{\Sigma} \widehat{n} \ d\sigma = 0
    \end{gather*}
\end{proof}
Non basta la risultante uguale a zero occorre anche che il momento delle forze sia uguale a 0.\\
Passiamo dunque a dimostrare questa cosa:
\begin{gather*}
    \oint_{\Sigma} \vec{r} \times p \widehat{u} \ d\sigma \equiv 0
\end{gather*}
Per vedere che la quantità di moto sia costante ruoto il mio volume\\
\begin{gather*}
    \delta r = \delta \alpha \widehat{\Omega}\times \vec{r}
\end{gather*}
di conseguenza\dots
\begin{gather*}
    \delta V = d\sigma \delta \alpha(\widehat{\Omega}\times \vec{r})\widehat{n}\\
    \Delta V = \int_{\Sigma(V)} d\sigma \ \delta \alpha(\widehat{\Omega}\times \vec{r})\widehat{n}\\
    \Delta V = \int_{\Sigma(V)} d\sigma \ \delta \alpha\widehat{\Omega}(\vec{r}\times \widehat{n}) = \delta \alpha \widehat{\Omega}\int(\widehat{r} \times \widehat{n}) \ d\sigma \equiv 0\\
\end{gather*}
Introduciamo il principio di Pascal, sostanzialmente dice che $p$ è costante.\\
\begin{gather*}
    p = \frac{F_1}{\Sigma_1} = \frac{F_2}{\Sigma_2} \\
    \text{bho ha cancellato}\\
\end{gather*}
Questo era l'esempio di pressa idraulica\\\\

Prendiam ora un volume piccolo, un parallelepipedo.
\begin{gather*}
    z: \frac{\rho g \ \Delta x \ \Delta y \ \Delta z}{\Delta x \ \Delta y \ \Delta z} -\frac{(p(x,y,z + \Delta z) \Delta x \ \Delta y) - p(x,y,z)\Delta x \ \Delta y }{\Delta x \ \Delta y \ \Delta z}= 0\\
    z: \rho g - \frac{p(x,y,z+\Delta z)-p(x,y,z)}{\Delta z}\\
    x: -\frac{p(x+\Delta x, y, z) \Delta y \ \Delta z p(x,y,z) \Delta y \ \Delta z}{\Delta x \ \Delta y \ \Delta z}=0\\
    x: \frac{-p(x + \Delta x, y, z) - p(x,y,z) }{\Delta x} = 0\\
    y: \frac{-p(x  , y + \Delta y, z) - p(x,y,z) }{\Delta y} = 0\\
\end{gather*}
Si puo riscrivere nelle forme di derivate parziali:
    \begin{multicols}{2}
        \noindent
        z:\\
        x:\\
        y:\\
        \columnbreak
        \begin{gather*}
            \begin{cases}
                    \rho g_z - \frac{\partial p}{\partial z} = 0\\
                    \rho g_x - \frac{\partial p}{\partial x} = 0\\
                    \rho g_y - \frac{\partial p}{\partial y} = 0\\
                \end{cases}
        \end{gather*}
    \end{multicols}
    

    \section{25/09/25}
    Consideriamo il ssitema sovrastante dell'ultima volta:\\
     \begin{gather*}
            \begin{cases}
                    \rho g_z - \frac{\partial p}{\partial z} = 0 \qquad x - \\
                    \rho g_x - \frac{\partial p}{\partial x} = 0 \qquad y - \\
                    \rho g_y - \frac{\partial p}{\partial y} = 0 \qquad z - \\
                \end{cases}
        \end{gather*}
        Ricordiamo la definizione di radiante di p.\\
        \begin{gather*}
            \overrightarrow{\nabla p} = \frac{\partial p}{ \partial x} \widehat{x} + \frac{\partial p}{ \partial y} \widehat{y} + \frac{\partial p}{ \partial z} \widehat{z}\\
        \end{gather*}
\begin{figure}[tbh]
    \centering
    \includesvg[width = 300 pt]{fluid1}
    \caption{ }\label{fig:testsvg}
\end{figure}
\begin{gather*}
    p(x,y) = x^2 + y^2\\
    p(x,y) = x^2 + y^2 = C\\
\end{gather*}

Da qui si ricava l'\underbar{EQUAZIONE FONDAMENTALE IDROSTATICA}\\
\begin{gather*}
    \text{\fbox{$
    -\overrightarrow{\nabla} p + \rho \vec{g} = 0
    $}}
\end{gather*}

In caso di fluido incomprimibile si ha che\\
\begin{gather*}
    \rho= \rho_0
\end{gather*}

\begin{figure}[tbh]
    \centering
    \includesvg[width = 200 pt]{fluid2}
    \caption{ }\label{fig:testsvg}
\end{figure}

\begin{gather*}
    \frac{\partial p}{ \partial x} = 0\\
    \frac{\partial p}{ \partial y} = 0\\
    \frac{\partial p}{ \partial z} = +\rho g \Rightarrow \int_{z_0}^{z} \frac{dp}{dz'} dz' = \int_{z_0}^{z} \rho g dz'\\
    \text{\fbox{$ p-p_0 = \rho y (z-z_0) $}}
\end{gather*}
Quest'ultima è la legge di Stevino

\begin{figure}[tbh]
    \centering
    \includesvg[width = 200 pt]{fluid3}
    \caption{Non ne ho la più pallida idea}\label{fig:testsvg}
\end{figure}

\begin{gather*}
    \rho_0 \approxeq 9/ cm^3 = 1kg/ litro = 10^3 gm/ m^3\\
    g\approxeq 10\\
    \text{\fbox{$P_{1m} \approxeq 10^4 N/ m^2$}}
\end{gather*}

\subsection{L'esempio della bacinella}
Adesso rempiamo una bacinella di liquido con un tubo senza fondo al centro, la pressione atmosferica esercitata al di fuori del tubo dovrà essere bilanciata da unaltra forza che si crea allinterno del tubo.
\begin{figure}[tbh]
    \centering
    \includesvg[width = 200 pt]{fluid4}
    \caption{Bacinella con liquido soggetta alla pressione atmosferica}\label{fig:testsvg}
\end{figure}

\begin{gather*}
    p = p_{\perp} g h\\
    \text{prendiamo il mercurio che reagirà alla pressione atmosferica con un valore di: } \ 760 mm Hg\\
    \text{(hG = mercurio nella tavola periodica)}\\
    760 mm Hg = 13,6 10^3 kg/ m^2 9,81 * 0,76 m =\\
    1,03 10^5 n/ m^2 \approxeq 10^5 Pascal\\
     = \ \text{\fbox{$1 Atm$}}
\end{gather*}
Per ricordare le equivalenze tra unità di misura:
\begin{gather*}
    10^5 N/m^2 = 1Bar \approxeq 1 Atm (1,03 10^5 Pascal) \approxeq 1Kg/cm^2\\\\
\end{gather*}


Se si riprende la \underbar{figura 7} si vede che più è $h$ più è la pressione esercitata \\\\
Prendiamo l'esempio di una diga:
\begin{gather*}
    \Delta p = \rho \ g \widehat{z}\\
    r(B) = \int_{B}^{0} = \rho \ g z \ dz = \frac{\rho \ g z^2}{2}\\
\end{gather*}

\begin{figure}[tbh]
    \centering
    \includesvg[width = 200 pt]{fluid6}
    \caption{Schematizzazione diga}\label{fig:testsvg}
\end{figure}
Dove ogni punto della diga deve sostenere la pressione della parte soprastante, anche per questo si usa questa forma.
E l'asse z rappresenta la forza di $\vec{g}$ che viene applicata su essa.

\subsection{Forza di Archimede}
Prendiamo ora un'altra situazione in cui si hanno 2 bacinelle e si immerge un corpo.
Il fluido reagirà spingendo il corpo verso l'alto vediamo el forze in gioco, come rappresentarle e cosa fanno.
\begin{figure}[tbh]
    \centering
    \includesvg[width = 200 pt]{fluid7}
    \caption{Bacinelle di liquido con un corpo}\label{fig:testsvg}
\end{figure}

\begin{gather*}
    \rho_F g V = R_\Sigma\\
    \overrightarrow{R_\Sigma = -\rho \vec{g} V}\\
    \overrightarrow{R_\Sigma} = \int_V \rho \ dV \vec{g} = -M_f \vec{g} = \rho \ V \vec{g}
\end{gather*}

\section{26/09/25}

\begin{gather*}
    \overrightarrow{\nabla} p = \rho \vec{g}
\end{gather*}

la legge/ relazione barotropica: $p = p(\rho)$\\\\

Mentre la seguente è la legge politropica;
\begin{gather*}
    p = C \rho^\gamma \Rightarrow c=\frac{p_0}{\rho_0^{\gamma}}\\
    \text{ è equialente alla seguente scrittura: }\\
    \frac{p}{p_0} = (\frac{\rho}{\rho_0})^\gamma
\end{gather*}

\begin{figure}[tbh]
    \centering
    \includesvg[width = 200 pt]{fluid8}
    \caption{Bacinelle di liquido con un corpo}\label{fig:testsvg}
\end{figure}


\begin{gather*}
    \frac{\partial p}{ \partial x} = 0\\
    \frac{\partial p}{ \partial y} = 0\\
    \frac{\partial p}{ \partial z} = -\rho g\\\\
    \frac{dp}{dz} = -\rho g = -\rho_0 \frac{p}{p_0} g\\
    \int_{p_0}^{p} \frac{dp'}{p'} = \int_{0}^{z} \frac{\rho_0 y}{p_0} dz'\\
    \ln(\frac{p}{p_0}) = \frac{-\rho_0 \ g}{p_0} z \Rightarrow \text{\fbox{$p(z) = p_0 \ e^{-\frac{\rho_0 \ g}{p_0}z}$}}
\end{gather*}

QUesta formula ci dice la pressione ad una certa altezza z la possiamo sintetizzare come:
\begin{gather*}
    p = p_0 e^{-z/H}\\
    \text{con H uguale a:}\\
    H = \frac{p_0}{ \rho_0 \ g}
\end{gather*}

Descriviamo ora la legge dei gas ideali:
\begin{gather*}
    PV = n R T = n N_a \frac{R}{N_a} T = N  K_B T\\ \text{con $N$ numero di atomi e $N_B$ la costante di Bolzman}\\
    P dv = dN K_B T \frac{kg T }{m} \Rightarrow P = \frac{dm}{dV} \frac{K_B \ T}{\overline{m}} \Rightarrow p = \frac{K_B \ T}{m} \rho \\
\end{gather*}


\begin{figure}[tbh]
    \centering
    \includesvg[width = 200 pt]{fluid9}
    \caption{Bacinelle di liquido con un corpo}\label{fig:testsvg}
\end{figure}

\hfil\\

\begin{gather*}
    \overrightarrow{\nabla} f = \frac{\partial f}{\partial x} \widehat{x} + \frac{\partial f}{\partial y} \widehat{y} + \frac{\partial f}{\partial z} \widehat{z} 
\end{gather*}
Non posso semplicemente passare a coordinate cilindriche perchè avrei $f/$lunghezza e $f/$angolo sommate insieme il che non torna dimensionalmente, quindi...\\

\begin{gather*}
    \overrightarrow{\nabla} f = \frac{\partial f}{\partial r} \widehat{r} + \frac{1}{r} \frac{\partial f}{\partial \theta} \widehat{\theta} + \frac{\partial f}{\partial z} \widehat{z} 
\end{gather*}

Scomponiamo la pressione nelle tre componenti nelle coordinate cilindriche
\begin{gather*}
    \overrightarrow{\nabla} p = \rho \vec{g}\\
    \begin{cases}
        \frac{\partial p}{\partial r} = \rho_0 w^2 r\\
        \cancel{ 1/r \frac{\partial p}{\partial \theta}} = 0 \quad \leftarrow\text{non lo uso}\\
        \frac{\partial p}{\partial f} = -\rho_0 g
    \end{cases}
\end{gather*}

E trovo:
\begin{gather*}
    p(r, z) -p_0 = f_1(r) + f_2(z)\\
    \text{ricordando che $p_0$ ha coordinate $z=0 r=0$}\\
    \begin{cases}
        \frac{df_1}{dr} = \rho_0 \omega^2 r \Rightarrow f_1 = \rho_0 \frac{\omega^2}{2} r^2 +{\not c_1}\\
        \frac{df_2}{dz} = -\rho_0 g \Rightarrow f_2 = -\rho_0 g z + \not c_2
    \end{cases}
\end{gather*}

Da questo otteniamo:
\begin{gather*}
    p_0 = \rho_0 \frac{\omega^2}{2} r^2 - \rho g z + p_0 \Rightarrow \cancel{\rho_0} \frac{\omega^2}{2} r^2 = \cancel{\rho_0} \ g \ z\\
    \text{\fbox{$p(p,z) = \rho_0 \frac{\omega^2}{2} r^2 - \rho_0 g z + \rho_0$}}
\end{gather*}

\begin{figure}[tbh]
    \centering
    \includesvg[width = 200 pt]{fluid10}
    \caption{Bacinelle di liquido con un corpo}\label{fig:testsvg}
\end{figure}


Da qui si trova l'equazione di questa mezza parabola:
\begin{gather*}
    \text{\fbox{$
    z = \frac{\omega^2}{2} \frac{r^2}{g}
    $}}
\end{gather*}

La gravità che si genera in un corpo celeste cerca di comprimerlo e la pressione generata da essa cerca di espanderlo, con questo si crea un equilibrio.
\\La forza esercitata su un unità di massa è:
\begin{gather*}
    \frac{\vec{F}}{m} = -G \frac{M(r)}{r^2} \widehat{r}
\end{gather*}

Si considera un volume di spessore infinitesimo $dV$ lungo tutta la circonferenza

\begin{figure}[tbh]
    \centering
    \includesvg[width = 100 pt]{fluid12}
    \caption{volume infinitesimo per una sfera}\label{fig:testsvg}
\end{figure}


\begin{gather*}
    \frac{\vec{F}}{m} = -G \frac{M(r)}{r^2} \rho dV \widehat{r}\\
    M(r) = \int_{0}^{r} 4\pi r^2 \rho(r) \ dr\\
    \overrightarrow{\nabla} p = \rho \vec{g}(r) = -G \frac{\rho m(r)}{r^2}\widehat{r}\\
\end{gather*}

usando le coordinate sferiche si trasforma
\begin{gather*}
    \vv{\nabla} f = \frac{\partial f}{\partial r} \widehat{r} + 1/r \frac{\partial f}{\partial \theta} \widehat{\theta} + \frac{1}{r \sin(\theta)}\frac{\partial f }{\partial \Phi} \widehat{\Phi}\\
\end{gather*}


\begin{gather*}
    \frac{dp}{dr} = -G \rho \frac{M(R)}{r^2} = -\frac{r\pi}{3} G \rho_0^2 \frac{r^{\cancel{3}}}{\cancel{r^2}}
\end{gather*}
dove $M(r)$ è la massa a distanza $r$ dall'origine della sfera.

\begin{gather*}
    p-p_0 = \int_{p_0}^{p} dp = -\int_{R_\oplus}^{r} \frac{4\pi}{3}G\rho_0^2 r' dr = -\frac{4\pi}{3} \rho_0^2 G/2  (r^2 - R_\oplus^2)\\
    p = p_0 +\frac{4\pi}{3} \rho_0^2 G/2 (R_\oplus^2 -r^2) = p_0 +\frac{4\pi}{3} \frac{R_\oplus^2 p_0^2}{2} G (1- \frac{r^2}{R_\oplus}) = \frac{4\pi}{3}p_0 R_\oplus^3 \frac{G}{R_\oplus} \frac{p_0}{2} (1-\frac{r^2}{R_\oplus^2}) \qquad \text{con } M_\oplus = \frac{4\pi}{3}p_0 R_\oplus^3\\
    M(r) = \int_{0}^{r} 4 \pi r^2 \rho(r) dr
\end{gather*}

proviamo ad applicarlo con questi dati:
\begin{gather*}
    \rho_0 = 5,5 10^3 kg/ m^3\\
    p = 10^{11} Pascal
\end{gather*}
Ora prendiamo questo pianeta e facciamolo ruotare su se stesso:
\begin{gather*}
    \vv{F}_g = -\frac{\rho G M(r)}{r^2} \widehat{r}\\
    \vv{F}_c = \rho \omega^2 h (\sin\theta \ \widehat{r} + \cos\theta \ \widehat{\theta})\\
    = \rho_0 \omega^2 r \sin(\theta)(\sin\theta \ \widehat{r} + \cos\theta \ \widehat{\theta})\\\\
    \begin{cases}
        \frac{\partial p}{\partial r} = -\rho_0 ^2 \frac{4\pi}{3} r G + \rho_0 \omega^2 r \sin^2 \theta\\
        1/r \frac{\partial p}{\partial \theta} = \rho_0 \omega^2 r \sin\theta \cos\theta\\
    \end{cases}
    \\p(r, \theta) -\rho_0 = -\rho_0^2 \frac{4\pi}{3} \frac{r^2}{2} G + \rho_0 \omega^2 \frac{r^2}{2} \sin^2\theta + \cancel{f(\theta)}\\
    \/r\frac{\partial p}{\partial \theta} = \rho_0 \frac{\omega^2 r^{\cancel{2}}}{\cancel{r}} \sin\theta - \cos\theta + \frac{f'(\theta)}{r} = \rho_0 \cancel{\omega^2}\\
    p-p_0 = -p_0  \frac{r^2}{2} (Gp_0 - \omega ^ 2 \sin^2\theta)\\
    p-p_0 = \frac{r^2 \rho_0 g_\oplus}{2R_\oplus}[1-\frac{\omega^2 r_\oplus}{g_\oplus} \sin^2\theta]
\end{gather*}
\newpage
30/09/25
\section{TERMODINAMICA}
Lo studio della termodinamica si concentra sullo studio macroscopica degli oggetti invece che sullo studio microscopico,
si userà quindi un piccolo numero di variabili e non ci interesseremo del comportamento delle molecole.\\
Queste variabili possono essere anche chiamate coordinate termodinamiche.\\
alcuni esempi sono la pressione ($p$) o la densita ($\rho$).
\\Prendiamo la massa ($M$) che sappiamo è legata alla densita ($\rho = \frac{M}{V}$) e solitamente si usa di piu quest'ultima invece della massa, poichè danno le stesse informazioni.
\\Si considerano i sistemi sia isolati sia non.\\
Consideriamo infatti un sistema $S$ e il suo ambiente $A$ e "definiamo" in gergo $S+A= U$ come universo, anche se non comprende tutto l'universo come lo conosciamo ma generalmente si intende tutto ciò che influisce ul mio sistema (es. se un sistema è in un ambiente stanza, "l'universo" è la stanza, che è l'ambiente complessivo che interagisce che sto considerando).
La densità e la pressione non sono in grado di descrivere completamente i fenomeni che sto considerando quindi mi serve introdurre una nuova grandezza $T$ ovvero la temperatura.\\
La temperatura non si puo considerare come una lunghezza o simili nel senso che se due sistemi hanno la stessa temperatura o la stessa lunghezza mi dicono cose fondamentalmente diverse. Nel senso che se ho un campione di lunghezza posso usare quello per determinare la lunghezza di un sistema quando non posso definire in maniera fisica/univoca un campione della temperatura.\\

\subsection{equilibrio termodinamico}
    Quello che vuol dire intrinsecamente è che un sistema $S$ è in equilibrio termodinamico se le sue variabili sono costanti nello spazio e nel tempo.\\
    Se queste condizioni si verificano solo per un punto si parla invece di equilibrio termodinamico locale, in altro modo si può definire che se una parte del mio sistema (un punto) è costante nello spazio (considero sempre quel punto) e non cambia nel tempo allora, quel punto, è in equilibrio termodinamico, se si considera il sistema $S$ si dice che è in equilibrio termodinamico locale.\\
    es.
    \begin{gather*}
        \rho = \rho(\vv{r})\\
        p = p(\vv{r})\\
    \end{gather*}
    \hfil\\
    Possiamo supporre che ci siano delle relazioni tra le mie variabili $p, V, T$ e quindi posso rappresentare questo equilibrio termodinamico cosa come:
    \begin{gather*}
        f(p,V,T) = 0\\
        \text{    questa prende il nome di \textbf{equazione di stato}}
    \end{gather*}
    Con questa relazione posso capire che se ho note due delle 3 variabili che sto considerando conoscendo $f$ posso ricavarmi la rimanente.\\

    Noi non siamo in grado di capire  e non siamo neanche interessati a scrivere cosa succede quando un sistema NON è in equilibrio ma ci interessa capire i cambiamenti di equilibrio.\\
    \begin{definition}
        Estensività:\\
        prendiamo in sistema $S$ e lo dividiamo in due sistemi $S_1$ e $S_2$ (che per semplicità, ma non mi limita, avranno la stessa grandezza)\\
        Per hp. ho la grandezza $X \in S$ d conseguenza avrò:
        \begin{gather*}
            x_1 \in S_1\\
            x_2 \in S_2\\
            x_1 = x_2 = \frac{X}{2}
        \end{gather*}
        Questa è di fatto la def. di additività ma in termodinamica l'estensività e l'additività sono la stessa cosa, anche se in generale non è vero.\\
    \end{definition}

\begin{definition}
    Un esempio nel quale questo non è vero è se la mia $X$ è la pressione, quando divido il mio sistema $S$ in due sottosistemi la pressione è la stessa quindi può succedere che:
    \begin{gather*}
        x_1 = x_2 = X
    \end{gather*}
    questa caratteristica è definita come intensiva\\
\end{definition}
    Se una caratteristica (variabile) non è ne estensiva ne intensiva non è una grandezza termodinamica (fondamentlmente non ci interessa).\\\\

    Consideriamo ora un problema:\\
    Se io ho solo densità pressione e temperatura, quindi tutte caratteristiche intensive,io non conosco le dimensioni del sistema quindi mi serve conoscere in tre variabili che scelgo almeno una estensiva.\\
    \begin{definition}
        Parete \underline{adiabatica} (termodinamicamente isolante)\\
        Una parete è adiabatica se due sistemi termodinamici possono coesistere all'equilibro termodinamico \\
        \begin{figure}[tbh]
    \centering
    \includesvg[width = 150 pt]{term1}
    \caption{}\label{fig:testsvg}
    \end{figure}
    Quindi qualunque sia il volume o pressione dei miei sistemi loro non "capiscono" di essere accanto se sono separati da una parete adiabatica, questa è ovviamente una condizione molto forte.
    \end{definition}
    Generalmente questo è un concetto ideale ma molto comodo per diversi modi.

    \begin{definition}
        Parete diatermica $=$ NON adiabatica.\\


    \begin{figure}[tbh]
    \centering
    \includesvg[width = 150 pt]{term2}
    \caption{}\label{fig:testsvg}
    \end{figure}

    Quando $S_1$ e $S_2$ si stabilizzano sono in equilibrio termico fra loro.
    \end{definition}

    Due sistemi in equilibrio termico godono di una proprietà fondamentale non ovvia scoperta sperimentalmente:
    \begin{proposition}
        Se ho 3 sistemi $A,B,C$ se ho equ. termico tra $A$ e $B$ e anche tra $A$ e $C$ e questo ci GARANTISCE che ho eq. termico tra $B$ e $C$.\\
        \begin{gather*}
            (A \sim B) \land (B \sim C) \Rightarrow (A \sim B)
        \end{gather*}
        Quindi l'eq. termico gode della prop. transitiva.\\\\
        Questo prende il nome di \textbf{principio zero della termodinamica}.\\
        In generale un principio è qualcosa che non si verifica ma non si riesce a confutare.\\
    \begin{figure}[tbh]
    \centering
    \includesvg[width = 150 pt]{term3}
    \caption{schematizzazione di eq termico con pareti adiabatiche e diotermiche}\label{fig:testsvg}
    \end{figure}


    \end{proposition}

    Introduciamo il concetto di equilibrio termico con questa intuizione:
    \begin{gather*}
        A \sim B \Rightarrow T_A = T_B
    \end{gather*}
    Mi chiedo ora se la temp. è estensiva o intensiva, e lo psso schematizzare in questo modo:
    \begin{figure}[tbh]
    \centering
    \includesvg[width = 150 pt]{term4}
    \caption{schematizzazione di eq termico con pareti adiabatiche e diotermiche}\label{fig:testsvg}
    \end{figure}
    La temperatura è difficile da misurare nel senso che ci serve una quantità "universale" non ci basta dire che qualcosa è caldo/freddo relativaemtne a qualcos'altro\\
    \hfil\\
    Verifichiamo la veridicità della eq. $A \sim B \Rightarrow T_A = T_B$.\\
    Prendiamo due grandezze $X,Y$ che non sono la temperature, e tre sistemi $A,B,C$ in equilibrio termico.\\
    In generale il fatto che due sistemi siano in eq. termico implica che esiste una relazione del tipo:
    \begin{gather*}
        f_{A,C}(X_A,Y_A,X_C,Y_C) = 0\\
        f_{B,C}(X_B,Y_B,X_C,Y_C) = 0\\
    \end{gather*}
    Non sto assumendo affatto che queste relazioni siano la stessa.\\
    
    Da queste due equazioni in quattro variabili uguali a zero scrivo $Y_C$ in funzione delle altre variabili e ricavo:
    \begin{gather*}
        Y_C = g_{A,C}(X_A, Y_A, X_C)\\
        Y_C = g_{B,C}(X_B, Y_B, X_C)\\
    \end{gather*}
    Da queste:
    \begin{gather*}
        g_{A,C}(X_A, Y_A, X_C) = g_{B,C}(X_B, Y_B, X_C)\\
        f_{A,B}(X_A,Y_A,X_B,Y_B) = 0
    \end{gather*}
    Siccome $f_{A,B}$ non dipende da $X_C$ e so che queste due funzioni sono equivalenti deve esistere una funzione del tipo:
    \begin{gather*}
        h_A (X_A,Y_A) = h_B(X_B,Y_B)\\
    \end{gather*}
    Quindi la funzione $h$ è la funzione $g$ con i fattori $X_C$ espressi in funzione di $X_A$, $X_B$,$Y_A$ e $Y_B$
    Questo non mi limita a dover avere una funzione con a che relaziona b ma anche:
    \begin{gather*}
        h_A (X_A,Y_A) = h_C(X_C,Y_C)\\
    \end{gather*}
    Da qui si evince che:
    \begin{gather*}
        h_A (X_A,Y_A) = h_B(X_B,Y_B) = h_C(X_C,Y_C) = \tau\\
    \end{gather*}
    ricordiamo che quest tre funzioni non sono le stesse ma devono essere uguali posso chiamare il valore di queste funzioni $\tau$ che chiamo \textbf{temperatura empirica}.\\
    Purtroppo non sappiamo risolvere queste funzioni però sappiamo l'esistenza di queste relazioni.\\
    Riflettiamo però su queste considerazioni, io so che:
    \begin{gather*}
        h(x,y) = \tau \ \text{isoterma}
    \end{gather*}
    Qundi tutti gli stati possibili devono rispettare questa relazioni, se considero $X,Y$ come coordinate una certa temperatura mi identifica una curva, chiamata \textbf{curva isoterma}.\\
    
    \subsection{misurare la temperatura in maniera operativa}
        Come si misura quindi la temperatura nella pratica?\\
        Serve definire, sfruttando il principio 0, uno strumento in grado di misurare la temperatura, un termomentro:\\
        Come per le forze scopriamo che ogni oggetto si deforma se ci applichiamo una forza, abbiamo quindi individuato una proprietà dinamometrica che si riconduce alle grandezze fondamnetali.\\
        Per la temperatura ci interessa fare la stessa cosa, ovvero cercare una proprietà termometrica per relazionare la temperatura alle grandezze fondamentali, il problema è che non conosco appunto questa relazione.\\
        Nel pratico:
        \begin{gather*}
            T_a = T_b \Rightarrow z_a = z_b\\
            T_a \neq T_b \Rightarrow z_a \neq z_b\\
        \end{gather*}
        In questo esempio z sarà la quantita che mi fa misurare T.\\

    \begin{figure}[tbh]
    \centering
    \includesvg[width = 150 pt]{term5}
    \caption{Schematizzazione termometro}\label{fig:testsvg}
    \end{figure}
    dove $T_A^A$ è il mio termometro.\\
    però ho il problema che non conosco la temp. iniziale di $S$, quindi anche se io ho usato un sistema.

    Per procedere quindi procedo a farmi $N$ sistemi termometri di varie $N$ masse decrescenti che misureranno tutti la temperatura del sistema $S$ che io sono in grado di riporare sempre allo stesso stato.\\
    Con questo ho una successione monotona che mi convergerà al mio ultimo:
    \begin{gather*}
        T= \lim_{M_A \to 0} T(M_A)
    \end{gather*}
    In generale se uso un termometro relativamente piccolo rispetto al sistema che sto misurando questo influenzerà il mio sistema originario sufficentemente poco da essere trascurabile.\\
    In generale in questo modo non ho una definizione univoca ovvero $\tau \to numero$, l'unica cosa che so essere univoca è quando due sistemi hanno la stessa temperatura (sono in eq. termico).

    \newpage
    \section{2/09/25}
    \subsection{Come sono fatti i termometri}
    Il metodo principale utilizzato è la proprietà dei liquidi di dilatarsi con la temperatura, il più usato era il mercurio poichè aveva un coefficiente di dilatazione termica molto grande, nonostante fosse velenoso/tossico.\\
    Non si usava ad es. l'acqua perchè ha un comportamento di fatto molto strano con i cambiamenti di temperatura.\\
    
    \begin{center}
    \begin{tikzpicture}
        \draw[->](0, 0) -- (4, 0) node[at end, below] {$x$};
        \draw[->](0, 0) -- (0, 4) node[at end, left] {$y$};
        \draw(1, 1) .. controls (1.4, 0.9) and (1.9, 2.2) .. (2, 3) node[at end, above] {$h(x, y) = T_1$};
        \draw(0, 2) -- (4, 2) node[at end, right] {$Y = Y_0$};
        \draw[dashed](1.78, 0) -- (1.78, 2) node[at start, below] {$x_1$};
        \draw(2, 1) .. controls (2.4, 0.9) and (2.9, 2.2) .. (3, 3) node[at end, right] {$h(x, y) = T_2$};
        \draw[dashed](2.78, 0) -- (2.78, 2) node[at start, below] {$x_2$};
    \end{tikzpicture}    
    \end{center}        

    In generale:

    \begin{gather*}
        T(x) = ax +b
    \end{gather*}
    Adesso mi interessa un modo per ricavare queste costanti $a$ e $b$.\\
    Per farlo sfrittuami quelli che sono chiamati come punti fissi termometrici, ovvero punti in cui la materia cambia di stato e questo è a punti assegnati della variabile termodinamica che io conosco.\\
    In particolare ci siamo interessati ai punti fissi \underline{normali}, che si verificano esattamente\\
    \begin{gather*}
         p = p_{standard} \quad (p=1 atm) = 1.01425 \ 10^5 Pa
    \end{gather*}
    In questo caso si misurano i punti fissi dellacqua ($H_2 O$) poichè comuni, facilmente accessibili poichè vicini tra loro e in generale è una sostanza comoda per lavorare. L'acqua si usa quindi come materiale per TARARE il termometro ma NON è il materiale che si usa per misurare il termometro (quello è il mercurio).\\
    Di conseguenza abbiamo due punti fissati:
    \begin{gather*}
        T_1 = ax_1 + b\\
        T_2 = ax_2 + b\\
    \end{gather*}
    Posso quindi ricavare $a$ e $b$ in questo modo:
    \begin{gather*}
        a= \frac{T_2-T_1}{x_2-x_1}\\
        b= T_1 \frac{T_2-T_1}{x_2-x_1} x_1
    \end{gather*}
    Non siamo quindi di fissare una $y$ e cerchiamo una relazione tra questa $y$ fissata e 
    \begin{gather*}
        T(x) = \frac{T_2-T_1}{x_2-x_1}(x-x_1) + T_1\\
        (y= cost.)
    \end{gather*}
    Di conseguenza ora posso definire la mia scala termometrica, la scala Celsius:
    \begin{gather*}
        T_1 = 0^{\circ} \\
        T_2 = 100^{\circ}\\
        t(x) = \frac{x-x_1}{x_2-x_1} 100^{\circ}
    \end{gather*}
    Quel coglione di Fahrenheit usava invece in una miscela stupida e pensando di usare come punti fissi --$>$ hmm si quando l'umano sta piu o meno bene e quando no
    \begin{gather*}
        t_f[\text{°}F] = \frac{9}{5}t[\text{°}C]+32
    \end{gather*}
    In generale purtroppo due termometri funzionanti a liquidi diversi (secondo principi diversi) anche su stesse temperature leggeranno valori diversi, questo perchè abbiamo presupposto che la relazione fosse lineare ma nessuno ci garantisce che lo sia.\\
    Quindi si prende un termometro campione e si usa per tarare i termometri su quello, in generale è la stessa cosa che si faceva con il metro, si prendeva un campione e si taravano gli altri metri su quello.\\
    \subsection{La storia dello svincolamento dalla taratura}
    Ovviamente questo metodo alla lunga non  sostenbile e man mano che si cerca nelle conversioni una relazione, si cerca anche un modo per svincolarsi da un termometro campione.\\
    Siamo partiti dalla misurazione di gas, Boyle misurava la pressione $p$ e il volume $V$ di vari gas, e trovava che se variava linearmente l'una rispetto a l'altra il loro prodotto restava costante:
    \begin{gather*}
        pV = cost.\\
        (t=cost)
    \end{gather*}
    Successivamente grazie a Gay Lussec scopri 
    \begin{gather*}
        p = p_0 (1+\beta t) \quad (V=cost.)\\
        V= V_0 (1+ \gamma t) \quad (p=cost.)
    \end{gather*}
    Da questo si pensa che il volume e la pressione siano dei buoni parametri per decidere temperatura in maniera impirica.\\
    In più $\alpha$ e $\beta$ erano piuttosto vicini tra loro.
    \begin{gather*}
        \beta \simeq \gamma = \frac{1}{273}\text{°}C^{-1}
    \end{gather*}
    Nel frattempo Avogadro scoprì che se ho la pressione e temperatura costante ho che il volume è proporzionale al numero di atomi in un sistema.
    \begin{gather*}
        p=cost. \quad t=cost. \Rightarrow V \propto \mathcal{N} 
    \end{gather*}
    POsso ora pensare come diminuivo la massa del termometro per avere un limite, posso fare la stessa cosa e fo tendere la pressione di un gas a 0 ovvero vedo come si comporta via via che riduco la pressione del suo sistema:
    \begin{gather*}
        p \to 0 \quad (V=cost.) \Rightarrow \text{Gas perfetto}\\
        \beta = \gamma = \frac{1}{273,15 \text{°}C}\\
        p(t) = p_0(1+\frac{1}{273.15}t)
    \end{gather*}
    Posso quindi definire la temperatura sulla base della pressione del gas perfetto e non su un evento sperimentale del comportamento del mercurio:
    \begin{gather*}
        t=\frac{273.15}{p_0}p-273.15[\text{°}C]
    \end{gather*}
    La cosa non ovvia è che ad ogni passaggio del limite le leggi di Gay Lussec rimangono valide.\\
    Di fatto è una procedura complicata ma sempre univoca e ben definita.\\
    In più si pensa che può essere utile definire una scala che invece partire dalla scala celsius parta dalla temperatura del gas perfetto, la scala Kelvin.
    \begin{gather*}
        T=273.15 \underset{[\text{°}C]}{t} [K]\\
    \end{gather*}
    In questo modo la temperatura assoluta diventa:
    \begin{gather*}
        T = \frac{p}{p_0}T_0 (V=cost.)
    \end{gather*}
    Oltre ai punti fissi esistono i punti tripli ovvero combinazioni di pressioni, temperature e volume fanno coesistere i tre stati della materia, solido liquido e gas.\\
    Ad esempio quello dell'acqua è
    \begin{gather*}
        T^{H_2O}_{\text{triplo}} = 0.01\text{°}C = 273.16 [K]\\
        p_{\text{triplo}}^{H_2O} = 611 Pa \approx 6 \ 10^{-3} atm
    \end{gather*}
    Purtroppo il gas perfetto non esiste e quindi  necessario fare tante misurazioni di termometri a gas per avere una buona stima.
    \begin{gather*}
        T^{(1)} = \frac{p^{(1)}}{p_{\text{triplo}}}^{(1)} T_{\text{triplo}} \qquad \text{prima iterazione}\\
        T^{(2)} = \frac{p^{(2)}}{p_{\text{triplo}}}^{(2)} T_{\text{triplo}} \qquad \text{seconda iterazione}\\
    \end{gather*}
    A ogni iterazione si dimezza il gas che si misura poichè il punto triplo non è costante perchè dipende dalla temperatura a cui misuro, man mano che itero la variazione del punto triplo sarà sempre minore e quindi posso calcolare il limite dei miei valori.
    
    \newpage
    \section{7/10/25}
    \subsection{Risassunto:}
    \begin{gather*}
        T [K]\\
        V = \frac{V_0}{T_0}T \quad (p=cost.)\\
        p = \frac{p_0}{T_0}T \quad (V=cost.)\\
        pV = cost \quad (T=cost.)\\
        V \propto \mathcal{N} \quad (p,T cost.)
    \end{gather*}
    Quando queste considerazioni sono vere si può definire il limite del gas \underline{perfetto} (gas.ideale)
    \begin{gather*}
        T=\underset{(p\to 0)}{\lim_{p_3 \to 0}} \frac{p}{p_3} T_3 \Longrightarrow T \geq 0 
    \end{gather*}
    Un'importante osservazione è che si può effettivamente dire che $600 K$ sono il doppio di $300 K$ ma non si può dire con esattezza ad es. che la temperatura di 40°C è il doppio di unaltra a 20°C\\
    \subsection{Equazione di stato del gas perfetto}
    I sistemi fluidi sono molto semplici da studiare da studiare dal punto di vista termodinamico.\\
    In ogni sistema termodinamico vale una equazione di stato, ovvero esistono tre grandezze legate fra loro che descrive il materiale dal punto di vista termodinamico, come quelle che abbiamo introdotto: $f(p,V,T) = 0$, ma non si sa la forma esatta di questa equazione, in alcuni casi si può ricavare l'eq di stato con metodi sperimentali.\\
    Se prendiam l'esempio di un gas perfetto siamo in grado infatti di ricavare l'equazione di stato di questo.\\
    Prendiamo ora l'esempio di un gas perfetto:\\
    Ricordiamo l'equazione di stato che sto cercando:
    \begin{gather*}
        f(p,V,T) = 0
    \end{gather*}
    \begin{gather*}
        \text{(prendo una mole di gas)} \ \mathcal{N} = 1
    \end{gather*}
    Prendo ora dei valori iniziali per ogni grandezza:
    \begin{gather*}
        T = T_0 = 273.15 K\\
        p = p_0 = 1 atm\\
        V = V_0 = 2.2414 \times 10^{-2} m^3 \approx 224 l \ \text{(litri)}
    \end{gather*}
    
    \begin{figure}[tbh]
    \centering
    \includesvg[width = 150 pt]{term7}
    \caption{}\label{fig:testsvg}
    \end{figure}
    Sappiamo che la temperatura ad un certo volume ha una certa pressione, so inoltre che la temperatura ha la forma di una curva isoterma e quindi costruisco sperimentalmente\\
    
    \begin{gather*}
        V' = V_0 \frac{T}{T_0}
    \end{gather*}
    Per andare da $( p_0,V')$ a $(p,V)$ cambiando la pressione e facendo attenzione che la temperatura non cambi. e quindi ho:
    \begin{gather*}
        pV = p_0
    \end{gather*}
    Da quì si sostituisce $V'$
    \begin{gather*}
        pV = \frac{p_0 V_0}{T_0}T
    \end{gather*}
    Ora definiamo la costante di gas:
    \begin{gather*}
        R = \frac{p_0V_0}{T_0}\\
        R = .3145 J/K\\
        pV = RT
    \end{gather*}
    Siccome il volume è $\propto$ al numero di moli $\mathcal{N}$, $pV \propto RT$ e per $\mathcal{N} = 1 \Rightarrow pV = RT$ sappiamo di conseguenza che:
    \begin{gather*}
        \text{\fbox{$pV = \mathcal{N}RT$}}
    \end{gather*}
    Che è proprio l'equazione di stato del gas perfetto.\\
    Si ricorda che $\propto$ implica relazione lineare.\\
    Fare attenzione che nonostante ci sia il numero di moli siamo in un sistema macroscopico, poichè il numero di avogadro non è strettamente necessario per definire la mole.\\
    \subsection{Sviluppo viliale dell'eqazione di stato}
    \begin{gather*}
        pV = \mathcal{N} RT (1+\frac{B_1}{V}+\frac{B_2}{V^2}+\frac{B_3}{V^3} + \ \dots \ )
    \end{gather*}
    I vari $B_n(T)$ sono proprio funzioni della temperatura.\\
    Concettualmente non è complicato ricavare questi coefficienti ma dal punto di vista pratico sperimentale bisogna essere molto precisi nel misurarli (dal punto di vista microscopico le cose sono molto complicate).\\
    Questi termini deformano leggermente la mia funzione originale che era un'iperbole equilatera. Questa modifica a l'eq. di stato originaria si chiama sviluppo del viliale e i termini al suo interno sono coefficienti del viliale.\\
    \subsection{Temperatura critica}
    Tempreratura ambiente: $300K$\\
    Temperatura critica ($T_c$): $126 K$\\
    Un gas ha ognuno la propria temperatura critica (quindi caratteristica intrinseca), che se viene superata il gas può essere compresso a piacere ma non diverrà mai liquido. \\

    \subsection{Cambi di stato a pressioni fissate}
    \begin{gather*}
        V = V(p,T)\\
        V = V_{\overline{p}}(T) = V(p,T) |_{p = \overline{p}}
    \end{gather*}

    Se io tengo una pressione fissata quando vario la temperatura avrò un valore in cui l'equilibrio termico si avrà un volume più grande (del triplo/quadruplo), fare attenzione a non pensare questa transizione come qualcosa di istantaneo, non ci interessa per ora di descrivere questi fenomeni nel tempo.\\
    Per ogni pressione fissata esiste solo un valore assegnato alla temperatura $T_{\overline{p}}$
    \begin{figure}[tbh]
    \centering
    \includesvg[width = 150 pt]{term6}
    \caption{}\label{fig:testsvg}
    \end{figure}

    \subsection{equazione di Van Der Wals}
    Per semplicità prendiamo $\mathcal{N} = 1$
    \begin{gather*}
        (p + \frac{a}{V^2})(V-b) = R T
    \end{gather*} 
    Questa formula riesce a decidere tutta la parte della temperatura superiore e inferiore da quella critica, e differisce dall'equazione di base solo di due COSTANTI e riesce a descrivere molto bene l'andamento della pressione in funzione del volume. Non predice la parte orizzontale anche se venne corretta da MAXWELL e riesce a descrivere anche la parte orizzontale del grafico in maniera qualitativa.\\
    Questa formula se pur molto precisa e molto comoda per ddescrivere landamento è comunque un'approssimazione e la trattiamo come tale.\\
    La correzzione di Maxwell è la seguente:
    \begin{gather*}
        p = \frac{RT}{(V-b)}
    \end{gather*}
    \begin{gather*}
        \frac{\partial p}{\partial V} |_{T=T_c} = 0
        \frac{\partial^2 p}{\partial V^2} |_{T=T_c} = 0
        \frac{\partial p}{\partial V} = -\frac{RT}{(V-b)^2}+\frac{2a}{V^3} \underset{\text{calcolato in }T = T_c}{=} 0\\
        \frac{\partial^2 p}{\partial V^2} = \frac{2RT}{(V-b)^3}-\frac{6a}{V^4} \underset{\text{calcolato in }T = T_c}{=} 0
    \end{gather*}
    \begin{gather*}
        \frac{RT_c}{(V_c-b)^2} = \frac{2a}{V_c^3}\\
        \frac{2RT_c}{(V_c-b)^3} = \frac{2a}{V_c^4}
    \end{gather*}
    \begin{gather*}
        RT_c = \frac{8a}{27b}\\
        V_c = 3b\\
        p_c = \frac{a}{27b^2}
    \end{gather*}

    \newpage
    \section{9/10/25}
    \subsection{riassunto equazione vander vals}
    La equazione si può riscrivere anche ne seguente modo:
    \begin{gather*}
        V^3p-V^2(RT+bp) + aV - db = 0
    \end{gather*}

        Maxwell pensò che per sistemare l'equazione di wander vals si potesse iniziare dal vedere lo scarto di quanto la sua equazione sbaglia in prossimità della temperatura critica:

    \begin{figure}[tbh]
    \centering
    \includesvg[width = 150 pt]{term8}
    \caption{}\label{fig:testsvg}
    \end{figure}
    Dove $A_1$ e $A_2$ sono le arre che scartano dall'equazione di vander vals dai valori reali aspettati.\\\\

    Vander Vals ha ricavato questa equazione che funziona molto bene per i valori maggiori o inferiori alla temperatura critica, da un punto di vista qualitativa è ottima, ma da un punto di vista quantitativo?\\
    \begin{gather*}
        1=\frac{8}{3} \frac{P_c V_c}{R T_C}
    \end{gather*}
    Se si prende una tabella con i valori aspettati calcolati dall'equazione e i valori ottenuti vediamo uno scarto comunque in base a come uno sta considerando una certa precisione anche considerevoli:\\


    \hfil\\\\\\\\
    Sappiamo che i valori critici sono caratteristiche intrinseche diverse per ogno sostanza, prendiamo dunque i tre parametri in base alle variabili che sto considerando:
    \begin{gather*}
        \overset{\sim}{T} = \frac{T}{T_c}\\
        \overset{\sim}{V} = \frac{V}{V_c}\\
        \overset{\sim}{p} = \frac{p}{p_c}\\
    \end{gather*}
    Se si mettono nell'eq. di vander vals si ottiene:
    \begin{gather*}
        (\overset{\sim}{p}+\frac{3}{\overset{\sim}{V^2}})(3\overset{\sim}{V}-1)
    \end{gather*}
    Noto che sono scomparsi i parametri $a$ e $b$, Questa caratteristica è detta universalità.\\
    Questa predizione non è vera con buona approssimazione in a tutti i valori ma man mano che mi avvicino ai valori critici diventa un'approssimazione sempre più vera e diventa esatta esattamente ai valori critici.QUesta approssimazione in generale non è del tutto vera per tutti i sistemi, di fatto cnee siamo resi conto quando si iniziò ad avere la possibilit di fare esperimenti più precisi.\\

    \newpage
    \subsection{stati di equilibrio e transizioni di fase}

    \begin{figure}[tbh]
    \centering
    \includesvg[width = 150 pt]{term9}
    \caption{}\label{fig:testsvg}
    \end{figure}
    \hfil\\
    Da questa figura si vede che l'equilibrio tra i tre stati della materia è solo $P_3$ metrne se cado con una combinazione di pressione e temperatura di stare su una parte della curva avrò coesistenza tra le varie fasi e sono dette curve di coesistenza.\\
    Fintantò che io sto all'interno di un area delimitata dalle curve di coesistenza la dipendenza tra i parametri che sto considerando ($p, V, T$) cambiano in maniera regolare, metre quando passo una curva noto immediatamente ch el'andamento diventa non regolare.\\

    Se ad esempio come nella linea rossa orizzontale tengo la pressione costante e vario la temperatura avrò uno scatto proprio come descritto nella figura 20, situazione simile per la linea rossa verticale, e con questa combinazione di percorsi possiamo identificare un punto critico.\\
    Se ingrandiamo possiamo però prendere un'atro percoorso (indicato dalla freccia bianca) che fa diventare il gas gradualmente un iquido, girando intorno al punto critico.\\
    In generale il volume è il terzo asse, quindi per il punto triplo le caratteristiche importanti sono la pressione e la temperatura per qualsiasi volume io consideri.\\
    Fare attenzione che tra liquido e solido il punto critico non c'è quindi quandi si passa tra solido e liquido si comporta in maniera singolare.\\
    Nel grafico è stata specificata anche una curva viola che specifica tutti i tipi di liquidi (tipo l'acqua) che si espandono quando passano allo stato solido.\\\\
    Il grafico seguente rappresenta gli stati di equilibrio considerando il volume con una rappresentazione in tre dimensioni:
    
    \subsection{Trasformazioni generali}
    Quello che per ora sappiamo fare è descrivere ciò che accade negli stati di equilibrio e si è in grado di ricorstruire se una transizione inizia e termina in uno stato di equilibrio, tutto il resto in generale (quindi apparte casi specifici che sappiamo come funziona) NON siamo in grado di spiegare ciò che sta succedendo con precisione.\\
    Cerchiamo ora di capire come descrivere una transizione quando parto da uno stato di equilibrio. Se parto da uno stato di equilibrio so cosa succede ma se vado molto lentamente so che se mi allontano poco, so ancora  cosa accade perchè sono vicino allo stato di equilibrio, continuo lentamente e capisco l'andamento della curva a quel punto descrivo l'andamento della transizione, questa trasformazione è detta quasi-statica, che è un'ottima approssimazione. Esistono anche le trasformazioni statiche che sono fondamentalmente la stessa cosa ma occorre un tempo infinito per essere completate e quindi si usano solo come modello ideale.\\

    Esistono anche trasformazioni cicliche in cui parto da un certo stato di equilibrio, accade una trasformazione, (se quasi-statica so cosa accade altrimenti no) e poi ritorno allo stesso punto di equilibrio.\\
    \begin{figure}[tbh]
    \centering
    \includesvg[width = 150 pt]{term10}
    \caption{}\label{fig:testsvg}
    \end{figure}
    Se ho una trasformazione a presisone costante è detta isobara, mentre se ho volume costante è detta isocora:
        \begin{figure}[tbh]
    \centering
    \includesvg[width = 150 pt]{term11}
    \caption{}\label{fig:testsvg}
    \end{figure}
    In generale se io so che le posizioni iniziali e finali sono a stati di equilibrio e nel caso di un gas perfetto siccome sappiamo descrivere la curva dell'andamento data la posizione iniziale sappiamo calcolare la posizione finale:\\
    
    \begin{figure}[tbh]
    \centering
    \includesvg[width = 150 pt]{term12}
    \caption{}\label{fig:testsvg}
    \end{figure}
    \newpage
    \subsection{Trasformazioni adiabatiche}
    Sono trasformazioni che sono contenute all'interno di pareti adiabatiche.\\
    \begin{figure}[tbh]
    \centering
    \includesvg[width = 150 pt]{term13}
    \caption{}\label{fig:testsvg}
    \end{figure}
    In generale:
    \begin{gather*}
        V_f > V_i
    \end{gather*}
    QUesta trasformazione NON è quasi-statica, in più osservo che $T_f < T_i$ sempre, ma di poco.\\
    Ripeto quindi l'esperimento con poco meno volume e scopro che la temperatura varia di un po' meno, quindi se reitero l'esperimento per arrivare all'hp di gas perfetto scopro che $T_i \to T_f$.
    
    \newpage
    \section{10/10/25}
    Espansione libera adiabatica di un Gas
    \begin{gather*}
        \Delta T = T_f -T_i\\
        \Delta T < 0\\
        \Delta t \underset{\text{gas perfetto}}{\to} 0
    \end{gather*}
    Per far si che questa espansione sia quasi statica devo rallentare questa transizione, per farlo introduco una forza per rallentare l'espansione che sarà ad ogni passo poco meno della forza che il gas applica per espandersi.

        \begin{figure}[tbh]
    \centering
    \includesvg[width = 150 pt]{term14}
    \caption{}\label{fig:testsvg}
    \end{figure}

    Questo oggetto è proprio un pistone adiabatico, poichè tutte le sue pareti sono adiabatiche.\\
    Se io ho che il gas si è espanso osservo che la temperatura è diminuita:
    \begin{gather*}
        \Delta V > 0 \to \Delta T < 0
    \end{gather*}
    Mentre se ho una compressione la mia temperatura aumenta:
    \begin{gather*}
        \Delta V < 0 \to \Delta T > 0
    \end{gather*}
    in generale:
    \begin{gather*}
        \Delta V \neq 0 \Rightarrow \Delta T \neq 0
    \end{gather*}
    Questo accade sempre, ANCHE nel caso di un gas perfetto.\\
    In generale ci sono anche trasformazioni non adiabatiche, anzi sono quelle reali.\\
    
    \subsection{termostato}
    Per introdurre queste trasformazioni ci serve introdurre un nuovo strumento: \underline{il termostato}, che viene semplicemente definito in questo modo:
    \begin{gather*}
        T = cost.
    \end{gather*}
    Ovvero è un sistema la cui temperatura NON cambia durante il cambio di temperatura al sistema con cui è a contatto.\\\\
    Ovviamente un oggetto perfetto del genere non esiste ma possiamo immaginarlo, se si vuole è l'opposto di un termometro, un termometro è infinitesimo mentre un termostato è infinito, un termostato è anche un sistema che fa si che qualsiasi sistema che va a contatto funzionerà come un termometro.\\
    In generale un termostato reale è definito come:
    \begin{gather*}
        T \simeq cost.\\
        \underset{S}{m} << \underset{T}{M} 
    \end{gather*} 
    Dove appunto $m$ è la massa del sistema $S$ e $M$ è la massa del sistema $T$.
    \subsection{pistoniii}
    Riconsideriamo ora l'esempio di prima con una modifica sostituisco una parete adiabatica con una diatermica a contatto col \underline{termostato}:
        \begin{figure}[tbh]
    \centering
    \includesvg[width = 150 pt]{term15}
    \caption{}\label{fig:testsvg}
    \end{figure}
    
    Ora siccome la mia trasformazione è quasi statica e mi muovo lentamente il gas ha tutto il tempo di mettersi in equilibrio termico col termostato e quindi la mia $\Delta T = 0$ poichè per ogni istante la temperatura del mio sistema rimane sempre quella del termostato.\\\\
    Consideriamo ora un altro esempio:

        \begin{figure}[tbh]
    \centering
    \includesvg[width = 150 pt]{term16}
    \caption{}\label{fig:testsvg}
    \end{figure}
        
    Queste trasformazioni sono dette isoterme.
    \newpage
    \subsection{lavoro di un fluido con un pistone}

        \begin{figure}[tbh]
    \centering
    \includesvg[width = 150 pt]{fluid17} %ops sarebbe term ma vabbè
    \caption{}\label{fig:testsvg}
    \end{figure}
    Sto considerando che il gas si espanda di un certo valore $dl$:
    \begin{gather*}
        \vv{dl} = dl \widehat{i}
    \end{gather*}
    E il volume si sta espandendo:
    \begin{gather*}
        V \to V+dV
    \end{gather*}
    \begin{gather*}
        SL = \vv{F} \vv{dl} = pA\widehat{i} dl \widehat{i} = p A dl = p dV\\
        \vv{F} = p A \widehat{i}
    \end{gather*}
E in generale posso dire che quando ho una trasforamzione infinitesima che mi varia il volume di un fluido le sue pressioni mi generano un lavoro:
\begin{gather*}
    \text{\fbox{$\delta L = p dV$}}
\end{gather*}
Il fatto che la trasformazione sia infinitesima mi garantisce che la mia $p$ non varia.\\
E si vede che il lavoro è positivo se il fluido si espande ed è negativo se si contrae.\\

\underline{il mondo non è fatto di pistoni}. Infatti vale per qualunque trasformazione infinitesima di un fluido (il fatto che sia infinitesima mi implica che sia quasi-statica).\\

        
\subsection{generalizzazione del concetto di lavoro di un fluido}

        \begin{figure}[tbh]
    \centering
    \includesvg[width = 150 pt]{term18} %ops sarebbe term ma vabbè
    \caption{}\label{fig:testsvg}
    \end{figure}

Se prendiamo un volume generico $V$ che si espande di un $dV$ e possiamo considerare ogni punto del volume si espande come un pistone di area infinitesima e che si espande di un $dl$ infinitesimo. Di conseguenza si può scrivere che il lavoro di un punto infinitesimo del fluido è:
\begin{gather*}
    \delta L_i = p dA_i + dl_i
\end{gather*}
Considero ora la somma di tutti questi lavori infinitesimi:
\begin{gather*}
    \delta L = \sum_{i} p dA_i dl_i = p \sum_{i} dA_i dl_i = p \sum_{i} dV_i = p dV
\end{gather*}
ho trovato che:
\begin{gather*}
    \text{\fbox{$\delta L = p dV$}}
\end{gather*}

\subsection{trasformazione di volume di area non infinitesima}
In generale NON so calcolare se un fluido si trasforma in maniera NON quasi statica.\\
Poichè sto considerando variazioni di volume infinitesimo\\
Se sono in trasforamzioni quasi statiche e SOLO in questo caso posso scrivere:
\begin{gather*}
    V_i \to V_f\\
    L = \int_{V_i}^{V_f} p dV \qquad p(V)
\end{gather*}
In cui appunto $V_i - V_f$ è un numero finito.\\
Quindi sono in un caso in cui so la dipendenza della pressione dal volume ($p(V)$).\\
In generale se so che è quasi statica so che $p = p(V,T)$\\
Se si conosce$p(V)$ sappiamo:
        \begin{figure}[tbh]
    \centering
    \includesvg[width = 150 pt]{term19} %ops sarebbe term ma vabbè
    \caption{}\label{fig:testsvg}
    \end{figure}
QUindi so che in generale il lavoro dipende dalla dipendenza di $p$ da $V$ poichè come si vede l'area rossa e l'area verde sono diverse ma il punto inziiale e finale sono uguali.\\
E quindi in genrale il lavoro NON è un differneziale (anche per questo che si indica con $\delta L$ invece di $dL$).
\newpage
\subsection{Calcolo effettivo del lavoro di un fluido}
Sono nel caso di trasf. quasi-statica:
\begin{gather*}
    L = \int_{V_i}^{V_f} p dV = p \int_{V_i}^{V_f} dV = p (V_i-V_f)
\end{gather*}
Prendiamo ora una trasf. sempre quasi-statica ma anche isoterma $(T = cost)$
\begin{gather*}
    L = \int_{V_i}^{V_f} p dV = ?
\end{gather*}
Ma se sono nel caso del gas perfetto so l'equazione di stato (o in casi particolari in cui so ricavarmela).
\begin{gather*}
    pV = \mathcal{N} R T\\
    p = \frac{\mathcal{N}RT}{V}\\
    L = \int_{V_i}^{V_f} p dV = \int_{V_i}^{V_f} \frac{\mathcal{N} RT}{V} dV \qquad \text{la temp. è cost.}\\
    = \mathcal{N}RT \int_{V_i}^{V_f} \frac{dV}{V} = \mathcal{N} RT ln(\frac{V_f}{V_i})
\end{gather*}
Fare attenzione che sono in un caso in cui ho tre precisazioni: gas perfetto trasf. quasi statica e temp costante, sono tante!\\
Vediamo come fare nel caso di trasf. non quasi statiche...

\subsection{caso nell'espansione libera di un gas}
So che un caso di questo tipo non è di sicuro quasi sitatica.\\
    \begin{figure}[tbh]
    \centering
    \includesvg[width = 150 pt]{termo20} 
    \caption{}\label{fig:testsvg}
    \end{figure}
Ricordiamo che un gas in espansione si ha quando la parete nel mezzo viene spostata istanntaneamente sulla parete a dx\\
Quanto lavoro fa un fluido in un espansione di questo tipo?\\
NON FA LAVORO. perchè il gas sta premendo solo su pareti fermi o sul nulla:
\begin{gather*}
    L = 0
\end{gather*}
Quindi posso avere trasformazioni del volume anche grandi ma il lavoro della pressione di esso è nullo.

\subsection{Lavoro dell'ambiente}
Queste forze di pressione del fluido del sistema fanno $\delta L>0$ sull'ambiente quando il sistema si espande: $ dV >0$ e $\delta L<0$ quando $dV <0$\\
È anche vero il contrario ovvero l'ambiente fa lavoro sul sistema, in generale:
\begin{itemize}
    \item Quando $\delta L >0$ si dice che il sistema esercita lavoro sull'ambiente. 
    \item Quando $\delta L >0$ si dice che l'ambiente esercita lavoro sul sistema.
\end{itemize}
Questo ci fa capire che il lavoro è il modo in cui i sistemi si scambiano energia meccanica tra di loro. Se io faccio lavoro meccanico questo lavoro puo essere usato per sollevare un peso o essere immagazzinato come energia potenziale, quindi se c'è lavoro c'è uno scambio di ENERGIA MECCANICA.\\
Prendiamo il caso in cui un fluido può fare lavoro e guardiamo la sua espansione potenziale:
\begin{multicols}{2}
    \noindent
\begin{center}
            \begin{tikzpicture}
                \draw[red](-1,1) -- (-1,2) node[midway, left]{h};
                \draw(1,0) -- (1,2);
                \draw[dashed](-1,2) --(1,2);
                \node[rectangle, draw, minimum width=2cm, minimum height=2cm] {\textcolor{green}{fluido}};
                \node at(0,1.3)[rectangle, draw, minimum width=0.2cm, minimum height=0.2cm] {m};
                \node at(0,2.3)[rectangle, dashed, draw, minimum width=0.2cm, minimum height=0.2cm] {m};
                \node at(1.4,1.03){$\leftarrow B $};
            \end{tikzpicture}
\end{center}
        \columnbreak
        \begin{gather*}
            \Delta E_{pot}(A) = mgh \qquad p = \frac{F}{B} = \frac{mg}{B}\\
            L(S) = p \Delta V\\
            \Delta V = h B\\
            L(s) = \frac{mg}{\cancel{B}}h \cancel{B} = mgh\\\\\\
        \end{gather*}
\end{multicols}
    Prendiamo ora un'altro esempio:

    \begin{center}
            \begin{tikzpicture}
                \draw(-1,2)--(-1,2.5);
                \draw(1,2)--(1,2.5);
                \draw[red](-1,1) -- (-1,2) node[midway, left]{h};
                \draw(1,0) -- (1,2);
                \draw[dashed](-1,2) --(1,2);
                \node[rectangle, draw, minimum width=2cm, minimum height=2cm] {\textcolor{green}{fluido}};
                \draw[decoration={aspect=0.3, segment length=1.5mm, amplitude=3mm,coil},decorate] (0.7,2) -- (0.7,2.5);             
                \draw[decoration={aspect=0.3, segment length=1.5mm, amplitude=3mm,coil},decorate] (-0.4,1) -- (-0.4,2.5);             
            \end{tikzpicture}
    \end{center}

    \subsection{estensività del lavoro}
    Mi chiedo ora il lavoro è una grandeza estensiva? ...si :
    \begin{gather*}
        \delta L = \underset{\text{int.}}{p} \underset{\text{est.}}{dV}
    \end{gather*}

    \subsection{lavoro in una trasformazione adiabatica}
    Abbiamo già visto un caso di trasformazione adiabatica in cui il lavoro era nulla ma era un caso molto particolare, in generale non è detto.\\
    Prendiamo ora un'altro esempio specifico (didattico):\\
    Se io ho un certo stato iniziale e finale con temperature rispettivamente iniziali o finai non c'è un solo modo per fare la transizione tra lo stato iniziale e finale, e fare tutto adiabaticamente.
    PRendo il volumme che scelgo $V_a$ che in generale $V_a > V_i$.
    Siccome stiamo considerando un gas perfetto so che la temperatura non varia, quindi posso predirre $V_a$ con precisione:

    \begin{figure}[tbh]
    \centering
    \includesvg[width = 150 pt]{term23} 
    \caption{}\label{fig:testsvg}
    \end{figure}
    Quindi:
    \begin{gather*}
        L_{ia} = 0\\
        L_{bf} = 0\\
        L_{ab} = \int_{V_a}^{V_b} p dV
    \end{gather*}
    In generale questo è UN modo per passare dallo stato iniziale allo stato finale ma non è l'unico:
    \begin{gather*}
        L^{(1)}_{if} = \underset{\underset{0}{||}}{L_{ia}} + \underset{\underset{\int_{V_a}^{V_b} p dV }{||}}{L_{ab}} + \underset{\underset{0}{||}}{L_{bf}} \\
        L^{(1)}_{if} = \int_{V_a}^{V_b} p dV\\
    \end{gather*}
    Posso spostare la parete e avere altri volumi $V_c$ e $V_d$:
    \begin{gather*}
        L^{(2)}_{if} = \int_{V_c}^{V_d} p dV\\
    \end{gather*}
    Già da qui si capisce che da qui si possono avere infiniti casi.
    Posso andare avanti in casi diversi:
    \begin{gather*}
        L^{(3)}_{if} = \int_{V_g}^{V_n} p dV\\
        L^{(4)}_{if} = \int_{V_c}^{V_d} p dV\\
    \end{gather*}
    In generale si può pensare:
    \begin{gather*}
        L^{(1)}_{if} \neq L^{(2)}_{if} \neq L^{(3)}_{if} \neq L^{(4)}_{if} \neq L^{(5)}_{if}
    \end{gather*}
    poichè sto facendo integrali diversi, MA in realtà:
    \begin{gather*}
        L^{(1)}_{if} = L^{(2)}_{if} = L^{(3)}_{if} = L^{(4)}_{if} = L^{(5)}_{if}
    \end{gather*}

    \newpage
    \section{13/10/25}
    Fissato uno stato inziiale e finale abbiamo calcolato il lavoro, e abbiamo scoperto che:
    \begin{gather*}
        L_{if}^{(1)} =L_{if}^{(2)}=L_{if}^{(3)}=L_{if}^{(4)}=L_{if}^{(5)}
    \end{gather*}
    Se noi ciamiamo i lavoro adiabatico, quello fatto in una trasformazione adiabatica, quello che si trova è che il lavoro adiabatico non dipende dalla trasformazione:
    \begin{gather*}
        \underset{\gamma_1}{L_{i \to f}^{(adiabatico)}} = \underset{\gamma_2}{L_{i \to f}^{(adiabatico)}} \qquad \forall \text{trasformazione } \gamma_1,\gamma_2
    \end{gather*}
    In più possiamo dire:
    \begin{gather*}
        \underset{\gamma}{L_{i \to i}^{(adiabatico)}} = 0 \qquad \forall \gamma \forall i,i
    \end{gather*}
    Dove $i$ è lo stato iniziale.\\
    Questo in generale è un risultato sperimentale e ci dice tanto sulla natura, dice infatti che quando un sistema compie una trasformazione adiabatica lo deve fare in maniera precisa, è di fatto un vincolo che da diverse informazioni che ci sarà molto utile.\\
    Questo prende anche il nome di \textbf{primo principio della termodinamica}.\\\\

    Ora ipotizzo (so) che esista una funzione generica del tipo:
    \begin{gather*}
        L_{i \to f}^{(adiabatico)} = F(i,f) = F(p_i,V_i,p_f,V_f)\\
        F(i,i) = 0
    \end{gather*}
    Siccome dipende che esiste solo dallo stato iniziale e finale E quando sono uguali mi fa $0$.\\
    \begin{definition}
        Esiste quindi una funzione $U$ che rappresenta l'\underbar{energia interna}, ovvero del lavoro complessivo fatto dal sistema:
        \begin{gather*}
            \Delta U = U(f)- U(i) = -L_{i \to f}^{(adiabatico)}
        \end{gather*}
        Il fatto di usare un meno è una convenzione(attenzione a non confondere questa con l'energia potenziale di una forza)\\
        Questa definizione è univoca a meno di una costante poichè se prendo $U' = U +c$ comunque soddisfa la definizione poichè sulla differenza si annulla la costante.\\
        Possiamo esprimere quindi questa funzione a meno di una costante spostando lo 0 su uno stato che mi fa comodo:
        \begin{gather*}
            U(0) = 0\\
            U(A)  = -L_{0 \to A}^{(adiabatico)}
        \end{gather*}
    \end{definition}
    \hfil\\
    Complessivamente la definizione la posso riscrivere:
    \begin{gather*}
        \exists U : U(f)-U(i) = -L_{i \to f}^{(adiabatico)}
    \end{gather*}
    In generale il lavoro non è esrpimibile con una funzione di stato ma in questo caso siccome si parla di lavoro adiabatico allora è un equazione di stato.\\
    Inoltre in generale:
    \begin{gather*}
        A \overset{\text{adiabatica}}{\rightarrow} B\\
        \nexists B \overset{\text{adiabatica}}{\rightarrow} A\\
    \end{gather*}
    In generale se un gas si espande non torna indietro e se lo comprimo per farlo tornare alo stato di volume iniziale aumenteràtemperatura e quindi non tornerò allo stato di partenza.
    \begin{gather*}
        A(V_1,T) \quad B(V_2,T) \qquad V_1 > V_2\\
        \exists ad \ A \to B\\
        \nexists ad \ B \to A
    \end{gather*}
    Dunque esiste sempre almeno una trasformazione in un verso ma non è assolutamente vero che esista una trasformazione inversa.\\
    Quindi:
    \begin{gather*}
        \forall A,B \text{ stati di equilibrio}\\
        \exists \text{ almeno una t.a.}\\
        A \overset{\text{adiabatica}}{\rightarrow} B\\
        B \overset{\text{adiabatica}}{\rightarrow} A
    \end{gather*}
    Questo È INCLUSO NEL \textbf{primo principio della termodinamica}\\
    Supponiamo che esista una funzione reversibile in entrambi i sensi:
    \begin{gather*}
        U(B) - U(A) = -L_{A \to B}^{(ad)}\\
        U(A) - U(B) = -L_{B \to A}^{(ad)}\\
    \end{gather*}        
    Questo mi porta ad avere due funzioni equivalenti che devono valere entrambe quindi:
    \begin{gather*}
        0 = -L_{A \to B}^{(ad)} -L_{B \to A}^{(ad)}\\
        L_{A \to B}^{(ad)} = -L_{B \to A}^{(ad)}
    \end{gather*}
    Che non è escontato, questo mi porta a dire:
    \begin{gather*}
        \forall A,B\\
        \text{\underline{se} } \exists \ A \overset{ad}{\to} B \quad U(B)-U(A) = -L_{A \to B}^{(ad)} 
        \text{\underline{se} non esiste \underbar{allora} }\exists B \overset{ad}{\to} A\\
        U(B)- U(A) = +L_{B \to A}^{(ad)}
    \end{gather*}
    In forma compatta la definizione di energia interna che segue il primo principio è:
    \begin{gather*}
        \Delta U = -L^{(ad)}
    \end{gather*}
    Da cui segue:
    \begin{gather*}
        \Delta U + L^{(ad)} = 0
    \end{gather*}
    Questo ci dice semplicemente che il lavoro del sistema è proporzionale alla diminuzione di questa energia interna del sistema.\\
    Ad esempio se il sistema fa lavoro sull'esterno avrò che l'energia interna del sistema diminusice.
    \begin{definition}
        E invece nel caso NON adiabatico?
        Siccome $U$ non è un'equazione di stato ma è in linea con lo stato del sistema allora è sempre esprimibile e ben definita.
        \begin{gather*}
            \overset{i \to f}{\Delta U + L} = Q \text{("quantità di calore")}
        \end{gather*}
        Questa funziona proprio da definizione per quantità di calore, normalmente si tente a scriverla nella forma:
        \begin{gather*}
            \text{\fbox{$Q-L = \Delta U$}}
        \end{gather*}
        E viene di solito detta: espressione matematica in forma integrale del I° principio.\\
    \end{definition}
    Anche quantità di calore non è unequazione di stato e quantità di calore e energia interna sono quantità estensive, e dimensionalmente sono energie(si misurano in Joule).\\
    Per il segno (che è arbitrario), quando $Q>0$ il sistema ha aquisito calore/energia, mentre per $Q < 0$ ha ceduto calore/energia. In generale indica la quantità di energia trasmessa con l'esterno che nelle trasformazioni adiabatiche è naturalmente nullo ($=0$)\\
    Quindi dire che una trasformazione è adiabatica o una trasformazione non scambia quantità di calore è la medesima cosa.
    In una trasformazione ciclica avrò $\Delta U= 0 \qquad Q-L = 0 \qquad Q=L$.\\
    \begin{example}
        Prendiamo in esempio una trasf. ciclica adiabatica.
    
    \begin{figure}[tbh]
    \centering
    \includesvg[width = 150 pt]{term24} 
    \caption{}\label{fig:testsvg}
    \end{figure}
    
    \end{example}  
    DAlla definizione operativa di quantitò di calore grazie al primo prncipio posso dire che la quantitòà di calore che scambio in una certa trasfromazione per definizione è data da:
    \begin{gather*}
        Q_{A \to B} = \underset{\underset{U(B)-U(A)}{||}}{\Delta U} + L_{A \to B}
    \end{gather*}
    Prendendo un gas perfetto quindi: $p = $ cost. e $L \approx 0$, so che $Q = \Delta U$.\\
    Purtroppo in generale questa è quanto vero quanto inutile poichè quasi mai si fanno calcoli effettivi con un gas perfetto. Quindi devo cercare una trasformazione adiabatica che mi trasmetta lo stesso stato iniziale allo stato finale.\\
    Un esempio è mulinemllo di Joule.\\
    Siccome
    \begin{gather*}
        Q = \Delta U \Rightarrow Q \propto \Delta T\\
        \delta Q = \mathcal{C} dT \qquad \mathcal{C} \text{ capacità termica}\\
        \mathcal{C} = \frac{\delta Q}{d T}
    \end{gather*}
    Questa in generale NON è una derivata.
    \begin{gather*}
        \mathcal{C} = (\frac{\delta Q}{d T})_x
    \end{gather*}
    Quindi tenendo fissa una variabile, analogmaente posso:
    \begin{gather*}
        \mathcal{C}_p = (\frac{\delta Q}{d T})_p
        \mathcal{C}_V = (\frac{\delta Q}{d T})_V
    \end{gather*}
    Se si vuole per una trasformazione adiabatica $\mathcal{C}_{\text{ad}} = 0$, mentre per una trasf. isotemra ($T$= cost.) è indefinità poichè $dT$ è esattamente $0$.
    \begin{gather*}
        \text{per un liquido} \mathcal{C}_p \approx \mathcal{C}_V\\
        \text{per un gas} \mathcal{C}_p \neq \mathcal{C}_V\\
    \end{gather*}
    la capacità termica è definita si ricorda come:
    \begin{gather*}
        \text{\fbox{$\mathcal{C}_x = (\frac{\delta Q}{dT})_x$}}\\
    \end{gather*}
    per ricavare le dimensioni:
    \begin{gather*}
        [\mathcal{C}] = \frac{[E]}{[T]}
    \end{gather*}
    Quindi si usano joule su kelvin ($J/K$).\\
    definiamo ora il calore specifico(per unità di massa):
    \begin{gather*}
        c = \frac{\mathcal{C}}{M}\\
        JK^{-1} kg^{-1}
    \end{gather*}
    e il calore molare come:
    \begin{gather*}
        c = \frac{\mathcal{C}}{\mathcal{N}}\\
        JK^{-1} mol^{-1}
    \end{gather*}
    La capacità termica mi indica proprio come il mio sistema reagisce ad uno stimolo esterno, come ad es. uno scambio di calore con l'esterno.\\
    Siccome è una quantità additiva si ha che 
    \begin{gather*}
        \mathcal{C} \geq 0
    \end{gather*}
    In generale per definire come un sistema reagisce agli stimoli esterni serve sia la capacità termica sia la quantità di calore.

    \newpage
    \section{16/10/25}
    Abbiamo definito la capacità termica come:
    \begin{gather*}
        \mathcal{C}_x =\left(\frac{\delta Q}{d T} \right)_x 
    \end{gather*}
    Una delle maggiori utilità della capacità termica è quella di definire in maniera precisa i termometri e termostati.\\
    La capacità termica indica sostanzialmente quanto un sistema scambia calore con l'esterno, se è piccola scambierà molta energia se è grande ne scambierà molta poca, infatti un termometro e un termostato \underbar{idealmente} :
    \begin{gather*}
        \text{termostato: } \mathcal{C} \to \infty
        \text{termometro: } \mathcal{C} \to 0
    \end{gather*}
    Di fatto questo è un caso ideale, prendiamo quindi in esempio un termostato $S_1$ che misura un sistema $S_2$ si ha che:
    \begin{gather*}
        \mathcal{C}(S_1) >> \mathcal{C}(S_2)
    \end{gather*}
    Analogamente un termometro $S_1$ che misura un sistema $S_2$
    \begin{gather*}
        \mathcal{C}(S_1) << \mathcal{C}(S_2)
    \end{gather*}
    \subsection{Calore latente}
    Durante i passaggi di stato è interessante vedere cosa accade, intanto sappiamo che la temperatura resta costante.
    \begin{wrapfigure}{r}{0.4\textwidth}
        \begin{tikzpicture}
        \draw[->](0,0)-- (3,0) node[at end, below]{$t$};
        \draw[->](0,0)-- (0,3) node[at end, left]{$T$};
        \draw(0,0)--(1,1);
        \draw(1,1)--(2,1);
        \draw(2,1)--(3,2);
        \draw[dashed](1,1)--(1,0);
        \draw[dashed](2,1)--(2,0);
        \node at(1.6,-0.3){coesistenza};
    \end{tikzpicture}
    \end{wrapfigure}
    \textbf{Calore latente} si definisce come:
    \begin{gather*}
        l_v^{H_2O} = 2.27 \times 10^6 J/kg\\
        l_F^{H_2O} = 3.33 \times 10^5 J/kg
    \end{gather*}
    L'energia interna dipende dalle variabili termodinamiche e dalla .\\
    Il calore latente dice quanto varia il calore nei passaggi di stato.\\
    Differenza di energia interna ad uguali valori termodinamici ma a diverse fasi. es l'acqua ha un energia interna più grande del ghiaccio di esattamente $l_F^{H_2O}$
    \subsection{notazioni e relazione della capacità termica}
    Consideriamo le variabili termodinamiche $p,V,T$, e scegliamo di considerare la nostra energia interna dal volume e dalla temperatura:
    \begin{gather*}
        U(v,T) \qquad dU = \left( \frac{\partial U}{\partial V} \right)_T dV + \left( \frac{\partial U}{\partial T} \right)_V dT 
    \end{gather*}
    Perchè scelgo questa notazione che si discosta dalle normale derivate parziali?(si indica col pedice la variabile che teniamo cost.)\\ 
    È dovuto al fatto che le funzioni in termodinamica potrebbero essere funzioni di diverse variabili, e vanno specificate. Quindi o si specifica (come abbiamo fatto in maniera rindondante in questo caso) $U(V,T)$ che specifica le variabili usate, o uso la notazione dei pedici per indicare quali variabili cosnidero.\\
    Specificato questo ora passiamo ad esprimere $dU$ in questo modo:
    \begin{gather*}
        dU = \delta Q - \delta L\\
        \delta Q - \delta L =  \left( \frac{\partial U}{\partial V} \right)_T dV + \left( \frac{\partial U}{\partial T} \right)_V dT 
    \end{gather*}
    Mettiamoci ora in un caso particolare in cui non faccio lavoro, quindi ad es. sono nel caso $V=\text{cost.}$
    \begin{gather*}
        \text{\fbox{$V=$cost.}} \quad dV = 0 \quad \delta L = 0\\
        \delta Q =  \left( \frac{\partial U}{\partial V} \right)_T dV + \left( \frac{\partial U}{\partial T} \right)_V dT        
    \end{gather*}
    Dato che sono a volume costante posso dire:
    \begin{gather*}
        \delta Q = \mathcal{C}_V dT\\
        \mathcal{C}_V dT = \left( \frac{\partial U}{\partial T}\right)_V dT 
    \end{gather*}
    QUesto risultato ci permette di dire che:
    \begin{gather*}
        \text{\fbox{$\mathcal{C}_V = \left( \frac{\partial U}{\partial T}\right)_V$}}
    \end{gather*}
    Questo risultato mi dice che la derivata dell'energia interna del sistema rispetto alla temperatura è proprio la capacità termica di esso, in questo caso ci dice che questo risultato è proprio un'equazione di stato, e significa che mentre una capacità termica in generale NON è una funzione di stato ma quando considero $V=$ cost. alora si.
    \subsection{gas perfetti}
\begin{center}
        \begin{tikzpicture}
            \draw[->](0,0) -- (0,2);
            \draw[->](0,0) -- (2,0);
            \draw(0.2, 1.2) ..controls(0.6,0.7) and(0.9,0.5).. (1.5,0.3) node[midway,above, right]{$_{\text{ curva isoterma}}$};
            \filldraw[fill=red](0.2,1.2) circle (1pt) node[above]{$i$};
            \filldraw[fill=red](1.5,0.3) circle (1pt) node[right]{$f$};
        \end{tikzpicture}
\end{center}
    \newpage
    \subsection{Trasformazioni adiabatiche}
    Sono trasformazioni che sono contenute all'interno di pareti adiabatiche.\\

\begin{center}
            \begin{tikzpicture}
                \fill[pattern=north east lines](0,0)rectangle(2,1);
                \filldraw[fill=page] (0.2,0.2) rectangle (1.8,0.8);
                \draw(0,0) rectangle (2,1);
                \fill[fill=cyan] (0.2,0.2) rectangle (1,0.8);
                \draw[->](2.5,0.5) -- (3.5, 0.5);
                \fill[pattern=north east lines](4,0)rectangle(6,1);
                \draw(4,0) rectangle (6,1);
                \filldraw[fill=cyan] (4.2,0.2) rectangle (5.8,0.8);
                \node at(0,-0.5){$V_A$};
                \node at(4,-0.5){$V_B$};
            \end{tikzpicture}
\end{center}
    Si ricorda che siamo nel caso di una trasforamzione adiabatica e quindi:
    \begin{gather*}
        L = 0 \quad Q = 0 \quad \Delta U = 0\\
        \Delta U = Q-L
    \end{gather*}
    Siamo nel caso di gas perfetto quindi:
    \begin{gather*}
        \Delta T = 0\\
        (V_A,T) \to (V_B,T)\\
        (p_A,T) \to (p_B,T)\\
        U(V_A,T) = U(V_B,T)
    \end{gather*}
    Siccome so di essere nel caso di un gas perfetto (quindi $\Delta T = 0$) so che $U$ non dipende da $V$.
    \begin{gather*}
        U(p:A,T) = U(p_B,T)
    \end{gather*}
    Per lo stesso modo $U$ non dipende da $p$\\
    Di conseguenza un gas perfetto dipende solo da $T$:
    \begin{gather*}
        \text{gas perfetto} \Rightarrow U = U(T)
    \end{gather*}
    Può dipendere solo dal prodotto di $p$ e $V$ che saranno uguali alla temperatura a meno di costanti (esprimo $T$ in funzione di $p$ e $V$).\\
    Ora se siamo a nel caso di $T=$cost. e quindi $\Delta U =0$ per un gas perfetto so che $Q=L$ poichè
    \begin{gather*}
        L = \int_{V_i}^{V_f} p dV = \mathcal{N} RT \ln(\frac{V_f}{V_i})
    \end{gather*}
    Sfruttando la relazione precedente della capacità termica posso dire:
    \begin{gather*}
        \mathcal{C}_V = \left( \frac{\partial U}{\partial T} \right)_V \overset{gas perf.}{\Rightarrow} \mathcal{C}_V = \frac{dU}{dT} \\
        U = U(T) \qquad \mathcal{C}_V = \mathcal{C}_V(T)
    \end{gather*}
    \begin{gather*}
        dU = \delta Q -\delta L
        \text{sono nel caso di gas perfetto}\\
        \mathcal{C}_V dT = \delta Q - p dV\\
        p = \text{cost.} \Rightarrow \mathcal{C}_V = \mathcal{C}_p dT -p dV\\
        (\mathcal{C}_p-\mathcal{C}_V) dT = p dV
    \end{gather*}
    So che c'è una relazione tra $V$, $p$ e $T$:
    \begin{gather*}
        V = V(p,T)\\
        dV = \left( \frac{\partial U}{\partial p} \right)_T dp + \left( \frac{\partial U}{\partial T} \right)_p dT\\
        (\mathcal{C}_p-\mathcal{C}_V) dT = \cancel{\left( \frac{\partial U}{\partial p} \right)_T dp} + \left( \frac{\partial U}{\partial T} \right)_p dT\\
        \mathcal{C}_p - \mathcal{C}_V = \left( \frac{\partial U}{\partial T} \right)_p dT\\
        V = \frac{\mathcal{N}RT}{p} \qquad \left( \frac{\partial U}{\partial T} \right)_p = \frac{\mathcal{N}R}{p}\\
        \text{\fbox{$\mathcal{C}_p - \mathcal{C}_V = \mathcal{N} R$}}
    \end{gather*}
    Posso anche esprimerla in maniera generale però diventa una affermazione più debike
    \begin{gather*}
        \mathcal{C}_p - \mathcal{C}_V \geq 0
    \end{gather*}
    \begin{gather*}
        \mathcal{C}_V = g(T)\qquad \mathcal{C}_p = g(T) + \mathcal{N}R
    \end{gather*}
    \newpage
    \section{17/10/25}
    \begin{gather*}
        \mathcal{C}_p-\mathcal{C}_V = \mathcal{N}R \qquad \mathcal{C}_V = \frac{dU}{dT}\\
        \mathcal{C}_V = g(T)\\
        \mathcal{C}_p = g(T) +R
    \end{gather*}
    Dagli esperimenti scopriamo che la capacitàtermica di un gas perfetto dipende dalla composizione chimica del gas perfetto, cosa che di solito non è rilevante.\\
    Però è una dipendenza molto debole dalla composizione chimica, più che altro dipende dal fatto che le molecole che costituiscono.\\
    In generale si differenziano in classi:
    \begin{itemize}
        \item caso siano monoatomico ($He, Kr, Ar$) (quindi formate da un singolo atomo)
        \item caso biatomico (ossigeno $O_2$), e non importa siano dello stesso atomo.
        \item infine caso poliatomico (non lineari), con condizione che se tutti gli atomi cadono sulla stessa linea si ricade nel caso biatomico.
    \end{itemize}

\begin{center}
        \begin{tabular}{ | m{5em} | m{2cm}| m{2cm} | m{2cm} |  } 
      \hline
       & \begin{center}$\mathcal{C}_V/R$\end{center} & \begin{center}$\mathcal{C}_p/R$\end{center} & \begin{center}$\gamma = \frac{\mathcal{\mathcal{C}}_p}{\mathcal{C}_V}$\end{center} \\
      \hline
      1 atomo \tab& \begin{center}$\frac{3}{2}$\end{center}& \begin{center}$\frac{5}{2}$\end{center} & \begin{center}$\frac{5}{3}$\end{center} \\
      \hline
      2 atomi \tab & \begin{center}$\frac{5}{2}$\end{center} & \begin{center}$\frac{7}{2}$\end{center} & \begin{center}$\frac{7}{5}$\end{center} \\ 
      \hline
      $>$2 atomi $_{\text{(non lineari)}}$  & \begin{center}3\end{center} & \begin{center}4\end{center} & \begin{center}$\frac{4}{3}$\end{center} \\ 
      \hline
    \end{tabular}
\end{center}
    Adesso consideriamo:
    \begin{gather*}
        U(T=0) \qquad\text{gas pefetto}\\
        \frac{d U}{dT} = \mathcal{C}_V\\
        \int_{0}^{T} \frac{d U}{dT} dT = \int_{0}^{T} \mathcal{C}_V dT\\
        U(T) - \underset{\underset{0}{||}}{U(0) }= \mathcal{C}_V T 
    \end{gather*}
    Noi sappiamo:
    \begin{gather*}
        \boxed{U(T)= \mathcal{C}_V T} \quad \boxed{*}\\
        \mathcal{C}_p - \mathcal{C}_V = \mathcal{N} R\\
        \mathcal{C}_p = \gamma \mathcal{C}_V\\
        \mathcal{C}_V = \frac{\mathcal{N}R}{\gamma -1}
    \end{gather*}
    possiamo quindi riscrivere l'equazione \fbox{*} come:
    \begin{gather*}
        U(T) = \frac{\mathcal{N}R}{\gamma -1} T = \frac{pV}{\gamma -1}
    \end{gather*}
    In particolare per un gas perfetto monoatomico
    \begin{gather*}
        U(T) = \frac{3}{Q} \mathcal{N} R T = \frac{3}{2}pV
    \end{gather*}
    \subsection{calcolare la curva isotermica}
    siccome siamo nel caso di un gas perfetto
    \begin{gather*}
        \delta Q = 0\\
        dU = \cancel{\delta Q} - \delta L \\
        \mathcal{C}_V dT = -pdV
    \end{gather*}
    Abbiamo:
    \begin{gather*}
        p = \frac{\mathcal{N} R T}{V}
    \end{gather*}
    Quindi:
    \begin{gather*}
        \mathcal{C}_V dT = -\frac{\mathcal{N} R T}{V} dV\\
        \frac{dT}{T} = - \frac{\mathcal{N}R}{\mathcal{C}_V} \frac{dV}{V}
    \end{gather*}
    Ci serve che:
    \begin{gather*}
        \mathcal{C}_V = \frac{\mathcal{N}R}{\gamma -1}
    \end{gather*}
    Quindi:
    \begin{gather*}
        \frac{dT}{T} = -(\gamma -1) \frac{dV}{V}
    \end{gather*}
    Questa è un'equazione differenziale a variabili separabili (già separate) e quindi facilmente risolvibile.\\
    Dobbiamo dunque fissare le variabili che sto considerando:
    \begin{gather*}
        O \equiv ( T_0,p_0,V_0) \qquad S \equiv (T,p,V)
    \end{gather*}
    Di fatto la pressione non la consideriamo ma comunque è fissata.\\
    Ricordiamo che siamo in una trasformazione quasi-statica:
    \begin{gather*}
        \int_{0}^{S} \frac{dT}{T} = - (\gamma -1) \int_{0}^{S} \frac{dV}{V} \\
        \int_{T_0}^{T} \frac{dT}{T} = - (\gamma -1) \int_{V_0}^{V} \frac{dV}{V}\\
        \ln(\frac{T}{T_0}) = -(\gamma -1) \ln(\frac{V}{V_0})\\
        \frac{T}{T_0} = (\frac{V}{V_0})^{-(\gamma -1)}\\
        TV^{\gamma -1} = T_0 V_0^{\gamma -1}
    \end{gather*}
    Ho quindi trovato:
    \begin{gather*}
        \boxed{TV^{\gamma -1} = \text{cost.}} 
    \end{gather*}
    E considerando $T = \frac{pV}{\mathcal{N}R}$, posso riscriverla come:
    \begin{gather*}
        \boxed{pV^\gamma = \text{cost.}}\\
    \end{gather*}
    Posso anche esprimerla in funz. di $T$:
    \begin{gather*}
        \boxed{T^\gamma p^{1-\gamma} = \text{cost.}}
    \end{gather*}
\begin{center}
        \begin{tikzpicture}
            \draw[->](0,0) -- (0,2)node [at end,left]{$p$};
            \draw[->](0,0) -- (2,0)node [at end,below]{$V$};
            \draw(0.2, 1.2) ..controls(0.6,0.7) and(0.9,0.5).. (1.5,0.3) node[at end,below, right]{$_{adiabatica}$};
            \draw[dashed](0.2, 1.2) ..controls(0.6,1) and(0.9,0.8).. (1.5,0.7) node[at end, above, right]{$_{isoterma}$};
            \draw[dashed](0.2,1.2) -- (0.2,0) node[at end, below]{$V_i$};
            \draw[dashed](1.5,0.3) -- (1.5,0) node[at end, below]{$V_f$};
        \end{tikzpicture}
\end{center}
    Questo ci dice anche se abbiamo un gas perfetto che si sta espandendo in una trasformazione adiabatica tra un certo $V_i$ e $V_f$, e dunque farà lavoro, e sarà proprio l'area sotto la curva adiabatica, e questo è sicuramente inferiore al lavoro fatto dall'isoterma.\\
    Il lavoro quindi è proprio:
    \begin{gather*}
        L = \int_{V_0}^{V} p \ dV
    \end{gather*}
    E la posso riscrivere considerando che $pV^\gamma = p_0V_0^\gamma$ e quindi $p = \frac{p_0 V_0^\gamma}{V^\gamma}$
    \begin{gather*}
        L = \int_{V_0}^{V} \frac{p_0 V_0^\gamma}{V^\gamma} dV = p_0 \int_{V_0}^{V} \frac{V_0^\gamma}{V^\gamma} dV \\
        = p_0 V_0 \int_{x=1}^{x=\frac{V}{V_0}} x^{-\gamma} dx
    \end{gather*}
    Dove:
    \begin{gather*}
        \int_{x=1}^{x=\frac{V}{V_0}} x^{-\gamma} dx  = \left[ \frac{1}{1+\gamma}x^{1-\gamma} \right]^{\frac{V}{V_0}}_1 
    \end{gather*}
    Quindi:
    \begin{gather*}
        L = \frac{p_0 V_0}{1 - \gamma}\left[ \left(\frac{V}{V_0}\right)^{1-\gamma} -1\right] = \frac{\mathcal{N}R T_0}{\gamma -1} \left[ 1 - \left(\frac{V}{V_0}\right) ^{1-\gamma} \right]  
    \end{gather*}

    \begin{gather*}
        pV^\gamma = p_0 V_0^\gamma = \mathcal{N} R T_0 \ V_0^{\gamma-1}
    \end{gather*}
    Osservo nel caso di $\gamma = 1$ e ottengo:
    \begin{gather*}
        pV = \mathcal{N}RT_0
    \end{gather*}
    Se questa cosa è coerente ( e lo è) vuol dire che il lavoro dovrebbe tornarmi il lavoro che mi viene da una curva isotemra, siccome non posso sostituire a 1 fo il limite:
    \begin{gather*}
        \lim_{\gamma \to 1} \frac{\mathcal{N} R T_0}{1-\gamma} \left[ \left(\frac{V}{V_0}\right)^{1- \gamma} -1 \right] = \mathcal{N} R T_0 \lim_{x \to 0} \frac{1}{x} \left[ \left( \frac{V}{V_0} \right)^x  -1 \right] = \mathcal{N} R T_0 \ln\left(\frac{V}{V_0}\right)   
    \end{gather*}
    Se ora definiamo delle classi in base a come varia $\gamma$ quindi tutte le trasformazioni per cui:
    \begin{gather*}
        pV^k = \text{cost.} \qquad k \in \mathbb{R}
    \end{gather*}
    Queste trasformazioni si chiamano trasformazioni \textbf{politropiche}.\\
    Per fare alcuni esempi con $k=0$ avrò una trasf. a pressione costante, mentre se fo tendere $k \to \infty$ avrò il volume costante.\\
    \underbar{Tutte le trasformazioni politropiche nel limite del gas perfetto sono equazioni di stato.}

    \begin{gather*}
        pV^k = p_0 V_0^k \qquad p = p_0 V_0^k V^{-k}\\
        \delta L = p \ dV = p_0V_0^kV^{-k} \ dV = p_0V_0^kd\left(\frac{V^{1-k}}{1-k}\right)\\
        = d\left( \frac{p_0V_0^kV^{-k}V}{1-k} \right)  = d\left(\frac{pV}{1-k}\right) = d\left(\frac{\mathcal{N}RT}{1-k}\right)\\
        \delta L = \frac{\mathcal{N}R}{1-k} dT
    \end{gather*}
    $\forall$ politropico gas perfetto quasi statica con $k\neq 1$
    \begin{gather*}
        L_k = \frac{\mathcal{N}R}{1-k}(T-T_0) = \frac{}{}
    \end{gather*}
    Nel caso in cui:
    \begin{gather*}
        k = 0\qquad \text{$p$ = cost.}\\
        L_0 = \mathcal{N} R \Delta T = \mathcal{N}R T - \mathcal{N} R T_0\\
        = pV -p_0V_0 = p_0V-p_0V_0 = p_0 (V-V_0)
    \end{gather*}
    E ottengo:
    \begin{gather*}
        (\delta Q)_k = dU + \delta L = \mathcal{C}_V dT + \frac{\mathcal{N}R}{1-k} dT\\
        \boxed{\mathcal{C}_k = \mathcal{C}_V + \frac{\mathcal{N}R}{1-k}}
    \end{gather*}
    Dove l'eq evidenziata è il risultato generale.\\
    
    \begin{center}
        \begin{tikzpicture}
            \draw[->](0,0) -- (2,0) node[at end,below]{$V$};
            \draw[->](0,0) -- (0,2) node[at end,left]{$p$};
            \draw(0.2, 1.2) ..controls(0.6,0.7) and(0.9,0.5).. (1.5,0.3) node[at end,below, right]{$k = 1$};
            \draw[cyan](0.1, 1) ..controls(0.6,0.7) and(0.9,0.5).. (1.5,0.6) node[at end,below, right]{$k <1$};
            \draw[orange](0.35, 1.4) ..controls(0.6,0.7) and(0.9,0.5).. (1.3,0.1) node[at end,below]{$k > 1$};
            \draw[red](0.8, 1.5) -- (0.8,0.1) node[at start,above]{$k = +\infty$};
            \draw[blue](0.2, 0.65) -- (1.5,0.65) node[at start, left,below]{$k = 0$};
        \end{tikzpicture}
    \end{center}

    Se infatti andiamo nel caso adiabatico e poniamo $k=\gamma$:
    \begin{gather*}
        \mathcal{C}_\gamma = \mathcal{C}_V + \frac{\mathcal{N}R}{1- \gamma} = 0 \qquad \mathcal{C}_V = \frac{\mathcal{N}R}{\gamma-1}
    \end{gather*}
    Se siamo nel caso in cui un sistema non varia la quantità di sostanza che contiene si può sempre scrivere nel caso di $p = $ cost.$V^{-1}$:
    \begin{gather*}
        pV^k = \text{cost.}\\
        p = \text{cost.} \rho^k
    \end{gather*}
    \subsection{Esperimento di Ruchardt}
    \begin{center}
        \begin{tikzpicture}
            \draw(0,0) rectangle (2,1);
            \draw(0.9,1) rectangle(1.2,1.2);
            \fill[fill = page] (0.908,0.8) rectangle (1.195,1.1);
            \fill[fill=cyan] (0.95,0.5) rectangle (1.15,1.193);
            \fill[fill=cyan] (0.1,0.1) rectangle (1.9,0.9);
            \filldraw[fill=red] (1.05,1.2) circle (2pt) node[above]{$m$};
        \end{tikzpicture}
    \end{center}
    \begin{gather*}
        p = p_0 + mg\\
        p = \frac{mg}{S} +p_0\\
        y \qquad dV = S y\\
        V^\gamma = \text{cost.}
    \end{gather*}
    Se la differenzio ottengo:
    \begin{gather*}
        V^\gamma \ dp + \gamma V^{\gamma -1} p \ dV = 0\\
        dp + \gamma \frac{V^{\gamma -1}}{V^\gamma} p \ dV = 0\\
        dp = - \frac{\gamma p}{V} dV = -\frac{\gamma pS}{V}y
    \end{gather*}
    \begin{gather*}
        F = S \ dp = -\gamma \frac{pS^2}{V}y
    \end{gather*}
    E quindi ho una equazione di moto che è proporzionale ad $y$:
    \begin{gather*}
        m \ddot{y} = -\frac{\gamma p S^2}{V}y
    \end{gather*}
    \begin{gather*}
        \omega = \sqrt{\frac{\gamma p S^2}{mV}}\\
        \tau = \frac{2 \pi}{\omega} = 2\pi \sqrt{\frac{mV}{\gamma p S^2}}\\
        \gamma = \frac{4 \pi m V}{p S^2 \tau^2} \qquad p = p_0 +\frac{mg}{S}\\
        \gamma = \frac{4\pi^2mV}{\tau^2S(P_0 S + mg)}
    \end{gather*}
\newpage
\section{21/10/25}
\begin{gather*}
    p(z) p_0 exp(-z/k)
\end{gather*}
Tempo fa abbiamo fatto queste considerazioni:
$\rho (z) \qquad p(z)$
Per fare questo abbiamo ipotizzato che $p = \text{cost. } \rho$\\
Cerchiamo ora di reinterpretare questo problema con le conoscenze attuali:
Ricordiamo che assumiamo che tutto sia stazionario, e otteniamo qualcosa che dipende dalla posizione, e questo è un esempio esplicito (dal punto di vista termodinamico) in cui dobbiamo considerare le var. term. dalla posizione ma non dal tempo. in questo caso si parla appunto di eq. termodinamico locale.\\
Ricoriamo che il sistema è immerso in un campo gravitazionale ( quello della terra).\\
l'atmosfera si ricorda che per il $99\%$ di $O_2$ e $N_2$ (azoto e ossigeno) il resto $1\%$ di altro ($CH_4 , CO_2, H_2O$)\\
Valutiamo le temp. critiche di questi gas:
\begin{gather*}
    T_c(N_2) \simeq 125 K\\
    T_c (O_2) \simeq 155 K 
\end{gather*}
Anche se consideriamo la temperatura ad alta quota ($200K $) vediamo che è molto al di sopra della temp. critica quindi l'approssimazione dellatmosfera ad un gas perfetto è perfettamente ragioneole.\\
Il modello che abbiamo usato finora sfruttava proprio l'hp. di $p = \text{cost. } \rho$\\
In generale non è vero sempre infatti abbiamo appena constatato che la temperatura si abbassa con l'innalzamento di quota, (stiamo ipotizzando che l'atmosfera sia isoterma).\\
Immaginiamo che per varie cause da variazione di temperatura una regione dell'atmosfera diventa legggermente meno densa del gas circostante, e quindi per la spinta di archimede inizia a salire, ma la pressione è minore se sale di quota e quindi si scalda di più. Questo è un classico esempio di equilibrio stabile.\\
È vero anche per il caso opposto ovvero delle aree meno dense cominciano a scendere.\\

Questo accade solo perchè l'atmosfera è stratificata e questo fenomeno si chiama covenzione.\\
In media abbiamo quante bolle che salgono, di fatto queste bolle stanno facendo trasformazioni quasi sttiche adiabatiche, poichè sono di fatto molto lenti (fondamentalmente non sono efficienti e dunque non hanno una dipendenza dal tempo), quindi ha senso pensare che invece considerarle come trasfromazioni isotermiche si usa lipotesi che le trasformazioni siano adiabATICHE E VEDIAMO COSA SUCCEDE:
\begin{gather*}
    dU = \cancel{\delta Q} - \delta L\\
    dU +dV = 0 \\
    dU = \mathcal{C}_V dT
\end{gather*}
Sostituisco la capacità termica:
\begin{gather*}
    \mathcal{C}_V = m c_V\\
    mc_V dT + pdV = 0
\end{gather*}
Dalla scorsa volta differenziando avevamo ottenuto:
\begin{gather*}
    pdV = \frac{-V}{\gamma} dp
\end{gather*}
Di conseguenza:
\begin{gather*}
    m c_V dT -\frac{V}{\gamma} dp = 0\\
    \frac{m}{V} c_V dT - \frac{dp}{\gamma} = 0\\
    p c_V dT -\frac{dp}{\gamma} = 0\\
\end{gather*}
Sfruttando $    \gamma = \frac{c_p}{c_V}$
\begin{gather*}
    p \cancel{c_V} dT - \frac{\cancel{c_V}}{c_p} dp = 0\\
    dp = c_p \rho dT 
\end{gather*}
Ora so che varia ma non so come varia perchè ho ancora $dT$ quindi uso la dipendenza da l'altezza di quota e so che g a differenze di quote di qualche km varia di poco:
\begin{gather*}
    dp = -\rho g dz\\
    c_p \rho dT = -\cancel{\rho} g dz\\
    \boxed{\frac{dT}{dz} = -\frac{g}{c_p}}
\end{gather*}
Se si integra questa eq. differenziale si ottiene:
\begin{gather*}
    T(z) = T_0 - \frac{g}{c_p} z\\
    \frac{dT}{dz} \simeq 0.98 \times 10^{-2} m^{-1} K
\end{gather*}
Quindi le nostre considerazioni ci hanno fatto dire che la dipendenza della temperatura dalla quota è di tipo lineare (ed è così quindi tutto bene).\\
Però se misuro l'ultima misura che ho ottenuto con la teoria avro circa:
\begin{gather*}
    \left(  \frac{dT}{dz} \right)_{\text{mis}} \simeq -0.7 \times 10^{-2} m^{-1} K 
\end{gather*}
Ed è uno scarto di circa il 30\% che non è trascurabile, perchè avviene questo?\\
So che non è $g$ perchè da uno scarto dell 5\%,in realtà è dato dal rimanente 1\% di gas atmosferici ovvero il cambiamento di stato del vapore acqueo, che ha una temperatura critica altissima e quindi non è approssimabile ad un gas perfetto. Infatti ogni volta che l'acqua condensa rilascia calore latente.\\
Di fatto ho una sorgente di energia che scalda l'atmosfera, e quindi si ha che si raffredda un po' meno.\\
Questo si puo verificare considerando l'umidità nell'aria se si fa questa misura in un punto arido della terra sarà motlo vicino al valore teorico, se è in un punto molto umido sarà molto lontano.\\

\newpage
\section{23/10/25}
Abbiamo parlato la scorsa volta di come ci siano delle regioni che trasportano energia verso l'alto. Se si considera un esempio in cui abbiamo una fonte di calore (ad es un termostato) con sopra degli strati di fluidi, il calore viene in questo modo trasportato per moto convettivo, e una volta arrivato in cima si raffredda una volta in contatto con l'ambiente sovrastante formando così una colonna di discesa.\\
Questo è detto \textbf{convezione} ed è uno dei tre modi per trasportare calore ed è un metodo molto efficiente (il più efficiente che vedremo).\\
Un esempio pratico di questo moto convettivo può essere una pentola che bolle, ci sono vapori che salgono cedono calore sul coperchio e poi scendono. È anche il motivo per il quale i termosifoni funzionano, quello che succede è che il termosifone scalda l'aria a contatto con una regione macroscopica di fluido e ne cambia la densità e lo fa muovere e mescolare nella stanza.
\begin{gather*}
    \frac{\delta Q}{d t} \propto T_p - T_f
\end{gather*}
Quindi più il termosifone è caldo più scalda la temperatura nella stanza.\\\\
Un altro modo per trasportare calore è l'\textbf{irraggiamento}, quando una sorgente emette radiazione elettromagnetica con una distribuzione di frequenza che dipende dalla temperatura, e rilascia calore.\\
Questo metodo è in realtà il più efficiente per trasportare calore nel vuoto, sostanzialmente se non avessimo la possibilità di trasmettere calore con l'utilizzo di mezzi materiali.\\
L'irraggiamento funziona per QUALUNQUE sistema (cosa che non funzionanella convenzione con i solidi ad esempio).\\\\
Il terzo meccanismo di trasporto del calore è la \textbf{conduzione}, che è utile nel caso io abbia un sistema separato da una parete reale, in cui ovviamente la convezione non funziona perchè non funziona con i solidi, l'irraggiamento funziona ma con un solido è estremamente poco efficiente, in questo caso la conduzione è appunto il metodo migliore che ho per trasferire calore.\\\\
\subsection{irraggiamento}
\begin{center}
    \begin{tikzpicture}
        \draw[->](0,0) -- (4,0) node[at end, below]{$x$};
        \draw[<->](0,-1) -- (2,-1) node[midway,below]{$l$};
        \draw(0,-2.5) -- (0,2.5) node[midway, left]{$T_1$};
        \draw(2,-2.5) -- (2,2.5) node[midway, below, right]{$T_2$};
        \draw[dashed](0,2.5) -- (0,2.8);
        \draw[dashed](0,-2.5) -- (0,-2.8);
        \draw[dashed](2,2.5) -- (2,2.8);
        \draw[dashed](2,-2.5) -- (2,-2.8);
    \end{tikzpicture}
\end{center}
\begin{gather*}
    q = \frac{\delta Q}{d S \ d t}
\end{gather*}
Questo è il flusso di calore, e gli esperimenti mi dicono che questo è proprio:
\begin{gather*}
    q = -l \frac{T_2-T_1}{l}
\end{gather*}
dove $k$ è una costante che si chiama \textbf{conducibilità termica}
La unità di misura è:
\begin{gather*}
    W m^{-1} k^{-1}
\end{gather*}
e dipende dal materiale stesso, ad esempio:
\begin{itemize}
    \item argento $k \approx 400 W m^{-1} K^{-1}$
    \item vetro $k \approx 0.8 W m^{-1} k^{-1}$
    \item sughero, fibra di vetro $k \approx 0.04 W m^{-1} K^{-1}$
\end{itemize}
È un cambiamento sostanziale molto grande.
\subsection{legge di Foulier}
Se prendiamo che la temperatura varia in base alla posizione possiamo definire:
\begin{gather*}
    T(x)
\end{gather*}
che ci porta a poter dire:
\begin{gather*}
    q = -k \frac{T(x+l) -T(x)}{l}\\
    q(x) = -k \lim_{l \to 0} \frac{T(x+l) - T(x)}{l}\\
    \boxed{q(x) = -k \frac{dT(x)}{dx}}
\end{gather*}
Quest'ultima è detta \textbf{legge di Fourier}\\
Ora abbiamo fissato un punto nel tempo.\\\\
Dobbiamo ora fare un passo avanti e abbandonare completamente l'eq. e quindi si assume che si puo definire un punto nello spazio che dipenda sia dalla temperat sia dal tempo.
Questo ci porta a riscrivere la legge di Foulier in questo modo:
\begin{gather*}
    \boxed{q(x) = -k \frac{\partial T}{\partial x}}
\end{gather*}
Consideriamo a questo punto un volume del nostro sistema:
\begin{center}
    \begin{tikzpicture}
        \draw(0,-2) -- (0,1) node[at start, left]{$x$};
        \draw(0.5,-2) -- (0.5,1) node[at start, right]{$x+dx$};
        \draw[->](0,0) -- (3,0) node[at end]{$x$};
        \draw(0,-0.5) -- (0.5,-0.5);
        \draw(0,-1) -- (0.5,-1);
        \node at(0.25,-0.8){$dV$};
        \draw[red](0.5,-0.5) --(0.5,-1)node[at end, below, right]{$dS$};
    \end{tikzpicture}
\end{center}
\begin{gather*}
    \delta Q = -q_{\text{tot}} dS \ dt\\
    \delta Q = \mathcal{C}_p dt\\
    \mathcal{C}_p dT= -q_{\text{tot}} dS \ dt\\
    -q_{\text{tot}} dS = \mathcal{C}_p \frac{\partial T}{\partial t}
\end{gather*}
Dove $q_{\text{tot}}$ è il flusso su tutte le pareti.
\begin{gather*}
    -q_{\text{tot}} dS = \rho dV \mathcal{C}_p \frac{\partial T}{\partial t}\\
     = \rho \mathcal{C}_p \frac{\partial T}{\partial t} dx \ dS\\
    -q_{\text{tot}} dS = \rho \mathcal{C}_p \frac{\partial T}{\partial t} dx \ dS
\end{gather*}
\begin{gather*}
    -q_{\text{tot}} = -\left( q(x+dx)-q(x) \right) \\
    = - \left( q(x) +\frac{\partial q}{\partial x} dx - q(x) \right) \\
    = -\frac{\partial q }{\partial x} dx
\end{gather*}
\begin{gather*}
    -q_{\text{tot}} dS = -\frac{\partial q}{\partial x} dx dS
\end{gather*}
\begin{gather*}
    -\frac{\partial q}{\partial x} \cancel{dx \ dS} = \mathcal{C}_p \rho \frac{\partial T}{\partial t} \cancel{dx \ dS}\\
    -q = k \frac{\partial T}{\partial x}\\
    -\frac{\partial q}{\partial x} = k \frac{\partial ^2 T}{\partial x^2}
\end{gather*}
\begin{gather*}
    \mathcal{C}_p \rho \frac{\partial T}{\partial t} = k \frac{\partial^2 T}{\partial x^2}
\end{gather*}
Definiamo ora il \textbf{coefficiente di diffusione termica}
\begin{gather*}
    D_T = k \frac{j}{\mathcal{C}_p \rho}
\end{gather*}
Possiamo così ricavare l'equazione di diffusione:
\begin{gather*}
    \boxed{\frac{\partial T(x,t)}{\partial t}= D_t \frac{\partial^2 T(x,t)}{\partial x^2}}
\end{gather*}
Siamo nel caso in cui si deve verificare:
\begin{gather*}
    \frac{d^2T}{d x^2} = 0
\end{gather*}
E questo si ha solo nelle relazioni lineari, da questo ragionamento possiamo ricavare che la relazione tra la temperatura $T$ e posizione $x$:
\begin{gather*}
    T(x) = T_1 + \frac{T_2-T_1}{l}x
\end{gather*}
Se io in generale ho una derivata seconda uguale a qualcosa, avro un caso in cui se ho una soluzione di $f(t)$ avrò anche una soluzione $f(-t)$. E questo mi dice che il fenomeno descritto dalla mia equazione è reversibile.\\
Questo si vede anche in $F=ma$ che è reversibile, infatti se vedo fenomeni fisici se qualcuno non mi conferma che sia un filmato al contrario io non vedo fenomeni non plausibili.\\\\
Una soluzione possibile è:
\begin{gather*}
    T(x,t) = (4 \pi D_T t )^{-\frac{1}{2}} e ^{-\frac{x^2}{4 D_T t}}
\end{gather*}
Questa si vede essere una gaussiana con $\sigma \propto \sqrt{t}$\\
Quindi è una gaussiana sempre più stretta, è di fatto una delta di Dirac $T(x,0) = \delta(x)$

\begin{center}
    \begin{tikzpicture}
\begin{axis}[
    view={135}{30}, % Angolo di visualizzazione 3D
    xlabel=$x$,
    ylabel=$t$,
    zlabel={$T(x,t)$},
    domain=-5:5, % Intervallo per x
    yticklabel style={/pgf/number format/fixed},
    xticklabel style={/pgf/number format/fixed},
    zticklabel style={/pgf/number format/fixed},
    ymin=0.01, % Evita t=0 per evitare divisioni per zero e logaritmi
]

% Definizione della funzione
\addplot3 [
    surf, % Tipo di grafico: superficie
    domain=-5:5, % Intervallo di x per la superficie
    y domain=0.01:5, % Intervallo di t per la superficie
    samples=50, % Numero di campioni per una migliore visualizzazione
    samples y=25,
    point meta=z, % Mappa il colore della superficie in base al valore di z
] {
    (4*pi*1.0*x)*% Definisci un valore arbitrario per D_T (qui 1.0, da adattare)
    (x-x)*% Definisci un valore arbitrario per D_T (qui 1.0, da adattare)
    exp(-(x^2)/(4*1.0*x)) % Definisci un valore arbitrario per D_T (qui 1.0, da adattare)
};

% Puoi anche aggiungere un grafico a linea per un valore fisso di t (ad esempio t=1)
\addplot3 [
    domain=-5:5,
    samples=100,
    blue, % Colore del grafico
    thick,
    samples y=1, % Disegna solo una linea
] {
    (4*pi*1.0*1)^(-0.5)* % Definisci D_T come sopra
    exp(-(x^2)/(4*1.0*1))
};
\end{axis}
\end{tikzpicture}
\end{center}

\newpage
\section{24/10/25}
Passiamo al caso realistico effettivo:
\begin{gather*}
    q(\vv{r},t) = -k \nabla T \cdot \vv{n}
\end{gather*}
Con $\vv{n}$ la direzione normale che stiamo considerando.\\
Di fatto questa è la legge di Foulier generalizzata in tre dimensioni.\\
Si ricorda che il gradiente in coordinate cartesiane è:
\begin{gather*}
    \nabla = \frac{\partial}{\partial x} \widehat{i} +\frac{\partial}{\partial y} \widehat{j} + \frac{\partial}{\partial z} \widehat{k}
\end{gather*}
Passando ora a l'ultima forma vista:
\begin{gather*}
    \frac{\partial T}{\partial t} = D_T \nabla^2 T
\end{gather*}
Si ricorda sempre:
\begin{gather*}
    \nabla^2 = \frac{\partial^2}{\partial x^2} + \frac{\partial^2}{\partial y^2} + \frac{\partial^2}{\partial z^2}
\end{gather*}
Quando parliamo in generale di trafsormazione termodinamiche distinguiamo tra trasformazioni generiche e quasi statiche, in più dovremmo introdurre la distinzione tra trasformazioni reversibili e irreversibili.\\
\subsection{trasformazioni termodinamiche reversibili e irreversibili}
Se ho un sistema ad uno stato iniziale $i$ e dopo la trasfromazione arrivo ad uno stato finale $f$, in generale non è sufficiente dire che se è possibile andare dallo stato finale $f$ allo stato iniziale $i$ perchè la trasforamzione sia reversibile, ma anche l'interazione con l'ambiente deve essere invertita.\\
Uno buon modo per definire questa cosa è utilizzare il lavoro ovvero se la mia trasformazione $i \to f$ fa un lavoro $L$, se è reversibile allora facendo $f \to i$ farà un lavoro $-L$.\\
Nel caso reale ovviamente non è possibile farlo in maniera perfetta ma ci potremo avvicinare molto.\\\\
Parliamo ora delle trasformazioni irreversibili, che generalmente sono tutte le trasfromazioni trovate in natura.\\
Prendiamo un esempio in cui abbiamo due sistemi a contatto con una parete diatermica:
\begin{center}
    \begin{tikzpicture}
        \draw(0,0) rectangle(2,2) node[midway]{$T_1$};
        \draw(2,0) rectangle(4,2) node[midway]{$T_2$};
        \draw[red](2,-0.5) -- (2,2.5);
        \draw[->,thick](4.5,1) -- (5.5,1) node[midway,above]{tempo...};
        \draw(6,0) rectangle(8,2) node[midway]{$T$};
        \draw(8,0) rectangle(10,2) node[midway]{$T$};
        \draw[red,thick](8,-0.5) -- (8,2.5);
    \end{tikzpicture}
\end{center}

\begin{center}
            \begin{tikzpicture}
                \fill[pattern=north east lines](0,0)rectangle(2,1);
                \filldraw[fill=page] (0.2,0.2) rectangle (1.8,0.8);
                \draw(0,0) rectangle (2,1);
                \fill[fill=cyan] (0.2,0.2) rectangle (1,0.8);
                \draw[->](2.5,0.5) -- (3.5, 0.5);
                \fill[pattern=north east lines](4,0)rectangle(6,1);
                \draw(4,0) rectangle (6,1);
                \filldraw[fill=cyan] (4.2,0.2) rectangle (5.8,0.8);
                \node at(0,-0.5){$V_i,T$};
                \node at(4,-0.5){$V_f,T$};
            \end{tikzpicture}
\end{center}
Questi sono esempi di trasformazioni non reversibili che però non sono quasi statiche, questo ci pone 
a chiederci: se una trasformazione non è quasi statica è reversibile? In generale è così quindi si dice che 
l'essere quasi statica è una condizione necessaria per essere reversibile, non è un qualcosa di logicamente provato ma in generale 
non si riescono a pensare modi per fare una trasformazione reversibile che non sia quasi statica.Ma è anche sufficiente? In altre 
parole tutte le trasformazioni quasi-statiche sono reversibili? Questo non è vero e prendiamo appunto un esempio:
\begin{center}
            \begin{tikzpicture}
                \fill[pattern=north east lines](0,0)rectangle(2,1);
                \draw(0.2,0.2) rectangle (1,0.8);
                \draw(1,0.2) rectangle (1.2,0.8);
                \filldraw[fill=page] (1.2,0.2) rectangle (2,0.8);
                \draw(0,0) rectangle (2,1);
                \fill[fill=page] (1.95,0.2) rectangle (2.05,0.8);
                \fill[fill=cyan] (0.2,0.2) rectangle (1,0.8);
                \draw(1.2,0.5) -- (2.5,0.5);
                %\draw[->](2.5,0.5) -- (3.5, 0.5);
                \node at(0,-0.5){$V_A$};
                \filldraw[pattern=north east lines](2.5,0.4)rectangle(2.8,0.5);
                \draw(2.3,0.2) rectangle(3,0.4) node[at end, right]{$T$};
            \end{tikzpicture}
\end{center}
 
\subsection{Enunciato di Kelvin-Planck (KP)}
\underbar{NON} è possibile realizzare una trasforamzione termodinamica il cui \underbar{unico} risultato sia la \underbar{completa} trasforamzione $Q$ (una certa quantità di calore) in laboro meccanico $L>0$ quando questa quantità di calore $Q$ è assorbita da \underbar{1 solo} termostato $T$.\\\\
Il dispositivo che stiamo descrivendo è:
\begin{center}
    \begin{tikzpicture}
        \draw(0,0) rectangle (2,1) node[midway]{$T$};
        \draw[->](1,0) -- (1,-0.5) node[midway, right]{$Q$};
        \draw(0.5,-1) -- (1.5,-1) -- (1,-0.5) -- (0.5,-1);
        \node at(2,-1){$L>0$};
    \end{tikzpicture}    
\end{center}
e questo \textbf{non} esiste.
\subsection{Enunciato di Clausius (C)}
\underbar{NON} è possibile avere una trasformazione termodinamica il cui \underbar{unico} risultato sia il passaggio di $Q$ da un termostato $T_B$ a $T_A$ se $T_A > T_B$.

\hfil\\
Questo sono due enunciati del \textbf{IL II PRINCIPIO DELLA TERMODINAMICA}.\\
Dobbiamo quindi dimostrare che siano equivalenti.
\begin{gather*}
    KP \Leftrightarrow C
\end{gather*}
Cominciamo a formulare una strategia,, intanto sono entrambe affermazioni per impossibilità.
Per fare questo possiamo ragionare al contrario e sapere che la negazione di $KP$ coimplica la negazione di $C$.
\begin{gather*}
    \sim KP \Leftrightarrow \sim C
\end{gather*}
Incominciamo con $\sim KP \Rightarrow \sim C$
Per la parte $\sim C \Rightarrow \sim KP$ serve introdurre degli oggetti chiamati macchine termiche.
\subsection{macchine termiche}
È un dispositivo che compie una trasformazione termodinamica scambiando energia con in maniera termica con dei termostati e nel fare questo compie un lavoro.\\
Un esempio è la macchina di Fermi, qindi un termostato che scalda un gas che fa un lavoro.\\
In generale se $L\geq 0$ si chiamano macchine termiche ma se $L < 0$ si chiamano macchine frigorifere.\\
L'esempio di fermi è una macchina termica a tutti gli effetti ma non ha una caratteristica che rende interessanti le macchine termiche ordinarie, ovvero che oltre a fare del lavoro sono \underbar{cicliche}, che rendono il lo strumento generalmente utile.\\
Se aggiungiamo anche la condizione della reversibilità allora si dicono macchina reversibili (ovviamente siamo nel limite ideale).
\begin{enumerate}[$i)$]
    \item $Q_1 < 0, Q_2 < 0$
    \item $Q_1 > 0, Q_2 > 0$
    \item $Q_1 < 0, Q_2 > 0$
    \item $Q_1 > 0, Q_2 > 0$
\end{enumerate}
Posso iniziare a fare alcune considerazioni per escludere alcuni casi:
\begin{gather*}
    Q = Q_1+Q_2 = L>0
\end{gather*}
Questo ci esclude il caso $i)$\\
Considerando  caso $iii)$ si puo vedere un sottocaso però irrilevante che è:
\begin{gather*}
    |Q_1| > |Q_2|
\end{gather*}
In conclusione si ha che l'unico caso plausibile è il caso $iv$

\newpage
\section{28/10/25}
La scorsa volta abbiamo dimostrato che:
\begin{gather*}
    \sim KP \Rightarrow \sim C\\
\end{gather*}
Ora puntiamo a dimostrare:
\begin{gather*}
    \sim C \Rightarrow \sim KP
\end{gather*}
\begin{center}
    \begin{tikzpicture}
        \draw(0,0) rectangle (2,1) node[midway]{$T_1$};
        \draw[->](0.5, -1) -- (0.5,0) node[midway, left]{$-Q$};
        \draw[->](1.5,0) -- (1.5, -1) node[midway, right]{$Q$};;
        \draw(0.2,-1.4) rectangle (0.9, -1) node[midway]{$\sim C$};
        \draw(1.5,-1.25) circle (6pt);
        \draw[->](0.5, -2) -- (0.5,-1.5) node[midway, left]{$Q$};
        \draw[->](1.5,-1.5) -- (1.5, -2) node[midway, right]{$Q_2$};
        \draw(0,-3) rectangle (2,-2) node[midway]{$T_1$};
        \node at(1.5,-1.25) {$M$};
    \end{tikzpicture}
\end{center}
Con:
\begin{gather*}
    T_1>T_2\\
    L = Q + Q_2
\end{gather*}
E ricordando che $M$ è una macchina ciclica.\\
Da questo si concolude la dimostrazione per assurdo e posso dire:
\begin{gather*}
    \boxed{CP \Leftrightarrow C}
\end{gather*}

\subsection{Considerazioni con n termostati}
Come abbiamo dimostrato è che non è possibile usare due termostati a temperature diverse e avere due quantità diverse di calore non concordi positivi,poichè ci si può rincondurre
 al caso ad un termostato il che viola il secondo principio, lo stesso ragionamento vale per $n$ termostati.\\
Dunque se ho
\begin{gather*}
    T_1,T_2, \dots , T_N    
\end{gather*}
Avrò delle relative temperature di calore:
\begin{gather*}
    Q_1,Q_2, \dots , Q_N
\end{gather*}
E devono essere alcuni positivi e alcuni negativi (di nuovo per non violare il secondo principio):
\begin{gather*}
    Q_1 , \dots , Q_j > 0\\
    Q_{j+1} , \dots , Q_N < 0\\
    Q_a = \sum_{i=1}^{j} Q_i\\
    Q_b = \sum_{i = j+1}^{N} Q_i
\end{gather*}
DOve possiamo identificare:
\begin{gather*}
    Q_a > 0\\
    Q_c < 0
\end{gather*}

\subsection{rendimento (o efficienza)}
è definito come:
\begin{gather*}
    \eta = \frac{L}{Q_a}
\end{gather*}
Consideriamo macchine cicliche, nelle quali valgono:
\begin{gather*}
    \eta = \frac{L}{Q_a} = \frac{Q}{Q_a} = \frac{Q_a + Q_c}{Q_a} = 1+ \frac{Q_c}{Q_a} = 1-\frac{|Q_c|}{Q_a}
\end{gather*}
Con $Q_a$ che è il calore acquisito e $Q_c$ è il calore ceduto.\\
Da queste considerazioni possiamo dire che:
\begin{gather*}
    0 \leq \eta < 1
\end{gather*}
Il caso $\eta = 0$ è perfettamente valido ma poco interessante perchè si verifica solo se $L=0$\\\\

Questo è cioò che possiamo dire con la sola ipotesi di macchina termica coclica, ora procediamo ad aggiungere ipotesi:\\
Consideriamo ora macchine termiche che scambiano calore con due soli termostati (ovviamente a temperature diverse):
\begin{gather*}
    T_1 \qquad T_1
\end{gather*}
\begin{theorem}[di Carnot]
    fra tutte le macchine termiche cicliche che laborano fra $T_1$ e $T_2$ assgnate\\
    si ha che:
    \begin{gather*}
        \eta_R \geq \eta
    \end{gather*}
    Che vuol dire che il rendimento delle macchine reversibili hanno rendimento maggiore di quelle non reversibili.
    \begin{gather*}
        \eta_{R'} = \eta_R
    \end{gather*}
    Che vuol dire che tutte le macchine reversibili hanno lo stesso rendimento.
\end{theorem}
Facciamo attenzione a come è posta la frase:\\
Se non poniamo fra $T_1$ e $T_2$, potrei creare macchine com piccolo rendimento ma reversibili, se io non fisso le temperature vorrebbe dire che tutte le altre macchine hanno rendimento 0, poichè sto maggiorando un numero con un numero arbitrariamente piccolo.\\\\
Dimostriamo ora la prima parte quindi date due temperature assegnate sappiamo che $\eta_R \geq \eta$:
\begin{proof}
    \begin{center}
    \begin{tikzpicture}
        \draw(0,0) rectangle (2,1) node[midway]{$T_1$};
        \draw[->](0.5, -1) -- (0.5,0) node[midway, left]{$Q_{1R}$};
        \draw[->](1.5,0) -- (1.5, -1) node[midway, right]{$Q_{1M}$};;
        \draw(1.5,-1.25) circle (6pt);
        \draw(0.5,-1.25) circle (6pt);
        \draw[->](0.5, -2) -- (0.5,-1.5) node[midway, left]{$Q_{2R}$};
        \draw[->](1.5,-1.5) -- (1.5, -2) node[midway, right]{$Q_{2M}$};
        \draw(0,-3) rectangle (2,-2) node[midway]{$T_1$};
        \node at(1.5,-1.25) {$M$};
        \node at(0.5,-1.25) {$R$};
        \draw[->](0.8,-1.25) -- (1.3,-1.25) node[midway, above]{$L_R$};
        \draw[->](1.8,-1.25) -- (2.3,-1.25) node[midway, above]{$L_M$};
    \end{tikzpicture}
    \end{center}
    Considero $N_R$ cicli e $N_M$ cicli, e avrò di conseguenza
    \begin{gather*}
        Q_{1R}' = N_R Q_{1R} L_R' = N_R L_R Q_{2R}' = N_L Q_{2R}
        Q_{1M}' = N_M Q_{1M} L_M' = N_M L_M Q_{2M}' = N_M Q_{2M}
    \end{gather*}
    Ora scegliendo opportunamente $N_R$ e $N_M$
    \begin{gather*}
        Q_{1R}' = N_R Q_{1R} = N_M Q_{1M} = Q_{1M}'
    \end{gather*}
    Purtroppo non posso farlo uguale, poichè $N_R$ e $N_M$ sono interi e devo avere un numero reale, ma mi basta poichè è approssimabile a piacere, e quindi è preciso a piacere.\\
    Vediamo ora cosa succede al rendimento:
    \begin{gather*}
        \eta_M' = \frac{L_M'}{ Q_{1M}'} = \frac{N_M L_M}{N_M Q_{1M}} = \frac{L_M}{Q_{1M}} = \eta_M\\
        \text{analogamente... } \eta_R' = \eta_R
    \end{gather*}
    In questo modo ottengo:
    \begin{center}
    \begin{tikzpicture}
        \draw(0,0) rectangle (2,1) node[midway]{$T_1$};
        \draw[->](0.5, -1) -- (0.5,0) node[midway, left]{$-Q_1$};
        \draw[->](1.5,0) -- (1.5, -1) node[midway, right]{$Q_1$};;
        \draw(1.5,-1.25) circle (6pt);
        \draw(0.5,-1.25) circle (6pt);
        \draw[->](0.5, -2) -- (0.5,-1.5) node[midway, left]{$-Q_{2R}$};
        \draw[->](1.5,-1.5) -- (1.5, -2) node[midway, right]{$Q_{2M}$};
        \draw(0,-3) rectangle (2,-2) node[midway]{$T_1$};
        \node at(1.5,-1.25) {$M$};
        \node at(0.5,-1.25) {$-R$};
        \draw[->](1.3,-1.25) -- (0.8,-1.25) node[midway, above]{$-L_R$};
        \draw[->](1.8,-1.25) -- (2.3,-1.25) node[midway, above]{$L_M$};
    \end{tikzpicture}
    \end{center}
    Ora ho una situazione in cui ho la mia macchina termica iniziale è uguale ma la macchina $R$ la ho girata ( e lo posso fare perchè è reversibile per hp.)\dots
    \begin{center}
        \begin{tikzpicture}
            \draw(0,0) rectangle (3,1) node[midway]{$T_2$};
            \draw[->](1.5,1) -- (1.5, 2) node[midway, right]{$Q_{2M} - Q_{2R}$};
            \draw(0.5,2) rectangle (2.5,3) node[midway]{$M - R$};
            \draw(2.5, 2.5) -- (4, 2.5) node[midway, above]{$L_M - L_R$};
        \end{tikzpicture}
    \end{center} 
    Di conseguenza si può vedere che $L_M - L_R \leq 0$, perchè dalla schematizzazione precedente si vede che le due macchine termiche non interagiscono più con il termostato $T_1$ e quindi ho la situazione in cui ho una macchina termica che cede calore ad un termostato è compie un lavoro, di conseguenza qusto lavoro sarà per forza negativo.\\
    E con questo si dice:
    \begin{gather*}
        L_M \leq L_R\\
        \frac{L_M}{Q_1} \leq \frac{L_R}{Q_1}\\
        \eta_M \leq \eta_R
    \end{gather*}
    \hfil\\
    Per dimostrare che $\eta_M = \eta_R$, si assume che anche la macchina $M$ sia reversibile, e posso ripetere il ragionamento che abbiamo fatto in precedenza, invertendo il procedimento di ogni macchina una alla volta ottengo:
    \begin{gather*}
        \eta_M \geq  \eta_R\\
        \eta_M \leq \eta_R
    \end{gather*}
    E valgono entrambe quindi necessariamente:
    \begin{gather*}
        \eta_M = \eta_R
    \end{gather*}
\end{proof}

Facciamo ora un esempio esplicito di una macchina termica reversibile ciclica a due temperature fissate $T_1$ e $T_2$ generiche, in questo modo possiamo sapere il rendimento di ogni MT rev. cilcica tra due temp generiche.\\
In primis consideriamo un gas perfetto e dobbiamo fargli fare una trasfromazione che torna al punto di partenza , può essere reversibile e deve fare lavoro positivo:\\
Consideriamo un caso particolare di ciclo detto ciclo di Carnot
\begin{multicols}{2}
    \noindent
    \begin{center}
        \begin{tikzpicture}
            \draw[->](0,0) -- (0,3) node[at end, left]{$p$};
            \draw[->](0,0) -- (3,0) node[at end, below]{$V$};
            \draw[->,orange](0.5, 2) ..controls(0.7,1.2).. (1.5,1);
            \draw[->,orange](1.5, 1) ..controls(1.7,0.4).. (2.5,0.2);
            \draw[->,orange](2.5,0.2) ..controls(0.8,0.4).. (0.5, 2);
            \node at(0.5,2)[above]{$A$};
            \node at(1.5,1)[above]{$B$};
            \filldraw[fill=black] (1,0.6) circle (1pt) node[below, left]{$D$};
        \end{tikzpicture}
    \end{center}
    \columnbreak
    Ha le seguenti caratteristiche:
    \begin{gather*}
        AB \quad T_1 \text{cost.}\\
        BC \quad \text{adiabatica}\\
        CD \quad T_2 \text{cost.}\\
        DA \quad \text{adiabatica}\\
    \end{gather*}
\end{multicols}
Consideriamo innanzitutto:
\begin{multicols}{2}
    \noindent
    \begin{gather*}
        U(A) = U(B)\\
        U(C) = U(D)
    \end{gather*}
    \columnbreak
    \begin{gather*}
        Q_1 = L_{AB} = \mathcal{N} RT_1 \ln(\frac{V_B}{V_A})\\
        Q_2 = L_{CD} = \mathcal{N} RT_2 \ln(\frac{V_C}{V_D})
    \end{gather*}
\end{multicols}
Sappiamo la formula generale per il rendimento, e anche altre considerazioni:
\begin{gather*}
    \eta = 1+ \frac{Q_2}{Q_1} \qquad \boxed{*}\\
    TV^{\gamma-1} = \text{cost.}\\
    \frac{Q_2}{Q_1} = \frac{T_2}{T_1}\frac{\ln(\frac{V_D}{VC})}{\ln(\frac{V_B}{V_A})}\\
    T_1 V_A^{\gamma-1} = T_2 V_D^{\gamma-1}\\
    T_1 V_B^{\gamma-1} = T_2 V_C^{\gamma-1}
\end{gather*}
Di conseguenza:
\begin{gather*}
    \left(\frac{V_A}{V_B}\right)^{\gamma-1} = \left( \frac{V_D}{V_C} \right)^{\gamma-1}\\
    \frac{V_A}{V_B} = \frac{V_D}{V_C}\\
    -\ln(\frac{V_A}{V_B}) = \ln(\frac{V_D}{V_C})\\
    \frac{Q_2}{Q_1} = \frac{T_2}{T_1}
\end{gather*}
Sostituendo nella formula iniziale si ottiene:
\begin{gather*}
    \boxed{\eta_R = 1 - \frac{T_2}{T_1}}
\end{gather*}
considerando quest'ultima e la equazione segnata \fbox{*} si arriva a dire che:
\begin{gather*}
    \frac{Q_2}{Q_1} = -\frac{T_2}{T_1}
\end{gather*}
E di conseguenza so che è vero anche:
\begin{gather*}
    \frac{|Q_2|}{|Q_1|} = \frac{T_2}{T_1}
\end{gather*}
In generale un termometro a gas perfetto non può misurare sotto la temperatura critica del materiale con cui lo ho concepito, quindi il vantaggio grosso è che la definizione della macchina termica non ha questo limite poichè "l'unica" cosa che richiede è di avere trasformazioni cicliche reversibili.\\
Posso generalizzare questo concetto cambiando la temperatura con il termine $\theta$
\begin{gather*}
    \theta = \theta_0 \frac{|Q|}{|Q_0|}
\end{gather*}
E quindi so che:
\begin{gather*}
    \eta = 1- \frac{|Q_2|}{|Q_1|} = 1- \frac{Q_2}{Q_1} \\
    \frac{\theta_2}{\theta_1} = \frac{T_2}{T_1}\\
\end{gather*}
E quindi si vede che la temperatura 0 non può essere raggiunta

\section{30/10/25}
\subsection{sistema ciclico}
Prendiamo come unica premessa un sistema termodinamico ciclico e che possa scambiare calore con un certo numero di termostati.\\
\begin{gather*}
    \sum_{i=1}^{N} \frac{Q_i}{T_i} \leq 0 \quad = 0 \text{reversibile}
\end{gather*}
Ma se ho una quantità $Q_i$ la posso pensare come:
\begin{gather*}
    Q_i = \sum_{k=1}^{M_i} \delta Q_k^{(i)}
\end{gather*}
Di conseguenza avrò che la somma precedente diventerà:
\begin{gather*}
    \sum_{i=1}^{N} \frac{Q_i}{T_i} = \sum_{i=1}^{N} \frac{1}{T_i} \sum_{k=1}^{M_i} \delta Q_k^{(i)} = \sum_{i=1}^{N} \sum_{k=1}^{M_i} \frac{\delta Q_k^{(i)}}{T_i}
\end{gather*}
Ora faccio in modo che:
\begin{gather*}
    M_i \to \infty\\
    \delta Q_k^{(i)} \to 0
\end{gather*}
Il che mi permette di scrivere:
\begin{gather*}
    \sum_{i=1}^{N} \int_{i} \frac{\delta Q^{(i)}}{T_i}
\end{gather*}
Se ho un numero finito di termostati (come in questo caso) posso scrivere:
\begin{gather*}
    \sum_{i=1}^{N} \frac{Q_i}{T_i} = \oint \frac{\delta Q}{T}
\end{gather*}
L'integrale $\oint$ è per ricordarmi che alla fine delle somme avrò qualcosa di bellino(bho idk).\\
Possiamo ora prendere questa relazione e mandare il numero di termostati all'infinito, ma devo ricordarmi che sto simulando uninterazione con l'ambiente che scambia una quantità finita di calore, quindi vuol dire nel pratico che ho  a disposizione una quantità continua di termostati:
\begin{gather*}
    \lim_{N \to \infty}\\
    Q_i \to \delta Q_i \to 0
\end{gather*}
E quindi il tutto diventa:
\begin{gather*}
    \lim{\underset{\delta Q_i \to 0}{N \to \infty}} \sum_{i=1}^{N} \frac{\delta Q_i}{T_i} = \oint \frac{\delta Q}{T}
\end{gather*}
Grazie al teorema di Clausius possiamo dire:
\begin{gather*}
    \oint \frac{\delta Q}{T} \leq 0\\
\end{gather*}
e nel caso reversibile:
\begin{gather*}
    \oint \left( \frac{\delta Q}{T} \right)_\text{REV} = 0 
\end{gather*}
Questa è anche detta \textbf{disuguaglianza di Clausius}, ed è una diretta conseguenza del secondo principio (sena di esso non sarebbe possibile scriverlo)\\
Si deve ricordare che questo non è un integrale di una funzione ad una variabile, è un modo compatto per scrivere la somma di termostati  per la quantità di calore scambiato diviso per la temperatura alla quale sono stati scambiati. Si mette l'integrale al posto della somma per tenere conto della possibilità della somma di tanti pezzi infinitesimi.
\subsection{entropia}
"foto"\\
la notazione di integrale è proprio la somma tra il segmento $AB$ e il segmento $BA$:
\begin{gather*}
    0 = \int_{A \ \gamma_1}^{B} \left(\frac{\delta Q}{T}\right)_\text{REV} + \int_{B \ \gamma_2}^{A} \left(\frac{\delta Q}{T}\right)_\text{REV} 
\end{gather*}
di conseguenza:
\begin{gather*}
    \int_{A \ \gamma_1}^{B} \left(\frac{\delta Q}{T}\right)_\text{REV} = -\int_{B \ \gamma_2}^{A} \left(\frac{\delta Q}{T}\right)_\text{REV}
\end{gather*}
In generale non è vero che se sommo una quantità lungo un percorso e poi sommo la stessa quantità con segno opposto lungo la direzione opposta dello stesso percorso, io ottenga 0. Però in questo caso lo è per via dell'ipotesi di reversibilità. E grazie a questa (e solo per questo) posso scrivere:
\begin{gather*}
    \int_{A \ \gamma_1}^{B} \left(\frac{\delta Q}{T}\right)_\text{REV} = -\int_{B \ \gamma_2}^{A} \left(\frac{\delta Q}{T}\right)_\text{REV} = \int_{A \ \gamma_2}^{B} \left( \frac{\delta Q}{T} \right)_\text{REV}
\end{gather*}
Ora consideriamo l'ultimo integrale e la pensiamo come una funzione:
\begin{gather*}
    \int_{A \ \gamma_2}^{B} \left( \frac{\delta Q}{T} \right)_\text{REV} = f(A,B)\\
    f(A,A) = 0
\end{gather*}
Questo ci permette di dire che presa una funzione di stato $S$:
\begin{gather*}
    \boxed{S(B) - S(A) = \int_{A}^{B} \left( \frac{\delta Q}{T} \right)_\text{REV} }
\end{gather*}
Questa è detta entropia (la funzione $S$ è proprio la funzione entropia, ed è una funzione di stato).\\
Anche questo viene direttamente dal secondo principio e la si puo definire come appunto la differenza tra le funzioni di stato nel punto $A$ e $B$ ma non è una proprietà della trasformazione.\\
Si ricorda di nuovo che questo vale solo nellipotesi di reversibilità.\\
L'entropia è sia estensiva si additiva.
\subsection{versione differenziale}
Se ora considero le due funzioni di stato infinitesimamente vicini tra di loro?
\begin{gather*}
    dS = \frac{\delta Q_\text{REV}}{T}
\end{gather*}
Elaboriamo, se abbiamo una quantità di calore reversibile NON è infinitesimo, ma quando la dividiamo per la temperatura alla quale è stata scambiata lo diventa.\\
In generale se qualcosa non è n differenziale ma si moltiplica per qualcosa lo diventa, quel qualcosa è detto fattore integrante.\\\\
Il fatto che sia reversibile è impattante siccome siamo nel caso infinitesimo? in generale è importante poichè se fosse vero che non dobbiamo sottolineare il fatto che è reversibile, potrei dimostrare questo senza l'utilizzo dell'ipotesi reversibilità.\\
In generale uno scambio di calroe è reversibile quando si scambia temperatura alla stessa temperatura del termostato.\\
PER NOTAZIONE poi si omette \textbf{REV} poichè altrimento non avrebbe senso, ma è estremamente importante.\\
(il fatto che la trasformazione sia quasi statica è sottointeso), ma abbiamo solo supposto che esista una trasformazione adiabatica che connetta uno stato con un'altro e non che sia reversibile, di fatto che sia adiabatica e quindi quasi statica lo implica.\\\\

Cerchiamo ora di togliere l'ambiguità dietro ad una trasformazione reversibile grazie alla definizione di entropia, in maniera aritmetica (facendo un calcolo o una misura):
\begin{center}
    \begin{tikzpicture}
        \draw[dashed](0,0) ..controls(1.5,3.8).. (3,3) node[midway, above]{1};
        \draw[->](3,3) ..controls(1,1.5).. (0,0) node[midway, below]{2};
        \filldraw[fill=black](0,0) circle (1pt) node[left]{$A$};
        \filldraw[fill=black](3,3) circle (1pt) node[right]{$B$};
    \end{tikzpicture}
\end{center}
\begin{gather*}
    A_1 B\\
    A_2 B \quad \text{REV}\\
    A_1 B + B_2 A = \text{ciclo}\\
    _1\int_{A}^{B} \frac{\delta Q}{T} + _2\int_{B}^{A} \left( \frac{\delta Q}{T} \right)_\text{REV} \leq 0\\
    _1\int_{A}^{B} \frac{\delta Q}{T} - _2\int_{A}^{B} \left( \frac{\delta Q}{T} \right)_\text{REV} \leq 0\\
\end{gather*}
Da questo si ottiene
\begin{gather*}
    S(B) - S(A) \geq _1\int_{A}^{B} \frac{\delta Q}{T} = I_{AB}
\end{gather*}
Possiamo fare delle considerazioni:
\begin{gather*}
    \Delta S > I_{AB} \Rightarrow A_!B \quad \text{irreversibile}
    \Delta S > I_{AB} \Rightarrow A_!B \quad \text{reversibile}
\end{gather*}
Nel caso \textbf{termicamente isolato}
\begin{gather*}
    S(f) - S(i) \geq 0
\end{gather*}
Si possono fare o trasformazioni che non cambiano entropia ma allora sono reversibili.\\
oppure che cambiano entropia si deve scambiare calore con l'ambiente.\\
Questa cosa è detta come principio dell'entropia.\\
\begin{gather*}
    \boxed{\Delta S \geq 0 } \text{ se } Q \equiv = 0
\end{gather*}
Di fatto si sa che se un sistema ha una certa entropia questa potrà solo crescere\\
Il secondo principio è enunciabile con questo principio. 

\newpage
\section{31/10/25}
\begin{gather*}
    \Delta S \geq 0
\end{gather*}
Se il sistema che stiamo considerando è termicamente isolato\\
Si ricorda che in gergo termodinamico $S+A = U$ (con $U$ inteso come "universo") e quindi:
\begin{gather*}
    \Delta S_U \geq 0
\end{gather*}
In generale non sappiamo se l'entropia universale cresca, o meglio supponiamo che lo sia ed è del tutto ragionevole pensarlo ma non lo sappiamo da queste definizioni.\\
In breve non può essere presa come implicazione del secondo principio della termodinamica.\\
Ora ragioniamo sul fatto che l'entropia complessiva è data da:
\begin{gather*}
    \Delta S = \Delta S_S + \Delta S_A
\end{gather*}
Si può quindi ridurre l'entropia del sistema o dell'ambiente MA facendo così l'altra aumenterà sempre.\\
\subsection{robe}
Ricordiamo la definizione di entropia:
\begin{gather*}
    \Delta S = S(B) - S(A) = \int_{A}^{B} \left( \frac{\delta Q}{T} \right)_\text{REV} 
\end{gather*}
Ricordiamo inoltre dalla disuguaglianza di Clausius sappiamo che:
\begin{gather*}
    I_{AB} = \int_{A}^{B} \frac{\delta Q}{T}
\end{gather*}
Ho di conseguenza i seguenti casi:
\begin{gather*}
    \Delta S > I_{AB} \Rightarrow A \to B \text{irrev.}\\
    \Delta S = I_{AB} \Rightarrow A \to B \text{rev.}
\end{gather*}
Prendiamo due stati termodinamici $A$ e $B$ connessi da una trasformazione adiabatica reversibile:
\begin{center}
    \begin{tikzpicture}
        \draw(0,0) ..controls(1,0.7).. (2,0.5) node[midway, above]{adiab. rev.};
        \node at(0,0)[left]{$A$};
        \node at(2,0.5)[right]{$B$};
    \end{tikzpicture}
\end{center}
Allora avrò che:
\begin{gather*}
    \Delta S = 0
\end{gather*}
Questo vale per tutti i tipi di gas (non solo gas perfetto)\\\\
Prendiamo ora il caso in cui sono connessi da una isoterma reversibile
\begin{center}
    \begin{tikzpicture}
        \draw(0,0) ..controls(1,0.7).. (2,0.5) node[midway, above]{isoterma rev.};
        \node at(0,0)[left]{$A$};
        \node at(2,0.5)[right]{$B$};
    \end{tikzpicture}
\end{center}
Questo vuol dire che si puo usare l'isoterma per calcolare l'entropia:
\begin{gather*}
    \Delta S = S(B) - S(A) = \int_{A}^{B} \left( \frac{\delta Q}{T} \right)_\text{REV} = \frac{1}{T} \int_{A}^{B} \left(\delta Q\right) _\text{REV} = \frac{Q}{T} 
\end{gather*}
\hfil\\
Per usare la relazione di Clausius utilizzo solo gli stati iniziali e finali, concentriamoci ora su altri stati.\\
In generale si guarda la differenza di \textbf{energia} tra stato iniziale e finale.\\
Consideriamo il punto di fusione dell'acqua in cui la temperatura a quel momento 0°C, e vogliamo calcolare l'entropia durante un passaggio di stato.\\
Siccome possiamo farlo reversibilmente e quindi ho un termostato di temperatura vicina a 0°C(quinsi diamo nel caso di isoterma reversibile)
\begin{gather*}
    \Delta S = \frac{Q}{T} = \pm \frac{m \ l}{T}
\end{gather*}
Quindi la massa del sistema per il calore latente, diviso la temperatura.\\
Calcoliamo ora l'integrale $\int_{A}^{B} \frac{\delta Q }{T}$:
\begin{gather*}
    I_{AB} = \frac{1}{T'} \int_{A}^{B} \delta Q = \pm \frac{m \ l}{T'}
\end{gather*}
\subsection{dim trasf. irreversibile con temperature uguali}
\begin{center}
            \begin{tikzpicture}
                \fill[pattern=north east lines](0,0)rectangle(2,1);
                \filldraw[fill=page] (0.2,0.2) rectangle (1.8,0.8);
                \draw(0,0) rectangle (2,1);
                \fill[fill=cyan] (0.2,0.2) rectangle (1,0.8);
                \draw[->](2.5,0.5) -- (3.5, 0.5);
                \fill[pattern=north east lines](4,0)rectangle(6,1);
                \draw(4,0) rectangle (6,1);
                \filldraw[fill=cyan] (4.2,0.2) rectangle (5.8,0.8);
                \node at(0,-0.5){$(V_i,T)$};
                \node at(4,-0.5){$(V_f,T)$};
            \end{tikzpicture}
\end{center}
Qundi abbiamo un volume iniziale che si espande in uno finale ma che mantiene la stessa temperatura, questa trasformazione è sicuramente reversibile. In generale non sarà mai adiabatica.\\
Di conseguenza possiamo calcoalre l'entropia:
\begin{gather*}
    \Delta S = \frac{Q}{T}
\end{gather*}
$Q$ in generale lo so calcolare oichè se sono nel caso di un gas perfetto so che:
\begin{gather*}
    Q = L = \mathcal{N} R T \ln\left(\frac{V_f}{V_i}\right)\\
    \Delta S = \mathcal{N} R T \ln\left(\frac{V_f}{V_i}\right) 
\end{gather*}
Per calcolarla ho dunque usato una trasformazione ausiliaria, ho capito che la trasformazione che sto considerando è possibile rappresentarla su un isoterma e quindi ho creato unatrasf. ausiliaria. Ho dimostrato che questa trasf. è irreversibile poichè ho dimostrato che $\Delta S > I_{AB}$.\\
\subsection{trasf. irreversibile tra due sistemi a temperature diverse}
\begin{center}
    \begin{tikzpicture}
        \filldraw[pattern=north east lines](-0.5,-0.5)rectangle(4.5,2.5);
        \filldraw[fill = page](-0.3,-0.3) rectangle(4.2,2.3);
        \filldraw[fill = page](0,0) rectangle (2,2) node[midway]{$T_1 \quad \boxed{1}$};
        \filldraw[fill = page](2,0) rectangle (4,2) node[midway]{$T_2 \quad \boxed{2}$};
        \draw[red](2,-0.2) -- (2,2.2);
    \end{tikzpicture}
\end{center}
Posso considerare anche il sistema $1+2$ che all'inizio ha un vincolo che va da una certa temperatura a $T_e$ ma in questo modo so che è termicamente isolato e posso sfruttare questa cosa.
\begin{gather*}
    T_2 > T_1\\
    T_2 > T_e > T_1
\end{gather*}
Con $T_e$ la temperatura ad eq. termico dei due sistemi. I sistemi sono isolati da pareti adiabatiche per non considerare il contributo dell'ambiente
\begin{gather*}
    i: \ T(1) = T_1 \qquad T(2) = T_2\\
    f: \ T(1) = T(e) = T(2) 
\end{gather*}
Ora per sapere se questa trasf. è irreversibile devo dimostrare che:
\begin{gather*}
    \Delta S = S(f) - S(i) > 0
\end{gather*}
Per fare questo calcolo devo trovare una trasf. reversibile che connetta lo stato finale a quello iniziale.\\
Per fare questo copriamo ogni intervallo di temperatura con un termostato (teoricamente infiniti)
\begin{center}
    \begin{tikzpicture}
        \draw(0,0) rectangle (1.5,0.8) node[midway]{$T_1 + dT$};
        \draw(1.5,0) rectangle (3,0.8) node[midway]{$T_1 + 2dT$};
        \draw(4,0) rectangle (5.5,0.8) node[midway]{$T_e - 2dT$};
        \draw(5.5,0) rectangle (7,0.8) node[midway]{$T_e - dT$};
        \node at(3.5,0){...};
        \draw(0,0) rectangle (1.5,-0.8) node[midway]{$1$};
        \draw(1.5,0) rectangle (3,-0.8) node[midway]{$1$};
        \draw(4,0) rectangle (5.5,-0.8) node[midway]{$1$};
        \draw(5.5,0) rectangle (7,-0.8) node[midway]{$1$};
    \end{tikzpicture}
\end{center}


\begin{gather*}
    \delta Q =\mathcal{C}_p \ dT
\end{gather*}
La quantitàdi calore che cede sarà proprio:
\begin{gather*}
    Q_1 = \int_{T_1}^{T_e} \delta Q = \int_{T_1}^{T_e} \mathcal{C}_1 \ dT >0
\end{gather*}
Quindi il sistema 1 scambia una quantità di calore positiva.\\
Ora procedo a calcolare l'entropia del sistema 1:
\begin{gather*}
    \Delta S_1 = \int_{T_1}^{T_e} \frac{\delta Q}{T} = \int_{T_1}^{T_e} \frac{\mathcal{C}_1}{T} dT > 0
\end{gather*}
Ora guardo il sistema 2:
\begin{gather*}
    Q_2 = \int_{T_2}^{T_e} \mathcal{C}_2 dT = - \int_{T_e}^{T_2} \mathcal{C}_2 dT < 0
\end{gather*}
E per l'entropia:
\begin{gather*}
    \Delta S_2 = \int_{T_2}^{T_e} \frac{\delta Q}{T} = \int_{T_2}^{T_e} \frac{\mathcal{C}_2}{T} dT = -\int_{T_e}^{T_2} \frac{\mathcal{C}_2}{T} dT < 0
\end{gather*}
Ora considero l'energia interna dei due sistemi. 
\begin{gather*}
    \Delta U = \Delta U_1 + \Delta U_2 = 0 \qquad \Delta U_1 = -\Delta U_2\\
    \Delta U_1 = Q_1\\
    \Delta U_2 = Q_2
\end{gather*}
e quindi:
\begin{gather*}
    Q_1 = -Q_2
\end{gather*}
\begin{gather*}
    \int_{T_1}^{T_e} \frac{\mathcal{C}_1}{T} \ dT > \int_{T_1}^{T_e} \frac{\mathcal{C}_1}{T_e} \ dT\\
    \int_{T_1}^{T_e} \frac{\mathcal{C}_1}{T} \ dT > \frac{1}{T_e}\int_{T_1}^{T_e} \mathcal{C}_1 \ dT = \frac{Q_1}{T_e}\\
    \boxed{\Delta S_1 > \frac{Q_1}{T_e}}
\end{gather*}
\begin{gather*}
    T>T_e\\
    \frac{1}{T} < \frac{1}{T_e} \text{ con } \delta Q > 0\\
    \int_{T_2}^{T_e} \frac{\mathcal{C}_2}{T} \ dT = \int_{T_2}^{T_e} \frac{\mathcal{C}_2}{T_e} \ dT\\
    \boxed{\Delta S_2 > \frac{Q_2}{T_e}}
\end{gather*}
Ora sfrutto la relazione $Q_2 = -Q_1$ insieme a queste due nuove relazioni scoperte
\begin{gather*}
    \boxed{\Delta S_2 < \frac{Q_1}{T_e}}
\end{gather*}
Da questo posso concludere la dimostrazione:
\begin{gather*}
    \Delta S_1 > \frac{Q_1}{T_e} > -\Delta S_2\\
    \Delta S_1 > - \Delta S_2\\
    \boxed{ \Delta S_1 + \Delta S_2 > 0}\\
    \boxed{ \Delta S > 0}
\end{gather*}
\subsection{variazione dell'energia}
Prendendo la funzione generale $U = U(T,V)$:
\begin{gather*}
    dU = \left( \frac{\partial U}{\partial T} \right)_V dT + \left(\frac{\partial U}{\partial V}\right)_T dV = \delta Q - p \ dV\\
    dS = \frac{1}{T}\left( \frac{\partial U}{\partial T} \right)_V dT + \frac{1}{T}\left[\left(\frac{\partial U}{\partial V}\right)_T + p\right]  dV 
\end{gather*}
\begin{gather*}
    S = S(V,T)\\
    ds = \left( \frac{\partial U}{\partial T} \right)_V dT + \left( \frac{\partial U}{\partial V} \right)_T dV\\
    \frac{\partial}{\partial V}\left( \frac{\partial U}{\partial T} \right) = \frac{\partial}{\partial T}\left( \frac{\partial U}{\partial V} \right)
\end{gather*}
\begin{gather*}
    \frac{\partial }{\partial V}\left[ \frac{1}{T} \left( \frac{\partial U}{\partial T} \right)_V \right] = \frac{\partial}{\partial T}\left\{ \frac{1}{T}\left[\left( \frac{\partial U}{\partial V} \right)_T  + p\right] \right\}  \\
    \cancel{\left(\frac{\partial^2}{\partial V \partial T}\right) } = -\frac{1}{T} \left[ \left(\frac{\partial U}{\partial V}\right)_T + p  \right] + \cancel{\left(\frac{\partial^2}{\partial V \partial T}\right) } + \left( \frac{\partial p}{\partial T} \right)_V\\
    \frac{1}{T} \left[\left(\frac{\partial U}{\partial V}\right) + p \right] = \left( \frac{\partial p }{\partial T} \right)_V \\
    \boxed{\left(\frac{\partial U}{\partial V}\right)_T = T \left( \frac{\partial p}{\partial T} \right)_V -p} \qquad \text{eq. dell'energia interna}  
\end{gather*}
\begin{gather*}
    T \ dS = \delta Q = dU + p \ dV\\
    T \ dS = \mathcal{C}_V \ dT + \left(\frac{\partial U}{\partial V}\right)_T \ dV + p \ dV\\
    T \ dS = \mathcal{C}_V \ dT +  \left(\frac{\partial U}{\partial V}\right)_T \ dV - \cancel{p \ dV} + \cancel{ p \ dV}\\
    \boxed{T \ dS = \mathcal{C}_V \ dT + T \left(\frac{\partial U}{\partial V}\right)_T \ dV} \qquad \text{eq. del TdS (I)}\\
    \boxed{T \ dS = \mathcal{C}_P \ dT -  T\left(\frac{\partial U}{\partial V}\right)_p \ dp} \qquad \text{eq. del TdS (II)}
\end{gather*}
\begin{gather*}
    \left(\frac{\partial S}{\partial T}\right)_V = \frac{\mathcal{C}_V}{T}\\
    \left(\frac{\partial S}{\partial V}\right)_T = \left(\frac{\partial p}{\partial T}\right)_V\\
    \left(\frac{\partial S}{\partial T}\right)_p = \frac{\mathcal{C}_p}{T}\\
    \left(\frac{\partial S}{\partial p}\right)_T = \left(\frac{\partial U}{\partial T}\right)_p
\end{gather*}
Risolviamo nel caso du un gas perfetto e l'equazione del TdS diventa:
\begin{gather*}
    dS = \frac{\mathcal{C}_V}{T} \ dT + \left(\frac{\partial p}{\partial T}\right)_V \ dV\\
    p = \frac{\mathcal{N} R T}{V} \qquad \left(\frac{\partial p}{\partial T}\right)_V = \frac{\mathcal{N} R}{V}\\
    dS = \mathcal{C}_V \frac{dT}{T} + \mathcal{N} R \frac{dV}{V}\\
\end{gather*}
Con $(T_0,V_0)$:
\begin{gather*}
    S(T,V) = \mathcal{C}_V \ \ln\left( \frac{T}{T_0} \right) + \mathcal{N} R \ln\left(\frac{V}{V_0}\right)\\
    S(T,V) = \mathcal{C}_V \left[ \ln\left(\frac{T}{T_0}\right) + \frac{\mathcal{N} R}{\mathcal{C}_V} \ln\left(\frac{V}{V_0}\right)   \right]   
\end{gather*}
\begin{gather*}
    S(T,V) = \mathcal{C}_V \left[\ln\left(\frac{T}{T_0}\right) + \ln\left(\frac{V}{V_0}\right)  \right]\\
    S(T,V) = \mathcal{C}_V \ln\left(\frac{T V^{\gamma -1}}{T_0 V_0^{\gamma -1}}\right)\\
    S(p,V) = \mathcal{C}_V \ln\left(\frac{p V^\gamma}{p_0 V_0^\gamma}\right)\\
    S(T,p) = \mathcal{C}_V \ln\left(\frac{T^\gamma p^{1-\gamma}}{T_0^\gamma p_0^{1-\gamma}}\right)    
\end{gather*}

\newpage
\section{4/11/25}
Riprendiamo la transizione di stato tra un volume liquido e uno cassoso:
\begin{center}
    \begin{tikzpicture}
        \draw[->](0,0) -- (0,2) node[at end, left]{$p$};
        \draw[->](0,0) -- (2.5,0) node[at end, below]{$V$};
        \draw(0.3,1.5) -- (0.8,1) node[midway, above]{$T$};
        \draw(0.8,1) -- (1.8,1);
        \draw(1.8,1) -- (2.1, 0.7);
        \draw[dashed](0.8,1) -- (0.8,0) node[at end, below]{$V_l$};
        \draw[dashed](1.8,1) -- (1.8,0) node[at end, below]{$V_g$};
    \end{tikzpicture}
\end{center}
\begin{gather*}
    T(ds) = T(\frac{\partial p}{\partial T})_V dV
\end{gather*}
Ma siccome non abbiamo la dipendenza dal volume $(\frac{\partial p}{\partial T})_V = \frac{d p_s}{dT}$ e di conseguenza la nostra equazione diventa:
\begin{gather*}
    T(ds) = T \frac{d p_s}{dT} \ dV \\
    Q = T \frac{d p_s}{dT} (V_g - V_l)
\end{gather*}
Ovvero la quantità di calore dal passaggio liquido a quello gassoso.\\
Per semplicità prendiamo che la quantità di calore trasmessa è una mole:
\begin{gather*}
    Q = Ml_v\\
    Ml_v = T  \frac{d p_s}{dT} (V_g- V_l)\\
    \boxed{\frac{dp_s}{d T} = \frac{Ml_v}{T(V_g-V_l)}}
\end{gather*}
Quest'ultima è detta equazione di Clapeyror.\\
Considero ora che $V_g >> V_l$ e ricordo le seguenti uguaglianze:
\begin{gather*}
    \frac{dp_s}{dT} = \frac{Ml_v}{TV_g}\\
    p_s V_g = R T\\
    V_g = \frac{RT}{p_s}
\end{gather*}
E a questo punto mi ricavo le seguenti equazioni
\begin{gather*}
    \frac{d p_s}{d T} = \frac{Ml_v}{ R T^2} p_s\\
    \frac{d T_l}{dp} = \frac{R T^2_l}{p \ ML_v}\\
    \frac{d p}{d z} = -\rho g \\
    d p = -\rho g dz
    \frac{d T_l}{d z} = - \rho g \frac{R T_l^2}{p \ Ml_v}
\end{gather*}
Da questo si capisce che per via della quota la pressione cambia, vogliamo capire di quanto:
\begin{gather*}
    M = M_{H_2 O} \qquad \rho = \rho_\text{aria}\\
    M_\text{aria} p = \rho_\text{aria} R T_\text{aria}\\
    \frac{d T_l}{dz} = \frac{- M_\text{aria}}{M_{H_2 O}}\frac{T_l^2}{T_\text{aria}} \frac{g}{l_v}\\
    \frac{d T_l}{dz} \simeq -3.5 \times 10^{-3} m^{-1} K
\end{gather*}
Questo è per il passaggio da liquido a gas ma questo vale anche da solido a liquido e il ragionamente è analogo.
\begin{gather*}
    \frac{dp}{dT}|_\text{fusione} = \frac{Ml_f}{T(V_l-V_s)} 
\end{gather*}
In qesto caso però si deve avere che $V_l-V_s < 0$, nel nostro caso lo stiamo facendo nel caso con l'acqua e siccome quest non è vero si invertono i segni
\begin{gather*}
    \frac{dp}{dT}|_\text{fusione} < 0 \ (H_2O)\\
    \frac{dT}{dp}|_\text{fusione} < 0 \ (H_2O)
\end{gather*}
questo è il motivo del perchè un ghiacciaio non sta fermo ma "cammina".\\
Un ghiacciaio ha una grande massa e la sua temperatura è inferiore agli 0°C, perchè la pressione del ghiacciaio è alzata dalla massa del ghiaccio e quindi la temperatura di fusione si alza e si fonde uno strato di ghiaccio e questo strato di acqua fa spostare il ghiacciaio.\\
Con lo stesso principio funzionano i pattini su ghiaccio perchè fanno sciogliere sulla lama uno strato di ghiaccio (per via della alta pressione).

\newpage
\section{6/11/25}
\subsection{Introduzione alla fisica statistica}
Si inizia con il considerare una serie di eventi:
\begin{gather*}
    x \in \{x_1,x_2, \ ... \ , x_M\}
\end{gather*}
A ciascuno di questi valori si associa una probabilità $p_i \in [0,1]$. L'insieme di tutti questi valori prende il nome di distribuzione probabilistica.
\begin{gather*}
    \{p_i\} = \{p_1 , p_2 , \ ... \ , p_M\}
\end{gather*}
Si definisce la \textbf{condizione di normalizzazione}, che fa si che la somma di tutte le probabilità deve essere pari a 1. Questo fa si che la distribuzione sia normalizzata o unitaria.
\begin{gather*}
    \sum_{i=0}^{M} p_i = 1
\end{gather*}
In questi casi si può definire un valore atteso, o valore medio, che si ottiene dalla media pesata di tutte le probabilità:
\begin{gather*}
    \mathbb{E}(x) = \sum_{i=1}^{M} p_i \ x_i= \left\langle x \right\rangle
\end{gather*}
Se ogni risultato ha la stessa probabilità per la condizione di normalizzazione si ha:
\begin{gather*}
    p_i = \frac{1}{M} \qquad \mathbb{E} (x) = \frac{1}{M} \sum_{i=1}^{M} x_i = \left\langle x\right\rangle 
\end{gather*}
Dalla media dei pesi si ottiene anche la varianza:
\begin{gather*}
    var(x) = \sigma^2 = \left\langle (x-\left\langle x \right\rangle )^2 \right\rangle = <x^2> - \left\langle x\right\rangle^2 
\end{gather*}
Ad ogni intervallo di un valore $x_0$ si associa un infinitesimo
\begin{gather*}
    dp(x_0) = f(x_o)dx \qquad x \in [x_\text{max},x_\text{min}]
\end{gather*}
Se si fa l'integrale di queste grandezze otterrò la probabilità totale, e siccome è normalizzata:
\begin{gather*}
    \int_{x_\text{min}}^{x_\text{max}} f(x) \ dx = 1
\end{gather*}
E il valore atteso diventa:
\begin{gather*}
    \mathbb{E}(x) = \left\langle x \right\rangle = \int_{x_\text{min}}^{x_\text{max}} f(x) \ dx  
\end{gather*}
Stiamo considerando che $dx$ è equivalente $\frac{dx}{\Delta x}$\\
Consideriamo la probabilità di un evento tra due intervalli:
\begin{gather*}
    p(a,b) = p[a<x<b] = \int_{a}^{b} f(x) \ dx
\end{gather*}
Se la generalizzo a due valori interni ottengo la probabilità di ottenere una variabile in questo intervallo:
\begin{gather*}
    p(x_j,x_k) = p[x_j<x<x_k] = \sum_{i=j}^{k} p_i
\end{gather*}
$N$ eventi indipendenti coon $P_i$:
\begin{gather*}
    \Pi 
\end{gather*}
\begin{gather*}
    \left\langle g(x)\right\rangle = \int_{x_\text{min}}^{x_\text{max}} g(x) f(x) \ dx 
\end{gather*}
Questa è stata detta anche come: Law of uncnscious statiscian.\\
Se si considera ora una funzione $g(x) = y$
\begin{gather*}
    f(y) = \sum_{x:g(x) = y} f(x) = \int_{x_\text{min}}^{x_\text{max}} dx \ f(x) \delta(g(x) - y)\\
    p_{yi} = \sum_{i=1}^{M} \delta_{g(x), y_i} p_i
\end{gather*}
\subsection{la legge dei grandi numeri}
\begin{theorem}[del limite centrale]
    Si suppone di avere $N$ variabili $x_i$ indipendenti tra di loro e identicamente distribuibile (i.i.d).
    Definiamo la media empirica
    \begin{gather*}
        \overline{x}_N = \frac{1}{N} \sum_{i=1}^{N} x_i \underset{N \to \infty}{\rightarrow} \left\langle x \right\rangle 
    \end{gather*}
    Ovvero la media tra i risultati dei vari valori $x_i$, che fanno parte di un campione nel nostro set di dati ( a differnza della media che è su tutti). Che è un valore reale a differenza della media normale che è una stima di un valore.
\end{theorem}\hfill\\
Questo è un limite di convergenza di probabilità che è noto che:
\begin{gather*}
    p( \lim_{N \to \infty} \overline{x}_N = \left\langle x \right\rangle ) = 1
\end{gather*}
Questo in fisica statistica è certo. È anche ciò su cui poggia un po' tutta la fisica statistica.\\
Il teorema del limite centrale suppone inoltre che:
\begin{gather*}
    var(x) < \infty
\end{gather*}
Quindi varianza finita.
Se si calcola:
\begin{gather*}
    p( \overline{x}_N ) \sim \text{ gaussiana}\\
    x_0 = \left\langle x \right\rangle \qquad var = \frac{\sigma^2}{N} 
\end{gather*}
Quindi si ottiene approssimativamente una gaussiana
\subsection{teoria cinetica}
Si definisce tramite l'utilizzo dei principi dei gas perfetti e la statistica generale.
\subsection{meccanica statistica}
(bho)\\
Si parte dall'energia cinetica dei gas
\hfil\\
\begin{center}
    \begin{tikzpicture}
        \draw[->](0,0) -- (0, 2.5) node[at end, left]{$z$};
        \draw[->](0,0) -- (2.5, 0) node[at end, below]{$y$};
        \draw[->](0,0) -- (-1.5, -1.5) node[at end, below]{$x$};
        \draw(-0.5,-0.5) -- (0.5,-0.5) -- (1,0) node[at end, below]{$L$};
        \draw(1,0) -- (1,1) node[midway, right]{$L$};
        \draw(-0.5,-0.5) -- (-0.5,0.5);
        \draw(0.5,-0.5) -- (0.5,0.5);
        \draw(-0.5,0.5) -- (0.5,0.5) -- (1,1) -- (0,1) -- (-0.5,0.5);
        \node at(0,0)[below, left]{o};
        \node at(0.5,1)[above]{$L$};
    \end{tikzpicture}
\end{center}
Si suppone di trattare il gas come gas perfetti, quindi estremamente rarefatti.\\
Assumiamo che la risultatnte dei momenti delle forze esterne sia nullo, quindi possiamo sempre scegliere un sistema di rifermimento inerziale fissato al nostro sistema.\\
E inoltre si assume che l'energia poteniale di interazione sia 0: $U_\text{interaz.} \equiv 0$
\begin{gather*}
    U(\vv{r}_1, \ ... \ , \vv{r}_n) = 0 \qquad E = \frac{m}{2} \sum_{i=1}^{N} |\vv{v}_i|^2
\end{gather*}
Con $m$ che è la massa di ciascuna particiella.\\
Prendo un i-esimo atomo e considero la sua velocità:
\begin{gather*}
    \vv{v}_i = v_x \widehat{i} + v_y \widehat{j} + v_z \widehat{k}
\end{gather*}
Quando si ha che una delle velocità si annulla vuol dire che si ha un urto con una parete, quello che accade è che la velocità della componente $x$ si ribalta
\begin{gather*}
    \vv{v}'_i = -v_x \widehat{i} + v_y \widehat{j} + v_z \widehat{k}
\end{gather*}
\begin{center}
    \begin{tikzpicture}
        \draw(0,0) -- (0,2);
        \draw[->](-1,0) -- (0,1) node[midway, below]{$\vv{v}_i$};
        \draw[->](0,1) -- (-1,2) node[midway, above]{$\vv{v}'_i$};
    \end{tikzpicture}
\end{center}
\begin{gather*}
    \Delta (m \vv{v}_i) = m\vv{v}'_i - m\vv{v}_i = -2m |V_i^x| \widehat{i} 
\end{gather*}
Di conseguenza se calcolo la forza tra il tempo prima e dopo l'impatto con la parete:
\begin{gather*}
    \int_{t}^{t'} \vv{F} \ dt = \Delta (m\vv{v}_i) = -2m (v_i^x) \widehat{i}
\end{gather*}
Se considero ora la forza che imprime ogni particiella su quella che stiamo considerando la trovo facendo:
\begin{gather*}
    \vv{F}_i\\
    \int_{t}^{t'} \vv{F}_i \ dt = 2m |v^x_i| \widehat{i}
\end{gather*}
Il tempo di impatto che abbiamo considerato $t^*$ che è tra: $t< t^*< t'$\\
Consideriamo ora il grafico:
\begin{center}
    \begin{tikzpicture}
        \draw(0,0) -- (4,0) node[at end, below]{$t$};
        \draw(0,0) -- (0,4) node[at end, left]{$\frac{F_i}{L^2}$};
        \draw(0,0) ..controls(1,0.2) .. (1.45,3.7) .. controls(1.5,4) .. (1.55,3.7) .. controls(2,0.2) .. (3,0);
        \node at(1.5,4)[above]{$t^*$};
    \end{tikzpicture}
\end{center}
Consideriamo $\tau$ come il tempo dello strumento macroscopico che sto utilizzando mentre $\tau_\text{urto}$ è il tempo dell'urto, per quanto preciso possa essere avrò comunque:
\begin{gather*}
    \tau >> \tau_\text{urto}
\end{gather*}
Ci chiediamo qual'è il tempo tra un urto di una parete e unaltro:
\begin{gather*}
    t_u = \frac{L}{|v^x_i|}
\end{gather*}
Il tempo complessivo è due volte perchè si parte dal punto in cui si è appena fatto un urto.\\
Il numero degli urti sarà quindi:
\begin{gather*}
    n_u = \left\lfloor \frac{\tau}{2t_n} \right\rfloor \simeq \frac{\tau}{2 t_n} = \frac{\tau}{2L} |v_i^x|\\
    \sum_{i=1}^{N} \frac{1}{\tau} \int_{0}^{\tau} F_i \ dt = \sum_{i = 1}^{N} 2 n_u m |v^x_i| = \sum_{i = 1}^{N} \frac{m}{N} (v^x_i)^2 = p
    p = \textcolor{red}{N} \frac{m}{V} \underbrace{\textcolor{red}{\frac{1}{N}} \sum_{i = 1}^{N} (v_i^x)^2}_\text{è la media empirica} 
\end{gather*}

\newpage
\section{7/11/25}
Dalla scorsa volta abbiamo siamo arrivati a:
\begin{gather*}
    p = \frac{m \ N}{V} \frac{1}{N} \sum_{i=1}^{N} (v^x_i)^2
\end{gather*}
E abbiamo visto che il secondo pezzo sarebbe proprio la media empirica. A questo punto non mi interessa esprimere la media empirica poichè la velocità delle varie particielle non le conosco.\\
Abbbiamo anche supposto che le velocità dell particielle fossero indipendenti tra di loro (abbastanza assurdo), e anche che la massa delle particielle siano tutte uguali.\\
Da questi ragionamenti possiamo scrivere che per $n \to \infty$ la media empirica diventa proprio $\left\langle (v_x)^2 \right\rangle $\\
E di conseguenza riscriviamo l'equazione come:
\begin{gather*}
    p = \frac{m \ N}{V} \left\langle (v^x)^2 \right\rangle 
\end{gather*}
\hfil\\
Consideriamo ora non solo le velocità lungo $x$ ma in tre dimensioni:
\begin{gather*}
    \left\langle (v^x)^2 \right\rangle \qquad \left\langle (v^y)^2 \right\rangle \qquad \left\langle (v^z)^2 \right\rangle
\end{gather*}
Ma poichè noi stiamo studiando cosa succede vicino alle temperature di equilibrio:
\begin{gather*}
    \left\langle (v^x)^2 \right\rangle = \left\langle (v^y)^2 \right\rangle = \left\langle (v^z)^2 \right\rangle
\end{gather*}
Adesso chiamo il modulo $\left\langle v^2 \right\rangle = \left\langle (v^x)^2+(v^y)^2+(v^z)^2 \right\rangle $ che è uguale a  $    \left\langle (v^x)^2 \right\rangle + \left\langle (v^y)^2 \right\rangle + \left\langle (v^z)^2 \right\rangle$\\
Quindi se considero solo una componente avrò che è un terzo del modulo totale:
\begin{gather*}
    \left\langle (v^x)^2 \right\rangle = \left\langle (v^y)^2 \right\rangle = \left\langle (v^z)^2 \right\rangle = \frac{1}{3}\left\langle v^2 \right\rangle 
\end{gather*}
Ritornando all'equazione iniziale sostituendo il nuovo modulo è:
\begin{gather*}
    \frac{1}{3} \frac{m \ N}{V} \left\langle v^2 \right\rangle\\
    p \ V = \frac{1}{3} m \ N \left\langle v^2 \right\rangle  
\end{gather*}
Ho quindi ritrovato la legge di boile, ora ricordandomi che $p V= \mathcal{N} R T$ dalla teoria dei gas perfetti:
\begin{gather*}
    \frac{1}{3} m \ N \left\langle v^2 \right\rangle = \mathcal{N} R T
\end{gather*}
Adesso proprio perchè siamo nel modello di gas perfetti siamo "forzati" a dover fare l'ipotesi che:
\begin{gather*}
    \frac{\frac{1}{3} m \ N \left\langle v^2 \right\rangle}{\mathcal{N} R} = T
\end{gather*}
Questo \textbf{solo} perchè siamo nel caso di gas perfetti.\\
Ora sfruttiamo questa relazione per ricavare alcune cose.\\
Ricordiamo che ad es. vale che il numero di atomi
\begin{gather*}
    N = \mathcal{N} N_A
\end{gather*}
con $N_A$ numero di avogadro.
\begin{gather*}
    \frac{1}{3} m  \cancel{\mathcal{N}} N_A \left\langle v^2 \right\rangle = \cancel{\mathcal{N}} \ R \ T 
\end{gather*}
\begin{gather*}
    T = \frac{M}{3R} \left\langle v^2 \right\rangle 
\end{gather*}
Con $\left\langle v^2 \right\rangle$ che è la velocità quadratica media e $M$ che è la massa di una mole.
Questa deriva da Clausius-Kroning.
\hfil\\
È utile adesso introdurre una costante dei gas per unità di atomi:
\begin{gather*}
    \frac{R}{N_A} = K_B \simeq 1.38 \times 10^{-23} JK^{-1}
\end{gather*}
costante di Bolzman.
Con questo riscriviamo l'equazione di Clausius-Kroning:
\begin{gather*}
    T = \frac{m}{3 K_B} \left\langle v^2 \right\rangle 
\end{gather*}
Questo è corretto ma per visualizzare meglio moltiplico da entrambe le parti per un fattore interessante:
\begin{gather*}
    \frac{3}{2} K_B T = \frac{1}{2} m \left\langle v^2 \right\rangle = \left\langle \frac{1}{2} m v^2 \right\rangle  
\end{gather*}
E l'ultimo termine è proprio l'energia cinetica media delle molecole, e fa capire che ovviamente è direttamente proporzionale ad essa.\\
Questo ci dice che è concettualmente sbagliato pensare che dalla temperatura dipende la velocità delle particielle, ma la temepratura influisce su quanto le varie velocità delle varie particielle si sparpagliano rispetto alla velocità medi adi tutte le particielle.\\
Questo anche perchè il modo corretto di interpretarlo è utilizzando come sistema di riferimento il centro di massa del sistema.\\
Se si considera la temperatura come la temperatura ambiente e si ha una massa di un atomo tipica (m), allora siccome rimane solo la costante e si puo ricavare così la velocità media che è di varie centinaia di metri al secondo.\\
\begin{gather*}
    E = \frac{m}{2} \sum_{i=1}^{N} v_i^2 \\
    = \frac{m}{2} N \underbrace{\frac{1}{N} \sum_{i=1}^{N} v_i^2}_{\text{per } n \to \infty \Rightarrow \left\langle v^2 \right\rangle }
\end{gather*}
Riscrivendo:
\begin{gather*}
    E = N \frac{1}{2} m \left\langle v^2 \right\rangle   = N \left\langle  \frac{1}{2} m v^2\right\rangle 
\end{gather*}
Adesso passando dal microscopico adl macroscopico possiamo associare a questa nuova quantità:
\begin{gather*}
    U = N \left\langle  \frac{1}{2} m v^2\right\rangle 
\end{gather*}
Questo è vero per il modello che stiamo considerando ma non è vero in generale, ovvero che l'energia interna è uguale a l'energia istantanea, ma vedremo che se associamo il valore dell'energia interna macroscopica all'energia meccanica attesa del sistema.
\begin{gather*}
    U = \left\langle E \right\rangle 
\end{gather*}
E questo si vedrà valere in generale.\\
Adesso con alcune identità che abbiamo possiamo trovare che:
\begin{gather*}
    U = N \left\langle  \frac{1}{2} m v^2\right\rangle = \frac{3N}{2} K_B T
\end{gather*}
Guardando la capacitò termica:
\begin{gather*}
    \mathcal{C}_V = \frac{dU}{dT} =\Rightarrow \text{ gas perfetto monoatomico}
\end{gather*}
E sappiamo che il termine nell'equazione precedente è proprio la capacitò termica:
\begin{gather*}
    \mathcal{C}_V = \frac{3}{2} N K_B = \frac{3}{2} \mathcal{N} R
\end{gather*}
Essedo quindi in un gas perfetto ricordiamo che valgono:
\begin{gather*}
    pV = N K_B T\\
    pV = \frac{2}{3} U
\end{gather*}
\hfil\\
Ricordiamo che abbiamo fatto una serie di considerazioni solo nel caso vicino all'equilibrio.\\
(Diciamo che l'ipotesi errata che abbiamo supposto ovvero che non ci fossero urti tra le particielle, in realtà non influenza il risultato finale).\\
Consideriamo ora l'identità della velocità media:
\begin{gather*}
     \left\langle v^2 \right\rangle = \int_{0}^{\infty} v^2 F(v) \ dv
\end{gather*}
Seguendo il procedimento fatto Maxwell, e andiamo a calcolare:
\begin{gather*}
    f(v_x) \qquad f(v_y) \qquad f(v_z)
\end{gather*}
Che sono l'analogo dell'integrale sopra.\\
Facciamo l'ipotesi che non ci sia ne una direzione en una posizione privilegiata, ovvero omogeneità. (non c'è un caxxo)\\
\begin{gather*}
    dp ( v_x, v_y, v_z) \overset{*}{=} f(v_x)f(v_y)f(v_z) \ dv_x \ dv_y \ dv_z  
\end{gather*}
* ipotesi di indipendenza.\\
Cerchiamo ora di sfruttare di più l'ipotesi di omogeneità, e quindi possiamo passare ad un altro sistema di assi che chiamo $\xi, \eta, z$ (z è uguale), ho dunque rotato di un certo angolo che chiamo $\varphi$ attorno all'asse z:
\begin{gather*}
    v_\xi = \cos(\varphi) v_x + \sin(\varphi) v_y
    v_\eta = -\sin(\varphi) v_x + \cos(\varphi) v_y
\end{gather*}
Per proprietà dello jacibiano 
\begin{gather*}
    dv_x \ dv_y = \begin{bmatrix}
        \text{\tiny det della matrice} \\
        \text{\tiny jacobiana che è 1 in}\\
        \text{\tiny questo caso pk è una rotaz.}
    \end{bmatrix} dv_\xi \ dv_\eta
\end{gather*}
\begin{gather*}
    f(v_x) f(v_y) f(v_z) \ dv_x \ dv_y \ dv_z
    = f(v_\xi) f(v_\eta) f(v_z) \ dv_\xi \ dv_\eta \ dv_z\\
    f(v_x) f(v_y) = f(v_\xi) f(v_\eta)\\
    f(v_x) f(v_y) = f(\cos(\varphi) v_x+ \sin(\varphi) v_y) f(-\sin(\varphi) v_x + \cos(\varphi) v_y)
\end{gather*}
\begin{gather*}
    \cos(\varphi) \simeq 1 + o(\varphi^2)\\
    \sin(\varphi) \simeq \varphi + o(\varphi^2)\\
    f(v_x)f(v_y) = f(v_x + \varphi v_y)f(v_y - \varphi v_x)\\
    f(v_x) f(v_y) = [f(v_x) + \varphi v_y f'(v_x)][f(v_y)-\varphi v_x f'(v_y)]\\
    = f(v_x) f(v_y) - \varphi v_x f(v_x) f'(v_y) + \varphi v_y f'(v_y) f(v_y)
\end{gather*}
\begin{gather*}
    \frac{f'(v_x)}{v_xf(v_x)} = \frac{f'(v_y)}{v_y f(v_y)}\\
    G(x_1) = G(x_2)\\
    \frac{f'(v_x)}{v_x f(v_x)} = \frac{f'(v_y)}{v_y f(v_y)} = C \quad \text{(cost.)}
\end{gather*}
\begin{gather*}
    \frac{f'(v_x)}{v_x f(v_x)} = c\\
    \frac{f'(v_x)}{f(v_x)} = c \ v_x\\
    \frac{d}{d(v_x)} \ln[f(v_x)] = c \ v_x\\
    \int_{0}^{v_x} \frac{d}{dy} \ln[f(y)] \ dy = c \int_{0}^{v_x} y \ dy\\
    \ln[\frac{f(v_x)}{f(0)}] = \frac{1}{2} c \ v_x^2\\
    f(v_x) = f(0) e^{\frac{c}{2} v_x^2}
\end{gather*}
Siccome la mia funzione descrive una probabilità deve necessariamente fare 1.
\begin{gather*}
    \int_{-\infty}^{+\infty} f(v_x) \ dv_x = 1\\
    f(0) \int_{-\infty}^{+\infty} e^{\frac{c}{2} x^2} \ dx = 1\\
    c = \frac{-2}{\omega^2}\\
    f(0) \int_{-\infty}^{+\infty} e^{-\frac{x^2}{\omega^2}} \ dx = 1\\
    \omega \int_{-\infty}^{+\infty} e^{-y^2} \ dy = \frac{1}{f(0)}\\
    \omega \sqrt{\pi} = \frac{1}{f(0)} \qquad f(0) = \frac{1}{\omega\sqrt{\pi}}\\
    f(v_x) = \frac{1}{\omega \sqrt{\pi}} e^{-(\frac{v_x^2}{\omega^2})}
\end{gather*}
distribuzione di probabilità di $v_x$
\begin{gather*}
    f(v_x) f(v_y) f(v_z) = \frac{1}{\omega \sqrt{\pi}} e^{-(\frac{v_x^2 + v_y^2 + v_z^2}{\omega^2})} =\frac{1}{\omega \sqrt{\pi}} e^{-(\frac{v}{\omega^2})} = f(\vv{v})
\end{gather*}
\begin{gather*}
    v_x , v_y , v_z \to (v, \theta , \varphi )\\
    dv_x \ dv_y \ dv_z = v^2 \sin(\theta) \ d\theta \ d\varphi dv\\
    F(v, \theta, \varphi) dv \ d\theta \ d\varphi = v^2 \sin(\theta) (\frac{1}{\omega\sqrt{\pi}})^3 e^{-\frac{v^2}{\omega^2}} \ dv \ d\theta \ d\varphi \\
    F(v) dv = 4\pi v^2 \frac{1}{(\omega \sqrt{\pi})^3}e^{-\frac{v^2}{\omega^2}} \ dv\\
    f(v) = \frac{4v^2}{\omega^3 \sqrt{\pi}} e^{-\frac{v^2}{\omega^2}}
\end{gather*}
\begin{gather*}
    \left\langle v^2 \right\rangle = \int_{0}^{\infty} v^2 F(v) \ dv\\
    \left\langle v \right\rangle =   \int_{0}^{\infty} v F(v) \ dv
\end{gather*}
Questi sono integrali del tipo:
\begin{gather*}
    cost.\int_{0}^{\infty} v^k e^{-\frac{v^2}{\omega^2}} \ dv\\
    \int_{0}^{\infty} x^n e^{-x^2} \ dx
\end{gather*}
Questi sono detti integrali gaussiani
\begin{gather*}
    \left\langle v \right\rangle = \frac{2}{\sqrt{\pi}} \omega\\
    \frac{3 k_B \ T}{m} = \left\langle v^2 \right\rangle = \frac{3}{2} \omega\\
    \omega^2 = \frac{2 k_B \ T}{m} \qquad \left\langle v \right\rangle \sqrt{\frac{8k_B \ T}{\pi m}}\\
    F'(\omega) = 0\\
    f(\vv{v}) = \left( \frac{m}{2\pi k_B \ T} \right)^\frac{3}{2} e^{-\frac{mv^2}{2k_B \ T}}\\
    F(v) = 4\pi \left( \frac{m}{2 \pi k_B \ T} \right)^\frac{3}{2} v^2 e^{-\frac{mv^2}{2 k_B \ T}}   
\end{gather*}
\begin{gather*}
    \left\langle v^2_r \right\rangle = \left\langle (\vv{v}_1 - \vv{v}_2)^2 \right\rangle = \left\langle v_1^2 \right\rangle + \left\langle v^2_1 \right\rangle - 2\left\langle \vv{v}_1 \cdot \vv{v}_2 \right\rangle     
\end{gather*}
Per calcolare il secondo termine sfrutto il fatto che è uguale a: $\left\langle \vv{v}_1 \right\rangle \cdot \left\langle \vv{v}_2 \right\rangle   $ che è $=0$
\begin{gather*}
    \left\langle v^2_r \right\rangle = 2\left\langle v^2 \right\rangle\\
    v_r = \sqrt{2} v  
\end{gather*}
Vediamo ora i limiti nel modello che abbiamo scelto.\\
La prima assunzione critica è che le particielle tra di loro non si urtino, che è sensatissimo in un gas perfetto, ma comunque qualche urto lo avrò lo stesso, cerchiamo di quantificare il numero di urti.\\
Introduciamo il concetto di sezione d'urto:
\begin{gather*}
    n = \frac{N}{V}
\end{gather*} 
Normalmente la sezione d'urto si indica con $\sigma$.\\
Quando è che due particielle si urtano? quando i loro centri distano circa due volte il raggio degli atomi.\\
Si può immaginare una circonferenza di raggio 2r e "altezza" $vt$ attorno ad ogni atomo e se un atomo entra completamente nel cerchio si ha un urto. Immaginandola nello spazio avrò una sezione cilindrica quindi ho che la mi ase. d'urto sarà:
\begin{gather*}
    \sigma = 4 \pi r^2
\end{gather*}
\begin{gather*}
    N^4 = n \sigma v t\\
    \frac{N^4}{t} = n \sigma v\\
\end{gather*}
Che è il numero di urti per unità di tempo
\begin{gather*}
    \tau_0 = \frac{1}{n\sigma v}
\end{gather*}
E questo è il tempo tipico fra un urto e un'altro e quindi:
\begin{gather*}
    \ell_0 = \frac{1}{n \sigma}
\end{gather*}
è la distanza tra un urto e unaltro, ed è detto libero cammino medio.\\
Si vede che il fattore che influenza di più questa distanza è la densità.\\
Se $v \to v_r$ si ha che $\ell = \frac{1}{\sqrt{2}n \sigma}$\\
Vedendo con una stima posso vedere che il numero di urti:
\begin{gather*}
    n \simeq 3 \times 10^{25} m^{-3}\\
    r \simeq 2  = 2 \mathring{A} \times 10^{-10} m\\
    \sigma = 4 \pi \times 4 \times 10^{-20} m^2 \simeq 5 \times 10^{-19} m^2\\
    \ell \approx t \times 10^{-8}
\end{gather*}
Ho una distanza tra un urto e unaltro molto piccola quindi ho un botto di urti.
\section{18/11/25}
Dalla scorsa volta sappiamo: $\left\langle n\right\rangle = \frac{t}{\tau} \quad P_n $\\
Divido un intervallo di tempo fissato in tanti $N$ intervalli.
\begin{center}
    \begin{tikzpicture}
        \draw[->](0,0) -- (3,0) node[at end, below]{$t$};
        \draw(0, -0.2) -- (0,0.2);
        \draw(0.5, -0.2) -- (0.5,0.2);
        \draw(1.5, -0.2) -- (1.5,0.2);
        \draw(1, -0.2) -- (1,0.2);
        \draw(2, -0.2) -- (2,0.2);
        \draw(2.5, -0.2) -- (2.5,0.2);
        \draw(3, -0.2) -- (3,0.2);
    \end{tikzpicture}
\end{center}
Dove ogni intervallino è di un tempo di : $\frac{t}{N}$
\begin{gather*}
    \frac{t}{N} \qquad 1-\frac{t}{N \tau}\\
    \left(\frac{t}{\tau N}\right)^n \left( 1-\frac{t}{N \tau} \right)^{N-n}\\
    \binom{n}{N} = \frac{N'}{n' (N-n)!} = \frac{N(N-1)(N-2) \dots (N-n +1)}{n!}\\
    P_n = \frac{N(N-1)(N-2) \dots (N-n + 1)}{n!}\left( \frac{t}{\tau N}  \right)\left( 1-\frac{t}{\tau N} \right)^{N-n}\\
    N \to \infty \qquad N(N-1)(N-2) \dots (N_n+1) \approx N^n\\
    \left(1 -\frac{t}{N \tau}\right)^{N-n} \approx \left( 1 - \frac{t}{\tau N}\right)^N\\      
\end{gather*}
Da qui:
\begin{gather*}
    p_n = \lim_{N \to infty} \frac{N^n}{n!} \left( \frac{t}{\tau N}\right)^n \left(1-\frac{t}{\tau N} \right)^N\\
    = \lim_{N \to \infty} \frac{1}{n!} \left(\frac{t}{\tau}\right)^n \left( 1 - \frac{t}{\tau N} \right)^N\\p
    _n = \frac{1}{n!} \left( \frac{t}{\tau}\right)^n e^{-\frac{t}{\tau}} = \frac{1}{n!} \left\langle n \right\rangle e^{\left\langle n \right\rangle }     
\end{gather*}
Si ricordano le equivalenze:
\begin{gather*}
    \lim_{N \to \infty} (1 + \frac{z}{N})^N = z\\
    \frac{t}{2} = \left\langle n \right\rangle 
\end{gather*}
\begin{gather*}
    \sum_{n = 0}^{\infty} p_n = e^{-\frac{t}{\tau}} \sum_{n=0}^{\infty} \frac{1}{n!} \left(\frac{t}{\tau}\right)^n = e^{- \frac{t}{\tau}} e^{\frac{t}{\tau}} = 1\\
    f(t) \ dt  \qquad p_0 = e^{-\frac{t}{\tau}} \qquad p_1(dt) = \frac{dt}{\tau} e^{\frac{dt}{\tau}} \approx \frac{dt}{\tau}\\
    f(t) \ dt = e^{-\frac{t}{\tau} \frac{dt}{\tau}} \qquad f(t) = \frac{1}{\tau} e^[-\frac{t}{\tau}]
\end{gather*}
Troviamo così la probabilità degli urti, questa si chiama in questo caso anche distribuzione esponenziale\\
Con $p_0$ che è la probabilità che non ci siano urti fino ad un intervallo t. e $p_1$ che è a questo punto il punto $P_0$ più un dt.
\begin{gather*}
    \left\langle t \right\rangle = \int_{0}^{\infty} dt \frac{t}{\tau} e^{-\frac{t}{\tau}} = \tau \int_{0}^{\infty} x e^{-x} \ dx = \tau\\
    \left\langle t^2 \right\rangle = \int_{0}^{\infty} dt \frac{t^2}{\tau} e^{-\frac{t}{\tau}} = \tau^2 \int_{0}^{\infty} x^2 e^{-x} \ dx = 2\tau^2\\
    \sigma^2_t = \left\langle t^2 \right\rangle - \left\langle t \right\rangle^2 = \tau^2 = \left\langle t \right\rangle^2 
\end{gather*}
Questa è la distribuzione dei tempi tra gli urti, e si suppone (più che ragionevolmente) che la distribuzione tra le distanze tra gli urti sarà analoga, quindi distribuzione esponenziale.
\begin{gather*}
    g(\lambda) = \frac{1}{e} e^{-\frac{\lambda}{e}}\\
    \left\langle \lambda \right\rangle = \int_{0}^{\infty} g(\lambda) \lambda \ d\lambda = e\\
    \left\langle \lambda^2 \right\rangle = \int_{0}^{\infty} g(\lambda) \lambda^2 \ d\lambda = 2e^2 \\
    \sigma^2_\lambda = \left\langle \lambda^2 \right\rangle - \left\langle \lambda \right\rangle^2 = e^2  
\end{gather*}
Con $\lambda$ che è la distanza tra due urti.\\
Da un urto passando al successivo cambia direzione, non è detto che la distanza sia sempre la stessa ma la probabilità sarà la stessa.\\
Questo è detto cammino aleatorio (o random walk).\\
Stiamo implicitamente supponendo che una particiella sbatte con unaltra ad una certa velocità e emergerà con una nuova, sto assumendo che la nuova velocità non dipende dalla precedente.\\
Questa ipotsi è ovviamente sbagliata perchè abbiamo a che fare con urti elastici.\\
\begin{center}
    \begin{tikzpicture}
        \draw(0,0) circle (5 pt);
        \draw(1,0) circle (5 pt);
        \draw[->](0,0) -- (-0.35,1) node[at end, left]{$\vv{v}_1^f$};
        \draw[->](1,0) -- (1.35,1)  node[at end, left]{$\vv{v}_2^f$};
        \draw(4,0) circle (5 pt);
        \draw(5,0) circle (5 pt);
        \draw[->](3.65,1) -- (4,0) node[at start, left]{$\vv{v}_1^i$};
        \draw[->](5.35,1) -- (5,0) node[at start, left]{$\vv{v}_2^i$};
    \end{tikzpicture}
\end{center}
*foto*
\begin{gather*}
    b = 2R \sin(\frac{\theta}{2})\\
    \frac{b}{2} = R \sin(\frac{\theta}{2})
\end{gather*}
C'è una relazione esplicita tra il parametro ($b$) di impatto e quello di uscita.

\newpage
\section{20/11/25}

Si lavora con una grandissima quantità di atomi (siamo sukl'ordine di $N_A \approx 6 \times 10^{23}$)\\
Abbiamo una conseguenza di diversi vettori posizione e velocità per ogni atomo nel caso di \textbf{macrostato} basato su: $p, V , T$.\\
Abbiamo quindi:
\begin{gather*}
    \vv{r_1}, \vv{r_2}, \dots, \vv{r_{N}}\\
    \vv{v_1}, \vv{v_2}, \dots, \vv{v_{N}}\\
\end{gather*}
Prendiamo ora quello che si dice un \textbf{mircostato}, ovvero una configurazione istantanea del sistema.\\
La possiamo esprimere come una funzione di \textbf{densità di probabilità}:
\begin{gather*}
    \rho(\vv{r}_1, \vv{r}_2, \dots, \vv{r_{N}}, t) d^3\vv{r}_1 d^3\vv{r}_2 \dots d^3\vv{r}_N d^3\vv{v}_1d^3\vv{v}_2 \dots d^3\vv{v}_N \\
\end{gather*}
Con $d^3$ che implica $dx dy dz$. Ovvero le tre componenti dei vettori posizione e velocità.\\
Abbiamo ora lo spazio di tutte le funzioni e di tutte le velocità che è detto \textbf{spazio delle fasi} $F$ di dimensione $6N$:
\begin{gather*}
    \rho:F \times \mathbb{R} \to \mathbb{R}\\
\end{gather*}
In caso di equilibrio termodinamico la funzione di densità di probabilità non dipende dal tempo e quindi è \underbar{risolto}.\\
Fuori dall'equilibrio termodinamico la funzione dipende dal tempo e quindi è \underbar{non risolto}. Conosciamo qualche risultato parziale ma non una soluzione generale.\\
\hfil\\
Studiamo ora il caso di equilibrio termodinamico e consideriamo le varie ipotesi:\\
\begin{enumerate}[$i)$]
    \item equilibrio $\rho$ non dipende dal tempo;
    \item omogeneità spaziale $\rho$ non dipende dalle posizioni assolute ma solo dalle differenze di posizione quindi $n= \frac{N}{V}$ costante, e invece di avere una vunzione di $6N$ variabili diventa di $3N$ variabili quindi la mia funzione diventa$\rho = \rho(\vv{v}_1 , \dots , \vv{v}_N)$;
    \item Indipendenza e identicamente distinte $\rho(\vv{v}_1 , \dots , \vv{v}_N) = \prod_{i=1}^N f((\vv{v}_i))$ 
    \item isotropia $\rho$ non dipende dalla direzione delle velocità ma solo dal loro modulo quindi:$f(\vv{v}) = f(v_x) f(v_y) f(v_z) \qquad f(\vv{v}) = f(v^2)$;
%    \item assenza di correlazioni tra posizione e velocità delle particelle;
%    \item assenza di correlazioni tra le particelle.
\end{enumerate}
Quello che possiamo assumere che nello spazio delle fasi ci sia una scala sotto la quale $\rho$ non varia più e questi valori sono completamente arbitrari.\\
Questo ci dice che non avremo mai una risoluzione esatta ma solo approssimata. Poichè abbiamo un'incertezza finita e quindi esiste un numero al disotto del quale non riusciamo più a sapere se ad esempio la variabile x cambia o meno poichè diventa infinitesima am la sua incertezza no.\\
Prendiamo ora il volume derivato da questa incerteza:
\begin{gather*}
    (\Delta x \Delta v)^{3N}
\end{gather*}
\begin{center}
    \begin{tikzpicture}
        \draw(0,0) circle(2);    
        \draw(0,0) rectangle (0.2,0.2);
        \draw(0.2,0) rectangle (0.4,0.2);
        \draw(0.4,0) rectangle (0.6,0.2);
        \draw(0.6,0) rectangle (0.8,0.2);
        \draw(0,-0.2) rectangle (0.2,0);
        \draw(0.2,-0.2) rectangle (0.4,0);
        \draw(0.4,-0.2) rectangle (0.6,0);
        \draw(0.6,-0.2) rectangle (0.8,0);
        \draw(0,-0.4) rectangle (0.2,-0.2);
        \draw(0.2,-0.4) rectangle (0.4,-0.2);
        \draw(0.4,-0.4) rectangle (0.6,-0.2);
        \draw(0.6,-0.4) rectangle (0.8,-0.2);
        \draw[->] (2,2) -- (0.1,0.1) node[at start, right] {$(\Delta \times \Delta V)^{3N}$ (di volume volume $a$)};
    \end{tikzpicture}
\end{center}
Abbiamo ora diversi volumetti all'interno dei quali la funzione resta costante e questo appunto accade quando $r$ e $v$ variano meno di $\Delta r$ e $\Delta v$. \\
Possiamo passare così da una funzione continua a una funzione discreta. dove la probabilità di cadere nel volume $a$ $\rho$ calcolato nei valori che mi daranno le coordinate che mi fanno cadere in quel volume\\
\begin{gather*}
    P_1, \dots , P_M \\
    P_a = \rho(\vv{r}_1, \dots , \vv{r}_N, \vv{v}_1, \dots , \vv{v}_N) \text{ con } \vv{r}_1, \dots , \vv{r}_N, \vv{v}_1, \dots , \vv{v}_N \in (\Delta r \Delta v)^{3N} \\
\end{gather*}
Con $M$ che è il numero di stati microscopici
Questo modello dello spazio delle fasi discreto è detto \textbf{coarse graining} e in questo caso si dice proprio che è \textbf{coarse-grained} per tradurlo sarebbe "a grana grosse"\\
Chiamo ora questi volumetti \textbf{microstati} si ha quindi:
\begin{gather*}
    p = p\left(\left\langle v^2 \right\rangle\right)  \\
    T = T\left(\left\langle v^2 \right\rangle \right) 
\end{gather*}
\begin{example}
    Prendiamo in esempio uno stato $k$-esimo con probabilità del 100\%
    \begin{gather*}
        P_k = 1 \\
        P_{a} = 0 \quad a \neq k\\
    \end{gather*}
    In questo caso abbiamo che:
    \begin{gather*}
        P_a = \frac{1}{M} \quad \forall a = 1, \dots , M
    \end{gather*}
\end{example}
Consideriamo ora l'incertezza dello stato:
\begin{gather*}
    H = - K \sum_{a=1}^M P_a \ln(P_a) = \left\langle - \ln(p) \right\rangle = \left\langle \ln \left(\frac{1}{p}\right)  \right\rangle 
\end{gather*}
Se consideriamo questa incertezza per il caso dell'esempio precedente ponendo le condizioni:
\begin{gather*}
    H = 0\\
    \text{se } p_k = 1 \text{ e } p_a = 0 \quad \forall a \neq k
\end{gather*}
\begin{gather*}
    H = - \sum_{a=1}^M  p_1 \ln\left(p_a\right) = -p_k \ln(p_k) -\sum_{a \neq k}^{M} p_a \ln\left( p_a \right) = -1 \ln(1) - 0 -\sum_{a \neq k}^{M} p_a \ln\left(p_a\right) 
\end{gather*}
Ricordandomi che
\begin{gather*}
    \lim_{x \to 0} x \ln(x) = 0
\end{gather*}
Ho trovato così un caso in cui $H$ è proprio 0.\\
Considerando ora come è costruita $H$ si vede 
che è sempre positiva $H \geq 0$ quindi col caso di prima ho trovato il minimo.\\
Cerco dunque il massimo vincolato di $H$ sul vincolo $\Sigma P_a = 1$
\begin{gather*}
    \frac{\partial F}{\partial P_k} = 0 \quad \forall k = 1 , \dots , M\\
    \frac{\partial F}{\partial \lambda} = 0\\
    \frac{\partial F}{\partial P_k} = -\frac{\partial}{\partial P_k} \sum_{a=1}^{M} P_a \ln(P_a) + \lambda \frac{\partial}{\partial P_k} \sum_{a=1}^{M} P_a\\
    \frac{\partial^2 H}{\partial P_K^2} = -\frac{d}{dp} (1 + \ln(P_k)) = -\frac{1}{P_k}\\
    P_k = \frac{1}{M}
\end{gather*}
Impongo $\nabla = 0$
\begin{gather*}
    - (\ln(P_k) + 1) + \lambda = 0\\
    \ln(P_k) = (-1 + \lambda) \quad \forall k = 1 , \dots , M\\
    \sum_{a = 1}^{M} P_a = 1 \Rightarrow P = \frac{1}{M} \text{ e } P_k = P
\end{gather*}
Ho ottenuto $P_a = \frac{1}{n}$\\
Dalla derivata seconda osservo che questo estremo è un massimo.\\
Ricapitolando: 
\begin{gather*}
    H_{max} = -\sum_{a=1}^{M} \frac{1}{M} \ln\left(\frac{1}{M}\right) = \ln(M) 
\end{gather*}
\begin{theorem}[di Sheldon]
    La funzione $H$:
    \begin{gather*}
        H = -\sum_{a = 1}^{M} P_a \ln(P_a)
    \end{gather*}
    è l'uncia con queste proprietà (a meno di moltiplicazione per costanti):
    \begin{enumerate}[$i)$]
        \item vale 0 sulla distribuzione certa ed massima sulla distribuzione uniforme
        \item è additiva per distribuzioni indipendenti
        \item monotona crescente in $M$
    \end{enumerate}
\end{theorem}
Abbiamo verificato la prima con i passaggi precedenti concentriamoci ora sulla seconda:
\begin{gather*}
    \{P_a^{(1)}\} \quad a = 1 \dots M_1\\
    \{P_a^{(2)}\} \quad a = 1 \dots M_2\\
    \pi_{ab} = P_a^{(1)} P_a{(2)}
\end{gather*}
\begin{gather*}
    H_{1+2} = -\sum_{a=1}^{M_1} \sum_{b=1}^{M_2} \pi_{ab} \ln(\pi_{ab}) = -\sum_{a=1}^{M_1} \sum_{b=1}^{M_2}\{P_a^{(1)}\} \{P_a^{(2)}\} \ln \left( \{P_a^{(1)}\}\{P_a^{(2)}\} \right) \\
    = -\sum_{a=1}^{M_1} \sum_{b=1}^{M_2} \{P_a^{(1)}\} \{P_b^{(2)}\} \left( \ln \{P_a^{(1)}\} + \ln \{P_b^{(2)}\} \right) \\
    = -\sum_{a=1}^{M_1} \sum_{b=1}^{M_2} \{P_a^{(1)}\} \{P_b^{(2)}\} \ln\left(\{P_a^{(1)}\}\right) -\sum_{a=1}^{M_1} \sum_{b=1}^{M_2} \{P_a^{(1)}\} \{P_b^{(2)}\} \ln\left(\{P_b^{(2)}\}\right) \\
    = \sum_{a=1}^{M_1} \{P_a^{(1)}\} \ln\left( \{P_b^{(2)}\} \right) \sum_{b=1}^{M_2} \{P_b^{(2)}\} - \sum_{a=1}^{M_1} \{P_a^{(1)}\} \sum_{b=1}^{M_2} \{P_b^{(1)}\} \ln\left( \{P_b^{(2)}\} \right)  \\
    = -\sum_{a=1}^{M_1} \{P_a^{(1)}\} - \sum_{b=1}^{M_2} \{P_2^{(2)}\} \ln \left(\{P_2^{(2)}\} \right) = H_1 - H_2 
\end{gather*}
\newpage
\section{21/11/25}
\begin{gather*}
    H = -\sum_{n=1}^{M} P_a \ln(P_a)
\end{gather*}
\begin{gather*}
    H[P] = \int p(x_1 \dots x_n, v_1 \dots v_n) \ln\left[ \rho(x_1, \dots , x_{3N} , v_1 ,  \dots, v_{3N}) \right]  \frac{dx_1 \dots dx_{3N} \ dv_1 \dots dv_{3N}}{(\Delta x \Delta v)^{3N}}
\end{gather*}
Queste due forme sono analoche, la seconda è consistente solo se la costante di integrazione è normalizzata.\\
Si ricorda che $H$ è una quantità adimensionale.\\
Ora guardiamo l'integrale estesu a tutti i cambi delle fasi.
\begin{gather*}
    \int  \frac{dx_1 \dots dx_{3N} \ dv_1 \dots dv_{3N}}{(\Delta x \Delta v)^{3N}} = M
\end{gather*}
Questo termine è necessario per far si che la prima e la seconda scrittura siano coerenti.\\
Si usano entrambi ma ci sono alcuni casi in cui la seconda forma è particolarmente pratica.
\hfil\\
\begin{problem}
Siamo capaci di dire come questa distribuzione di probabilità evolve nel tempo?\\
Possiamo quindi scrivere:
\begin{gather*}
    \frac{d}{dt}P_a(t) = ?
\end{gather*}
In generale questa probabilità dipende dal tempo.
\end{problem}
La intuizione generale venne da Bolzman noi ne vediamo una versione più moderna:
\begin{gather*}
    \frac{d}{dt} P_a(t) = -\sum_{b} T_{ab} P_a(t) + \sum_{b} T_{ba} P_a(t)
\end{gather*}
Generalmente questa è un ipotesi NON una conseguenza.\\
Considerando che $P_a(t)$ sono $N$ equazioni, ci chiediamo come evolve nel tempo e ASSUMIAMO di volerla scrivere in qeusto modo.\\
Quindi come una somma su tutti gli altri stati, che descrive tutti i possibili processi che all'istante $t$ avremo uno stato $a$ e a $t+dt$ saremo usciti da questo stato. 
Se ad un certo stato si ha una certa posizione e una certa velocità, all'istante successivo avrà variato sicuramente la posizione e molto probabilmente la velocità (potrebbe avere vel costante). 
Però allo stesso modo potremmo avere stati che provengono da unaltro stato e arrivano in qeullo stato.\\
Quindi questa forma ci esprime che ogni transizione ha una certa probabilità per unità di tempo che una certa quantità di atomi si toglie dallo stato che stiamo considerando o si immette nello stato che stiamo considerando.\\
Prendiamo il primo termine $[T_{ab}] = \frac{1}{[t]}$ che è la probabilità per unità di tempo che ogni atomo cambi dallo stato che sto considerando. \\
In generale stiamo considerando $m \times n$ numeri, quindi ho una \textbf{matrice di transizione} rappresentano la probabilità per unità di tempo di passare da un certo stato a un'altro.
\hfil\\
Perchè lo esprimo come un'ipotesi e non posso averla come conseguenza?\\
Lo è perchè in questa equazione non stiamo considerando cosa è accaduto prima, quindi ogni stato è scorrelato dal precedente.\\
Questa è un ipotesi \textbf{MOLTO} forte ed è detta \textbf{Ipotesi di Morkov}\\
Di fatto non interessa a questa equazione come siamo arrivati in un certo caso ma ci da solo una stima di ciò che può accadere in futuro.\\
Considerando questo la forma in cui abbiamo espresso questo concetto è di fatto il più generale possibile, questo tipo di equazioni è detta equazione molesta (moter equation)\\
\hfil\\
Partiamo da ciò che sappiamo e andiamo a scriverci tutti gli stati possibili:
\begin{gather*}
    \frac{d}{dt} \sum_{a} P_a = \sum_a \frac{d}{dt} P_a = - \sum_{a} \sum_b T_{ab} P_a +\sum_a \sum_b T_{ba} P_b = -\sum_a \sum_b T_{ba} P_a + \sum_a \sum_b T_{ab} P_a = 0
\end{gather*}
Questa condizione conserva la normalizzazione \tab\tab 1° punto \cmark\\
Ovviamente possono esistere degli stati che non sono connessi tra loro e quindi non ci sia nessun modo per passare da uno stato ad un'altro.\\
ES: un sistema completamente isolato quindi l'energia si conserva e stati ad energia diverse sono sconnessi gli uni dagli altri quando.\\
Quindi è del tutto inutile mettere nell'equazione degli stati in cui le matrici ha tutti zeri, quindi escludiamo tutti gli stati che non sono raggiungibili, ovvero quelli $\neq 0$\\
\hfil\\
Servono alcune ipotesi sulle matrici di transizione che considero (per fortuna sono semplici).\\
La prima è che queste matrici siano delle matrici simmetriche ovvero $T_{ab} = T_{ba}$, questa è l'ipotesi di \textbf{microreversibilità}\\
Questa ipotesi dice sostanzialmente che in un'unità di tempo la probabilità per passare da uno stato $b$ a uno stato $a$ sia la stessa di passare da uno stato $a$ a uno stato $b$. quindi stiamo trattando trasformazioni reversibili.\\
Questa ipotesi è molto meno forte di quella di Morkov.
Questo ci porta a poter scrivere l'equazione nel seguente modo:
\begin{gather*}
    \frac{d}{dt} P_a(t) = \sum_b T_{ab} \left[ P_b(t)-P_a(t) \right] 
\end{gather*}
Prendiamo ora due stati $P_1$ e $P_2$ a un generico istante $t$\\
\begin{gather*}
    \frac{d}{dt} P_a(t) = T_{12} \left(P_2(t) - P_1(t)\right) 
\end{gather*}
Almeno sotto questa ipotesi si vede che la probabilità si tende a stabilizzare al passare del tempo. E quindi si arriva ad uno stato di equilibrio.\\
Questo vale anche nel caso più generale oltre che a quello di solo due stati.\\
\begin{gather*}
    H(t) = -\sum_{a=1}^{N_c} P_a (t) \ln(P_a(t))\\
    \frac{dt t}{dt} = -\sum_{a=1}^{N_c} \frac{dP_a}{dt} \ln(P_a) - \sum_{a=1}^{N_c} \frac{dP_a}{dt} = '' = - \frac{d}{dt} \underbrace{\sum_{a=1}^{N_c} P_a }_{=0}\\
    \frac{dtt}{dt} = .\sum_{a=1}^{N_c} \frac{dP_a}{dt} \ln(P_a) \\
    \frac{dtt}{dt} = \sum_{a=1}^{N_c} \sum_{b=1}^{N_c} T_{ab} \left[ P_a(t)-P_b(t) \right] \ln(P_a(t))\\
    \frac{dtt}{dt} = \sum_{a=1}^{N_c} \sum_{b=1}^{N_c} T_{ab} \left[ P_b(t) - P_a(t) \right] \ln(P_b(t))\\
    \frac{dtt}{dt} = \frac{1}{2} \sum_{a=1}^{N_c} \sum_{b=1}^{N_c} T_{ab} \left[ P_a(t) -P_b(t) \right]\left[ \ln(P_a(t)) - \ln(P_b(t)) \right]  
\end{gather*}
Questa è una somma di termini sempre positivi o al massimo 0 (nel caso in cui $P_a = P_b$)\\
\begin{gather*}
    \Longrightarrow \quad \boxed{\frac{dtt}{dt} \geq 0}
\end{gather*}
Quindi $H$ è una funzione crescente.\\
Questo è il teorema H di Bolzman.\\
Questo ci diceh che se non esistono altri vincoli l'evoluzione di $H$ va verso il massimo che è uno stato uniforme, quindi allo scorrere del tempo il sistema si stabilizza.\\
Attenzione a vincoli aggiuntivi, possiamo avere un caso in cui c'è un gas ideale i quali atomi si scontrano solo con le pareti e mai tra di loro, \\
quindi ogni particiella conserva il modulo della sua velocità. Questo è un caso in cui abbiamo una grandissima quantità di vincoli aggiuntivi, e è vero che $H$ tende a crescere ma più di tanto non riesce proprio per il gran numero di vincoli.\\
In questo modo ad esempio non si troverebbe mai l'evoluzione di Maxwell, quindi la distribuzione delle probabilità non varia in nessun modo cosa che abbiamo visto che non accade affatto.\\
Infatti ci che fa stabilizzare il mio sistema sono proprio le collisioni anche nel caso di una conservazione totale, quindi l'energia non viene dispersa ma i vari atomi scontrandosi tra di loro fanno si che il sistema tende verso l'equilibrio, e questo accade quando $H$ si massimizza.\\
In generale si può pensare che la distribuzione a cui tende la probabilità sia quella uniforme, ma cene sono altre diverse che sono generate da come il mio sistema interagisce con l'esterno (come ad esempio il contatto con un termostato).
\hfil\\
Abbiamo fatto si che processi microscopici reversibili producono un risultato irreversibile.\\
Al tempo suscitò diverse critiche perchè il risultato di Bolzman mancava di precisazioni:\\
In primo luogo questo caso non è assoluto ci possono essere casi in cui per esempio $H$ diminuisce, anche se altamente imporbabili sono possibili.\\
Il fatto che $H$ varia nel tempo è dato dal fatto che siamo in un caso probabilistico, quindi non esiste una sola funzione ma esiste una banda che sono mediate e espresse come $H$.\\
\subsection{Meccanica statistica dell'equilibrio}
Possiamo definire da un punto di vista micriscopico cos'è uno stato di equilibrio termodinamico:
\begin{definition}[eq. TD dal punto di vista macroscopico]
    Uno stato in cui la distribuzione di probabilità:
    \begin{gather*}
        P_1 \dots P_M\\
        \rho(x_1, \dots , x_{3N},v_1, \dots , v_{3N})
    \end{gather*}
    Con le due scritture che sono totalmente equivalenti, e devono essere \underbar{tali che} $H$ sia massima \textbf{compatibilmente con i vincoli}\\
    Con i vincoli che sono le informazioni certe che abbiamo sul sistema.\\
    Ad esempio se sappiamo che il nostro sistema è completamente isolato.
\end{definition}
Noi conosciamo un caso di una grandezza termodinamica che in un caso piò solo crescere, ovvero l'entropia.\\
Questo è vero appunto per un sistema completamente isolato, questo ci suggerisce che ci sia una correlazione tra l'entropia e la funzione $H$.\\
Procediamo quindi a postularla e a vedere se ci porta a qualcosa di corretto.\\
Non possiamo eguagliarla poichè una è adimensionale mentre l'altra no quindi:
\begin{gather*}
    \text{\underbar{dall'equilibrio} } S = K_b H
\end{gather*}
Con $K_b$ si ricorda essere la costnte di Bolzman, facendo questo torna dimensionalmente.\\
\tiny{Non sitamo supponendo che il nostro sistema sia isolato}\normalsize\\
Quindi l'entropia $S$ diventa:
\begin{gather*}
    S = -K_b \sum_a P_a \ln(P_a)
\end{gather*}
Con le $P_a$ solo quando corrispondono all'equilibrio.\\
Riprendiamo l'identità di $H$ in forma integrale:
\begin{gather*}
    S = -K_b \int \rho(x_1, \dots , x_{3N} , v_1 , \dots , v_{3N}) \ln\left[ \rho\left(x_1, \dots , x_{3N} , v_1 , \dots , v_{3N}\right)  \right] \frac{dx_1 \dots dx_{3N dv_1 \dots dv{3N}}}{\left( \Delta x \Delta v \right)^{3N} }
\end{gather*}
\subsection{Bolzmann-Gibbs (1900)}
\begin{center}
    \begin{tikzpicture}
        \draw(0,0) rectangle (0.2,0.2);
        \draw(0.2,0) rectangle (0.4,0.2);
        \draw(0.4,0) rectangle (0.6,0.2);
        \draw(0.6,0) rectangle (0.8,0.2);
        \draw(0,-0.2) rectangle (0.2,0);
        \draw(0.2,-0.2) rectangle (0.4,0);
        \draw(0.4,-0.2) rectangle (0.6,0);
        \draw(0.6,-0.2) rectangle (0.8,0);
        \draw(0,-0.4) rectangle (0.2,-0.2);
        \draw(0.2,-0.4) rectangle (0.4,-0.2);
        \draw(0.4,-0.4) rectangle (0.6,-0.2);
        \draw(0.6,-0.4) rectangle (0.8,-0.2);
        \draw[gray!20](0.35,0) circle (1);
        \draw[red,->](0.5,-0.1) ..controls(0.4,1)and(1.6,1).. (0.5,-0.15);
        \node at(1.35,1){$F$};
        \draw[gray, ->](-1,1) -- (0.45,-0.13) node[at start, left]{stato $k$-esimo};
    \end{tikzpicture}
    \\Avrò che dopo un certo tempo, data una \textcolor{red}{trasformazione}, ritornerò sullo stato \textcolor{gray}{$k$-esimo}.
\end{center}
Immagino che ci sia una traiettoria dinamica che ad un certo tempo passa per un $k$-esimo stato, e dopo un altro tempo ci ripassa.\\
\begin{gather*}
    P_k = \frac{\Delta t_k}{\tau}\\
    \tau \to \infty
\end{gather*}
Con $P_k$ che è il $k$-esimo stato, e $\tau$ il tempo che sto considerando (ovviamente non sarà praticamente infinito ma sarà molto grande rispetto al normale).\\
Questa era la visione di Bolzman. Ed è di tipo dinamico.
\hfil\\
La probabilità all'equilibrio di stare al $k$-esimo stato è:
\begin{gather*}
    P_k = \frac{\mathcal{N}_k }{\mathcal{N}}\\
    \mathcal{N} \to \infty
\end{gather*}
Con $\mathcal{N}$ che è il numero di repliche del sistema.\\
Questa è la visione di Gibbs di fatto approccia il problema in maniera statica.
\hfil\\
Noi facciamo l'ipotesi che queste affermazioni siano equivalenti:
\begin{gather*}
    \lim_{\tau \to \infty} \frac{\Delta t_k}{\tau} = \lim_{\mathcal{N} \to \infty} \frac{\mathcal{N}_k}{\mathcal{N}} \quad \forall k
\end{gather*}
Introduciamo ora l'ipotesi ergodica: "da scrivere" \\
Qualunque sfera macroscopica $O$ si può identificare una certa $\mathcal{O}$
\begin{gather*}
    O = \overline{\mathcal{O}}_\tau = \frac{1}{\tau} \int_{0}^{\tau} \mathcal{O} \left( x_1(t), \dots, x_{3N}(t), v_1(t), \dots v_{3N}(t) \right) dt\\
    \text{Hp. erg. } \Rightarrow O = \lim_{\tau \to \infty} \overline{\mathcal{O}}_\tau = \left\langle \mathcal{O} \right\rangle = \sum_a P_a \mathcal{O}\\
    = \int \mathcal{O} (x,v) \frac{d^{3N} \times d^{3N}v}{\Delta \Gamma} \qquad (\Delta x\Delta v)^{3N} = \Delta \Gamma
\end{gather*}
$\Gamma$ è sepsso usata per identificare lo stato delle fasi complessivo.\\
In generale è del tutto ragionevole considerare l'energia interna come $U = \left\langle E(x,v) \right\rangle $
\begin{gather*}
    E = \frac{m}{2} \sum_{i=1}^{N} \left\lvert \vv{v}_i \right\rvert^2 + U^{int} \left( x_1, \dots , x_{3N} \right)  
\end{gather*}
\subsection{insieme statistico microcanonico}
Partiamo dal caso più semplice: un sistema completamente isolato con l'esterno.\\
\begin{gather*}
    V = cost. \qquad N = cost.\\
    E(x,v) = E = U
\end{gather*}
\begin{gather*}
\begin{cases}
        P_a = 0\quad \forall a,i \ E_a \neq E\\
        P_a = \frac{1}{\mathcal{N}(E,V)}
\end{cases}\\
S(U,V) = -K_b \sum_a P_a \ln(P_a)\\
= -K_b \sum_a \frac{1}{\mathcal{N}(U,V)} \ln\left(\mathcal{N}(U,V)\right) \\
= K_b \ln\left( \mathcal{N}(U,V) \right)
\end{gather*}
\begin{gather*}
    S(U,V) = K_b \ln\left(\mathcal{N}(U,V)\right) 
\end{gather*}
Con $\mathcal{N}$ che è il numero di microstati.
\begin{gather*}
    \mathcal{N} = \int_{\underset{V=V}{E(x,v) = U}} \frac{d^{3N} \times d^{3N}v}{\Delta \Gamma} = \int \delta\left[ U-E(x,v) \right] \frac{d^{3N} \times d^{3N}v}{\Delta \Gamma} 
\end{gather*}
In linea di principio risolve il problema ma calcolarlo per cose diverse dai gas perfetti è quasi impossibile.\\
Se si è in grado di risolvere questo integrale si risolve facilmente tutta la termodinamica del sistema
\begin{gather*}
    \frac{1}{T} = \left(\frac{\partial S}{\partial U}\right)_V \qquad  \frac{1}{T} = \left(\frac{\partial S}{\partial V}\right)_U
\end{gather*}
In generale si riesce a fare anche in casi poco più complessi del gas perfetto.


\end{document}

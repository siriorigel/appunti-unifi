\documentclass[a4paper, oneside]{article}
\usepackage{graphicx}
\usepackage{amsthm}
\usepackage{amsmath}
\usepackage{amssymb}
\usepackage[a4paper,
            bindingoffset=0.2in,
            left=2cm,
            right=2cm,
            top=2cm,
            bottom=2cm,
            footskip=.25in]{geometry}
\usepackage[italian]{babel}
\usepackage{pgfplots}
\usepackage{tabularx}
\usepackage{tikz-3dplot}
\usepackage{wrapfig}
\usepackage{color}
\usepackage{multicol}
\usepackage{arydshln}
\usepackage{mathtools}
\usepackage{enumerate}
\usepackage{graphicx}
\usepackage{svg}
\usepackage{cancel}
\usepackage[d]{esvect}
\usepackage[dvipsnames]{xcolor}
\usepackage{pgfplots}
\usepackage{pifont}
%\usepackage{animate}
%\usepackage{xfp} % utile se vuoi fare calcoli aggiuntivi
\pgfplotsset{compat=1.18}
\usetikzlibrary{tikzmark}
\newcommand{\TikzNCbar}[4][10pt]{
\tikz[overlay,remember picture]{\draw[#2] (#3) --++(0,-#1) -| (#4);}}

\graphicspath{ {images/} }

\definecolor{redish}{rgb}{255, 0, 30}
\definecolor{page}{rgb}{0.129,0.157,0.212}
\pagecolor{page}
\color{white}   
\graphicspath{ {./images/} }
\usetikzlibrary{shapes.geometric}
\usetikzlibrary{datavisualization}
\usetikzlibrary{datavisualization.formats.functions}
\pgfplotsset{width=10cm,compat=1.9}

\setlength\dashlinedash{0.2pt}
\setlength\dashlinegap{1.5pt}
\setlength\arrayrulewidth{0.3pt}

\newcommand\eqq{\stackrel{\mathclap{\normalfont\mbox{?}}}{=}}
\newcommand\bulletout  {\labelitemfont \textbullet}
\newcommand{\tab}{\hspace*{2em}}
\newcommand{\xmark}{
\tikz[scale=0.23] {
    \draw[line width=0.7,line cap=round] (0,0) to [bend left=6] (1,1);
    \draw[line width=0.7,line cap=round] (0.2,0.95) to [bend right=3] (0.8,0.05);
}}
\newcommand{\cmark}{
\tikz[scale=0.23] {
    \draw[line width=0.7,line cap=round] (0.25,0) to [bend left=10] (1,1);
    \draw[line width=0.8,line cap=round] (0,0.35) to [bend right=1] (0.23,0);
}}
 \newcommand{\hookbox}[1]{
\begin{center}
\hfill\break
\begin{tikzpicture}
\node[inner sep=0pt,outer sep=0pt,anchor=base] (A) {
\begin{minipage}{\dimexpr\linewidth-5em}
\centering
#1
\end{minipage}
};
% Draw the left bracket
\draw ([xshift=0pt]A.north west) -- ++(0, 0.5) -- ++(0.4, 0);
% Draw the right bracket
\draw ([xshift=0pt]A.south east) -- ++(0, -0.5) -- ++(-0.4, 0);
\end{tikzpicture}
\end{center}} 
\title{Analisi II}
\author{Gariboldi Alessandro}
\date{ }


\begin{document}

\newtheoremstyle{theoremEnv}
                {}          % Space above
                {}          % Space below
                {\slshape}  % Body font
                {}          % Indent amount
                {\bfseries} % Head font
                {.}         % Punctuation after head
                {\newline}         % Space after theorem head
                {}          % Theorem head spec
\theoremstyle{theoremEnv}

\newtheorem{definition}{Definizione}[section]
\newtheorem{theorem}{Teorema}[section]
\newtheorem{lemma}{Lemma}[section]
\newtheorem{observation}{Oss.}[section]
\newtheorem{corollary}{Corollario}[theorem]
\newtheorem{example}{Esempio}[section]
\newtheorem{problem}{Problema}[section]
\newtheorem{solution}{Soluzione}[section]
\newtheorem{proposition}{Proposizione}[section]


\maketitle
\section{ESERCIZI PRIMO PARZIALE}

\begin{enumerate}
    \item equazioni differenziali di primo grado:
    \begin{itemize}
        \item variabili separabili
        \item eq del tipo $z'+ a(x) = 0$ (metodo con $e^{A(x)}$)
        \item eq del tipo $z' + a(x) = f(x)$ (metodo con $e^{A(x)}$ o variazione dele costanti)
    \end{itemize}
    \item equazioni differenziali di secondo grado:
    \begin{itemize}
        \item \textbf{metodo generale per soluzione omogenea}
        \item metodo variazione delle costanti 
        \item metodo somiglianza
        \item studio qualitativo
    \end{itemize}
    \item eq. diff. di primo e secondo ordine a coeff. variabili
    \item es. su spazi metrici
    \item limiti in due variabili
    \item continuità, derivabilità e differenziabilità
    \item trovare massimi e minimi liberi e vincolati (classificazione dei punti critici)
    \item approssimazioni con taylor. \tab trovare rette o piani tangenti ad un punto dato
    \item esercizi funzioni implicite (teorema del dini in 2 o 3 variabili)
    \item Curve parametriche:
    \begin{itemize}
        \item se è semplice e regolare
        \item trovare la retta o piano tangente ad un punto dato
        \item calcolare la lunghezza
    \end{itemize}
    \item integrali di linea di prima specie
    \item integrali di linea di seconda specie
\end{enumerate}
    \newpage
    
    \begin{gather*}
        z' + a(x)z = 0\\\\
    \end{gather*}
    \underbar{Sia} $A(x)$ una primitiva (finita senza le costanti) di $a(x)$. Quindi $A'(x) = a(x)$\\
    \subsection{metodo con $e^{A(x)}$}
        Prendiamo l'equazione differenziale omogenea:
        \begin{gather*}
            z'+ a(x)z = 0
        \end{gather*}
        Consideriamo $A(x)$ una primitiva (fissata, senza costanti ) di $a(x)$, dunque $A'(x) = a(x)$\\
        Mi ricavo $A(x)$ calcolando l'integrale: $\int a(x) dx = A(x)$\\
        ora moltiplico per $e^{A(x)}$ e quindi ottengo:\\
        \begin{gather*}
            z' e^{A(x)} +a(x) z e^{A(x)} = 0\\
        \end{gather*}
        E da questo si sa che questa espressione è uguale a: 
        \begin{gather*}
            (e^{A(x)} z )' = 0
        \end{gather*}
        Ora posso integrare questa espressione e ottenere:
        \begin{gather*}
           z \ e^{(A(x))} = c \rightarrow z = c \ e^{-A(x)}\\ 
        \end{gather*}
        Così ci siamo ricondotti ad una forma $z = \ 'qualcosa'$ e di conseguenza risolto l'eq differenziale omogenea.
        \\\\Facciamo ora la stessa cosa con un'equazione differenziale non omogenea:\\
        \begin{gather*}
            z' + a(x) \ z = f(x)\\
        \end{gather*}
        Ricavo di nuovo $A(x)$ calcolando $\int a(x) dx$ e moltiplico sempre per $e^{A(x)}$ di conseguenza otterrò lo stesso caso sulla \emph{sinistra} ma dove prima avevo $0$ ora ho $f(x)$ e quindi avrò:\\
        \begin{gather*}
            z' \ e^{A(x)} + z \ e^{A(x)} \ a(x) = e^{A(x)} f(x)\\
        \end{gather*}
        Ora pongo: 
        \begin{gather*}
            (z \ e^{A(x)})' = f(x) e^{A(x)} \text{E integro: }\\
            z \ e^{A(x)} = \int f(x) \ e^{A(x)} dx\\
        \end{gather*}
        Adesso ho una situazione "simile" ma per ricondurmi alla forma $z = 'qualcosa'$ devo risolvere l'integrale $\int f(x) e^{A(x)} dx$, una volta fatto questo la mia equazione differenziale non omeogenea è risolta.
    \begin{gather*}
    %    e^{A(x)}z' + a(x) e^{A(x)}z = 0\\
    %    (e^{A(x)}z)' = 0\\
    %    e^{A(x)} = c \quad c \in \mathbb{R}\\
    %    z = c r^{-A(x)}\\
    %    y' e^{A(x)} + a(x) e^{A(x)} y = f(x) r^{A(x)}\\
    %    (ye?{A(x)})' = f(x) e^{A(x)}\\
    %    \text{\underbar{se}} \ F(x) \text{indica una \underbar{primitiva} di} f(x)e^{A(x)}\\
    %    ye^{A(x)} = F(x) + c \quad c \in \mathbb{R}\\
    %    \text{\fbox{$
    %    y=ce^{-A(x)}+e^{-A(x)}F(x)
    %    $}}
    \end{gather*}
    \subsection{Metodo generale risoluzione EDO ordine II}
    Avendo una equazione differenziale del II° ordine della forma:
    \begin{gather*}
        \text{\textcolor{cyan}{a}}y''(x) + \text{\textcolor{cyan}{b}}y'(x) + \text{\textcolor{cyan}{c}}y(x) = 0 \qquad \text{con \textcolor{cyan}{a},\textcolor{cyan}{b} e \textcolor{cyan}{c} numeri reali}
    \end{gather*}
    So che l'insieme delle soluzioni sarà uno spazio vetttoriale di dimensione 2\\
    Quindi la soluzione generale sarà della forma:
    \begin{gather*}
        y(x) = c_1 y_1(x) + c_2 y_2(x)
    \end{gather*}
    Dove $c_1$ e $c_2$ sono parametri liberi e $y_1(x)$ e $y_2(x)$ sono la base dello spazio delle soluzioni.\\
    \\\\$\Rightarrow$ Per trovare una base devo risolvere in $\mathbb{C}$ l'equazione caratteristica:
    \begin{gather*}
        \text{\textcolor{cyan}{a}}\lambda^2 + \text{\textcolor{cyan}{b}}\lambda + \text{\textcolor{cyan}{c}}
    \end{gather*}
    \textbf{Avrò tre casi distinti:}
    \begin{enumerate}[$\rightarrow$]
        \item 2 soluzioni reali distinte: $\lambda_1 \neq \lambda_2 \ \ \ \ \Rightarrow \ \ \ \ \ \ \ \ \ \overset{\text{BASE}}{e^{\lambda_1 x}, e^{\lambda_2 x}} \ \ \ \ \ \ \ \ \ \ \Rightarrow \ \ \ \ \overset{\text{SOLUZIONE GENERALE}}{y(x) = c_1 e^{\lambda_1 x} + c_2 e^{\lambda_2 x}}$
        \item 2 soluzioni reali coincidenti: $\lambda_1 = \lambda_2 \Rightarrow \ \ \ \ \ \ \ \ \ e^{\lambda_1 x}, xe^{\lambda_2 x} \ \ \ \ \ \ \ \ \Rightarrow \ \ \ \ \ y(x) = c_1 e^{\lambda_1 x} + c_2 xe^{\lambda_2 x}$
        \item 2 soluzioni complesse: $\lambda_{1,2} = \alpha \pm i \beta \ \Rightarrow e^{\alpha x} \cos(\beta x) , e^{\alpha x} \sin(\beta x) \Rightarrow y(x) = c_1e^{\alpha x} \cos(\beta x) + c_2 e^{\alpha x} \sin(\beta x)$
    \end{enumerate}

    \subsection{Metodo Variazione Costanti}
    Metodo generale per risolvere le equazioni differenziali del II° ordine che richiede di saper risolvere l'equazione differenziale omogenea associata.\\
    Consideriamo l'equazione generica:
    \begin{gather*}
        ay''(x) + by'(x) +cy(x) = \text{\textcolor{orange}{$f(x)$}}
    \end{gather*}
    \textbf{\underbar{Passo 1°}}: Determinare la soluzione generale dell'equazione omogenea associata:
    \begin{gather*}
        y_0(x) = c_1 \text{\textcolor{cyan}{$y_1(x)$}} + c_2 \text{\textcolor{yellow}{$y_2(x)$}}
    \end{gather*}
    \textbf{\underbar{Passo 2°}}: trovare una soluzione particolare della forma:
    \begin{gather*}
        y_p(x) = c_1(x) \textcolor{cyan}{y_1(x)} + c_2(x) \textcolor{yellow}{y_2(x)}
    \end{gather*}
    Per trovare $c_1(x)$ e $c_2(x)$ bisogna risolvere il sistema:
    \begin{gather*}
        \begin{cases*}
            c_1'(x) \text{\textcolor{cyan}{$y_1(x)$}} + c_2'(x) \text{\textcolor{yellow}{$y_2(x)$}} = 0\\
            c_1'(x) \text{\textcolor{cyan}{$\underline{y_1'(x)}$}} + c_2'(x) \text{\textcolor{yellow}{$\underline{y_2'(x)}$}} = \text{\textcolor{orange}{$f(x)$}}\\
        \end{cases*}
    \end{gather*}
    Risolvo per $c_1'(x)$ e $c_2'(x)$ e li dovrò integrare per ottenere $c_1(x)$ e $c_2(x)$.\\\\
    \textbf{\underbar{Passo 3°}}:Calcolare la soluzione generale sommando $y_0(x)$ e $y_p(x)$:
    \begin{gather*}
        y(x) = y_0(x) + y_p(x)
    \end{gather*}
    \begin{example}
        Prendiamo in esempio l'equazione: $y'' -2y + y = \frac{e^x}{x^4}$\\
        \fbox{1}: Calcolo le soluzioni dell'omogenea associata:\\
        \begin{gather*}
            \lambda^2 -2\lambda +1 = 0\Rightarrow (\lambda - 1)^2 \Rightarrow \lambda_1 = \lambda_2 = 1\\
            y_0(x) = c_1 \text{\textcolor{cyan}{$e^x$}} + c_2 \text{\textcolor{yellow}{$xe^x$}}
        \end{gather*}
        \fbox{2}: Risolvo il sistema:\\
        \begin{gather*}
            \begin{cases}
                c_1' (x) \text{\textcolor{cyan}{$e^x$}} + c_2'(x) \text{\textcolor{yellow}{$xe^x$}} = 0\\
                c_1' (x) \underset{[e^x]'}{\text{\textcolor{cyan}{$e^x$}}} + c_2'(x) \underset{[xe^x]'}{\text{[\textcolor{yellow}{$e^x+ xe^x$}]}} = \text{\textcolor{orange}{$\frac{e^x}{x^4}$}}\\
            \end{cases}\\
            \Rightarrow c_1' = \frac{-c_2'(x) x \cancel{e^x}}{\cancel{e^x}} = -x c_2'(x)\\
            \cancel{-xc_2'(x)}+ c_2'(x) \cancel{e^x}+ \cancel{c_2'(x) x e^x} = \frac{e^x}{x^4}\\
            \Rightarrow \begin{cases}
                c_1'(x) = -\frac{1}{x^3}\\
                c_2'(x) = \frac{1}{x^4}\\
            \end{cases}
        \end{gather*}
        Ora procediamo a integrare:
        \begin{gather*}
            c_1(x) = \int -\frac{1}{x^3} dx = \frac{1}{2x^2}\\
            c_2(x) = \int \frac{1}{x^4} dx = \frac{1}{3x^3}\\
        \end{gather*}
        Da qui si trova $y_p$:
        \begin{gather*}
            y_p(x) = \frac{1}{2X^2}e^x - \frac{1}{3x^3} xe^x = \frac{e^x}{6x^2}
        \end{gather*}
        \fbox{3}: metto insieme soluzione dell'omogenea associata e soluzione particolare:
        \begin{gather*}
            y(x) = y_0(x) + y_p(x)\\
            c_1 e^x + c_2 xe^x + \frac{e^x}{6x^2}
        \end{gather*}

    \end{example}
    %\begin{gather*}
    %    y(x) = c \ e^{A(x)} = \text{\fbox{$c(x)e^{-A(x)}$}}\\
    %    y' = c' e^{-A(x)} - ce^{-A(x)} a\\
    %    y' + a(x) y= f\\
    %    c' e^{-A(x)} - a c e^{-A(x)} + a(x) c e^{-A(x)} = f\\
    %    c' = f(x) e^{A(x)} \text{cioè se } c(x) \text{è una primitiva di } f(x) e^{A(x)}\\
    %    \text{Cioè se } c = F(x) + d \quad d \in \mathbb{R}\\
    %    \text{Quindi le soluzioni sono: }\\
    %    y(x) = c(x)e^{-A(x)} \text{cioè } y(x) = (F(x) + d)e^{-A(x)}
    %\end{gather*}   

\subsection{Metodo di somiglianza}
    È un metodo per risolvere equazioni differenziali del II° ordine del tipo:
    \begin{gather*}
        y'' + a y' + b y = P(x) e^{\Lambda x} \underset{\sin(\mathcal{B} x)}{\cos(\mathcal{B} x)}
    \end{gather*}
    Per prima cosa come di consueto si passa a risolvere la omogenea associata:
    \begin{gather*}
        y_0(x) = c_1 y_1(x) + c_2 y_2(x)
    \end{gather*}
    In cui si ricorda $y_1(x)$ e $y_2(x)$ sono generate in base alle soluzioni dell'equazione caratteristica, e più in particolare per quello che ci interessa alle soluzioni di quest'ultima:\\ 
    $\lambda_1$ e $\lambda_2$ oppure soluzioni complesse $\alpha \pm i \beta$\\
    Ora si guarda la funzione nota a destra dell'uguale e consideriamo i diversi casi in cui le varie tipologie di soluzionicoincidono o meno con $\Lambda$ e $\mathcal{B}$:
    \begin{gather*}
        f(x) e^{\lambda x} P(x) \rightarrow
        \begin{cases}
            \lambda_{1,2} \neq \Lambda \land \beta = 0 \to y_p(x) = \overline{P}(x) \ e^{\lambda x} \\
            \text{(soluz. distinte) }\lambda_1 = \Lambda \land \lambda_2 = \Lambda \land \beta = 0 \to y_p(x) = x \ \overline{P}(x) \ e^{\lambda x} \\
            \text{(soluz. coincidenti) }\lambda_1 = \lambda_2 = \Lambda \land \beta = 0 \to y_p(x) = x^2 \ \overline{P}(x) \ e^{\lambda x} \\
            \alpha \neq \Lambda \lor \beta \neq \mathcal{B} \to y_p(x) = e^{\Lambda x} \ \left[ \ \overline{P_1}(x) \cos(\mathcal{B} x) + \overline{P_2}(x) \sin(\mathcal{B} x) \ \right]\\
            \alpha = \Lambda \land \beta = \mathcal{B} \to y_p(x) = x \ e^{\Lambda x} \ \left[ \ \overline{P_1}(x) \cos(\mathcal{B} x) + \overline{P_2}(x) \sin(\mathcal{B} x) \ \right]\\
        \end{cases}
    \end{gather*}
    Dove $\overline{P_1}(x)$ e $\overline{P_2}(x)$ sono dei polinomi generici dello stesso grado di $P(x)$ (che è il polinomio noto a destra dell'uguale dell'equazione) con coefficienti da determinare.\\
    Per determinarli si calcolano $y_p'(x)$ e $y_p''(x)$ e si sostituiscono nell'equazione differenziale di partenza.\\
    Dopodichè si raccoglie a secondo i termini in $\cos(\mathcal{B} x)$ e $\sin(\mathcal{B} x)$ (se presenti) e si uguaglia i coefficienti a quelli di $f(x) e^{\Lambda x} P(x)$ per trovare i coefficienti dei polinomi.\\
    Infine la soluzione generale sarà data da:
    \begin{gather*}
        y(x) = y_0(x) + y_p(x)
    \end{gather*}



    \begin{example}
        Prendiamo l'equazione differenziale:
        \begin{gather}
            y'' -2y' +2y = 5 \sin(x)
        \end{gather}
        \fbox{1}: Risolvo l'omogenea associata:
        \begin{gather*}
            \lambda^2 -2\lambda +2 = 0 \Rightarrow \lambda_{1,2} = 1 \pm i\\
            y_0(x) = c_1 e^{x} \cos(x) + c_2 e^{x} \sin(x)
        \end{gather*}
        \fbox{2}: Cerco la soluzione particolare.\\
        Qui $\alpha = 1 \neq 0 = \Lambda$ e $\beta = 1 \neq 1 = \mathcal{B}$ quindi:
        \begin{gather*}
            y_p(x) = \overline{P_1}(x) \cos(x) + \overline{P_2}(x) \sin(x)\\
            \text{con } \overline{P_1}(x) = A \quad \overline{P_2}(x) = B \quad A,B \in \mathbb{R}\\
        \end{gather*}
        Siccome $P(x)$ è un polinomio di grado 0 (una costante) allora anche $\overline{P_1}(x)$ e $\overline{P_2}(x)$ saranno di grado 0 quindi costanti generiche $A$ e $B$.\\
        Inoltre tutto sarebbe moltiplicato per $e^\Lambda$ ma siccome $\Lambda = 0$ avremmo $e^0$ che è 1 e quindi si omette.\\
        Procediamo ora con calcolare le derivate della nostra $y_p(x)$\\
        \begin{gather*}
            y_p(x) = A \cos(x) + B \sin(x)\\
            y_p'(x) = -A \sin(x) + B \cos(x)\\
            y_p''(x) = -A \cos(x) - B \sin(x)\\
        \end{gather*}
        Ora si sostituiscono queste espressioni nell'equazione (1) e si raccolgono i termini con $\cos(x)$ e $\sin(x)$:
        \begin{gather*}
            (-A \cos(x) - B \sin(x)) -2(-A \sin(x) + B \cos(x)) + 2(A \cos(x) + B \sin(x)) = 5 \sin(x)\\
            \overset{*}{-A \cos(x)} - \overset{o}{B \sin(x)} + \overset{o}{2A \sin(x)} - \overset{*}{2B \cos(x)} + \overset{*}{2A \cos(x)} + \overset{o}{2B \sin(x)} = 5 \sin(x)\\
            ( A - 2B ) \cos(x) + ( 2A + B ) \sin(x) = 5 \sin(x)\\
        \end{gather*}
        Da qui si può impostare il sistema, poichè vediamo che per essere soddisfatta il termine del coseno deve essere 0, e il termine del seno deve essere 5:
        \begin{gather*}
            \begin{cases}
                A - 2B = 0\\
                2A + B = 5\\
            \end{cases}
            \begin{cases}
                A = 2B\\
                4B + B = 5\\
            \end{cases}\to B = 1 \to A = 2
        \end{gather*}
        \fbox{3} infine calcolo la soluzione generale esatta:
        \begin{gather*}
            y(x) = y_0(x) + y_p(x)\\
            = c_1 e^{x} \cos(x) + c_2 e^{x} \sin(x) + 2 \cos(x) + 1 \sin(x)\\
        \end{gather*}
    \end{example}


\newpage
\subsection{limiti in due variabili}



        \newpage
        \subsection{\textbf{Metodo per trovare massimi e minimi}:}
            fare le derivate parziali rispetto a $x$ e $y$ e porle in un sistema uguale a 0 (di fatto pongo il \underbar{gradiente} uguale a 0)\\
            \begin{gather*}
                \begin{cases}
                    \frac{\partial f}{\partial x} = f_x(x,y) = 0\\
                    \frac{\partial f}{\partial y} = f_y(x,y) = 0
                \end{cases}
            \end{gather*}
            Risolvere il sistema e trovare i punti in cui il \underbar{gradiente} si annulla:\\
            \begin{gather*}
                \text{es. } \textcolor{cyan}{P_1} = (\textcolor{cyan}{x_1},\textcolor{cyan}{y_1}),\textcolor{yellow}{P_2} = (\textcolor{yellow}{x_2},\textcolor{yellow}{y_2})
            \end{gather*}
            Calcolare le derivate miste e le derivate seconde rispetto a $x$ e $y$.
            \begin{gather*}
                \frac{\partial^2 f}{\partial x^2} = \textcolor{BrickRed}{f_1}(x,y)\\
                \frac{\partial^2 f}{\partial y^2} = \textcolor{magenta}{f_2}(x,y)\\
                \frac{\partial^2 f}{\partial yx} = \textcolor{green}{f_3}(x,y)\\
                \frac{\partial^2 f}{\partial xy} = \textcolor{orange}{f_4}(x,y)\\
            \end{gather*}
            Scriversi la matrice hessiana:
            \begin{gather*}
                \begin{pmatrix}
                    \textcolor{BrickRed}{f_1}(x,y) & \textcolor{green}{f_3}(x,y)\\
                    \textcolor{orange}{f_4}(x,y) & \textcolor{magenta}{f_2}(x,y)
                \end{pmatrix}
            \end{gather*}
            Calcolare il \textcolor{Emerald}{determinante} dell'hessiano nei punti $\textcolor{cyan}{P_1}$ e $\textcolor{yellow}{P_2}$\\
            si ricorda il \textcolor{Emerald}{determinante} di una matrice $2\times2$ è:
            \begin{gather*}
                \begin{pmatrix}
                    a & b\\
                    c & d
                \end{pmatrix} = ad-bc
            \end{gather*}
            \begin{gather*}
                \det
                \begin{pmatrix}
                    \textcolor{BrickRed}{f_1}(\textcolor{cyan}{x_1},\textcolor{cyan}{y_1}) & \textcolor{green}{f_3}(\textcolor{cyan}{x_1},\textcolor{cyan}{y_1})\\
                    \textcolor{orange}{f_4}(\textcolor{cyan}{x_1},\textcolor{cyan}{y_1}) & \textcolor{magenta}{f_2}(\textcolor{cyan}{x_1},\textcolor{cyan}{y_1})
                \end{pmatrix} = \text{\textcolor{Emerald}{\text{\fbox{\textcolor{white}{$d_1$}}}}}\qquad
                \det
                \begin{pmatrix}
                    \textcolor{BrickRed}{f_1}(\textcolor{yellow}{x_2},\textcolor{yellow}{y_2}) & \textcolor{green}{f_3}(\textcolor{yellow}{x_2},\textcolor{yellow}{y_2})\\
                    \textcolor{orange}{f_4}(\textcolor{yellow}{x_2},\textcolor{yellow}{y_2}) & \textcolor{magenta}{f_2}(\textcolor{yellow}{x_2},\textcolor{yellow}{y_2})
                \end{pmatrix} = \text{\textcolor{Emerald}{\text{\fbox{\textcolor{white}{$d_2$}}}}}
            \end{gather*}
            \underbar{caso I}: \textcolor{Emerald}{Determinante} \textbf{positivo} $\Rightarrow$ ho \textbf{min/max}\\
            \underbar{caso II}: \textcolor{Emerald}{Determinante} \textbf{negativo} $\Rightarrow$ ho punto di \textbf{sella}\\\\
            Per vedere se nel \underbar{caso I} si ha punti di min. o max. si vede se la \textbf{\underbar{\textcolor{Orchid}{traccia}}} della matrice è positiva o negativa:
            \begin{gather*}
                tr
                \begin{pmatrix}
                    \textcolor{BrickRed}{f_1}(\textcolor{cyan}{x_1},\textcolor{cyan}{y_1}) & \textcolor{green}{f_3}(\textcolor{cyan}{x_1},\textcolor{cyan}{y_1})\\
                    \textcolor{orange}{f_4}(\textcolor{cyan}{x_1},\textcolor{cyan}{y_1}) & \textcolor{magenta}{f_2}(\textcolor{cyan}{x_1},\textcolor{cyan}{y_1})
                \end{pmatrix} = \text{\textcolor{Orchid}{\text{\fbox{\textcolor{white}{$t_1$}}}}}\qquad
                tr
                \begin{pmatrix}
                    \textcolor{BrickRed}{f_1}(\textcolor{yellow}{x_2},\textcolor{yellow}{y_2}) & \textcolor{green}{f_3}(\textcolor{yellow}{x_2},\textcolor{yellow}{y_2})\\
                    \textcolor{orange}{f_4}(\textcolor{yellow}{x_2},\textcolor{yellow}{y_2}) & \textcolor{magenta}{f_2}(\textcolor{yellow}{x_2},\textcolor{yellow}{y_2})
                \end{pmatrix} = \text{\textcolor{Orchid}{\text{\fbox{\textcolor{white}{$t_2$}}}}}
            \end{gather*}
            Se la \textcolor{Orchid}{traccia}($t_1/t_2$) è \textbf{positiva} $\Rightarrow$ ho un punto di \textbf{min.} (ho un paraboloide definito positivo)\\
            Se la \textcolor{Orchid}{traccia}($t_1/t_2$) è \textbf{negativa} $\Rightarrow$ ho un punto di \textbf{max.} (ho un paraboloide definito negativo)\\
\newpage
\subsection{esercizi dini}
\underline{Richiamo teorema e oss. utili sul DINI:}
\begin{theorem}[del DINI]
    $A_{ap} \subseteq \mathbb{R}^2 \ F:A\to \mathbb{R}, F \in C^1 (A)$\\
    \underbar{sia} $P_0 \equiv (x_0,y_0) \in A \ t.c. \ F(x_0,y_0) = 0 , P_0 $ Regolare (cioè $DF(x_0,y_0)\neq 0$)\\
    \underbar{Allora} $\exists U$ intorno di $x_0$ e $\exists V$ intorno di $y_0$ $ \ t.c. \ F(x,y) = 0$ definisce un grafico di funzione $y= f(x)$ oppure $x = g(y)$ in $U\times V$ intorno di $P_0$
\end{theorem}
\underbar{\textbf{In particolare:}}\\
\begin{itemize}
    \item  se $F_y(x_0,y_0) \neq 0$ allora $\exists! y = f(x):U\to V \ t.c. \ F(x,f(x)) = 0$, $f \in C^1(U)$ e $ f'(x) = -\frac{F_x(x,f(x))}{F_y(x,f(x))} \quad \forall x \in \overset{\circ}{U}$
    \item se $F_x(x_0,y_0) \neq 0$ allora $\exists! x = g(y): V \to U \ t.c. \ F(g(y),y) = 0$, $g \in C^1(V)$ e $g'(y) = -\frac{F_y(g(y),y)}{F_x(g(y),y)} \quad \forall y \in \overset{\circ}{V}$
\end{itemize}
$F(x,y) = x^2+y^2-1$\\
$DF = \begin{pmatrix}
    2x\\
    2y
\end{pmatrix}$\\
punti critici $\Leftrightarrow DF = \underbar{0} \Leftrightarrow P \equiv(0,0) \cancel{\in}$ linea $(F(x,y) = 0) \Rightarrow $ Tutti i punti di $F(x,y) = 0$ sono regolari.\\
\begin{observation}
    $F \in C^2$ ( e vale teorema Dini) $\Rightarrow f \in C^2$\\
    Da cui $f(x) = \underset{\textcolor{cyan}{\underset{y_0}{||}}}{f(x_0)} = f(x_0) + \underset{\textcolor{yellow}{= -\frac{F_x(x_0,y_0)}{F_y(x_0,y_0)}}}{f'(x_0)}(x-x_0) + \textcolor{orange}{\boxed{\textcolor{white}{\frac{f''(x_0)}{2}}}}(x-x_0)^2 + o((x-x_0)^2)$
    per $x \in I_{x_0}$
    \begin{gather*}
        f''(x) = \left(-\frac{F_x(x,f(x))}{f_y(x,f(x))}\right)' = -\frac{(F_x(x,f(x)))'F_y(x,f(x))-F_x(x,f(x))(F_y(x,f(x)))'}{\left(F_y(x,f(x))\right)^2 }\\
        = -\frac{1}{(F_y)^2}[(F_{xx} + F_{xy}f')F_y-F_x(F_{\textcolor{yellow}{xy}}+F_{yy}f')]\vert_{x,f(x)} 
    \end{gather*}
\end{observation}
\subsubsection{caso 2 variabili:}
In generale si ha sempre un punto ($P_0$) in cui calcolare una funzione data, a volte è chiesta apprissimazione di taylor o piano tangente in quel punto,oppure vedere se il punto è un massimo o minimo per quella funzione\\
Per prima cosa si fanno le derivate parziali della funzione data, $F(x,y)$ e si calcola nel punto dato:
\begin{gather*}
    \frac{\partial F}{\partial x}(P_0) \qquad \text{se } = 0 \rightarrow  \exists \text{ una param. per cui } y=f(x) \text{ è un grafico/ funzione di } x\\
    \frac{\partial F}{\partial y}(P_0) \qquad \text{se } = 0 \rightarrow  \exists \text{ una param. per cui } x=g(y) \text{ è un grafico/ funzione di } y
\end{gather*}
Adesso sia nel caso sia richiesto se sia un massimo/minimo sia nel caso di piano tangente/taylor si deve procedere calcolando la derivata prima e seconda calcolate in $P_0$, per farlo si applica il teorema del dini, che ci dice che vale:
\begin{gather*}
    f'(x) = -\frac{F_x(x,f(x))}{F_y(x,f(x))} = 0 \text{ (la derivata risp. a $x$ calcolata in $P_0$)} \rightarrow P_0 \text{ è max o min della funzione } f(x)\\
    g'(y) = -\frac{F_y(y,g(y))}{F_x(y,g(y))} = 0 \text{ (la derivata risp. a $y$ calcolata in $P_0$)} \rightarrow P_0 \text{ è max o min della funzione } g(y)
\end{gather*}
Fare attenzione al segno meno.\\
Ora per la derivata seconda si può usare:
\begin{gather*}
    f''(x) = -\frac{1}{(F_y)^2}[(F_{xx} + F_{xy}f')F_y-F_x(F_{\textcolor{yellow}{xy}}+F_{yy}f')]\vert_{x,f(x)}\\%g''(y) = -\frac{1}{(F_y)^2}[(F_{xx} + F_{xy}f')F_y-F_x(F_{\textcolor{yellow}{xy}}+F_{yy}f')]\vert_{x,f(x)}\\
\end{gather*}
In generale si può anche derivare la propria funzione ma questa è comunque molto veloce se si ha a disposizione, ovviamente per utilizzarla si deve calcolare le derivate parziali seconde:
\begin{gather*}
    F_{xx} = \frac{\partial^2 F}{\partial x^2}\\
    F_{xy} = \frac{\partial^2 F}{\partial x \partial y}\\
    F_{yy} = \frac{\partial^2 F}{\partial y^2}\\
    F_{yx} = \frac{\partial^2 F}{\partial y \partial x}\\
\end{gather*}
A questo punto si ha sia la derivata prima che seconda del grafico di $f(x) \ / \ g(y)$ e si può facilmente sapere se il punto è di max o di min (grazie alla derivata seconda pos./neg. per la concavità), o scrivere il piano tangente/appross. con taylor (si spera ci si fermi al secondo ordine altrimenti c'è da derivare quel mostro)
\begin{gather*}
    \text{piano tangente in } P_0 \Rightarrow                 z = <Df(x_0,y_0);\begin{pmatrix}
                    x-x_0\\
                    y-y_0
                \end{pmatrix}> + \underset{\overset{||}{f(x_0,y_0)}}{z_0}\\
    \text{appros. al II ordine con Taylor in } P_0 \Rightarrow f(x_0)+f'(x_0)(x-x_0) + \frac{f''(x_0)}{2} (x-x_0)^2 + o((x-x_0)^2)\\
    \text{punto max/min in } P_0 \Rightarrow f''(x_0)/g''(y_0) < 0 \to \text{max} \quad f''(x_0)/g''(y_0) > 0 \to \text{min}
\end{gather*}
Dove $x_0,y_0$ sono le coordinate di $P_0$ ovvero $(x_0,y_0)$
\subsubsection{caso a 3 variabili}
È di fatto analogo al caso a due si ricorda che il punto $P_0$ ha tre coordinate $(x_0,y_0,z_0)$ e si devono fare le derivate parizali prime e seconde anche rispetto a z.
\begin{theorem}
    Dini in 3 variabili\\
    $F(x,y,z) \in C^1(A) \quad A_{ap} \subseteq \mathbb{R}^3$\\
    $P_0 \equiv (x_0,y_0,z_0) \in A$ Regolare e $t.c. \ F(x_0,y_0,z_0) = 0$\\
    supponiamo $F_z(x_0,y_0,z_0) \neq 0$ \underbar{allora}  $\exists \mathcal{U}_{P_0} \subseteq A$\\
    $t.c. \ \{F(x,y,z) = 0\} \cap \mathcal{U}_{P_0}$ è grafico $z=f(x,y)$\\
    inoltre $f\in C^1$ e:\\
    $\frac{\partial f}{\partial x} f(x,y) = -\frac{F_x(x,y,f(x,y))}{F_z(x,y,f(x,y))}$ localmente vicino a $(x_0,y_0)$\\
    $\frac{\partial f}{\partial y} f(x,y) = -\frac{F_y(x,y,f(x,y))}{F_z(x,y,f(x,y))}$
\end{theorem}
\newpage
\subsection{curve parametriche}
Per questi esercizi iene data una curva parametrica del tipo:
\begin{gather}
    \varphi(t) = \begin{pmatrix}
        x(t)\\
        y(t)\\
        z(t)
    \end{pmatrix} t \in [a,b]
\end{gather}
In generale è richiesto:
\begin{enumerate}[$i)$]
    \item provare che una curva è regolare
    \item calcolarne la lunghezza
\end{enumerate}
\textbf{Punto $i)$}\\
Una curva è regolare se è \textbf{semplice} (è implicita) e se la sua \textbf{derivata è $\neq \underline{0}$} (è diversa dal vettore nullo).\\
Quindi per questo punto basta verificare queste due ipotesi.
\textbf{punto $ii)$}\\
Per calcolare la lunghezza si usa la definizione:
\begin{gather}
    \mathcal{L}(\varphi) = \int_{a}^{b} \left\lVert \underline{\varphi}'(t) \right\rVert dt = \int_{a}^{b} \sqrt{\dot{x}^2(t) +\dot{y}^2(t) + \dot{z}^2(t)} dt
\end{gather}
\subsection{integrali curvilinei di prima specie}
Per la "forma base" di un integrale curvilineo si intende l'area del sottografico di una curva parametrica ad una funzione (ad esempio $f(x,y)$), ed è:
\begin{gather}
    \int_{a}^{b} f(x(t),y(t)) \sqrt{\dot{x}^2(t) + \dot{y}^2(t)} dt = \int_{\gamma} f(x,y) ds
\end{gather}
Si possono avere casi analoghi a tre o piu variabili. In generale questo descrive il volume della curva parametrica il che viene usato anche in altre applicazione.\\
Data una curva parametrica integrali curvilinei di seconda spece si possono presentare anche come il calcolo del centro geometrico di una curva.\\
Per questi casi viene data una funzione che descrive la densita della curva: $\rho(t)$.\\
Il modo per trovare le coordinate del centro geometrico $G\equiv(x_G,y_G,z_G)$ è il seguente:
\begin{gather}
    x_G  = \frac{1}{m_\gamma} \int_{\gamma} \rho x ds = \frac{1}{m_\gamma} \int_{a}^{b} \rho(\underline{\varphi}(t)) x(t) \left\lVert \underline{\dot{\varphi}}(t) \right\rVert dt \\
    y_G  = \frac{1}{m_\gamma} \int_{\gamma} \rho y ds = \frac{1}{m_\gamma} \int_{a}^{b} \rho(\underline{\varphi}(t)) y(t) \left\lVert \underline{\dot{\varphi}}(t) \right\rVert dt \\
    z_G  = \frac{1}{m_\gamma} \int_{\gamma} \rho z ds = \frac{1}{m_\gamma} \int_{a}^{b} \rho(\underline{\varphi}(t)) z(t) \left\lVert \underline{\dot{\varphi}}(t) \right\rVert dt \\
\end{gather}




\subsection{integrali curvilinei di seconda specie}
\subsubsection{caso con campi vettoirali}
Dati una curva $\gamma$ con param. $\varphi:[a,b] \to A$ e un campo vettoriale $F$
\begin{gather}
    \int_{a}^{b} \left\langle F(\varphi(t)) ;  \dot{\varphi}(t)\right\rangle dt \\
    \int_{a}^{b} \left\langle F,\tau \right\rangle ds 
\end{gather}
come input a $F$ si mettono le componenti delle curve, al posto di x e y
\subsubsection{caso con forme differenziali}
Data una forma differenziale $\omega(x,y) = f(x,y)dx + g(x,y)dy$ e una curva $\gamma$ e relativa param. $\varphi[a,b] \to A$
\begin{gather}
    \int_{\gamma} \omega = \int_{a}^{b} \omega(\varphi(t))\left[ \frac{\dot{\varphi}(t)}{\left\lVert \cancel{\dot{\varphi}(t)} \right\rVert } \right] \cancel{\left\lVert \dot{\varphi}(t) \right\rVert } \ dt \\ 
    = \int_a^b \left( a_1(\varphi(t))\ dx_1 + a_2(\varphi(t))\ dx_2 + a_3(\varphi(t))\ dx_3 \right) \text{[applicato a]} \ \left[ \dot{\varphi_1}(t),\dot{\varphi_2}(t),\dot{\varphi_3}(t) \right] \ dt  
\end{gather}
Al posto di $dx$ e $dy$ si sostituiscono le derivate delle componenti della curva, e al posto di $x$ e $y$ si sostituiscono componenti della curva non derivata

\section{Integrali doppi}
Dato un integrale dopio e il suo dominio del tipo:
\begin{gather}
    \iint_\mathbb{D}  \qquad \mathbb{D} = \{ (x,y) \in \mathbb{R}^2: \quad g_1(x) \leq/\geq a_1 \dots g_n \leq / \geq a_n\}
\end{gather}
con $g_i(x)$ vincolo generico e $a_i \int \mathbb{R}$\\
Un esempio può essere $\mathbb{D} = \{ (x,y) \in \mathbb{R}^2: \quad x+y \leq 1\}$
Si procede a scrivere il dominio come x-semplice o y-semplice $\Rightarrow$ Bisogna dividere l'area in un numero finito di insiemi semplici.\\
Fatto ciò, posso passare all'integrale nella forma:
\[\int_{a}^{b}dx \ \int_{g_1(x)}^{g_2(x)}f(x,y,)dy\]

\end{document}
\documentclass[a4paper, oneside]{article}
\usepackage{wrapfig}
\usepackage{graphicx}
\usepackage{amsthm}
\usepackage{amsmath}
\usepackage{amssymb}
\usepackage[a4paper,
            bindingoffset=0.2in,
            left=2cm,
            right=2cm,
            top=2cm,
            bottom=2cm,
            footskip=.25in]{geometry}
\usepackage[italian]{babel}
\usepackage{pgfplots}
\usepackage{tabularx}
\usepackage{tikz-3dplot}
\usepackage{wrapfig}
\usepackage{color}
\usepackage{multicol}
\usepackage{arydshln}
\usepackage{mathtools}
\usepackage{enumerate}
\usepackage{graphicx}
\usepackage{svg}
\usepackage{cancel}
\usepackage[d]{esvect}
\usepackage[dvipsnames]{xcolor}
\usepackage{pgfplots}
\usepackage{pifont}
\usetikzlibrary{patterns}
\makeindex
%\usepackage{animate}
%\usepackage{xfp} % utile se vuoi fare calcoli aggiuntivi
\pgfplotsset{compat=1.18}
\usetikzlibrary{tikzmark}
\newcommand{\TikzNCbar}[4][10pt]{
\tikz[overlay,remember picture]{\draw[#2] (#3) --++(0,-#1) -| (#4);}}

\graphicspath{ {images/} }

\definecolor{redish}{rgb}{255, 0, 30}
\definecolor{page}{rgb}{0.129,0.157,0.212}
\pagecolor{page}
\color{white}   
\graphicspath{ {./images/} }
\usetikzlibrary{shapes.geometric}
\usetikzlibrary{datavisualization}
\usetikzlibrary{datavisualization.formats.functions}
\pgfplotsset{width=10cm,compat=1.9}

\setlength\dashlinedash{0.2pt}
\setlength\dashlinegap{1.5pt}
\setlength\arrayrulewidth{0.3pt}

\newcommand\eqq{\stackrel{\mathclap{\normalfont\mbox{?}}}{=}}
\newcommand\bulletout  {\labelitemfont \textbullet}
\newcommand{\tab}{\hspace*{2em}}
\newcommand{\xmark}{
\tikz[scale=0.23] {
    \draw[line width=0.7,line cap=round] (0,0) to [bend left=6] (1,1);
    \draw[line width=0.7,line cap=round] (0.2,0.95) to [bend right=3] (0.8,0.05);
}}
\newcommand{\cmark}{
\tikz[scale=0.23] {
    \draw[line width=0.7,line cap=round] (0.25,0) to [bend left=10] (1,1);
    \draw[line width=0.8,line cap=round] (0,0.35) to [bend right=1] (0.23,0);
}}
% Comando:
%   \potato[opzioni]{(x,y)}{scala}
%
% Opzioni = facoltative (es. fill=red!20, draw=black, thick)
% (x,y)   = centro della patata
% scala   = fattore di scala
%
\def\potatoshape{
  (1,0) (2,1.5) (1.6,3) (0.3,2.7) (-0.4,1.2)
}
\newcommand{\potato}[3][draw=white]{
  \begin{scope}[shift={#2}, scale=#3]
    \draw[#1]
      plot [smooth cycle, tension=1]
      coordinates {\potatoshape};
  \end{scope}
}
 \newcommand{\hookbox}[1]{
\begin{center}
\hfill\break
\begin{tikzpicture}
\node[inner sep=0pt,outer sep=0pt,anchor=base] (A) {
\begin{minipage}{\dimexpr\linewidth-5em}
\centering
#1
\end{minipage}
};
% Draw the left bracket
\draw ([xshift=0pt]A.north west) -- ++(0, 0.5) -- ++(0.4, 0);
% Draw the right bracket
\draw ([xshift=0pt]A.south east) -- ++(0, -0.5) -- ++(-0.4, 0);
\end{tikzpicture}
\end{center}} 
\title{Fisica II}
\author{Gariboldi Alessandro}
\date{ }


\begin{document}

\newtheoremstyle{theoremEnv}
                {}          % Space above
                {}          % Space below
                {\slshape}  % Body font
                {}          % Indent amount
                {\bfseries} % Head font
                {.}         % Punctuation after head
                {\newline}         % Space after theorem head
                {}          % Theorem head spec
\theoremstyle{theoremEnv}

\newtheorem{definition}{Definizione}[section]
\newtheorem{theorem}{Teorema}[section]
\newtheorem{lemma}{Lemma}[section]
\newtheorem{observation}{Oss.}[section]
\newtheorem{corollary}{Corollario}[theorem]
\newtheorem{example}{Esempio}[section]
\newtheorem{problem}{Problema}[section]
\newtheorem{solution}{Soluzione}[section]
\newtheorem{proposition}{Proposizione}[section]


\maketitle
\section{Introduz. al corso}
\date{23/02/26}\\

professori:
\begin{itemize}
\item Prof. Massimo Gurioli (@unifi.it)
\item Prof. Giovanni Ferioli (@unifi.it)
\item Prof. Francesco Biccari (@unifi.it)
\end{itemize}
Libri:
\begin{itemize}
    \item C. Mencuccini, V. Silvestrini
    \item Fisica 2 Liguori editore J.D. Jackson
    \item Classical Electrodynamics 3rd edition.
\end{itemize}
Modalità d'esame:\\
Lo scritto ha validità di un anno, si passa con 18, con 17 anche ma si fa un esercizio all'orale.\\
Ci sono due parziali per sostituire che gli scritti , 16 a in ciascuno, e 18 in media.\\
Ogni scritto ha 3 esercizi con tre domande ciascuno. si ha un massimo di 36 punti e sono concessivi su piccli errori, il terzo è in media più difficile dei primi due.\\
L'orale non fa media aritmetica, decide il voto finale.\\
\hfil\\
Generale introduzion storica:\\
Dalla bottiglia di Leila (primo esempio di condensatore) si cercò di misurare la carica che produceva.\\
Coulomb definì la forza tra due cariche, e da lì si definì la costante dielettrica.\\
\begin{gather*}
\vec{F} = k_e \frac{q_1 q_2}{r^2} \hat{r} \\
\end{gather*}
$q$ aveva una precisione molto bassa oggi dopo varie misurazioni si è arrivati a $q \approx 10^{-16}$\\
\subsection{Legge di Coulomb}
\begin{center}
    \begin{tikzpicture}
        \draw[->] (0,0) -- (2,0) node[right] {$y$};
        \draw[->] (0,0) -- (0,2) node[above] {$z$};
        \draw[<->] (0.1,0.1) -- (0.9,0.9) node[above, midway] {$r$};
        \draw[->] (0,0) -- (-1,-1) node[above] {$x$};
        \draw[->] (1,1) -- (1.5,1.5) node[above] {$\widehat{r}$};
        \filldraw (0,0) circle (2pt) node[below] {$Q$};
        \filldraw (1,1) circle (2pt) node[below] {$q$};
    \end{tikzpicture}
    \begin{tikzpicture}
        \draw[->] (0,0) -- (2,0) node[right] {$y$};
        \draw[->] (0,0) -- (0,2) node[above] {$z$};
    \end{tikzpicture}
\end{center}
Date due cariche puntiformi ad una certa distanza la forza che la carica $q$ sente dalla carica $Q$ è esattamente:
\begin{gather*}
    F_q = \frac{q \ Q}{r^2} \frac{1}{4 \pi \epsilon_0} \widehat{r}
\end{gather*}
Con $r$ che p la distanza tra la carica $Q$ e la carica $q$. E il versore indica il verso, qeusto perchè la forza può essere reulsiva (quando si hanno $q$ e $Q$ di segno opposto) o attrattiva (quando hanno segno concorde).\\
Si ha poi:
\begin{gather*}
    \boxed{\vec{E}(\vec{r}) = \lim_{q \to 0} \frac{\vec{F}_q}{q}}
\end{gather*}
Da qui troviamo che il campo elettrico nel limite di carica puntiforme ($q \to 0$) è:
\begin{gather*}
    \vec{E}(\vec{r}) = \frac{Q}{4 \pi \epsilon_0 r^2} \widehat{r}
\end{gather*}
In breve il corpo due se \underbar{non} c'è il corpo uno, il corpo due non subisce nessuna forza.\\
In più si deve pensare che il campo elettrico sia qualcosa che è fatto di energia ma è proprio un costituente della natura, alla pari della materia, sono qualcosa che permiano lo spazio e sono regolati da leggi fisiche ben precise.\\
Cos'è un campo elettrico?\\
Di fatto è matematicamente un campo vettoriale ovvero associa ad ogni punto dello spazio un vettore che rappresenta l'intensità e la direzione del campo elettrico in quel punto.\\
Il campo in un punto dello spazio è la somma delle forze in quel punto.\\
Iniziamo col generalizzare la legge di Coulomb senza che la carica $Q$ sia nell'origine del sistema di riferimento scelto, altra cosa non si può cercare la forza tra due cariche nello stesso punto, poichè si considera la somma delle due.\\
\begin{center}
    \begin{tikzpicture}
        \draw[->] (0,0) -- (2,0) node[right] {$y$};
        \draw[->] (0,0) -- (0,2) node[above] {$z$};
        \draw[->] (0.1,0.1) -- (0.9,0.9) node[above, midway] {$\vec{r'}$};
        \draw[->] (0,0) -- (-1,-1) node[above] {$x$};
        \draw[->] (1,1) -- (2,1) node[above] {$\vec{r}-\vec{r'}$};
        \draw[->] (0,0) -- (2,1) node[below, midway] {$\vec{r}$};
        \filldraw (2,1) circle (2pt) node[below] {$q$};
        \filldraw (1,1) circle (2pt) node[below] {$Q$};
    \end{tikzpicture}
    \begin{tikzpicture}
        \draw[->] (0,0) -- (2,0) node[right] {$y$};
        \draw[->] (0,0) -- (0,2) node[above] {$z$};
        \draw[->] (0,0) -- (-1,-1) node[above] {$x$};
        \draw[->] (0,0) -- (2,1) node[below, midway] {$\vec{r_1}$};
        \draw[->] (0,0) -- (1,2) node[below, midway] {$\vec{r_2}$};
        \filldraw (2,1) circle (2pt) node[below] {$Q_1$};
        \filldraw (1,2) circle (2pt) node[below] {$Q_2$};
    \end{tikzpicture}
\end{center}
\begin{gather*}
    \vec{E}(\vec{r}) = \frac{1}{4 \pi \epsilon_0} \frac{Q}{|\vec{r}-\vec{r'}|^3} (\vec{r}-\vec{r'})
\end{gather*}
Considerando la fig. 2 ho sapendo che ho: 
\begin{gather*}
    \vec{E} = \vec{E_1} + \vec{E_2} = \frac{1}{4 \pi \epsilon_0} \left( \frac{Q_1}{|\vec{r}-\vec{r_1}|^3} (\vec{r}-\vec{r_1}) + \frac{Q_2}{|\vec{r}-\vec{r_2}|^3} (\vec{r}-\vec{r_2}) \right)\\
    =\frac{1}{4 \pi \epsilon_0}  \sum_{i=1}^{N_c} \frac{Q_i}{|\vec{r}-\vec{r_i}|^3} (\vec{r}-\vec{r_i})
\end{gather*}
Con $N_c$ che è il numero di cariche presenti.\\
\subsection{densità di carica}
Cerchiamo ora a introdurre il concetto di densità di carica, prendendo un volume generico, e cerco di farne l'integrale di volume.
\begin{center}
    \begin{tikzpicture}
        \draw[->] (0,0) -- (2,0) node[right] {$y$};
        \draw[->] (0,0) -- (0,2) node[above] {$z$};
        \draw[->] (0,0) -- (-1,-1) node[above] {$x$};
        \draw[->] (0,0) -- (1.2,1.2) node[above, midway] {$\vec{r'}$};
        \potato{(1,1)}{0.5};
        \draw (1.2,1.2) rectangle (1.4,1.4) node[above, midway] {$\Delta \tau$};
    \end{tikzpicture}
\end{center}
Considero il volumetto infinitesimo quindi $\Delta \tau \to 0$ e considero il numero di volumetti che tende all'infinito quindi $N \to \infty$.\\
\begin{gather*}
    \vec{E} (\vec{r}) = \frac{1}{4 \pi \epsilon_0} \int_\tau \frac{\rho(\vec{r'})}{|\vec{r}-\vec{r'}|^3} (\vec{r}-\vec{r'}) d\tau\\
    \boxed{\vec{E}(\vec{r}) = \frac{1}{4 \pi \epsilon_0} \int_\tau \frac{\rho(\vec{r'})}{|\vec{r}-\vec{r'}|^3} (\vec{r}-\vec{r'}) d\tau}
\end{gather*}
Con $\rho$ che è la densità di carica, e si ritrova nella relazione:
\begin{gather*}
    dQ = \rho(\vec{r'}) d\tau
\end{gather*}
\begin{example}
    Prendiamo un cavo e un punto ad una distanza $x$, e cerchiamo di calcolare il campo elettrico in quel punto.
    \begin{center}
        "foto"
    \end{center}
    Cosa accade invece se ho $L \gg x \gg d$
    Avrò invece che:
    \begin{gather*}
        \vec{E} (x) = \frac{1}{4 \pi \epsilon_0} \frac{\lambda}{x} \widehat{x}
    \end{gather*}
    Quindi se considero un filo infinito ho che il campo va a 0 come $1/x$, e non come normalmente accade che va come $1/x^2$.

\end{example}
\subsection{il teorema di Gauss}
È di fatto la prima equazione di Maxwell espresso in forma integrale e dice che presa una superficie chiusa \\
Prendiamo un tubo:\\
\begin{tikzpicture}
    \draw (0,0) -- (2,0);
    \draw (0,1) -- (2,1);
    \draw (0,0.5) ellipse (0.25 and 0.5);
    \draw (2,0.5) ellipse (0.25 and 0.5);
\end{tikzpicture}
\begin{gather*}
    \frac{\Delta M}{\Delta t} = \rho v \cancel{\Delta t} S = \vec{J} \widehat{n} S = \phi_{\vec{J}}(S) = \int_S \vec{J}(r') \cdot \widehat{u} dS\\
    \text{preso } \vec{J} = \rho \vec{v}
\end{gather*}
Con che è il flusso di $J$ attraverso la superficie $S$.\\
Se consideriamo una delle sezioni inclinata e abbiamo quindi una situazione di questo tipo:
\begin{tikzpicture}
     \draw (0.1,0) -- (2,0);
    \draw (-0.55,1) -- (2,1);
    \draw [rotate=45, scale=1.22] (0.2,0.4) ellipse (0.25 and 0.5);
    \draw (2,0.5) ellipse (0.25 and 0.5);
\end{tikzpicture}
In questo caso otteniamo che:
\begin{gather*}
    \frac{\Delta M}{\Delta t} = \rho v S \cos \theta = \vec{J} \cdot \widehat{n} S = \phi_{\vec{J}}(S)
\end{gather*}
\begin{theorem}[di Gauss]
    \begin{gather*}
        d\phi = \vec{E} \cdot \widehat{n} dS = \frac{Q}{4 \pi \epsilon_0 r^2} \widehat{r} \cdot \widehat{n} dS
    \end{gather*}
    Considero ora invece che un pezzetto di piano sia perpendicolare rispetto a $Q$ e quindi abbia $\widehat{n} = \widehat{r}$, in questo caso ottengo che da un pezzetto di piano $dS$ ottengo un pezzetto di sfera $dS_n$:
    \begin{gather*}
        d\phi = \frac{Q}{4 \pi \epsilon_0} \frac{dS_n}{r^2} = \frac{Q}{4 \pi \epsilon_0} d\Omega
    \end{gather*}
    Sono passato infine ad una quantità che è il rapporto tra l'area normale e il quadrato della distanza, che è proprio la definizione di angolo solido.\\
    \begin{gather*}
        \phi = \int_{S} d\phi = \frac{Q}{4 \pi \epsilon_0} \int_S d\Omega = \frac{Q \cancel{4 \pi}}{\cancel{4 \pi} \epsilon_0}
    \end{gather*}
    \begin{center}
        \begin{tikzpicture}
            \draw[->] (0,0) -- (2,0) node[right] {$y$};
            \draw[->] (0,0) -- (0,2) node[above] {$z$};
            \draw[->] (0,0) -- (-1,-1) node[above] {$x$};
            \filldraw (0,0) circle (2pt) node[below] {$Q$};
            \potato{(-0.5,-0.5)}{0.5};
            \draw (0.3,0.3) rectangle (0.5,0.5) node[right, midway] {$dS$};
        \end{tikzpicture}
    \end{center}
    \noindent
    Consideriamo ora una carica esterna alla superficie chiusa, bisogna considerare che un angolo solido generico proveniente da quella carica o nonintercetta la superficie o la intercetta un numero pari di volte:
    \begin{tikzpicture}
    \filldraw (-1,0) circle (2pt) node[below] {$Q$};
    \potato{(0,0)}{0.5};
    \end{tikzpicture}
    \begin{gather*}
        d\phi = \vec{E_1} \widehat{n_1} S_1 + \vec{E_2} \widehat{n_2} S_2 = \frac{Q}{4 \pi \epsilon_0} \left( \frac{\cos(\theta_1) dS_1}{r_1^2} + \frac{\cos(\theta_2) dS_2}{r_2^2} \right) = 0
    \end{gather*}
    Fa una quantità di angolo solido positiva e una negativa, che si annullano reciprocamente.\\
    In particolare si ricorda che $0 \leq \theta_2 \leq \frac{\pi}{2}$ e $\frac{\pi}{2} \leq \theta_1 \leq \pi$.\\
    Quindi se $Q$ è interna:
    \begin{gather*}
        \phi = \frac{Q}{\epsilon_0}
    \end{gather*}
    Se $Q$ è esterna:
    \begin{gather*}
        \phi = 0
    \end{gather*}
    \begin{gather*}
        \vec{E} = \sum_{i=1}^{N_c} \vec{E_i}
        \phi = \int_S \sum_{i=1}^{N_c} \vec{E_i} \cdot \widehat{n} dS = \sum_{i=1}^{N_c} \int_S \vec{E_i} \cdot \widehat{n} dS = \sum_{i=1}^{N_c} \frac{Q_i}{\epsilon_0} = \frac{Q_\text{int}}{\epsilon_0}
    \end{gather*}
\end{theorem}

\end{document}
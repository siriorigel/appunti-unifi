\documentclass[a4paper, oneside]{article}
\usepackage{wrapfig}
\usepackage{graphicx}
\usepackage{amsthm}
\usepackage{amsmath}
\usepackage{amssymb}
\usepackage[a4paper,
            bindingoffset=0.2in,
            left=2cm,
            right=2cm,
            top=2cm,
            bottom=2cm,
            footskip=.25in]{geometry}
\usepackage[italian]{babel}
\usepackage{pgfplots}
\usepackage{tabularx}
\usepackage{tikz-3dplot}
\usepackage{wrapfig}
\usepackage{color}
\usepackage{multicol}
\usepackage{arydshln}
\usepackage{mathtools}
\usepackage{enumerate}
\usepackage{graphicx}
\usepackage{svg}
\usepackage{cancel}
\usepackage[d]{esvect}
\usepackage[dvipsnames]{xcolor}
\usepackage{pgfplots}
\usepackage{pifont}
\usetikzlibrary{patterns}
\makeindex
%\usepackage{animate}
%\usepackage{xfp} % utile se vuoi fare calcoli aggiuntivi
\pgfplotsset{compat=1.18}
\usetikzlibrary{tikzmark}
\newcommand{\TikzNCbar}[4][10pt]{
\tikz[overlay,remember picture]{\draw[#2] (#3) --++(0,-#1) -| (#4);}}

\graphicspath{ {images/} }

\definecolor{redish}{rgb}{255, 0, 30}
\definecolor{page}{rgb}{0.129,0.157,0.212}
\pagecolor{page}
\color{white}   
\graphicspath{ {./images/} }
\usetikzlibrary{shapes.geometric}
\usetikzlibrary{datavisualization}
\usetikzlibrary{datavisualization.formats.functions}
\pgfplotsset{width=10cm,compat=1.9}

\setlength\dashlinedash{0.2pt}
\setlength\dashlinegap{1.5pt}
\setlength\arrayrulewidth{0.3pt}

\newcommand\eqq{\stackrel{\mathclap{\normalfont\mbox{?}}}{=}}
\newcommand\bulletout  {\labelitemfont \textbullet}
\newcommand{\tab}{\hspace*{2em}}
\newcommand{\xmark}{
\tikz[scale=0.23] {
    \draw[line width=0.7,line cap=round] (0,0) to [bend left=6] (1,1);
    \draw[line width=0.7,line cap=round] (0.2,0.95) to [bend right=3] (0.8,0.05);
}}
\newcommand{\cmark}{
\tikz[scale=0.23] {
    \draw[line width=0.7,line cap=round] (0.25,0) to [bend left=10] (1,1);
    \draw[line width=0.8,line cap=round] (0,0.35) to [bend right=1] (0.23,0);
}}
% Comando:
%   \potato[opzioni]{(x,y)}{scala}
%
% Opzioni = facoltative (es. fill=red!20, draw=black, thick)
% (x,y)   = centro della patata
% scala   = fattore di scala
%
\def\potatoshape{
  (1,0) (2,1.5) (1.6,3) (0.3,2.7) (-0.4,1.2)
}
\newcommand{\potato}[3][draw=white]{
  \begin{scope}[shift={#2}, scale=#3]
    \draw[#1]
      plot [smooth cycle, tension=1]
      coordinates {\potatoshape};
  \end{scope}
}
 \newcommand{\hookbox}[1]{
\begin{center}
\hfill\break
\begin{tikzpicture}
\node[inner sep=0pt,outer sep=0pt,anchor=base] (A) {
\begin{minipage}{\dimexpr\linewidth-5em}
\centering
#1
\end{minipage}
};
% Draw the left bracket
\draw ([xshift=0pt]A.north west) -- ++(0, 0.5) -- ++(0.4, 0);
% Draw the right bracket
\draw ([xshift=0pt]A.south east) -- ++(0, -0.5) -- ++(-0.4, 0);
\end{tikzpicture}
\end{center}} 
\title{Metodi matematici}
\author{Gariboldi Alessandro}
\date{ }


\begin{document}

\newtheoremstyle{theoremEnv}
                {}          % Space above
                {}          % Space below
                {\slshape}  % Body font
                {}          % Indent amount
                {\bfseries} % Head font
                {.}         % Punctuation after head
                {\newline}         % Space after theorem head
                {}          % Theorem head spec
\theoremstyle{theoremEnv}

\newtheorem{definition}{Definizione}[section]
\newtheorem{theorem}{Teorema}[section]
\newtheorem{lemma}{Lemma}[section]
\newtheorem{observation}{Oss.}[section]
\newtheorem{corollary}{Corollario}[theorem]
\newtheorem{example}{Esempio}[section]
\newtheorem{problem}{Problema}[section]
\newtheorem{solution}{Soluzione}[section]
\newtheorem{proposition}{Proposizione}[section]


\maketitle
\section{25/02/26}
\subsection{introduzione al corso}
Questo è un corso che estende gli argomenti di analisi e geometria, e servono in generale per studiare meccanica quantistica 
e altri argomenti di fisica avanzata.\\
Strutturazione del corso:
\begin{itemize}
    \item Analisi complessa (funzioni a valori complessi, serie complesse e integrali complessi)
    \item Analisi armonica (come le funzioni posso essere sviluppate come serie armoniche, usando seni e coseni)
    \item Spazi di Hilbert (spazi vettoriali a dimensione infinita, che generalizzano lo spazio euclideo)
\end{itemize}
Strutturazione dell'esame:
\begin{itemize}
    \item Scritto: 3 esercizi uno per parte del corso, in generale simili a quelli del corso (si passa con 16, durata di 3 ore, validità di 1 anno a consigliato di farlo prima).
    \item Orale: Meno incentrato sulle dimostrazioni, più sulla discussione di esercizi o semplici dimostrazione (poche) o argomenti
\end{itemize}
In generale questo perchè lo scopo di questo corso non è di fornire una stuttura formale, ma di fornire strumenti utili per la fisica.\\
In generale per gli orali si possono fare anche fuori dagli appelli basta organizzarsi, nel caso è meglio organizzarsi a piccoli gruppi.\\
Materiale didattico è presente su moodle in forma di dispense. 
\subsection{Riassunto numeri complessi}
I numeri complessi sono un'estensione dei numeri reali, ed è ottenuta dal numero $i$ che è la radice quadrata di $-1$.\\
Questa è l'unità immaginaria che ci serve per estendere i numeri reali al campo dei complessi.\\
Quindi un generico numero $z \in \mathbb{R}$ è della forma $z = x + iy$ con quindi un generico numero reale più un numero reale considerato $i$ volte.\\
Ovviamente quando $i = 0$ otteniamo un sottoinsieme di $\mathbb{C}$ che è proprio $\mathbb{R}$.\\
Quindi i numeri complessi sono un'estensione dei numeri reali, e possiamo rappresentarli come punti in un piano, chiamato piano complesso.\\
Questo perchè $\mathbb{C} \simeq \mathbb{R} \times \mathbb{R}$ (dove $\simeq$ vuol dire isomorfo)come spazio vettoriale, quindi ogni numero complesso può essere rappresentato come un punto in un piano.\\
\begin{center}
    \begin{tikzpicture}
        \draw[->] (-3,0) -- (3,0) node[right] {$\Re$};
        \draw[->] (0,-3) -- (0,3) node[above] {$\Im$};
        \draw[blue, thick] (1,2) node[above] {\textcolor{white}{$z = x + iy$}} -- (0,0);
        \draw[blue, thick] (1,-2) node[above] {\textcolor{white}{$\frac{1}{z}$}} -- (0,0);
        \draw[red,thick] (1,0) node[below] {$x$} -- (0,0);
        \draw[red,thick] (0,2) node[left] {$y$} -- (0,0);
        \draw[dashed] (1,0) -- (1,2);
        \draw[dashed] (0,2) -- (1,2);
        %angolo
        \draw[->] (0.5,0) arc (0:63.4:0.5);
        \node at (0.8,0.2) {$\phi$};
        %arco che rappresenta il passaggio da z a 1/z
        \draw[->] (1,2) arc (63.4:243.4:2.236);
        \node at (1.5,-1) {$*$};
    \end{tikzpicture}
\end{center}
    \hfill\\
    \noindent
    Il modulo (valore assoluto) di un numero complesso è la distanza dall'origine, quindi:
    \begin{gather*}
        |z| = \sqrt{x^2 + y^2} \in \mathbb{R}^+\\
        |z| = 0 \Leftrightarrow z = 0
    \end{gather*}
    La coniugata di un numero complesso è ottenuta cambiando il segno della parte immaginaria, quindi:
    \begin{gather*}
        z \to z^* \equiv \overline{z} = x - iy\\
    \end{gather*}
    Se ora riscriviamo il modulo in termini di $z$ e $z^*$ otteniamo:
    \begin{gather*}
        |z|^2 = z \ z^* = z^* \ z = |z^*|^2\\
        = (x + iy)(x - iy) = x^2 + y^2
    \end{gather*}
    \begin{gather*}
        \frac{1}{z} = \frac{z^*}{z^* \ z} = \frac{z^*}{|z|^2}
    \end{gather*}
    Un'altra rappresentazione dei numeri complessi è la forma polare, che è data da:
    \begin{gather*}
        x + i y = r \cos \phi + i r \sin \phi = r (\cos \phi + i \sin \phi)\\
    \end{gather*}
    Spesso si prende $\phi \in [\alpha, \alpha+ 2\pi[$ che prende un giro completo del pian, ma può essere utile anche prendere ad esempio $\phi \in ]-\infty, +\infty[$ che prende più giri del piano.\\
    Consideriamo ora il prodotto tra due numeri complessi:
    \begin{gather*}
        z_1 z_2 = r_1 r_2 (\cos \phi_1 + i \sin \phi_1)(\cos \phi_2 + i \sin \phi_2)\\
        = r_1 r_2 [\cos (\phi_1 + \phi_2) + i \sin (\phi_1 + \phi_2)]
    \end{gather*}
    Se estendiamo questo al concetto di potenze intere:
    \begin{gather*}
        z^2 = r^2 [\cos (2\phi) + i \sin (2\phi)]\\
        z^n = r^n [\cos (n\phi) + i \sin (n\phi)]
    \end{gather*}
    Questo fa capire bene anche il coniugato che è dato da:
    \begin{gather*}
        \frac{1}{z} = \frac{1}{r} [\cos (\phi) - i \sin (\phi)]
    \end{gather*}
    Che è esattamente il coniugato poichè ha stessa parte reale e parte immaginaria ribaltata rispetto all'asse reale.\\
    Concentriamoci ora sull trovare le radici:
    \begin{gather*}
        z^{\frac{1}{n}} \qquad w^n = z\\
        w = \rho (\cos \theta + i \sin \theta)\\
        z = r (\cos \phi + i \sin \phi)\\
        \rho^n (cos n\theta + i \sin n\theta) = r (\cos \phi + i \sin \phi)\\
        \begin{cases}
            \rho = r^{\frac{1}{n}}\\
            n \theta = \phi + 2 k \pi \quad k \in \mathbb{Z} \Rightarrow \theta = \frac{\phi}{n} + \frac{2 \pi k}{n}
        \end{cases}
    \end{gather*}
    Questi sono proprio poligoni iscritti nella circonferenza di raggio $r^{\frac{1}{n}}$ con centro nell'origine:
    \begin{center}
        \begin{tikzpicture}
            \draw[->] (-3,0) -- (3,0) node[right] {$\Re$};
            \draw[->] (0,-3) -- (0,3) node[above] {$\Im$};
            \draw (0,0) circle (2);
            \draw[red,thick] (2,0) node[below, right] {$r^{\frac{1}{n}}$} -- (0,0);
            \draw[rotate = 45](-1.4,-1.4) rectangle (1.4,1.4);
        \end{tikzpicture}
    \end{center}
    \subsection{Proprietà dei complessi}
    Definiamo una topologia come di intorni di un punto, quindi una successione di insiemi che contengono il punto e che sono sempre più piccoli.\\
    Introduciamo ora il concetto di convergenza su $\mathbb{C}$ sfruttando la topologia su $\mathbb{C}$\\
    Lo facciamo partendo dalla definizione di convergenza su $\mathbb{R}$, prendendo il limite sulla parte reale e sulla parte immaginaria.
    \begin{definition}
        Una successione $\left\{z_n\right\} _{n \in \mathbb{N}}$ di numeri complessi converge a $z \in \mathbb{C}$ se e solo se:
        \begin{gather*}
            \begin{cases}
                \lim_{n \to +\infty} \Re(z_n) = \Re(z)\\
                \lim_{n \to +\infty} \Im(z_n) = \Im(z)
            \end{cases}
        \end{gather*}
        In modo equivalente possiamo scrivere:
        \begin{gather*}
            \lim_{n \to +\infty}\left\lvert  z_n -z \right\rvert = 0 
        \end{gather*}
    \end{definition}
    \noindent
    Consideriamo che $\mathbb{C}$ è uno spazio metrico con la distanza data da:
    \begin{gather*}
        d(z_1, z_2) \equiv |z_1 - z_2|
    \end{gather*}
    Questo soddisfa gli assiomi di distanza.\\
    \hfill\\
    Vediamo ora che gli intorni di $z$ sono tutte le palle aperte con centro $z$ e raggio $r > 0$:
    \begin{gather*}
        B(z, r[
    \end{gather*}
    Da questo concetto di intorno segue naturalmente la convergenza dei limiti.\\
    \hfill\\
    Chiediamoci ora se $\mathbb{C}$ è completo, cioè se ogni successione di Cauchy converge.\\
    Si può dimostrare velovemente considerando che la parte reale e la parte immaginaria sono entrambe complete, quindi anche $\mathbb{C}$ è completo (ogni successione di Cauchy ha limite) e di conseguenza anche $\mathbb{C}^n$ è completo.\\
    \hfill\\
    Su $\mathbb{C}$ possiamo definire anche la norma oltre che alla distanza, e questa è data da:
    \begin{gather*}
        \left\lVert z \right\rVert  \equiv |z|
    \end{gather*}
    Quindi $\mathbb{C}$ è anche spazio normato.
    Siccome $\mathbb{C}$ è completo e normato, allora è anche uno spazio di Banach.\\
    Su $\mathbb{C}$ purtroppo si perde la proprietà di ordinamento, quindi non è uno spazio ordinato e non hanno senso le disuguaglianze tra numeri complessi al massimo è possibile eseguirle tra i moduli.\\ 
    \subsection{serie di potenze complesse}
    \begin{definition}
        Data una successione di numeri complessi $\left\{z_n\right\} _{n \in \mathbb{N}} \in \mathbb{C}$, e definiamo le ridotte parizali come:
        \begin{gather*}
            \sum_{n=0}^N z_n \equiv S_N
        \end{gather*}
    \end{definition}
    \begin{definition}
        $\left\{z_n\right\} _{n \in \mathbb{N}}$ è una serie convergente se $S_N$ è una successione convergente, ovvero:
        \begin{gather*}
            \lim_{N \to +\infty} \sum_{n=0}^N z_n = \lim_{N \to +\infty} S_N = S \in \mathbb{C}
        \end{gather*}
    \end{definition}
    \begin{theorem}
        Condizione necessria (ma non sufficiente) per la convergenza di una serie è che $|z_n|$ sia una successione infinitesima ovvero:
        \begin{gather*}
            \lim_{n \to +\infty} |z_n| = 0
        \end{gather*}
    \end{theorem}
    \begin{theorem}[Criterio del rapporto]
        $\sum_n z_n$ è convergente se $\exists p : 0<0<1$ e $\exists \overline{n} \in \mathbb{N}$ tale che $\forall n > \overline{n}$:
        \begin{gather*}
            \left\lvert \frac{z_{n+1}}{z_n} \right\rvert \leq p
        \end{gather*}
    \end{theorem}
    \begin{theorem}[Criterio della radice]
        $\sum_n z_n$ è convergente se $\exists p : 0<0<1$ e $\exists \overline{n} \in \mathbb{N}$ tale che $\forall n > \overline{n}$:
        \begin{gather*}
            \sqrt[n]{|z_n|} \leq p
        \end{gather*}
    \end{theorem}
    \begin{definition}[Convergenza assoluta]
        $\sum_n z_n$ è convergente assolutamente se $\sum_n |z_n|$ è convergente.
    \end{definition}
    \begin{theorem}
        La convergenza assoluta $\overset{\text{implica}}{\Rightarrow}$ la convergenza, ma non viceversa.
    \end{theorem}
    \begin{theorem}[Criterio del confronto]
        Se $\left\{z_n\right\} _{n \in \mathbb{N}}$ è maggiorata in \underbar{modulo} ($|z_n| \leq a_n$)\\
        da una successione di numeri reali positivi $\left\{a_n\right\} _{n \in \mathbb{N}}$ che è convergente, \\
        \underbar{allora} anche $\sum_n z_n$ converge (assolutamente).\\
    \end{theorem}
    \begin{proposition}[Serie di potenze]
        \begin{gather*}
            \sum_{n=0}^{\infty} a_n (z - z_0)^n
        \end{gather*}
        Con $a_n, z, z_0 \in \mathbb{C}$, è una serie di potenze con centro in $z_0$.\\
    \end{proposition}
    Per semplicità consideriamo $z_0 = 0$ e quindi la serie di potenze diventa:
    \begin{gather*}
    \sum_{n=0}^{\infty} a_n z^n
    \end{gather*}
    \begin{definition}[Insieme di convergenza]
        Definiamo come insieme di convergenza:
        \begin{gather*}
            C_a \equiv \left\{ z \right\}_{\in \mathbb{C}} \ \text{t.c} \ \sum_n a_n z^n \ \text{converge}
        \end{gather*}
    \end{definition}
    \begin{lemma}[Di Abel]
        Sia $w \in \mathbb{C}$ t.c. $a_n w^n$ sia limitata, \underbar{allora}\\
        $\forall z \in \mathbb{C}$ t.c. $|z| < |w|$ la serie $\sum_n a_n z^n$ converge assolutamente.
    \end{lemma}
    \begin{center}
        \begin{tikzpicture}
            \draw[->] (-3,0) -- (3,0) node[right] {$\Re$};
            \draw[->] (0,-3) -- (0,3) node[above] {$\Im$};
            \filldraw[pattern = north east lines] (0,0) circle (2);
            \node at (0.5,2.2) {$C_a$};
            \filldraw (1.5,1.5) circle (2pt) node[right] {$w \in C_a$};
        \end{tikzpicture}
    \end{center}
    \begin{proof}
        \begin{gather*}
            \left\lvert a_n w^n\right\rvert \leq M\\
            \left\lvert a_n z^n\right\rvert = \left\lvert a_n w^n (\frac{z^n}{w^n})\right\rvert \\
            = \left\lvert a_n w^n \right\rvert \left\lvert \frac{z}{w}\right\rvert^n \leq M \underbrace{\left\lvert \frac{z}{w} \right\rvert^n}_{= p \ <1}
        \end{gather*}
    \end{proof}
    \begin{definition}[Raggio di convergenza]
        \begin{gather*}
            R_a \equiv \sup \left\{|z|: z \in C_a\right\} 
        \end{gather*}
    \end{definition}
    \begin{theorem}
        L'insieme di convergena $C_a$ contiene il cerchio aperto $B(0, R_a)$ ed è contenuto nel cerchio chiuso $B(0, R_a]$.
    \end{theorem}
    \hfill\\
    Le serie geometriche valgono anche su $\mathbb{C}$, quindi:
    \begin{gather*}
        \sum_{n=0}^{\infty} z^n = \frac{1}{1-z}
    \end{gather*}
    


\end{document}
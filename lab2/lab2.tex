\documentclass[a4paper, oneside]{article}
\usepackage{wrapfig}
\usepackage{graphicx}
\usepackage{amsthm}
\usepackage{amsmath}
\usepackage{amssymb}
\usepackage[a4paper,
            bindingoffset=0.2in,
            left=2cm,
            right=2cm,
            top=2cm,
            bottom=2cm,
            footskip=.25in]{geometry}
\usepackage[italian]{babel}
\usepackage{pgfplots}
\usepackage{tabularx}
\usepackage{tikz-3dplot}
\usepackage{wrapfig}
\usepackage{color}
\usepackage{multicol}
\usepackage{arydshln}
\usepackage{mathtools}
\usepackage{enumerate}
\usepackage{graphicx}
\usepackage{svg}
\usepackage{cancel}
\usepackage[d]{esvect}
\usepackage[dvipsnames]{xcolor}
\usepackage{pgfplots}
\usepackage{pifont}
\usetikzlibrary{patterns}
\makeindex
%\usepackage{animate}
%\usepackage{xfp} % utile se vuoi fare calcoli aggiuntivi
\pgfplotsset{compat=1.18}
\usetikzlibrary{tikzmark}
\newcommand{\TikzNCbar}[4][10pt]{
\tikz[overlay,remember picture]{\draw[#2] (#3) --++(0,-#1) -| (#4);}}

\graphicspath{ {images/} }

\definecolor{redish}{rgb}{255, 0, 30}
\definecolor{page}{rgb}{0.129,0.157,0.212}
\pagecolor{page}
\color{white}   
\graphicspath{ {./images/} }
\usetikzlibrary{shapes.geometric}
\usetikzlibrary{datavisualization}
\usetikzlibrary{datavisualization.formats.functions}
\pgfplotsset{width=10cm,compat=1.9}

\setlength\dashlinedash{0.2pt}
\setlength\dashlinegap{1.5pt}
\setlength\arrayrulewidth{0.3pt}

\newcommand\eqq{\stackrel{\mathclap{\normalfont\mbox{?}}}{=}}
\newcommand\bulletout  {\labelitemfont \textbullet}
\newcommand{\tab}{\hspace*{2em}}
\newcommand{\xmark}{
\tikz[scale=0.23] {
    \draw[line width=0.7,line cap=round] (0,0) to [bend left=6] (1,1);
    \draw[line width=0.7,line cap=round] (0.2,0.95) to [bend right=3] (0.8,0.05);
}}
\newcommand{\cmark}{
\tikz[scale=0.23] {
    \draw[line width=0.7,line cap=round] (0.25,0) to [bend left=10] (1,1);
    \draw[line width=0.8,line cap=round] (0,0.35) to [bend right=1] (0.23,0);
}}
% Comando:
%   \potato[opzioni]{(x,y)}{scala}
%
% Opzioni = facoltative (es. fill=red!20, draw=black, thick)
% (x,y)   = centro della patata
% scala   = fattore di scala
%
\def\potatoshape{
  (1,0) (2,1.5) (1.6,3) (0.3,2.7) (-0.4,1.2)
}
\newcommand{\potato}[3][draw=white]{
  \begin{scope}[shift={#2}, scale=#3]
    \draw[#1]
      plot [smooth cycle, tension=1]
      coordinates {\potatoshape};
  \end{scope}
}
 \newcommand{\hookbox}[1]{
\begin{center}
\hfill\break
\begin{tikzpicture}
\node[inner sep=0pt,outer sep=0pt,anchor=base] (A) {
\begin{minipage}{\dimexpr\linewidth-5em}
\centering
#1
\end{minipage}
};
% Draw the left bracket
\draw ([xshift=0pt]A.north west) -- ++(0, 0.5) -- ++(0.4, 0);
% Draw the right bracket
\draw ([xshift=0pt]A.south east) -- ++(0, -0.5) -- ++(-0.4, 0);
\end{tikzpicture}
\end{center}} 
\title{Laboratorio di fisica II}
\author{Gariboldi Alessandro}
\date{ }


\begin{document}

\newtheoremstyle{theoremEnv}
                {}          % Space above
                {}          % Space below
                {\slshape}  % Body font
                {}          % Indent amount
                {\bfseries} % Head font
                {.}         % Punctuation after head
                {\newline}         % Space after theorem head
                {}          % Theorem head spec
\theoremstyle{theoremEnv}

\newtheorem{definition}{Definizione}[section]
\newtheorem{theorem}{Teorema}[section]
\newtheorem{lemma}{Lemma}[section]
\newtheorem{observation}{Oss.}[section]
\newtheorem{corollary}{Corollario}[theorem]
\newtheorem{example}{Esempio}[section]
\newtheorem{problem}{Problema}[section]
\newtheorem{solution}{Soluzione}[section]
\newtheorem{proposition}{Proposizione}[section]


\maketitle
\section{23/02/26}
Definiamo le varie quantità che misureremo, e partiamo dalla legge di Ohm:
\begin{gather*}
    V = R I
\end{gather*}
Le misure che facciamo in laboratorio sono macroscopiche anche se si parla di cariche di elettroni che sono quantità microscopiche.\\
La carica dell'elettrone è stata definita esattamente (nel 2019) come:
\begin{gather*}
    e = 1.602176634 \times 10^{-19} C
\end{gather*}
Da quì si definisce l'ampere come:
\begin{gather*}
    1 A = \frac{1 C}{1 s}
\end{gather*}
La corrente la definiamo come:
\begin{gather*}
    I = \frac{dq}{dt}
\end{gather*}
Ovvero quanta carica passa in un certo intervallo di tempo, ma non si guardano i singoli elettroni ma si misura una quantità macroscopica.\\
Questo è vero ma non del tutto, per capire meglio è comodo introdurre il concetto di \textbf{shot noise}: sostanzialmente invece di pensare agli elettroni 
come un flusso omogeneo che può essere associao ad un rumore bianco, si considerano come pacchetti ben definiti 
di carica finita che si può ricondurre se si vuole al rumore della pioggia, da quì shot noise. Questo perchè ogni elettrone ha una carica 
definita ben precisa e siccome passano in momenti leggermente diversi ma casuali, seguono le leggi della statistica e si hanno delle fluttuazioni attorno al valore medio.\\
Quindi l'aspetto microscopico è comunque importante e bisogna ricordarsene.\\
Tutte le tecniche moderne per misurare le correnti sotto lo shot noise, sono le tecniche che vanno al di sotto del singolo elettrone.\\
In generale bisogna sempre avere in mente il limite delle misure che si stanno eseguendo.\\
\subsection{atto pratico}
Supponiamo di avere un conduttore e abbiamo degli ioni (+) che sono posti ad una certa distanza e alcuni elettroni liberi di muoversi, associati ognuno ad uno ione.
\begin{center}
    \begin{tikzpicture}
        \draw[thick] (0,0) sin (0.5,1) cos (1,0) sin (1.5,-1) cos (2,0) sin (2.5,1) cos (3,0) sin (3.5,-1) cos (4,0) sin (4.5,1) cos (5,0);
            \foreach \x in {1.5, 3.5} {
                \node[circle, fill=red!50, minimum size=0.3cm] at (\x,0) {+};
            }
            \foreach \x in {2.5, 4.5} {
                \node[circle, fill=blue!50, minimum size=0.3cm] at (\x,1.2) {-};
            }
            \draw[->, thick] (5,-0.5) -- (5.5,-0.5);
    \end{tikzpicture}
\end{center}
Se prendiamo un cavo di rame con una densità di $d=9 g/cm^3$, e quindi $63,5 g/mol$
\begin{gather*}
    n = \frac{N_A \cdot d}{M} = 8.5 \times 10^{22} atomi/cm^3
\end{gather*}
Questo vuol dire che in un $\mu m$ cubo di rame ci sono circa $10^{11}$ elettroni liberi.\\
Ora si capisce bene il concetto di densità di carica che è:
\begin{gather*}
    \rho = n \cdot e
\end{gather*}
Ovvero la quantità di carica per unità di volume.\\
Questo non è sempre vero perchè non tutti gli elettroni sono liberi, ma è una buona approssimazione.\\
\subsection{cos'è una carica}
Se ho un campo elettrico con una carica all'interno, questo campo applicherà una forza alla mia carica e quest'ultima avrà una certa accelerazione, quindi si ha che la carica dipende anche dal tempo?\\
Questo melo spiega la legge di Ohm che mi dice che mi dice che la corrente è proporzionale alla tensione, e quindi se la tensione è costante allora la corrente è costante, e quindi la carica che passa in un certo intervallo di tempo è costante.\\
\begin{example}
    Prendiamo un gas  a una temperatura $T$, avremo che $\frac{1}{2} m v_t^2 = \frac{3}{2} k_B T$ è una relazione che ci dice che la velocità termica è proporzionale alla temperatura, e quindi se la temperatura è alta allora la velocità termica è alta, e quindi gli elettroni si muovono più velocemente, e quindi la corrente è più alta.\\
    Se infatti la calcoliamo otteniamo:
    \begin{gather*}
        v_t = \sqrt{\frac{3 k_B T}{m}} = \sqrt{\frac{3 k_b T}{n_e}} = \sqrt{\frac{\cancel{3} 1,3810^{-23} T/k \cancel{3}00 k}{\cancel{9},1 10 ^{-31}}kg} = 1,16 \times 10^5 m/s
    \end{gather*}
    Ricordiamo che il moto degli elettroni in un metallo è quantistico ma classicamente possiamo dire che da un moto di agitazione termica di velocità di centinaia di km/s (la stessa velocità a cui viaggiano gli atomi nell'aria), quindi la corrente viaggia più o meno alla stessa velocità?\\ 
\end{example}

\subsection{urti nel campo elettrico}
Consideriamo ora che gli elettroni che si muovono in un corpo, considero i loro urti in un intervallo generale $\Delta t$ avrò quindi che la variazione della velocità avrà un'accelerazione:
\begin{gather*}
    \Delta \vec{v} = \vec{a} \Delta t = \frac{\vec{F}}{m} \Delta t = \frac{-e \vec{E}}{m} \Delta t\\
    \Delta \vec{v} = -\frac{e \vec{E}}{m} \Delta t
\end{gather*}
\begin{gather*}
    \vec{v_d} =-\frac{e \vec{E}}{2m} \Delta t
\end{gather*}
Ma $\Delta t= \frac{\Delta l}{v_t}$ e siccome $v_d \ll v_t$
\begin{gather*}
    \vec{v_d} = - \frac{e \vec{E}}{2m} \frac{\Delta l}{v_t} = k(t) \vec{E}
\end{gather*}
Ricordiamo che $v_d := \left\langle \Delta v\right\rangle_T$ è la velocità media degli elettroni liberi in un conduttore.
verifichiamo ora l'ipotesi di $v_d \ll v_t$
\subsection{densità di corrente}
Si parlerà dell'equazione di continuità che deriva dal principio di conservazione della carica.\\
E si arriverà alla prima legge di kirkhoff.\\
\hfil\\
Prendiamo un elettrone in concizioni stazionarie (che non vuol dire che la corrente è stazionaria che vuol dire che $\vec{E} = 0$ cosa che comunque considereremo per parte di questo corso).\\
Ora definiamo:
\begin{gather*}
    I = \int_S \vec{J} \cdot \vec{dS} = \int_S \vec{J} \widehat{n} dS
\end{gather*}
Cos'è $J$?
\begin{gather*}
    \vec{J} = n q \vec{v_d} \propto \vec{E}
\end{gather*}
Da questo si vede che vale sempre che $\vec{J} \parallel \vec{E}$ indipendentemente dalla carica.\\
$\vec{J}$ è una densità di corrente, e si misura in $A/m^2$.\\
\begin{gather*}
    [J] = \frac{C}{s \cdot m^3} \frac{m}{s} = \frac{A}{m^2} 
\end{gather*}
\section{25/02/26}
La scorsa lezione abbiamo detto che possiamo avere delle cariche $n \sim 9 \times 10^{22} \text{ atomi/cm}^3$, che possono avere una certa densità di carica $\rho$\\
e che possono essere "accelerate" da un campo elettrico $\vv{E}$ ma che sotto le nostre ipotesi di $v_d \ll v_t$ e di urti che avvengono in un intervallo di tempo $\Delta t$ possiamo dire che $\vv{v_d} \propto \vv{E}$.\\
Inoltre abbiamo visto che la corrente è data da $I = \int_S \vv{J} \cdot \hat{n} dS$ e che $\vv{J} = n q \vv{v_d} \parallel \vv{E}$, quindi possiamo dire che $I \propto E$.\\
Come ultima cosa siamo arrivati alla legge di Ohm in forma locale:
\begin{gather*}
    \vv{J}(\vv{r}) = \sigma_c \vv{E}(\vv{r})
\end{gather*}
Dove $\sigma_c$ è la conducibilità elettrica, che è una proprietà del materiale.\\
L'unica cosa che avevamo dimostrato era la velocità di deriva, ma avevamo visto che $\vv{v_t}$ è molto grande circa $100 km/s$\\
Altro aspetto importante è che la legge di ohm si può riscrivere come:
\begin{gather*}
    \vv{E} = \rho_c \vv{J}
\end{gather*}
Dove $\rho_c$ è la resistività elettrica, che è l'inverso della conducibilità ovvero $\rho_c = \frac{1}{\sigma_c}$.\\
\subsection{calcoliamo $v_d$}
Prendiamo un cavo con una sezione di raggio $r = 1mm$ e $I = 1A$.\\
Di conseguenza avremo la sezione $S = \pi r^2 = \pi (1mm)^2$\\
Ora passando a $I$
\begin{gather*}
    I = 1A = n \ q \ v_d \ S =\\ 
    %\frac{1 \cancel{C}/s}{9 \times 10^{22} el/cm^\cancel{3} \cdot 1.6 \times 10^{-19} \cancel{C} \cdot \pi (10^{-1})^2 \cancel{cm}^2} = 7 \times 10^{-5} m/s
\end{gather*}
\begin{gather*}
    -dQ = \int_S \vv{J} \cdot \vv{dS} \\
    -\frac{dQ(t)}{dt} = \int_S \vv{J} \cdot \vv{dS}
    -\frac{dQ(t)}{dt} = -\int_V \frac{\partial \rho(x,y,z,t)}{\partial t} dx dy dz\\
\end{gather*}
Usando il teorema della divergenza ottengo:
\begin{gather*}
    -\int_V \frac{\partial \rho}{\partial t} dV = \int_S \vv{J} \vv{dS} = \int_V \vv{\nabla} \vv{J} dV\\
    -\frac{\partial \rho}{\partial t} = \vv{\nabla} \cdot \vv{J} \Rightarrow \boxed{\vv{\nabla} \cdot \vv{J} + \cancel{\frac{\partial \rho}{\partial t}} = 0}\\
    \Rightarrow \boxed{\vv{\nabla} \cdot \vv{J} = 0} \Rightarrow \boxed{\int_{S_\text{chiusa}} \vv{J} \cdot \vv{dS} = 0}
\end{gather*}
\subsection{nodi}
Un nodo è un punto di incontro tra due o più conduttori.
La somma algebrica delle correnti che entrano in un nodo è uguale alla somma algebrica delle correnti che escono da un nodo.\\
Se ho una superficie chiusa che racchiude il nodo $n$ considerando $\int_S \vv{J} \widehat{n} dS = 0$
\begin{center}
    \begin{tikzpicture}
        \draw[dashed] (0,0) circle (1.4);
        %curve intersecanti
        \draw (-1,-1) to [bend left=20] (1,1);
        \draw (-1,1) to [bend right=20] (1,-1);
        \draw(-1,-1) circle (3pt) node[below] {$S_3$};
        \draw(1,1) circle (3pt) node[above] {$S_1$};
        \draw(-1,1) circle (3pt) node[above] {$S_4$};
        \draw(1,-1) circle (3pt) node[above] {$S_2$};
        %versori n uscenti
        \draw[->, thick] (-1,-1) -- (-1.4,-1.4) node[above] {$\hat{n} \quad I_4$}; 
        \draw[->, thick] (1,1) -- (1.4,1.4) node[above] {$\hat{n} \quad I_2$}; 
        \draw[->, thick] (-1,1) -- (-1.4,1.4) node[above] {$\hat{n} \quad I_1$}; 
        \draw[->, thick] (1,-1) -- (1.4,-1.4) node[above] {$\hat{n} \quad I_3$}; 
    \end{tikzpicture}
\end{center}
\begin{gather*}
    \sum_{i=1}^n I_i = 0
    I_1 + I_2 + I_3 + I_4 = 0
\end{gather*}
\subsection{Maglie e rami}
\begin{center}
    \begin{tikzpicture}[scale = 0.5]
        \draw (-1,-1) to [bend left=20] (1,1);
        \draw (-1,1) to [bend right=20] (1,-1);\
        \filldraw(-0.4,0) circle (3pt) node[below] {$N$};
    \end{tikzpicture}
    $\qquad\qquad$
    \begin{tikzpicture}
        \draw (-1,0) -- (1,0) node[midway, above] {$R$};
        \filldraw (-1,0) circle (2pt);
        \filldraw (1,0) circle (2pt);
        \draw (-1,0) -- (-1.5,-1);
        \draw (-1,0) -- (-1.5,1);
        \draw (1,0) -- (1.5,-1);
        \draw (1,0) -- (1.5,1);
    \end{tikzpicture}
    $\qquad\qquad$
    \begin{tikzpicture}
        \draw (0,0) rectangle (2,2) node[midway] {$M$};
        \filldraw (0,0) circle (2pt);
        \filldraw (2,0) circle (2pt);
        \filldraw (0,2) circle (2pt);
        \filldraw (2,2) circle (2pt);
    \end{tikzpicture}
\end{center}
Un ramo è un conduttore che collega due nodi, una maglia è un percorso chiuso che non contiene altri nodi al suo interno.\\
Più maglie formano una rete.\\
\subsection{Seconda legge di Kirchhoff e il potenziale}
rotore uguale a 0 vuol dire conservativo.
\begin{gather*}
    \vv{\nabla} \land \vv{E} = 0
\end{gather*}
\begin{center}
    \begin{tikzpicture}
        %assi
        \draw[->] (0,0) -- (3,0) node[below] {$x$};
        \draw[->] (0,0) -- (0,3) node[left] {$y$};
        \draw[->] (0,0) -- (-1,-1) node[below] {$z$};
        \draw[->] (0,0) -- (2,1) node[midway, above] {$\vv{r_A}$};
        \filldraw (2,1) circle (1pt) node[above] {$A$};
        \draw[->] (0,0) -- (1,-0.5) node[midway, below] {$\vv{r_B}$};
        \filldraw (1,-0.5) circle (1pt) node[below] {$B$};
        \draw[->, thick] (2,1) to [bend left=20] (1,-0.5);
        \node at(2,0.5) {$\Gamma$};
        \node at(0,0) [left] {$O$};
    \end{tikzpicture}
\end{center}
\begin{gather*}
    \vv{E}(\vv{r}) = \frac{Q}{4 \pi \epsilon_0 r^2} \hat{r}\\
    \mathcal{L}_{AB}^\Gamma = \int_\Gamma \vv{F} \vv{dl} = \int q \vv{E} \vv{dl} = \frac{qQ}{4 \pi \epsilon_0} \int_{r_a}^{r_b}  \frac{1}{r^2} \overset{\overset{1}{||}}{\widehat{r} \widehat{r}} dr = \int \frac{1}{r^2} dt = -\frac{1}{r}
\end{gather*}
\begin{gather*}
    \mathcal{L}_{AB}^\Gamma = \frac{qQ}{4 \pi \epsilon_0} \left( -\frac{1}{r_b} + \frac{1}{r_a} \right)
\end{gather*}
Quest'ultimo è il principio di conservatività del campo (1a legge di Kirchhoff)\\
Continuando otteniamo:
\begin{gather*}
    \lim_{q \to 0} \frac{U}{q} = \boxed{V(r) = \frac{Q}{4 \pi \epsilon_0} \frac{1}{r} + cost.}\\
    V_A - V_B = \int_{A}^{B} \vv{E} \cdot \vv{dl}
\end{gather*}
Ricordiamo che se abbiamo una curva chiusa $\Gamma$ allora:
\begin{gather*}
    \oint_\Gamma \vv{E} \cdot \vv{dl} = 0
\end{gather*}
\begin{center}
    \begin{tikzpicture}
        \draw (-1,-1) rectangle (1,1);
        \filldraw (1,1) circle (1pt) node[above] {$C$};
        \filldraw (-1,-1) circle (1pt) node[below] {$A$};
        \filldraw (-1,1) circle (1pt) node[above] {$B$};
        \filldraw (1,-1) circle (1pt) node[below] {$D$};
        \draw[dashed] (0,0) circle (1.4);
        \node at(-1,0) [right] {$R_1$};
        \node at(1,0) [left] {$R_3$};
        \node at(0,1) [below] {$R_2$};
        \node at(0,-1) [above] {$R_4$};
        \node at(0,-1.4)[below] {$\Gamma$}; 
    \end{tikzpicture}
\end{center}
\begin{gather*}
    \oint_\Gamma \vv{E} \cdot \vv{dl} = \int_{A}^{B} \vv{E} \cdot \vv{dl} + \int_{B}^{C} \vv{E} \cdot \vv{dl} + \int_{C}^{D} \vv{E} \cdot \vv{dl} + \int_{D}^{A} \vv{E} \cdot \vv{dl} = 0\\
    (V_A - V_B) + (V_B - V_C) + (V_C - V_D) + (V_D - V_A) = 0
\end{gather*}
\subsection{parliamo dipiù della resistività}
Partendo da $\vv{E} \rho_R \vv{J}$ e considerando l'unità di misura della resistività:
\begin{gather*}
    [\rho_R] = \left[\frac{N}{C} \frac{m^2}{A}\right]  = \qquad V = \frac{J}{C}\\
    \left[ \frac{J m}{C A} \right] = \left[ V \frac{m}{A} \right] = \left[ \Omega m\right]   
\end{gather*}
Prendiamo l'esempio del traliccio dell'alta tensione, dove alcune mucche morivano, questo succede perchè anche se ovviamente il cavo è la parte che fa meno resistenza e il traliccio ha un punto di materiale isolante, comunque una parte 
della corrente passa attraverso il traliccio e quindi una piccola parte della corrente passa attraverso il traliccio, questo accadeva con le mucche per via dei passi molto lunghi, quindi 
facevano si di avere una differenza di potenziale tra le due zampe, e quindi una corrente che passava attraverso il loro corpo, e questo era sufficiente per ucciderle.\\
\hfill\\
Altra cosa è il fatto che la differenza di potenziale tra il terreno e le nuvole e circa $200 V$ per $m$, noi non prendiamo la scossa perchè fungiamo da conduttori, ma quando si accumula sulle nuvole comunque da qualche parte devono 
scaricarsi, e quindi quando si accumula una certa quantità di carica sulle nuvole, questa si scarica attraverso un fulmine\\
\subsection{Conduttore a sezione costante}
\begin{gather*}
    \vv{E} = \rho_R \vv{J}
\end{gather*}
\begin{center}
    \begin{tikzpicture}
        %cilindro
        \draw (-1,0) -- (1,0);
        \draw (-1,1) -- (1,1);
        \draw (-1,0.5) ellipse (0.3 and 0.5) node[above] {$S$};
        \draw (1,0.5) ellipse (0.3 and 0.5) node[above] {$S$};
        \node at(0,1) [above] {$\ell$};
        \draw[thin] (-1,0.5) -- (1,0.5);
        \draw[->] (1.1,0.5) -- (1.5,0.5) node[above] {$\widehat{n}$};
    \end{tikzpicture}
\end{center}
\begin{gather*}
    \int_{A}^{B} \vv{E} \cdot \vv{dl} = V_A - V_B = \int_{A}^{B} \rho_R \vv{J} \cdot \vv{dl} = \int_{A}^{B} \rho_R \frac{\vv{J} S}{S} \cdot \widehat{n} dl=\\
    I \underbrace{\int_{A}^{B} \frac{\rho_R}{S} dl}_{R} = I \frac{\rho_R \ell}{S} = I R
    \Delta V = I R
\end{gather*}
Quindi $R = \frac{\rho_R \ell}{S}$ è la resistenza elettrica, che è una proprietà del conduttore, e si misura in $\Omega$.\\
\subsection{applicazione delle resistenze}
\begin{center}
    \begin{tikzpicture}
        \draw(-2,0) -- (-0.5,0);
        \draw(0.5,0) -- (2,0);
        %resistenza
        \draw (-0.5,0) -- (-0.3,-0.5);
        \draw (-0.3,-0.5) -- (-0.1,0.5);
        \draw (-0.1,0.5) -- (0.1,-0.5);
        \draw (0.1,-0.5) -- (0.3,0.5);
        \draw (0.3,0.5) -- (0.5,0);
        \node at(0,0.5) [above] {$R$};
    \end{tikzpicture}
\end{center}
\subsection{esercizio extra}
Prendiamo un filo di rame $(Cu)$ lunghezza $\ell = 1m$ e raggio rella sezione $r = 1mm$, calcoliamo $R$:
\begin{gather*}
    R = \frac{1,7 \times 10^{-8} \Omega \cancel{m} 1 \cancel{m}}{\pi ( 10^{-3} \cancel{m})^2}
\end{gather*}
\end{document}
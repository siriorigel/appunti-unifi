\documentclass[a4paper, oneside]{article}
\usepackage{graphicx}
\usepackage{amsthm}
\usepackage{amsmath}
\usepackage{amssymb}
\usepackage[a4paper,
            bindingoffset=0.2in,
            left=2cm,
            right=2cm,
            top=2cm,
            bottom=2cm,
            footskip=.25in]{geometry}
\usepackage[italian]{babel}
\usepackage{pgfplots}
\usepackage{tabularx}
\usepackage{tikz}
\usepackage{wrapfig}
\usepackage{color}
\definecolor{page}{rgb}{0.129,0.157,0.212}
\pagecolor{page}
\color{white}
\graphicspath{ {./images/} }
\usetikzlibrary{shapes.geometric}
\usetikzlibrary{datavisualization}
\usetikzlibrary{datavisualization.formats.functions}
\pgfplotsset{width=10cm,compat=1.9}

\title{DOmande Geom}
\author{Tommaso Miliani}
\date{}

\begin{document}
\newtheoremstyle{theoremEnv}
                {}          % Space above
                {}          % Space below
                {\slshape}  % Body font
                {}          % Indent amount
                {\bfseries} % Head font
                {.}         % Punctuation after head
                {\newline}         % Space after theorem head
                {}          % Theorem head spec
\theoremstyle{theoremEnv}

\newtheorem{definition}{Definizione}[section]
\newtheorem{theorem}{Teorema}[section]
\newtheorem{lemma}{Proposizione}[section]
\newtheorem{observation}{Osservazione}[section]
\newtheorem{corollary}{Corollario}[theorem]
\newtheorem{example}{Esempio}[section]

\maketitle

\section{Domande}
\begin{itemize}
    \item Dimostrazione matrice ortogonale
    \item Dimostrazione relazione rango-determinante
    \item Operazioni di secondo tipo come affliggono il determinante di una matrice?
    \item Equivalenza rango per righe e rango per colonne
    \item Perché l'immagine di una matrice è generata dalle colonne di della matrice
    \item Dimostrazione dimensione dello spazio di partenza di una funzione lineare
    \item Polinomio caratteristico di una matrice
    \item Criterio di diagonalizzabilità
    \item Sottospazio somma 
    \item Molteplicità algebrica e geometrica
    \item Forma hermitiana e matrice hermitiana
    \item Gram Schmidt e Gram Schmidt due. 
    \item Modulo di una forma bilineareo
    \item Cosa è una base di uno spazio vettoriale
    \item Spazio duale
\end{itemize}

\end{document}
\documentclass[a4paper, oneside]{book}
\usepackage{graphicx}
\usepackage{amsthm}
\usepackage[a4paper,
            bindingoffset=0.2in,
            left=1in,
            right=1in,
            top=1in,
            bottom=1in,
            footskip=.25in]{geometry}
\usepackage[italian]{babel} % italian package
\usepackage{pgfplots} % plots
\usepackage{tabularx} % tables
\usepackage{chemfig} %chemical figures 
\usepackage[version = 4]{mhchem} %chem
\usepackage{wrapfig} %wrapping images in text
\graphicspath{ {./images/} }
\usetikzlibrary{datavisualization}
\usetikzlibrary{datavisualization.formats.functions}
\pgfplotsset{width=10cm,compat=1.9}

\title{Chimica I}
\author{Tommaso Miliani}
\date{2024/2025}

\begin{document}
\theoremstyle{definition}
\theoremstyle{theorem}
\theoremstyle{lemma}

\newtheorem{definition}{Definizione}[chapter]
\newtheorem{theorem}{Teorema}[chapter]
\newtheorem{lemma}{Proposizione}[theorem]

\maketitle

\tableofcontents

\chapter{Modello atomico della materia}
\section{La materia}
\subsection{Introduzione}
Le sostanze che ci circondano sono razionalizzabili sulla base del modello atomico
della materia: tutta la materia è costituita da miscele omogenee o eterogenee di sostanze
pure o individui chimici che si presentano in diversi stati di aggregazione.
Le \textbf{miscele omogenee} sono miscele di sostanze che non si possono fisicamente
distinguere tra loro; le \textbf{miscele eterogenee} sono miscele contenenti più fasi 
fisicamente distinte l'une dalle altre. \\
Una sostanza può anche essere \textbf{pura} se è un sistema che possiede una composizione costante: 
le sostanze pure si dividono in \textbf{elementari} (ossia atomi della stessa specie) o
\textbf{composta} se costituita da atomi di specie diverse come l'acqua.

\subsection{Elementi e formule delle sostanze}
Le specie naturali di elementi sono individuate sulla tavola periodica fino al numero
atomico 93. Dall'elemento 93 in poi iniziano gli elementi artificiali (o transuranici),
che presenti in natura. Gli atomi si uniscono per formare le \textbf{molecole}. \\
Le sostanze inoltre si possono rappresentare attraverso le \textbf{formule chimiche}:
\begin{enumerate}
    \item La \textbf{formula minima}, stechiometrica o elementare, si ricava
    dall'analisi elementare della sostanza come \ce{NaCl}, \ce{H2O}...
    \item La \textbf{formula molecolare} fornisce più informazioni di quella minima:
    nonostante possa sembrare simile, la formula molecolare ci dice esattamente
    di quanti e quali atomi è costituita una molecola e non le proporzioni tra gli atomi.
    (ad esempio \ce{NaSO4} è la minima di \ce{Na2S2O8}).
    \item Quando si vuole rappresentare bidimensionalmente una struttura molecolare, si ricorre alla
    \textbf{formula di struttura}: \vspace{0.5cm}\\
    \chemfig{O = C = O} \hspace{1cm} \chemfig{H - C(-[2]H)(-[-2]H) - H}
\end{enumerate}
Le sostanze inoltre, si dicono \textbf{organiche} se contengono atomi di carbonio
e idrogeno.

\section{L'atomo}
\subsection{Il modello atomico}
L'atomo è composto da un \textbf{nucleo} all'interno del quale ci sono i \textbf{protoni} ed i \textbf{neutroni},
mentre all'esterno del nucleo si muovono gli \textbf{elettroni}. Di queste particelle solo l'elettrone
è elementare, infatti più avanti si vedrà come il protone ed il neutrone siano
composti da particelle ancora più piccole chiamate \textbf{quark}.
Di seguito le proprietà delle particelle che compongono l'atomo:
\begin{center}
    \begin{tabular}{| m{3cm} | m{3cm} | m{3cm} | m{3cm} |}
        \hline
        Particella & Protone & Neutrone & Elettrone \\
        \hline
        Massa/kg & $1,673 \cdot 10^{-27}$ & $1,675 \cdot 10^{-27}$ & $9,1094 \cdot 10^{-31}$ \\
        \hline
        Massa/uma & $1,007277$ & $1,008665$ & $5,4851 \cdot 10^{-4}$\\
        \hline
        Carica/C & $1,6022 \cdot 10^{-19}$ & - & $-1,6022 \cdot 10^{-19}$ \\
        \hline
        Raggio/m & $8,751(61) \cdot^{-16}$ & $\sim 8 \cdot 10^{-16}$ & $10^{-18} > r > 10^{-22}$ \\ 
        \hline
    \end{tabular}
\end{center}
Il numero di protoni in un atomo si chiama \textbf{numero atomico} indicato con $Z$, mentre
il numero di protoni più il numero di neutroni dà il numero di massa. Un atomo, inoltre,
può essere uno ione se non ha più carica neutra: (+ = \textbf{catione}, - = \textbf{anione}).

\subsection{Isotopi e difetto di massa}
Dal momento che il numero di massa atomica dipende dal numero di protoni e neutroni ed il
numero di protoni definisce un elemento, ogni elemento può avere più masse atomiche pur
mantenendo le stesse proprietà: questi elementi si chiamano \textbf{isotopi}. \\
Dal momento che la massa è contenuta per la maggior parte nel nucleo, come è possibile
che i protoni possano rimanere così tanto attaccati pur risentendo di una forte forza di
repulsione? Se il protone ed il neutrone fossero scinti dal nucleo, intanto, il protone potrebbe
continuare a vagare per ben $10^{25}$ anni prima di poter avere il $50\%$ di possibilità
di decadere; mentre il neutrone decadrebbe dopo soli $886$ secondi. \\
Ciò che accade è che esiste una forza, chiamata \textbf{forza nucleare forte}, mediata
da delle particelle chiamate \textbf{gluoni} le quali agiscono sui \textbf{quark} con
una intensità $100$ volte maggiore della forza elettrostatica che fa respingere tra loro
i protoni.\\
Per poter scindere un atomo, bisognerebbe utilizzare un'energia pari alla massa atomica per
la velocità della luce, come dimostrato dall'equazione $E = mc^{2}$.
Tuttavia, la massa effettiva di un atomo è minore della massa dei protoni e dei neutroni combinati.
Questa perdita di massa  si chiama \textbf{difetto di massa} $\delta m$ e rappresenta
l'energia necessaria per scindere il nucleo nei suoi componenti (anche chiamata
\textbf{energia di legame nucleare})che cresce quasi linearmente con l'aumentare dei protoni.
Il principio di fusione nucleare sfrutta il fatto che fondendo due atomi di idrogeno in deuterio,
questi incontrano difetto di massa e fondendoli a loro volta in elio, questi subiscono un'ulteriore
difetto di massa che libera moltissima energia.

\subsection{Nuclidi stabili}
Generalmente un nuclide è stabile se rimangono inalterati per moltissimo tempo (più di quello
che siamo in grado di misurare con accuratezza) e generalmente gli atomi più stabili sono
quelli con numero Z e N pari, in quanto tendono ad accoppiarsi a due a due.
Via via che il numero di protoni aumenta però, ci si rende conto che il numero di neutroni tende
ad essere sempre $\geq$ al numero di protoni in modo da poter bilanciare le cariche nel nucleo.
Nuclei molto grandi tuttavia, diventano molto instabili e tendono ad avere un tempo di
decadimento sempre più basso. In natura esistono $206$ nuclidi stabili.

\section{Peso atomico degli elementi}
\subsection{Massa atomica e peso atomico degli elementi}
L'unità di misura della massa atomica è definita \emph{massa relativa} e come unità
di misura è stato scelto un dodicesimo della massa del nuclide \ce{^{12}C}.
Per determinare la massa atomica di un elemento formato da più isotopi bisogna fare 
una media pesata delle masse dei singoli nuclidi che lo compongono; per \ce{^{1}H} e \ce{^{2}H}:
\[
    \ce{^{1}H} = 1,007825 (99,985\%) \quad \ce{^{2}H} = 2,014102 (0,015\%)
\]
La massa atomica media risulta essere dunque $1,007976$.
Questo numero si chiama anche \textbf{peso atomico} (è solo un numero puro però) mentre
il \textbf{peso atomico standard} è il peso atomico per la miscela isotopica naturale
di un dato elemento. Si può anche definire il \textbf{peso molecolare}, ossia la somma dei 
pesi atomici in una molecola. Se non si è certi del numero di molecole di una sostanza, allora
si parla di \textbf{peso formula}, ossia la somma dei pesi molecolari degli
atomi che compaiono nella formula della sostanza come $\ce{NaCl} = 58,443$. 

\section{Grandezze fisiche fondamentali e mole}
\subsection{La mole}
Il SI ha 7 grandezze fondamentali (vedere lab I); quella più importante per la chimica
è sicuramente la \textbf{mole} ossia la misura della quantità di sostanza.
La mole in chimica definisce la quantità di sostanza in quanto non si è in grado
di poter lavorare con unità di misura come il kilo poiché gli oggetti trattati 
sono dell'ordine dei nanometri.
L'unità dio misura della mole è definita dunque come la \emph{quantità di sostanza
che contiene esattamente $6,02214076 \cdot 10^{23}$ unità elementari}, anche chiamata
\textbf{costante di Avogadro}:
\[
    N_A = 6,02214076 \cdot 10^{23} mol^{-1}
\]
La \textbf{massa molare} è il rapporto tra massa e quantità di sostanza indicata 
con $M$ e coincidendo (alla $9^a$ cifra decimale) al peso atomico della sostanza.
Per la \ce{CO2} è:
\[
    M = \frac{m (g)}{n (mol)} = \frac{44,011 g}{1 mol} = 44,011 g \cdot mol^{-1}. 
\] 

\section{Energia, temperatura e stati della materia}
\subsection{Energia}
L'\textbf{energia} è la capacità di svolgere lavoro da un sistema, manifestandosi
sotto diversi aspetti:
\begin{enumerate}
    \item L'\textbf{energia cinetica}:
    \[
        E_{cin} = \frac{1}{2}mv^{2},
    \]
    \item L'\textbf{energia potenziale}:
    \[
        E_{pot} \ (gravitazionale) = mgh,
    \]
    \item L'\textbf{energia potenziale elettrostatica} che risente un elettrone nel campo di un nucleo:
    \[
        E_{pot} \ (elettrostatica) = -\frac{Ze^{2}}{r}
    \]
    \item L'\textbf{energia complessiva}:
    \[
        U = E_{pot} + E_{cin}
    \]
\end{enumerate}

\subsection{Stati di aggregazione della materia}
Le particelle si aggregano tra di loro quando l'energia del sistema aggregato
diminuisce rispetto a quella del sistema isolato. Nei gas vale:
\[
    \langle E_{cin}\rangle = \frac{1}{2}m\langle v^{2}\rangle = \frac{3}{2}k_B T.
\]
dove $k_B = 1,380649 \cdot 10^{-23} J K^{-1}$ è la costante di Boltzmann. Se si fornisce
energia ad un gas, le sue molecole si muoveranno più velocemente e di conseguenza
aumenterà la temperatura del gas stesso.
Aumentando sempre di più la temperatura si raggiunge la ionizzazione di un gas e lo stato
della materia adesso prenderà il nome di \textbf{plasma}, ossia uno stato in cui gli
elettroni non appartengono èiù ad un atomo ma sono liberi di muoversi.
Ci sono diversi tipi di legame tra atomi:
\begin{center}
    \begin{tabular}{| m{3cm} | m{3cm} |}
        \hline
        Legame & Energia \\
        \hline
        Covalente & $10^{2} - 10^{3} \ kJ \ mol^{-1}$\\
        \hline
        Ionico & $10^{2} - 10^{3} \ kJ \ mol^{-1}$\\
        \hline
        Metallico & $10^{1} - 10^{3} \ kJ \ mol^{-1}$\\
        \hline
        Idrogeno & $10 - 40 \ kJ \ mol^{-1}$\\
        \hline
        Van der Waals & $0,1 - 10 \ kJ \ mol^{-1}$\\
        \hline
    \end{tabular}
\end{center}

\section{Unità di misura nel SI e loro definizione}
\begin{center}
    \begin{tabular}{| m{1.2cm} | m{1cm} | m{0.8cm} | m{1.5cm} | m{8.7cm} |}
        \hline
        Nome & Unità & Dim.  & Dicitura & Definizione \\
        \hline
        secondo & s, \emph{t} & T  & tempo & La durata di 9192631770 periodi della 
        radiazione corrispondente alla transizione tra due livelli iperfini
        nell'atomo di cesio 133 al suo stato fondamentale\[
            1 \ s = \frac{9192631770}{\Delta \nu_{Cs}} 
        \] \\
        \hline
        metro & m, \emph{l} & L &  lunghezza & La distanza percorsa dalla luce nel
        vuoto in 1/299792458 secondi \[
            1 \ m = \frac{9192631770}{299792458} \cdot \frac{c}{\Delta \nu_{Cs}}
        \] \\
        \hline
        kilo & kg, \emph{m} & M & massa & Il kilogrammo è definito ponendo la
        costante di Planck \emph{h} esattamente uguale a $6,62607015 \cdot 10^{-34}
        J \cdot s \cdot (kg \cdot m^2 \cdot s)$ \[
            1 \ kg = \frac{299792458^2}{9192631770 \cdot 6,62607015 \cdot 10^{-34}}
            \cdot \frac{h \Delta \nu_{Cs}}{c^2}
        \]\\
        \hline
        ampere & A, \emph{i} & I & corrente elettrica & Il flusso di esattamente $1,602176634
        \cdot 10^{-19}$ volte la carica elementare per secondo: \[
            1 \ A = \frac{e \Delta \nu_{Cs}}{1,602176634 \cdot 10^{-19} \cdot 9192631770}
        \] \\
        \hline
        kelvin & K, \emph{T} &  & temp. termodinamica & Il Kelvin è definito
        ponendo il valore numerico della costante di Boltzmann: \[
            1 \ K = \frac{1,380649 \cdot 10^{-23}}{9192631770 \cdot 6,62607015 \cdot 10^{-34}}
            \cdot \frac{k_B}{h \Delta \nu_{Cs}} 
        \] \\
        \hline
        mole & mol, \emph{n} & N & quantità di sostanza & La quantità di sostanza che
        contiene esattamente $6,02214076 \cdot 10^{23}$ unità elementari. \\
        \hline
        candela & cd & J & intensità luminosa & L'intensità luminosa, in una qualunque
        direzione, di una sorgente che emette una radiazione monocromatica alla frequenza
        di $5,4 \cdot 10^{14}$ Hz (verde) e che ha un'intensità radiante di 
        1/683 watt per steradiante. 
        \[
            1 \ cd = \frac{1}{683 \cdot 9192631770^2 \cdot 6,62607015 \cdot 10^{-34}}
            \cdot \frac{K_{cd}h \Delta \nu_{Cs}^{2}}{sr}
        \] \\
        \hline
    \end{tabular}
\end{center}

\chapter{Modello elettronico degli atomi e proprietà periodiche}
\section{Introduzione alla meccanica quantistica}
L'elettrone è la particella dotata di  carica negativa mentre il protone 
quella con carica positiva. Sarebbe dunque logico se i protoni e gli elettroni nel nucleo
si potessero attrarre tra di loro causando il collassino degli elettroni  nel nucleo. 
Tuttavia nessun modello fisico è in grado di spiegare come questo non accada. L'unica
teoria che è in grado di spiegarlo è la \textbf{meccanica quantistica}, che tratta l'energia
come se fossero dei pacchetti chiamati \textbf{quanti}, e dei quali non è possibile 
determinare contemporaneamente posizione e quantità di moto.

\subsection{Radiazione elettromagnetica}
\begin{wrapfigure}{r}{0.35\textwidth} \label{Fig 2.1}
    \centering
    \caption{Un'onda elettromagnetica}
    \includegraphics[width=0.35\textwidth]{onda}
\end{wrapfigure}
La luce è una porzione ristretta della radiazione elettromagnetica, la quale si propaga
nel vuoto con velocità $c$, ed ha altre proprietà come la \textbf{frequenza}, ossia il numero
di volte in cui il vettore campo elettrico assume l'intero ciclo di valori in un secondo ($\nu$),
e la lunghezza di questa oscillazione, ossia la \textbf{lunghezza d'onda} ($\lambda$).
Inoltre, l'\textbf{ampiezza} dell'onda è il massimo valore assunto dal campo elettrico o magnetico,
mentre l'\textbf{intensità} della radiazione è proporzionale al quadrato dell'ampiezza delle onde elettrica
o magnetica. Data una direzione di propagazione, esistono ortogonali ad essa ed ortogonali fra loro
un campo elettrico ed uno magnetico, ed i loro vettori hanno andamenti sinusoidali in
fase tra l'uno e l'altro. \\ 
Se si potesse vedere un'onda elettromagnetica, si osserverebbe un fenomeno chiamato
\textbf{treno d'onda}, o \textbf{pacchetti d'onda}, in quanto la luce, è emessa a "pacchetti",
poiché ha una frequenza limitata. Così come vedremmo passare i vagoni di un treno
con una frequenza se i vagoni sono corti, più è corta la lunghezza d'onda, maggiore 
sarà la sua frequenza.
In una lampadina, i fasci di luce sono emessi in modo \textbf{non coerente}, ossia
sono emessi con fasi diverse portando ad un'interferenza distruttiva.
Quando un fascio di luce ha tutti i pacchetti della medesima frequenza, la radiazione
diventa \textbf{monocromatica}, altrimenti si chiama \textbf{policromatica}. I fasci
di luce si separano per \textbf{diffrazione}, mentre l'insieme di fasci luminosi si
chiama \textbf{spettro} (Fig 2.2).
\begin{figure}
    \label{Fig 2.2}
    \centering
    \caption{Lo spettro elettromagnetico}
    \includegraphics[width=0.90\textwidth]{onde}
\end{figure}

\subsection{Quantizzazione dell'energia}
Ogni sistema fisico non può assumere qualsiasi valore di energia ma può solo assumere 
determinati valori di energia, per cui per poter passare da un livello $E_1$ ad un livello
$E_1$ dovremmo fornire un'energia pari a:
\[
    E_2 - E_1 = h \nu
\] 
\textbf{assorbendo} questo quanto di energia, il sistema innalzera il suo livello di energia
e potrà comunque tornare al livello energetico precedente attraverso l'\textbf{emissione spontanea}
di un quanto. Se il sistema a livello $E_2$ è colpito da un quanto di luce e torna allo stato
energetico $E_1$, allora il sistema avrà emesso un quanto con la stessa fase di quello
da cui è stato colpito ma di intensità doppia (\textbf{emissione stimolata}).
Dall'equazione di Einstein deriva che:
\[
    E = mc^2 \quad \Rightarrow \quad h\nu = mc^2 
\]
La \textbf{massa relativistica} dei fotoni è (con massa invariante nulla):
\begin{equation}
    m = \frac{h \nu}{c^2}
\end{equation}
Ne risulta che fotoni con lunghezza d'onda minore hanno massa maggiore (e quindi più energia).

\subsection{Dualismo onda particella e principio di indeterminazione di Heisenberg}
Un'elettrone che si muove con velocità v ha un'energia $E = mv^2 = h \nu$, quindi:
\begin{equation}
    \lambda = \frac{h}{mv}.
\end{equation}
Questa è l'idea di base della \emph{meccanica ondulatoria}: la luce è sia
un'onda che una particella poiché si comporta come un'onda ma possiede una massa.
Nel mondo microscopico la dualità onda particella non può essere ignorata mentre nel
mondo macroscopico, date le enormi masse rispetto a quelle delle particelle subatomiche,
le lunghezze d'onda sono talmente piccole da essere impossibile da rilevare. \\
Heisenberg nel 1925 arriva a concludere che alla base della meccanica quantistica
sta il \textbf{principio di indeterminazione}: secondo questa legge il prodotto degli errori 
nella determinazione della quantità di moto e della posizione di un elettrone non 
può essere minore di una certa quantità:
\begin{equation}
    \Delta x \Delta p \geq \frac{h}{2\pi}
\end{equation}
Nel mondo macroscopico questi errori sono talmente piccoli da essere completamente 
trascurabili, mentre per un elettrone no: se si avesse un'errore di $1/10^{10}$ sulla 
quantità di moto di un elettrone, l'errore risultante sulla posizione darebbe circa $10^6$ metri!.
Il principio di indeterminazione si applica a tutte le misure \textbf{coniugate}, ossia tutte
quelle misure che hanno le dimensioni di h (qta. di moto e posizione o energia e tempo).
Cercare di osservare la quantità di moto di un pacchetto o la posizione è 
come decidere se cercare di osservare un'onda o una particella. 

\section{Interazione radiazione elettromagnetica - elettrone}
\subsection{L'effetto fotoelettrico}
Esponendo una lastra di metallo sottovuoto ad una radiazione elettromagnetica, si ha
emissione di elettroni, i quali, sono raccolti su un elettrodo carico positivamente.
Gli elettroni sono legati al metallo con una energia $E_0$ che dipende dal tipo di metallo,
\[
    se \ E = h \nu \geq E_0
\]
si ha emissione di elettroni, la frequenza al di sopra della quale si ha emissione si
chiama ($\nu_0$) \textbf{frequenza di soglia}. Ciò che è importante notare è che non
importa quanti fotoni al di sotto della frequenza di soglia si mandino al metallo; nessun
elettrone sarà emesso. Il numero di elettroni emessi dipende dunque dal numero di 
fotoni con $E \geq E_0$. L'energia in eccesso diventa energia cinetica:
\begin{equation}
    \Delta E = E - E_0 = \frac{1}{2}mv^2
\end{equation} 
Questo effetto fu spiegato da Einstein nel 1905 e questo gli garantì il nobel nel 1921.

\subsection{Lo spettro dell'atomo di idrogeno}
L'atomo di idrogeno è costituito da un protone e da un elettrone, essendo che l'elettrone
può assumere solo alcune energie permesse, se bombardato con l'intero spettro luminoso,
l'idrogeno emetterà soltanto le frequenze che è riuscito ad assorbire: i due spettri risultanti
si chiameranno rispettivamente \textbf{spettro di emissione} e \textbf{spettro di assorbimento}
Le energie permesse dall'elettrone sono quindi vincolate dalla relazione:
\begin{equation}
    E = - \frac{1}{n^2}\cdot \frac{2k_C^{2}\pi^2me^4}{h^2}
\end{equation}
dove $n \in N$, m la massa dell'elettrone, $e$ la sua carica e $k_C$  una costante 
pari a $8,99 \cdot 10^9 JmC^2$. $n$ è il \textbf{numero quantico principale}, tutti gli stati
superiori ad uno nell'atomo di idrogeno si chiamano $stati \ eccitati$. \\
Un'elettrone colpito da un certo fotone in grado di farlo eccitare, ha due possibilità
una volta che il suo n è aumentato: emettere un fotone e tornare ad n = 1 oppure
passare prima per uno stadio intermedio emettendo così due fotoni. L'energia di
transizione da uno stato eccitato $n_j$ ad uno stato fondamentale o di minor energia $n_i$:
\begin{equation}
    \Delta E = h\nu = \frac{2k_C^{2}\pi^2me^4}{h^2}\left(\frac{1}{n_i^2}-\frac{1}{n_j^2}\right)
\end{equation} 
Un livello di energia dato da un $n = \infty$ è il livello di energia di un elettrone
che non interagisce più con un dato nucleo

\section{L'elettrone nell'atomo di idrogeno (Il modello di Bohr)}
\subsection{Equazione di Schrodinger}
Determinare un elettrone all'interno di un orbitale non è facile come determinare
un elettrone che si muove nello spazio, Schrodinger sviluppò un'equazione differenziale
le cui soluzioni, chiamate \textbf{funzioni d'onda}, sono le funzioni desiderate:
\begin{equation}
    \hat{H}\psi = E\psi
\end{equation}
dove $\psi$ è la funzione d'onda e $\hat{H}$ l'operatore hamiltoniano dell'energia.
\begin{definition}
    $\hat{H}$ è un'operatore \textbf{monadico}, ossia agisce sulla funzione trasformandola,
    a differenza di un operatore \textbf{diadico} (come +, -, $\times$ e /) che ne  ha bisogno
    di due. Se invece un operatore moltiplica una funzione per una costante senza alterarla,
    come: $f(x) = x \ \Rightarrow 3f(x) = 3x$, il numero 3 è un \textbf{autovalore} e la nuova funzione
    è un \textbf{autofunzione} di quell'operatore. \\ 
    Nell'equazione di Schrodinger $\psi$ è un'autofunzione dell'operatore hamiltoniano, per l'atomo
    di idrogeno questo diventa: 
    \[
        \hat{H} = \frac{\partial^2}{\partial x^2} + \frac{\partial^2}{\partial y^2} +
        \frac{\partial^2}{\partial z^2} + \frac{8 \pi^2 m}{h^2}V
    \]
    Nell'elettrone inoltre $V = -e^2/r$ dove r è la distanza dal nucleo.
\end{definition}
L'equazione nella sua forma più estesa diventa quindi:
\begin{equation}
    \frac{\partial^2 \psi}{\partial x^2} + \frac{\partial^2 \psi}{\partial y^2} +
    \frac{\partial^2 \psi}{\partial z^2} + \frac{8 \pi^2 m}{h^2}V \psi = E\psi
\end{equation}
$\psi$ ammette infinite soluzioni all'equazione, a ciascuna delle equali è associata
$\psi_i$ ed un corrispondente valore dell'energia data da $E_i$. Questo tuttavia
non contraddice il principio di indeterminazione poiché non si potrà mai conoscere
il tempo per cui l'elettrone manterrà quel livello di energia: l'elettrone si
deve trovare dunque in uno \textbf{stato stazionario}.
Inoltre, per ogni stato $n_i$ di energia $E_i$ dell'atomo di idrogeno corrispondono
ben $n^2$ funzioni $\psi$ diverse. \\
Prendendo un volume $d\tau$ molto piccolo, la funzione $\psi^2 d\tau$ è la probabilità
di trovare l'elettrone in quel determinato volume, ossia la risoluzione dell'integrale
di questa funzione in quella regione, $\psi^2$ è la \emph{densità elettronica} in quel volume.

\subsection{L'atomo di idrogeno: numeri quantici ed orbitali}
Risolvendo l'equazione di Schrodinger per l'atomo di idrogeno, si nota che più funzioni
d'onda possono dare la stessa energia e sono dette quindi \textbf{degeneri}. Ogni
funzione d'onda è caratterizzata da tre parametri chiamati \textbf{numeri quantici}.

\begin{wrapfigure}{r}{0.50\textwidth} \label{Fig 2.3}
    \centering
    \caption{Alcuni orbitali}
    \includegraphics[width=0.50\textwidth]{orbitali}
\end{wrapfigure}

\begin{enumerate}
    \item \textbf{Numero quantico principale}: indicato col simbolo $n$ ed è un numero
    naturale e determina le energie permesse.
    \item \textbf{Numero quantico secondario}: indicato con $l$, assume qualsiasi valore intero
    $0 \leq l \leq n -1$ e determina il \emph{momento angolare} di un sottolivello.
    \item \textbf{Numero quantico magnetico}: indicato come $m_i$, assume tutti i valori interi
    $-l \leq m_i \leq l$.
\end{enumerate}

E' da notare come la funzione d'onda tenda a dare lo $0\%$ di probabilità quando
il raggio r dal nucleo tende ad infinito. Le forme geometriche degli orbitali hanno
dunque senso se e solo se si sceglie un valore di probabilità inferiore al 100\%.
Un'orbitale diventa dunque una regione di spazio delimitato dalla superficie $\psi^2$
costante. Le forme degli orbitali nelle figure sono sempre volumi al cui interno
si ha il 90\% di probabilità di trovare l'elettrone. La forma geometrica dell'orbitale cambia
dunque a seconda del valore di $\psi^2$ selezionato. \\
Il numero quantico $l$ è responsabile per identificare la forma dell'orbitale mentre il
numero magnetico $m$ definisce l'orientazione dell'orbitale.

\subsection{La distribuzione della probabilità radiale}
Prendendo in considerazione il volume di un cubo molto piccolo $d\tau$ ($\psi^2$), la probabilità di
trovare l'elettrone nell'orbitale 1s ha densità elettronica finita sul nucleo: addirittura
Fermi nel 1930 propose che l'orbitale 1s potesse stare a contatto col nucleo.

\begin{wrapfigure}{l}{0.50\textwidth} \label{Fig 2.4}
    \centering
    \caption{Funzione orbitale 1s}
    \includegraphics[width=0.40\textwidth]{Orbitale-1s.png}
\end{wrapfigure}

E' possibile rappresentare dunque rappresentare la funzione dell'orbitale utilizzando
una sfera di raggio r molto piccolo definita come $4\pi r^2 dr$. La frazione totale di
elettrone che si trova distanza r dal nucleo è data dal prodotto $\psi^2$ per il guscio
appena descritto ed ha il massimo in $a_0 = 53 pm$. \\
Invece, nell'orbitale 2s, le tre funzioni nel grafico diventano zero quando raggiungono
$2a_0$, quindi la probabilità di trovare l'elettrone $\psi^2 = 0$, ovvero diventa una
\textbf{superficie nodale}. Un'orbitale nello strato n avrà $n-1$ superfici nodali,
la cui geometria sarà definita da l ed m. \\
I valori di $\psi$ dipendono dai tre numeri quantici, e possono avere sia segno negativo
che positivo: questo perché in un atomo i vari orbitali si sovrappongono ed il segno di
$\psi$ ci dice che la sovrapposizione delle funzioni d'onda è sempre zero, mentre il segno di
di $\psi^2$ è sempre positivo dato il quadrato. Quando invece le densità elettroniche,
che sono proporzionali a $\psi^2$ si sovrappongono e si compenetrano, vi è competizione
tra i vari orbitali dello stesso strato, per cui gli orbitali s risulteranno sempre quelli
più penetranti e si troveranno con più probabilità vicini al nucleo rispetto agli orbitali
p e d.

\begin{figure}
    \label{Fig 2.5}
    \centering
    \caption{Funzione orbitale 2s}
    \includegraphics[width=0.40\textwidth]{Orbitale-2s.png}
\end{figure}

\subsection{Il numero quantico di spin}
A tutte le particelle elementari è associata una proprietà intrinseca chiamata \emph{spin},
le particelle si comportano infatti come se ruotassero attorno al proprio asse: gli elettroni
possono assumere solo $+1/2$ o $-1/2$. Queste particelle con spin semintero (dette fermioni),
non possono coesistere nello stesso stato energetico con lo stesso spin per il principio di Pauli.\\
Associato al momento angolare c'è sempre un \textbf{momento magnetico}: questo momento
magnetico fa si che una particella si orienti all'interno di un campo magnetico, dando
proiezioni dello spin della particella sull'asse del campo magnetico
chiamata \textbf{asse di quantizzazione}.

\section{Configurazione elettronica degli atomi polielettronici e tavola periodica}
L'Equazione di Schrodinger può essere risolta in modo accurato solo per l'atomo di
idrogeno perché per atomi con più elettroni l'equazione può essere risolta solo in modo
approssimativo a causa del potenziale del nucleo dotato da più protoni ed elettroni. \\
Anche se approssimative, le soluzioni dell'equazioni ci dice che si può ancora parlare
di orbitali 1s, 2s... anche per gli atomi polielettronici. In questi atomi però, l'energia
dell'orbitale non dipende più dal numero n ma anche da l. Tuttavia, all'aumentare di Z,
l'energia dell'orbitale diminuisce in modo sostanziale, anche se non come ci si aspetti poiché
il potenziale del nucleo è "nascosto" dagli altri elettroni nell'atomo. Per queste motivazioni
l'orbitale 4s ha meno energia dell'orbitale 3d e, per questo, il 4s si riempie prima, ma dopo Z = 0,
l'energia del 3d diminuisce ancora, permettendo di riempirlo prima. \\
\begin{wrapfigure}{l}{0.20\textwidth} 
    \label{Fig 2.6}
    \centering
    \caption{Ordine di riempimento degli orbitali}
    \includegraphics[width=0.20\textwidth]{Ordine-orbitali.jpg}
\end{wrapfigure}
Lo stesso comportamento si osserva per gli orbitali 4d e 4f, e questo spiega la posizione
anomala degli elementi di \emph{transizione, lantanoidi e attinoidi} nella tavola periodica.  \\
Generalmente la configurazione elettronica di un atomo si indica come segue:
\[
    Carbonio \ (C) = \ 1s^2 2s^2 2p^2 \qquad oppure \qquad [He]2s^2 2p^2
\]
La prima notazione è la configurazione elettronica completa, mentre la seconda utilizza
la configurazione elettronica del gas nobile più "a sinistra" mentre $2s^2 2p^2$ è la 
\textbf{configurazione elettronica esterna}.
Il riempimento degli orbitali segue la regola proposta in \ref{Fig 2.6} anche se
esistono molti elementi che completano prima orbitali con energia superiore di altri.
La tavola periodica è inoltre suddivisa in diversi gruppi di elementi: il gruppo 1
è il gruppo dei metalli alcalini, quelli del gruppo 2 sono i metalli alcalino-terrosi,
quelli dei gruppi 3-12 sono i metalli di transizione, quelli del gruppo 15 sono i
pnicogeni, il 16 sono i calcogeni, il 17 sono gli alogeni ed il 18 i gas nobili.

\section{Proprietà periodiche}
\subsection{Energia di ionizzazione}
\begin{wrapfigure}{l}{0.65\textwidth} 
    \label{Fig 2.7}
    \centering
    \caption{Energia di prima ionizzazione}
    \includegraphics[width=0.65\textwidth]{Ionizzazione.png}
\end{wrapfigure}
L'energia di ionizzazione è l'energia minima che occorre fornire ad un atomo
\emph{isolato gassoso} per togliergli un elettrone. L'\textbf{energia di prima ionizzazione}
è quando si sottrae un elettrone ad un atomo neutro. Dal momento che ogni volta che
si sottrae un elettrone ad un atomo esso riduce le sue dimensioni, l'energia di seconda,
terza e di n-esima ionizzazione saranno via via crescenti. Sulla tavola periodica l'energia
di prima ionizzazione cresce lungo ciascun periodo e salendo lungo un gruppo. La crescita
lungo un periodo tuttavia non è costante ma è discontinua a causa della configurazione elettronica
che spesso si altera seconda degli orbitali che vengono toccati: nei metalli di transizione spesso
accade che si tolgano elettroni dello strato $ns$, il che li avvicina alla configurazione dell'atomo
precedente, ma dal momento che gli altri orbitali sono più vicini al nucleo, questi risentiranno
di una carica positiva molto più alta che porta ad avere una aumento di energia di ionizzazione.

\subsection{Affinità elettronica ed elettronegatività}
L'\textbf{affinità elettronica} è l'energia necessaria per legare un elettrone ad un atomo gassoso isolato,
se è positiva allora il processo è molto favorevole.
Generalmente il processo di aggiunta di un secondo elettrone è sempre sfavorevole
in termini energetici a causa della carica globale negativa dell'atomo che ha già un elettrone di troppo.
Il concetto di affinità elettronica e di energia di prima ionizzazione ci permettono
di definire anche il concetto di \textbf{elettronegatività}, ossia la misura della
tendenza di un atomo ad attrarre le coppie di elettroni di legame. L'elettronegatività
si determina a partire dall'equazione di Mulliken: 
\begin{equation}
    \chi = k (E_I + A)
\end{equation}

Utilizzando la scala di Mulliken non si hanno le elettronegatività di tutti gli elementi,
così si utilizza la scala Allred-Rochow, la quale calcola il valore di attrazione di
un nucleo sull'elettrone di un doppietto. \\
Generalmente l'elettronegatività cresce lungo il periodo e diminuisce lungo un gruppo.

\subsection{Dimensioni atomiche}
Le dimensioni di un atomo non possono essere misurate empiricamente in quanto variano
a seconda degli altri atomi a cui sono legati, tuttavia la scala di raggio atomico proposta
da J.C Slater ci fornisce una indicazione del raggio atomico di atomi legati in sostanze
elementari con una deviazione standard di circa $12 pm$.
IL raggio atomico cresce lungo un gruppo tranne che per il passaggio tra $4d$ e $5d$ in
quanto dopo l'Afnio si ha la \textbf{contrazione lantanoidica} a causa del riempimento
degli orbitali $4f$ (interni). \\
Quando un atomo perde un elettrone, esso si carica positivamente e i suoi orbitali si
contraggono (la contrazione è più accentuata quando scompare completamente un orbitale),
mentre quando si carica negativamente, gli orbitali si espandono.



\chapter{Legame covalente}
\section{Il legame chimico}
Con la parziale eccezione dei gas nobili, tutti gli elementi della tavola periodica hanno
la tendenza a formare legami con altri atomi, questa tendenza nasce dal principio di 
\textbf{minima energia}: secondo questo principio gli atomi tendono a formare legami
tra loro se questo abbassa la loro energia. 

\subsection{Il legame covalente nelle molecole biatomiche}
La formazione di un legame chimico avviene solo quando l'energia dei due atomi legati è
minore di quella dei due atomi isolati: ossia se le forze di repulsione/attrazione si bilanciano
perfettamente. \\
Nei gas nobili generalmente questo non avviene in quanto hanno tutti gli orbitali completi e 
generalmente si respingono tra loro, tuttavia per altri atomi, gli elettroni che si 
avvicinano di entrambi gli atomi tenderanno a disporsi con spin antiparallelo e quindi
soddisfano il principio di minima energia andando a mettere in comune un elettrone e
facendo così aumentare la densità elettronica tra i due nuclei. Se si avvicinano troppo,
le forze repulsive li faranno allontanare, altrimenti, se si dispongono alla \textbf{distanza di legame},
formeranno un legame covalente.

\section{Il legame chimico secondo il metodo dell'orbitale molecolare}
Un modello che possa spiegare il legame covalente deve riflettere la sua natura quantomeccanica,
in analogia con quella dell'elettrone: estendendo il concetto di struttura elettronica sovrapponendo
gli orbitali e considerando il diverso potenziale elettrostatico dovuto alla presenza di più
di un nucleo e quindi bisogna considerare l'estensione degli orbitali su almeno due atomi della
molecola.

\subsection{Molecole e ioni biatomici del primo periodo}
Risolvendo l'equazione di Schrodinger si ottengono funzioni d'onda del sistema (considerati idrogeno ed elio)
che sono in questo caso orbitali molecolari. Il potenziale è tuttavia descritto non più da
un solo nucleo ma da ben due nuclei atomici a distanza r, per cui l'operatore Hamiltoniano
diventa per due orbitali 1s dei due atomi isolati:
\[
    \psi_L = \frac{1}{\sqrt{2}}\psi_1 + \frac{1}{\sqrt{2}}\psi_2
\]
ossia l'\textbf{orbitale molecolare di legame} e si indica con $\sigma$, l'altro è invece
descritto da;
\[
    \psi_A = \frac{1}{\sqrt{2}}\psi_1 + \frac{1}{\sqrt{2}}\psi_2
\]
ed è l'\textbf{orbitale molecolare di antilegame} e si indica come $\sigma^*$.
Questa è tuttavia un'approssimazione (con buona precisione) in quanto le
equazioni valgono solo con un elettrone in gioco.
L'aumento della densità elettronica di $\psi_L$ rispetto ai due atomi isolati
riduce la repulsione tra i nuclei e quindi l'energia 
\[
    \hat{H}_M\psi_L = E_L\psi_L
\]
è minore rispetto a quella della combinazione dei due orbitali atomici.
L'energia
\[
    \hat{H}_M\psi_A = E_A\psi_A.
\]
è invece maggiore di quella dei due orbitali atomici.
La molecola di \ce{H2} possiede due elettroni che si possono sistemare entrambi con spin
antiparallelo nell'orbitale molecolare di legame $\sigma$. Il risultato è che il sistema
ha diminuito la propria energia. In questo caso l'orbitale di legame $\sigma$ presenta
due elettroni e si indica come $\sigma^2$, mentre l'orbitale di antilegame sarà
$\sigma^{*}$ poiché non ci sono elettroni non di legame mentre nel caso di
\ce{H2^{-}} l'orbitale di antilegame sarà $\sigma^{*1}$. \\
Si ottiene quindi che \textbf{l'ordine di legame} si ottiene sottraendo al numero
degli elettroni negli orbitali di legame quello degli elettroni 
negli orbitali di antilegame e dividendo per due. Una molecola come \ce{He2} non può
esistere poiché gli orbitali sarebbero $\sigma^2$ e $\sigma^{*2}$ per cui l'energia
che si è ridotta andando a creare l'orbitale $\sigma^2$ è bilanciata da quella spesa
per l'orbitale di antilegame.

\subsection{La sovrapposizione degli orbitali atomici e gli orbitali
molecolari per le molecole biatomiche del secondo periodo}
Nelle condizioni di formazione di un legame, i due elettroni occupano la stessa
regione caratterizzata da alta densità elettronica, quest'area in comune è chiamata
\textbf{area di sovrapposizione}. Si può dire che il legame chimico covalente risulta 
dalla sovrapposizione di due orbitali atomici di due atomi contenenti ciascuno un elettrone
creando così un unico orbitale molecolare di legame e di antilegame. Il risultato è identico
anche se si avvicinano un \ce{H^{+}} ed un \ce{H^{+}}. \\
Perché dopo il primo periodo non si considerano più gli orbitali $1s$ nei legami?
Questo è dovuto alla dispersione energetica: una volta che $\sigma$ e $\sigma^*$ sono stati
riempiti, non c'è più guadagno energetico e quindi gli orbitali si combinano per
ridare funzioni atomiche: la loro forma è più vicina alla forma degli orbitali
atomici di partenza. \\
Per descrivere ora il legame che intercorre tra un orbitale s ed uno p, bisogna
ricordare che $\psi$ descrive l'orbitale e che per formare il legame occorre che la
sovrapposizione tra i due orbitali avvenga fra zone in cui le funzioni $\psi$ abbiano
lo stesso segno: la sovrapposizione diventa più efficace quanto maggiore è il valore
assoluto della somma delle funzioni $\psi$. Nei legami con orbitali di tipo $p$, i legami
possono essere sia di tipo $\sigma$ se i loro assi sono collineari, ma anche di tipo $\pi$
se i piani nodali coincidono. A differenza di un legame $\sigma$, nel legame $\pi$ 
l'ampiezza della zona di sovrapposizione varia ruotando un atomo rispetto all'altro:
essa è massima quando i piani nodali coincidono e zero quando sono ortogonali.
\begin{wrapfigure}{l}{0.65\textwidth} 
    \label{Fig 3.1}
    \centering
    \caption{Forma schematica (\textbf{a}) di un orbitale molecolare
    $\sigma(s-s)$, (\textbf{b}) di un orbitale $\sigma(s-p_z)$ e (\textbf{c}) di un $\pi(p_x-p_x)$}
    \includegraphics[width=0.65\textwidth]{orbitali-a-confronto.png}
\end{wrapfigure}
Il primo legame è sempre di tipo $\sigma$, infatti si può pensare che gli atomi
si orientino e si dispongano in modo tale da far si che si ottenga la migliore 
sovrapposizione. Ne deriva che il legame $\pi$ sia molto meno stabile e che blocchi la
rotazione poiché i piani nodali devono coincidere e quindi gli assi degli orbitali sono paralleli.
Formato un legame $\sigma$, si possono formare solo altri due legami $\pi$, poiché una volta
formati i primi orbitali $\sigma$ e $\sigma^*$, questi non danno più contributo successivo
alla densità elettronica e quindi ulteriori elettroni non possono occupare lo spazio
dell'orbitale formato dal legame $\sigma$. L'unica scelta rimasta è poter fare altri due
legami $\pi$. Si noti inoltre che l'orbitale $\sigma^*$ ha energia minore di un orbitale
successivo $\pi$ e quindi viene riempito per primo. \\
Dal momento che la sovrapposizione degli orbitali $p$ nel legame $\pi$, risulta che la
separazione energetica degli orbitali $\pi-\pi^*$ è minore della separazione energetica
degli orbitali $\sigma_{pz}-\sigma^*_{pz}$. Quanto minore è la separazione energetica tra
gli orbitali $2s$ e $2p$, tanto maggiore è la separazione tra i diversi orbitali $\sigma$
per cui l'orbitale di legame $\sigma_{pz}$ è ad energia maggiore rispetto a quelli $\pi$.
Questo spiegherebbe come sia possibile l'esistenza di molecole come \ce{N2} in
quanto il suo ordine di legame raggiunge 3 utilizzando 4 orbitali di legame ed un di antilegame
(utilizza $\sigma_2$ insieme a tutti gli orbitali di legame $\pi$ mentre per l'antilegame usa
solo l'orbitale $\sigma_1^*$).

\section{Il formalismo di Lewis e la regola dell'ottetto}
A questo punto è chiaro che il modello dell'orbitale molecolare giustifichi come
gli atomi mettano in compartecipazione tanti elettroni quanti ne servono per
raggiungere l'ottetto (la configurazione del gas nobile più vicino). \\
G.N. Lewis propose un metodo per la rappresentazione grafica di questo modello
attraverso l'utilizzo di trattini per rappresentare o i legami covalenti o
i doppietti elettronici non di legame:
\begin{center}
    \chemfig{\charge{90=\|,180=\|,270=\|}{F}-\charge{0=\|,90=\|,-90=\|}{F}}
\end{center}
Inoltre, le coppie di non legame tenderanno a distanziarsi tra loro:
\[
    Sbagliato:\chemfig{\charge{90 = \|, -90 = \|}{O}=\charge{90 = \|, -90 = \|}{O}} \qquad
    Giusto: \charge{135=\|,225=\|}{O}=\charge{45 = \|, -45=\|}{O}
\]
Questa formulazione non è in grado tuttavia di descrivere le proprietà paramagnetiche
di certe sostanze come l'ossigeno, non fa distinzione tra legami $\sigma$ e $\pi$
e non ci dice da quali orbitali derivano i legami.

\section{Geometria e legame nelle molecole.}
\subsection{La regola delle repulsioni delle coppie di valenza}
In una qualsiasi molecola \ce{AB2}, la sua configurazione può variare notevolmente ed essere come:
\chemfig{B-A-B} oppure \chemfig{B-[:42.75]A-[:-42.75]B}, etc..
Le disposizioni delle molecole sono le seguenti:
\begin{center}
    \begin{tabular}{| m{3cm} | m{2cm} | m{7cm} |}
        \hline
        Nome & Angolo & Geometria delle molecole ed esempio \\
        \hline
        Planare & 180° & \chemfig{\charge{135 = \|, 225 = \|}{O}-C-\charge{45 = \|, -45 = \|}{O}} \\
        Trigonale planare & 120° & \chemfig{H-[:30]B(-[:90]H)-[:-30]H}  \\
        Trigonale angolare & 120° & \chemfig{O-[:30]\charge{90 = \|}{S}-[:-30]O} \\
        Tetraedrica & 109,5° & \chemfig{(H>:[:45])C(-[2]H)(<[:170]H)(-[:-20]H)} \\
        Tetraedrica piramidale & 107,3° & \chemfig{(H>:[:65])\charge{90 = \|}{N}(<[:195]H)(-[:-15]H)} \\
        Tetraedrica angolare &  104,5° & \chemfig{H-[:+42.75]\charge{45 = \|, 135 = \|}{O}-[:-42.75]H} \\
        Trigonale bipiramidale & 120° e 90° & \chemfig{(Cl>:[:30])P(-[2]Cl)(<[:165]Cl)(-[0]Cl)(<|[-2]Cl)}\\
        Ottaedrica & 90° & \chemfig{(F>:[:30])S(-[2]F)(<[:150]F)(-[-2]F)(<:[:-30]F)(<:[:30]F)}\\
        \hline
    \end{tabular}
\end{center}

\subsection{GLi orbitali ibridi}
\begin{wrapfigure}{r}{0.45\textwidth}
    \label{Fig. 3.2}
    \centering
    \caption{Ibridazione di un orbitale s e tre orbitali p che dà luogo a quattro
    orbitali sp. Sotto le sezioni di superficie di $\psi^2$}
    \includegraphics[width=0.45\textwidth]{Orbitali-sp.png}
\end{wrapfigure}
Per rimanere collegati al concetto di orbitale molecolare, avendo già visto che due o
più orbitali sia atomici che molecolari, possono combinarsi tra loro per dare lo stesso
numero di orbitali diversi per forma ed energia (pur mantenendo l'energia totale del sistema).
Combinando queste funzioni d'onda si ottengono dunque diverse combinazioni di
orbitali che minimizzano la repulsione elettrostatica: si parla di \textbf{orbitali ibridi}.
Combinando per esempio orbitali $s$ e $p$ si ottengono nuove funzioni d'onda
chiamati orbitali $s-p$. Questi orbitali sono orientati
tra loro a 120° l'uno dall'altro e, a causa della formazione di un nuovo orbitale,
la loro energia è minore di un orbitale $p$. \\
\vspace{2cm}
\label{Tabella 3.2}
\begin{center}
    \begin{tabular}{| m{1.8cm} | m{4.4cm} | m{5.5cm} | m{2.5cm} |}
    \hline
    Ibridazione & Orbitali atomici che formano orbitale ibrido & disposizione spaziale 
    degli assi degli orbitali ibridi & Esempi di molecole \\
    \hline
    \[sp\] & \[s + p_z\] & \chemfig{-} lineare & \ce{CO2}, \ce{XeF2}  \\
    \[sp^2\] & \[s + p_x + p_z\] & \chemfig{-[:30](-[:-30])-[2]} trigonale planare 
    & \ce{BF3}, \ce{CO3^{2-}}, \ce{SO2} \\
    \[sp^3\] & \[s + p_x + p_y + p_z\] & \chemfig{-[:30](-[:-30])(-[:-60])-[2]}
    Tetraedrica & \ce{CH4}, \ce{NH4^{+}}, \ce{NH3}, \ce{H2O}. \\
    \[dsp^2\] & \[s + d_{x^2-y^2} + p_x + p_y\] & \chemfig{-[:15](-[-1])(-[:15])(-[:135])}
    Quadrata planare & --- \\
    \[dsp^3\] & \[\left\{\begin{array}{c}
        s + d_{x^2 - y^2} + p_x + p_y + p_z \\
        \space \\
        \space \\
        \space \\
        \space \\
        \space \\
        s + d_{z^2} + p_x + p_y + p_z \\
    \end{array}\right.\] &
    \[
    \begin{array}{c}
        \chemfig{-[:15](-[-1])(-[:15])(-[2])(-[3])} \text{piramidale quadrata} \\
        \chemfig{-[:60](-[-2])(-[0])(-[2])(-[3])} \text{trigonale bipiramidale}
    \end{array}
    \] & \ce{PCl5} \\
    \[d^2sp^3\] & \[s + d_{x^2 - y^2} + d_{z^2} + p_x + p_y + p_z\] &
    \chemfig{-[:15](-[-2])(-[-1])(-[:15])(-[2])(-[3])} Ottaedrica &
    \ce{IF5}, \ce{SF6} \\
    \hline
    \end{tabular}
\end{center}

\subsection{Geometria molecolare di \ce{CH4}, \ce{NH3} e \ce{H2O}}
Sulla base della teoria VSEPR, queste sono le disposizioni spaziali delle molecole:
\[
    \chemfig{C(<|[:-30]H)(-[2]H)(<:[:210]H)(<[:255]H)} \qquad
    \chemfig{\charge{90 = \|}{N}(<|[:-30]H)(<:[:210]H)(<[:255]H)} \qquad
    \chemfig{\charge{90 = \|, -30 = \|}{O}(<:[:180]H)(<[5]H)}
\]
La molecola di \ce{CH4} ha una struttura tetraedrica semplice poiché nessun doppietto
è escluso dai legami e dunque le molecole di idrogeno sono equidistanti. Nella molecola
di \ce{NH3} invece essendoci solo tre legami ed un doppietto, i tre idrogeni
si disporranno in modo tale da formare la base del tetraedro mentre lo spazio
che nel \ce{CH4} sarebbe destinato all'ultimo idrogeno è occupato dal doppietto
non di legame: questo fa si che l'angolo H-N-H sia minore a 107°.
Nella molecola d'acqua invece ci sono due doppietti di non legame,
e quindi vale lo stesso ragionamento dell'\ce{NH3}, tuttavia essendoci due doppietti
l'angolo H-O-H sarà più piccolo a solo 105°.

\subsection{La regola dell'ottetto ed i suoi limiti}
Generalmente un atomo che impiega solo orbitali $s$ e $p$ tende a formare un massimo
di quattro legami e ad assumere una configurazione esterna con 8 elettroni di valenza.
Se c'è un numero di elettroni dispari, allora un elettrone occupa da solo un orbitale
Nella molecola \ce{NO} anche se non otteziale, comunque si hanno gli orbitali tutti
impiegati o nei legami o da elettroni di non legame. \\
Negli atomi come il fosforo, seppur dello stesso gruppo dell'azoto, ha anche gli elettroni
del gruppo $d$, che gli consentono di fare più legami (l'energia richiesta per gli
orbitali d è compensata da un maggior numero di legami). 
In generale tutte le volte che sono coinvolti gli orbitali $d$ si ha l'\textbf{espansione 
dell'ottetto}: lo zolfo (analogo dell'ossigeno) può fare fino a sei legami grazie agli
orbitali $d$. \\
PEr Z molto grande invece non si usano più orbitali dello stesso livello energetico in quanto
la differenza di energia tra gli orbitali va via via riducendosi.
I gas nobili invece dà luogo a composti promuovendo gli orbitali $s$ e $p$ completi.

\section{Numero di ossidazione e polarità dei legami}
\subsection{Polarità dei legami}
Nei legami tra molecole con forte differenza di elettronegatività, accade che 
gli elettroni siano attratti più verso l'atomo più elettronegativo e quindi
la densità elettronica aumenta nei pressi di quest'ultimo. Il legame che ne deriva è
il legame covalente polare (omopolare quando è tra atomi uguali). Quando il 
$\Delta$elettronegatività è molto alto allora si parla di \textbf{legame ionico}
(Cap. 4).

\subsection{Numero di ossidazione}
Il \textbf{numero di ossidazione} è la carica (positiva o negativa) di un atomo
in un legame o composto quando tutti gli atomi di legame sono assegnati all'atomo
più elettronegativo determinata a partire dal numero di elettroni in più o in meno
rispetto all'atomo neutro. \\
Il numero di ossidazione di un composto non dipende dal formalismo con cui è scritta
la formula (nel caso di specie contenenti due atomo dello stesso elemento legato
tra loro il numero di ossidazione dipende anche dal numero di legami covalenti
tra i due atomi) e non dipende dal numero di legami covalenti formati da un dato atomo,
inoltre la somma di tutti i numeri di ossidazione in un composto è sempre pari alla
carica di quel composto.\\
\[
    \begin{array}{c}
            \chemfig{H-\charge{0 = \|, 90 = \|, -90 = \|}{Cl}} \\
             \text{H ha numero di ossidazione +1}
    \end{array}
    \begin{array}{c}
        \chemfig{H-[:28]\charge{45 = \|, 135 = \|}{O}-[:-28]H} \\
        \text{O ha numero di ossidazione -2.}
    \end{array}
\]


\section{La risonanza}
\subsection{Formule limite}
Accade spesso che alcune formule di struttura possano essere scritte in più modi:
\[
\left[ \ \chemfig{\charge{120 = \|, 210 = \|, -60 = \|}{O}-[:30]\charge{90 = \|}{N}=[:-30]\charge{15 = \|, -75 = \|}{O}} \ \right]^{-}
\quad \longleftrightarrow \quad \left[
    \ \chemfig{\charge{120 = \|, 210 = \|, -60 = \|}{O}=[:30]\charge{90 = \|}{N}
    -[:-30]\charge{15 = \|, -75 = \|}{O}} \
\right]^{-}
\]
Le due formule sono identiche ma dal momento che l'ossigeno che non ha il doppio legame deve
completare l'ottetto, la copia di legame $\pi$ è delocalizzata sui tre atomi della molecola
e "passa" da un atomo all'altro degli ossigeni. notazione usata sopra per \ce{NO2} è
definita come \textbf{formula limite}  e la freccia indica cambiamento di distribuzione
elettronica ma non di geometria. Si dice anche che è un \textbf{ibrido di risonanza}.
Entrambe le formule limite contribuiscono alla definizione della formula reale in quanto 
ogni formula limite ci dice quali legami sono fatti ed hanno la stessa geometria.
La risonanza tuttavia avviene anche tra numeri diversi di legame:
\[
    \chemfig{\charge{180= \|}{N}~N-\charge{0=\|, -90 = \|, 90 = \|}{O}} \quad
    \longleftrightarrow \quad 
    \chemfig{\charge{135= \|, 225 = \|}{N}=N=\charge{-45 = \|,45 = \|}{O}}
\]
Esse non sono energeticamente equivalenti poiché i legami $\pi$ di \chemfig{N-N} non 
ha la stessa energia di quello \chemfig{N-O}, tuttavia sono formule limite perché
differiscono nella distribuzione degli elettroni.
Se le due formule fossero equivalenti allora le frazioni del legame $\pi$ di risonanza 
sarebbero rispettivamente $1/2$ ed $1/2$.

\subsection{Contributo relativo delle formule limite alla descrizione della struttura molecolare}
Se si misurasse l'energia di una formula limite si troverebbe che essa è maggiore della
della molecola reale (e quindi meno stabile): quindi la formula con energia minore tra
quelle limite è quella che fornisce una descrizione più reale della molecola effettiva.
Il concetto di \textbf{carica formale} è utile per determinare le formule di risonanza
e la loro energia: questa è la differenza tra il numero di elettroni dell'atomo libero
e quello attribuitegli formalmente in una formula di struttura. Nella seguente struttura
le cariche formali sono rispettivamente -1, +1 e 0.
\[
\chemfig{\charge{135= \|, 225 = \|}{N}=N=\charge{-45 = \|,45 = \|}{O}}
\]
La \textbf{separazione di carica} è l'esistenza di cariche formali di segno opposto
nella stessa formula; se una formula ha una separazione di carica minore allora ha
anche energia minore. Se in due formule limite entrambe hanno la stessa separazione di
carica allora è più stabile (e quindi più probabile) quella con gli elettroni sull'atomo
più elettronegativo.

\subsection{Formule limite con diverso numero di legami}
Nella molecola \ce{BF3}, generalmente il boro fa tre legami $\sigma$, uno con ogni \ce{F},
tuttavia, esso ha ancora un orbitale vuoto e quindi può formare un legame $\pi$ con 
uno dei fluori. Il legame $\pi$ \chemfig{B-F} non è molto forte poiché c'è grande
$\Delta$elettronegatività e quindi le formule limite per la molecola sono tre con un doppio
legame con uno dei fluori e una con solo legami sigma. Ogni legame \chemfig{B-F} è formato
da un legame $\sigma$ e solo da $<10\%$ di legame $\pi$. \\
Con atomi del terzo periodo, avendo a disposizione anche orbitali $d$, si possono ottenere
molte più formule di risonanza, anche se per ragioni di schematicità si possono rappresentare
con una sola formula, trascurando gli orbitali $d$, la formula otteziale di alcune molecole
e che mostra che i legami devono essere equivalenti.

\section{Sommario regole per la scrittura delle formule di struttura}
Data una formula molecolare e conoscendo l'atomo centrale (in genere l'atomo meno
elettronegativo tranne l'idrogeno), si procede per la scrittura delle formule
nella seguente maniera:
\begin{enumerate}
    \item Legare l'atomo centrale tutti gli atomi con un legame semplice;
    \item Sommare gli elettroni di valenza degli atomi, addizionate/sottratte al numero di
    cariche negative o positive;
    \item Disporre tutti i possibili elettroni sugli atomi periferici e disporre 
    anche gli elettroni rimanenti dopo aver sistemato gli elettroni di legame
    e le coppie solitarie degli atomi periferici sull'atomo centrale;
    \item determinare le posizioni degli atomi intorno all'atomo centrale attraverso
    la repulsione delle coppie elettroniche;
    \item distribuire gli elettroni in modo da scrivere formule o formule limite accettabili.
\end{enumerate}

\section{Energia di legame}
Le caratteristiche principali di un legame \chemfig{A-B} sono la distanza di legame,
e l'energia di legame. Per convenzione si definisce l'energia di legame l'\textbf{
energia necessaria per rompere il legame} e quindi è per convenzione positiva,
mentre è negativa l'energia immagazzinata nel sistema \chemfig{A-B}. Sperimentalmente
l'energia di legame è misurata come il $\Delta$\textbf{entalpia} nella trasformazione:
\[
    \chemfig{A-B_{(g)}} \ \to \ \ce{A_{(g)}}+ \ce{B_{(g)}}
\] 
Il numero di legami si chiama \textbf{ordine di legame} e può essere un numero intero
oppure una frazione nel caso di formule limite.

\section{Orbitale molecolare per molecole poliatomiche}
L'orbitale molecolare per le molecole biatomiche, anche gli orbitali non di legame
partecipano al legame in quanto l'energia degli orbitali di antilegame spesso hanno energia
maggiore che del terzo orbitale $p_z$ come nel caso di \ce{H2O}. Se $p_z$ non fosse
occupato da due elettroni, allora l'angolo di legame sarebbe di soli 90°, poiché l'energia
degli orbitali di antilegame aumenta a causa delle sovrapposizione degli orbitali 
$2s$ dell'ossigeno e $1s$ dell'idrogeno. Aumentando dunque l'angolo della molecola,
si osserva che l'energia del sistema diminuisce maggiormente e in questo modello 
l'orbitale $2p_z$ rimane invariato, mentre nel modello degli orbitali
ibridi si sovrappone a $s$ e $p$. Secondo quindi il modello OMC, ossia degli orbitali
molecolari canonici, l'orbitale $2p_z$ rimane invariato, mentre nel modello degli orbitali
ibridi anche questo si sovrappone agli altri orbitali.\\
Nel caso del \ce{CH4} è invece molto più facile poiché secondo la teoria
dell'OMC, i quattro orbitali di legame hanno uno energia più bassa degli altri tre,
mentre secondo la teoria del VSEPR si hanno quattro orbitali ibridi di pari energia
e coincidono con la descrizione degli orbitali ibridi $sp^3$.


\chapter{Il legame metallico}
\section{I metalli}
I metalli sono gli elementi sulla tavola periodica che possiedono pochi elettroni
esterni ed hanno energie di ionizzazione molto basse. Sono metalli non solo gli
elementi del primo e secondo gruppo (tranne l'idrogeno), ma anche alcuni che si trovano
in fondo agli ultimi gruppi della tavola periodica e anche quelli di 
transizione del blocco $d$. \\
Nella struttura dei metalli si possono trovare fino a 12 altri atomi intorno ogni
atomo metallico e generalmente possiedono molti meno elettroni di valenza che
orbitali disponibili. Questo rende impossibile il modello del legame covalente
a coppia di elettroni localizzati tra due atomi, poiché ci sarebbero molte formule di
risonanza valide (nel caso dei metalli alcalini fino a 7).\\
Il modello di legame tra metalli può quindi essere interpretato come un continuo
scambio di elettroni tra gli ioni metallici nel reticolo che, se sottoposti a
campo elettrico, si muovono dando vita a conduzione elettrica.

\section{Il modello a bande}
Il legame dei metalli è dunque un modello a bande, considerata un'estensione dell'orbitale
molecolare.
\begin{wrapfigure}{r}{0.40\textwidth} \label{Fig 4.1}
    \centering
    \caption{Schema di orbitale molecolare per la molecola \ce{Li2}}
    \includegraphics[width=0.40\textwidth]{Schema-Litio.png}
\end{wrapfigure}
Prendendo come esempio il Litio (con soli tre elettroni), nella figura \ref{Fig 4.1}
è raffigurato il modello dell'orbitale molecolare (non considerando la teoria degli
orbitali ibridi), gli elettroni stanno dunque nell'orbitale $\sigma_s$, ad energia più
alta si trovano gli orbitali vuoti $\sigma_s^*, \sigma_p, \pi_p, \pi_p^*$ ed infine
$\sigma_p^*$, continuando ad aggiungere altri atomi di litio, gli orbitali
si infittiscono sempre di più fino a disporsi secondo delle bande di energia,
se la banda $\sigma_s$ si tocca con un la banda degli orbitali corrispondenti
$\sigma_s^*$, allora si ha un'unica banda $2s$ \textbf{semiriempita}, e quindi un
elettrone può passare da un orbitale riempito ad uno vuoto e l'energia
necessaria sarà quella cinetica.

\section{Il modello della conduzione elettrica in un solido}
Nel modello a bande si ha la risposta alla conduttività/isolamento di un materiale.
Se si applica un certo campo elettrico ad un materiale, gli elettroni tenderanno
a muoversi e due elettroni che vanno in verso opposto uno risulterà accelerato
dal campo, l'altro rallentato. Tuttavia se il campo non riesce a creare sufficiente 
$\Delta$Energia, allora non si avrà alcun salto quantico e gli elettroni non
saranno accelerati in quanto il sistema può assorbire solo determinate energie
quantizzate. \\
Se si applica un campo elettrico ad un materiale con bande piene e vuote a contatto,
allora gli elettroni possono saltare da un livello energetico all'altro in quanto
le bande, dal momento che sono fitte, permettono passaggio di elettroni con salto
quantico di energia \textbf{uguale} a quella impartita dal campo. Se invece una banda
è completamente piena, allora il materiale è isolante in quanto è richiesta
più energia per passare alla banda sopra. \\
Un conduttore è un materiale che presenta \textbf{bande parzialmente occupate} e gli
isolanti bande complete di elettroni separate da quelle vuote da un grande salto di energia. \\
Se le due bande sono separate ma il salto energetico è piccolo, allora si hanno le 
caratteristiche di un \textbf{semiconduttore}. La banda inferiore è chiamata
\emph{banda di valenza}, mentre quella superiore \emph{banda di conduzione}.

\section{Configurazione elettronica e proprietà elettriche}
Nel gruppo 1, i metalli alcalini utilizzano la banda semiriempita $ns^1$, mentre
gli alcalino-terrosi del gruppo 2 utilizzano gli orbitali $p$ (vuoti) che si sovrappongono
a quelli $s$: gli elettroni di valenza occuperanno gli orbitali di minore energia, e in
questo caso si ha guadagno netto di energia e si spiega il legame solido degli elementi
oltre che alla conducibilità elettrica; si applica lo stesso ragionamento per i 
metalli del gruppo 13 ($ns^2np^1$). \\
Nel gruppo 14 invece il carbonio esiste in due forme allotropiche (grafite e diamante),
nel diamante si formano tanti legami quanti elettroni di valenza (ossia 4), quindi si
hanno $4N$ orbitali di legame e $4N$ orbitali di antilegame, il che fa sì che ci
sia un grande salto energetico tra le due bande di legame e di antilegame, il
carbonio nella sua forma a diamante è perciò un isolante elettrico. 
Negli altri elementi del gruppo 14, via via che si scende nel gruppo diminuisce
il salto energetico tra le bande, facendo sì che lo \ce{Sn} sia conduttore a causa
della maggiore \textbf{delocalizzazione} dei legami, mentre Silicio e Germanio
sono semiconduttori, la cui conducibilità aumenta con la temperatura. \\
La grafite invece è un buon conduttore elettrico poiché esistono bande di tipo
$\pi$ e $\pi^*$ che sono molto vicine le une alle altre e comprese tra quelle
$\sigma$ e $\sigma^*$. \\
Aggiungendo invece atomi con più o meno elettroni ad un semiconduttore (nel caso
del Silicio si usa \ce{P, As} oppure il \ce{B}), la conduzione può essere aumentata
(\textbf{drogaggio}). Quando il Silicio è drogato con atomi con più elettroni, allora
si ha che gli elettroni in più si localizzano sulla banda di conduzione, altrimenti
sulla banda di valenza.

\section{Conduzione termica}
Quando si fornisce energia ad un metallo conduttore la sua conduzione diminuisce
poiché la velocità superiore degli elettroni aumenta il moto casuale e quindi non
passano dalle bande di valenza a quelle di conduzione. \\
Una proprietà interessante dei metalli e la loro alta conducibilità termica
dovuta alla delocalizzazione degli elettroni su tutto il solido che redistribuiscono
l'energia assorbita ovunque. Se un materiale non riesce a redistribuire questa energia
con facilità, allora si parla di \textbf{isolante termico}.

\section{Energia del legame metallico}
L'energia del legame metallico è definita nello stesso modo, qualunque sia 
il tipo di legame nei metalli. L'energia del legame varia notevolmente a seconda
del numero di elettroni coinvolti nel legame metallico, dal momento che gli elettroni
sono delocalizzati su tutto il solido, questi schermano le cariche degli elettroni
interni non di valenza (\textbf{\emph{core} ionico} del metallo). Questo mare di
elettroni è quindi il collante che tiene insieme il metallo.
Le temperature di fusione ed ebollizione aumentano lungo il periodo fino ai gruppi
5-6-7, successivamente diminuiscono poiché ci sono più elettroni che orbitali vuoti
e gli elettroni tendono dunque a localizzarsi vicino al nucleo con spin antiparallelo.

\section{L'idrogeno}
Sebbene si sia detto come l'idrogeno non sia un metallo, esso in condizioni molto estreme,
ossia con altissime pressioni e temperatura, dà luogo non più a molecole di \ce{H2} ma
a molecole di \ce{H}, le quali sono tenute insieme da un legame metallico. Questo
è stato scoperto da una sonda della navicella Galileo nel '95 che ha confermato
la presenza di idrogeno metallico (liquido) nel mantello di Giove.


\chapter{Il legame ionico}
\section{Il modello del legame ionico}
Il legame ionico si associa ad una tipologia di composti che prendono il nome di
\textbf{composti ionici} i quali a temperatura ambiente sono solidi cristallini 
elettricamente neutri. La condizione di minima energia tra due ioni di 
carica opposta è la minima energia data dalla legge di Coulomb:
\begin{align}
    E_{pot} = k_C \frac{Q_A Q_B}{r}
\end{align}
Dove $k_C$:
\begin{align}
    k_C = \frac{1}{4\pi\epsilon_0} = 8.9875\cdot 10^{9} JmC^{-2}  
\end{align}
Dove $\epsilon_0$ è la \textbf{permittività nel vuoto}:
\begin{align}
    \epsilon_0 = 8.8542 \cdot 10^{-12}J^{-1}m^{-1}C^{2}    
\end{align}
Se $Q_AQ_B$ è un prodotto negativo allora la potenziale è negativa, il che
vuol dire che si attraggono, altrimenti si respingono. Per decidere ora se la
coppia formata da due particelle cariche $A^+$ e $B^{-} $ sua stabile energeticamente
si valutano i contributi delle singole energie: l'energia di Ionizzazione di
A è positiva e quindi è sfavorevole, l'affinità di B è positiva e quindi favorevole
e l'energia di interazione elettrostatica è grande e negativa (il che va bene se questa
più l'affinità bilanciano l'energia di ionizzazione di A). La maggior parte delle
coppie ioniche è molto più stabile in legame che da sole (come \ce{NaCl}), in questi
casi un livello energetico minore si ha con reticoli cristallini tridimensionali
in cui ogni particella $A^{+}$ si circonda di particelle $B^{-}$ e viceversa
rendendo solido il composto a temperatura ambiente. In ogni caso il
rapporto tra cationi ed anioni all'interno di un legame ionico è quello dell'uguaglianza
tra cariche positive e negative. \\
Inoltre il legame ionico non è direzionale in quanto l'attrazione si ha in tutte le direzioni. \\
Assumendo ora che gli ioni siano sfere perfette , l'energia complessiva di attrazione
e repulsione dipende dalla carica degli ioni e dalla distanza tra i loro baricentri:
\begin{align}
    E_{pot} = k_C N_A M \frac{Q_BQ_A}{r}
\end{align}  
Dove M è la costante di \emph{Madelung} che è sempre maggiore di uno: il che ci porta
a dire che è più stabile un sistema con gli ioni legati tra loro piuttosto che senza interazione.

\section{Energia reticolare e costante di Madelung}
Nel caso di \ce{Cl^{-}} e \ce{Na^{+}} il reticolo cristallino cubico fa si che si
alternino strati del cubo con contributi negativi a contributi positivi
ognuno sempre minore in valore assoluto di quello prima, andando a formare una serie
convergente ad un certo valore di M ($M = 1.7475$ per \ce{NaCl}) quando il 
numero di reticoli sono $\to \infty $. Così moltiplicando M per $-k_Ce^{2}/r$ si ottiene
l'energia di interazioni di uno ione \ce{Cl^{-}} con gli altri ioni e
quindi si ottiene l'\textbf{energia potenziale di interazione}di una mole \ce{NaCl}:
\begin{gather*}
    E = -k_CMN_A\frac{e^{2} }{r}
\end{gather*}  
Il valore della costante di Madelung dipende sopratutto dalla struttura cristallina
della sostanza ma non dalla sua natura chimica ed è sempre $> 1$.

\section{Ioni che costituiscono le sostanze ioniche}
\begin{tabular}{| m{1cm} | m{2.5cm} | m{5cm} |}
    \hline
    Gruppo & Configurazione & Ioni \\
    \hline
    1 & $ns^{1}$ & \ce{M^{+}} (tranne H) \\
    \hline
    2 & $ns^{2}$ & \ce{M^{2+}}  \\
    \hline
    13 & $ns^{2}np^{1}$ & \ce{M^{3+}} e poliatomici negativi \\
    \hline
    14 & $ns^{2}np^{2}$ & poliatomici negativi e monoatomici positivi    \\
    \hline
    15 & $ns^{2}np^{3}$ & poliatomici negativi \\
    \hline
    16 & $ns^{2}np^{4}$ & \ce{X^{2-}} e poliatomici negativi \\
    \hline
    17 & $ns^{2}np^{5}$ & \ce{X^{-}} e ioni poliatomici negativi  \\
    \hline    
\end{tabular}

\section{Geometria locale dei composti ionici}
Ogni ione è circondato dal maggior numero di ioni di carica opposta
ed il limite degli ioni che possono circondare un altro ione di segno opposto
è dato dal rapporto tra i raggi dello ione positivo fratto quello negativo.
\begin{tabular}{m{3cm} m{3cm}}
    \hline
    Numero di coordinazione & $\frac{r^{+} }{r^{-} }$ \\
    \hline
    3 & 0.155 - 0.225 \\
    4 & 0.255 - 0.414 \\
    6 & 0.414 - 0.732 \\
    8 & 0.732 - 1
\end{tabular}
Il \textbf{numero di coordinazione} è il numero di ioni di segno opposto
attorno allo ione più piccolo del composto ionico. La geometria di coordinazione
invece tende a rendere distanti gli ioni di segno uguale e per questo
solo a determinati raggi si può avere un certo numero di coordinazione.

\section{Validità del modello ionico}
Questo modello è valido se e solo se l'energia trovata 
corrisponde all'energia necessaria per la dissociazione del legame ionico
misurata sperimentalmente: nel caso di \ce{NaCl} queste sono rispettivamente
867 e 768 kJ $mol^{-1}$, il che è abbastanza preciso per un modello che non tiene in
considerazione la repulsione elettronica: se si volesse un modello che tenga conto di questo,
allora dovremmo correggere la formula come segue:
\begin{align}
    E_{rec} = -E_{pot} = k_CN_A\frac{Me^{2} }{r}\left( 1 - \frac{1}{n} \right).
\end{align}
Dove n è esattamente un numero che varia da 6 a 10 a seconda del tipo di ioni.
La Validità del modello può anche dipendere dal carattere covalente di un composto:
alcuni composti con un carattere molto covalente si ha una sovrapposizione degli 
orbitali che porta ad imprecisioni nel calcolo teorico dell'energia di dissociazione.

\section{Carattere ionico, covalente e numeri di ossidazione degli atomi nei composti ionici}
Il carattere covalente in un legame ionico dipende dalla direzionalità del legame:
se un legame ionico non ha molte molecole di coordinazione allora ha un carattere tendente
al covalente, altrimenti è puramente ionico. Questo accade perché in un legame covalente si ha una
forte penalizzazione per la violazione degli angoli di legame è molto più grande rispetto
allo ionico in quanto in quest'ultimo non ci sono coppie elettroniche che si respingono e
gli ioni si impacchettano in modo più efficiente: è per questo che
nell'\ce{NaCl} ogni ione è circondato da altri 6 ma il carbonio forma al
massimo 4 legami. \\
Anche se è corretto parlare di aumento di carattere ionico quando aumenta il $\Delta$ elettronegatività
,tuttavia non si raggiungerà mai il 100\% di carattere ionico in quanto si formerebbe una 
coppia ionica non stabile. \\
Per quanto riguarda il numero di ossidazione invece si ha che nei composti ionici
corrisponde alla carica dei singoli ioni, mentre nei poliatomici vale sempre
la regola di assegnare gli elettroni a quello più elettronegativo.

\chapter{Interazioni di Van der Waals ed il ponte ad idrogeno}
\section{Interazioni di Van der Waals}
\subsection{Modello per dispersione}
Le sostanze a stato elementare sono molecole discrete ed i legami al loro intero solo tali
da far raggiungere la configurazione del gas nobile più vicino a tutti gli atomi interessati dal legame. 
Come è possibile che quindi queste sostanze possano essere solide se non formano
altri legami? \\
Esistono quindi delle \textbf{forze di coesione intermolecolari} spiegate
attraverso \textbf{dipoli momentanei reciprocamente indotti}. In linea teorica in un atomo
gassoso non si dovrebbe avere la formazione di dipoli momentanei poiché il baricentro delle cariche
positive e negative dovrebbero coincidere col centro dell'atomo, tuttavia, muovendosi, gli elettroni,
tendono a modificare il campo elettrostatico generando \textbf{dipoli momentanei}, che,
essendo causali, la loro media è nulla nel tempo. \\
\begin{wrapfigure}{r}{0.4\textwidth}
    \centering
    \label{Fig 6.1}
    \caption{Il moto degli elettroni intorno al nucleo genera dipoli momentanei}
    \includegraphics[width=0.4\textwidth]{Dipoli.png}
\end{wrapfigure}
Prendendo in considerazione più atomi si osserva come questi dipoli momentanei
possano influenzare anche gli atomi vicini andando a far muovere in modo coordinato
gli elettroni di atomi adiacenti la cui media non sarà più zero poiché sono orientati
testa-coda con altri atomi creando delle vere e proprie forze momentanee di attrazione
chiamate \textbf{forze di attrazione} o di \textbf{London} con energia potenziale:
\begin{align}
    E_{AB} = -\frac{3}{2}\frac{I_AI_B}{2(I_A + I_B)} \frac{\alpha_A \alpha_B}{r^{6} }
\end{align}
Dove $I_A$ e $I_B$ sono le energia di ionizzazione delle molecole coinvolte e r è la distanza
fra i baricentri delle molecole e $\alpha$ è la loro \textbf{polarizzabilità} ossia la facilità
con cui si può deformare la distribuzione degli elettroni delle molecole in presenza
di un campo elettrico in modo tale che i baricentri delle cariche positive e negative non coincidano più. \\
La polarizzabilità è una proprietà periodica che cresce scendendo lungo un gruppo
e da destra verso sinistra. L'energia dovuta alla forza di London è positiva poiché
è l'energia dovuta per rompere l'interazione.

\subsection{I dipoli}
Un dipolo elettrico è costituito da due cariche elettriche di segno opposto a distanza $d$, e con
un vettore momento dipolo chiamato $\mu$ che va dal polo positivo a quello negativo il cui modulo vale
\begin{align}
    \mu = Qd
\end{align}
mentre l'energia potenziale di un dipolo è data da:
\begin{align}
    U = -\mu
\end{align}

\subsection{Molecole polari ed interazioni per orientazione ed induzione}
Quando i baricentri delle cariche positive e negative in una molecola non coincidono
allora tale molecola possiede un momento polare permanente  e la direzione, verso
e modulo dipendono dalla geometria della molecola. Questo può dipendere da molti fattori
come l'orientazione degli orbitali esterni e l'elettronegatività delle molecole. 
Le molecole polari si attraggono tra loro in modo da massimizzare l'interazione dipolo-dipolo
e quando sono libere di orientarsi allora la loro attrazione si media a:
\begin{align}
    E_{AB} = -\frac{2}{3k_BT}\frac{1}{k_C^{2}}\frac{\mu_A^{2} \mu_B^{2} }{r^{6} }
\end{align}
Questa interazione è anche chiamata \textbf{interazione per orientazione} perché le molecole
polari si orientano in un campo elettrico. \\
L'energia dovuta alla polarità di una molecola però è solo una componente
dell'energia totale nelle interazioni molecolari: infatti ogni dipolo induce un'altro 
dipolo con un energia di \textbf{interazione di Debye} molto piccola:
\begin{align}
    E_{AB} = - \frac{2}{k_C^{2} r^{6} } (\mu_A^{2} \alpha_B + \mu_B^{2} \alpha_A )
\end{align}

\section{Il legame ponte ad idrogeno}
\begin{wrapfigure}{r}{0.4\textwidth}
    \centering
    \label{Fig 6.2}
    \caption{Il legame ad idrogeno nella molecola d'acqua}
    \includegraphics[width=0.4\textwidth]{Legami-idrogeno.png}
\end{wrapfigure}
Alcune molecole polari mostrano forze di coesione molto maggiori di quelle
prevedibili in base alla semplice orientazione del dipolo o alla loro polarizzabilità
e son o caratterizzate dalla presenza di atomi elettronegativi e legami con atomi
di idrogeno. In queste molecole le interazioni deboli prendono il
nome di \textbf{ponte ad idrogeno} in quanto si ha interazione elettrostatica tra
l'idrogeno e gli atomi elettronegativi. Questo tipo di legame è più forte delle interazioni
deboli e per questo vi si riferisce come se fosse un debole legame chimico in quanto è
solo un ordine di grandezza più debole di quello covalente. Questo tipo
di legame spiegherebbe il comportamento anomalo dell'acqua che è meno densa
allo stato solido che allo stato liquido: questo perché allo stato solido si
devono rompere alcuni di quei legami.

\section{Stato di aggregazione di una sostanza molecolare ed energia del legame intermolecolare}
La temperatura standard di ebollizione di un liquido può essere espressa come la somma
delle forze di interazione intermolecolare allo stato condensato. Molecole con 
basse temperature di fusione, ebollizione hanno delle interazioni intermolecolari
molto deboli rispetto a quelle che hanno punti molto più alti. Inoltre 
un altro fattore importante è la polarizzabilità, la quale, aumentando, tende
a far aumentare la temperatura di fusione ed ebollizione, così come la grandezza
della molecola.

\chapter{Stati e proprietà della materia}
\section{Il modello e le proprietà dello stato solido}
\subsection{Il principio di massimo impacchettamento}
\begin{wrapfigure}{r}{0.4\textwidth}
    \centering
    \label{Fig 7.1}
    \caption{\textbf{a)} massimo impacchettamento di un piano di sfere,
    \textbf{b)} struttura non a massimo impacchettamento}
    \includegraphics[width=0.5\textwidth]{Massimo-impacchettamento.png}
\end{wrapfigure}
Lo \textbf{stato solido} o \textbf{stato cristallino} è uno stato della materia
in cui le molecole di una data sostanza presentano il massimo impacchettamento possibile
con una disposizione ordinata che si ripete periodicamente nello spazio. Nonostante
il massimo impacchettamento possibile sia dovuto al principio di minima energia, 
alcuni fattori come la direzionalità delle interazioni, la non sfericità delle
molecole e le dimensioni diverse di ogni particella 

\subsection{impacchettamento ideale di sfere di uguali dimensioni}
Le varie strutture di massimo impacchettamento (posto che le molecole costituenti siano
delle sfere perfette ) si ha con numero di coordinazione 12 a livello
tridimensionale. In generale, qualsiasi sia la struttura di massimo impacchettamento
data da sfere piene, la loro frazione di volume occupata sarà sempre di
$\frac{\pi}{3\sqrt{2}} \sim 0.7405$. Tra le strutture più comuni adottate dalle
sostanze, si hanno la \textbf{cubica a corpo centrato} e quella
\textbf{cubica semplice}.

\subsection{La forma delle cavità comprese fra le sfere nelle strutture a massimo impacchettamento}
\begin{wrapfigure}{r}{0.4\textwidth}
    \centering
    \label{Fig 7.2}
    \caption{Forma delle cavità: \textbf{b)} tetraedrica generata da \textbf{a)}
    e \textbf{d)} generata da \textbf{c)}.}
    \includegraphics[width=0.5\textwidth]{Cavita.png}
\end{wrapfigure}
All'interno delle strutture a massimo impacchettamento si hanno generalmente due
tipologie di cavità: una \textbf{tetraedrica} ed una \textbf{ottaedrica}. Nelle
strutture piane, la cavità che si forma è triangolare, e quando vi si appoggia
sopra un'altra struttura si formano delle cavità tetraedriche. Generalmente
si originano cavità tetraedriche quando si ha uno spazio all'interno di 4 sfere
e si originano cavità ottaedriche quando si ha uno spazio all'interno di 6 sfere.

\section{Strutture dei composti ionici}
Solitamente nei composti ionici quando si raggiunge il massimo impacchettamento
sono gli anioni che si avvicinano moltissimo tra di loro fino a toccarsi quasi
(posto che si stia modellizzando gli atomi come sfere perfette) mentre
i cationi tendono a disporsi all'interno delle cavità lasciate dagli
anioni. Nella struttura dell' \ce{NaCl}, gli ioni \ce{Cl^{-}} tendono
ad avvicinarsi così tanto da essere in contatto tra di loro ed acquisiscono
un numero di coordinazione 12 mentre gli ioni \ce{Na^{+}} è solo 6. 

\section{Strutture dei solidi molecolari e covalenti}
\subsection{Sostanze costituite da molecole discrete}
Quando le molecole hanno una forma sferica le forze di interazione sono adirezionali e 
la sostanza allo stato solido assume una conformazione ben precisa ed il baricentro
(nei gas nobili e negli alogeni ) tende ad essere al centro della sfera
nel reticolo a massimo impacchettamento poiché agiscono solo forze di London. \\
all'interno di sostanze con interazioni più forti come i ponti ad idrogeno, allora
la struttura non è mai quella di massimo impacchettamento come nel caso del ghiaccio:
avendoci 4 ponti ad idrogeno, allora 4 atomi di ossigeno sono circondati
da solo 4 atomi di idrogeno non rendendola compatta.

\subsection{Solidi con struttura covalente polimera}
I solidi polimeri sono quelle sostanze che presentano legami solo ed
esclusivamente covalenti; questo vuol dire che presentano un carattere che si
discosta molto dalle strutture di massimo impacchettamento: il legame covalente è un
legame fortemente direzionale e quindi queste molecole raggiungono la stabilità soltanto
quando raggiungono il numero di legami massimo possibile e non quando raggiungono
il il maggior numero possibile di altri atomi intorno. \\
\begin{wrapfigure}{r}{0.4\textwidth}
    \centering
    \label{Fig 7.3}
    \caption{\textbf{a)} disposizione tetraedrica dei legami nel diamante;
    \textbf{b)} reticolo cubico del diamante}
    \includegraphics[width=0.4\textwidth]{diamante.png}
\end{wrapfigure}
La struttura modello è quella del diamante: ogni atomo di carbonio forma 4 legami covalenti
con altri 4 atomi di carbonio, i quali, a loro volta, formano altri 4 legami
covalenti e così via: non si ha quindi una struttura ben definita a massimo impacchettamento
poiché ogni atomo di carbonio si lega solo con altri 4 atomi di carbonio anche se la
struttura è assimilabile a quella di un reticolo cubico.\\
Un'altra particolarità è proprio l'impossibilità di determinare in modo geometrico
come si distribuiscono spazialmente le sostanze covalenti: questo perché spesso sono
costituiti da strati di atomi tenuti insieme dalle forze di Van der Waals come
nel caso della grafite. Nella grafite ogni atomo di carbonio è legato con 3 altri 
atomi di carbonio formando un reticolo con numero di coordinazione 3; l'orbitale $\pi$ 
è invece condiviso con gli altri reticoli in modo da formare una struttura a strati 
con un sistema di elettroni delocalizzati. \\
Ne deriva che diamante e grafite sono sostanze \textbf{allotropiche} del carbonio.

\section{Cristalli reali e difetti reticolari}
\begin{wrapfigure}{r}{0.5\textwidth}
    \centering
    \label{Fig 7.4}
    \caption{\textbf{a)} difetti di Schottky; \textbf{b)} difetti di Frenkel}
    \includegraphics[width=0.4\textwidth]{Difetti-reticolari.png}
\end{wrapfigure}
Tutta la trattazione fino ad ora è stata effettuata secondo l'assunzione che i cristalli
fossero perfetti; tuttavia questo non è sempre il caso (non lo è mai) e quindi inevitabilmente
presenteranno delle impurezze, che possono essere ridotte mediante la ricristallizzazione.
Quando gli atomi vibrano e traslano (traslano poco a temp. ambiente) possono aprirsi delle cavità
in cui si insediano degli atomi estranei, i quali creano dei \textbf{difetti reticolari}. Esistono
imperfezioni di \textbf{Schottky} quando si parla di imperfezione del reticolo
ideale (ossia quando mancano ioni o molecole) o imperfezioni di \textbf{Frenkel} quando
un atomo o ione si insedia in una cavità che normalmente dovrebbe essere vuota. \\
Inoltre se il processo di cristallizzazione avviene troppo velocemente allora accade
che gli atomi non riescono a disporsi in modo ideale e quindi danno origine a
\textbf{dislocazioni}. Si ha anche un difetto ottico delle sostanze ioniche chiamato
\textbf{centri di colore}: ossia quando per qualche motivo uno ione negativo perde il proprio
elettrone (il quale rimane nella stessa posizione poiché è vincolato dal campo di
potenziale degli altri ioni) divenendo quindi neutro e traslando per via del moto termico. \\
Cosa c'è alla fine di un solido? Generalmente gli atomi si avvicinano tra di loro oppure
come nei metalli aumentano il numero di ossidazione e si legano a specie ossidanti come
l'ossigeno. 

\section{Alcune proprietà delle sostanze solide correlate al legame chimico}
\begin{enumerate}
    \item \textbf{Durezza}: della durezza(ossia la resistenza alla scalfitura)
    è responsabile l'energia dei legami covalenti e viene misurata secondo la \textbf{
    scala di Mohs} (il cui limite massimo è il diamante con 10, ed il limite inferiore
    è il talco con 1) nella quale ogni gradino rappresenta una durezza 10 volte maggiore
    dell'elemento prima sulla scala;
    \item \textbf{Fragilità}: della Fragilità (ossia la possibilità di frattura
    del cristallo per sollecitazioni meccaniche) è responsabile la direzionalità del legame:
    legami covalenti portano alla rottura del cristallo poiché molto direzionali mentre
    solidi tenuti insieme da interazioni deboli sono molto malleabili;
    \item \textbf{Malleabilità} e \textbf{Duttilità}: La Malleabilità dipende soprattutto dalla adirezionalità
    dei legami (quindi solo interazioni deboli e ionici) poiché prevede lo slittamento
    di lamine di cristallo sopra le altre.
\end{enumerate}
Quando un metallo è molto malleabile ma poco duro questo dipende dalla barriera di
potenziale che si oppone allo scorrimento di un piano di atomi rispetto ad un altro
e dalle proprietà atomiche dell'elemento.

\section{Modello e proprietà dello stato gassoso}
\subsection{Equazione di stato dei gas ideali}
Le particelle che costituiscono un sistema allo stato gassoso possiedono
energia cinetica maggiore dell'energia di interazione ed occupano tutto lo spazio
disponibile. La materia allo stato gassoso possiede 4 proprietà: due estensive
ossia mole e volume e due intensive ossia temperatura e pressione; dall'equazione
dei gas ideali è possibile ricavare da solo tre di esse la quarta:
\begin{align}
    PV = nRT
\end{align}
Il valore di R è:
\begin{align}
    R = 8.31446261815324 mol^{-1} K^{-1}  
\end{align}
Questa legge contiene anche le leggi di \emph{Boyle, Gay-Lussac e Avogadro}
che dicono rispettivamente che a temperatura costante un aumento della pressione
causa una diminuzione del volume; dipendenza del volume dalla temperatura a pressione
costante e viceversa; e la terza ci dice che a volumi uguali di gas ideali
alla stessa temperatura e pressione contengono la stessa quantità di sostanza.
Un esempio di validità è la seguente relazione:
\begin{gather*}
    \ce{CH4 + 2O2 -> CO2 + 2H2O}
\end{gather*}

\subsection{Postulati e risultati della teoria cinetica dei gas: il modello ideale}
Queste conclusioni possono permetterci di razionalizzare il comportamento di un gas
in modo ideale con poche modifiche: (anticipando la teoria cinetica dei gas) i gas
ideali hanno le seguenti proprietà:
\begin{enumerate}
    \item Le particelle di un gas ideale sono tutte uguali tra di loro e si muovono
    di moto rettilineo uniforme in direzioni diverse;
    \item il volume delle particelle è trascurabile;
    \item Non ci sono interazioni tra le particelle e con le pareti del contenitore
    e gli urti sono perfettamente elastici.
\end{enumerate}
\begin{figure}
    \centering
    \label{Fig 7.5}
    \caption{Energia cinetica media del gas ideale a varie temperature}
    \includegraphics[width=0.65\textwidth]{energia-cinetica-media.png}
\end{figure}

SI può trovare con queste leggi la distribuzione delle particelle dei gas chiamata
\emph{distribuzione di Maxwell-Boltzmann} nella quale il punto della curva più alto
contiene la frazione più grande di particelle con una data energia e la curva non è 
mai simmetrica ma tende a decrescere velocemente verso zero e a decrescere
lentamente verso energie molto più alte. Aumentando la temperatura aumenta l'energia
più probabile così come l'energia media. QUesta curva è descritta con
la seguente formula:
\begin{align}
    f(E) = 2(k_BT)^{-\frac{3}{2}} \sqrt{\frac{E_{cin}}{\pi}}\cdot \exp\left(-\frac{E_{cin}}{k_BT}\right) 
\end{align}
Da cui l'energia cinetica media delle particelle  $\left< E_{cin}\right>$ è data da:
\begin{align}
    E_{cin} = \frac{3}{2}k_BT
\end{align} 

\section{Gas reali}
\subsection{l'equazione di Van Der Waals per i gas reali}
l'Equazione di stato che rappresenti il comportamento di un gas reale con maggior 
precisione è sicuramente l'equazione di Wan Der Waals:
\begin{align}
    \left( P  + a\frac{n^{2} }{V^{2} } \right) (V - nb) = nRT
\end{align}
Dove a e b sono fattori correttivi specifici per ogni gas: il valore di b
è chiamato covolume o volume proprio ed è il volume vero delle molecole in stato
gassoso poiché in stato condensato è minore. Inoltre non essendo gli urti
perfettamente elastici ed essendoci forze di attrazione e repulsione fra le particelle
allora il termine $a(n^{2 }/V^{2})$ chiamato \textbf{pressione interna} rappresenta
questa correzione (più le particelle interagiscono tra loro e minore è la pressione
esercitata sulle pareti).

\subsection{Discostamento dei gas reali dal comportamento ideale}
Un approccio utile per poter determinare il comportamento dei gas reali è 
utilizzando il \textbf{fattore di comprimibilità}:
\begin{align}
    Z = \frac{V^{R}_m }{V^{I}_m }
\end{align}
Dove $V^{R}_m$ è il volume occupato da una mole di gas reale e l'altro 
quando è ideale. Utilizzando la relazione dei gas ideali si ottiene:
\begin{align}
    Z = \frac{V^{R}_m P^{R} }{RT}
\end{align} 
Per i gas reali ora si ottiene:
\begin{align}
    Z = \frac{V_m^{R} }{V^{R} - b } - \frac{a}{V^{R}RT }
\end{align}
Il quale varia a seconda della temperatura e a seconda del gas:
al di sopra della \textbf{temperatura di Boyle} il fattore è sempre
$> 1$ per qualsiasi gas.

\subsection{Uso dell'equazione di stato del gas ideale per i gas reali}
L'equazione di stato dei gas ideali mette in relazione il volume di un gas
con la qta. di moli gas e viceversa. In modo analogo possiamo definire quindi
il \textbf{volume molare standard} ossia il volume alle condizioni standard
(temp = 0°C) e ad un bar per il gas ideale come:
\begin{align}
    V_M = \frac{RT}{P}
\end{align}
Da questo si calcola la \textbf{densità standard} per qualsiasi gas:
\begin{align}
    d_G = \frac{M_G}{V_M} = \frac{PM}{RT}
\end{align}
Dove M è la massa molare e G è il gas considerato.

\subsection{I gas nelle miscele gassose}
Nelle miscele gassose i gas si comportano tutti come se fossero un unico gas con
delle proprietà ben definite (nell'assunzione che il gas reale si possa
approssimare ad un gas ideale); tuttavia si può conoscere la
pressione di un solo gas (ossia la \textbf{pressione parziale}) di un gas
se fosse l'unico nel recipiente. Se n moli di un gas A occupano un volume V
allora la sua pressione parziale è data da:
\begin{align}
    P_A = \frac{n_ART}{V}
\end{align}
Quindi generalizzando si ha per N gas:
\begin{align}
    P_{tot} = \sum_{i = 1}^{N}& P_i = n_{tot} \frac{RT}{V} \\
    P_A &= \frac{n_A}{N} 
\end{align}

\subsection{Effusione dei gas}
L'effusione dei gas è la diffusione di un gas da un recipiente pieno ad uno vuoto
e la sua velocità (ossia la quantità di moli che attraversa il foro per unità di tempo)
dipende dalla velocità(temperatura e pressione) e dalla massa molare.
Secondo quindi la \textbf{Legge di Graham} la velocità di effusione è:
\begin{align}
    v_A \propto \left(\frac{T}{M_A}\right)^{\frac{1}{2}} 
\end{align}
Se ci sono due gas in effusione allora la legge ci dice che:
\begin{align}
    \frac{v_A}{v_B} = \left(\frac{M_B}{M_A}\right)^{\frac{1}{2}}  
\end{align}
Se il gas A è 4 volte più leggero del gas B, allora esso passerà 2 volte
più velocemente dell'altro. 

\section{Modello e proprietà dello stato liquido}
\subsection{Proprietà dello stato liquido}
Lo stato liquido è lo stato intermedio tra gassoso e solido poiché se
da un lato le particelle possiedono un certo moto caotico in tutte le direzioni
come nei gas, dall'altro invece sono comunque vincolate dalle deboli forze di 
interazione che le tengono unite e dall'energia potenziale che è negativa
(molto di più di quella gassosa).
\begin{enumerate}
    \item \textbf{isotropia} :  le proprietà dei liquidi sono le stesse in qualunque direzione esse sono misurate.
    \item \textbf{tensione superficiale}: la forza che si oppone in un liquido all'aumento della superficie esterna,
    la quale tende a far assumere ai gruppi di molecole una forma sferica. 
    \item \textbf{capillarità}: la capillarità è una proprietà che deriva proprio dalla tensione
    superficiale che tende a far salire un liquido in un capillare quando le forze con le
    pareti di un capillare (forze adesive) sono maggiori della forza peso. 
    \item \textbf{viscosità}: la viscosità è la caratteristica di un liquido di opporsi
    al passaggio di una sostanza al suo interno (o la resistenza allo slittamento di una
    lastra di particelle rispetto all'altra). Ovviamente sostanze con ponti ad idrogeno tra le
    molecole sono quelle più viscose. 
\end{enumerate}
\begin{figure}
    \centering
    \includegraphics[width=0.8\textwidth]{energia-interna.png}
    \caption{Comparazione delle energie interne tra solidi, liquidi e gas}
    \label{Fig 7.6}
\end{figure}

\subsection{Modello strutturale dei liquidi}
All'interno dei liquidi esistono cavità più o meno grandi tra le molecole: questo determina
il comportamento di alcune molecole ad oscillare (se non sono vicine ad una cavità
proprio come nei solidi) oppure a convertire la propria energia cinetica in traslazionale
andando ad occupare queste cavità; dal momento che questo comportamento è tipico dei
gas e che l'energia cinetica aumenta all'aumentare della temperatura, allora, mano a mano
che la temperatura cresce, aumenta il carattere gassoso dei liquidi.

\section{Sostanze vetrose o amorfe}
Alcune sostanze sono chiamate amorfe o vetrose quando hanno proprietà 
intermedie fra liquidi e solidi. Questo tipo di sostanze, sebbene presenti delle
proprietà riconducibili ai solidi, non presenta un reticolo cristallino geometrico,
ordinato e periodico, il che le rende molto più simili ad un sistema liquido. Una sostanza
vetrosa può essere immaginata come un liquido senza proprietà traslazionali
ossia come un liquido ad altissima viscosità. \\
La formazione di questo tipo di sostanze si ha quando si raffredda velocemente 
un liquido: questo porta all'abbassamento repentino dell'energia traslazionale
che porta le molecole a congelarsi in posizioni casuali, impedendo la formazione
di un reticolo ordinato e periodico. Le sostanze ioniche tendono a non dare
vita a sostanze vetrose in quanto presentano strutture cristalline molto semplici.

\chapter{Equazioni chimiche, stechiometria e nomenclatura}
\section{Significato quantitativo delle formule}
\subsection{Numeri di ossidazione degli elementi nei composti}
Nelle reazioni che coinvolgono l'ossigeno, questo ha sempre numero di ossidazione
pari a $+2$ e l'idrogeno sempre $+1$. Nelle reazioni la somma dei numero di ossidazione deve
sempre fare 0.

\subsection{Periodicità dei numeri di ossidazione}
I numeri di ossidazione seguono un andamento periodico che è correlato al numero di
elettroni nello strato più esterno e quindi dipendendo dall'elettronegatività
i primi 3 gruppi sono quelli che di solito hanno numero di ossidazione positivo.

\section{Reazioni chimiche}
\subsection{Equazioni chimiche}
Le equazioni chimiche permettono di rappresentare su carta cosa accade nel mondo della
chimica all'interno della quale deve valere il principio di conservazione
della massa per cui se una reazione non è completa bisogna procedere con il processo
di bilanciamento della reazione. 

\subsection{Bilanciamento delle equazioni chimiche}
Bilanciare un'equazione chimica significa fare in modo che il numero di atomi di ciascun
elemento sia lo stesso da entrambe le parti dell'equazione. Si dividono le reazioni in classi:\\
\textbf{Reazioni acido base}: \\
Una reazione acido-base consiste nel trasferimento di un protone da una specie all'altra
(da acido a base): 
\begin{gather*}
    \ce{H3O^{+}  + OH^{-}  -> 2H2O}
\end{gather*}
La somma algebrica in queste equazioni delle cariche deve essere la stessa da ambedue le parti.\\
\textbf{Reazioni di formazione di composti di coordinazione o complessazione}: \\
Per esempio aggiungendo \ce{NH3} ad una soluzione acquosa di sale di Nichel si ottiene:
\begin{gather*}
    \ce{Ni^{2+} + 6NH3 -> Ni(NH3)_{6}^{2-}}
\end{gather*}
\textbf{Reazioni di precipitazione}
Le reazioni di precipitazione è una reazione nella quale si forma dei prodotti che non solubili
nel solvente utilizzato come nel caso di:
\begin{gather*}
    \ce{2Ag^{+}  + H2S - > Ag2S + 2H^{+} }
\end{gather*}

\textbf{Reazioni di ossidoriduzione}
Di gran lunga le più articolate, queste sono delle reazioni in cui almeno una coppia di atomi
cambia il proprio numero di ossidazione come nella seguente:
\begin{gather*}
    \ce{O2 + 2H2 -> 2H2O }
\end{gather*}
Dal momento che ogni atomo di idrogeno passa da zero a +1 come numero di ossidazione ed
ogni ossigeno da 0 a -2, il numero di elettroni persi da una specie deve essere uguale al
numero di elettroni acquisiti dall'altra. Si scrive quindi la seguente reazione:
\begin{gather*}
    \ce{Zn + 2H^{+}  -> Zn^{2+} + H2}
\end{gather*}
Poiché lo zinco è attaccato da un acido, allora perde due elettroni portando alla seguente:
\begin{gather*}
    \ce{Zn -> Zn^{2+} + 2e^{-} }
\end{gather*}
E quindi ogni atomo di idrogeno
\begin{gather*}
    \ce{2H^{+}  + 2e^{2}  -> H_{2}}
\end{gather*}
La specie che perde elettroni si chiama \textbf{riducente} e quella che acquisisce
elettroni \textbf{ossidante} e acquista elettroni riducendo il proprio numero di
ossidazione. \\
Nella seguente reazione si ha un'ossidazione del magnesio:
\begin{gather*}
    \ce{MnO^{-}_{4}  + 5e^{-}  -> Mn^{2+}} 
\end{gather*}
Così si aggiungono 8 idrogeni (poiché avviene in ambiente acido)
per bilanciare le cariche:
\begin{gather*}
    \ce{MnO^{-}_{4}  + 8H^{+}  + 5e^{-}  -> Mn^{2+} + 4 H2O}
\end{gather*}
Se nello stesso ambiente abbiamo anche la seguente
\begin{gather*}
    \ce{Fe^{2+} -> Fe^{3+} + e^{-} }
\end{gather*}
Allora il numero di elettroni il numero di elettroni dell'ossidante deve essere uguale 
al numero di elettroni del riducente e quindi moltiplichiamo per 5 quest'ultima 
e sommando tutto si ottiene:
\begin{gather*}
    \ce{MnO^{-}_{4} + 8H^{+}  + 5Fe^{2+} -> Mn^{2+} + 5Fe^{3+} + 4H_{2}O}
\end{gather*}

Generalmente invece delle semireazioni è più rapido bilanciare la reazione seguendo tre passaggi:
\begin{enumerate}
    \item Bilanciamento del numero di elettroni;
    \item bilanciamento delle cariche;
    \item bilanciamento degli atomi di idrogeno ed ossigeno.
\end{enumerate}
Per esempio la seguente reazione:
\begin{gather*}
    \ce{Cr_{2}O^{2-}_{7} + I^{-}  -> I_{2} + Cr^{3+}}
\end{gather*}
Dal momento che avviene in ambiente acido, si osserva che:
\begin{gather*}
    \ce{2I^{-}  -> I_{2} + 2e^{-} } \\
    \ce {Cr2O^{2-}_{7} + 6e^{-}  -> 2Cr^{3+}}  
\end{gather*}
Si bilanciano gli elettroni e quindi:
\begin{gather*}
    \ce{6I^{-}  -> 3I_{2} + 6e^{-} } \\
    \ce {Cr2O^{2-}_{7} + 6e^{-}  -> 2Cr^{3+}}  
\end{gather*}
Dal momento che c'è una differenza di 14 cariche, si procede adesso
a bilanciare le cariche aggiungendo a destra 14\ce{OH-} oppure 14\ce{H+} a sinistra
poiché l'ambiente è acido, allora si utilizzano gli idrogeni e si aggiungono le molecole d'acqua:
\begin{gather*}
    \ce{Cr2O^{2-}_{7} + 6I^{-}  + 14H^{+}  -> 3I_{2} + 2Cr^{3+}+ 7H2O} 
\end{gather*}

\subsection{Agenti ossidanti e riducenti}
Sono potenziali agenti ossidanti tutte quelle sostanze che fanno alzare il numero di ossidazione
ad una sostanza all'interno di una reazione chimica e riducenti tutte quelle sostanze
che invece fanno ridurre il numero di ossidazione delle altre sostanze. \\
Quando invece una sostanza in una reazione chimica genera due specie chimiche diverse
allora si parla di reazione di \textbf{dismutazione} come nella seguente:
\begin{gather*}
    \ce{Cl_{2} + H_{2}O -> Cl^{-}   + ClO^{-}  + 2H^{-} } \\
    \ce{3MnO^{2-}_{4} + 4H^{+}  -> 2 MnO^{-}_{4} + MnO_{2} + 2H_{2}O }
\end{gather*} 
Nella seconda per esempio, per bilanciarla, si segue la stessa fatta per le ossidoriduzioni
ma scrivendo che uno ione \ce{MnO^{2-}_{4}} reagisce con un'altro uguale:
\begin{gather*}
    \ce{MnO^{2-}_{4} + MnO^{2-} -> MnO^{-}_{4}  + MnO_{2}}
\end{gather*}
E poi bilanciando secondo il metodo delle ossidoriduzioni.

\subsection{significato quantitativo delle reazioni chimiche}
La formula per ottenere il numero di moli dati i grammi di una certa sostanza ed il suo
peso molare è la seguente:
\begin{align}
    \frac{m(g)}{M(g\cdot mol^{-1}  )} = n(mol)
\end{align} 

\section{Soluzioni e la loro composizione}
Quando si tratta di \textbf{soluzioni} ci si riferisce a sistemi omogenei liquidi costituiti da
più di una sostanza e con una composizione variabile (possono essere anche solide). Un sistema gassoso
è sempre miscibile e quindi sempre omogeneo qualsiasi gas li costituisca. \\
Nelle soluzioni si possono ricavare con semplici formule la percentuale di un soluto
come massa o volume:
\begin{align}
    \frac{m(soluto)}{m(soluzione)} \cdot 100 &= \%m(soluto) \\
    \frac{V(soluto)}{V(soluzione)} \cdot 100 &= \%V(soluto)
\end{align}
Si può anche esprimere la composizione di una sostanza come il rapporto tra 
quantità di soluto e e quantità di sostanza totale ottenendo la \textbf{frazione molare}
del soluto 
\begin{align}
    x = \frac{n(soluto)}{n(soluto) + n(solvente)}
\end{align}
E la frazione molare del solvente come:
\begin{align}
    x(solvente) = \frac{n(solvente)}{n(soluto) + n(solvente)}
\end{align}
Di solito la concentrazione di una soluzione è espressa come \textbf{concentrazione
molale} espressa come $mol \cdot  kg^{-1}$ e indicata generalmente col simbolo $b$ oppure $C_m$ 
oppure come  \textbf{concentrazione di qta. di sostanza} $mol\cdot dm^{-3}$ indicata
come $c$ o $C_m$. Mentre la prima non dipende dalla temperatura la seconda si
così come la \textbf{solubilità}.

\section{Regole per la nomenclatura}
\subsection{Ioni mono e poli atomici e gruppi funzionali inorganici}
Gli ioni monoatomici positivi mantengono il nome dell'elemento, se invece sono
legati ad idrogeno perdono il suffisso -\emph{onio}. Gli anioni monoatomici
mantengono il nome dell'elemento e aggiungono il suffisso -\emph{uro} mentre gli anioni
poliatomici prendono il suffisso -\emph{uro} mentre la maggior parte dei poliatomici
ha il suffisso ato con il numero di ossidazione dell'atomo centrale (di, tri, tetra-...)
In tabella ci sono alcuni dei gruppi funzionali più importanti.
\begin{center}
    \begin{tabular}{c | c}
        -\ce{CO} & carbonile \\
        -\ce{CrO2} & cromile \\
        -\ce{NO} & nitrosile \\
        -\ce{NO2} & nitrile \\
        -\ce{OH} & ossidrile \\
        -\ce{PO} & fosforile \\
        -\ce{SO} & tionile \\
        -\ce{SO2} & solfonile \\
        -\ce{VO} & vanadile \\
        -\ce{UO2} & uranile 
    \end{tabular}
\end{center}

\subsection{Composti binari}
Nei composti binari tra metalli e non metalli il nome del metallo non subisce cambiamenti
rispetto al nome dell'elemento libero mentre quello del non metallo prende il suffisso -\emph{uro}
e si può utilizzare il numero di ossidazione per indicare il composto come cloruro di ferro(3) 
indica che il cloro è ossidato con numero -3.\\
Gli ossidi di elementi non metallici con comportamento acido si chiamano \textbf{anidridi}.

\subsection{Ossiacidi ed i loro sali}
Per gli ossiacidi e per i i sali si seguono le regole per la nomenclatura dei composti
binari utilizzando i suffissi -ico ed ato rispettivamente come nella tabella aggiungendo
il numero di ossidazione dell'atomo centrale.
\begin{center}
    \begin{tabular}{c | c}
        Acido \\
        \hline 
        Prefisso & Suffisso \\
        \emph{ipo-} & \emph{-oso} \\
        & \emph{-oso} \\
        & \emph{-ico} \\
        \emph{per-} & \emph{-ico} 
    \end{tabular}
    \begin{tabular}{c | c}
        Anione \\
        \hline
        Prefisso & Suffisso \\
        \emph{ipo-} & \emph{-ato} \\
        & \emph{-ito} \\
        & \emph{-ato} \\
        \emph{per-} & \emph{-ato} 
    \end{tabular}
\end{center}
Tabella per identificare (dall'alto verso il basso) i composti con numero
di ossidazione maggiore nei composti con più numeri di ossidazione. \\

Il prefisso \emph{perosso-} è usato per tutti i composti che contengono il perossido
\chemfig{-O-O-}. \\
I nomi di sali che contengono idrogeni hanno il prefisso \emph{idrogeno-}

\subsection{Composti di coordinazione}


\chapter{Proprietà delle sostanze correlate ai modelli del legame chimico}
\section{Sostanze elementari, composti e modelli di legame che li razionalizzano}
\subsection{Struttura delle sostanze elementari}
La prima grande suddivisione delle sostanze è tra non metalli e  metalli; quest'ultimi 
costituiscono la maggioranza degli elementi in natura ed il loro legame generalmente è a
massimo impacchettamento e anche allo stato liquido mantengono un'alto numero di coordinazione
ed il legame metallico; mentre i non metalli presentano molti tipi di legami different
tutti in base agli elettroni di valenza. \\
Generalmente la tendenza a formare legami di tipo $\pi$ diminuisce scendendo lungo ciascun
gruppo a causa dell'aumento del raggio atomico.  \\
Una grossa differenza tra metalli e non metalli allo stato elementare risiede nella quantità
di elettroni allo strato di valenza: prima del gruppo 13 gli elementi non hanno elettroni a 
sufficienza per poter completare la loro configurazione, il che li porta a preferire un legame
delocalizzato ossia il legame metallico. Solo il Boro è un'eccezione in quanto tende a
forare legami covalenti boro boro per dar vita a molecole molto grosse. \\
Da questo deriva che tutti gli elementi dopo il gruppo 14 debbano essere non metalli, tuttavia
non è così perché alcuni elementi nei periodi sottostanti sono ancora metalli come lo stagno e
il piombo poiché aumentando il periodo aumenta la compenetrazione degli orbitali s e quindi
si hanno sempre meno elettroni disponibili per il legame. \\
Gli atomi nelle \textbf{sostanze elementari polimere} sono sempre tenuti insieme da legami
covalenti e a seconda di come si struttura il legame possono essere completamente covalenti
se hanno strutture tridimensionali oppure avere interazioni deboli altrimenti. \\
Esistono inoltre moltissime \textbf{sostanze elementari molecolari} i cui legami covalenti
sono presenti solo all'interno delle molecole, il che porta ad avere interazioni di Van Der
Waals per tenere insieme le varie molecole . \\
Infine le molecole monoatomiche dei gas nobili presentano deboli interazioni di Van Der Waals.

\subsection{Strutture dei composti}
Passando dalle sostanze elementari ai composti si ha la stessa suddivisione tra polimeri e composti
molecolari, più la classe dei composti ionici. Nei \textbf{composti covalenti polimeri}
come \ce{SiO2} si hanno legami covalenti che creano strutture tridimensionali, mentre in altri
composti covalenti polimeri si hanno legami solo su due dimensioni. \\
Quando i legami covalenti non si estendono a tre dimensioni allora si hanno le interazioni
di Van Der Waals che tengono insieme i "fogli molecolari" oppure si hanno i ponti ad idrogeno. \\
Anche i \textbf{composti covalenti molecolari} nei loro stati condensati si ripete l'analogia
con le sostanze elementari molecolari: legami covalenti intramolecolari e interazioni deboli
intermolecolari. \\
Esistono infine anche molti composti ionici con \textbf{ioni polimerici} , così come
legami ionici all'interno di strutture molecolari con legami metallici e covalenti soprattutto
se presenti ioni poliatomici.  Nei composti ionici invece ci possono anche essere molecole 
d'acqua che occupano cavità specifiche del cristallo o che interagiscono con altre
componenti del composto.
\begin{figure}
    \centering
    \includegraphics[width=0.8\textwidth]{Legami-elementi.png}
    \caption{Tipi di legame degli elementi allo stato elementare. Il blu indica
    elementi con legami metallici, il verde con legami covalenti polimeri e l'arancione
    quello con legami covalenti molecolari.}
    \label{Fig 9.1}
\end{figure}

\section{Serie di composti inorganici}
\subsection{Composti binari con idrogeno: idruri}
L'atomo di idrogeno ha solo un elettrone ed un protone; questo gli conferisce 
la proprietà di impiegare solo legami covalenti oppure diventare lo ione idruro \ce{H^{-}}
Nella figura qua sotto sono rappresentati i principali tipi di idruri.
\begin{figure}
    \centering
    \includegraphics[width=0.8\textwidth]{Tipi-idruri.png}
    \caption{I principali tipi di idruri}
    \label{Fig 9.2}
\end{figure}
Questi sono suddivisi in 4 gruppi:
\begin{enumerate}
    \item Idruri ionici (in blu);
    \item Idruri molecolari (in arancione);
    \item Idruri polimeri con atomi di idrogeno a ponte (in verde);
    \item Idruri di elementi di transizione (non in tabella);
\end{enumerate}
Gli idruri ionici contengono lo ione \ce{H^{-}} ed hanno caratteristiche basiche
mentre gli idruri del gruppo 17 e 18 sono forti acidi. Gli idruri dal gruppo 14 al
gruppo 17 sono tutti gassosi tranne che per \ce{H2O} e \ce{HF} che sono liquidi. \\
Nell'idruro di boro invece l'alta energia di ionizzazione del boro porta alla
formazione di un composto ionico anche se una molecola discreta come \ce{BH3} è favorita
poiché il boro ha comunque 4 orbitali dando vita ad una molecola del tipo \ce{B2H6} che
ha 2 idrogeni a ponte tra i due atomi di boro dando vita a due orbitali molecolari
di legame che contengono due elettroni ciascuno e deriva dalla sovrapposizione
di 3 orbitali atomici. \\
Coi metalli di transizione si possono anche avere dei composti non stechiometrici solidi
con proprietà metalliche ma più fragili ed in genere conduttori elettrici con la stessa
struttura dei metalli puri ma più espansa poiché gli atomi di idrogeno si dispongono nelle
cavità.

\subsection{Composti binari con gli alogeni: alogenuri}
Gli elementi del gruppo 17 hanno i più elevati valori di affinità elettronica e di
elettronegatività dell'intera tavola periodica. Pertanto tendono quasi sempre ad acquisire
elettroni oppure a formare legami covalenti con numero di ossidazione -1. Qui sotto
sono rappresentati composti più comuni con gli alogeni ed il loro carattere:
\begin{figure}
    \centering
    \includegraphics[width=0.6\textwidth]{Alogenuri-composti.png}
    \caption{Gli alogenuri più comuni}
    \label{Fig 9.3}
\end{figure}
\begin{enumerate}
    \item Alogenuri ionici(blu);
    \item alogenuri covalenti molecolari (arancione);
    \item ALogenuri covalenti polimeri (verdi);
\end{enumerate}
Generalmente gli alogenuri con metalli sono tutti ionici, altrimenti covalenti; tuttavia
esistono anche composti covalenti molecolari con metalli nel gruppo 13. \\
Alogenuri molecolari monomeri sono tipici del boro e degli elementi elettronegativi
del gruppo 14, 15, 16. Mano a mano che l'elettronegatività degli elementi cresce
ci sono sempre più elementi legati con \ce{F} e \ce{Cl} poiché sono gli elementi più elettronegativi.
Negli alogenuri con i metalli più pesanti invece quest'ultimi tendono ad assumere numero
di ossidazione più piccolo di quelli permessi.\\
I rapporti degli alogenuri più stabili sono quelli con rapporto 1:1, mentre mano a mano
che cresce il numero di  alogeni nella molecola e più reattivi diventano.

\subsection{Composti binari con l'ossigeno}
L'ossigeno è l'elemento più elettronegativo dopo il fluoro e per questo tende ad assumere
sempre numero di ossidazione negativo (tranne quando lega col fluoro) di -2 negli \textbf{ossidi},
altrimenti di -1 nei \textbf{perossidi}. Nella tabella sono riportati i composti più comuni:
\begin{figure}
    \centering
    \includegraphics[width=0.6\textwidth]{Composti-ossigeno.png}
    \caption{Composti con l'ossigeno}
    \label{Fig 9.4}
\end{figure}
\begin{enumerate}
    \item ossidi ionici (blu)
    \item ossidi molecolari (arancione);
    \item ossidi polimeri (verde);
\end{enumerate}
Gli ossidi del primo e secondo gruppo sono praticamente ionici e caratterizzati dalla presenza
dello ione \ce{O^{2-}} e sono solidi e con temperature di fusione molto elevate. \\
Con gli elementi del gruppo 14 invece si hanno numeri di ossidazione variabili nei loro composti
con l'ossigeno e gli ossidi più comuni in genere sono quelli con il numero di ossidazione che 
corrisponde al massimo e a quello minore di due unità. \\
In generale gli ossidi, spostandosi da sinistra verso destra lungo un periodo variano
lo stato di aggregazione e le loro proprietà chimiche portando a da solidi ionici a ossidi molecolari
bassofondenti o liquidi ed infine a ossidi gassosi passando contemporaneamente da basici 
ad acidi (poiché più e forte il carattere ionico maggiore è il carattere di \ce{O^{2-}} che è una base forte).

\subsection{Composti ternari con ossigeno ed idrogeno}
I composti formati da un metallo, un semimetallo o un non metallo e da idrogeno ed ossigeno
sono chiamati idrossidi. Nella tabella i più comuni idrossidi.
\begin{figure}
    \centering
    \includegraphics[width=0.6\textwidth]{Tipi-idrossidi.png}
    \caption{idrossidi}
    \label{Fig 9.5}
\end{figure}
\begin{enumerate}
    \item Idrossidi ionici (blu);
    \item Idrossidi polimeri (verde);
    \item Idrossidi molecolari (arancione);
\end{enumerate}
I composti Ionici e polimeri sono tutti solidi , mentre quelli molecolari sono liquidi o solidi
e mai gassosi a causa dei ponti ad idrogeno. Esistono sia idrossidi basici che ossiacidi
con non metalli e idrossidi anfoteri che invece sono insolubili.

\section{Composti ionici formalmente derivati dagli acidi: i sali}
\subsection{Sali binari}
Nei sali binari i legami ionici sono l'unico tipo di legame e generalmente appartengono
a questa categoria i solfuri, i seleniuri e gli altri sali calcogeni. Generalmente solubili,
i sali binari con elementi alcanino-terrosi non sono molto solubili e quelli del gruppo 14
sono ancora meno solubili.

\subsection{Sali ternari e più complessi}
Molto più numerosi sono invece quelli ternari nei quali, oltre al legame ionico, c'è anche
un legame covalente 

\section{Composti organici}
I composti organici formano le \textbf{sostanze organiche} caratterizzate da legami covalenti
tra idrogeno e carbonio. La grande stabilità del legame C--C tende a far formare catene molto lunghe
di carbonio. I composti organici prendono diversi nomi a seconda dei legami, della lunghezza
della catena e della forma della catena o seconda dei \textbf{gruppi funzionali} a loro collegati.\\
Si chiamano \textbf{idrocarburi} divisi in \textbf{alifatici} o \textbf{aromatici} a second
delle loro caratteristiche di legame. Si dividono ulteriormente a seconda del numero di legami dei
carboni coinvolti: quando si hanno quattro legami si parla allora di \textbf{alcani}, 
altrimenti se fanno un legame doppio \textbf{alcheni} e se ne fanno un legame triplo \textbf{alchini}.
\begin{center}
    \begin{tabular}{| c | c | c | c |}
        \hline
        & Alcani & Alcheni & Alchini \\
        \hline 
        Formula minima & \ce{C_{n}H_{2n+2}} & \ce{C_{n}H_{2n}} &  \ce{C_{n}H_{2n-2}} \\
        \hline
    \end{tabular}
\end{center}
Gli alcani possiedono moltissimi isomeri soprattutto grazie al fatto che i
gruppi funzionali possono attaccarsi alla catena principale di carbonio virtualmente ovunque.
I \textbf{cicloalcani} sono degli idrocarburi formati da catena chiusa ad anello con formula
minima uguale a quella degli alcheni. Dal momento che i legami $\pi$ sono più deboli di
quelli $\sigma$, gli alcheni ed gli alchini sono molto più instabili.\\
Gli \textbf{idrocarburi aromatici} sono composti con struttura ad anello chiuso planare con
legami $\sigma$ tra gli atomi di carbonio e di idrogeno con orbitali $\pi$ delocalizzati sopra e sotto
l'anello. Di seguito i gruppi funzionali più comuni e dopo i sostituenti costituiti da atomi di carbonio ed idrogeno.
\begin{center}
    \begin{tabular}{c | c | c}
        Gruppo & Nome & Composti \\
        \hline
        \chemfig{-X} & Alogeno & Alogenuri \\
        \hline
        \chemfig{-OH} & Ossidrile & Alcoli e fenoli \\
        \hline
        \chemfig{-SH} & Solfidrile & Tioalcoli e tiofenoli \\
        \hline
        \chemfig{-[-1]O-[-3]} & Ossigeno etereo & Eteri \\
        \hline
        \chemfig{-[-1]S-[-3]} & Zolfo tioetereo & Tioeteri \\
        \hline
        \chemfig{-[-1]{C}(-[-3])(=O)} & Carbonile & Chetoni \\
        \hline 
        \chemfig{H-[-1]{C}(-[-3])(=O)} & Formile & Aldeidi \\
        \hline 
        \chemfig{-C(=[1]O)(-[-1]OH)} & Carbossile & Acidi carbossilici \\
        \hline
        \chemfig{-NH_{2}} & Ammino & Ammine \\
        \hline
        \chemfig{-[-1]{C}(-[-3])(=NH)} & Immino & Immine \\
        \hline 
        \chemfig{-C(=N)} & Ciano & Nitrili \\
        \hline 
        \chemfig{-C(=[1]O)(-[-1]HN^{-})} & Carbossiammidico & Ammidi \\
        \hline
        \chemfig{-C(=[1]O)(-[-1]OR)} & Alcossicarbonile & Esteri \\
    \end{tabular}
\end{center}
\begin{center}
    \begin{tabular}{c | c}
        Gruppo & Nome \\
        \hline
        \chemfig{-CH_{3}} & Metile \\
        \hline
        \chemfig{-CH_{2}CH_{3}} & Etile \\
        \hline
        \chemfig{-(CH_{2})_{n}H} & Alchile \\
        \hline
        \chemfig{-C=C-} & Alchenile \\
        \hline
        \chemfig{-C=C-} & Alchinile \\
        \hline
        \chemfig{--[1]=-[-1]=[-3]-[-4]=[-5]} & Fenile \\
    \end{tabular}
\end{center}
All'interno di una catena carboniosa ci possono essere più sostituenti così come più gruppi funzionali;
di grande interesse sono soprattutto gli \textbf{amminoacidi} ossia un gruppo amminico, un carbossilico,
un idrogeno ed un sostituente:
\begin{gather*}
    \chemfig{R-C(-[2]H)(-[-2]NH_{2})-C(=[1]O)(-[-1]OH)}
\end{gather*}

\section{Composti di coordinazione}
\subsection{Definizione}
questi sono composti nei quali:
\begin{enumerate}
    \item Un atomo centrale è circondato da altri atomi con cui forma legami covalenti polari;
    \item Il numero di legami che forma è sempre maggiore del suo numero di ossidazione;
    \item Gli atomi che si legano donano una loro coppia di elettroni da far finire nell'orbitale 
    vuoto di questo atomo centrale;
    \item L'identità di quelli solubili in acqua rimane inalterata in soluzione.
\end{enumerate}

TODO PAGINA 280
\section{Isomeria}
L'isomeria strutturale degli idrocarburi non è un fenomeno raro nella chimica;
tornado agli amminoacidi, anche essi presentano due isomeri: uno "destrorso" ed
uno "mancino" a seconda se il sostituente si trova a destra o a sinistra del gruppo
amminico: solo quelli levogiri ("mancini") sono utilizzati nelle cellule. \\
Esistono vari tipi di isomeria ma quando si parla di composti in cui atomi o gruppi di atomi
possono occupare nella molecola posizioni diverse allora si parla di \textbf{isomeria
configurazionale} ed \textbf{isomeria geometrica}. \\
Esiste anche un tipo di isomeria chiamata \textbf{isomeria ottica} che si riferisce
alla proprietà di alcune molecole di avere un centro di simmetria interno o
esterno alla molecola stessa. Le molecole che non hanno piani di simmetria interni
sono chiamate \textbf{chirali}. \\
Due molecole le cui formule di struttura siano immagini speculari non sovrapponibili
sono chiamate isomeri ottici o \textbf{enantiomeri}. Anche se sembrano diverse queste
molecole, hanno proprietà fisiche e chimiche esattamente identiche ma si differenziano
tra loro poiché fanno ruotare in senso diverso i fasci di luce polarizzati. Quando
si mischiano insieme due sostanze che fanno ruotare di un certo grado la luce in un senso
e di un certo grado nel senso opposto e facendovi passare un fascio di luce polarizzata
non si vede alcuna modifica, allora siamo di fronte ad un \textbf{racemo}. \\
Quando inoltre due molecole che sono speculari ma non sovrapponibili, che se ruotate diventano
sovrapponibili, allora prendono il nome di \textbf{diasteroisomeri}. 

\section{Colore delle sostanze}
Il colore dei solidi può essere osservato per trasparenza oppure per riflessione; nel
primo caso la sostanza è attraversata dalla luce bianca ed assorbe alcune frequenze quantizzate
lasciando passare le altre mentre per la riflessione dipende dalle proprietà della superficie
della sostanza anche se le frequenze riflesse sono uguali a quelle della trasparenza.
Sostanze come il vetro ed il diamante (se costituite da un solo cristallo senza
imperfezioni) sono completamente trasparenti mentre le sostanze formate da molti
cristalli tendono a non esserlo (come il sale) poiché disperdono la luce in tutte le direzioni. \\
In generale le sostanze che sono colorate sono quelle le cui radiazioni riflesse e/o
lasciate passare cadono proprio nell'intervallo della radiazione luminosa.  I metalli in
particolare sono in grado di assorbire un ampia gamma di radiazione luminosa data il loro
ampio intervallo di stati elettronici eccitati e quindi, poiché ogni elettrone eccitato
tende a tornare subito nel suo stato iniziale, riemette tutta la radiazione assorbita
andando così a risultare completamente riflettenti. 

\section{Proprietà magnetiche delle sostanze}
Ogni elettrone possiede un leggero campo magnetico intrinseco legato al suo spin il cui valore è:
\begin{align}
    \mu = 2.00232\sqrt{s(s + 1)} \frac{he}{4\pi m_e} = 1.734 \frac{he}{4\pi m_e} = 1.724 \mu_B 
\end{align}
La costante in frazione è chiamata \textbf{magnetone di Bohr} che esprime il momento magnetico
per cui elettroni nello stesso orbitale hanno stesso modulo di momento ma con segno opposto. \\
Quando le sostanze hanno momento magnetico pari a zero sono chiamate \textbf{diamagnetiche} mentre
le specie che possiedono elettroni spaiati sono \textbf{paramagnetiche} con numero di spin pari alla somma
dello spin dell'elettrone.  A temperatura ambiente queste proprietà magnetiche
non sono visibili in quanto il moto casuale degli elettroni tende a mascherarle, tuttavia se poste
all'interno di un campo magnetico allora viene \textbf{indotto} un momento magnetico
proporzionale alla forza del campo e inversamente proporzionale alla temperatura. Una sostanza
diamagnetica quindi tende ad essere spinta verso zone di minore intensità del campo magnetico mentre
una sostanza paramagnetiche viene attratta verso il punto di massima intensità. \\
Se si toglie il campo magnetico le sostanze paramagnetiche tornano ad essere diamagnetiche
poiché l'energia cinetica dei dipoli li porta ad orientarsi in modo casuale, tuttavia le
sostanze \textbf{ferromagnetiche} mantengono localmente (all'interno di \textbf{domini}) l'orientazione dei dipoli
continuando a generare un campo magnetico anche dopo essere state tolte da un campo magnetico esterno
poiché i singoli domini si allineano dando vita ad un campo magnetico che prima non c'era.
Le sostanze \textbf{ferrimagnetiche} invece hanno un campo magnetico naturale originato
dalla combinazione di molecole paramagnetiche di tipi diversi e con elettroni spaiati. \\
Le sostanze paramagnetiche sono generalmente composti di coordinazione formati da metalli
che hanno gli orbitali $d$ semiriempiti e quindi con elettroni spaiati. Quando sono presenti
molti elettroni (ma non tutti quelli necessari per riempire gli orbitali) allora essi possono
disporsi in due modi: seguendo la regola di Hund (rimanendo spaiati e quindi dando origine
ad una configurazione ad \textbf{alto spin}) oppure se la separazione è maggiore del 
guadagno energetico allora tenderanno ad assumere una configurazione tale per cui
si disporranno negli orbitali già semiriempiti andranno a creare una configurazione con
pochi elettroni spaiati (\textbf{basso spin}).
\begin{figure}
    \centering
    \includegraphics[width=0.8\textwidth]{spin-magneti.png}
    \caption{Configurazioni ad alto (blu) e basso spin(arancione) di alcuni  complessi ionici}
    \label{Fig 9.6}
\end{figure}

\chapter{Termodinamica}
\section{Concetti di base}
\subsection{Entropia, entalpia e energia libera}
La termodinamica è una delle branche fondamentali della fisica e che si applica pure alla chimica
sebbene si trattata con una giustificazione intuitiva basata sull'osservazione microscopica e sulle implicazioni chimiche.
Nella termodinamica si considera il contributo all'abbassamento dell'energia totale del sistema
che è non può essere correlato solo alle variazione dell'\textbf{energia interna} $\Delta U$. Se 
si cede calore a un gas si fa aumentare l'energia cinetica delle particelle che lo compongono,
aumentandone la temperatura; tuttavia se il volume rimane costante, allora deve anche aumentare
la pressione (oppure il contrario). Per cui tenendo conto anche della teoria cinetica dei gas
si chiama la variazione di energia interna (i cui contributi sono $\Delta U$ e $P\Delta V$ (
ossia la produzione eventuale di lavoro meccanico)) come
variazione di \textbf{entalpia} $\Delta H$. \\
L'\textbf{energia libera} è la grandezza che il sistema tende a minimizzare in una trasformazione 
e si definisce come
\begin{align}
    \Delta G = \Delta H - T\Delta S
\end{align} dove $-T\Delta S$ è il contributo a 
raggiungere la configurazione di particelle statisticamente più favorevole. In questa definizione
$\Delta S$ è la variazione della funzione di stato chiamata \textbf{entropia}. \\
Una trasformazione è \textbf{spontanea} quando tende a far abbassare l'energia libera. Il \textbf{
calore} è un mezzo con cui l'energia può essere scambiata tra due sistemi mentre il \textbf{lavoro}
è una forma di energia che il sistema scambia con l'ambiente durante la trasformazione. Detto questo
si enunciano le leggi della termodinamica:
\begin{enumerate}
    \item Se due sistemi sono in equilibrio termico con un terzo, allora sono in equilibrio anche tra loro;
    \item L'energia non si crea né si distrugge ma può solo cambiare forma e l'energia totale dell'Universo è costante;
    \item L'entropia di un sistema isolato non in equilibrio aumenta e raggiunge un massimo quando il sistema raggiunge l'equilibrio;
    \item Se la temperatura di un sistema si avvicina allo zero assoluto, la sua entropia diminuisce fino a zero o fino ad
    un valore molto piccolo.
\end{enumerate} 

\subsection{Informazione, ordine e disordine}
L'energia può essere ripartita tra le particelle in moltissimi modi (\textbf{stati}) e a sua volta si possono anche
ottenere delle permutazioni tra i valori energetici delle particelle chiamati \textbf{microstati} e l'entropia
è correlata direttamente al numero di questi microstati:
\begin{align}
    S = k_N \ln W
\end{align}
Dove W è il numero di microstati (1 per un cristallo perfetto). Il concetto di microstato
è strettamente legato sia alla temperatura che alla posizione delle particelle per cui
in gas per ogni valore di T, P e V ci sono tantissimi microstati: più grande è il numero di
microstati e più "disordinato" è il sistema e quindi si ha una perdita di informazioni
sullo stato effettivo del sistema. E' per questo che l'entropia è definita anche come la
"misura del disordine" poiché il disordine è rappresentato dall'alto numero di microstati
delle particelle di una sostanza. L'entropia quindi nelle sostanze ha la seguente gerarchia:
solido $<$ liquido $<$ gas, ed il suo valore zero corrisponde alla situazione ideale secondo la quale
si hanno zero microstati. 

\section{Capacità termica}
L'aumento di temperatura di una sostanza dipende sia dal calore ceduto che da una proprietà del 
materiale chiamata \textbf{capacità termica} per cui:
\begin{align}
    \Delta T = \frac{q}{C}
\end{align}
Questa è una proprietà estensiva della materia per cui dipende dalla sostanza e dalla sua quantità;
esiste anche un la \textbf{capacità termica molare} che è propria di una mole di sostanze. \\
La capacità termica di una sostanza dipende anche dalla pressione: mentre nei solidi e liquidi
non è apprezzabile l'espansione termica, i gas tendono ad utilizzare parte dell'energia fornitegli
per espandersi e per questo la capacità termica a pressione costante è maggiore di quella a volume
costante. Per un gas ideale si ha:
\begin{gather*}
    C_P = \frac{\Delta H}{\Delta T} = \frac{\Delta U + P \Delta V}{\Delta T} \\
    C_V = \frac{\Delta U}{\Delta T}
\end{gather*}
\begin{align}
    C_P = \frac{\Delta U}{\Delta T} + nR = C_V + nR
\end{align}
La \textbf{capacità termica molare a pressione costante} è quindi
\begin{align}
    C_P = C_V + R
\end{align}
Essendo per la teoria cinetica dei gas l'energia cinetica l'unica energia interna, allora un aumento di
energia a volume costante porta ad aumento di energia cinetica:
\begin{gather*}
        \Delta E = \frac{3}{2\Delta T}R = \Delta U, \Rightarrow 
\end{gather*}
\begin{align}
    C_V = \frac{3}{2}R
\end{align}
Per i gas biatomici vi è anche un'energia cinetica rotazionale degli atomi attorno al loro baricentro,
il che porta a dare i seguenti valori:
\begin{gather*}
    C_V = \frac{5}{2}R \\
    C_P = \frac{7}{2}R
\end{gather*}
Le molecole triatomiche invece hanno tre componenti rotazionali e dunque:
\begin{gather*}
    C_P = 4R
\end{gather*}
La \textbf{legge di Doulong e Petit} permette di calcolare la capacità termica molare di un solido ideale costituito
da atomi disposti con regolarità in un reticolo cristallino data da $C_P = C_V = 3R$.

\section{Transizioni di stato spiegate con il modello microscopico dei gas}
\subsection{Entalpia, entropia, energia libera ed evaporazione di un liquido}
Qualunque sostanza tende ad evolversi nel tempo e a cambiare il proprio stato; dal punto di vista 
di minima entalpia, è favorita la condensazione ad uno stato solido in quanto l'energia
cinetica delle particelle si abbassa.  \\
Un gas in equilibrio con la sua fase liquida risente di una pressione costante a temperatura
costante chiamata \textbf{tensione di vapore}: nel caso di un liquido in un barattolo, le particelle
del liquido devono vincere la forza di reciproca attrazione per diventare gassose, il che richiede
energia ed è quindi un processo \textbf{endoentalpico} anche se l'entropia favorisce questo 
processo fino al raggiungimento della tensione di vapore. \\
Il passaggio da liquido a gassoso causa un aumento di entalpia e anche un aumento di entropia; all'inizio
del processo l'aumento del prodotto TS è massimo, poi tende a ridurre il suo incremento fino a che
non si arriva alla fase di equilibrio $TS = \Delta H$. Il tutto avviene senza variazione di energia libera
poiché nella situazione di equilibrio si ha che $\Delta G = \Delta H - T\Delta S = 0$. In un sistema 
a temperatura costante si hanno alcune particelle che hanno un'energia cinetica maggiore a quella
di legame e quindi tendono a staccarsi da liquido ed entrare in fase gassosa. Se non ci fosse
un termostato il liquido si raffredderebbe, altrimenti continuerebbe a rimanere alla stessa temperatura
e quindi la frazione di molecole tali che $E_{cin} > E_{inter}$ è costante. Si raggiunge dunque un equilibrio
\textbf{dinamico} nel quale alcune molecole di gas tornano liquide le quali sono bilanciate dalle molecole
liquide che tornano gassose. 

\subsection{Il campo di esistenza di un liquido}
La tensione di vapore non dipende dallo spazio a disposizione del gas, dalla superficie di separazione liquido-gas
ma aumenta all'aumentare della temperatura non solo perché il gas aumenta la propria energia cinetica ma anche perché
cresce la sua quantità per unità di volume. La dipendenza dalla tensione di vapore dalla temperatura è
espressa dalla seguente legge di Clausius-Clapeyron :
\begin{align}
    \ln \frac{P_2}{P_1} = \frac{\Delta H_{vap}^{o} }{R}\left(\frac{1}{T_1}-\frac{1}{T_2}\right)
\end{align}
Questa equazione ci indica che la diminuzione della pressione al diminuire della temperatura è maggiore per
quelle sostanze che hanno maggiore entalpia di vaporizzazione, ossia che hanno legami intermolecolari più forti. \\
Se l'evaporazione del liquido avviene in un recipiente aperto lo spazio a disposizione della fase gassosa è infinito
e la pressione parziale del gas non raggiunge mai il valore della tensione di vapore e quindi $T\Delta S$ sarà
sempre maggiore e porterà all'evaporazione totale. Se l'evaporazione avviene velocemente la temperatura del liquido
diminuisce rispetto a quella ambiente poiché è più veloce la sottrazione del calore dal liquido dovuta all'evaporazione. \\
Se invece si mantiene il vuoto in un contenitore con un liquido dentro, allora si ha il processo di \textbf{ebollizione}
durante il quale le molecole del liquido passano allo stato gassoso dentro il liquido. \\
La temperatura alla quale la tensione di vapore  è una atmosfera è chiamata \textbf{temperatura normale di ebollizione} e
rimane costante durante tutto il processo di ebollizione.

\subsection{Transizione solido-gas}
I criteri discussi per la trasformazione liquido-gas possono essere estesi anche alla trasformazione solido-gas. \\ 
Dal momento che i solidi hanno energia potenziale minore in valore assoluto rispetto ai gas, il processo di trasformazione
solido-gas è sfavorito dal punto di vista entalpico ma non dal punto di vista entropico poiché in un solido ci saranno
sempre delle particelle che avranno un $E_{cin} > E_{inter}$ e quindi si definisce anche qui una \textbf{tensione di vapore del solido}
che aumenta sempre con la temperatura ma è minore di quella dei liquidi a causa dei legami più forti. \\
La temperatura alla quale la pressione del gas in equilibrio col solido è di una atmosfera si chiama \textbf{temperatura normale di sublimazione},
il processo di sublimazione consiste nella rottura del reticolo cristallino e al passaggio allo stato gassoso. Solo
alcune sostanze tuttavia sublimano, molte di esse passano invece allo stato liquido col processo di \textbf{fusione} ma
tutte possono sublimare se la pressione esterna è minore della tensione di sublimazione. 

\subsection{Transizione solido-liquido}
La temperatura a cui avviene questa transizione è chiamata \textbf{temperatura di fusione} ed in queste condizioni
la tensione  del solido è uguale a quella del liquido e l'aumento entalpico ed entropico sono piccoli poiché non si perde
molta energia nei legami e non è così caotico come nel passaggio allo stato gassoso.

\section{Entalpie di vaporizzazione, sublimazione e fusione e proprietà strutturali}
Si definisce \textbf{entalpia molare standard di vaporizzazione} ($\Delta H_{vap}^{o} $) la variazione di entalpia
necessaria per vaporizzare una mole di liquido, la quale dipende strettamente dalla forza di legame intermolecolare. \\
Si definisce invece \textbf{entalpia molare standard di sublimazione} ($\Delta H_{subl}^{o} $) la variazione di entalpia necessaria
per sublimare una mole di sostanza alla temperatura normale di sublimazione, che si può esprimere come l'energia
di legame delle molecole di un solido poiché corrisponde al lavoro meccanico di espansione per mole di sostanza dallo
stato solido a quello gassoso. \\
L'\textbf{entropia molare standard di sublimazione} ($\Delta S_{subl}^{o} $) non è molto differente da sostanza a sostanza
per cui si ricavano indicazioni sulle forze che agiscono allo stato solido. \\
L'\textbf{entalpia molare standard di fusione} ($\Delta H_{fus}^{o}$) è definita in modo analogo alle entalpia
di vaporizzazione e sublimazione ed è il calore che bisogna fornire per fondere una mole di sostanza. Dal momento
che l'energia potenziale di un liquido non è trascurabile come quella di un liquido allora l'entalpia di fusione e la 
temperatura di fusione sono indice di come variano le forze di coesione sono in sostanze con caratteristiche simili. \\
Tuttavia anche in quelle sostanze in cui la temperatura di fusione è molto bassa rispetto a sostanze simili 
(come nel caso del mercurio) le temperature di ebollizione sono comunque molto alte poiché le forze di coesione
sono molto forti e che si mantengono pressoché inalterate allo stato liquido. 

\section{Diagrammi di stato a un componente}
I valori di temperatura e pressione per cui una sostanza è stabile a stato solido o liquido si chiama
\textbf{diagramma a stato}. Sul diagramma di stato vi sono tre linee: la curva di equilibrio tra solido e liquido
che è verticale(ad alte temperature inizia a diventare una curva), la curva di equilibrio solido-gassoso e la curva di equilibrio liquido-gassoso. Inoltre sono presenti due punti: 
il \textbf{punto triplo} nel quale tutte e tre le fasi sono in equilibrio ad una certa temperatura ed ad una certa pressione e
il punto a cui giunge la curva di equilibrio liquido-gas a partire dal punto triplo ossia il \textbf{punto critico}
in cui la tensione di vapore è così grande che il gas assume la densità del liquido e le due fasi non sono più distinte. Sopra
alla \emph{temperatura e pressione critici} si ha un \textbf{fluido supercritico}
\begin{figure}
    \centering
    \includegraphics[width=0.6\textwidth]{punto-triplo.png}
    \caption{Diagramma a stato di una sostanza generica}
    \label{Fig 10.1}
\end{figure}

\subsection{Diagramma si stato di \ce{H2O} e \ce{CO2}}


\section{La transizione liquido-vapore in un sistema di due liquidi completamente miscibili}
La tensione di vapore di una miscela omogenea ideale di due liquidi completamente miscibili è 
la somma della tensione di vapore moltiplicato per la frazione molare x (\textbf{legge di Raoult}):
\begin{align}
    P = x_1 P_1^{o}  + x_2 P_2^{o} 
\end{align} 
Con la legge di Raoult e quella di Dalton si ottiene la frazione molare di un componente nella fase vapore:
che è in equilibrio con con il liquido di composizione x:
\begin{align}
    g_1 = x_1 \frac{P_1^{o} }{P_{tot}}
\end{align}
\begin{figure}
    \centering
    \includegraphics[width=0.6\textwidth]{diagramma-a-stato-due-liquidi.png}
    \caption{Diagramma a stato di due liquidi completamente miscibili, la linea verde rappresenta la curva
    del vapore, sopra la quale è completamente gassosa la soluzione e quella blu la curva del liquido sotto
    la quale la soluzione è completamente liquida.}
    \label{Fig 10.2}
\end{figure}

\subsection{Miscele azeotropiche}
Una soluzione liquida a due componenti può deviare così tanto dal comportamento ideale che gli equilibri
possono avere un minimo o un massimo, la soluzione che ha la massima temperatura è chiamata \textbf{miscela azeotropica}:
essa è in equilibrio alla temperatura di ebollizione con un vapore della medesima composizione. \\
Queste miscele oltre ad avere un punto di minimo e di massimo, permettono di separare liquidi completamente
miscibili tra di loro. 

\section{Le soluzioni ideali e le proprietà colligative}
\subsection{Tensione di vapore delle soluzioni poco volatili: innalzamento ebullioscopico e abbassamento crioscopico}
Fra i sistemi reali è utile esaminare il comportamento di soluzioni diluite ottenute da soluti solidi o comunque
poco volatili rispetto al solvente ($P_1 << P_2$) per cui il prodotto con $P_1$ è trascurabile e l'equazione di
Raoult diventa:
\begin{align}
    P = x_2P_2
\end{align}
la tensione di vapore è sicuramente minore della soluzione minore di quella del solvente puro e l'abbassamento
della tensione è uguale alla frazione molare del soluto (poiché $x_2$ è minore di 1):
\begin{align}
    \frac{P_2 - P}{P_2} = x_1
\end{align}
L'effetto dell'aggiunta di un soluto non volatile è quella di spostare la curva di equilibrio solido-gas verso
temperature più alte. (\textbf{innalzamento ebullioscopico}) poiché si abbassa la tensione di vapore mentre
la curva solido-liquido si sposta verso temperature più basse (\textbf{abbassamento crioscopico}). Sono vere poiché
un soluto non volatile ha interazioni più forti con il solvente rispetto ai legami intermolecolari(come nel caso dell'acqua). \\
La relazione tra queste due e la concentrazione molale della soluzione è ricavabile dalla precedente ed è data:
\begin{align}
    \Delta T_{eb} &= iK_{eb}b \\
    \Delta T_{cr} &= iK_{cr}b
\end{align} 
dove b è la concentrazione della soluzione in $mol kg^{-1}$ ed $i$ è il coefficiente \textbf{Van't Hoff} che
indica il rapporto tra la concentrazione della sostanza considerata b della specie e la concentrazione di
particelle della specie in soluzione. Le costanti $K_{eb}$ e $K_{cr}$ sono caratteristiche di ogni dato solvente
e non variano al variare del soluto e si chiamano \textbf{costante ebullioscopica molale} e \textbf{costante crioscopica molale}. 

\subsection{La pressione osmotica}
Alcune membrane hanno dei pori che permettono l'attraversamento solo di sostanze molto piccole come l'acqua, sono chiamate
allora  \textbf{membrane semipermeabili} ed il passaggio si chiama invece \textbf{osmosi}. Mettendo in contatto
una soluzione ed un solvente che è in grado di passare questa membrana si osserva un innalzamento della soluzione
rispetto al solvente considerato; il processo termina solo quando la pressione idrostatica si bilancia con la differenza di altezza
dei liquidi. Questo processo avviene poiché è favorito dall'entropia poiché un sistema diluito è meno ordinato di uno più concentrato. \\
La pressione osmotica $\pi$ è definita allora come la pressione da applicare sulla superficie della soluzione
quando è in contatto con un solvente attraverso una membrana semipermeabile affinché non avvenga l'osmosi. 
\begin{align}
    \pi = icRT
\end{align}
L'unica costante sconosciuta è \emph{c}, la quale è la \textbf{concentrazione di quantità di sostanza}. 
COme le altre proprietà colligative la pressione osmotica dipende solo dalla quantità di soluto e non dalla
sua natura chimica. Due soluzioni con la stessa pressione osmotica si dicono \textbf{isotoniche}


\chapter{Equilibrio chimico e termodinamica delle reazioni}
\section{La termodinamica delle reazioni}
\subsection{I parametri termodinamici delle trasformazioni chimiche}
Quando un sistema è chiuso, allora esso non scambia materia con l'ambiente; considerate le seguenti reazioni
\begin{gather*}
    \ce{aA + bB -> cC + dD} \\
    \ce{cC + dD -> aA + bB}
\end{gather*}
Entrambe le reazioni possono avvenire fino a quando l'entropia non minimizza l'energia del sistema. 
Questa posizione, chiamata \textbf{equilibrio dinamico}, le reazioni evolvono per in modo tale da far 
\textbf{diminuire l'energia libera del sistema}, ovvero a far sì che $\Delta G$ relativo sia minore di zero. 
Ci aspetta invece che l'entropia raggiunga un picco e poi scenda nuovamente, poiché la miscela è più disordinata
quando è composta al 50\% di prodotti e al 50\% di reagenti, mentre l'entalpia tende sempre a scendere. \\
Dal momento che vale la relazione $G = H -TS$ allora anche G avrà un minimo, che è più a destra rispetto
al minimo della funzione entropica (ossia quando l'entropia è massima) poiché il termine H continua a scendere. 
Dove $\Delta G = 0$ il sistema è in equilibrio. 

\subsection{Entalpia standard di una reazione}
Il calore sviluppato o assorbito da un sistema durante un trasformazione di una mole di sostanza alla temperatura
di 25° è pari a una variazione chiamata $\Delta H^{o}$ ossia \textbf{entalpia molare di reazione}. Quando invece
si forma una mole di composto partendo dalle sostanze elementari, con tutti i reagenti e prodotti a condizioni standard,
allora si chiama \textbf{entalpia molare standard di formazione} $\Delta H_{f}^{o}$, che concettualmente è uguale a zero
poiché altrimenti varierebbe la concentrazione di sostanze, il suo valore standard anche in presenza di forma allotropiche
è quello più stabile. Un valore negativo contribuisce alla stabilità dei composti rispetto alle sostanze 
elementari. L'Entalpia di trasformazione è indipendente dal cammino percorso ed è quindi una funzione di stato che
dipende solo dai parametri iniziali e finali (\textbf{legge di Hess}).

\subsection{Entropia standard di reazione}
Analogamente all'entropia, le condizioni standard ell'entropia assoluta sono convenzionalmente riferite ad 1 bar
e 25°. A differenza dell'entalpia standard delle sostanze elementari, l'\textbf{entropia assoluta} è diversa da zero. Essa
aumenta al diminuire della pressione e all'aumentare della temperatura. L'entropia assoluta è quindi definita
come la differenza tra l'entropia dei prodotti meno quella dei reagenti:
\begin{align}
    \Delta S_{reazione}^{o} = \sum_{i = 1}^{n} c_i S_{i(prodotti)}^{o} - \sum_{i = 1}^{n} c_i S_{i(reagenti)}^{o}.   
\end{align}
Come da teoria, l'entropia misurata delle sostanze gassose è maggiore della corrispettiva liquida. INoltre
l'entropia diminuisce aumentando la complessità della molecola e la forza dei legami.

\subsection{L'energia libera standard di una reazione}
Come nelle trasformazioni di fase, anche nelle trasformazioni liquide si ha variazione di \textbf{energia libera del sistema}
che determina se i reagenti si trasformano nei prodotti e viceversa. Il \textbf{principio di minima energia} dipende e fa
riferimento solo al sistema per cui $\Delta G < 0$ fino a che $\Delta G = 0$, analogamente l'energia
libera è data da:
\begin{align}
    \Delta G_{(reazione)}^{o} = \sum_{i =  1}^{n} c_i \Delta G_{fi(prodotti)}^{o}   - \sum_{i =  1}^{n} c_i \Delta G_{fi(reagenti)}^{o}   
\end{align}  
dove $\Delta G_{f}^{o}$ è l'\textbf{energia libera di trasformazione} di ciascun composto, partendo dalla  definizione di $\Delta H_{f}^{o}$.  \\
I Composti che hanno $\Delta G_{f}^{o} < 0$ sono stabili nel senso che non si dissociano spontaneamente nelle rispettive sostanze
elementari. 

\subsection{Energia libera di una reazione e temperatura}
Quando una reazione avviene a temperatura diversa da quella standard e con pressione di ciascun reagente e prodotto diversa da un bar
l'energia libera di reazione a condizione ambiente l'energia libera non è più standard ma le differenze di entropia ed entalpia
non sono grandi e quindi si possono considerare standard entro un certo intervallo di temperature:
\begin{align}
    \Delta G^{o} (T) = \Delta H^{o} - T\Delta S^{o}    
\end{align}

\section{L'aspetto fenomenologico dell'equilibrio}
\subsection{La costante di equilibrio come manifestazione dell'energia libera}
Si può dimostrare che la dipendenza dall'energia libera di una qualunque sostanza gassosa è data dalla seguente:
\begin{align}
    G(P, T) = G^{o}(T) + RT\ln(P) 
\end{align}
Poiché $\Delta G = \Delta H - T\Delta S$ allora si può scrivere:
\begin{align}
    \Delta G(T, P) = \Delta G^{o}(T) + RT \ln(Q) 
\end{align}
dove Q è il \textbf{quoziente di reazione} ossia il rapporto del prodotto delle pressioni dei prodotti ed il 
prodotto delle delle pressioni dei reagenti elevate al coefficiente stechiometrico alle condizioni standard:
\begin{align}
    Q = \frac{\prod_{i}^{prodotti} p_i^{c_i} }{\prod_{i}^{reagenti} p_i^{c_i}}
\end{align}
Il valore del quoziente di reazione all'equilibrio prende il nome di \textbf{costante di equilibrio} $K_p$:
\begin{align}
    \Delta G^{o}(T) = -RT \ln K_p
\end{align}
QUesta relazione ci permette di sapere quanto una reazione è spostata rispetto alla condizione di equilibrio
ed il valore di $\Delta G$ necessario per raggiungerlo.

\subsection{L'uso della costante di equilibrio ed i suoi formalismi}
La costante di equilibrio della reazione è esprimibile anche mediante la concentrazione della quantità di sostanza
utilizzando l'equazione di stato ideale: 
\begin{align}
    K_p (RT)^{-\Delta n} = K_c = \frac{\prod_{i}^{n}[Prodotti]^{c_i}_{eq}}{\prod_{i}^{n}[Reagenti]^{c_i}_{eq}} 
\end{align}
Il pedice eq sarà sottinteso in base al contesto della reazione. Il valore della costante di equilibrio dipende soprattutto
dal simbolismo utilizzato per scriverla: il valore di $K_{eq}$ dipende solo dalla temperatura ma non dal resto. \\
Un'altra grandezza utile è il \textbf{grado di dissociazione} di una sostanza: ossia il rapporto tra la quantità di sostanza
che siè dissociata all'equilibrio e quella presente dall'inizio della reazione, ossia il la \textbf{frazione molare} 
della sostanza che si dissocia:
\begin{align}
    \alpha = \frac{n_{A_{iniziale}} - n_{A_{eq}}}{n_{A_{iniziale}}} = \frac{c_{A_{iniziale}} - c_{A_{eq}}}{c_{A_{iniziale}}} = \frac{p_{A_{iniziale}} - p_{A_{eq}}}{p_{A_{iniziale}}}
\end{align}

\subsection{Gli effetti di perturbazione esterne sullo stato di equilibrio di una reazione: il principio di Le Chatelier-Braun}
Se uno dei parametri che determinano le condizioni di equilibrio cambiano in una reazione, allora il 
\textbf{principio di Le Chatelier-Braun} afferma che il sistema \textbf{si oppone alla modifica apportata} tendendo
a minimizzarne gli effetti cercando un nuovo stato di equilibrio. 


\chapter{Equilibri acido-base}
\section{Equilibrio chimico in fase liquida}
Per molte specie chimiche pure allo stato liquido si osserva un'equilibrio con altre specie chimiche
liquide, e molto interessanti sono gli equilibri con gli atomi di idrogeno, i quali possono passare
da una molecola all'altra secondo il processo di \textbf{autoprotolisi}, l'autoprotolisi più importante
è quella dell'acqua e la sua costante di equilibrio si scrive come:
\begin{gather*}
    K_{eq} = \frac{[\ce{H3O^{+} }][\ce{OH^{-} }]}{[\ce{H2O^}]^{2} }
\end{gather*}
Il prodotto ionico dell'acqua è dato come il prodotto delle specie al numeratore:
\begin{gather*}
    K_w = [\ce{H3O^{+} }][\ce{OH^{-} }] = 1,0 \cdot 10^{-14} 
\end{gather*}
In una soluzione con l'acqua come solvente e aumenta la quantità di \ce{H3O^{+}} allora la soluzione
assume un carattere acido, altrimenti se aumenta la quantità di \ce{OH^{-}} la soluzione assume un carattere basico.

\section{Equilibri acido-base}
\subsection{Equilibri secondo Lowry e Bronsted}
Si definisce acido qualunque sostanza che sia capace di cedere protoni e base una sostanza che riceve protoni. \\ 
Un'acido agisce come tale se e solo se c'è una base e viceversa ed il prodotto della loro reazione 
è sempre un acido + una base. La stessa reazione di autoprotolisi dell'acqua è una reazione di Lowry
poiché una molecola di acqua si comporta come acido e l'altra come base

\subsection{Reazione con l'acqua degli acidi e basi di Lowry-Bronsted}
L'equilibrio acido-base risponde alle leggi dell'equilibrio chimico le equazioni tipo acide e basiche sono:
\begin{gather*}
    \ce{HA + H2O <-> H3O^{+} + A^{-}  } \\
    \ce{B + H2O <-> HB^{+} OH^{-} }
\end{gather*}
Si misura con la legge d'equilibrio la tendenza di una reazione ad essere acida con la seguente costante:
\begin{align}
    K_a = \frac{[\ce{H3O^{+} }][\ce{A^{-}} ]}{[\ce{HA}]}
\end{align}
e per la base:
\begin{align}
    K_b = \frac{[\ce{OH^{-} }][\ce{HB^{+}} ]}{[\ce{B}]}
\end{align}
Le costanti sono utilizzate per misurare la \textbf{forza} di un acido o di una base ossia la loro
tendenza a cedere ioni \ce{H^{+}} all'acqua o da acquistarlo rispettivamente. 

\subsection{Il pH e pOH}
I simboli pH, pOH e pK indicano il logaritmo cambiato di segno di un numero uguale a quello che esprime 
la concentrazione di $mol \ dm^{-3}$ di \ce{H^{+}}, \ce{OH^{-}} e K. \\
Una soluzione neutra ha pH, pOH = 7(ossia $10^{-7}\ mol \ dm^{3}$ ), sono possibili sulla scala anche valori
di pH $<$ 0 e $>$ 14 anche se le leggi dell'equilibrio non sono più valide per soluzioni molto concentrate.
\begin{figure}
    \centering
    \includegraphics[width=0.8\textwidth]{pH.png}
    \caption{La scala del pH e del pOH delle soluzioni acquose}
    \label{Fig 12.1}
\end{figure}

\subsection{Livellamento della forza degli acidi e delle basi in acqua}
Tutti gli acidi con $K_a > 1$ danno origine alla stessa concentrazione di \ce{H3O^{+}} poiché
in acqua queste risultano \textbf{livellate } e quindi \ce{H3O^{-}} è l'acido più forte in acqua. 
Lo stesso vale per la base \ce{OH^{-}}. L'intervallo di pH di un solvente è chiamato dunque \textbf{
finestra di discriminazione acido-base}.

\subsection{Aspetti degli equilibri acido-base in acqua}
Nel caso di acidi e basi forti, il contributo dell'autoprotolisi  si calcola conosciuta la concentrazione di
dell'acido:
\begin{align}
    K_w = (c_{HA} + x)x
\end{align}
Dove x è la concentrazione di \ce{H3O^{+}} = \ce{OH^{-}}. Quando $c_{AH} = 1 \cdot  10^{-7}  $ allora il contributo
dell'autoprotolisi non è più trascurabile e va considerato nella formula poiché gli acidi e basi
forti reagiscono completamente con l'acqua. 
\begin{wrapfigure}{r}{0.4\textwidth}
    \centering
    \includegraphics[width=0.4\textwidth]{ph-pc.png}
    \caption{pH di soluzioni acide(rosso) e soluzioni basiche(blu) in funzione del cologaritmo 
    della concentrazione dell'acido o della base (pc)}
    \label{Fig 12.2}
\end{wrapfigure}
Quando siamo invece in presenza di acidi e basi deboli allora  $pH > pc_{HA}$ e lo stesso vale
per le basi. PEr questo si definisce il \textbf{grado di dissociazione} di un acido debole, ossia il
rapporto tra la quantità che è dissociata (che ha reagito con l'acqua) ossia la frazione molare
dell'acido o della base che all'equilibrio ha reagito con l'acqua:
\begin{align}
    \alpha = \frac{[\ce{H3O^{+}}]}{[acido]}
\end{align}
Un acido debole ha quindi come concentrazione:
\begin{align}
    K_a = \frac{x^{2} }{c_{HA} - z}
\end{align}
Dove x è la concentrazione di \ce{H3O^{+}}.

\subsection{Reazione fra $K_a$ e $K_b$ di una coppia coniugata acido-base}
Sciogliendo un acido forte in acqua ed aggiungendo la sua base coniugata, un acido è tanto forte
quanto è forte la reazione della sua base coniugata con l'acqua.
\begin{align}
    K_a K_b = K_w 
\end{align}

\subsection{Reazioni acido-base}
I valori di $K_a$ e $K_b$ indicano quanto è spostata la reazione e servono a trovare la costante di equilibrio di una qualsiasi reazione
acido-base, se le loro costanti sono $> 1$ allora hanno reagito con l'acqua in modo quantitativo allora 
si ha una costante di equilibrio molto grande $K_{eq} = K_w^{-1}$. Un acido reagisce effettivamente con qualunque
base che sia meno debole(ossia che abbia $K_b$ più grande) della propria base coniugata. Tanto maggiore
è il $K_b$ e più la reazione è spostata a destra ed in generale una reazione è spostata a destra se:
\begin{align}
    K_a K_b > K_w 
\end{align}
L'interazione tra un acido ed una base forte avvengono principalmente tra \ce{H3O^{+}} e \ce{OH^{-}} generati
in quanto queste specie reagiscono fortemente con l'acqua; se invece sono acidi o basi deboli, allora interagiscono
di più tra di loro poiché interagiscono meno con l'acqua (questo non influisce sul raggiungimento dell'equilibrio).

\subsection{Acidi poliprotici e basi poliacide}
Gli acidi che hanno la possibilità di cedere più di uno ione \ce{H^{+}} hanno tendenza decrescente a cedere all'acqua 
i protoni successivi al primo e si chiamano \textbf{poliprotici} e danno luogo a tanti equilibri con l'acqua quanti protoni
possono cedere (ognuno con un $K_a$ diverso tale che $K_{a_1} > ...  > K_{a_n}$), i quali esistono tutti simultaneamente 
in medesima concentrazione nella medesima soluzione. 
La costante \textbf{complessiva } di un acido poliprotico o di una base poliacida è data dal 
prodotto delle costatanti a cui possono dare vita:
\begin{align}
    K_c = K_{a1} \cdot  ... \cdot K_{an}
\end{align} 


\section{Applicazioni degli equilibri acido-base}
\subsection{Acidi e basi polifunzionali}
Facendo reagire l'acido \ce{H2SO4} con l'acqua si ottengono due reazioni; per cui chiamato
x la concentrazione di \ce{H3O^{+}} e \ce{HSO4^{-}}  proveniente dal primo equilibrio e con y la concentrazione proveniente dal secondo equilibrio,
allora si possono ottenere le concentrazioni
\begin{gather*}
    \left\{\begin{array}{c}
        K_{a_1} = \frac{[\ce{H3O^{+}}][\ce{HSO4^{-}}]}{[\ce{H2SO4}]} = \frac{(x + y)(x - y)}{c_{\ce{H2SO4}} - x} \\
        K_{a_1} = \frac{[\ce{H3O^{+}}][\ce{SO4^{2-}}]}{[\ce{HSO4^{-}}]} = \frac{(x + y)y}{x- y}
    \end{array}\right.
\end{gather*}
Se $K_{a_1} >> K_{b_1}$ allora si può utilizzare un altro metodo di calcolo per x ed y che dà un errore minore:
la concentrazione di \ce{H3O^{+}} è trascurabile al secondo membro e si può usare questo metodo anche quando
$K_{a_1}/K_{a_2}$ è dell'ordine di $10^{4}-10^{5} $. Lo stesso ragionamento e meccaniche si applicano anche per le basi. \\
Per quanto riguarda invece il calcolo del pH invece, vanno considerati entrambi i contributi (o più) degli acidi polifunzionali
e quindi:
\begin{align}
    pH \sim \frac{(pK_{a_1} + \dots + pK_{a_n}) }{n}
\end{align}  

\section{Le soluzioni tampone}
Se si mettono in soluzione un acido ed una base coniugati in concentrazioni comparabili, per esempio
mettendo in soluzione \ce{CH3CO2H} e \ce{NaCH3CO2} il sistema all'equilibrio deve soddisfare:
\begin{gather*}
    K_a = \frac{[\ce{H3O^{+}}][\ce{CH3CO2^{-}}]}{[\ce{CH3CO2H}]} = 1.8 \cdot 10^{-5} 
\end{gather*}
Poiché le concentrazioni simili e per la legge di Chatelier-Braun che tende a far regredire
i prodotti degli acidi e delle basi, e poiché in acqua le reazioni sono spostate a sinistra, allora si scrive:
\begin{gather*}
    [\ce{H3O^{+} }] = K_a \frac{[\ce{CH3CO2H}]}{[\ce{CH3CO2^{-}}]}
\end{gather*}
e quindi si arriva alla formula:
\begin{align}
    pH = pK_A - \log\left(\frac{[\ce{CH3CO2H}]}{[\ce{CH3CO2^{-}}]}\right)
\end{align}
Questo tipo di soluzione è chiamata \textbf{soluzione tampone}: l'aggiunta di una piccola quantità
di \ce{H3O^{-}} fa produrre una piccola quantità di acido coniugato alla base e viceversa una piccola aggiunta
di \ce{OH^{-}} fa produrre una piccola quantità di base coniugata facendo variare il pH di poco. \\
Essendo le concentrazioni all'equilibrio dell'acido e della base coniugati in un tampone non esattamente uguali alle concentrazioni
stechiometriche, sciogliendo quantità uguali di acido il pH varia da 7 a pK dell'acido scelto e quindi
\begin{align}
    \ce{H3O^{+}} = K_a
\end{align}
A seconda che il $pK_a$ sia minore o maggiore di 7, allora il sistema dovrà fornire (tampone acido) o
sottrarre (tampone basico) protoni all'acqua 


\section{Curve di distribuzione delle specie}
\begin{wrapfigure}{r}{0.4\textwidth}
    \centering
    \includegraphics[width=0.4\textwidth]{curve-specie-acidi.png}
    \caption{Curva di distribuzione della coppia acido-base considerata}
    \label{Fig 12.3}
\end{wrapfigure}
Considerando la distribuzione in soluzioni di specie costituenti una coppia coniugata quando il pH assume 
certi valori in seguito all'aggiunta di acidi o basi forti, e preso come esempio la coppia del paragrafo scorso,
si osserva che, con l'aggiunta di 1 mole per $dm^{3}$ di acido cloridrico, \ce{CH3CO2H} regredisce
ulteriormente grazie all'acido forte  per cui la dalla costante acida in presenza di una mole di \ce{H3O^{-}}:
\begin{gather*}
    K_a = \frac{[\ce{CH3CO2^{-}}]\cdot 1.0}{1.0 \cdot  10^{-2}-[\ce{CH3CO2^{-}}]}
\end{gather*}
Si ricava che il grado di dissociazione dell'acido è $\alpha = 1.8 \cdot  10^{-5}$ a pH = 0. Aumentando il pH
mano a mano incrementa la frazione di acido dissociato fino a raggiungere il massimo di x = 1 a pH 7.  


\section{Acidi e basi di Lewis}
\subsection{Definizione di acidi e basi di Lewis}
Gli ossidi dei non metalli possiedono proprietà acide nonostante non corrispondano alla definizione di acido di Lowry,
queste sostanze sono chiamate \textbf{acidi di Lewis}: sono delle specie che possono accettare in compartecipazione
una coppia di elettroni da un'altra specie; sono basi invece le specie che in una reazione possono ceder
una coppia di elettroni ad un'altra sostanza. L'ossidazione di un metallo è una reazione acido-base che
per Bronsted-Lowry non lo è. 

\subsection{Gli equilibri coinvolgenti ioni complessi}
Quando si aggiunge un legante ad una soluzione che contiene uno ione metallico, si ha un equilibrio
che dipende dalla costante di equilibrio che in questo caso è chiamata \textbf{costante di stabilità} del complesso
e la costante relativa alla reazione inversa di dissociazione dello ione complesso è chiamata invece
\textbf{costante di instabilità} (ossia l'inversa).

\chapter{Sistemi a più componenti e equilibri eterogenei}
\section{Equilibri eterogenei}
\subsection{Equilibri eterogenei implicanti gas}
Nelle reazioni \textbf{eterogenee} si ha una costante di equilibrio data dal rapporto tra i prodotti ed i reagenti. Tuttavia
essendo che le sostanze solide non mutano la costante di equilibrio che coinvolge la reazione sarà sempre il rapporto
tra i reagenti ed i prodotti ma senza le componenti solide. 


\section{Dissoluzione di un soluto in solvente}
\subsection{Il meccanismo della solubilizzazione dei composti ionici }
Questo processo si basa sulle proprietà strutturali del composto ionico e del solvente: dissolvendo
\ce{NaCl} in acqua, ogni atomo di Na ed ogni atomo di \ce{Cl} sono circondati da sei molecole d'acqua
secondo il processo di \textbf{solvatazione} (\textbf{idratazione} in questo caso). Dal punto di vista
termodinamico il processo è molto più complicato perché in adesso non ci si spiega come possa la solubilità
aumentare esponenzialmente all'aumentare della temperatura.

\subsection{Termodinamica della solubilizzazione di una sostanza}
La dissoluzione di un composto ionico in un solvente è schematizzata con il \textbf{ciclo di Hess}:
distruzione del reticolo cristallino (con passaggio allo stato gassoso degli ioni) e idratazione degli ioni gassosi. \\
Il passaggio allo stato gassoso porta alla \textbf{dissociazione} ($\Delta H_{diss} = \Delta H_{ret}$)  mentre al secondo
stadio l'entalpia di idratazione è negativa che è circa uguale all'energia potenziale fra gli ioni e le molecole d'acqua
che non porta variazione di lavoro né di energia cinetica. \\
L'\textbf{Entalpia di solubilizzazione} $\Delta H_{sol}$ è data da:
\begin{align}
    \Delta H_{sol} = \Delta H_{ret} + \Delta H_{idr}
\end{align}
Mentre $\Delta H_{ret}$ è costante durante tutto il processo, il l'altro termine diminuisce in valore
assoluto diventando meno favorevole anche se non è mai abbastanza grande da non permettere la solubilizzazione e per
questo qualsiasi processo di solubilizzazione è \textbf{endoentalpico o endotermico}: Agitando una soluzione,
l'energia cinetica viene impiegata nella solubilizzazione portando ad un abbassamento della temperatura della soluzione. \\
La capacità di disciogliere un soluto in un solvente è invece determinata dal contributo entropico: il quale è molto 
grande nel processo di dissoluzione e gli ioni vanno in soluzione fino a che $T \Delta S > \Delta H$. Mano a mano 
$\Delta S$ diminuisce sempre di più fino a che $\Delta H_{sol} - T \Delta S = 0 $: il sistema raggiunge l'equilibrio 
e la soluzione è \textbf{satura}. \\
La solubilità di un solido ionico in un solvente dipende quindi l'entalpia relativa alla distruzione
dei legami, di solvatazione e di dissoluzione giocano un ruolo fondamentale nel passaggio dallo stato solido a quello gassoso. 
Il termine $\Delta S$ aumenta all'aumentare della temperatura e quindi determina una solubilità maggiore.

\subsection{Equilibri eterogenei in soluzione: il prodotto di solubilità}
La costante di equilibrio prende il nome di \textbf{prodotto di solubilità} ed  è una funzione della sola temperatura 
e la quantità di solido, come in ogni equilibrio eterogeneo non influisce sulla posizione di equilibrio ed il prodotto
di solubilità è costante ad ogni temperatura (è il prodotto di degli ioni elevati ciascuno al coefficiente stechiometrico).

\subsection{La dissoluzione di composti molecolari liquidi e solidi}
Il meccanismo di dissoluzione dei composti molecolari è diverso da quelli ionici: l'interazione
fra le molecole è di Van der Waals e quindi se è liquido-liquido l'entropia fà si che soa solubile illimitatamente
così come nelle interazioni gas-gas ma solo se i loro valori di $\Delta H_{sol}$ sono piccoli. Se i liquidi 
sono costituiti da molecole strutturalmente diverse, allora sono immiscibili. Questo perché il contributo
favorevole ma negativo di $\Delta H$ dovuto alle interazioni soluto-solvente compensa solo in minima parte il contributo
positivo ma sfavorevole di $\Delta H$ dovuto alla perdita soluto-soluto e solvente-solvente nella soluzione. \\
Se il soluto è un solido molecolare la solubilità ha un certo limite.

\subsection{Solubilità di solidi covalenti polimeri e metalli}
I solidi covalenti polimeri hanno grande entalpia di reticolare permette ai solidi di essere immiscibili così come
i metalli nei liquidi (sono miscibili solo nel mercurio nel processo di \emph{amalgama} il quale avviene anche tra solidi).

\subsection{Solubilità di composti con legame a ponte idrogeno}
Sostanze liquide in cui agiscono legami ad idrogeno sono fra loro miscibili spesso in tutte le proporzioni quando l'entropia
di mescolamento supera la piccola entalpia. 

\subsection{Solubilità dei gas}
Un gas non ha forze di interazione apprezzabili, quindi il processo di solubilizzazione di un gas è generalmente favorito
dall'entalpia di solvatazione e la solubilizzazione della maggior parte dei gas è in processo 
\textbf{esoentalpico o esotermico}. L'entropia non favorisce la dissoluzione di un gas in un liquido poiché il gas
passerebbe ad uno stato di disordine minore e inoltre potrebbe infilarsi nelle cavità del liquido dando ancora più 
ordine alla soluzione. Secondo la \textbf{legge di Henry} la solubilità dei gas è direttamente proporzionale alla
loro pressione parziale. Con l'aumentare della temperatura diminuisce la solubilità di un gas a causa dell'aumento dei
microstati e quindi dell'aumento di entropia. 

\subsection{Effetto della temperatura sulla solubilità}
Per prevedere l'effetto dell'aumento della temperatura sulla solubilità occorre sapere se il processo di solubilizzazione
all'equilibrio è endoentalpico o esoentalpico: a tale scopo si mette altro solvente e sciogliere altro soluto. 
Se la temperatura si innalza allora è endoentalpico, altrimenti è esoentalpico. \\
Se il processo di solubilizzazione all'equilibrio è esoentalpico come nei gas, allora alla diminuzione
dell'energia cinetica provoca altra solubilizzazione dei gas, allora si genera calore che si oppone a questa dissoluzione. 

\subsection{Ripartizione di un soluto fra solventi immiscibili}
Quando una soluzione è aggiunta in un recipiente con un altro solvente immiscibile e viene agitata il soluto si ripartisce fra le due
fasi liquide e la concentrazione sarà maggiore nel solvente in cui la sua solubilità è maggiore. Il rapporto tra le concentrazioni
del soluto nelle due soluzione è uguale al rapporto fra le solubilità è chiamato \textbf{costante di ripartizione} del soluto
fra due solventi. 

\section{I diagrammi di stato a due componenti}

\subsection{Il diagramma \ce{H2O}-\ce{NaCl}}
\begin{wrapfigure}{r}{0.4\textwidth}
    \centering
    \label{Fig 13.1}
    \caption{Diagramma di stato Acqua-sale}
    \includegraphics[width=0.4\textwidth]{Acqua-sale.png}
\end{wrapfigure}
Il diagramma di equilibrio solido-solido alla pressione di un'atmosfera del sistema acqua sale è dato
dalla figura \ref{Fig 13.1}.
I punti della curva EB rappresentano le temperature di equilibrio tra la fase liquida ed il solido
formato da \ce{NaCl} puro e quindi il tratto EB indica la dipendenza della solubilità ed il punto
E è chiamato il \textbf{punto eutettico} e la temperatura e la composizione corrispondenti sono la
\textbf{temperatura eutettica e composizione eutettica} e rappresenta il punto di equilibrio fra la soluzione,
e le due fasi solide costituite da ghiaccio e cloruro di sodio. 

\section{I sistemi dispersi}
\subsection{Le dispersioni colloidali}
Un \textbf{sistema omogeneo} è definito come un sistema all'interno del quale le proprietà chimiche e fisiche costanti 
per ogni volume piccolo a piacere. La \textbf{sospensione } di una sostanza insolubile in un dato solvente costituisce 
un \textbf{sistema eterogeneo}. Il passaggio da sospensione eterogenea a soluzione è idealmente 
un processo continuo: è possibile avere sospensioni o aggregati microscopici. In questo caso si parla
\textbf{dispersioni colloidali}. \\
La fase disperdente può essere gassosa e liquida e la fase dispera solida, liquida o anche gassosa come i fumi o
gli aerosol più comuni e sono elettricamente cariche respingendosi tra di loro. \\
L'energia potenziale delle particelle che riescono ad aggregarsi ha energia minore se è di forma
sferica. 

\subsection{Sol e gel}
Le dispersioni di particelle di sostanze solide in liquidi si chiamano sol. Se si sala un sol cioè se si 
aggiungono ioni positivi e negativi di un composto ionico le particelle micelle si scarico e quindi avviene
la \textbf{flocculazione} o precipitazione della fase dispersa. \\
Le micelle di alcune sostanze specifiche formano una struttura tra di loro che occupa tutto il volume
occupato prima dai sol e la fase disperdente è intrappolata al suo interno prendendo il nome di \textbf{gel}. 
Raffreddando un sol si ottiene un materiale liofilizzato che è simile ad un gel ma molto più fragile . Le proprietà dei
colloidi invece sono molto peculiari in quanto sono diverse dai solidi in sospensione poiché rimangono
omogeneamente distribuiti nella soluzione e possono rifrangere la luce secondo l'effetto \textbf{Tyndal}.

\subsection{Emulsioni e schiume}
Nelle \textbf{emulsioni} un liquido è disperso in un'altro liquido. Generalmente queste emulsioni
tendono a stratificarsi dopo molto tempo ma con l'aggiunta di una sostanza \textbf{tensioattiva}  perché hanno
la proprietà di abbassare la tensione superficiale di un liquido. COn la diminuzione della tensione superficiale
le sostanze tensioattive possono formare una \textbf{schiuma} sulla superficie del liquido

\section{Le soluzioni solide}
Esistono anche le soluzioni solide (spesso ottenute mediante raffreddamento di soluzioni liquide):
nei metalli esse si chiamano \textbf{leghe}, le quale sono formate da molecole di metalli diversi all'apparenza
compatte. Un'altro tipo di soluzione solida è un atomo o ione sostituisce un altro simile all'interno di un
cristallo di un composto.  

\chapter{Cinetica chimica}
\section{Aspetti termodinamici e cinetici di una reazione}
I parametri $\Delta G^{o}$ o $K_{eq}$ di una reazione permettono di decidere se una reazione avviene o non avviene spontaneamente
e quali sono le condizioni sperimentalmente che la favoriscono. L'aspetto termodinamico di una reazione non è tuttavia la sola
proprietà che è necessario conoscere: la \textbf{velocità} è determinata dal'\textbf{aspetto cinetico}. Tutto dipende
dalla costante di equilibrio: se essa è molto grande allora una reazione può essere molto lenta, altrimenti molto veloce. 

\section{La velocità di una reazione}
\subsection{La misura della velocità di reazione}
La velocità di reazione è espressa come:
\begin{align}
    v_R(t) = \frac{d[R]}{dt}
\end{align}
Tutti i reagenti all'interno di una reazione hanno la stessa velocità di reazione percui in una reazione 
generica del tipo \ce{aA + bB <-> cC + dD} la sua velocità al tempo t sarà data da:
\begin{gather*}
    v_R = -\frac{1}{a} \frac{d[A]}{dt} = -\frac{1}{b} \frac{d[B]}{dt}= \frac{1}{c} \frac{d[C]}{dt}= \frac{1}{d} \frac{d[D]}{dt}
\end{gather*}
I reagenti hanno velocità negativa poiché diminuiscono mentre i prodotti positiva poiché vanno ad aumentare

\section{Il meccanismo delle reazioni elementari}
\subsection{Teoria delle collisioni ed energia di attivazione}
\begin{wrapfigure}{r}{0.4\textwidth}
    \centering
    \label{Fig 14.1}
    \caption{Variazione dell'energia potenziale tra due molecole AB e C}
    \includegraphics[width=0.4\textwidth]{energia-attivazione.png}
\end{wrapfigure}
Il modello microscopico che descrive l'andamento di una reazione è quello della \textbf{toeria delle collsisioni} che si basa
sull'assunzzione che per reagire e dar luogo a dei prodotti dei reagenti collidano tra di loro. Questa però 
è solo una condizione necesaria ma non sufficiente e la quantità di urti che si hanno si calcola mediante la
teoria cinetica dei gas e solo una frazione è effettivamente utile per la generazione di prodotti (\textbf{urti efficaci})
poiché possiedono l'energia minima per far avvenire la reazione chiamata \textbf{energia di attivazione}.
Facendo avvicinare due molecole AB e C si ottiene il grafico in figura \ref{14.1}. Durante l'avvicinamento
si arriva allo \textbf{stato di transizione} in cui si forma un debole legame tra B e C ed in cui il legame tra A e B
si è indebolito  dando quindi luogo alla reazione:
\begin{gather*}
    \ce{AB + C <-> A + BC}
\end{gather*}
Solo le particelle che hanno $E_{cin} > E_{a+}$ possono dare luogo alla reazione, le altre non reagiscono a meno 
che non si fornisca calore al sistema. \\
Dalla teoria delle collisioni si ricava la velocità di reazione chiamata \textbf{legge cinetica}, la quale è $\propto$ 
al numero degli urti tra le molecole, la quale è $\propto$ alla concentrazione delle particelle. 

\subsection{Costante di velocità ed energia di attivazione}
La costante di proporzionalità, $k_+$ è definita come \textbf{costante cinetica} o costante di velocità della reazione
e tiene conto della frazione di molecole che possiedono quell'energia cinetica di attivazione della reazione, la quale
aumenta all'aumentare della temperatura (a causa di una maggiore agitazione delle molecole).\\
Reazioni che avvengono con solo uno stadio(monostadio) (ossia senza trasformazioni intermedie) prendono il nome di \emph{bimolecolari}
e si dice anche che la loro \emph{molecolarità} sia di 2. La dipendenza della costante di velocità dall'energia di attivazione è data dalla:
\begin{align}
    k_+ = Ae^{-\frac{E_{a+}}{RT}} 
\end{align}
Dove A è il \textbf{fattore di frequenza} che dipende dalla frazione di urti favorevoli rispetto al totale. 

\subsection{Velocità della reazione inversa}
Presa come esempio una reazione monostadio generica con molecolarità 2:
\begin{gather*}
    \ce{A + B <-> C + D}
\end{gather*}
L'espressione della velocità per la reazione diretta è data dall'equazione differenziale di primo ordine:
\begin{gather*}
    -\frac{d[A]}{dt} = k_+ [A][B]
\end{gather*}
Se $K_{eq}$ è molto grande allora la velocità segue la stessa legge fino alla fine, altrimenti
diventa importante anche la reazione inversa  per cui la velocità di produzione di A diventa:
\begin{gather*}
    \frac{d[A]}{dt} = -\frac{d[C]}{dt} = k_-[C][D]
\end{gather*}
dove $k_-$ è data da:
\begin{align}
    k_- = Ae^{-\frac{E_{a-}}{RT}} 
\end{align}
con A che  è il fattore di frequenza della reazione e in questo caso si avra che $E_{a-} > E_{a+}$ poiché
ci sarà da superare una barriera energetica molto più forte  per ottenere i reagenti dai prodotti.
L'espressione generale della velocità di reazione durante tutta la reazione è dunque:
\begin{align}
    -\frac{d[A]}{dt} = k_+[A][B] - k_-[C][D]
\end{align}

\subsection{Relazione tra costanti cinetiche e costante di equilibrio termodinamico}
Se si sostituisce all'energia potenziale l'energia libera del sistema  il grafico risultante(analogo a quello dell'energia potenziale)
ci illustra bene la relazione fra l'aspetto cinetico e termodinamico delle reazioni. Le costanti di velocità
possono essere espresse in funzione dell'energia libera di attivazione si ottiene come?????

\section{Meccanismi delle reazioni a due o più stadi: i processi elementari}
Considerata nuovamente la reazione generica, se la reazione è molto spostata a destra si può trascurare il contributo
della reazione inversa e che A non possa reagire con B ma che debba prima trasformarsi in I; allora si tratta di una
reazione a due stadi in cui il primo è più lento perché ha un'attivazione maggiore del secondo e quindi la velocità di reazione
dipende interamente dalla velocità di formazione di I. 

\section{Molecolarità dei processi elementari ed ordine di reazione}
\subsection{Ordine di reazione}
Generalmente, data una reazione del tipo:
\begin{gather*}
    \ce{aA + bB <-> cC}
\end{gather*}
la legge cinetica può essere esperessa dall'equazione
\begin{gather*}
    -\frac{1}{a}\frac{d[A]}{dt} = k_+ [A]^{x}[B]^{y}  
\end{gather*}
queste costanti x ed y esprimono la \textbf{molecolarità dello stadio lento} e coincidono
con i coefficienti stechiometrici dei reagenti che partecipano allo stadio lento:  esse rappresentano
l'\textbf{ordine della reazione} rispetto al componente A e al componente B; la somma x + y rappresenta l'ordine 
complessivo della reazione che può essere espresso da un numero intero o frazionario indipendente per ogni reazione.\\

\subsection{Immaginare un meccanismo di reazione}
Misurate sperimentalmente la velocità di reazione iniziale a diverse concentrazioni, 
si può spiegare ogni reazione che avviene semplicemente supponendo che avvengano delle mini reazioni
al suo interno per portare il prodotto finale. 

\subsection{La costante di equilibrio della reazione nelle reazioni multistadio}
La velocità delle reazioni è in realtà la risultante della velocità della reazione diretta meno quella inversa, 
per cui il rapporto tra la velocità della reazione diretta e la reazione inversa è sempre costante durante tutta la reazione.
Inoltre se la reazione si può scomporre in più reazioni si può scrivere:
\begin{align}
    \frac{k_{+_{1}}}{k_{+_{1}}}\cdot \dots \cdot \frac{k_{+_{n}}}{k_{+_{n}}} = K_{eq}
\end{align}

\section{Le leggi cinetiche integrate delle reazioni del primo e del secondo ordine}
\subsection{Cinetiche del primo ordine}
Per conoscere la dipendenza dal tempo delle concentrazioni delle specie che prendono parte ad una reazione occurre integrare l'equazione
per cui si ottiene per una reazione generica integrata: 
\begin{align}
    -\frac{d[A]}{dt} = k_+[A] \Rightarrow  -\log \left(\frac{[A]_t}{[A]}\right) = k_+t
\end{align}

\subsection{Cinetiche del secondo ordine}
Dalla reazione del secondo ordine:
\begin{gather*}
    \ce{A + B -> prodotti}
\end{gather*}
la legge cinetica ci da:
\begin{gather*}
    -\frac{d[A]}{dt} = k[A][B] 
\end{gather*}
L'integrazione nel caso in cui [A] = [B] perché si ha che:
\begin{align}
    \int_{[A]_0}^{[A]_t} \frac{d[A]}{[A^{2} ]} = -kt \quad \Rightarrow \quad \frac{1}{[A]_t}-\frac{1}{[A]_0} = kt 
\end{align}
Nel caso in cui $[A] \neq [B]$ allora si ha:
\begin{align}
    \frac{1}{[B]_0 -[A]_0} \log \left(\frac{[B][A]_0}{[A][B]_0}\right) = kt
\end{align}


\section{Controllo cinetico delle reazioni}
\begin{wrapfigure}{r}{0.4\textwidth}
    \centering
    \label{FIg 14.2}
    \includegraphics[width=0.4\textwidth]{Controllo-cinetico.png}
\end{wrapfigure}
Se due reagenti  danno due prodotti diversi mediante la stessa reazione la dipendenza da energia libera in funzione 
della coordinata di reazione è diversa e perciò i prodotti hanno stabilità diversa. Se uno dei due prodotti richiede un
energia di attivazione maggiore, allora la velocità di reazione può controllare quale dei prodotti si formerà: se
il prodotto D può formarsi solo da A e B, per velocità basse si formerà sempre e solo C e una quantità trascurabile di D.

\section{Reazioni in soluzione e reazioni eterogenee}
\subsection{Il meccanismo delle reazioni in soluzione: l'effetto del solvente}
Se si fa avvenire una reazione all'interno di un solvente, allora la velocità di reazione è determinata
a partire dalla libertà di movimento dei reagenti all'interno del solvente per poter ottenere degli urti efficaci:
in acqua la reazione più efficace e più veloce è proprio tra \ce{H3O^{+}} e \ce{OH^{-}}.

\subsection{Le reazioni eterogenee}
Analogamente alle reazioni omogenee, nelle reazioni eterogenee il meccanismo di reazione porta alla collisione
delle molecole in fase gassosa o liquida con quelle del solido. In questo caso la velocità di reazione dipende
sia dalla superficie del solido che dalla possibilità che i prodotti lascino la superficie del solido
per poter fare altri urti efficaci, oltre che da fattori più complicati come il grado di suddivisione
del solido e la diffusione dei reagenti e prodotti. 

\section{Catalisi}
\subsection{I catalizzatori}
Generalmente per aumentare la velocità di reazione si può incrementare la temperatura anche se questo non sempre è
applicabile per due motivi: in primis limiti tecnologici e termodinamici ed in secondo luogo alcune reazioni che
sono esoentalpiche l'equilibrio si sposterebbe a sinistra. Il metodo generale per aumentare la velocità di reazione
è mediante l'impiego di un \textbf{catalizzatore}, ossia una sostanza che aggiunta alla reazione permette di formare dei
prodotti intermedi, velocizzando la reazione. La sostanza che fa il contrario si chiama invece \textbf{inibitore}.
Se il catalizzatore si trova in fase coi reagenti e prodotti si parla di \emph{catalisi omogenea}, altrimenti \emph{catalisi eterogenea}.

\subsection{Catalisi eterogenea}
Nella catalisi eterogenea il catalizzatore si trova in fase solida mentre i reagenti e prodotti e reagenti in fase liquida 
o gassosa. Durante la reazione si forma un composto intermedio fra i reagenti ed alcune parti specifiche chiamati \textbf{centri attivi}.
In genere gli atomi sulla superficie del cristallo sono molto reattivi poiché hanno numero di coordinazione
ridotto rispetto agli altri. Maggiore è la superficie del solido, maggiore è la velocità di reazione. \\
Alla fine di ogni processo la superficie del catalizzatore è ripristinato e non figura quindi tra i prodotti.

\subsection{Catalisi enzimatica}
Negli esseri viventi i catalizzatori sono chiamati \textbf{enzimi}, senza gli enzimi la vita non potrebbe esistere
poiché le reazioni necessarie per la vita sarebbero troppo lente. 

\section{Reazioni fotochimiche}
Si parla di reazioni fotochimiche quando intervengono radiazioni elettromagnetiche. Le radiazioni possono
permettere la rottura di un legame dando vita a frammenti di molecole chiamati \textbf{radicali} in cui sono
presenti elettroni spaiati, altamente reattivi. Quando una reazione forma dei prodotti che possono reagire coi
reagenti e formare ancora più prodotti allora si dice che è una reazione a catena. 


\chapter{Elettrochimica}
\section{Celle Voltaiche}
\subsection{Celle volatiche e le reazioni di ossidoriduzione}
I sistemi elettrochimici trasformano energia chimica in energia elettrica sono chiamati \textbf{celle voltaiche o galvaniche}
ed il processo inverso avviene nelle \textbf{celle elettrolitiche}. Le \textbf{reazioni di ossidoriduzione} sono alla base
di ambedue i sistemi elettrochimici. Introducendo una lamina di zinco in una soluzione con ioni \ce{Cu^{2+}} la lamina si
ricopre di rame metallico secondo la reazione:
\begin{gather*}
    \ce{Cu^{2+} + Zn -> Zn^{2+} + Cu} 
\end{gather*}
Questa reazione è spontanea e produce energia termica se sono nello stesso contenitore ma si possono collegare
i due recipienti contenenti le due lamine. 

\subsection{Il funzionamento delle celle volatiche}
L'energia elettrica è il prodotto tra la carica per il potenziale:
\begin{align}
    C \times V = J
\end{align}
Congiungendo due contenitori contenuti una lamina di zinco ed una di rame ed immersi in una soluzione
salina \ce{Na2SO4} che fa da ponte salino si ottiene la pila di Daniell. che sul filo conduttore
delle due lamine legge una $\Delta V = 1,10 V$. Grazie al ponte salino le soluzioni rimangono neutre
in modo da far avvenire solo il passaggio di elettroni che altrimenti non avverrebbe. \\
Le due lamine sono \textbf{elettrodi}, il \textbf{catodo} è dove avviene la riduzione e l'\textbf{anodo} dove
avviene l'ossidazione. L'anodo è sempre \textbf{l'elettrodo negativo} se vi avviene l'ossidazione(che libera elettroni)
mentre il catodo è \textbf{l'elettrodo positivo} dove avviene la riduzione che sottrae elettroni.  \\
La \textbf{forza elettromotrice} o \textbf{f.e.m.} è la differenza di potenziale tra i due elettrodi
ossia la differenza tra il potere riducente e quello ossidante, calcolato a livello standard con l'elettrodo standard
ad idrogeno. 

\section{I potenziali di elettrodo}
\subsection{I potenziali standard di riduzione}
La cella formata da zinco e dall'elettrodo standard ad idrogeno è rappresentata come:
\begin{gather*}
    \ce{Zn|Zn^{2+}||H^{+}|H2(Pt)}
\end{gather*}
La differenza di potenziale di questa pila a 298K è di $0.76 V$ ed il suo valore misura la tendenza
complessiva della reazione:
\begin{gather*}
    \ce{Zn  +  2H^{+}  -> Zn^{2+} + H2}
\end{gather*}
Il \textbf{potenziale standard di elettrodo} è quello relativo alla \textbf{semireazione di riduzione} di ossidazione
o riduzione che avviene nei confronti dell'elettrodo standard ad idrogeno ed il potenziale può essere sia positivo
che negativo. \\
Il \textbf{potenziale standard di riduzione } si ottiene sottraendo la differenza di potenziale dell'anodo a quella del catodo.

\subsection{Dipendenza dei potenziali d'elettrodo dalle concentrazioni}
Come variano i potenziali di elettrodo in funzione dei parametri che definiscono il sistema? 
Il potenziale d'elettrodo dipende dalle concentrazioni di tutte le specie che compaiono nella semireazione
e si calcola secondo l'\textbf{equazione di Nernst}:
\begin{align}
    E = E^{o} - \frac{RT}{zF}\log \left( \frac{\prod [prodotti]}{\prod [reagenti]}\right).
\end{align}
dove z è il numero di elettroni, F la costante di Faraday (carica di una mole di elettroni).\\
L'energia in gioco nelle reazioni di ossidoriduzione è determinata da diversi contributi difficilmente
prevedibili per cui i potenziali standard non hanno andamento periodico 

\section{Il potenziale standard delle celle voltaiche e l'equilibrio chimico }
In una pila di Daniels, mano a mano che fluiscono gli elettroni l'agente ossidante acquisice sempre
più elettroni e l'agente riducente ne rilascia sempre di più fino a che ad un certo punto la tensione
inizia a diminuire fino a che non arriva a zero quando tutto l'agente ossidante è ridotto e quando
l'agente riducente è completamente ossidato. Da questo ragionamento seguono le seguenti relazioni:
\begin{align}
    K_{eq} =& e^{\frac{zF\Delta E^{o} }{RT}} \\
    K_{eq} =& e^{\frac{\Delta G ^{o} }{RT}} \\
    \Delta G^{o} =& -zF\Delta E^{o}     
\end{align} 

\section{Le reazioni elettrochimiche che coinvolgono alcuni metalli}
\subsection{L'attacco dei metalli}
La reazione di attacco dei metalli è la seguente:
\begin{gather*}
    \ce{M -> M^{z+} + ze^{-} }
\end{gather*}
Qualunque ossidante con potenziale standard apprezzabilmente maggiore del potenziale della semireazione
\begin{gather*}
    \ce{M^{z+} +ze^{-} -> M }
\end{gather*}
è capace di ossidare il metallo. I metalli che hanno un potenziale $E^{o} > 0$ e non sono dunque
attaccabili dall'idrogeno sono chiamati metalli nobili o seminobili (oro e platino sono i migliori esempi).  

\subsection{Corrosione e protezione dei metalli}
La corrosione dei metalli avviene per ossidazione superficiale da parte degli agenti atmosferici
ed in particolare dell'ossigeno che ha potenziale $1.23V$. Tutti i metalli dovrebbero ossidarsi all'aria
(tranne l'oro) ma ciò non avviene completamente: questo perché molti di questi formano uno strato di
ossido e in questo modo si autoproteggono attraverso la \textbf{passivazione}. \\
L'umidità dell'aria ed il contatto con un metallo più nobile favoriscono il processo di attacco: in acqua
avvengono le stesse condizioni che avverrebbero in una cella galvanica: il metallo più nobile
funziona da elettrodo su cui si riduce l'ossigeno ed il metallo meno nobile funziona da elettrodo solubile (anodo). \
Si protegge infatti il ferro avvolgendolo con un materiale meno nobile rispetto al ferro come lo Zinco in modo
che si ossidi quest'ultimo e non il ferro. Questa protezione prende il nome di \textbf{protezione catodica}.  

\section{I potenziali standard ed alcuni equilibri particolari}
\subsection{Il potenziale di elettrodo in condizioni lontane da quelle standard}
Quando un elettrodo non si trova in condizione standard, allora il suo potenziale può variare sensibilmente
rispetto a quello determinato standard: l'ambiente acido o basico possono introdurre variazioni del potenziale considerevoli. \\
Si utilizza quindi un elettrodo a potenziale noto e costante ossia l'\textbf{elettrodo a calomelano}-

\subsection{Reazioni di dismutazione}
Molte specie chimiche possiedono più di due numeri di ossidazione, le reazioni in cui una specie di numero
di ossidazione intermedio si trasforma in un numero di ossidazione maggiore o minore con reazioni di \textbf{dismutazione}.

\section{Cella a concentrazione}
L'equazione di Nernst fa prevedere che si possano avere delle reazioni agli elettrodi anche quando specie
reagenti sono le stesse:
\begin{gather*}
    \ce{Cu -> Cu^{2+} + 2e^{-}  } \\
    \ce{Cu^{2+} + 2e^{-} -> Cu}
\end{gather*}
E' una cella galvanica a bassa tensione che smette di erogare corrente elettrica nel momento in cui la quantità
di \ce{Cu^{2+}} è la stessa nelle due semicelle.

\section{I potenziali standard fanno prevedere i prodotti delle reazioni di ossidoriduzione}
Come nelle celle galvaniche, anche nelle ossidoriduzioni avviene la riduzione della semireazione con
$E^{o}$ maggiore e l'ossidazione nel senso opposto: nelle seguenti
\begin{gather*}
    \ce{MnO4^{-} +  8H^{+} + 5e^{-} -> Mn^{2+} + 4H2O} \\
    \ce{Fe^{3+} + e^{-} -> Fe^{2+} } 
\end{gather*} 
La seconda reazione non può avvenire perché ha $E^{o}$ ha un potenziale minore e quindi
deve avvenire per forza:
\begin{gather*}
    Fe^{2+} -> Fe^{3+} + e^{-} 
\end{gather*} 
Ed il potenziale di riduzione garantisce la sponteneità della reazione che è spostata a destra. 

\section{Elettrolisi}
\subsection{Le celle galvaniche e l'elettrolisi}
Una cella eroga energia con differenza di potenziale decrescente fino ad esaurimento di una della specie. 
Se ad una pila si applica una differenza di potenziale uguale a quella di erogazione allora non avverrà nulla 
mentre se viene ne applicata una maggiore allora la reazione procede nel senso opposto e quindi fa avvenire
la reazione di ossidoriduzione al contrario. La cella dove avviene questo è chiamata \textbf{cella elettrolitica}
ed il fenomeno è chiamato \textbf{elettrolisi}. Il segno dei poli nelle celle galvaniche è l'effetto delle reazioni
spontanee che avvengono.

\section{Sovratensione}
La differenza fra il potenziale d'elettrodo calcolato in base ai potenziali standard e quello
necessario per far avvenire l'elettrolisi è chiamato \textbf{sovratensione}. La sovratensione
aumenta all'aumentare della velocità con cui la reazione avviene.

\section{La conduzione elettrica nelle soluzioni di elettroliti}
La conduzione elettrica in una soluzione acquosa è prodotta dal movimento degli ioni positivi e neagitivi
di un sola o di una ssotanza che in acqua si dissocia in ioni e qundin conducono corrente elettric. \\
La \textbf{conduttività molare} $V_M$ è la conducibilità di un volume di soluzione che contiene 
una mole di elettrolita. E' una caratteristica di ciascun elettrolità ad ogni temperatura e concentrazione.

\chapter{Reazioni nucleari}
\section{Il decadimento radioattivo}
\subsection{I modi di decadimento naturale}
La distribuzione normale dei nuclidi stabili costituisce una fascia che segue la bisettrice del grafico in cui si riporti Z sulle
ascisse e N sulle ordinate per nuclidi leggeri . I nuclidi instabili iniziano dopo il numero Z del piombo
o anche gli isotopi di elementi prima del piombo (deuterio e trizio)- In base al principio di minima energia
si prevede che un nuclide instabile subisca qualche trasformazione che lo porti ad essere più stabile.
Queste trasformazioni si chiamano \textbf{decadimenti radioattivi}, e sono provessi medianti i quali un nuclide
si trasforma spontaneamente in un altro liberando energia sottoforma di radiazione elettromagnetica.
\begin{enumerate}
    \item \textbf{Emissione di particelle $\alpha$}: Le particelle $\alpha$ sono nuclei di Elio che vengono
    emessi da nuclidi che hanno meno neutroni che servono per essere stabili in modo tale che che si conservi
    la il numero di massa a destra e a sinistra e fa aumentare il rapporto $n/p^{+}$ in modo che aumenti la stabilità. 
    \item \textbf{Emissione di particelle $\beta^{-}$}: Le particelle $\beta^{-}$ sono elettroni. Il nucleo non contiene 
    elettroni ma li può emettere come se un neutrone facesse la seguente reazione:
    \begin{gather*}
        \ce{n -> p^{+} + \beta^{-} + \bar{\nu}_e }
    \end{gather*}  
    L'antineutrino è creato nel processo di decadimento per conservare il momento angolare e questa reazione
    avviene anche per i neutroni liberi. Questa reazione fa diminuire il rapporto $n/p^{+}$ e quindi è comune
    per i nuclidi per quei nuclidi che stanno sopra la fascia di stabilità.
    \item \textbf{Emissione di particelle $\beta^{+}$}: Le particelle $\beta^{+}$ hanno la stessa massa dell'Elettrone
    ma carica di segno opposto ossia sono l'antiparallela dell'elettrone e sono chiamati positroni. La seguente reazione
    \begin{gather*}
        \ce{p^{+} -> n + \beta^{+} + \nu_e  }
    \end{gather*}  
    Dal momento che il positrone è l'antiparallelo dell'elettrone allora quando si incontrano si 
    annichiliscono rilasciando raggi $\gamma$ con direzione e polarizzazione opposta.
    \item \textbf{Cattura elettronica}: Invece che genrare un positrone il nucleo può catturare un elettrone
    di quelli che sono negli strati più interni per cui:
    \begin{gather*}
        \ce{p^{+} + \beta^{-} -> n + \nu_e}  
    \end{gather*} 
    e quindi il nucleo si traforma nel caso 3, ossia quando il numero atomico diminuisce mentre A rimane costante.
\end{enumerate}

\subsection{Il tempo di dimezzamento}
Il decadimento radioattivo è estremamente casuale ma la probabilità è differente per tutti i nuclidi
e questo porta al fatto che, in un campione macroscopico di materiale, il numero di decadimenti è proporzionale
al numero di atomi nel campione:
\begin{align}
    -\frac{dN}{dt} = \lambda N
\end{align}
Si definisce \textbf{tempo di dimezzamento} è il tempo necessario affinché un certo numero di nuclidi
sia ridotto alla metà. Tutti gli elementi da 1 a 83 hanno alemeno un isotopo stabile, lo stesso non vale
per quelli maggiori di 83.

\subsection{Elementi dal polonio all'uranio}
Gli isotopi degli elementi dal polonio all'uranio (84 a 92) sono tutti radioattivi. La loro esistenza 
non implica che debbano avere tempi di dimezzamento lunghi poiché potrebbero essere prodotti di
altri decadimenti. TUtti gli elementi tra \ce{^{238}U} a \ce{^{206}Pb} sono tutti naturali. Tutti gli elementi
hanno decadimento di tipo $\alpha$ oppure $\beta^{-}$.

\subsection{Reazioni di fusione e fissione nucleare}
LA fissione nucleare dovrebbe essere spontanea perché $E_1/A$ aumenta quando un nucleo 
si divide in due più leggeri, invece essa non avviene perché c'è un altra barriera di potenziale 
da superare prima che il nucleo si spezzi. Il modo più facile per far avvenire la fissione è con il
bombardamento di neutroni lenti, ed atomi che possono dare prodotti dalla loro rottura in facilmente
è \textbf{fissili}. In media ogni volta che si spezza un atomo di uranio si formano 2.5 neutroni
pper reazione, la quale procede dunque a catena. Solo \ce{^{235}U} è fissile, ma è l'isotopo meno
abbondante, per cui l'uranio naturale deve essere \textbf{arricchito} in modo tale da poter
far avvenire le reazioni. Se si mette insieme troppa massa di materiale fissile e si raggiunge la
\textbf{massa critica} la fissione ha una progressione esplosiva. \\
Si controlla la reazione con materiali come il Cadmio che assorbono i neutroni e l'utilizzo dell'acqua 
pesante per rallentare i neutroni liberi. \\
La reazione di \textbf{fusione nucleare } invece avviene o in modo incontrollato superando la massa critica
oppure in modo controllato facendo fondere due atomi di deuterio e trizio in modo da ottenere 
elio più energia oppure per creare gli elementi transuranici.

\section{Elementi transuranici}
Non sono dipsonibili in natura.


\end{document}
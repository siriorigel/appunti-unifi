\documentclass[a4paper]{article}
\usepackage{graphicx}
\usepackage{amsthm}
\usepackage{amsmath}
\usepackage{wrapfig}
\usepackage[a4paper,
            bindingoffset=0.2in,
            left=0.8in,
            right=0.8in,
            top=0.8in,
            bottom=0.7in,
            footskip=.25in]{geometry}
\usepackage[italian]{babel}
\usepackage{pgfplots}
\usepackage{tabularx}
\usepackage{tikz}
\usepackage{multicol}
\usetikzlibrary{datavisualization}
\usetikzlibrary{datavisualization.formats.functions}
\pgfplotsset{width=10cm,compat=1.9}

\setlength{\premulticols}{1pt}
\setlength{\postmulticols}{1pt}
\setlength{\multicolsep}{1pt}
\setlength{\columnsep}{2pt}

\pagestyle{empty}
\title{Analisi 1 - Ultimate Cheatsheet}

\begin{document}
\maketitle
\section{Trigonometria}
\begin{multicols}{2}
    \subsection*{Angoli e valori}
    \begin{center}
        \begin{tabular}{| c | c | c | c |}
        \hline 
        Angolo & sin & cos & tan \\
        \hline
        $0$ & $0$ & $1$ & $0$ \\
        \hline
        $\frac{\pi}{6}$ & $\frac{1}{2}$ & $\frac{\sqrt{3}}{2}$ & $\frac{\sqrt{3}}{3}$ \\
        \hline
        $\frac{\pi}{4}$ & $\frac{\sqrt{2}}{2}$ & $\frac{\sqrt{2}}{2}$ & $1$ \\
        \hline
        $\frac{\pi}{3}$ & $\frac{\sqrt{3}}{2}$ & $\frac{1}{2}$ & $\sqrt{3}$ \\
        \hline
        $\frac{\pi}{2}$ & $1$ & $0$ & - \\ 
        \hline
        $\frac{3\pi}{2}$ & $\frac{\sqrt{3}}{2}$ & $-\frac{1}{2}$ & $-\sqrt{3}$ \\
        \hline
        $\frac{3\pi}{4}$ & $\frac{\sqrt{2}}{2}$ & $-\frac{\sqrt{2}}{2}$ & $-1$ \\
        \hline
        $\frac{5\pi}{6}$ & $\frac{1}{2}$ & $-\frac{\sqrt{3}}{2}$ & $-\frac{\sqrt{3}}{3}$ \\
        \hline
        $\pi$ & $0$ & $1$ & $0$ \\
        \hline
        \end{tabular}
    \end{center}
    \subsection*{Angoli associati}
    \begin{align*}
        &\sin(-x) = -\sin(x) \\
        &\cos(-x) = \cos(x) \\
        &\tan(-x) = -\tan(x) \\
        &\tan\left(\frac{\pi}{2} - x\right) = \tan(x) \\
        &\sin\left(\frac{\pi}{2} - x\right) = \cos(x) \\
        &\cos\left(\frac{\pi}{2} - x\right) = \sin(x)
    \end{align*}
    \subsection*{Identità fondamentali}
    \begin{align*}
        \sin^2(x) + \cos^2(x) &= 1 \\
        \tan^2(x) + 1 &= \frac{1}{cos^2(x)} \\
        \tan(x) &= \frac{\sin(x)}{\cos(x)} \\
        \cot(x) &= \frac{\cos(x)}{\sin(x)} \\
        \csc(x) &= \frac{1}{\sin(x)} \\
        \sec(x) &= \frac{1}{\cos(x)} \\
        Posto \ \tan\left(\frac{x}{2}\right) &= t \\
        \cos(x) &= \frac{1 - t^{2}}{1 + t^{2}} \\
        \sin(x) &= \frac{2t}{1 + t^{2} }
    \end{align*}
    \subsection*{Somma e differenza}
    \begin{align*}
        \cos(\alpha \pm \beta) &= \cos(\alpha)\cos(\beta) \mp \sin(\alpha)\sin(\beta) \\
        \sin(\alpha \pm \beta) &= \sin(\alpha)\cos(\beta) \pm \sin(\beta)cos(\alpha) \\
        \tan(\alpha \pm \beta) &= \frac{\tan(\alpha) \pm \tan(\beta)}{1 \mp \tan(\alpha)\tan(\beta)}
    \end{align*}
    \subsection*{Duplicazione, bisezione, abbassamento di grado}
    \begin{align*}
        \tan(2x) &= \frac{2\tan(x)}{1-\tan(x)} \\
        \sin(2x) &= 2\sin(x)\cos(x) \\
        \cos(2x) &= \cos^2(x) - \sin^2(x) \\
        \tan\left(\frac{x}{2}\right) &= \pm \sqrt{\frac{1-\cos(x)}{1+\cos(x)}} \\
        \sin\left(\frac{x}{2}\right) &= \pm \sqrt{\frac{1-\cos(x)}{2}} \\
        \cos\left(\frac{x}{2}\right) &= \pm \sqrt{\frac{1+\cos(x)}{2}} \\
        \sin(x)^{2} &= \frac{1 - \cos(2x)}{2} \\
        \cos(x)^{2} &= \frac{1 + \cos(2x)}{2}
    \end{align*}
    \subsection*{Formule di prostaferesi}
    \begin{align*}
        &\sin(\alpha) + \sin(\beta) = 2\sin\left(\frac{\alpha + \beta}{2}\right)\cos\left(\frac{\alpha -\beta}{2}\right) \\
        &\sin(\alpha) - \sin(\beta) = 2\cos\left(\frac{\alpha + \beta}{2}\right)\sin\left(\frac{\alpha -\beta}{2}\right) \\
        &\cos(\alpha) + \cos(\beta) = 2\cos\left(\frac{\alpha + \beta}{2}\right)\cos\left(\frac{\alpha -\beta}{2}\right)\\
        &\cos(\alpha) - \cos(\beta) = -2\cos\left(\frac{\alpha + \beta}{2}\right)\sin\left(\frac{\alpha - \beta}{2}\right)
    \end{align*}
    \subsection*{Seno e coseno iperbolici}
    \begin{align*}
        \sinh &= \frac{e^{x} -e^{-x} }{2} \\
        \cosh &= \frac{e^{x} +e^{-x} }{2}
    \end{align*}
\end{multicols}

\section{Limiti}
\begin{multicols}{2}
    \subsection*{Limiti di successioni}
    \begin{align*}
        \lim_{n \rightarrow + \infty} a^n &= \left\{ \begin{array}{l}
            + \infty, \qquad se \quad a > 1, \\
            1, \qquad se \quad a = 1, \\
            0, \qquad se \quad -1 < a < 1, \\
            non \exists, \qquad se \quad a \leq -1, \\
        \end{array}\right. \\
        \lim_{n \rightarrow + \infty} \sqrt[n]{a} &= \lim_{n \rightarrow + \infty}a^{\frac{1}{n}} = 1. \\
        \lim_{n \rightarrow +\infty} \sqrt[n]{n^b} &= 1. \\
        \lim_{n \to \infty} \log_a(n) &= \left\{ \begin{array}{c}
            -\infty, \qquad se \quad 0 < a < 1, \\
            +\infty, \qquad se \quad a > 1
        \end{array}\right. \\
        \lim_{n \to \infty} \sqrt[n]{a_n} &= \lim_{n \to \infty} \frac{a_{n+1}}{a_n} \\
        \lim_{n \to \infty} \ \frac{a_n}{n} &= \lim_{n \to \infty} \ (a_{n+1} - a_n) \\
        \lim_{n \to \infty} \left(1 + \frac{1}{n}\right)^n &= e
    \end{align*}
    \subsubsection{Definizioni}
    \begin{align*}
        \lim_{x \to x_0} f(x) =& l \Longleftrightarrow \ \forall \epsilon > 0, \exists
        \delta > 0 : |f(x) - l| < \epsilon, \\ & \ \forall x \in A : 0 \ne |x - x_0| < \delta \\
        &\\
        \lim_{x \to x_0} f(x) =& + \infty \Longleftrightarrow \ \forall M > 0, \exists \delta > 0 : f(x) > M, \\
        & \ \forall x \in A : 0 \ne |x - x_0| < \delta. \\
        &\\
        \lim_{x \to x_0} f(x) =& - \infty \Longleftrightarrow \ \forall M > 0, \exists \delta > 0 : f(x) < -M, \\
        & \ \forall x \in A : 0 \ne |x - x_0| < \delta. \\
        &\\
        \lim_{x \to + \infty} f(x) =& l \Longleftrightarrow \ \forall \epsilon > 0, \exists k : |f(x) - l| < \epsilon,
        \\ &\ \forall x \in A : x > k. \\
        &\\
        \lim_{x \to + \infty} f(x) =& + \infty  \Longleftrightarrow \ \forall M > 0, \ \exists k : f(x) > M, \\
        & \ \forall x \in A : x > k. \\
        \lim_{x \to + \infty} f(x) =& - \infty  \Longleftrightarrow \ \forall M > 0, \ \exists k : f(x) < - M, \\
        & \ \forall x \in A : x > k. \\
        \lim_{x \to - \infty} f(x) =& l \Longleftrightarrow \ \forall \epsilon > 0, \exists k : |f(x) - l| < \epsilon,
        \\ &\ \forall x \in A : x < - k. \\
        &\\
        \lim_{x \to - \infty} f(x) =& + \infty  \Longleftrightarrow \ \forall M > 0, \ \exists k : f(x) > M, \\
        & \ \forall x \in A : x <- k. \\
        \lim_{x \to - \infty} f(x) =& - \infty  \Longleftrightarrow \ \forall M > 0, \ \exists k : f(x) < - M, \\
        & \ \forall x \in A : x < - k. 
    \end{align*}
    \subsection*{Limiti di funzioni}
    \begin{align*}
        \lim_{x \to 0} \frac{\log(1 + f(x))}{f(x)} &= 1 \\
        \lim_{x \to 0} \frac{\log_a(1 + f(x))}{f(x)} &= \frac{1}{\log(a)} \\
        \lim_{x \to 0} \frac{e^{f(x)} - 1}{f(x)} &= 1 \\
        \lim_{x \to 0} \frac{a^{f(x)} - 1}{f(x)} &= \log(a) \\
        \lim_{x \to \pm \infty} \left(1 + \frac{1}{f(x)}\right)^{f(x)} &= e \\
        \lim_{x \to 0} \frac{(1 + f(x))^c - 1}{f(x)} &= c \\
        \lim_{x \to 0} \frac{\sin(f(x))}{f(x)} &= 1 \\ 
        \lim_{x \to 0} \frac{1 - \cos(f(x))}{f(x)^2} &= \frac{1}{2}
    \end{align*}
    \subsection*{Gerarchia degli infiniti}
    Per $n \to +\infty$:
    \begin{align*}
        &\log_a(n) << n^b << c^n << n! << n^n \\
        &con \ a,b,c \in R
    \end{align*}
    \subsubsection{Definizioni}
    \begin{align*}
        &\lim_{n \rightarrow + \infty} a_n = a \quad oppure \quad a_n \rightarrow a, \\
        &\forall \epsilon > 0 \ \exists v \ : \ |a_n - a| < \epsilon \ \forall n > v. \\
        &\\
        &\lim_{n \rightarrow + \infty} a_n = + \infty \quad oppure \quad a_n \rightarrow + \infty. \\
        &\forall \ M > 0 \ \exists v \in R \ : \ a_n > M \ \forall \ n > v \\
        &\\
        &\lim_{n \rightarrow + \infty} a_n = - \infty \quad oppure \quad a_n \rightarrow - \infty. \\
        &\forall \ M > 0 \ \exists v \in R\  :\ a_n < -M \ \forall \ n > v 
    \end{align*}
    \subsection{Teoremi utili}
    \subsubsection{Successioni}
    TEOREMA DELL'UNICITA' DEL LIMITE \\
    TEOREMA DELLA PERMANENZA DEL SEGNO \\
    TEOREMA DEI CARABINIERI \\
    CRITERIO DEL RAPPORTO \\
    TEOREMA DI BOLZANO-WEIERSTRASS
    \subsubsection{Funzioni}
    TEOREMA DI WEIERSTRASS \\
    TEOREMA DEI VALORI INTERMEDI (I e II) \\
    CRITERIO DI INVERTIBILITA'
\end{multicols}

\section{Derivate}
\begin{multicols}{2}
    \subsection{Derivate fondamentali}
    \begin{align*}
        D \ cf(x) &= cf'(x)\\
        D\ (f(x) \pm g(x)) &= f'(x) \pm g'(x)\\
        D\ (f(x)\cdot g(x)) &= f'(x)g(x) + f(x)g'(x)\\
        D\ \left(\frac{f(x)}{g(x)}\right)& = \frac{f'(x)g(x) - g'(x)f(x)}{g^2}\\
        D\ f(g(h(x))) &= f'(g(h(x)))\cdot g'(h(x)) \cdot h'(x)\\
        D\ f^{-1}(y) &= \frac{1}{f'(x)} = \frac{1}{f'(f^{-1}(y))}\\
        D \ x^n &= nx^{n-1} \\
        D \ \log_a(x) &= \frac{1}{x}\log_a(e), \ \forall x > 0, a > 0, a \ne 1. \\
        D \ \sin(x) &= \cos(x)\\ 
        D \ \cos(x) &= -\sin(x)\\
        D \ \tan(x) &= \frac{1}{\cos^2(x)} = 1 + \tan^2(x).
    \end{align*}
    \subsection{Definizione e retta tangente}
    \begin{align*}
        &\lim_{h \to 0} \ \frac{f(x + h) - f(x)}{h} \\
        &y = f(x_0) + f'(x_0)(x-x_0). \\
        &\lim_{x \to x_0} \ \frac{f(x)-f(x_0)}{x - x_0}.
    \end{align*}
    \subsection{Teoremi utili}
    TEOREMA DI FERMAT \\
    TEOREMA DI ROLLE \\
    TEOREMA DI LAGRANGE \\
    CRITERIO DI MONOTONIA (e stretta)\\
    CRITERIO DI CONVESSITA' \\
    TEOREMA DI CAUCHY \\
    TEOREMA DI L'HOPITAL 
\end{multicols}

\section[s]{Utilizzo dei teoremi sui limiti}
\begin{multicols}{2}
    TEOREMA DI UNICITA' DEL LIMITE: \\
    Una successione convergente non può avere due limiti distinti. \\
    TEOREMA DELLA PERMANENZA DEL SEGNO: 
    \begin{align*}
        se \ \lim_{n \rightarrow +\infty} a_n = a>0 \ \exists \ v \ | \ a_n > 0 \ \forall n > v.
    \end{align*}
    
    TEOREMA DEI CARABINIERI: 
    \begin{align*}
        &a_n \leq c_n \leq b_n, \qquad \forall n \in N. \\
        &\lim_{n \rightarrow + \infty} a_n = \lim_{n \rightarrow + \infty} b_n = a \\
        &c_n \to a.
    \end{align*}
    CRITERIO DEL RAPPORTO:
    \begin{align*}
        &b_n = a_{n+1}/a_n \\
        &se \ b_n \to b, b < 1 \Rightarrow a_n \to 0.
    \end{align*}
    TEOREMA DI BOLZANO-WEIERSTRASS:
    \begin{align*}
        &a_n \ limitata \\
        &a_{n_k} \to l
    \end{align*}
    TEOREMA DI WEIERSTRASS:
    \begin{align*}
        &f(x) \ derivabile \ in [a, b]\\
        &\exists x_1, x_2 \in [a, b] : f(x_1) \leq f(x) \leq f(x_2), \ \forall x \in [a, b]
    \end{align*}
    TEOREMA DEI VALORI INTERMEDI:
    \begin{align*}
        &\forall x \in [a, b] \ se \ f(x) \ continua \\ 
        &\Rightarrow \ \exists f(x) = y. \\
        &\forall x \in [a, b] \ se \ f(x) \ continua \\
        &\Rightarrow \ min\{f(x)\} \leq f(x) \leq max\{f(x) \}.
    \end{align*}
    CRITERIO DI INVERTIBILITA':
    \begin{align*}
        &se \ f(x) \nearrow \nearrow \ \vee f(x) \searrow \searrow \\
        &f(x) \ invertibile
    \end{align*}
   
\end{multicols}

\section[]{Utilizzo dei teoremi sulle derivate}
\begin{multicols}{2}
    TEOREMA DI FERMAT:
    \begin{align*}
        &f\in[a, b], \ f(x_0) = min\{f(x)\} \ \vee \ max\{f(x)\} \\
        &\Rightarrow f'(x_0) = 0.
    \end{align*}
    TEOREMA DI ROLLE
    \begin{align*}
        &f(x) \ continua \in [a, b] \ e \ derivabile \in (a, b) \\
        &se \ f(a) = f(b) \Rightarrow \exists x_0 \in (a, b) : f'(x_0) = 0.
    \end{align*}
    TEOREMA DI LAGRANGE:
    \begin{align*}
        &f(x) continua \in [a, b] \ e \ derivabile \in (a, b) \\
        &\Rightarrow \exists x_0 \in (a, b) : f'(x_0) = \frac{f(b)-f(a)}{b-a}
    \end{align*}
    CRITERIO DI MONOTONIA:
    \begin{align*}
        &f(x) continua \in [a, b] \ e \ derivabile \in (a, b) \\
        &f'(x)\geq 0, \qquad \forall x \in (a, b) \ f \nearrow \ \in [a, b] \\
        &f'(x) \leq 0, \qquad \forall x \in (a, b) \ f \searrow \ \in [a, b]
    \end{align*}
    CRITERIO DI CONVESSITA':
    \begin{align*}
        &f(x) \ continua \in [a, b] \ e \ derivabile \in (a, b) \\
        &f(x) \ convessa \in [a, b]; \\
        &f'(x) \ crescente \in [a, b]; \\
        &f''(x) \geq 0 \ \forall x \in (a, b).
    \end{align*}
    TEOREMA DI CAUCHY:
    \begin{align*}
        &f(x) continua \in [a, b] \ e \ derivabile \in (a, b) \\
        &g'(x) \ne 0 \ \forall x \in (a, b) \\
        &\Rightarrow \exists x_0 \in (a, b) : \frac{f'(x_0)}{g'(x_0)} = \frac{f(b)-f(a)}{g(b)-g(a)}.
    \end{align*}
    TEOREMA DI L'HOPITAL:
    \begin{align*}
        &f(x), g(x) \to 0 \ derivabili \ in \ un \ intorno \ di \ x_0\\
        &se \ g'(x) \ne 0 \ \forall x \ne x_0 \\
        &\Rightarrow \lim_{x \to x_0} \frac{f(x)}{g(x)} = \lim_{x \to x_0} \frac{f'(x)}{g'(x)}.
    \end{align*}
\end{multicols}

\section{Integrali}
\subsection{Integrali immediati}
\begin{multicols}{2}
    \begin{align*}
        \int x^{b}  \ dx &= \frac{x^{b + 1}}{b+ 1} + c; \\
        \int \frac{1}{x} \ dx &= \log(x) + c; \\
        \int a^{x}  \ dx &= \frac{a^{x} }{\log(a)} + c; \\
        \int e^{x}  \ dx &= e^{x}  + c; \\
        \int \sin (x) \ dx &= -\cos(x) + c; \\
        \int \cos(x) \ dx &= \sin(x) + c; \\
        \int \frac{1}{\cos^{2}(x)} \ dx &= \tan(x) + c; \\
        \int \frac{1}{\sqrt{1 - x^{2}}} \ dx &= \arcsin(x) + c; \\
        \int -\frac{1}{\sqrt{1 - x^{2} }} \ dx &= \arccos(x) + c; \\
        \int \frac{1}{1 + x^{2} } \ dx &= \arctan(x) + c; \\
        \int \frac{1}{a^{2} +x^{2} } \ dx &= \frac{1}{a} \arctan\left(\frac{x}{a}\right) + c; \\
        \int \log(x) =& x\log(x) -x + c \\
        \int \frac{1}{\sin^{2}(x)} \ dx &= -\cot(x) + c; \\
        \int \cosh (x) \ dx &= \sinh x + c; \\
        \int \sinh (x) \ dx &= \cosh(x) + c; \\
        \int \frac{1}{\cosh^{2}(x)} \ dx &= \tanh (x) + c; \\
        \int \frac{1}{\sinh^{2}(x)} \ dx &= -\coth(x) + c; \\
        \int \frac{1}{\sqrt{1 + x^{2} } } \ dx &= arcisinh(x) + c; \\
        \int \frac{1}{\sqrt{x^{2}-1} } \ dx &= arccosh(x) + c;
    \end{align*}
    \subsection{Altri integrali immediati}
    \begin{align*}
        \int \frac{b}{(cx + a)^{2} } \ dx &= -\frac{b}{c} \cdot  \frac{1}{(x + \frac{a}{c})^{2} } + C \\
        \int \frac{1}{\sin(x)} \ dx &= \int \csc(x) \ dx = -\log\left(\tan\left(\frac{x}{2}\right)\right) + c \\
        &\text{si risolve moltiplicando per $\cot(x) + \csc(x)$}\\
        \int \frac{1}{\cos(x)} \ dx &= \int \sec(x) \ dx = \ln\left(\frac{\sin\left(\frac{x}{2}\right) + \cos\left(\frac{x}{2}\right)}{\cos\left(\frac{x}{2}\right) - \sin\left(\frac{x}{2}\right)}\right) + c \\
        &\text{Si risolve moltiplicando per $\tan(x) + \sec(x)$}\\
        \int \sqrt{x^{2} + 1} \ dx &= \frac{1}{4}(\sinh(2x) - 2x) \\
        &\text{Si risolve con la sostituzione di $x = \sinh(t)$} \\
    \end{align*}
    \subsection{Tecniche di integrazione}
    PER PARTI
    \begin{gather*}
        \int f(x) g'(x) \ dx = f(x) g(x)  - \int f'(x)g(x) \ dx 
    \end{gather*}
    DERIVATE DI COMPOSTE
    \begin{gather*}
        \int F'(g(x) )\cdot g'(x) \ dx = F(g(x) ) + c
    \end{gather*}
    INTEGRAZIONE PER SOSTITUZIONE
    \begin{gather*}
        \int f(x) \ dx \Rightarrow t = g(x), dt = g'(x)dx
    \end{gather*}
    \subsection{Integrali impropri e loro carattere}
    \begin{gather*}
        \lim_{c \to +\infty } \int_{x_0}^{c}f(x) \ dx = \lim_{c \to +\infty } F(c) -F(x_0)  
    \end{gather*}
    Se due funzioni integrabili sono tali che $0 \leq f(x) \leq g(x)$,
    allora se la maggiore converge anche la minore converge,
    altrimenti se la maggiore diverge anche la minore diverge.
    \subsection{Teoremi degli integrali}
    TEOREMA DI CANTOR \\
    TEOREMA DI INTEGRABILITA' DELLE FUNZIONI CONTINUE \\
    PRIMO TEOREMA DELLA MEDIA
    \begin{gather*}
        m(b - a)\leq \int_{a}^{b}f(x) \ dx \leq M(b - a) 
    \end{gather*}
    SECONDO TEOREMA DELLA MEDIA
    \begin{gather*}
        \int_{a}^{b}f(x) \ dx = f(x_0)(b- a) 
    \end{gather*}
    INTEGRABILITA' DELLE FUNZIONI MONOTONE
\end{multicols}


\section{Taylor}
\subsection{Sviluppi di alcune funzioni}
\begin{align*}
    f(x) =& f(x_0) + f'(x_0)(x-x_0) + \frac{f''(x_0)}{2}(x-x_0)^{2}+ .... \\ 
    e^{x} =& 1 + x + \frac{x^{2}}{2} + \frac{x^{3}}{3!} + \dots + \frac{x^{n}}{n!} + o(x^{n}); \\
    \log(1 + x) =& x - \frac{x^{2}}{2} + \frac{x^{3}}{3} - \dots + (-1)^{n+1} \frac{x^{n}}{n} + o(x^{n}); \\
    \sin(x) =& x - \frac{x^{3}}{3!} + \frac{x^{5}}{5!} - \dots + (-1)^{n} \frac{x^{2n+1}}{(2n + 1)!} + o(x^{2n+2}): \\
    \cos(x) =& 1 - \frac{x^{2}}{2} + \frac{x^{4} }{4!} - \dots + (-1)^{n} \frac{x^{2n}}{(2n)!} + o(x^{2n + 1}); \\
    \tan (x) =& x + \frac{x^{3}}{3} + \frac{2x^{5} }{15} + \frac{17x^{7} }{315} + o(x^{8} ) \\
    \sinh(x) =& x + \frac{x^{3} }{6} + \frac{x^{5} }{5!} + \dots + \frac{x^{2n + 1}}{(2n + 1)!} + o(x^{2n + 2} ) \\
    \cosh(x) =& 1 + \frac{x^{2} }{2} + \frac{x^{4} }{4!} + \dots + \frac{x^{2n} }{(2n)!} +o(x^{2n + 2} )\\
    \tanh(x) =& x - \frac{x^{3} }{3} + \frac{2x^{5} }{15} - \frac{17x^{7} }{315} + o(x^{8}) \\
    arctanh(x) =& x + \frac{x^{3} }{3} + \frac{x^{5} }{5} + \dots + \frac{x^{2n + 1} }{2n + 1} + o(x^{2n + 2} ) \\
    \arctan(x) =& x  - \frac{x^{3} }{3} + \frac{x^{5} }{5} - \dots + (-1)^{n} \frac{x^{2n + 1} }{2n + 1} + o(x^{2n + 2});  \\
    \arcsin(x) =& x + \frac{x^{3} }{6} + o(x^{4} ) \\
    \arccos(x) =& \frac{\pi}{2} - x - \frac{x^{3} }{6} + o(x^{4} ) \\
    (1 + x)^{\alpha} =& 1 + \alpha x + \frac{\alpha(\alpha - 1)x^{2} }{2} +  \frac{\alpha(\alpha - 1)(\alpha - 2)x^{3} }{6} + \dots \begin{pmatrix} \alpha \\
    n \end{pmatrix} x^{n} + o(x^{n} ) \\
    \frac{1}{1+x} =& 1 + x + x^{2} + x^{3} + \dots + x^{n} + o(x^{n}  ) \\
    \sqrt{1 + x} =& 1 + \frac{x}{2} - \frac{x^{2} }{8} + o(x^{2} ).   
\end{align*}
\subsection{Proprietà degli o piccoli}
\begin{align}
    o(x^{n}) + o(x^{n}) &= o(x^{n}) \\
    c \cdot o(x^{n}) &= o(cx^{n}) = o(x^{n}) \\
    o(x^{n}) - o(x^{n}) &= o(x^{n}) \\
    x^{m} \cdot o(x^{n}) &= o(x^{m+n} ) \\
    o(x^{m}) \cdot o(x^{n}) &= o(x^{m+n} ) \\
    o(o(x^{n} )) &= o(x^{n} ) \\
    o(x^{n}  + o(x^{n} )) &= o(x^{n})
\end{align}

\section{Serie numeriche}
\begin{multicols}{2}
    \subsection{Somma parziale}
    \begin{gather*}
        S_N = \sum_{n = 1}^{N} a_n
    \end{gather*}
    \subsection{Serie geometriche}
    \begin{gather*}
        \sum_{n = 0}^{\infty } q^{n} = \frac{1}{1-q}, se \ q < |1|
    \end{gather*}
    \subsection{Serie armoniche}
    \begin{gather*}
        \sum_{n = 1}^{+\infty } \frac{1}{n^{p}}  
    \end{gather*}
    Converge se $p > 1$, diverge se $p \leq 1$.
    \begin{gather*}
        \sum_{n = 2}^{+\infty } \frac{1}{n^{\alpha} \log(n)^{\beta} } 
    \end{gather*}
    \begin{enumerate}
        \item Converge se $\alpha > 1$ e $\forall \beta$.
        \item Converge se $\alpha = 1$ e $\beta > 1$
        \item Diverge per $\alpha = 1$ e $\beta \leq 1$
        \item Diverge per $\alpha < 1$ 
    \end{enumerate}
    \begin{gather*}
        \sum_{n = 2}^{+\infty } \frac{1}{e^{n\gamma}n^{\alpha}\log(n)^{\beta}} 
    \end{gather*}
    \begin{enumerate}
        \item Converge per $\forall \gamma > 0, \forall a, b$
        \item Diverge per $\gamma < 0$
        \item Per $\gamma = 0$ vedere sopra
    \end{enumerate}
    \subsection{Teoremi delle serie}
    CONDIZIONE NECESSARIA PER LA CONVERGENZA DELLA SERIE
    \begin{gather*}
        Se \sum_{n = 1}^{\infty } a_n \ converge \Rightarrow \lim_{n \to \infty } a_n = 0  
    \end{gather*}
    CRITERIO DEL CONFRONTO
    \begin{gather*}
        Prese \ 0 \leq a_n \leq b_n \\
        Se \ \sum_{n = 1}^{\infty } b_n < + \infty \Rightarrow \sum_{n = 1}^{\infty } a_n < + \infty \\
        Se \   \sum_{n = 1}^{\infty } a_n = + \infty \Rightarrow \sum_{n = 1}^{\infty } b_n = + \infty 
    \end{gather*}
    CRITERIO DEL CONFRONTO ASINTOTICO
    \begin{gather*}
        a_n \geq 0, b_n > 0 \Rightarrow  \\
        \exists \lim_{n \to +\infty } \frac{a_n}{b_n} = L \in (0, +\infty ), L \neq 0. \Rightarrow  \\
        \sum_{n = 1}^{\infty } a_n \ e \ \sum_{n = 1}^{\infty } b_n\\
        hanno \ lo \ stesso \ carattere   
    \end{gather*}
    CRITERIO DEGLI INFINITESIMI 
    \begin{gather*}
        Se \ \lim_{n \to \infty } n^{p}a_n = l \\
        l \ne + \infty , p > 1 \Rightarrow \sum_{n = 1}^{\infty } a_n < + \infty \\
        l \ne 0, p \leq 1, \Rightarrow \sum_{n = 1}^{\infty }a_n = + \infty    
    \end{gather*}
    CRITERIO DELLA RADICE
    \begin{gather*}
        \lim_{n \to \infty } \sqrt[n]{a_n} = l  \Rightarrow  \\
        l < 1 \Rightarrow  \sum_{n = 1}^{\infty }a_n < + \infty \\
        l > 1 \Rightarrow  \sum_{n = 1}^{\infty } a_n = + \infty 
    \end{gather*}
    CRITERIO DEL RAPPORTO 
    \begin{gather*}
        \lim_{n \to \infty} \frac{a_{n + 1}}{a_n} = l \Rightarrow \\
        l < 1 \Rightarrow  \sum_{n = 1}^{\infty }a_n < + \infty \\
        l > 1 \Rightarrow  \sum_{n = 1}^{\infty } a_n = + \infty    
    \end{gather*}
    CRITERIO DI LEIBNIZ
    \begin{gather*}
        Se \ \lim_{n \to \infty } a_n = 0 \ e \ decresce \Rightarrow \\
        \sum_{n = 1}^{\infty }(-1)^{n}(a_n) = l 
    \end{gather*}
    CRITERIO DELL'INTEGRALE \\
    L'integrale improprio di una funzione continua e decrescente e la sua serie
    associata hanno lo stesso carattere.
\end{multicols}



\appendix
\section{Studio di funzione}
\begin{enumerate}
    \item \emph{Dominio} di $f(x)$;
    \item parità($f(-x)=f(x), \forall x$) o disparità($-f(-x) = f(x), \forall x$), oppure periodica;
    \item intersezioni con gli assi cartesiani;
    \item segno della funzione;
    \item limiti agli estremi del dominio;
    \item asintoti orizzontali, verticali o obliqui, trovare asintoti obliqui:
    \begin{align*}
        &m = \lim_{x \to \infty} \ \frac{f(x)}{x}; \\
        &q = \lim_{x \to \infty} \ (f(x) - mx)\\
        &y = mx + q
    \end{align*}
    \item intervalli di monotonia;
    \item massimi e minimi relativi e relativi valori (quando ci sono massimi e minimi assoluti);
    \item intervalli di concavità o convessità e punti di flesso con la derivata seconda;
    \item classificazione delle discontinuità (se presenti).
\end{enumerate}
\section{Polinomi}
\subsection{Divisione tra polinomi}
Esempio di divisione tra polinomi
\begin{gather*}
    \frac{x^{2} + 4x}{x + 2} = 
    \begin{tabular}{c | c}
        $x^{2} + 4x$ & $x + 2$ \\
        \hline
        $-x^{2} -2x$ & $x + 2$ \\
        $2x$ & \\
        $-2x - 4$ & \\
        $-4$
    \end{tabular} = x + 2 -\frac{4}{x + 2}
\end{gather*}
\subsection{Falso quadrato}
Il falso quadrato è molto utile per la scomposizione dei cubi:
\begin{align*}
    &x^3-y^3 = (x-y)(x^2 +xy +y^2) \\
    &x^3+y^3 = (x+y)(x^2 -xy + y^2)
\end{align*}
dove al secondo membro $x^2 \pm xy +y^2$ è il falso quadrato.

\section{Dimostrazioni per induzione}
Supponiamo che una proposizione dipendente da un indice $n \in N$
sia vera per $n=1$ e che inoltre, supposta vera per n, sia vera anche per il successivo
n + 1. Allora la proposizione è vera $\forall n \in N$. Esempio pratico: dimostrare che
$n^2+n+3$ è dispari $\forall n \in N$. Posto che l'ipotesi sia vera, allora trovo un
indice $n$ per cui questa vale (banalmente per n = 1). Ponendo ora n+1, voglio vedere se
vale ancora:
\begin{align*}
    (n+1)^2+(n+1)+3 \ = \ n^2+2n+1+n+1+3\  = \ n^2 + n + 1 + 2n + 2 
\end{align*}
Per ipotesi anche questa è verificata in quanto $2n + 2$ è un numero pari
mentre $n^2 + n + 1$ è la nostra ipotesi $-2$, per induzione è quindi verificata.

\section{Combinatoria}
Definizione di numero fattoriale:
\begin{align*}
    n! = n(n-1)(n-2)\dots(n-(n-1)).
\end{align*}
Il numero di disposizioni di $k$ elementi tra $n$ elementi dati è:
\begin{align*}
    n(n-1)\dots(n-k+1) = \frac{n!}{(n-k)!}
\end{align*}
Il numero di combinazioni di $k$ elementi è:
\begin{align*}
    \left(\begin{array}{l}
        n \\
        k
      \end{array}\right) = \frac{n!}{(n - k)!k!}.
\end{align*}
Da cui seguono le identità:
\begin{align*}
    \left(\begin{array}{l}
        n \\
        k
    \end{array}\right) = \left(\begin{array}{l}
        \ \ n \\
        n - k
    \end{array}\right), \qquad
    \left(\begin{array}{l}
        n \\
        k
    \end{array}\right) = \left(
        \begin{array}{c}
            n-1 \\ k
        \end{array}
    \right) +
    \left(
        \begin{array}{c}
            n-1 \\ k-1
        \end{array}
    \right).
\end{align*}
Binomio di Newton:
\begin{align*}
    (a + b)^{n} = \left(\begin{array}{l}n\\0\end{array}\right) a^{n} +
        \left(\begin{array}{l}n\\1\end{array}\right)a^{n - 1}b+
        \dots + &\left(\begin{array}{l}n\\k\end{array}\right)a^{n - k}b^{k} + 
        \dots + \left(\begin{array}{l}\ \ n\\n - 1\end{array}\right)ab^{n - 1} +
        \left(\begin{array}{l}n\\n\end{array}\right)b^{n}. = \\ 
    &= \sum_{k = 0}^{n} \left(
        \begin{array}{c}
            n \\ k
        \end{array}
    \right)a^{n-k}b^{k}
\end{align*}

\section{Integrazione di fratte}
Si può dividere una fratta nei seguenti modi: \\
Prima di tutto si cerca di ridurre il numeratore al primo grado attraverso la divisione tra polinomi. Passando ora al delta del denominatore (posto che sia di secondo grado),
nel caso in cui il $\Delta > 0$ allora si procede come segue:
\begin{gather*}
    \int \frac{ax + b}{x^{2} + cx + d}\\
    \int \frac{ax + b}{(x + e)(x + f)} = \int \frac{A}{x + e} + \int \frac{B}{x + f}
\end{gather*} 
A questo punto si ricombinano le due frazione e si ottengono i valori di A e B per cui si
ottiene il numeratore di partenza. \\
$\Delta = 0$ \\
Si procede in maniera simile: 
\begin{gather*}
    \int \frac{ax + b}{x^{2} + cx + d}\\
    \int \frac{ax + b}{(x + e)(x + e)} = \int \frac{A}{x + e} + \int \frac{B}{(x + e)^{2}}
\end{gather*}
Se $\Delta < 0$ si prova un raccoglimento parziale o a scomporre la frazione in modo da risolvere per
parti o da ricondurre all'integrale dell'arcotangente col completamento del quadrato. Altrimenti esiste
la forma (grazie Alessandro):
\begin{gather*}
    \int \frac{ax + b}{(x - 1)(cx^{2} + dx + e)} \ dx = \int \frac{A}{x-1} + \frac{B + Cx}{(cx^{2} +dx + e)} \ dx
\end{gather*}

\section{Integrazione di trigonometriche (principalmente seno e coseno) e $e^{x}$ }
Nel caso in cui un integrale contenga seno e coseno oppure $e^{x}$ (o anche entrambi), si cerca
di integrare per parti in modo tale da avere a destra dell'uguale lo stesso integrale di 
partenza: così possiamo portarlo dall'altra parte e dividere per il coefficiente risultante
ed ottenere il risultato dell'integrale. 

\section{Numeri complessi}
\begin{multicols}{2}
    \subsection*{Forma algebrica e formule di base}
    \begin{align*}
        &a + ib \\
        &i^2 = -1 \\
        &z' = a-ib \ (coniugato) 
    \end{align*}
    \subsection*{Forma trigonometrica e radici}
    \begin{align*}
        &z = \rho(\cos(\theta) + i\sin(\theta)) = \rho e^{i\theta} \\
        &z^n = \rho^n(\cos(n\theta) + i\sin(n\theta))
    \end{align*}
\end{multicols}

\end{document}
\documentclass[a4paper, oneside]{book}
\usepackage{graphicx}
\usepackage{amsthm}
\usepackage{amsmath}
\usepackage[a4paper,
            bindingoffset=0.2in,
            left=2cm,
            right=2cm,
            top=2cm,
            bottom=2cm,
            footskip=.25in]{geometry}
\usepackage[italian]{babel}
\usepackage{pgfplots}
\usepackage{tabularx}
\usepackage{tikz}
\usepackage{wrapfig}
\definecolor{page}{rgb}{0.129,0.157,0.212}
\pagecolor{page}
\color{white}
\usetikzlibrary{datavisualization}
\usetikzlibrary{datavisualization.formats.functions}
\pgfplotsset{width=10cm,compat=1.9}

\title{Esperienze di Lab. I}
\author{Tommaso Miliani}
\date{2024/2025}

\begin{document}

\maketitle

\tableofcontents

\chapter{Esperienza bilancia}
\section{Scopo dell'esperienza}
In questa esperienza si misureranno la \textbf{densità media assoluta} di 
più corpi ricordando che la densità assoluta è
\begin{gather*}
    \delta = \frac{m}{V} \ \frac{kg}{m^{3} }
\end{gather*}
Si indica inoltre con $\delta_r$ ed è il rapporto tra la \textbf{densità media assoluta} e
la densità dell'acqua alla temperatura di $T = 3.98\text{°}C$:
\begin{align}
    \delta_r = \frac{\delta}{\delta_A(277.13K)}
\end{align}
Dove la densità dell'acqua a quella temperatura è
\begin{gather*}
    \delta_A(277.13 K) = 0.999973 \frac{g}{cm^{3} } 
\end{gather*}
Dato che il volume di un oggetto è direttamente collegato alla sua temperatura, per il
fenomeno della \textbf{dilatazione termica} avremmo anche una certa dipendenza da parte
di $\delta$ rispetto alla temperatura: l'andamento tipico del volume rispetto alla
temperatura è data da:
\begin{gather*}
    \frac{\Delta V}{V_0} \approx \beta(\Delta T)
\end{gather*}

\section{Gli strumenti}
\subsection{La bilancia di precisione}
E' una piccola bilancia che sfrutta il funzionamento di una molla per misurare
la reazione vincolare sottoposta all'oggetto da pesare e per questo deve essere posta su una
superficie orizzontale per avere una lettura precisa. \\
La bilancia presenta diverse funzioni: 
\begin{enumerate}
    \item $ON/OFF$: il tasto che permette di spegnere o accendere lo strumento
    le cui prestazioni sono migliori solo se lo strumento è già acceso da qualche
    ora in modo tale da poter stabilizzare la molla e quindi la lettura.
    \item $TARE$: il tasto che permette di azzerare il peso percepito e va premuto
    prima di ogni misura in modo da misurare solo la forza aggiuntiva al
    peso "normale" sulla bilancia;
    \item $MODE/CAL$: Serve per la calibrazione della bilancia ponendo sopra la
    bilancia un oggetto con la massa di $1kg$ (ossia della portata massima della bilancia)
    in modo ta ottenere il valore corretto di $g$ e quindi poter iniziare
    ad eseguire letture dopo aver premuto il taso $SET$. 
\end{enumerate}

Caratteristiche della bilancia di laboratorio:
\begin{enumerate}
    \item Prontezza: $3s$;
    \item Portata: $1kg$;
    \item Errore di sensibilità: $0.01 g$ (ossia l'errore di riproducibilità);
    \item Errore di linearità: $0.03g$, ossia il valore misurato potrebbe discostarsi di massimo questo valore rispetto al reale (misurato con uno strumento più preciso). 
\end{enumerate}
Queste caratteristiche (specificate dal costruttore) rimangono vere solo sotto
certe condizioni specifiche di temperatura e di inclinazione etc..  \\
E' possibile per la bilancia poter determinare l'\textbf{errore di linearità} attraverso
una serie di misure con oggetti di massa crescente. Una volta misurato il peso
di $20$ masse pre-pesate con strumenti più precisi e con massa
crescente si è in grado di determinare il valore dell'errore di linearità
che sarà  dato dallo scarto massimo del peso misurato dalla bilancia rispetto
al peso misurato da una bilancia più precisa. Una volta misurato lo confrontiamo
con quello del costruttore e scegliamo quello più alto.  \\
Si può inoltre valutare l'\textbf{errore di riproducibilità}, ossia l'errore che si
ha misurando più volte($6$) la massa di uno oggetto. 

\section{La misura della densità relativa e picnometro}
In laboratorio si hanno vari oggetti assimilabili a solidi geometrici per cui si calcolano
la massa in maniera diretta (bilancia) ed il volume in maniera indiretta (calibro e palmer). 
Dando per scontato che i solidi misurati siano \textbf{regolari}, ossia che la loro densità non
cambi durante le misure né che la densità non sia uniforme. L'errore sulla massa di un certo
oggetto sarà data allora da
\begin{align}
    \Delta m = \Delta_{SM} + \Delta_{Lin}
\end{align}
Ossia la somma tra lo scarto massimo o di riproducibilità (si prende quello più alto) e errore di linearità.
Il picnometro è un'ampolla con un tappo smerigliato che si adatta
perfettamente al collo della bottiglia con un tappo che un foro capillare.
All'interno è possibile immettere un liquido (acqua distillata) sul tappo è possibile
regolare precisamente il volume di acqua attraverso un segno sul foro capillare. \\
Questo strumento ci permette di calcolare la densità relativa di un oggetto in maniera
indiretta attraverso la differenza di volume occupato dal solido secondo due procedure 
distinte.

\subsection{Procedura 1}
Si eseguono tre misure di massa:
\begin{enumerate}
    \item La prima con solo l'oggetto da misurare che chiamiamo $m_1$;
    \item La seconda sia con l'oggetto che con il picnometro riempito di 
    acqua fino al livello prefissato che chiamiamo $m_2$;
    \item La terza con il picnometro con all'interno l'oggetto e di nuovo riempito
    di acqua che chiamiamo $m_3$. 
\end{enumerate}
Si può ricavare la densità relativa tramite la relazione funzionale
\begin{align}
    \delta_{r_m} = \frac{m_1}{m_2 - m_3}
\end{align} 
che possiamo riscrivere (considerando $m_2 = m_x + m_p$ e $m_3 = m_x + m_p - m_A$) come
\begin{gather*}
    \delta_{r_m} = \frac{m_x }{m_A} = \frac{\delta}{\delta_A}
\end{gather*}
L'incertezza associata la possiamo ottenere con la tecnica della derivata logaritmica
\begin{gather*}
    d \frac{\delta_{r_m}}{\delta_{r_m}} = d \ln \delta_{r_m} = d \ln m_1 - d\ln(m_2 - m_3) = 
\end{gather*}
\begin{align}
    \frac{\Delta \delta_{r_m} }{\delta r_m} = \left|\frac{1}{m_1}\right| \Delta m_1 + \left|\frac{1}{m_2 - m_3}\right|\Delta m_2 + \left|\frac{1}{m_2 - m_3}\right| \Delta m_3 
\end{align}

\subsection{Procedura 2}
La seconda procedura è molto simile alla prima e consiste nel modificare la prima misura
misurando solamente la massa del picnometro($m_a$) con acqua distillata mentre $m_2$ e $m_3$ rimangono
inalterate. Si osserva che (chiamate $m_2 = m_b$ e $m_3 = m_c$):
\begin{align}
    \delta_{r_m} = \frac{m_b - m_a}{m_b - m_c}
\end{align}
Con l'incertezza relativa si ha che
\begin{gather*}
    \frac{\Delta \delta_{r_m}}{\delta_{r_m}} = \frac{\Delta m_a}{m_b - m_a} + \left|\frac{1}{m_b - m_a} - \frac{1}{m_b - m_c}\right| \Delta m_b + \frac{\Delta_c}{m_b - m_a}
\end{gather*}

\subsection{Scegliere la procedura corretta}
In linea teorica, dato che sono validi entrambi i modelli, potremmo scegliere
di calcolare i valori per entrambe le procedure e scegliere quella più precisa, ma
conviene invece ragionare a priori sulle incertezze relative (ricordando che $m_p = m_a$).\\
Supponendo che gli errori siano confrontabili (dato che abbiamo scelto gli stessi strumenti)
e supponendo che l'errore sulle masse sia uguale posso calcolare le incertezze relative come
\begin{gather*}
    \varepsilon_1 = \Delta m \left(\frac{1}{m_1} + \frac{2}{m_2 - m_3}\right) = \Delta m \left(\frac{1}{m_x} + \frac{2}{m_A}\right)\\
    \varepsilon_2 = \Delta m \left(\frac{1}{m_x} + \left|\frac{1}{m_x} - \frac{1}{m_A}\right| + \frac{1}{m_A}\right)
\end{gather*}
Possiamo allora semplificare di più queste espressioni sommando e semplificando $m_a$ con le relazioni
di massa che si conoscono dal metodo uno ottenendo
per la prima procedura un errore del tipo (con $\frac{m_x}{m_a} = \delta_{r_m}$ ):
\begin{gather*}
    \varepsilon_1 = \frac{\Delta m}{m_A} \left(\frac{m_A + 2m_x}{m_x}\right) \\
    \varepsilon_1 = \frac{\Delta m}{m_A} \cdot  \frac{1 + 2\delta_{r_m}}{\delta_{r_m}} 
\end{gather*}
Mentre per la seconda procedura, svolgendo, si ottiene la seguente formulazione:
\begin{gather*}
    \varepsilon_2 = \Delta m \left(\frac{m_A + m_x}{m_xm_A} + \left|\frac{m_A - m_x}{m_x m_A}\right|\right)  =\\
    \frac{\Delta m}{m_A} \left(\frac{m_A + m_x}{m_x} + \left|\frac{m_A - m_x}{m_x}\right|\right)  =\\
    \frac{\Delta m}{m_A} \left(\frac{1 + \delta_{r_m}}{\delta_{r_m}} + \left| \frac{1 - \delta_{r_m}}{\delta_{r_m}} \right| \right) \\
    \varepsilon_2 = \frac{\Delta m}{m_A} \cdot  \left(\frac{1 + \delta_{r_m}}{\delta_{r_m}} + \frac{\left| 1 - \delta_{r_m} \right| }{\delta_{r_m}}\right)
\end{gather*}
Adesso vediamo i vari casi $\varepsilon_2$:
\begin{gather*}
    \text{Se } \delta_{r_m} < 1 \begin{array}{l}
        \varepsilon_1 = \frac{\Delta m}{m_A} \cdot  \frac{1 + 2 \delta_{r_m}}{\delta_{r_m}} \\
        \varepsilon_2 = \frac{\Delta m}{m_A} \cdot  \frac{2}{\delta_{r_m}}
    \end{array}
\end{gather*}
Si ottiene allora che in questo caso $\varepsilon_1 < \varepsilon_2$ quando $\delta_{r_m} < 0.5$.
Se invece si avesse che $\delta_{r_m} > 1$ allora $\varepsilon_1 < \varepsilon_2$ MAI. In generale
la procedura 2 è più precisa quando $\delta_{r_m} > 0.5$. 


\section{Correzioni dovute alla temperatura e alla forza di Archimede}
Dato che risulta complesso lavorare alla temperatura precisa di $3.98^{\circ} C$ allora 
avremmo che $\delta_{r_m} \neq \delta_r$. Si può allora correggere
questo valore con un fattore correttivo per la temperatura 
\begin{gather*}
    \delta_r = \frac{\delta}{\delta_A(3.98^{\circ} C)} \cdot \frac{\delta_A(T)}{\delta_A(T)} = \delta_{r_m}f(T), \quad f(T) = \frac{\delta_A(T)}{\delta_A(3.98^{\circ} C)}
\end{gather*}
I valori della densità dell'acqua sono dati in un grafico e quindi per
trovare il valore vero della densità alle temperature a cui si opera in
laboratorio è necessario interpolare il grafico. \\
Adesso analizziamo l'altro fattore correttivo: ossia la correzione sulla forza di Archimede:
tutti i corpi immersi in un liquido tendono a risentire di una forza verso l'alto chiamata
forza di Archimede  che è uguale alla differenza tra il peso teorico di un oggetto e la reazione
vincolare esercitata sull'oggetto. Se un corpo è misurato all'aria:
\begin{gather*}
    P - S = m_x g - m_a g \\
    P - S = m_x g - \delta V_x g\\
    m = \frac{P- S}{g} = m_x - \delta_a V_x = m_x - \delta_a \frac{m_x}{\delta_x}=  m_x \left(1 - \frac{\delta_a}{\delta_x}\right) \\
    m_x = \frac{m}{1 - \frac{\delta_a}{\delta_x}} \approx m\left(1 + \frac{\delta_a}{\delta_x}\right)
\end{gather*}
Anche qui per trovare la densità dell'aria mi rifaccio ai grafici di densità ottenendo che
\begin{align}
    \delta \approx \delta_m \left(1 + \frac{\delta_a}{\delta_x}\right) = \delta_m f_{arc}
\end{align}
All'interno del picnometro invece come cambia? \\
Prendendo le tre misure $m_1, m_2, m_3$ in ognuna si avrà la forza di Archimede dell'aria e
dunque, chiamato $\sigma_x$ la spinta di Archimede sul corpo ($\sigma_x = \delta_a V_x$) e
$m_p$ la massa del picnometro pieno solo di acqua con la sua spinta di Archimede relativa $\sigma_b$:
\begin{gather*}
    m_1 = m_x - \sigma_x \\
    m_2 = (m_x - \sigma_x) + (m_p - \sigma_b) \\
    m_3 = (m_x + m_p - m_a) - \sigma_b
\end{gather*}
Dove $\sigma_x$ è definito come la spinta di Archimede e quindi
$\sigma_x = \delta_a V_x$ e così per le altre masse. Si possono allora
riscrivere le masse in funzione della densità e dunque ricavare la $\delta$. 
\begin{gather*}
    m_1 = m_x \left(1 - \frac{\delta_a}{\delta_x}\right) \ \Rightarrow  \ m_x = \frac{m_1}{1 - \frac{\delta_a}{\delta_x}}\\
    m_2 - m_3 = -\sigma_x + m_a = m_a\left(1 - \frac{\delta_a}{\delta_A}\right) \Rightarrow m_a = \frac{m_2 - m_3}{1 - \frac{\delta_a}{\delta_A}}
\end{gather*}
Allora si ottiene la densità relativa come
\begin{gather*}
    \delta_r = \frac{m_x}{m_a} = \frac{m_1}{1 - \frac{\delta_a}{\delta_x}} \frac{1 - \frac{\delta_a}{\delta_A}}{m_2 - m_3} = \delta_{r_m} \frac{1 - \frac{\delta_a}{\delta_A}}{1 - \frac{\delta_a}{\delta_x}} \\
    \approx \delta_{r_m} \left(1 - \frac{\delta_a}{\delta_A}\right) \left(1 + \frac{\delta_a}{\delta_x}\right) = \delta_{r_m} \left(1 - \frac{\delta_a}{\delta_A} + \frac{\delta_a}{\delta_x}\right)
\end{gather*}
Allora il contributo della spinta di Archimede diventa:
\begin{align}
        f_{arch} = 1 - \frac{\delta_a}{\delta_A} + \frac{\delta_a}{\delta_x}
\end{align}

\section{Il tabulato di dati}
\begin{itemize}
    \item Verificare la taratura della bilancia con almeno $20$ masse diverse;
    \item L'errore di linearità della bilancia sarà la massima deviazione;
    \item L'errore di riproducibilità lo calcoliamo eseguendo $6$ misure misurandone
    lo scarto massimo;
    \item L'errore totale sarà la somma dei due;
    \item Disegnare l'oggetto di cui si vuole ricavare la densità;
    \item Misurare la massa dell'oggetto $5$ volte e determinare valore medio e scarto massimo;
    \item Lo stesso ragionamento si fa per le misure di lunghezza;
    \item Misurare l'effetto della spinta di Archimede;
    \item Ricavare la densità assoluta dell'oggetto dopo aver trovato
    la densità relativa e confrontarle per vedere che siano compatibili 
    e scegliere la più appropriata. 
\end{itemize}

\section{Misura della densità relativa di un liquido}
Per la stima della densità di un liquido attraverso l'utilizzo del
picnometro dobbiamo eseguire una procedura simile a quella adottata
per l'oggetto solido che abbiamo misurato prima:
\begin{enumerate}
    \item Misurare $m_1$ ossia il picnometro vuoto;
    \item Misurare $m_2$ ossia il picnometro con acqua distillata;
    \item Misurare $m_3$ ossia il picnometro riempito con il liquido di cui vogliamo
    trovare la densità relativa.
\end{enumerate}
A questo punto ci basta trovare 
\begin{gather*}
    \delta_{r_m} = \frac{m_3 - m_1}{m_2 - m_1} = \frac{m_x}{m_A} = \frac{\delta}{\delta_A}
\end{gather*}
Se volessimo tenere conto anche della temperatura e della spinta di Archimede allora
basterà correggere questa densità relativa con i rispettivi fattori correttivi.
\begin{align}
    \delta_r = \delta_{r_m} f(T)f_{arch}
\end{align}
E allora possiamo esprimere (ricordando che $m_b$ è la massa del picnometro pieno e
$m_a$ la massa dell'aria spostata )
\begin{gather*}
    \left\{\begin{array}{l}
        m_1 = m_p - m_a - \sigma_p \\
        m_2 = m_p - \sigma_p \\
        m_3 = m_p - m_a + m_x - \sigma_x
    \end{array}\right.
\end{gather*} 
Ossia gli stessi ragionamenti applicati per l'oggetto da misurare.


\chapter{Esperienza volano}
\section{Parte pratica}
\begin{wrapfigure}{r}{0.4\textwidth}
    \begin{tikzpicture}
        \draw (0, 0) circle (1);
        \draw (0, 0) circle (0.3);
        \draw[<->] (0, 1.3) arc (90:180:1.3);
        \draw (-0.2, 0.23) -- (2,2); 
        \draw[dashed, red] (0.2, -0.25) -- (1.85, 1.9);
        \draw (2.05, 1.68) circle (0.3);
        \draw (2.35, 1.68) -- (2.35, 0);
        \draw (2.15, -0.4) rectangle (2.55, 0) node[midway] {$M$};
        \draw[->] (2.35, -0.4) -- (2.35, -1) node[at end, below] {$\vec{a}$};
        \draw[->] (3, 2) -- (3, -3) node[at end, right] {$y$};
        \draw[->] (2.8, -0.2) -- (4.5, -0.2) node[at end, below] {$x$} node[at end, above] {$t_0, y_0$};
        \draw (2.8, -0.8) -- (3.2, -0.8) node[at end, right] {$t_1, y_1$};
        \draw (2.8, -2.5) -- (3.2, -2.5) node[at end, right] {$t_2, y_2$};
    \end{tikzpicture}
\end{wrapfigure}
Nell'esperienza del volano si calcola l'accelerazione di gravità di una piccola
massa collegata ad un volano, ossia una ruota metallica vincolata
ad un asse orizzontale. Mentre il peso scende il filo si avvolge
sul volano e vogliamo allora misurare l'accelerazione di caduta
$\vec{a}$. Il corpo subisce le due forze $T$ e $P$ per trovare 
il vettore e quindi
\begin{gather*}
    \vec{P} + \vec{T} = m\vec{a}   
\end{gather*}  
Prendiamo come ipotesi che $a$ sia costante dato che il momento 
delle forze di attrito è costante. Possiamo ricavare l'accelerazione
con una relazione con il moto rettilineo uniformemente accelerato. \\
Lasciando andare da fermo il corpo e sfruttando le fotocellule si può trovare 
il tempo di caduta e, chiamato $y = 0$ la posizione di un corpo
da fermo $y_1$ e $y_2$ la posizione della seconda e terza fotocellula 
e quindi sfruttando le leggi della cinematica 
\begin{gather*}
    y = y_0 + v_0 t + \frac{1}{2} a t^{2} \\
    \left\{\begin{array}{l}
        y_1 = y_0 + v_0 t_1 + \frac{1}{2} a t_1^{2} \\
        y_2 = y_0 + v_0 t_2 + \frac{1}{2} at_2^{2}  
    \end{array}\right. 
\end{gather*}
Moltiplicando per entrambi i membri $t_2$ per quella sopra e
$t_1$ per quella sotto, si ottiene la seguente:
\begin{gather*}
    y_2 t_1  - y_1 t_2 = \frac{1}{2}a(t_1 t_2^{2} - t_1^{2} t_2)
\end{gather*}

E quindi l'accelerazione sarà
\begin{align}
    a = 2 \frac{y_2 t_1 - y_1 t_2}{t_1 t_2 (t_2 - t_1)}
\end{align}
Possiamo misurare allora le distanze dal metro legato alle
aste portanti dell'apparato: dobbiamo quindi misurare le distanze
$y_1$ e $y_2$ ossia le distanze tra la prima e la seconda fotocellula e
la distanza tra la seconda e la terza fotocellula ed inseriamoli 
calcolatore dell'apparato per trovare $T_1$, $t_2$ e l'accelerazione
senza preoccuparsi dell'incertezza sul tempo in quanto le fotocellule sono
precise a microsecondo.


\section{Analisi statistica dei dati}
Con l'analisi dei dati ora possiamo trovare la distribuzione 
dei valori in modo tale da poter determinare se i dati rispettano una distribuzione
gaussiana oppure no. Abbiamo a disposizione $12$ bin per
definire gli intervalli per l'istogramma, per scegliere il numero
\begin{wrapfigure}{r}{0.4\textwidth}
    \begin{tikzpicture}
        \draw[|-|] (0, 0) -- (0.5, 0) node[at start, below] {$a_{min}$};
        \draw[-|] (0.5, 0) -- (1, 0);
        \draw[-|] (1, 0) -- (1.5, 0);
        \draw[-|] (1.5, 0) -- (2, 0);
        \draw[-|] (2, 0) -- (2.5, 0);
        \draw[-|] (2.5, 0) -- (3, 0) node[at end, below] {$a_{max}$};
        \draw[<->] (1.55,-0.3) -- (1.95, -0.3) node[midway, below] {$\Delta_{bin}$};
    \end{tikzpicture}
\end{wrapfigure}
di bin necessari si osservano i valori dell'accelerazione massima e minima in modo 
tale da scegliere un numero ottimale di intervalli ben definiti.
\begin{wrapfigure}{r}{0.4\textwidth}
    \begin{tikzpicture}
        \draw[->](0, 0) -- (6, 0) node[at end, below] {$a (m/s)$};
        \draw[->](0, 0) -- (0, 4) node[at end, left] {$f_k$};
        \draw[<->] (1.55, -0.3) -- (2.45, -0.3) node[midway, below] {$\Delta_{bin}$};
        \draw (0.5, 0) rectangle (1.5, 1.5);
        \draw[|-|, red] (1, 1.2) -- (1, 1.8);
        \filldraw[red] (1, 1.5) circle (2pt);
        \draw (1.5, 0) rectangle (2.5, 2.5);
        \draw[|-|, red] (2, 2) -- (2, 3);
        \filldraw[red] (2, 2.5) circle (2pt);
        \draw (2.5, 0) rectangle (3.5, 3.5);
        \draw[|-|, red] (3, 3.2) -- (3, 3.8);
        \filldraw[red] (3, 3.5) circle (2pt);
        \draw (3.5, 0) rectangle (4.5, 2.8);
        \draw[|-|, red] (4, 2.5) -- (4, 3.1);
        \filldraw[red] (4, 2.8) circle (2pt);
        \draw (4.5, 0) rectangle (5.5, 2);
        \draw[|-|, red] (5, 1.5) -- (5, 2.5);
        \filldraw[red] (5, 2) circle (2pt);
    \end{tikzpicture}
\end{wrapfigure}
In questo modo è possibile costruire l'istogramma delle frequenze
per ogni intervallo e quindi mi aspetto che dietro a 
quell'istogramma c'è una distribuzione gaussiana che possiamo
descrivere con la migliore stima con valore medio e
deviazione standard. \\
Per verificare tale approssimazione individuiamo i punti medi dei bin e 
li calcoliamo con la seguente
\begin{gather*}
    P(a) = \frac{1}{\sqrt{2\pi}\sigma} \exp \left(-\frac{(a - \overline{a} )^{2} }{2\sigma^{2} }\right)
\end{gather*}
e chiamiamo questo valore $f_{Gauss}$ e valutiamo se 
sono sufficientemente confrontabili; allora se i conti tornano possiamo
allora concludere che non vi sono stati effetti sistematici altrimenti 
dobbiamo prendere in considerazione il fatto che potrebbero
essere sopraggiunti effetti sistematici. \\
Possiamo associare, ad ogni bin, una certa incertezza:
se $N_k$ è il numero di conteggi nel bin possiamo ricavare 
un incertezza del tipo $N_k = N_k \pm \sqrt{N_k}$. Questa
incertezza può essere propagata in $F_k$ e $f_k$ e la consideriamo in $\sigma_{f_k}$
ad ogni canale dell'istogramma possiamo aggiungere un intervallo per
verificare che l'istogramma approssimi la gaussiana.
Possiamo prendere un foglio millimetrato e disegnare l'istogramma
della densità i frequenza e per le proporzioni prendere rapporti semplici.



\chapter{Elasticità}
\section{Introduzione all'elasticità}
Limitandosi allo studio dell'elasticità relativa
ai corpi solidi, in questi corpi è importante il punto di applicazione
della forza ed il concetto di centro di massa. \\
È importante in questo sistema considerare anche il \textbf{momento della forza}
definito come
\begin{align}
    \vec{M}_O = \vec{r} \times \vec{F}   
\end{align}
Sappiamo che in un sistema rigido agiscono, se il corpo è in quiete,
delle forze che producono risultati nulli e nel momento che 
immetto una forza esterna nel sistema  si creerà 
uno spostamento di gran parte del corpo e si formerà una nuova
situazione di equilibrio con un cambiamento delle forze esterne. 
Se la sollecitazione applicata cessa può succedere che
il sistema ritorni alla configurazione di equilibrio precedente o
il sistema non ritorna allo stato iniziale. \\
L'esperienza dell'elasticità ha lo scopo di determinare se un oggetto
si è comportato come un corpo rigido durante tutta la durata dell'esperienza
mediante misurazioni indirette delle posizioni relative tra il corpo rigido e 
gli altri oggetti. Esistono anche i corpi non rigidi, che sono più complicati
e che vanno analizzati caso per caso: un sistema di punti materiali con un corpo
esteso che ha una certa forma, allora se partiamo da una situazione di equilibrio 
e la sua accelerazione è nulla la sommatoria delle forze sulla masse mi dà l'accelerazione.
Se un punto applica una forza esterna quello che avviene è che il corpo
si deforma e cambiano quindi le distanze tra gli oggetti e su di lui sono
applicate delle forze esterne oltre che a quelle interne .
\begin{gather*}
    \vec{a} = \frac{\sum \vec{F}_{int} + \vec{F}_{ext} }{m_i} 
\end{gather*}
In generale quando si ha un corpo esteso sottoposto a deformazioni e quando siamo 
in una situazione di equilibrio allora si bilanciano le forze esterne ed interne e 
se si smette di applicare delle forze esterne allora le forze interne agiranno
sul sistema deformandolo. Se il corpo torna però alle condizioni iniziali dopo
questo agire di forze, allora il corpo è elastico, altrimenti plastico.\\
Quando si applicano delle forze su di un corpo, esso può deformarsi in vari modi 
che andiamo ad analizzare.


\section{Trazioni e compressioni}
\begin{wrapfigure}{r}{0.4\textwidth}
    \centering
    \label{FIg 1.1}
    \caption{Parallelepipedo}
    \begin{tikzpicture}
        \draw (0, 0) -- (1, 0);
        \draw (0, 0) -- (0, 2) node[at start, left] {$\Delta L$} node[midway, left] {$L$};
        \draw(1, 0) -- (1.5, 0.25);
        \draw(1.5, 0.25) -- (1.5, 2.25);
        \draw(0, 2) -- (1, 2);
        \draw (1, 2) -- (1.5, 2.25);
        \draw(0, 2) -- (0.5, 2.25) ;
        \draw(0.5, 2.25) -- (1.5, 2.25);
        \draw (1, 2) -- (1, 0);
        \draw[dashed] (0, 0.5) -- (1, 0.5);
        \draw[dashed] (1, 0.5) -- (1.5, 0.75);
        \draw[->] (0.75, 0) -- (0.75, -1) node[at end, right] {$\vec{F}$};
    \end{tikzpicture}    
\end{wrapfigure}
Nel caso in cui il parallelepipedo con lati $w, h, l$ poggi su di un piano indeformabile 
allora si comprimerà e quindi si modificherà di un certo $\Delta L$ e 
aumenterà la sua sezione, se sottoposto a trazione invece si allungherà causando
una riduzione la sua sezione. Applicando un sistema di forze equivalenti ad una forza
$\perp$ alla base libera, chiamato allora $V = whl$ il volume dell'oggetto e $S = wh$ la sua sezione, la deformazione
che si ha è proprio $\Delta L$. \\
Sappiamo che, dato che i punti all'equilibrio hanno $\vec{a} = 0$, allora la risultante delle
forze deve essere nulla. Quindi ognuno subisce una forza $\vec{F}$ stessa e quindi per
ogni $\vec{F}$ si oppone una forza $-\vec{F}$. Nella realtà un corpo soggetto a trazione
si comprime lungo il lato perpendicolare alla forza e si allunga lungo il lato parallelo. \\
Ogni materiale è dotato di una caratteristica chiamata \textbf{sforzo specifico(normale)}
che si calcola nella seguente maniera:
\begin{align}
    \sigma = \frac{|\vec{F} |}{S}
\end{align}


\begin{wrapfigure}{r}{0.4\textwidth}
    \centering
    \label{FIg 2.1}
    \caption{Il grafico $\sigma $ deformazione relativa}
    \begin{tikzpicture}
        \draw(0, 0) -- (4, 0) node[at end, below] {$\frac{\Delta L}{L}$};
        \draw(0, 0) -- (0, 4) node[at end, left] {$\sigma$};
        \draw(0, 0) -- (0.8, 2);
        \draw[dashed](0, 1.4) -- (0.6, 1.4) node[at start, left] {$\sigma_p$};
        \draw[dashed](0.6, 0) -- (0.6, 1.4);
        \begin{scope}[rotate around={20:(2.2,3)}]
            \foreach \x in {0,...,3}{
              \draw  (0.5+\x*0.15772870662460567,2.5) -- ++(0.15772870662460567,0.1) -- ++ (0, -0.1);
            }
        \end{scope}
        \draw[dashed](0, 2) -- (0.85, 2) node[at start, left] {$\sigma_e$};
        \draw[dashed](0, 2.25) -- (1.3, 2.25) node[at start, left] {$\sigma_s$};
        \draw (1.34, 2.15) .. controls  (1.9, 2.5) ..(2.5, 2.5);
        \draw[dashed] (0, 2.5) -- (2.5, 2.5) node[at start, left] {$\sigma_r$};
        \draw[dashed] (2.5, 2.5) .. controls (3, 2.5) .. (3.5, 2.3);
        \filldraw (0, -0.1) circle (0pt) node[anchor = north] {elasticità};
        \filldraw (3, -0.1) circle (0pt) node[anchor = north] {plasticità};
    \end{tikzpicture}    
\end{wrapfigure}Si esprime in $Pa$, $\frac{N}{m^{2} }$ anche se non è una misura di pressione.
Variando lo sforzo specifico si potrebbero causare
delle deformazioni differenti e finché questo sforzo si mantiene sotto un certo
limite chiamato \textbf{sforzo elastico} allora la deformazione è elastica e 
si indica come $\sigma_{e}$.
Dal grafico, dopo la fase di elasticità, ad una piccola variazione dello sforzo
corrisponde una grande variazione del rapporto $\Delta L /L$, in genere valido per
molti dei materiali.
Tutti i materiali hanno una loro costante elastica per cui le deformazioni sono
proporzionali alla forza impiegata.  (Adottabile se e solo se il valore di $\sigma$ sta
dentro la curva) la forza sarà quindi:
\begin{align}
    F = E S \frac{\Delta L}{L}
\end{align}
$[E] = [kg_p /mm^{2}]$ ossia il \textbf{modulo di Young}, il quale dipende dalla geometria dell'oggetto 
mentre $S$ è la sezione del parallelepipedo considerato.
Questa equazione deriva dal fatto che più la mia sezione è grande e più forza ci devo mettere
mentre pù e lungo e meno forza ci devo mettere per ottenere deformazioni. Ogni materiale ha certi livelli di stress come nel grafico.
\begin{itemize}
    \item Il valore $\sigma_p$ rappresenta il limite di validità dell'equazione precedente ed
    è anche chiamato \textbf{limite di proporzionalità};
    \item Fino al valore $\sigma_e$ il materiale si comporta elasticamente ma la 3.2
    non è più sufficiente per poter descrivere accuratamente il comportamento del materiale;
    \item Il valore $\sigma_s$ si chiama \textbf{limite di snervamento}, oltre questo limite
    a minime sollecitazioni corrispondono grandi allungamenti in quanto il materiale ha perso
    la capacità di resistere alle piccole sollecitazioni;
    \item Il valore $\sigma_r$ rappresenta invece il \textbf{limite di rottura} dopo il quale
    il materiale si spezza.
\end{itemize}
Con l'equazione di Young si ottiene le seguenti formule:
\begin{align}
    \sigma = \frac{\left|\vec{F}\right|}{S} = E\frac{\left| \Delta L \right| }{L}
\end{align}
Si possono descrivere le proprietà dei corpi di cambiare le loro
dimensioni trasverse con il coefficiente di Poisson ($\gamma$). 
Si nota sperimentalmente che dopo una trazione/compressione
a livello fisico succede questo:
\begin{align}
    \frac{\Delta W}{W} = -\gamma \frac{\Delta L}{L} = \frac{\Delta h}{h}
\end{align}
Il rapporto tra allungamento del lato parallelo al cui si applica la forza
è proporzionale all'accorciamento o allungamento del lato perpendicolare all'applicazione
della forza di un fattore $\gamma$. Il volume dell'oggetto considerato è infatti:
\begin{gather*}
    V = whL
\end{gather*} 
Per cui si ottiene che il rapporto $\frac{\Delta V}{V}$ sarà:
\begin{align}
    \frac{\Delta V}{V} = \frac{\Delta L}{L} (1 - 2\gamma)
\end{align}
A seconda del valore di $\gamma$ si hanno diverse situazioni fisiche: $\gamma > \frac{1}{2}$ avremo allora un aumento di volume 
a seguito di un allungamento, mentre se è minore, allora si ha una diminuzione del volume dopo
un allungamento; si nota allora che $\gamma > 0.5$ è empiricamente
impossibile per un corpo isotropo. Se prendessimo infatti una spugna e la riempissimo d'acqua,
esso sarà soggetta a delle forze di Archimede e quindi il corpo non potrà che
diminuire di volume alla minima sollecitazione e quindi su ogni
asse si avrebbe un certo effetto descrivibile come
\begin{gather*}
    \left(\frac{\Delta a}{a}\right)_x = \left(\frac{\Delta b}{b}\right)_y = \left(\frac{\Delta c}{c}\right)_z = - \frac{\sigma}{E}
\end{gather*}
Dato allora che sugli assi lungo il quale la forza risulta trasversa si utilizza il
coefficiente di Poisson, allora si avrà che
l'effetto complessivo delle deformazioni su ogni asse sarà il contributo
della forze di Archimede
\begin{gather*}
    \frac{\Delta a}{a} = -\frac{\sigma}{E} + \gamma\frac{\sigma}{E} + \gamma\frac{\sigma}{E}
\end{gather*} 
Si ottiene allora che la deformazione rispetto al volume sarà data da
\begin{align}
    \frac{\Delta V}{V} = \frac{\Delta a}{a} + \frac{\Delta b}{b} + \frac{\Delta c}{c} = -3(1  - 2\gamma)\frac{\sigma}{E}
\end{align}
Se fosse, data questa definizione di deformazione del volume, che il coefficiente $\gamma > \frac{1}{2}$ allora si avrebbe
che la spugna bagnata si espanderebbe quando sottoposta ad una trazione, ma questo non è
possibile in quanto è un corpo isotropo. \\ 
L'effetto delle forze su di un corpo sottoposto ad una compressione
oppure un estensione risulta essere proprio la somma dei tre contributi rispetto 
ad ogni asse
\begin{align}
    \sigma = \frac{E}{3(1 - 2\gamma)}\frac{\Delta V}{V} = -K \frac{\Delta V}{V}
\end{align}
Possiamo allora definire la costante $K$ come il \textbf{modulo di comprimibilità}
e quindi nella pratica queste considerazioni valgono se e solo se il
corpo è \textbf{isotropo}, in quanto in questi corpi la disposizione
degli elementi nella struttura non dipende da una particolare direzione.

\section{Il modulo di scorrimento, anche chiamata sollecitazione da taglio}
\begin{wrapfigure}{r}{0.3\textwidth}
    \centering
    \caption{Oggetto vincolato su una base}
    \begin{tikzpicture}
        \draw (0, 0) -- (0, 3);
        \draw (0, 0.5) -- (1, 0.5);
        \draw (0, 2.5) -- (1, 2.5);
        \draw(1, 0.5) -- (1, 2.5);
        \draw[dashed](0, 2.5) -- (1, 2.25);
        \draw[dashed](0, 0.5) -- (1, 0.25);
        \draw(0.8, 2.5) arc (0:-15:0.8) node[midway, right] {$\alpha$};
        \draw[->](1, 0.5) -- (1, -0.5) node[at end, below] {$\vec{F}$}; 
    \end{tikzpicture}     
\end{wrapfigure}
Vincoliamo un oggetto ad una base rigida ma l'altra base parallela la
vogliamo non vincolata. In questo caso la base che è sottoposta alla forza parallela rispetto
alla base esterna fa scivolare una base rispetto all'altra. Questo
è immaginabile come una sorta di scorrimento di tanti strati del materiale uno
rispetto all'altro che dà luogo alla variazione della forma dell'oggetto. 
Matematicamente si può esprimere lo sforzo specifico come:
\begin{align}
    \sigma = \frac{F}{S} = G \frac{\left| \Delta x \right| }{L}
\end{align}
dove $G$ rappresenta il \textbf{modulo di scorrimento}, inoltre si ha che:
\begin{gather*}
    G \frac{\left| \Delta x \right| }{L} = G\tan\alpha
\end{gather*}
E si potrebbe approssimare $\tan\alpha \approx \alpha$ per angoli molto piccoli, dunque si ottiene che
Il modulo di Young è dato da
\begin{align}
    E = 2(1 + \gamma)G
\end{align}
Da cui si può ottenere una delle tre date le altre due ed è dimostrabile:
prendiamo un blocco cubico di materiale omogeneo e studiamo cosa succede a questo cubo
se intervengono delle forze di trazione e di compressione:
avremmo accorciamento / allungamento delle forze che lo porterà a mutare la sua forma ed immaginiamo poi un cilindro
diretto lungo la diagonale e dimostriamo che il cilindro subirà una deformazione di taglio che potremmo
relazionare a $G$ ed $E$.
Possiamo dire che, per le considerazioni geometriche e dato che le forze
sono tutte uguali in modulo che:
\begin{gather*}
    \left(\frac{\Delta BC}{BC}\right)_y = \frac{\sigma}{E} \qquad     \left(\frac{\Delta AB}{AB}\right)_x = -\frac{\sigma}{E}\\
    \left(\frac{\Delta BC}{BC}\right)_x = -\gamma \left(\frac{\Delta AB}{AB}\right)_x = \gamma \frac{\sigma}{E} \\
    \Rightarrow \frac{\Delta BC}{BC} = \frac{\sigma}{E} + \gamma \frac{\sigma}{E} = (1 + \gamma) \frac{\sigma}{E} = \frac{\Delta L}{L} = \frac{\Delta AB}{AB}
\end{gather*}
Studiamo ora la deformazione del cilindro, il quale avrà un angolo di scorrimento
di $\alpha = \frac{\sigma}{G}$ e mettiamolo in relazione con quella del cubo : sappiamo 
che $\theta = 2\phi$, ossia l'angolo tra le diagonali del cubo è il doppio dell'angolo tra 
il minimo angolo tra una diagonale ed uno spigolo del cubo: \\
\begin{gather*}
    \theta = \frac{\pi}{2} - \alpha \\
    (L + \Delta L) \tan\phi = L - \delta L 
\end{gather*}

\begin{wrapfigure}{r}{0.35\textwidth}
    \centering
    \begin{tikzpicture}
        \draw[->](-1, -2) -- (0, -2) node[midway, above] {$\vec{F_3}$};
        \draw[->](1, 0) -- (1, 1) node[midway, right] {$\vec{F}_2$};
        \draw[->](3, -2) -- (2, -2) node[midway, above] {$\vec{F}_4$};
        \draw[->](1,  -4) -- (1,  -5) node[midway, right] {$\vec{F}_1$};
        \filldraw (0, 0) circle (0pt) node[anchor = south] {$D$};
        \filldraw (2, 0) circle (0pt) node[anchor = south] {$C$};
        \filldraw (0, -4) circle (0pt) node[anchor = north] {$A$};
        \filldraw (2, -4) circle (0pt) node[anchor = north] {$B$};
        \draw(0, 0) rectangle(2, -4);
        \draw(0, 0) -- (2, -4);
        \draw(0, -4) -- (2, 0);
        \draw(0.5, -1) arc (120:60:1) node[midway, above] {$\theta$};
        \draw(1.5, -3) arc (120:90:1) node[midway, above] {$\phi$};
    \end{tikzpicture} 
\end{wrapfigure}
E allora sostituendo si ottiene che
\begin{gather*}
    \tan \frac{\frac{\pi}{2} - \alpha}{2} = \frac{L - \Delta L}{L + \Delta L} = \frac{1 - \frac{\Delta L}{L}}{1 + \frac{\Delta L}{L}} = \\
    \tan\left(\frac{\pi}{4} - \frac{\alpha}{2}\right) = \frac{1 - \tan \frac{\alpha}{2}}{1 + \tan\frac{\alpha}{2}} \Rightarrow \tan\frac{\alpha}{2} = \frac{\Delta L}{L}
\end{gather*}
Possiamo allora approssimare con Taylor la tangente e dunque ottenere che 
\begin{gather*}
    \frac{\sigma}{2G} \approx \frac{\alpha}{2 } = \frac{\Delta L}{L} = (1 + \gamma) \frac{\sigma}{E}
\end{gather*}
E allora si ottiene la tesi eguagliando quella a destra con quella a sinistra.
Inoltre si nota anche il corollario, ossia che
\begin{gather*}
     0 < \gamma < \frac{1}{2} \\
    \ \Longrightarrow \ E \approx 2G \sim 3G
\end{gather*}
Ossia $E$ è contenuto nell'intervallo tra $2G$ e $3G$. Qualsiasi valore
al di fuori di questo intervallo è impossibile per quanto dimostrato prima con le
sollecitazioni da torsione e trazione.

\section{Flessione}
\begin{wrapfigure}{R}{0.4\textwidth}
    \centering
    \label{Fig 2.2}
    \caption{La flessione}
    \begin{tikzpicture}
        \draw(0, 0) -- (0, 4);
        \draw (0, 2) -- (3, 2);
        \draw(0, 3) -- (3, 3);
        \draw(3,3) -- (3, 2);
        \draw[->](3, 2) -- (3, 2) node[at end, right] {$\vec{F}$};
        \draw[|-|] (0, 3.5) -- (3, 3.5) node[midway, above] {$L$};
        \draw[dashed](0, 3) .. controls (1.5, 3) .. (2.75, 2.25);
        \draw[dashed](0, 2) .. controls (1.5, 2) .. (2.25, 1.25);
        \draw[dashed](2.75, 2.25) -- (2.25, 1.25);
        \draw[dashed](0, 2.5) -- (1.75, 2.5);
        \draw(2.5, 1.75) -- (1.25, 2.5);
        \draw(1.75, 2.5) arc (0:-30: 0.5) node[midway, right] {$\phi$};
    \end{tikzpicture}    
\end{wrapfigure}
Considerando un cilindro di materiale omogeneo ed isotropo con una base
vincolata ad una superficie rigida. Se applico un sistema di forze tali
da rendere risultante nulla ma da creare un momento diverso da $0$ si osserverà che il cilindro si flette. \\
La forma che assume l'asse longitudinale tende ad assumere la deformazione
di un arco di circonferenza e si allungheranno le fibre sopra l'asse longitudinale
e si comprimeranno quelle sotto. Le fibre che compongono l'asse latitudinale invece rimangono
invariate. 
Posso allora definire \textbf{l'angolo di flessione} $\phi$ e la tangente $\perp$ alla base che interseca l'asse
ortogonale dal cilindro con un angolo che è sempre $\phi$. \\
\begin{align}
    \phi = \frac{1}{E} \frac{4}{\pi} \frac{L}{r^{4} }m
\end{align}
Per dimostrare la validità di questa formula possiamo considerare 
una fetta sottile del cilindro prima della deformazione: dopo la deformazione non sarà
più un cilindro ma un trapezio. \\
Prendo allora un asse $y$ $\perp$ all'asse delle fibre e parallelo al vincolo:
l'angolo ottenuto dal prolungamento delle due superfici inclinare lo
chiamo $\alpha$ e possiamo allora esprimere in maniera quantitativa
\begin{gather*}
    \alpha \approx \frac{l}{R}
\end{gather*}
Dove $R$ è il raggio della circonferenza centrata nell'origine $\alpha$ che identifica 
la posizione delle fibre neutre. Possiamo anche notare che
\begin{gather*}
    \alpha \approx \frac{l + \Delta l}{R + y}
\end{gather*}

\begin{wrapfigure}{r}{0.4\textwidth}
    \centering
    \caption{}
    \begin{tikzpicture}
        \draw[cyan, very thick](0, 0) -- (1.5, 0);
        \draw[cyan, very thick](0, 0) -- (0.5, -2);
        \draw[cyan, very thick](0.5, -2) -- (1, -2);
        \draw[cyan, very thick](1, -2) -- (1.5, 0);
        \draw[thin, ->](0.75, -4) -- (0.75, 1) node[at end, right] {$y$};
        \draw[thin] (0.5, -2) -- (0.75, -3);
        \draw[thin] (1, -2) -- (0.75, -3);
        \draw(0.2, 0) rectangle (1.3, -2);
        \draw[|-|] (2, -1) -- (2, -3) node[at end, right] {$R$};
        \draw[|-|] (2.5, -1) -- (2.5, -2) node[midway, right] {$r$};
        \draw[dashed , thin] (0.2, -1) -- (2.5, -1);
        \draw[|-|] (0, 0.25)  -- (1.5, 0.25) node[midway, above] {$l$}; 
        \draw[->] (-1.5, -1) -- (3.5, -1) node[at end, above] {$x$};
        \draw[very thin, dashed] (-1, 0) rectangle (3, -2);
    \end{tikzpicture}    
\end{wrapfigure}
Dove $y$ è un quota qualsiasi sopra le fibre neutre che può essere positiva o negativa.
Posso allora semplificare la relazione:
\begin{gather*}
    \frac{l}{R} = \frac{l + \Delta l}{R + y} \Rightarrow \frac{\Delta l}{l} = \frac{y}{R}
\end{gather*}
Allora, tramite il \textbf{modulo di Young} posso esprimere questa mettendola 
in relazione con lo sforzo delle deformazioni. Considero la faccia
del cilindro ed isolo una piccola striscia $ds$ sopra la sezione delle fibre neutre
le deformazioni allora saranno causate dalle forze infinitesime $d\vec{F}$ e posso
quindi scrivere:
\begin{gather*}
    \frac{dF}{ds} = E \frac{\Delta l}{l} \Rightarrow \frac{dF}{ds} \approx E \frac{y}{R}
\end{gather*} 
Individuiamo allora il braccio della forza e quindi possiamo esprimere il
momento infinitesimo come
\begin{gather*}
    d\vec{m} = \vec{r} \times d\vec{F} \ \Rightarrow \  dm = rdF\sin\theta = rd\vec{F} \approx y dF 
\end{gather*}

\begin{wrapfigure}{r}{0.4\textwidth}
    \centering
    \caption{}
    \begin{tikzpicture}
        \draw(0, 0) -- (1.5, 0);
        \draw(0, 0) -- (0.5, -2);
        \draw(0.5, -2) -- (1, -2);
        \draw(1, -2) -- (1.5, 0);
        \draw[cyan, very thick, ->] (1.3, -0.8) -- (2, -1) node[at end, above] {$d\vec{F}$};
        \draw[cyan, very thick, ->] (1.25, -1) -- (1.3, -0.8) node[midway, left] {$\vec{r}$};
    \end{tikzpicture}    
\end{wrapfigure}
E allora possiamo, sostituendo, ottenere il momento infinitesimo come
\begin{gather*}
    dm = E \frac{y^{2} }{R} ds 
\end{gather*}
Ma questa è solo una parte del momento dovuta alla deformazione sulla piccola
striscia $ds$ considerata, dobbiamo allora considerare anche tutti i momenti infinitesimi
\begin{gather*}
    m = \int \ dm = \int \frac{Ey^{2} }{R} \ ds = \frac{E}{R}\int y^{2} \ ds = \frac{EI}{R} 
\end{gather*}
Ossia il \textbf{momento areolare di inerzia} (riferito ad una superficie) ed è una grandezza
dipendente dalla forma e non dal materiale di un oggetto:
\begin{align}
    I = \int_S y^{2} \ ds 
\end{align} Si ottiene allora l'equazione
di \textbf{Eulero-Bernoulli}:
\begin{align}
    m = \frac{EI}{R}
\end{align}
In questo caso è stata ricavata per il cilindro ma potremmo ricavarla per qualsiasi forma; tornando
indietro osservo che $\alpha = \phi$ e quindi posso esprimerlo come
\begin{gather*}
    \phi = \frac{L}{R} \ \Rightarrow \ \phi = \frac{Lm}{E} \frac{4}{\pi} \frac{1}{r^{4} }\qed
\end{gather*}
Per calcolare i momenti areolari di inerzia occorre scegliere opportunamente
la sostituzione da fare per $ds$, nel caso del parallelepipedo si sa che la piccola
superficie $ds$ è data da $dxdy$:
\begin{gather*}
    I = \int_S y^{2} ds = \int_{-\frac{a}{2}}^{\frac{a}{2}}\ dx \ \int_{-\frac{b}{2}}^{\frac{b}{2}}y^{2} \ dy = \frac{1}{12} ab^{3}      
\end{gather*}
mentre nel caso del cilindro si ottiene che
la sezione infinitesima è data da $ds = 2dy |x(y)|$ e quindi sapendo che
$x^{2} = r^{2} - y^{2}$ si ottiene: $ds = 2 \sqrt{r^{2} - y^{2} }dy$. 
\begin{gather*}
    I = 2\int_{-r}^{r} y^{2} \sqrt{r^{2} - y^{2}  }dy = \dots = \frac{1}{4}\pi r^{4}    
\end{gather*}

\section{Misura del modulo di Young e apprato pratico}
Non sempre il raggio di curvatura esiste, talvolta devo considerare raggi di curvatura
locali avendo $R = R(x)$. Presa quindi una curva generica posso definire
una \textbf{circonferenza osculatrice} che avrà raggio di curvatura $R(x)$ 
dipendente dalla coordinata $x$ e quindi dall'analisi sappiamo che
\begin{gather*}
    \frac{1}{R(x)} = \frac{\left| \frac{d^{2} y}{dx^{2} } \right| }{\left(1 + \left(\frac{dy}{dx}\right)^{2} \right)^{\frac{7}{2}} } 
\end{gather*}
\begin{wrapfigure}{r}{0.4\textwidth}
    \centering
    \caption{}
    \begin{tikzpicture}
        \draw[->](0, 0) -- (5, 0) node[at end, below] {$x$};
        \draw[->](0, -3) -- (0, 2) node [at end, right] {$y$};
        \draw(2, -0.25) arc (-60:-90: 4);
        \draw(2, 0.25) arc (-60:-90: 4);
        \draw(2, -0.25) -- (2, 0.25);
        \draw[cyan, ->](2, 0.25) -- (2, 1.25) node[at end, right] {$\vec{N}$};
        \draw[|-|] (0, -1) -- (2, -1) node[midway, below] {$\frac{L}{2}$};
        \draw[->, cyan] (0, -0.75) -- (0, -1.75) node[at end, right] {$\vec{P}$};
    \end{tikzpicture}    
\end{wrapfigure}
Durante le misure si vuole studiare la curva che assume la superficie delle
fibre mentre è sottoposta a sforzo per calcolare il modulo di Young e dunque prendiamo una sbarretta a
base rettangolare. Essa è appoggiata a $2$ supporti e quindi avrà due reazioni vincolari
e la pressione sarà dovuta alle forze schematizzate come $N = \frac{P}{2}$ e studieremo solo metà del corpo per 
semplicità e quindi prendendo un sistema di assi in modo tale il cui centro sia proprio al centro
della sbarretta, si può scegliere un vettore $\vec{r}$ tale che parta da un punto
variabile sull'asse delle $x$ ed il modulo del momento associato alla forza sarà
\begin{gather*}
    m(x) = rN\sin\theta = rN = \frac{P}{2}(\frac{L}{2} - x)
\end{gather*} 
E sappiamo, dalla flessione, che il momento si può anche esprimere in relazione al modulo
di Young come
\begin{gather*}
    m(x) \approx \frac{EI}{R(x)} \approx EI \cdot  \frac{d^{2} y}{dx^{2}  }
\end{gather*}
E quindi si ottiene l'equazione differenziale 
\begin{gather*}
    \frac{d^{2} y}{dx^{2} } = \frac{P}{2EI}\left(\frac{L}{2} - x\right)
\end{gather*}
Prima di poter risolvere integrando è necessario mettere delle condizioni, ossia noi vogliamo che
quando non c'è sforzo, la nostra curvatura sia zero e che quindi si abbia il minimo 
della curvatura stessa per definizione. Allora posso integrare come segue:
\begin{gather*}
    \int_{0}^{x} \frac{d^{2} y}{dx^{2} } \ dx = \int_{0 }^{x} \frac{P}{2EI}\left(\frac{L}{2} - x \right) \ dx \Rightarrow \frac{dy}{dx} = \frac{P}{2EI} \left(\frac{L}{2}x - \frac{x^{2} }{2}\right)
\end{gather*}
Integrando nuovamente allora si ottiene 
\begin{gather*}
    \int_{0}^{x} \frac{dy}{dx} = \int_{0}^{x} \frac{P}{2EI} \left(\frac{L}{2}x - \frac{x^{2} }{2}\right) \ dx \ \Rightarrow \ y(x) = \frac{P}{2EI} \left(\frac{L}{4}x^{2} - \frac{x^{3} }{6} \right)
\end{gather*}
Ossia una curva che è la combinazione tra una parabola ed una cubica. Posso ricavarmi il valore
a $L/2$:
\begin{gather*}
    y(\frac{L}{2}) = \frac{L^{3} }{4Eab^{3} }P
\end{gather*}
Poiché ho sostituito il momento areolare di inerzia della sbarretta che già conoscevo. 
La flessione allora si può calcolare come l'abbassamento della sbarretta ossia
\begin{align}
    f = \frac{L^{3} }{4Eab^{3} }P \approx cP
\end{align}


\section{Parte pratica e stime a priori}
\subsection{Apparato sperimentale}
\begin{wrapfigure}{r}{0.4\textwidth}
    \centering
    \label{Fig 4.2}
    \caption{Il modello della deformazione della sbarretta}
    \begin{tikzpicture}
        \draw(0, 0) -- (0.5, 0);
        \draw(0, 0) -- (0.25, 0.5);
        \draw(0.25, 0.5) -- (0.5, 0);
        \draw(3, 0) -- (3.5, 0);
        \draw(3, 0) -- (3.25, 0.5);
        \draw(3.25, 0.5) -- (3.5, 0);
        \draw(0.25, 0.5) -- (1.75, 3);
        \draw(1.75, 3) -- (3.25, 0.5);
        \draw[dashed](1.75, 3) -- (1.75, 0.6);
        \draw[|-|, very thick ](1.75, 0.6) -- (1.75, 0.1) node[at start, left] {$f$};
        \draw(0.25, 0.5) arc (240:300:3);
        \draw(0.3, 0.6) arc (240:300:2.9);
        \draw[|-|](0.2, 0.7) -- (0.1, 0.5) node[midway, left] {$y$};
        \draw[|-|](0.25, -0.5) -- (3.25,-0.5) node[midway, below] {$L$};
        \draw (1.2, 2) arc (240:270:1) node[midway, below] {$\alpha$}; 
    \end{tikzpicture}    
    \smallskip\par
    \begin{tikzpicture}
        
    \end{tikzpicture}
\end{wrapfigure}
L'apparato sperimentale in laboratorio è costituito da dei supporti che possiedono dei coltelli
per poter appoggiare la sbarretta sopra in moto da ottimizzare la perpendicolarità delle
reazioni vincolari e dove $L$ è la distanza tra questi due coltelli e non la
lunghezza della sbarretta. Possiamo aggiungere dei pesi su questa sbarretta appoggiandoli sul
cavaliere, ossia una sorta di supporto attaccato al centro della sbarretta. Con ogni
pesetto possiamo allora misurare di quanto vale $f$ attraverso il truschino. \\
Il truschino è uno strumento formato da una scala legata ad una base pesante con piedini
regolabili e con una livella a bolla per garantire la verticalità. La scala del 
nonio cinquantesimale garantisce una precisione sulle misure di $0.02mm$ e presenta un 
laser che può essere regolato in intensità per poter centrare il fascio di luce 
al centro della croce del cavaliere.
Per le misure di peso si utilizza la bilancia elettronica e per velocizzare l'esperienza si 
leggono i grammi come grammi peso.  \\
Possiamo allora eseguire le misure di variazione di quota con più pesi ottenendo
una tabella altezza-peso e tenendo conto dell'altezza massima $h_0$ la cui incertezza è 
calcolata sullo scarto massimo di $10$ misure. 
Possiamo allora trovare la relazione che lega la flessione
al peso come
\begin{gather*}
    f_i = h_0 - h_i \approx cP_i
\end{gather*}

\begin{wrapfigure}{r}{0.4\textwidth}
    \centering
    \label{Fig 4.3}
    \caption{Grafico del fit lineare delle misure}
    \begin{tikzpicture}
        \draw(0, 0) -- (3, 0) node[at end, below] {$P$};
        \draw(0, 0) -- (0, 3) node[at end, left] {$h$};
        \draw[|-|](0.5, 2.75) -- (0.5, 2.5);
        \draw[|-|](1, 2.25) -- (1, 2);
        \draw[|-|](1.5, 1.75) -- (1.5, 1.5);
        \draw[|-|](2, 1.25) -- (2, 1);
        \draw[|-|](2.5, 0.75) -- (2.5, 0.5);
    \end{tikzpicture}    
\end{wrapfigure}
Successivamente si riportano i valori su di un grafico inserendo le opportune e uguali
incertezze. In maniera grafica si può ricavare il coefficiente
$c$ e da $c$ si ottiene con le formule inverse il modulo di Young.
Si può ottenere $c$ in modo grafico oppure col metodo dei minimi quadrati e scegliendo 
poi il valore migliore. Tuttavia, perché queste considerazioni possano 
essere valide, bisogna accertarsi di trovarsi nella regione elastica del materiale
verificando quindi che la sbarretta ritorni alla sua posizione originale. Per
determinare quindi il peso massimo dobbiamo eseguire una stima iniziale del peso 
massimo utilizzabile e verificare poi che, quando ho tolto tutte le masse dal 
cavaliere, l'altezza $h_0$ sia compatibile con quella inizialmente considerata. Se non lo è
si va incontro a due possibilità:
\begin{itemize}
    \item La stima a priori del peso era troppo alta e ho deformato la sbarretta;
    \item In qualche modo si è modificato l'apparato sperimentale durante le 
    misure e questo ha portato ad una falsa lettura dell'altezza iniziale al termine dell'esperienza. 
\end{itemize}

\subsection{Stima a priori del peso massimo utilizzabile}
Prendendo la sbarretta sui coltelli, possiamo approssimare la sua
curvatura ad un arco di circonferenza e dunque ottenere l'individuazione
dei diversi raggi delle fibre e sappiamo che l'angolo dell'arco di circonferenza
individuato dalla sbarretta piegata è
\begin{gather*}
    \frac{L}{R} = \phi = \frac{L + \Delta L}{R + y} \ \Rightarrow \ \frac{\Delta L}{L} = \frac{y}{R}
\end{gather*}
E quindi si ha che con il caso più estremo, ossia quello delle fibre della
parte sotto della sbarretta, e quindi per come si è schematizzato il problema possiamo 
sfruttare il teorema di pitagora ottenendo che
\begin{gather*}
    R^{2} \approx (R - f)^{2} + \left(\frac{L}{2}\right)^{2}   
\end{gather*}
E dunque se $f^{2} << RL$ posso trascurare questo termine e ottenere che
il raggio di curvatura è dato proprio da
\begin{gather*}
    \frac{1}{R} \approx \frac{8f}{L^{2} }
\end{gather*}
E allora 
\begin{gather*}
    \frac{\Delta L}{L} = \frac{b}{2}\cdot \frac{8f}{L^{2} }, \qquad y = \frac{b}{2}
\end{gather*}
Posso allora ottenere la relazione che lega la variazione di lunghezza di una fibra
allo sforzo ed al modulo di Young:
\begin{gather*}
    \frac{\Delta L}{L} = \frac{\sigma }{E } \ \Rightarrow  \ \sigma_P = \frac{4bfE}{L^{2} }
\end{gather*}
E quindi si ottiene che il peso massimo della sbarretta con una certa distanza tra i coltelli
ed un certo spessore ed una certa lunghezza è data da
\begin{gather*}
    P < \frac{ab^{2} }{L}\sigma_P
\end{gather*}
Si utilizza il coefficiente di proporzionalità invece che quello elastico
per evitare di fare misure con pesi troppo vicini al limite di elasticità.


\subsection{Stime delle grandezze in gioco nell'esperienza}
Si può, inoltre, stimare a priori altri valori come $c$ dalla formula
\begin{gather*}
    h = h_0 - cP \ \Rightarrow \ c = \frac{h_0 - h}{P}
\end{gather*}
E quindi ottenere l'errore relativo a priori come
\begin{gather*}
    \frac{\Delta c}{c} = \frac{2\Delta h_0}{h - h_0} + \frac{\Delta P}{P}
\end{gather*}
Così come per il modulo di Young:
\begin{gather*}
    \frac{\Delta E }{E} = 3 \frac{\Delta L}{L} + \frac{\Delta a}{a} + 3 \frac{\Delta b}{b} + \frac{\Delta c}{c}
\end{gather*}

\chapter{Pendolo di torsione}
\section{Sollecitazione di torsione}
\begin{wrapfigure}{r}{0.4\textwidth}
    \centering
    \caption{Sollecitazione di torsione}
    \begin{tikzpicture}
        \draw (0, 0) circle (1);
        \draw[->](0, 1) -- (1,1) node[at end, right] {$\vec{F}$};
        \draw[->](0, - 1) -- (-1, -1) node[at end, left] {$-\vec{F}$};
    \end{tikzpicture}    
\end{wrapfigure}
Consideriamo un cilindro e sottoponiamolo ad una serie di forze che rendono 
una sua base vincolata e l'altra sottoposta ad una serie di forze 
che rendano la risultante nulla ed un momento torcente diverso da zero.
Immaginando di sezionare infinite volte il cilindro possiamo immaginare 
elementi differenziali di superficie che ruotano l'uno rispetto all'altro. Dalla prima
e l'ultima superficie si individua l'angolo di rotazione $\theta$ 
e, chiamando $L$ la lunghezza del cilindro e $R$ il suo raggio, prendo un guscio
cilindrico interno al cilindro originale con la stessa lunghezza e con $r$ il suo raggio. \\
Posso allora prendere un blocchetto di questo guscio che avrà spessore
$\delta r$ e altezza $\delta l$ e a seguito della deformazione (di tipo taglio) il rettangolino
varierà la sua forma, così che è possibile individuare il suo angolo di deformazione $\alpha$ come
\begin{gather*}
    \theta = \frac{l}{r} \qquad \alpha \approx \frac{l}{L} \ \Longrightarrow \ \alpha \approx \frac{\theta r}{L}
\end{gather*}
Nella sollecitazione di taglio avremo due $\delta \vec{F}$ applicate alle due
facce che portano ad uno sforzo di
\begin{gather*}
    \sigma = \frac{\delta F}{\delta r \delta l}
\end{gather*} 

\begin{wrapfigure}{r}{0.3\textwidth}
    \centering
    \caption{Il blocchetto cilindrico considerato}
    \begin{tikzpicture}[scale=2]
        \draw(0, 0) -- (1, 0);
        \draw(0, 0) -- (0, 1);
        \draw(1, 0) -- (1, 1);
        \draw(0, 1) -- (1, 1);
        \draw(0, 1) -- (0.25, 1.25);
        \draw(0.25, 1.25) -- (1.25, 1.25);
        \draw(1, 1) -- (1.25, 1.25);
        \draw(1.25, 1.25) -- (1.25, 0.25);
        \draw(1, 0) -- (1.25, 0.25);
        \draw[|-|](-0.25, 0) -- (-0.25, 1) node[midway, left] {$\delta l$};
        \draw[|-|](1.25, 0) -- (1.5, 0.25) node[midway, below] {$\delta r$};
        \draw[cyan](0, 0) -- (1.25, 0.63);
        \draw[cyan](0, 0) -- (0, 1);
        \draw[cyan](0, 1) -- (1.25, 1.63);
        \draw[cyan](1, 0.5) -- (1, 1.5);
        \draw[cyan](1.25, 1.63) -- (1.25, 0.63);
        \draw[cyan, ->](1.3, 1) -- (1.3, 1.63) node[midway, right] {$\delta \vec{P}$ };
        \draw[cyan, ->](-0.1, 0.63) -- (-0.1, 0) node[midway, left] {$\delta \vec{P}$ };
        \draw[cyan](0.5, 0.25) arc (30:0:0.5) node[midway, right] {$\alpha$};
    \end{tikzpicture}    
\end{wrapfigure}

Possiamo ora considerare quantitativamente il momento rispetto al contributo infinitesimo
delle forze applicate al blocchetto come:
\begin{gather*}
    \delta m_{\delta l \delta r} = \delta F r \sin \frac{\pi}{2} = \delta Fr = \sigma r \delta r \delta l
\end{gather*}
Sommiamo a tutti i contributi $\delta l = 2\pi r^{2} $ (che è una approssimazione):
\begin{gather*}
    \delta m_{ \delta r} = \sigma \delta r 2\pi r^{2} 
\end{gather*}
Per trovare allora il momento complessivo per passare da un singolo guscio all'intera
superficie integrando, dato $\sigma$ che dipende da $r$ infatti possiamo dire che
per deformazioni abbastanza piccole
\begin{gather*}
    \sigma \approx G\alpha = G \frac{\theta r}{L} \ \Rightarrow \ \delta m_{\delta r} = 2\pi \frac{G\theta}{L}r^{3} \delta r 
\end{gather*}
Possiamo ora integrare rispetto al raggio infinitesimo:
\begin{gather*}
    m = \frac{2G\theta\pi }{L} \int r^{3} \delta r 
\end{gather*}
Si ottiene allora che
\begin{align}
    \theta = \frac{1}{G}\frac{2}{\pi} \frac{L}{r^{4} }m
\end{align}
Per deformazioni sufficientemente piccole l'angolo di torsione
risulta direttamente proporzionale al momento applicato.

\section{Misura empirica della torsione}
La misura sperimentale verrà svolta attraverso il pendolo di torsione:
Esso consiste in un filo appeso e legato ad un oggetto solidale
chiamato \textbf{equipaggio mobile} a cui posso applicare un certo momento che torce
il filo di un certo angolo $\theta$. \\
L'istante prima del rilascio il sistema è in equilibrio poiché il momento e le
forze elastiche sono bilanciate e quindi, sapendo che il momento sull'asse
è dato dal momento precedente, possiamo esprimere il momento risultante come 
\begin{gather*}
    m_z = - G \frac{\pi}{2} \frac{R^{4} }{L} \theta = -c_t \theta
\end{gather*}
Con le equazioni cardinali della dinamica si ha 
\begin{gather*}
    \left\{\begin{array}{l}
        \frac{d\vec{P} }{dt} =  \vec{F} \\
        \frac{d\vec{L} }{dt} = \vec{m} - \vec{v} \times \vec{P}    
    \end{array}\right.
\end{gather*}

\begin{wrapfigure}{r}{0.4\textwidth}
    \centering
    \caption{}
    \begin{tikzpicture}
        \draw(0, 0) circle (2.8);
        \filldraw(0, 0) circle (1pt) node[anchor =east] {$C$};
        \draw(0, 0) -- (3, 0);
        \draw(0, 0) -- (3, 3);
        \draw(1, 1) arc (45:0:1.48) node[midway, right] {$\theta$};
        \draw[cyan, ->](2, 2) -- (1, 1) node[at end, above] {$\vec{F}_N$};
        \draw[cyan, ->](2, 2) -- (0.6, 3.4) node[at end, above] {$\vec{v} $};
        \draw[cyan, ->](2, 2) -- (1, 3) node[at end, above] {$\vec{F}_T$}; 
    \end{tikzpicture}    
\end{wrapfigure}
Posso utilizzare la seconda equazione e verificare che possiamo 
prendere un polo facile per eliminare il secondo termine.
Prendo un punto materiale su una circonferenza di centro $C$ e orientato
con un angolo $\theta$ rispetto ad un asse fisso. Esso avrà la velocità
$\vec{v}$ data da una forza $\vec{F} = \vec{F}_T + \vec{F}_N  $. \\
Calcoliamo il momento 
\begin{gather*}
    \vec{m} = \vec{r} \times \vec{F} = \vec{r} \times \vec{F}_T + \vec{r} \times \vec{F}_N \\
    \vec{m} = m_z \hat{k} \\
    m_z = m = rF_{\perp} \sin \frac{\pi}{2} = rm a_T         
\end{gather*}
Dato che imponiamo il moto circolare 
\begin{gather*}
    S = \theta r\  \Rightarrow \ \frac{ds}{dt} = \dot{\theta}r + \theta r \\
    a_T = \frac{d^{2} s}{dt^{2} } = \ddot{\theta}r + \theta \dot{r} = \ddot{\theta}r  \ \Rightarrow \ m_z = m r^{2} \ddot{\theta} 
\end{gather*}
Possiamo allora provare a considerare la quantità di moto
\begin{gather*}
    \vec{L} = \vec{r} \times m \vec{v}    \\
    L = L_z = rm v \sin \frac{\pi}{2} = mrv = mr^{2} \dot{\theta} \\
    \vec{L} = mr^{2}\dot{\theta} \hat{k}    
\end{gather*}
Notiamo che 
\begin{gather*}
    \frac{d \vec{L} }{dt} = mr^{2} \ddot{\theta}\hat{k} = \vec{m}    \qedhere
\end{gather*}
Sapendo che il momento di inerzia è  
\begin{gather*}
    I = \sum m_i r_i^{2} 
\end{gather*}
E quindi posso scrivere che
\begin{gather*}
    \left\{\begin{array}{l}
        L_z = I \dot{\theta} \\
        m_z = I \ddot{\theta}
    \end{array}\right.
\end{gather*}
Studiamo allora il moto dell'equipaggio mobile 
che è dato dalla derivata rispetto alla quantità di moto:
\begin{gather*}
    \frac{dL_z}{dt} = m_z = -c_t \theta \\
    L_z = I_{em} \dot{\theta} \\
    I_{em} \ddot{\theta} = -c_t \theta \ \Rightarrow \ 
\end{gather*}
\begin{align}
    I_{em} \ddot{\theta} + c_t \theta = 0
\end{align}
Sull'equipaggio mobile sotto al soffitto sono presenti dei morsetti 
che dobbiamo escludere quando si prende la misura della lunghezza e la risoluzione dell'equazione è
\begin{gather*}
    \theta(t) = A \cos(\omega t - \gamma)
\end{gather*}
E' quindi una funzione periodica e possiamo studiare la derivata sapendo 
che $\omega = \frac{2\pi}{T}$ prende il nome di pulsazione:
\begin{gather*}
    \dot{\theta}(t) = -A \omega \sin(\omega t - \gamma) \\
    \ddot{\theta}(t) = -A \omega^{2} \cos(\omega t - \gamma) = -\omega^{2}\theta(t)  
\end{gather*}
Se sostituisco le due soluzioni nell'equazione differenziale si ottiene che
\begin{gather*}
    \omega = \sqrt{\frac{c_t}{I_{em}}} \ \Rightarrow \ T = 2\pi \sqrt{\frac{I_{em}}{c_t}} 
\end{gather*}
Misurando il periodo di oscillazione del pendolo di torsione 
le caratteristiche geometriche dell'oggetto si ottiene il modulo di scorrimento
E quindi :
\begin{gather*}
    \theta(t) = \theta_0 \cos(\omega t)
\end{gather*}
Per studiare il moto occorre però ottenere una misura del momento di inerzia
rispetto all'asse del filo.


\section{Misura della costante di torsione}
Possiamo ora misurare la costante di torsione ricavandola dal periodo e dal momento torcente, considerando anche
il fatto che il momento di inerzia è additivo. 
Possiamo infatti considerare $I_{tot}$ come somma di due momenti di inerzia diversi riferiti a più
parti separate del sistema e se modifichiamo il pendolo di torsione aggiungendo 
un pezzo, il nuovo momento di inerzia sarà allora la somma dei due momenti di inerzia
e quindi si può sfruttare i periodi ed ottenere
\begin{gather*}
    \left\{\begin{array}{l}
        T_1 = 2\pi \sqrt{\frac{I_{em}}{c_t}} \\
        T_2 = 2\pi \sqrt{\frac{I_{em} + I_{an}}{c_t}}  
    \end{array}\right.
\end{gather*}
Il \textbf{momento di inerzia} dei materiali è facilmente ricavabile date le loro
geometrie elementari e dunque si può trovare $\vec{m}$ per trovare $c_t$. 
Allora svolgendo il sistema otteniamo: 
\begin{gather*}
    \frac{T_2^{2} }{T_1^{2} } = \frac{I_{em} + I_{an}}{I_{em}} = 1 + \frac{I_{an}}{I_{em}}
\end{gather*} 
Facendo allora il rapporto tra le equazioni e dunque
\begin{gather*}
    I_{em} = I_{an} \frac{T_1^{2} }{T_2^{2} - T_1^{2} }
\end{gather*}
E quindi posso sostituire nell'equazione di $c_t$:
\begin{align}
    c_t = 4\pi^{2}\frac{I_{em}}{T_1^{2} } = 4\pi^{2} \frac{I_{an}}{T_2^{2} - T_1^{2} }  
\end{align}
Possiamo allora ricavare la costante $G$
\begin{align}
    G = \frac{2}{\pi} \frac{L}{R^{4} }4\pi^{2} \frac{I_{an}}{T_2^{2} - T_1^{2} } = 8\pi\frac{L}{R^{4} }\frac{I_{an}}{T_2^{2} - T_1^{2} } 
\end{align}


\subsection{Calcolo del momento di inerzia del cilindro pieno e cavo}
Prendiamo un cilindro pieno di raggio $R$ e altezza $h$, allora data la relazione
\begin{gather*}
    I = \sum m_ir_i^{2} 
\end{gather*}
è possibile passare dal caso discreto al caso continuo uscendo dall'integrale
e ricordando la densità, possiamo studiare localmente gli elementi
infinitesimi e quindi posso dire che il volume infinitesimo del
cilindro è proprio
\begin{gather*}
    I = \int_M r^{2} \ dm = \int_{V} r^{2} \delta \ dV = \delta \int r^{2} \ dV   
\end{gather*}
Immaginando allora di prendere un guscio cilindrico infinitesimo
possiamo dunque calcolare il suo volume e sarà di
\begin{gather*}
    d V = 2h\pi r dr  \ \Rightarrow \ \delta \int_V r^{2} \ dV = 2\delta \int_{0}^{R} r^{2} h \pi r \ dr = \frac{1}{2}mr^{2}  
\end{gather*}
Calcoliamo ora il momento di inerzia di un cilindro cavo:
\begin{gather*}
    V = h\pi (R_e^{2} - R_i^{2}  ) \\
    I = \delta \int_{V} r^{2} \ dr = \delta \int_{R_i}^{R_e} r^{2} h 2 \pi r \ dr  
\end{gather*}
E quindi
\begin{align}
    \frac{1}{2}m (R_e^{2} + R_i^{2} )
\end{align}


\section{Parte pratica}
\subsection{Presa della misure}
\begin{wrapfigure}{r}{0.4\textwidth}
    \centering
    \caption{Il pendolo di torsione}
    \begin{tikzpicture}
        \draw(0, 0) -- (3, 0) ;
        \draw(1.5, 0) -- (1.5, -2);
        \draw(0, -2) rectangle (3, -2.5);
        \draw(1.5, -2) circle(0.5);
    \end{tikzpicture}    
\end{wrapfigure}
La presa delle misure nell'esperienza della torsione avviene nella seguente maniera:
\begin{itemize}
    \item misureremo non i raggi ma i diametri dell'anello con strumenti simili a quelli
    delle esperienze precedenti (calibro);
    \item Nella presa delle misure dobbiamo misurare la
    lunghezza del filo $L$ a cui è attaccato l'equipaggio mobile: il filo va staccato dai morsetti che 
    lo tengono saldo all'apparato sperimentale e teso leggermente per poter simulare l'effetto della forza
    peso e in modo da poterne misurare l'effettiva lunghezza;
    \item Si misurano le masse utilizzando nuovamente la bilancia di precisione;
    \item Si misura  lo spessore del filo con cui stiamo
    lavorando tramite una compasso di palmer millesimale;
    \item Si svolgono le misure di tempo di oscillazione del pendolo di torsione attraverso un cronometro,
    stando attenti alla posizione rispetto al goniometro posto sotto al pendolo.
\end{itemize}
Ruotando ora l'estremo libero del filo, esso oscillerà di un certo angolo
$\theta$ rispetto al goniometro posto sotto ed il filo reagirà con un sistema di forze equivalente
ad un momento di forze diretto lungo l'asse verticale e di modulo dato dalla seguente 
\begin{align}
    \theta \leq \frac{L}{r}\alpha_{max} = \frac{L}{r}\frac{\sigma_G}{G}
\end{align}
dove $L$ e $r$ sono la lunghezza ed il raggio del filo considerato e $G$ la costante
di scorrimento del materiale di cui è costituito
il filo e $\sigma_G$ il valore limite dello sforzo specifico tangenziale. Dato che si tratta di una stima
\begin{gather*}
    \frac{\sigma_G}{G} \approx \frac{\sigma_E}{E}
\end{gather*}
Se lasciamo il pendolo libero partendo da questa posizione iniziale il pendolo si muoverò sotto
l'effetto della sola presenza della forza di reazione elastica del filo.
Il moto dell'equipaggio sarà allora descritto dall'equazione differenziale
\begin{align}
    I_{em} \ddot{\theta} = M_e = -c_t \theta
\end{align}
Dove 
\begin{align}
    c_t = \frac{\pi}{2}G\frac{r^{4} }{L}
\end{align}
La soluzione dell'equazione oraria allora è
\begin{gather*}
    \theta(t) = A \cdot  \cos(\omega t + \beta)
\end{gather*}
Dove $A$ e $\beta$ dipendono dalle condizioni iniziali e 
$\omega = \sqrt{c_t / I_{em}}$. Se si impone che $\theta(0) = \theta$ e
la derivata è nulla allora si ha che $A = \theta$ e $\beta = 0$. 
\begin{gather*}
    T = 2\pi \sqrt{\frac{I_{em}}{c_t}} 
\end{gather*} 
La misura del periodo di oscillazione ci permette la misura indiretta della costante 
di torsione del filo $c_t$ e quindi della costante di scorrimento del materiale
di cui è costituito il filo. Possiamo ridurre l'errore sul periodo 
misurando più oscillazioni in una sola misurazione e dividendo per il numero di oscillazioni
(ovviamente, dato che ogni oscillazione tenderà a ridurre la sua ampiezza, devo
scegliere un numero di oscillazioni in modo tale che il moto non sia troppo smorzato dalle
forze di attrito):
\begin{gather*}
    t = nT \ \Longrightarrow \ \frac{\Delta t}{t} = \frac{\Delta T}{T}
\end{gather*}
Allora posso dire che la migliore stima per il periodo sia:
\begin{align}
    T = \frac{t}{n} \pm \frac{\Delta t}{n}
\end{align}
Dove l'errore su  $t$ è proprio l'errore di sensibilità del cronometro (generalmente un $n = 5$ è sufficiente per
ottenere misure precise di periodo). Se non operassimo in questa maniera, l'errore sull'oscillazione
diventerebbe l'errore umano di circa $0.3 s$.


\subsection{Propagazione degli errori}
Ci danno l'informazione su come modificare eventuali procedure di misura:
\begin{gather*}
    G = 8\pi \frac{L}{R^{4} } \frac{I_{an}}{T_2^{2} - T_1^{2}  }
\end{gather*}
\begin{align}
    \frac{\Delta G}{G} = \frac{\Delta L}{L} + 4\frac{\Delta D}{D} + \frac{\Delta I_{an}}{I_{an}} + 2\frac{T_2\Delta T_2 - T_1 \Delta T_1}{T_2^{2}  - T_1^{2} }
\end{align}
\begin{gather*}
    I_{an} = \frac{1}{2} m(R_e^{2} + R_i^{2} )
\end{gather*}
\begin{align}
    \frac{\Delta I_{an}}{I_{an}} = \frac{\Delta m}{m} + 2\frac{R_e\Delta R_e + R_i \Delta R_i}{R_e^{2} + R_i^{2} }
\end{align}


\subsection{Riflessione sulla schematizzazione e ipotesi di validità}
Oltre all'effetto dello smorzamento c'è quello della relativa approssimazione
\begin{gather*}
    \sigma  \approx G \alpha
\end{gather*}
Per essere sicuri di non uscire dalla regione di elasticità possiamo stimare l'angolo
massimo $\theta$ e quindi se consideriamo la relazione
\begin{gather*}
    \alpha L = \theta R
\end{gather*}
Se imponessi allora $\sigma < \sigma_G$ allora ottengo che l'angolo
massimo  è dato da
\begin{gather*}
    \sigma < \sigma_G \ \Rightarrow \ \theta < \frac{\sigma_G}{G} \frac{L}{R} = \theta_{max}
\end{gather*}
Una volta in laboratorio dovremmo stimare questo $\theta_{max}$ con il pendolo
che avremo a disposizione e quindi $\theta_{max} >> 90°$ è conveniente scegliere
$\theta = 90°$ dato che semplifica notevolmente la misura. Un'altra ipotesi
che abbiamo fatto è quella che prevede la somma dei momenti di inerzia e dunque dato che
l'anello non è solidale e dato che esso non sarà piazzato esattamente al centro che stiamo
imponendo un'altra approssimazione. \\
Stiamo supponendo che questa non perfetta simmetria non da errori troppo elevati e 
quindi si utilizza il teorema di Huygens-Steiner  che ci permette di calcolare
il momento di inerzia rispetto a un asse diverso da quello che conosco già,
il momento di inerzia è dunque
\begin{gather*}
    I = I_{cdm} + md^{2} 
\end{gather*}
e per corpi geometrici corrisponde al centro geometrico del corpo il $CDM$. Questo teorema
ci permette di stimare l'errore derivante dalla non coassialità dell'anello. \\
Tutte queste considerazioni valgono se e solo se si mantengono determinate ipotesi di validità:
\begin{enumerate}
    \item Dato che stiamo trascurando le forze di smorzamento, dobbiamo
    utilizzare il cronometro attivandolo quando passa dalla posizione di equilibrio e non quando
    torna alla posizione di partenza (in quanto non ci tornerà mai).
    \item Il moto ha una velocità angolare variabile e l'incertezza con cui si misura il passaggio
    varia poiché varia la velocità angolare. DAto che nei punti centrali è massima, allora, sapendo che
    $\Delta t = \frac{\Delta \theta}{\dot{\theta}}$, avremmo allora $\Delta t$ minimo.
\end{enumerate}


\subsection{Stima dell'errore di posizionamento} 
Sapendo che 
\begin{gather*}
    I_{an} = I_{an} + md^{2} = m\left(\frac{1}{2}(R_e^{2} + R_i^{2}  ) + d^{2} \right) 
\end{gather*}
\begin{gather*}
    Md^{2} << \Delta I_{an} \ \Rightarrow  \ d^{2} << \frac{\Delta I_{an}}{M} \approx d^{2} << 1.4 mm  
\end{gather*}
Solo se riusciamo a fare coincidere l'asse dell'anello con quello del filo con una
precisione maggiore di $1.4 mm$ possiamo non tenere conto di questa componente 
nell'errore dei calcoli (durante l'esperienza possiamo non tenere conto di questa componente 
a prescindere, anche se è bene verificare la precisione con cui questa componente
si può ignorare nel nostro apparato).

\section{Smorzamento}
Nell'esperienza esso risulta opzionale anche se a livello di schematizzazione
fisica risulterebbe corretto il suo studio per determinare la bontà delle approssimazioni
(a livello pratico non si studia la dimostrazione passo per passo, basta sapere che il pendolo dopo
un certo numero di oscillazioni tende a ridurre l'angolo di oscillazione e dunque tende a 
fermarsi dopo un certo tempo, l'importante è sapere le formule per i conti nell'esperienza). \\
Siamo partiti infatti dalla legge
\begin{gather*}
    \frac{dL_z}{dt} = M_z^{ext} \qquad \qquad I_{em} \ddot{\theta} = -c_t \theta 
\end{gather*}
Abbiamo quindi dedotto che si tratta di moto armonico e anche se nella realtà 
è presente anche un momento dovuto agli attriti dell'aria che frena il moto. 
Supponiamo che nell'attrito viscoso $F = \alpha v$ definisco il fattore di attrito 
con l'aria come $-b \dot{\theta}$. 
\begin{gather*}
    I_{em} \ddot{\theta} + b \dot{\theta} + c_t \theta = 0
\end{gather*}
Risolvendo otteniamo 
\begin{gather*}
    \theta(t) = A'\exp\left(-\frac{b}{2I_{em}}t\right)\cos(\omega' t - \gamma')
\end{gather*}
Verifichiamo che non sia realmente soluzione:
\begin{gather*}
    \dot{\theta}(t) = -\frac{bA'}{2I_{em}}\exp\left(-\frac{b}{2I_{em}}t\right)\cos(\omega' t - \gamma') - \omega'A'\exp\left(-\frac{b}{2I_{em}}t\right)\sin(\omega't - \gamma') \\
    \ddot{\theta}(t) = \left(\frac{b}{2I_{em}}\right)^{2} \theta(t) + \frac{b}{2I_{em}}\omega'A'\exp\left(-\frac{b}{2I_{em}}t\right) \sin(\omega' t - \gamma') - \omega^{2'}\theta(t) 
\end{gather*}
Sostituendo nell'equazione si ottiene
\begin{gather*}
    \frac{b^{2} }{4I_{em}}\theta(t) - I_{em} \omega^{'2} \theta(t) - \frac{b^{2} }{2I_{em}}\theta(t) + c_t\theta(t) = 0 
\end{gather*}
Possiamo procedere e semplificare con $\theta(t)$ e quindi
\begin{gather*}
    -\frac{b^{2} }{4I_{em}} - I_{em}\omega^{'2} + c_t = 0 \\
    \omega^{'2} = \frac{c_t}{I_{em}} - \left(\frac{b}{2I_{em}}\right)^{2}   
\end{gather*}
La seconda è minore della pulsazione senza attriti $\omega^{2}$ e quindi si può dire che
\begin{align}
    \omega' = \omega \sqrt{1 - \left(\frac{b}{2I_{em}\omega}\right)^{2} }  
\end{align} 
E allora il periodo sarà
\begin{align}
    T' = T\frac{1}{\sqrt{1 - \left(\frac{b}{2I_{em}\omega}\right)^{2} } } >T
\end{align}

Posso allora prendere come condizioni iniziali 
\begin{gather*}
     \theta(0) = \theta_0 \\
     \dot{\theta}(0)  = 0
\end{gather*}
Ottenendo i seguenti valori per le costanti:
\begin{gather*}
    A' = \frac{\theta_0}{\cos\gamma} \\
    \tan\gamma' = \frac{b}{2I_{em}\omega'}
\end{gather*}
Posso anche riscrivere l'equazione con le sostituzioni:
\begin{gather*}
    \theta(t) = \theta_0 \exp\left(-\frac{b}{2I_{em}t}\right)\left(\cos(\omega't) + \frac{b}{2I_{em}\omega'}\sin(\omega' t)\right)
\end{gather*}
Bisogna porre attenzione nell'approssimare tali grafici a dei moti armonici
e verifichiamo l'errore commesso sull'approssimazione dove $T \sim T'$:
\begin{gather*}
    \theta(nT') = \theta_0\exp\left(-\frac{b}{2I_{em}nT'}\right)\left(\cos\left(\frac{2\pi}{T'}nT'\right) + \frac{b}{2I_{em}}\sin\left(\frac{2\pi}{T'}nT'\right)\right) = \theta_0\exp\left(-\frac{b}{2I_{em}}nT'\right)
\end{gather*}
Possiamo considerare il grafico di tale funzione, linearizzandola con la tecnica
del logaritmo e quindi 
\begin{gather*}
    \ln(\theta(nT')) = \ln\theta_0 - \frac{b}{2I_{em}}T'n
\end{gather*}
In questo grafico ci aspettiamo una dipendenza lineare e quindi possiamo con il metodo dei
minimi quadrati pesati oppure col metodo grafico. Come valutazione dell'errore
possiamo utilizzare l'errore di sensibilità del goniometro con cui si misura
l'angolo del pendolo di torsione. Con il grafico è possibile risalire al periodo
$T'$ con il quale posso ricavare il fattore correttivo e ha senso applicarlo solo se 
produce variazioni comparabili con l'incertezza di misura. Per verificare
ciò è possibile utilizzare l'equazione e approssimare con Taylor
\begin{gather*}
    T' = \frac{T}{\sqrt{1 - \frac{b}{2I_{em}\omega'}} } \approx T\left(1 + \frac{1}{2}\left(\frac{b}{2I_{em}\omega'}\right)^{2} \right) = T + \delta.
\end{gather*} 


\begin{wrapfigure}{r}{0.4\textwidth}
    \centering
    \caption{Il grafico per lo smorzamento}
    \begin{tikzpicture}
        \draw[->](0, 0) -- (4, 0) node[at end, below] {$n$}; 
        \draw[->](0, 0) -- (0, 3) node[at end, left] {$\ln(\theta(nT))$};
        \filldraw(0, 1.75) circle (1pt) node[anchor = east] {$\ln\theta_0$};
        \draw[|-|] (0.25, 1.5) -- (0.25, 1.75);
        \draw[|-|] (0.75, 1.35) -- (0.75, 1.60);
        \draw[|-|] (1.25, 1.20) -- (1.25, 1.45);
        \draw[|-|] (1.75, 1.05) -- (1.75, 1.30);
        \draw[|-|] (2.15, 0.90) -- (2.15, 1.15);
    \end{tikzpicture}    
\end{wrapfigure}

Posso allora ragionare sul rapporto tra la componente $\delta$ ed il periodo
$T'$ smorzato e confrontarlo con l'errore relativo ed ottenere le seguenti
considerazioni:
\begin{gather*}
    \frac{\delta}{T'} << \frac{\Delta T'}{T'} \ \Rightarrow \ T' \approx T \\
    \frac{\delta}{T'} \approx \frac{\Delta T'}{T'} \ \Rightarrow \ T'  = T + \delta
\end{gather*}
Per fare le misurazioni sullo smorzamento del periodo di oscillazione 
utilizzando dei traguardi angolari (ossia valutando dopo quante oscillazioni
il pendolo raggiunge una certa ampiezza angolare). Dato che però abbiamo calcolato
il fattore correttivo per $I_{em}$ e non per il complesso $I_{em} + I_{an}$. Dato che
richiederebbe molti conti e tempo stimare il fattore correttivo per lui, possiamo utilizzare
una approssimazione più blanda calcolando prima $T'$ e poi $b$
ed semplicemente sostituendo $I_{em}$ con $I_{em} + I_{an}$. 

\chapter{Esperienza pendolo}
\section{Obbiettivi esperienza}
L'esperienza del pendolo consiste nel misurare indirettamente
il modulo dell'accelerazione locale attraverso:
\begin{enumerate}
    \item Pendolo che si muove sul piano verticale.
\end{enumerate}
Il valore di $g$ infatti dipende da diverse condizioni:
\begin{enumerate}
    \item Latitudine $\phi$: all'aumentare della latitudine si ha un aumento di $g$
    \item La quota $h$ rispetto al livello medio del mare;
    \item La disomogeneità locale della distribuzione di massa
\end{enumerate}
Il pendolo è allora il sistema fisico il cui studio è stato cruciale
per lo sviluppo della scienza.

\section{Il pendolo: quando è approssimabile ad un pendolo semplice}
\begin{wrapfigure}{r}{0.4\textwidth}
    \centering
    \caption{Il pendolo}
    \begin{tikzpicture}
        \draw(0, 0) circle (0.5);
        \draw[cyan](0, 4) -- (0, -0.5);
        \filldraw(0, 0) circle (1pt) node[anchor = east] {$C$};
        \filldraw[red](0, 4) circle (1pt) node[anchor = east] {$O$};
        \draw(-1, 3.5) -- (2, 3.5);
        \draw(-1, 3.5) -- (-1.5, 5);
        \draw(-1.5, 5) -- (1.5, 5);
        \draw(1.5, 5) -- (2, 3.5);
        \draw[|-|](0.5, 4) -- (0.5, 0.5) node[midway, left] {$l_f$};
        \draw[|-|](0.75, 4) -- (0.75, -0.5) node[midway, right] {$l$};
    \end{tikzpicture}    
\end{wrapfigure}
\subsection{Approssimazione ad un moto semplice}
Il pendolo in laboratorio consiste in una sferetta di piombo (che possiamo
approssimare a liscia e uniforme cosicché si possa evitare di considerare
la resistenza dell'aria) fissata con un filo (che presupponiamo inestensibile)
al punto $O$ detto di sospensione. Un punto materiale di massa $m$ vincolato con un vincolo ideale a muoversi
sotto l'azione della forza peso, su di un arco di circonferenza di raggio 
prefissato $l$ giacente in un piano verticale. Trascurando
allora attriti ed effetti vari, il pendolo semplice è in condizioni
di moto ideale. \\
La schematizzazione del pendolo semplice vale se e solo se valgono le
seguenti condizioni:
\begin{enumerate}
    \item Il corpo appeso al filo è approssimato ad un punto
    materiale di massa $m$ ossia \textbf{massa puntiforme} se e solo se $R << l$;
    \item La fune è ideale, ossia la $m_f = 0$ ($m_f << m$) ed è
    inestensibile, ossia $l$ costante in modo tale che le tensioni
    da un capo e dall'altro della fune siano uguali e che la fune non
    sia in nessun modo elastica. 
    \item Trascurando eventuali forze di attrito ed i loro effetti.
    \item Inoltre devo dire che l'incertezza sulla lunghezza deve essere 
    molto maggiore della variazione di lunghezza del filo durante il moto $\delta l << \Delta l$.
\end{enumerate}
Nel momento in cui valgono tutte queste ipotesi, allora possiamo parlare di moto di
pendolo semplice ideale. Questo è il caso che andremo a studiare durante l'esperienza di laboratorio.

\subsection{Il moto del pendolo semplice}
\begin{wrapfigure}{r}{0.4\textwidth}
    \centering
    \caption{Il moto del pendolo semplice ed ideale}
    \begin{tikzpicture}
        \filldraw(0, 0) circle (1pt) node[anchor = east] {$O$};
        \draw[->](0, 0) -- (1.5, 0) node[at end, above] {$x$};
        \draw[->](0, 0) -- (0, 0.5) node[at end, left] {$y$};
        \filldraw(0, -2.9) circle (1pt) node[anchor = north] {$C$};
        \draw[dashed](0, 0) -- (0, -2.85);
        \draw[dashed](-1.5, -2.5) arc (240:300:3);
        \draw(0, 0) -- (1.5, -2.5) node[midway, above] {$l$};
        \filldraw(1.5, -2.5) circle (1pt) node[anchor = east] {$P$}; 
        \draw[red, ->](1.5, -2.5) -- (1, -1.68) node[at end, right] {$\vec{T}$};
        \draw[->] (1.5, -2.5) -- (1.5, -3.5) node[at end, left] {$m\vec{g}$};
        \draw[->] (1.5, -2.5) -- (2, -2.2) node[at end, below] {$\vec{u}_{\phi}$};
        \draw[->] (0, -1) arc (270:300:1) node[midway, below] {$\phi$};
    \end{tikzpicture}    
\end{wrapfigure}
Il moto del pendolo semplice ed ideale ora, deve godere
di un ulteriore proprietà: la velocità iniziale deve necessariamente
essere nulla (in particolar modo la componente
perpendicolare al piano $xy$). 
Dato che il moto avviene sul piano verticale, basterà descriverlo con solo 
una coordinata, ossia $\phi$ per cui, per un instante generico
$\phi = \phi(t)$. 
Applicando il secondo principio della dinamica, istante per istante
possiamo scomporre le forze lungo i due versori e quindi ottenere:
\begin{gather*}
    \left\{\begin{array}{l}
        \vec{a} = a_{\phi}\hat{u}_{\phi} + a_n \hat{u}_n  \\ 
        \vec{T}  = T\hat{u}_n \\
        m\vec{g} = -mg\cos\phi \hat{u}_n - mg\sin\phi\hat{u}_{\phi}    
    \end{array}\right.
\end{gather*}
Lungo le due direzioni normale e tangenziale allora:
\begin{gather*}
    T - mg\cos\phi  = ma_n \\
    -mg\sin\phi = ma_{\phi}
\end{gather*}
Dato che l'accelerazione si esprime come:
\begin{gather*}
    a_{\phi} = \frac{d\vec{v}_{\phi} }{dt} = \frac{d}{dt}\frac{ds}{dt} = \frac{d}{dt}l\frac{d\phi}{dt}
\end{gather*}
Allora, dato $\Delta s = l \Delta \phi$:
\begin{align}
    a_{\phi} = l\ddot{\phi}
\end{align}
Dove $\ddot{\phi}$ è l'accelerazione angolare e quindi
l'equazione del moto lungo la traiettoria tangenziale diventa:
\begin{align}
    \ddot{\phi} + \frac{g}{l}\sin\phi = 0
\end{align}
Ossia l'equazione del pendolo semplice con moto ideale che si
risolve in due modi. Il primo metodo è quello dell'analisi
del caso limite nel quale l'equazione semplifica diventando risolvibile
analiticamente con il limite delle piccole oscillazioni. \\
Il secondo metodo è invece quello dell'equazione nella sua forma
generale.

\section{Risolvere l'equazione di moto}
\subsection{Il limite delle piccole oscillazioni}
L'equazione di moto generale che abbiamo appena ricavato diventa
analiticamente risolvibile in una condizione limite puramente matematica. Il
caso limite delle piccole oscillazioni è la situazione nella quale
si ha un'oscillazione di $\phi_0$ sufficientemente piccola
da poter rendere valido lo sviluppo di $\sin\phi$, ossia
solo nel caso in cui $\phi << 1$ e allora l'equazione differenziale
diventa
\begin{gather*}
    \ddot{\phi} + \frac{g}{l}\phi = 0 \ \Longrightarrow \  \ddot{\phi}\alpha + \phi = 0, \alpha = \frac{l}{g}
\end{gather*}
In questo limite allora l'equazione di moto ha la generica soluzione:
\begin{align}
    \phi(t) = \alpha\cos(\omega_0t+\beta), \qquad\omega_0 = \frac{2\pi}{T_0}
\end{align}
Derivando rispetto al tempo allora si ottiene
\begin{gather*}
    \left\{\begin{array}{l}
        \dot{\phi} = -\omega_0 \alpha \sin(\omega_0 t + \beta) \\
        \ddot{\phi} = -\omega_0^{2}\alpha \cos(\omega_0 t+ \beta) = -\omega _0^{2}\phi (t) 
    \end{array}\right.
\end{gather*}
Sostituendo nella differenziale allora si ottengono 
\begin{gather*}
    \omega_0 = \sqrt{\frac{g}{l}} = \frac{2\pi}{T_0}
\end{gather*}
Posso escludere il secondo risultato della differenziale in quanto
questo ci descrive un caso molto banale, ossia il caso in cui il
pendolo sia fermo e quindi non è interessante ai fini dell'esperienza. 
Posso ottenere allora il periodo come:
\begin{align}
    T_0 = 2\pi \sqrt{\frac{l}{g}}
\end{align}
Date le condizioni iniziali per cui $v(t = 0s) = 0$, allora
$\phi(t = 0s) = \phi_0$, posso ricavare le costanti $\alpha$ e $\beta$:
\begin{gather*}
    \phi(0) = \alpha\cos\beta = \phi_0 \\
    \dot{\phi}(0) = -\omega_0 \alpha \sin\beta = 0
\end{gather*}
E quindi la soluzione diventa
\begin{gather*}
    \phi(t) = \phi_0 \cos(\omega_0 t)
\end{gather*}
In questo caso limite allora, posso ricavare
il valore di $g$ tramite la relazione del periodo scritta sopra
\begin{align}
    g = 4\pi^{2} \frac{l}{T_0^{2} }
\end{align}

\subsection{Il caso generale: le oscillazioni di ampiezza generica}
Seguiamo adesso il secondo percorso per l'analisi del moto
del pendolo ma non nel limite delle piccole oscillazioni; data l'equazione di moto
\begin{align}
    \ddot{\phi} + \frac{g}{l}\sin\phi = 0
\end{align}
Moltiplicando per $2\dot{\phi}$ si ottiene la relazione
\begin{gather*}
    2\dot{\phi}\ddot{\phi} = (-\sin\phi\cdot\dot{\phi})\cdot  2\frac{g}{l}
\end{gather*}
E quindi integrando da entrambe le parti si ottiene 
\begin{gather*}
    \dot{\phi}^{2} - \frac{2g}{l}\cos\phi = CONST 
\end{gather*}
Imponendo le condizioni iniziali allora si ottiene che la costante
è 
\begin{gather*}
    -2\frac{g}{l}\cos\phi_0 = CONST
\end{gather*}
Possiamo ottenere la costante di integrazione attraverso $\dot{\phi} = 0$ e
$\phi = \phi_0$ al tempo $t = 0$ e quindi la costante è $-2\frac{g}{l}\cos\phi_0$: 
\begin{align}
    \dot{\phi}^{2} = \frac{2g}{l}(\cos\phi - \cos\phi_0) 
\end{align}
Se $\dot{\phi}^{2} \geq 0$:
\begin{gather*}
    \cos\phi \geq \cos\phi_0 \ \Longrightarrow \  -\phi_0 \leq \phi \leq \phi_0
\end{gather*}
Utilizzando anche le informazioni sull'accelerazione angolare
dalle eq. di moto, allora si ottiene $\ddot{\phi} = -\frac{g}{l}\sin\phi$
possiamo mostrare che il moto è oscillatorio e periodico anche nel caso generale
di ampiezza generica. \\
L'oscillazione allora si divide in 4 fasi:
\begin{enumerate}
    \item $\phi \in [0, \phi_0], \dot{\phi} < 0$;
    \item $\phi \in [-\phi_0, 0], \dot{\phi} < 0$;
    \item $\phi \in [-\phi_0, 0], \dot{\phi} > 0$;
    \item $\phi \in [0, \phi_0], \dot{\phi} > 0$;
\end{enumerate}
\begin{enumerate}
    \item $\ddot{\phi} < 0$: la derivata diminuisce in segno e arriva al valore negativo in $\phi = 0$;
    \item $\ddot{\phi} > 0$: la derivata cresce fino a $\dot{\phi} = 0$ e quindi la derivata seconda diventa positiva;
    \item $\ddot{\phi} > 0$: la derivata cresce fino a $\dot{\phi}_{MAX}$ in $\phi = 0$ ed il pendolo torna indietro;
    \item $\ddot{\phi} < 0$: la velocità angolare scende per tornare ad annullarsi in $\phi = \phi_0$.  
\end{enumerate}
Ogni fase ha quindi un periodo di $T/4$.
Dato che non si è in grado di trovare una soluzione analitica per l'equazione 
differenziale del moto per oscillazioni generiche, bisogna per forza
integrare entrambi i membri. Nell'ultimo quarto dell'oscillazione in particolare la radice del secondo
membro diventa positiva e dunque posso dire che
\begin{gather*}
    \frac{d\phi}{dt} = \sqrt{\frac{2g}{l}(\cos\phi - \cos\phi_0) }
\end{gather*}
Si ottiene inoltre, per separazione di variabili:
\begin{gather*}
    d\phi = dt\sqrt{\frac{2g}{l}(\cos\phi - \cos\phi_0) } \\
    \frac{d\phi}{\sqrt{2(\cos\phi - \cos\phi_0)} } = dt\sqrt{\frac{g}{l}}
\end{gather*}
Integrando allora fra $\frac{3T}{4}$ e $T$ si ottiene la seguente uguaglianza:
\begin{align}
    \left(\frac{g}{l}\right)^{\frac{1}{2}} \int_{3\frac{T}{4}}^{T}\ dt = \int_{0}^{\phi_0}\frac{d\phi}{(2(\cos\phi - \cos\phi_0))^{\frac{1}{2}} } 
\end{align}
Il secondo integrale è complicato in quanto bisogna eseguire la sostituzione $\sin\frac{\phi}{2} = \sin\frac{\phi_0}{2}\sin\theta$ 
e quindi ottenere $K(\sin^{2} \frac{\phi_0}{2})$. L'uguaglianza di partenza è allora
\begin{align}
    T = 4\sqrt{\frac{l}{g}}K(\sin^{2} \frac{\phi_0}{2})
\end{align}
Possiamo allora rappresentare $\sin^{2}\frac{\phi_0}{2}$ con una serie di potenze convergenti e scrivendo
esplicitamente i primi termini della serie come
\begin{gather*}
    F(\phi_0) = 1 + \frac{1}{4}\sin^{2}\frac{\phi_0}{2} + \frac{9}{64}\sin^{4}\frac{\phi_0}{2} + o(\sin^{4}\frac{\phi_0}{2} )  
\end{gather*}
Allora il periodo per le oscillazioni di ampiezza generica
\begin{align}
    T = T_0 F(\phi_0)
\end{align}
Quindi per il pendolo semplice in moto ideale ed oscillazioni di ampiezza generica,
il periodo dipende dalla ampiezza angolare $\phi_0$ dell'oscillazione: questa dipendenza
moltiplica il periodo nel limite delle piccole oscillazioni $T_0$ e quindi 
il periodo risulta essere
\begin{gather*}
    T(l, g, \phi_0) = T_0(l, g) \cdot  F(\phi_0)
\end{gather*}
La funzione $F(\phi_0)$ può essere valutata con la precisione che vogliamo e
che è opportuna a seconda del caso: data l'ampiezza angolare $\phi_0$ dell'oscillazione
possiamo rappresentazione il periodo $T$ e la dipendenza da $\phi_0$. Nella configurazione
del laboratorio, gli angoli per i quali possiamo fare misure precise per il periodo
è dato dalla seguente tabella
\begin{gather*}
    \begin{tabular}{c | c | c | c}
    $\phi_0$ & $\phi_0$ & $\frac{1}{4}\sin^{2}\frac{\phi_0}{2}$ & $\frac{9}{64}\sin^{4}\frac{\phi_0}{2} $ \\
    $\approx 10°$ &  $\approx 0.175$ &  $\approx 1.9 \cdot  10^{-3} $ &  $\approx 8 \cdot  10^{-6} $ \\
    $\approx 20°$ &  $\approx 0.349$ &  $\approx 7.5 \cdot  10^{-3} $ &  $\approx 1.3 \cdot  10^{-4} $ \\
    $\approx 25°$ &  $\approx 0.436$ &  $\approx 1.17 \cdot  10^{-2} $ &  $\approx 3.1 \cdot  10^{-4} $ 
\end{tabular}
\end{gather*}
La rappresentazione più semplice ma pur sempre la più significativa, dato che il primo termine
è quello più grande, è quello del periodo come
\begin{align}
    T = T_0(1 + \frac{1}{4}\sin^{2}\frac{\phi_0}{2})
\end{align}
Nel limite delle piccole oscillazioni, che in laboratorio corrisponde 
agli angoli $10 \sim 25$° poiché è limitato dalle dimensioni 
della zona di lavoro del laboratorio.

\section{Errore di schematizzazione relativo}
Possiamo allora trovare l'errore di schematizzazione relativo come attraverso
il seguente ragionamento e confrontarlo poi con l'errore sperimentale per determinare
quale formula del periodo utilizzare per i conti.
\begin{gather*}
    \left(\frac{\delta T}{T}\right)_{schem} = \frac{\left| T(l, g, \phi_0) - T_0(l, g)\right| }{T(l, g, \phi_0)} \\
    \frac{\delta T}{T} = \frac{T_0(1 + \frac{1}{4}\sin^{2}\frac{\phi_0}{2} ) - T_0}{T_0(1 + \frac{1}{4}\sin^{2} \frac{\phi_0}{2})} 
\end{gather*}
Sostituendo e ricordando che $(1 + x)^{-1} \approx 1 - x$ (e quindi semplificando il termine che si ottiene e considerando solo
quello più grande) si può ottenere la seguente approssimazione della schematizzazione dell'errore
sulla schematizzazione
\begin{align}
    \left(\frac{\delta T}{T}\right)_{schem} \approx \frac{1}{4}\sin^{2}\frac{\phi_0}{2} 
\end{align}
In ambito sperimentale dobbiamo considerare un errore molto maggiore dell'errore
ottenuto attraverso metodi di approssimazione e modellizazione chiamato $\frac{\Delta T}{T}$. Si distinguono allora due casi \\
1.) Se $\frac{\delta T }{T} << \frac{\Delta T}{T}$ allora si ottiene che
\begin{gather*}
    \frac{T - T_0}{T} << \frac{\Delta T}{T} \Rightarrow \frac{1}{4}\sin^{2}\frac{\phi_0}{2} << \frac{\Delta T}{T} 
\end{gather*} 
la differenza dovuta alla differente schematizzazione che posso scegliere per integrare è trascurabile
perché non rilevabile rispetto alle misure che compiamo. Se la differenza allora è non significativa
posso utilizzare il limite delle piccole oscillazioni. 
Cerchiamo così una definizione del limite delle piccole oscillazioni da un punto di vista puramente fisico-sperimentale: la situazione
in cui la precisione della misura del periodo del moto non è sufficiente ad evidenziare la dipendenza
di $T$ da $\phi_0$. \\
2.) $\frac{\delta T}{T} \geq \frac{\Delta T}{T}$ allora l'errore di schematizzazione che si commetterebbe considerando il periodo
misurato come interpretabile nel limite delle piccole oscillazioni sarebbe
superiore o dell'ordine dell'incertezza sperimentale. Nella configurazione
sperimentale in laboratorio generalmente si ha $\frac{\Delta T}{T} \approx 10^{-3} $.
Riprendendo la tabella sopra, per $\phi_0 \approx 25$° l'errore è già maggiore
dell'errore sperimentale, e allora il limite delle piccole oscillazioni sarebbe inadeguato in
quanto condurrebbe a risultati lontani dagli errori sperimentali. Dobbiamo allora utilizzare 
la seguente schematizzazione del periodo come
\begin{gather*}
    T = T_0(1 + \frac{1}{4}\sin^{2}\frac{\phi_0}{2} )
\end{gather*}
Possiamo quindi, con la schematizzazione attuale, ignorare tutti i termini dopo il primo 
per angoli inferiori a $25$° e quindi condurre la nostra esperienza con angoli
piccoli in modo da semplificare i conti seppur ottenendo una precisione soddisfacente (pur
rimanendo fuori dal limite delle piccole oscillazioni). 

\section{Apparato sperimentale in laboratorio}
\subsection{Strumenti}
Durante l'esperienza del pendolo si utilizzano i seguenti strumenti:
\begin{enumerate}
    \item Righello millimetrico a doppia scala con due errori di sens di $0.5mm$ e $1mm$;
    \item un calibro ventesimale con errore di sens di $0.05mm$;
    \item un cronometro manuale con errore di sens di $0.01s$;
    \item una scala centimetrica sul bancone con un errore di $0.5cm$;
\end{enumerate}

\subsection{Le misure da trovare}
Avendo già la distanza tra il punto di sospensione ed un punto $F$ dove c'è un nastro nero 
che è di circa $H_0 \pm 0.02 m$, devo trovare:
\begin{enumerate}
    \item la lunghezza $l$, ossia la lunghezza del filo tra il centro della pallina ed il punto di sospensione;
    \item l'angolo $\phi_0$ (dato in relazione alla scala centimetrica posta sul bancone);
    \item il periodo in funzione dell'angolo;
    \item la distanza tra il nastro ed il punto di contatto con la sfera;
    \item Il raggio della sfera (misura indiretta poiché si determina il diametro). 
\end{enumerate}

\subsection{Come eseguire le misure}
\begin{wrapfigure}{r}{0.4\textwidth}
    \centering
    \caption{La situazione del pendolo in lab.}
    \begin{tikzpicture}
        \filldraw(0, 0) circle (1pt) node[anchor = east] {$O$};
        \draw[dashed](0, 0) -- (0, -4);
        \draw(0, 0) -- (2, -3.4);
        \draw(0, -1) arc (270:300:1) node[midway, below] {$\phi_0$};
        \filldraw(0, -4) circle (1pt) node[anchor = east] {$C$};
        \filldraw(2, -3.4) circle (1pt);
        \filldraw(0, -3.4) circle (1pt) node[anchor = east] {$H$};
        \draw(0, -4) circle(0.25);
        \draw[dashed] (0, -3.4) -- (1.75, -3.4) node[midway, above] {$x_0$} node[at end, above] {$P$};
        \filldraw (1.75, -3.4) circle (1pt);
        \draw[dashed] (0, -4) .. controls (1, -3.8) .. (2, -3.4);
        \draw(2, -3.4) circle (0.25);
    \end{tikzpicture}    
\end{wrapfigure}
Misurare l'ampiezza angolare dell'oscillazione $\phi_0$: spostando il pendolo dalla sua posizione
di equilibrio a quella scelta per il rilascio, corrispondente a $\phi = \phi_0$, il centro di
massa della sferetta si troverà ad una distanza dalla verticale pari a $\overline{HP} = x_0$,
allora posso dire che $\sin\phi_0 = \frac{\overline{HP} }{\overline{OC} } = \frac{x_0}{l}$ e allora
si ha
\begin{align}
    \phi_0 = \arcsin\left(\frac{x_0}{l}\right)
\end{align}
Per trovare $x_0$ si utilizza un cilindretto forato, montato su di una base, che
ci permette di poter leggere la distanza della superficie della sferetta a contatto
con il cilindro; chiamata allora $x_0'$ questa lettura, posso determinare la distanza
del centro della sferetta rispetto al filo in posizione di riposo: $x_0 = x_0' - R$. E allora, dato che è una
misura diretta, avrà un certo errore $\Delta x_0 = \Delta x_0' + \Delta R$. \\
Trovare l'incertezza sulla'ampiezza angolare con il metodo delle derivate logaritmiche:
\begin{gather*}
    \sin\phi_0 = \frac{x_0}{l} \ \Longrightarrow \ d\ln(\sin\phi_0) = d\ln(\frac{x_0}{l})
\end{gather*}
Che diventa allora 
\begin{align}
    \frac{\Delta \phi_0}{\phi_0} = \frac{x_0}{\sqrt{l^{2} - x_0^{2} } } \left(\frac{\Delta x_0}{x_0} + \frac{\Delta l}{l}\right)
\end{align}
Gli altri errori da trovare
\begin{enumerate}
    \item $\Delta  x_0' = 5 mm$, ossia l'errore della scala posta sul bancone;
    \item $\Delta R$ potrebbe essere più problematico dovuto alla non
    perfetta sfericità della pallina. dato che questo errore è comunque trascurabile rispetto a $x_0'$, posso ignorarlo nei conti futuri;
    \item $\Delta x_0 = \Delta x_0' + \Delta R$ e quindi posso evitare di usare $\Delta R$.
    \item $\Delta l$ è misurata prima. 
\end{enumerate}


\section{Misure di tempo, loro ottimizzazioni ed accorgimenti per migliorarne la precisione}
Durante l'esperienza si misura il periodo per diversi
valori dell'ampiezza angolare, i quali sono scelti in modo tale da essere sufficientemente diversi tra di loro,
attraverso l'utilizzo di un cronometro ad azionamento manuale con errore
di sensibilità di soli $0.01 s$ anche se ci si aspettano degli errori superiori
dovuti ai tempi di reazione umani.
Per migliorare la precisione delle misure si può utilizzare i seguenti accorgimenti:
\begin{enumerate}
    \item Fare attenzione a rilasciare la sferetta in modo corretto (senza imprimere velocità iniziali e facendo
    in modo che non vi sia alcuna oscillazione perpendicolare al piano del moto);
    \item mantenere il filo teso senza tirarlo;
    \item evitare di ruotare la sferetta e di imprimerle una velocità iniziale;
    \item una sola persona deve lasciare la sferetta (RIP Giulio);
    \item scelta del riferimento per l'azionamento del cronometro ricordando l'errore umano sia sull'attivazione che sull'incertezza della posizione della massa.
\end{enumerate}
L'intervallo di tempo da misurare direttamente è la durata di $n$ oscillazioni:
\begin{gather*}
    \tau = \tau(n) = nT
\end{gather*}
E ricavare allora $T$ indirettamente dalla misura di $\tau$ 
diretta e si ottiene allora il periodo medio come
\begin{align}
    T = \frac{\tau(n)}{n} \pm \frac{\Delta \tau(n)}{n}
\end{align}
Per misurare i periodi delle oscillazioni, bisogna disporsi in modo tale che
si riduca l'effetto di parallasse e di far partire il cronometro nel momento in cui si ha la terza
fase dell'oscillazione: ossia quando il pendolo passa nuovamente da $\phi_0$.
A questo punto con questo metodo di misurare più oscillazioni in una singola volta
posso limitare l'errore sulla misura del periodo in quanto posso dividerlo
per il numero di oscillazioni compiute durante una prova.


\subsection{Limitazioni sul numero di oscillazioni $n$ e sull'angolo $\phi_0$}
Perché non si possono fare tante $n$ misurazioni? Questo perché 
risultano progressivamente smorzate dalla resistenza dell'aria e
quindi bisogna scegliere un valore di $n$ adatto per poter
avere misure precise oltre che corrette. Inoltre, devo assicurarmi di
scegliere un angolo $\phi_0$ sufficientemente grande affinché il pendolo abbia
abbastanza energia per non poter essere facilmente smorzato dalle varie forze di attrito
per alcune oscillazioni. Tuttavia, angoli troppo grandi introdurrebbero molti errori
(oltre al fatto che si uscirebbe dal limite delle piccole oscillazioni). Devo allora scegliere un intervallo di distanze dal punto di
riposo del pendolo di circa $30-95cm$, che corrispondono indicativamente all'intervallo di angoli $10 \sim 25$°. 

\subsection{Misure di tempo: procedure, analisi ed anisocronismo}
\begin{wrapfigure}{r}{0.4\textwidth}
    \centering
    \caption{Grafico anisocronismo}
    \begin{tikzpicture}
        \draw[->](0, 0) -- (4, 0) node[at end, below] {$\frac{1}{4}\sin^{2}\frac{\phi_0}{2}$};
        \draw[->](0, 0) -- (0, 4) node[at end, left] {$T(\phi_0)$};
        \filldraw(0, 0) circle (1pt) node[anchor = east] {$O$}; 
        \draw[|-|](0.5, 0.5) -- (0.5, 1);
        \filldraw(0.5, 0.75) circle (1pt);
        \draw[|-|](1.2, 0.7) -- (1.2, 1.3);
        \filldraw(1.2, 1) circle (1pt);
        \draw[|-|](2.0, 1.0) -- (2.0, 2.0);
        \filldraw(2.0, 1.5) circle (1pt);
        \draw[|-|](2.5, 1.2) -- (2.5, 1.9);
        \filldraw(2.5, 1.55) circle (1pt);
        \draw[|-|](3.5, 1.5) -- (3.5, 2.4);
        \filldraw(3.5, 1.95) circle (1pt);
    \end{tikzpicture}    
\end{wrapfigure}
Per ogni $\phi_0$ da $\tau \pm \Delta \tau_i$ posso ricavare il periodo
come misura indiretta e allora, dato che è una misura indiretta, la sua
incertezza non è limitata all'errore sistematico dello strumento ma può essere anche più piccolo:
posso allora ignorare l'errore di sensibilità del cronometro e l'errore di reazione
umana se ho utilizzato tutti gli accorgimenti e ho eseguito delle buone misure per il periodo
complessivo di $n$ oscillazioni. \\
Facendo il grafico che sulle ascisse ha $\frac{1}{4}\sin^{2}\frac{\phi_0}{2}$ e sulle
ordinate il valore di $T(\phi_{0_i})$ si può verificare che più l'angolo è grande e più il periodo 
è grande e che questo incremento abbia un andamento lineare in modo tale da avere anisocronismo.
Questo controllo tuttavia non può essere effettuato tra angoli troppi vicini in quanto non
si osserverebbe alcuna distinzione significativa. Nel caso in cui 
non ci fosse questa crescita lineare non si avrebbe anisocronismo. 

\section{Misura di $g$}
\begin{wrapfigure}{r}{0.4\textwidth}
    \centering
    \caption{Consistenza misure di $T_0$}
    \begin{tikzpicture}
        \draw[->](0, 0) -- (4, 0) node[at end, below] {$i$};
        \draw[->](0, 0) -- (0, 4) node[at end, left] {$T_0$};
        \draw[|-|](0.5, 0.5) -- (0.5, 1);
        \draw[|-|](1.2, 0.7) -- (1.2, 1.2);
        \draw[|-|](2.0, 0.6) -- (2.0, 1.0);
        \draw[|-|](2.5, 0.7) -- (2.5, 1.2);
        \draw[|-|](3.5, 0.8) -- (3.5, 1.4);
    \end{tikzpicture}    
\end{wrapfigure}
Il valore $T_0$, ossia il periodo nel limite delle piccole oscillazioni è fatto da
una caratteristica del pendolo di lunghezza misurata nello specifico luogo
in cui si opera. Data allora la relazione che lega $T(\phi_0)$ a $ T_0$, da ciascuno dei valori 
$T(\phi_{0i}) = T_i \pm \Delta T_i$ possiamo allora ricavare una stima di $T_0$.
Nel limite delle piccole oscillazioni posso dire che
\begin{gather*}
    g = \frac{4\pi^{2}l}{T_0^{2} }
\end{gather*}
Altrimenti nel caso generale posso dire
\begin{align}
    T_{0i} = \frac{T_i}{1 + \frac{1}{4}\sin^{2}\frac{\phi_{0i}}{2}}
\end{align}
Così come le incertezze $\Delta T_i$, anche i vari $\Delta T_{0i}$ dovrebbero avere
incertezze di ordini di grandezze comparabili. Per fare la verifica di consistenza delle varie stime
$T_0$ ottenute nella nostra misura riportiamo sulle ordinate i valori di $T_0$ con le rispettive
barre di errore in funzione di una qualunque quantità che indichi da quale $T_{0_i} \pm \Delta T_{0_i}$ ogni
valore di $T_0$ sia stato ricavato. Se esiste una regione di sovrapposizione delle barre di errore
di tutti i valori $T_{0_i} \pm \Delta T_{0_i}$ allora questi sono tutti consistenti, 
allora la misura di $g$ diventa
\begin{align}
    g = \frac{4\pi^{2}l}{T_0^{2} }
\end{align}
La cui incertezza, col metodo delle logaritmiche diventa (posto che le misure 
siano indipendenti tra loro (e non lo sono)):
\begin{align}
    \frac{\Delta g}{g} = \frac{\Delta l}{l} + 2\frac{\Delta T_0}{T_0}
\end{align}
Anche se questa è possibile utilizzarla come buona approssimazione dell'errore relativo su $g$, 
altrimenti dovremmo derivare anche rispetto alla lunghezza del pendolo in quanto $T(\phi_0)$ dipende
anche dalla lunghezza. Per il calcolo di $g$ utilizzerò la misura di tempo con il minore
$\frac{\Delta T_0}{T_0}$ per ridurre al minimo l'errore.

\subsection{L'incertezza sulle misure del tempo}
Dato che
\begin{gather*}
    T = T_0\left(1 + \frac{1}{4}\sin^{2}\frac{\phi_0}{2}\right)
\end{gather*}
abbiamo allora da trovare $T_0$ e quindi trovare la sua incertezza relativa usando le derivate logaritmiche:
\begin{gather*}
    \frac{dT_0}{T_0} = d\ln T - d\ln\left(1 + \frac{1}{4}\sin^{2}\frac{\phi_0}{2}\right)
\end{gather*}
e quindi, con passaggi semplici e dato che $\phi_0 > 0$:
\begin{align}
    \frac{\Delta T_0}{T_0} = \frac{\Delta T}{T} + \frac{1}{8} \frac{\sin\phi_0}{1 + \frac{1}{4}\sin^{2}\frac{\phi_0}{2}} \cdot  \Delta \phi_0
\end{align}
Possiamo allora dare anche una stima a priori dell'incertezza: infatti, dato che il termine 
dipendente dal seno nel calcolo di $T$ è dell'ordine di $10^{-2}$ (o anche minore a seconda degli angoli),
se volessimo dare una sola valutazione dell'ordine di grandezza del periodo, allora potrei dire
che $T = T_0 = 2\pi\sqrt{\frac{l}{g}}$, usando il valore approssimativo di $9.8 m/s^{2}$ per $g$. 
Possiamo anche valutare l'ordine di grandezza che moltiplica $\Delta \phi_0$ a priori
ossia di circa $10^{-4}$ e dunque l'errore relativo su $T$ è molto più influente
di questo termine. Sarebbe quindi ragionevole non aspettarsi valori di
$\Delta T_0$ troppo distanti dal valore di $\Delta T$. 

\subsection{Incertezza di $g$}
Possiamo allora passare alla derivazione di $g$ attraverso la relazione  $g = 4\pi^{2}\frac{l}{T_0^{2} }$ 
e ottenere la sua incertezza relativa con il metodo delle logaritmiche:
\begin{align}
    \frac{\Delta g}{g} = \frac{\Delta l}{l} + \frac{2\Delta T_0}{T_0}
\end{align}
Ma in realtà tale relazione non è formalmente corretta in quanto dobbiamo
ricordare che $T_0 = T_0(T, \phi_0)$. La precedente relazione dunque è
giusta se e solo se supponiamo $l$ e $\phi_0$ come indipendenti (il che non è vero). 
Possiamo tuttavia vedere come questa differenza è in realtà irrilevante ai fini della
nostra esperienza in laboratorio: infatti riprendendo un passo dello sviluppo
della logaritmica:
\begin{gather*}
    \frac{d g}{g} = \frac{dl}{l} - 2\frac{dT_0}{T_0}, \qquad \frac{d T_0}{T_0} = \frac{dT}{T} - \frac{1}{8}\frac{\sin\phi}{1 + \frac{1}{4}+ \sin^{2}\frac{\phi_0}{2}}d\phi_0
\end{gather*}
E quindi ricordando che 
\begin{gather*}
    \sin\phi_0 = \frac{x_0}{l}, \qquad  \Delta \phi_0 = \tan\phi_0 \left(\frac{\Delta x_0}{x_0} + \frac{\Delta l}{l}\right)
\end{gather*}
Allora si ottiene che
\begin{gather*}
    d\phi_0 = \tan\phi_0 \frac{dx_0}{x_0} - \tan\phi_0 \frac{dl}{l}
\end{gather*}
Sostituendo allora in $\frac{d T_0}{T_0}$ si ottiene che l'errore relativo su $g$ è
\begin{gather*}
    \frac{\Delta g}{g} = \frac{\Delta l}{l}\left| 1 - \frac{1}{4} \frac{\sin\phi_0 \tan\phi_0}{1+  \frac{1}{4} \sin^{2}\frac{\phi_0}{2} } \right| + 2\frac{\Delta T}{T} + \frac{\sin\phi_0\tan \phi_0}{1 + \frac{1}{4}\sin^{2}\frac{\phi_0}{2} } \frac{\Delta x_0}{x_0} 
\end{gather*}
Possiamo ora ragionare sulla differenza tra le due rappresentazioni.
\begin{enumerate}
    \item $\Delta l$: dato che è dell'ordine di un decimo di millimetro $(0.25mm)$ allora posso trascurare il suo
    fattore moltiplicativo.
    \item $\Delta T_0$ l'unica differenza è un termine a prodotto $\tan\phi_0 \frac{\Delta x_0}{x_0}$ che è trascurabile
    in quanto $\Delta \phi_0 = \tan\phi_0 \frac{\Delta x_0}{x_0} + \tan\phi_0 \frac{\Delta l}{l}$ e dato che il primo termine è
    maggiore del secondo in quest'ultima relazione, posso dire che il termine che si somma dopo l'errore
    relativo su $T$ è essenzialmente non distinguibile da $\frac{\Delta T_0}{T_0}$. 
\end{enumerate}
Date queste considerazioni, possiamo allora continuare ad utilizzare la relazione più semplice trovata prima
in quanto l'influenza sulla misura dell'errore è minima in questa esperienza.

\section{Oltre al moto ideale, le condizioni di validità}
Nel pendolo semplice in moto ideale, si ha una descrizione appropriata
degli aspetti fondamentali del fenomeno del moto del pendolo, tuttavia queste
considerazioni sono possibili se e solo se il pendolo ha delle "caratteristiche
che lo rendono ideale". Generalmente con la configurazione fisica del 
laboratorio tali effetti sono trascurabili se e solo se non sono 
rilevabili quantitativamente con la nostra misura. Si trattano per
poter capire entro quali limiti l'effetto è trascurabile e quindi 
riprendendo le ipotesi di validità: 
\begin{enumerate}
    \item Pendolo di massa $m$ assimilabile a puntiforme e con dimensioni e distribuzioni di massa trascurabili;
    \item Filo con massa nulla;
    \item Filo inestensibile;
    \item Moto ideale: nessun effetto dissipativo dovuto all'attrito nell'aria.
\end{enumerate}
A partire da queste, possiamo ricavare tre situazioni fisiche "reali" che possiamo 
isolare una ad una:
\begin{itemize}
    \item Caso A: Non valgono le ipotesi $1$ e $2$;
    \item Caso B: Non vale l'ipotesi 3 (non nel programma);
    \item Caso C: Non vale l'ipotesi 4.
\end{itemize}

\section{Caso A: Il pendolo fisico}
\begin{wrapfigure}{r}{0.4\textwidth}
    \centering
    \caption{Il pendolo approssimabile ad un corpo rigido}
    \begin{tikzpicture}
        \draw(0, 0) rectangle (0.5, -4);
        \draw(0.25, -4.5) circle (0.5);
        \filldraw(0.25, 0) circle (1pt) node[anchor = south] {$O$};
        \filldraw(0.25, -2) circle (1pt) node[anchor = south] {$B$};
        \filldraw(0.25, -4) circle (1pt) node[anchor = south] {$P$};
        \filldraw(0.25, -4.5) circle (1pt) node[anchor = south] {$C$};
    \end{tikzpicture}    
\end{wrapfigure}
Innanzitutto consideriamo che il raggio della sfera non sia trascurabile
così come la massa del filo: ovviamente sempre secondo 
le approssimazioni che il raggio della sfera sia molto minore
del filo e che la massa del filo sia molto minore della massa
della sfera. Per poter trattare questo caso assumiamo che il filo
sia approssimabile ad una sbarretta rigida sottile ed omogenea, il che
è valido in quanto il filo è teso e si hanno le ipotesi
di inestensibilità. Se questa sbarretta sottile e cilindrica è ideale,
allora il pendolo costituirà un corpo rigido.
\begin{gather*}
    OP = L_f \qquad OB = BP = \frac{L_f}{2} \qquad OC = l = L_f + R
\end{gather*}
Sotto l'azione delle forze il sistema inizierà un moto oscillatorio
se lo lasciamo andare con un certo angolo $\phi$ rispetto alla
verticale e quindi l'angolo durante l'oscillazione sarà dato dalla seguente:
\begin{gather*}
    \phi = \phi(t)
\end{gather*}
Senza nessun attrito il vincolo in $O$ è considerato ideale
e le sole forze che hanno effetto sul moto del sistema sono le forze peso
che possiamo considerare applicate in $B$ e $C$. Essenzialmente il pendolo
si mette a fare un moto di rotazione oscillatorio attorno ad un asse fisso e quindi 
possiamo descriverlo con la seconda cardinale della dinamica si ottiene che in una sua forma sferica
per i corpi rigidi in rotazione attorno ad un asse fisso.
\begin{gather*}
    I_0 \dot{\omega} = M_{O, z}
\end{gather*}
$\dot{\omega}$ è l'accelerazione angolare, ossia $\ddot{\phi}$, mentre
$M$ è il momento assiale totale delle forze esterne rispetto al Polo $O$.
In questa equazione si ha il momento di inerzia del corpo rigido
rispetto all'asse fisso di rotazione passante per $O$, che è una grandezza
scalare estensiva che esprime le caratteristiche di inerzia di un corpo rispetto 
alla rotazione; è sempre riferito ad un asse e dipende sempre 
dalla distribuzione di massa del corpo rispetto al'asse 
a cui è riferito. Per un sistema di $N$ punti materiali il momento di inerzia
rispetto ad un dato asse $A$ è dato da
\begin{gather*}
    I_A = \sum m_i r_{A_i}^{2} 
\end{gather*}
Mentre per un corpo con distribuzione continua di massa si ha
\begin{gather*}
    I_A = \int r_A^{2} \ dm 
\end{gather*}
Ora per il Teorema di Huygens-Steiner si ha una relazione tra i momenti di inerzia
relativi ad assi paralleli: in questo caso il teorema ci dice che è possibile sommare
i contributi per ottenere il momento di inerzia definitivo e sommare il prodotto della
massa totale per il quadrato della distanza tra i due assi. 
\begin{align}
    I_A = I_{CdM \parallel A} + md^{2} 
\end{align}
Il nostro pendolo fisico rispetto ad un asse di rotazione fisso passante per $O$
avrà un momento di inerzia come
\begin{gather*}
    I_0 = I_{sfera, O} + I_{filo, O} 
\end{gather*}
Con
\begin{gather*}
    I_{sfera, O} = I_{sfera, C} + ml^{2} \\
    I_{filo, O} = I_{filo, C} + m_f\left(\frac{L_f}{2}\right)^{2}  
\end{gather*}
dobbiamo allora valutare i momenti di inerzia dei due componenti del nostro corpo
rigido e quindi il sistema (filo-sbarretta) rispetto agli assi passanti per i rispettivi
centri di massa e paralleli agli assi di rotazione. 

\subsection{Il momento di inerzia di una sfera rispetto ad un asse passante per il centro di massa}
\begin{wrapfigure}{r}{0.4\textwidth}
    \centering
    \caption{Il momento di una sfera: schematizzazione}
    \begin{tikzpicture}[scale=2]
        \draw[thick] (0,0) circle(1); % Cerchio esterno

% Asse z
\draw[->, thick] (0,-1.2) -- (0,1.2) node[above] {\(z\)};

% Etichette z = -R, z = 0, z = +R
\node at (0,-1.05) {\(z = -R\)};
\node at (0,0) [left] {\(C\), \(z=0\)};
\node at (0,1.05) {\(z = +R\)};

% Raggio R
\draw[dashed] (0,0) -- (0.866,0.5); % Raggio inclinato, coordinate (cos(30), sin(30))
\node at (0.5,0.3) {\(R\)};

% Piano orizzontale (sezione sferica)
\draw[dashed] (-0.8,0.6) arc (180:360:0.8 and 0.15); % Parte inferiore visibile
\draw[dashed] (0.8,0.6) arc (0:180:0.8 and 0.15);   % Parte superiore tratteggiata

% Etichetta z = r
\node at (0,0.65) [right] {\(z = r\)};

% raggio orizzontale
\draw[dashed] (0,0.6) -- (0.8,0.6);
\node at (0.4,0.65) {\(r\)};

% punto P
\filldraw[red] (0,0.6) circle(0.02);
\node[red] at (-0.05,0.6) {\(P\)};

% punto sulla circonferenza sezione
\filldraw[red] (0.8,0.6) circle(0.02);

% Griglia o linee tratteggiate nel piano sezione (opzionale)
\foreach \x in {-0.6,-0.3,0,0.3,0.6}
    \draw[gray!60, dashed] (\x,0.6) -- (\x+0.1,0.6);
    \end{tikzpicture}    
\end{wrapfigure}

Nel caso di una sfera omogenea, si ha massa $m$ e raggio $R$  e quindi il momento di
inerzia rispetto ad un asse $A$ diventa 
\begin{gather*}
    I_A = \int r_A^{2} \ dm = \rho \int r_A^{2} \ dV  
\end{gather*}
dove $r_A$ è la distanza dell'elemento di volume infinitesimo rispetto all'asse A.
Per calcolare allora il momento di inerzia facciamo a fette la sfera con fette normali all'asse
$z$ e dunque con spessore infinitesimo e raggio $r = (R^{2} - z^{2}  )^{1/2}$. 
Dove $z$ è la distanza tra $C$ e $P$. Ogni fetta è quindi un disco di raggio
$r$ definito prima e spessore infinitesimo $dz$ e quindi il momento di inerzia
totale della sfera è 
\begin{gather*}
    I_{sfera, C} = \int d I_{disco} \dots
\end{gather*}
Possiamo allora considerare il generico disco e con $-R \leq z \leq R$ ottengo tutti i 
dischi infinitesimi che compongono la sfera: allora per calcolare il momento di inerzia 
del singolo disco, devo potere esprimere una relazione tra il volume e la massa del singolo dischetto:
ossia posso individuare gli elementi di volume $dV$ con punti alla solita distanza
$x$ dal centro del disco così che poss valutare tutti i contributi infinitesimi al
disco considerato e, dato $dV = 2\pi x \ dx dz$ ottengo:
\begin{gather*}
    dI_{disco} = \rho \int_{0}^{r}x^{2} \ dV = \rho \int_0^{r}  x^{2}2\pi x \ dx dz = \\
    2\pi \rho \ dz \int_0^{r} x^{3} \ dx = 2\pi \rho \ dz \frac{r^{4} }{4} 
\end{gather*}
Ossia
\begin{align}
    I_{disco} \frac{\pi}{2}\rho r^{4}(z) \ dz 
\end{align}

Posso allora integrare tutti i dischi ed ottenere:
\begin{gather*}
    I_{Sfera, C} = \int d I_{disco}(z) = \frac{\pi}{2}\rho\int_{-R}^{R}(R^{2}  - z^{2} )^{2}\ dz   
\end{gather*}
Ottenendo infine
\begin{align}
    I_{sfera, C} = \frac{2}{5}mR^{2} 
\end{align}

\subsection{Momento di inerzia di una sbarretta rigida sottile e omogenea rispetto ad un asse 
passante per il centro di massa e parallelo a quello di rotazione}
Preso in considerazione il disegno del pendolo fisico fatto prima,
allora sappiamo che
\begin{gather*}
    OB = BP = \frac{L_f}{2} \Rightarrow OP = L_f
\end{gather*}
Allora il momento di inerzia del filo (o meglio della sbarretta rigida in questa
schematizzazione), dove $S$ è l'area della sezione:
\begin{gather*}
    I_{filo, B\parallel} = \rho_f \int y^{2} \ dV = \rho_f S \int_{ - \frac{L_f}{2}}^{\frac{L_f}{2}} y^{2} \ dy 
\end{gather*}
Ottenendo, sapendo che $\rho_f S L_f = m_f$ 
\begin{align}
    I_{filo, B\parallel} = m_f \frac{L_f^{2} }{12}
\end{align} 

\subsection{Il moto ideale del pendolo fisico}
NOTA: questa sezione non è importante saperla "a memoria" in quanto
gran parte di queste cose sono solo passaggi e non servono ai fini dell'orale.
Data l'equazione di moto 
\begin{gather*}
    I_0 \ddot{\phi} = M_{O, z}
\end{gather*}
Il momento di inerzia totale rispetto alla rotazione diventa: con
il teorema di Huygens-Steiner
\begin{gather*}
    I_0 = m\left(l^{2} + \frac{2}{5}R^{2} \right) + m_f \frac{L_f^{3} }{3}
\end{gather*}
Come si può vedere il momento di inerzia dipende strettamente 
dai parametri fisici del sistema e quindi possiamo valutare il momento assiale. \\
Il momento assiale ha come vincolo ideale $O$ e si presuppone che permetta
un moto ideale e trascurando quindi ogni forza di attrito esistente 
le sole forze esterne contribuiscono al momento assiale rispetto al
polo $O$ sono le forze peso e quindi per il calcolo del momento delle forze peso possiamo considerare 
$m\vec{g}$ applicata in $C$ e $m_f\vec{g}$ applicata in $B$ e quindi
dato che $\vec{g} = -g\cos\phi \hat{e}_n - g\sin\phi \hat{e}_{\phi}$ il momento diventa
\begin{gather*}
    \vec{M}_0 = \overline{OC} \times m\vec{g} + \overline{OB} \times m_f \vec{g} = \\
       (-l\hat{e}_n ) \times (-mg\cos\phi \hat{e}_n - mg \sin\phi \hat{e}_{\phi}  ) + \left(-\frac{L_f}{2} \hat{e}_n \right) \times (-m_f g \cos\phi \hat{e}_n - m_f g \sin \phi \hat{e}_{\phi}  )
\end{gather*}
Allora si ottiene che il vettore momento delle forze peso rispetto al polo $O$
diventa proprio
\begin{align}
    \vec{M}_0 = -\left(ml + m_f \frac{L_f}{2}\right)g\sin\phi \hat{e}_z 
\end{align}

Ora per ottenere il momento assiale lungo l'asse di rotazione
fisso $z$ proiettiamo il momento risultante di quelle forze e 
moltiplicandolo per $\hat{e}_z$ e quindi tornando all'equazione di moto

\begin{align}
    \ddot{\phi} + \left(g\frac{\left(ml + m_f \frac{L_f}{2}\right)}{I_0}\right)\sin\phi = 0
\end{align}
Rispetto al pendolo semplice che è invece
\begin{gather*}
    \ddot{\phi} + \frac{g}{l}\sin\phi = 0
\end{gather*}

Possiamo definire una variabile definita dai parametri del
sistema $(m, m_f, R, l, L_f)$ e ottenere quindi
\begin{gather*}
    L_e = \frac{I_0}{ml + m_f\frac{L_f}{2}}
\end{gather*}
Possiamo allora riassumere le relazioni per il pendolo semplice e fisico come segue:
\begin{align*}
    &FISICO& \qquad \qquad &SEMPLICE&\\
    &\ddot{\phi} + \left(\frac{g}{L_e}\right)\sin\phi = 0& \qquad &\ddot{\phi} + \left(\frac{g}{l}\right)\sin\phi = 0& \\
    &T_{0_f} = 2\pi \sqrt{\frac{L_e}{g}}& \qquad &T_{0_s} = 2\pi \sqrt{\frac{l}{g}}& \\
    &T_f = 2\pi \sqrt{\frac{L_e}{g}} F(\phi_0)& \qquad &T_s = 2\pi \sqrt{\frac{l}{g}} F(\phi_0) &
\end{align*}
Dove ovviamente 
\begin{gather*}
    L_e = \frac{m\left(l^{2} + \frac{2}{5}R^{2}  \right) + \frac{m_fL_f^{2} }{3}}{ml + m_f\frac{L_f}{2}}
\end{gather*}
Se poniamo il fattore correttivo come
\begin{gather*}
    F_c = \frac{L_e}{l} = \frac{1 + \frac{2}{5}\frac{R^{2} }{l^{2} } + \frac{1}{3}\frac{m_f}{m}}{1 + \frac{1}{2}\frac{m_f}{m}}
\end{gather*}
E quindi tornando ai periodi nelle due schematizzazioni si ha che
diventano rispettivamente
\begin{gather*}
    T_{f} = T_s \sqrt{F_c} \\
    T_{0_f} = T_{0_s} \sqrt{F_c}  
\end{gather*}
Il termine $\sqrt{F_c}$ tiene conto della differenza tra i periodi
delle due schematizzazioni sempre tenendo conto delle nostre condizioni,
ossia $R << l$ e $m_f << m$ e quindi
\begin{gather*}
    \sqrt{F_c} = \left(1 + \left(\frac{2}{5}\frac{R^{2} }{l^{2} } + \frac{1}{3}\frac{m_f}{m}\right)\right)^{\frac{1}{2}} \cdot  \left(1 + \frac{1}{2}\frac{m_f}{m}\right)^{-\frac{1}{2}} = (1 + x)^{\frac{1}{2}}\cdot (1 + x')^{-\frac{1}{2}}     
\end{gather*} 
Posto che i termini sommati ad $1$ siano molto più piccoli di $1$, allora
posso utilizzare gli sviluppi di Taylor:
\begin{gather*}
    \sqrt{F_c} \approx 1 + \frac{x}{2}- \frac{x'}{2} \\
    \sqrt{F_c} \approx 1 + \frac{1}{5}\frac{R^{2} }{l^{2} } - \frac{1}{12}\frac{m_f}{m} 
\end{gather*}
Con le condizioni $R << l$ e $m_f << m$
\begin{align}
    \sqrt{F_c} \approx 1 + \frac{1}{5}\frac{R^{2} }{l^{2} } - \frac{1}{12}\frac{m_f}{m} = \frac{T_f}{T_s} = \frac{T_{0_f}}{T_{0_s}}
\end{align}
Adesso è possibile valutare se gli effetti del pendolo fisico possono o meno
incidere sulla descrizione della nostra esperienza

\subsection{Confronto delle due schematizzazioni}
Il nostro scopo è quello di valutare la differenza tra le due schematizzazioni
per identificare possibili differenze nel calcolo del periodo
del pendolo. Possiamo innanzitutto cercare l'errore di schematizzazione
relativo con la differenza relativa tra i periodi:
\begin{gather*}
    \left(\frac{\delta T}{T}\right)_{f, schem.} = \frac{|T_f -T_s|}{T_s} = \frac{|T_S \cdot (1 + \frac{1}{5}\frac{R^{2} }{l^{2} } - \frac{1}{12}\frac{m_f}{m}) - T_s|}{T_s} = \left|\frac{1}{5}\frac{R^{2} }{l^{2} } - \frac{1}{12}\frac{m_f}{m}\right|
\end{gather*}
Anche in questo caso per comprendere se la differenza sui 
periodi prevista è rilevabile e significativa bisognerà confrontare
con l'incertezza sperimentale relativa delle misure
\begin{gather*}
    \left(\frac{\delta T}{T}\right)_{f, schem} << \left(\frac{\Delta T}{T}\right)_{sper}
\end{gather*}
In questo caso si ha che la schematizzazione di pendolo semplice 
era effettivamente adeguata e quindi le differenze rispetto all'incertezza erano
trascurabili e non significative né sensibili. Altrimenti occorreva
adottare la schematizzazione di pendolo fisico. 
Per la nostra esperienza in laboratorio si ha che
\begin{gather*}
    \left(\frac{\Delta T}{T}\right)_{sper} \approx 10^{-3} 
\end{gather*}
Generalmente in laboratorio conviene applicare la schematizzazione di
pendolo semplice in moto ideale. 

\subsection{Quanto influisce l'errore di schematizzazione sulla misura di $g$?}
Supponendo che $m_f$ non sia trascurabile, in tal caso bisognerebbe uguagliare il periodo
$T$ misurato a $T_f$ e quindi il  $g$ vero sarà
\begin{gather*}
    g_{vero} = \frac{4\pi^{2}l }{T_{0_f}^{2} } \left(1 - \frac{1}{12}\frac{m_f}{m}\right)^{2} 
\end{gather*}
In questo caso dunque, si avrebbe (dato lo sviluppo $(1 - x)^{2} \approx 1 - 2x$ ) che
\begin{gather*}
    g_{vero} \approx \frac{4\pi^{2}l }{T_{0_f}^{2} } \left(1 - \frac{1}{6}\frac{m_f}{m}\right)
\end{gather*}
Se in queste condizioni dovessi allora adottare la schematizzazione di 
pendolo semplice ideale, allora lo stesso $T_0$ misurato dovrebbe essere uguagliato a 
$T_0$ e allora si otterrebbe
\begin{gather*}
    g_s = \frac{4\pi^{2} l}{(T_0)^{2}_{misurata}}
\end{gather*}
E si ottiene che $g_{vero} < g_s$. Se avessi una incertezza tale da poter mettere
a confronto le differenze allora dovrei mettere a confronto:
\begin{gather*}
    \frac{|g_f - g_s|}{g_s} \approx  \frac{1}{6}\frac{m_f}{m}
\end{gather*}
Che per la schematizzazione di pendolo semplice è sufficiente considerare $g_s$.

\section{Il caso C: il pendolo ideale ma in presenza di attrito}
Realisticamente si osserva che nel moto del pendolo in laboratorio
ha una progressiva diminuzione dell'ampiezza angolare dell'oscillazione 
col crescere del numero di oscillazioni compiute dal rilascio del pendolo
smorzando l'oscillazione. Posso allora determinare l'angolo
di oscillazione in funzione del numero totale di oscillazioni. \\
Dall'analisi sperimentale si è in grado di quantificare gli attriti
presenti nel moto reale e quindi determinare se l'adozione della schematizzazione
di moto ideale sia effettivamente adeguata e corretta.  \\
Inoltre dobbiamo determinare se $T_0$ che abbiamo misurato indirettamente sia effettivamente
corretto e quindi determinare se la descrizione degli attriti di piccola entità che 
agiscono sul pendolo il periodo ne rimanga influenzato significativamente
dalla presenza e dall'azione di forze di attrito. 
\begin{gather*}
    \left(\frac{\delta T}{T_0}\right) = \frac{|T_0' - T_0|}{T_0}
\end{gather*}
Con $T_0'$ indico il periodo limite nel caso reale e quindi l'effetto
sul periodo nel limite delle piccole oscillazioni è tale che l'incertezza
della schematizzazione sia molto minore di quella sperimentale:
Se $T_0' \approx T_0$ il periodo non differisce molto da $T_0$ e quindi mi trovo nel caso
ideale in modo significativo per la nostra configurazione sperimentale e l'effetto
della presenza di attriti sul periodo di piccole oscillazioni è trascurabile per il nostro sistema pur
essendo osservabile sull'ampiezza delle piccole oscillazioni. Inoltre, la nostra stima del
periodo $T_0$ è corretta rispetto alle incertezze di misura  e quindi 
$g$ è ugualmente corretta. 

\subsection {Come descrivere le forze di attrito}
Esistono diverse possibili forze di attrito: la resistenza della viscosità dell'aria e gli
attriti del sistema al punto di sospensione del pendolo. Sia la rilevanza che la descrizione delle forze dipendono da
come è stato realizzato il pendolo ed è in generale molto complessa.
Possiamo rappresentare la viscosità dell'aria secondo una 
semplificazione che dipende da:
\begin{itemize}
    \item Caratteristiche del fluido;
    \item Dimensioni e caratteristiche geometriche
    \item Velocità del corpo nel fluido.
\end{itemize}

Sempre secondo una semplificazione notevole, si possono
identificare due regimi limite per il fenomeno distinti da un valore caratteristico
della velocità indicato con $v_l$:
\begin{itemize}
    \item $v >> v_l$: l'attrito si esplica in modo vorticoso tramite la formazione di
    vortici nella scia del corpo in moto e un questo caso la descrizione è molto complessa;
    \item $v < v_l$: si ha un regime laminare ossia l'attrito  è dato da lamine di fluido che
    scorrono sopra al corpo.
\end{itemize}

Ai fini della semplicità della schematizzazione scegliamo di scrivere
la forza di attrito come
\begin{gather*}
    \vec{F}_a = -b \vec{v}  
\end{gather*}

\subsection{Equazione di moto per il pendolo semplice in moto reale}
Torniamo all'equazione che esprime il moto del pendolo ideale aggiungendo
òle forze agenti su $C$ dell'attrito e quindi si avrà che
\begin{gather*}
    \vec{F}_a + \vec{T} + m\vec{g} = m\vec{a}    \\
    ml \ddot{\phi} + bl \dot{\phi} + mg\sin\phi = 0 \\
    \ddot{\phi} + \frac{b}{m}\dot{\phi} + \frac{g}{l}\sin\phi = 0
\end{gather*}
In questa forma generale anche questa equazione non
è risolubile analiticamente in modo semplice ma anche in questo
caso esiste un limite in cui l'equazione di moto si linearizza e 
ha una risoluzione analitica semplice data la 
terza condizione

\subsection{Limite delle piccole oscillazioni}
In modo del tutto analogo al caso del pendolo semplicemente
in moto ideale possiamo anche per il presente caso di moto non
ideale definire il limite delle piccole oscillazioni come il caso in cui
le oscillazioni sono di ampiezza $\phi_0$ sufficientemente piccola da consentire 
di poter approssimare con Taylor $\sin\phi \approx \phi$ e quindi 
l'equazione diventa
\begin{gather*}
    \ddot{\phi} + \frac{b}{m}\dot{\phi} + \frac{g}{l}\phi = 0
\end{gather*}
Ponendo allora $\omega_0^{2} = \frac{g}{l}$ l'equazione di moto è 
diventato una equazione differenziale del secondo ordine lineare
omogenea e a coefficienti costanti e quindi
risolubile analiticamente. \\
Con una integrazione numerica dell'equazione nella sua forma generale
si può allora dimostrare che fintantoché gli effetti degli
attriti sono di piccola entità, e quindi la relazione
$\phi_0 = \phi_0(n)$ rientra nel limite delle piccole oscillazioni,
vale per descrivere l'andamento di $\phi_0$ con $n$ 
anche nel caso di oscillazioni di ampiezza generica e potremmo quindi
descrivere il fenomeno dello smorzamento basandoci su una legge
che è la stessa che ricaveremo dallo studio del caso limite 
delle piccole oscillazioni.

\section{Il limite delle piccole oscillazioni per il pendolo semplice in moto reale}
Data l'equazione di moto
\begin{gather*}
    \ddot{\phi} + \frac{b}{m}\dot{\phi} + \omega_0^{2}\phi = 0, \qquad \omega_0^{2} = \frac{g}{l}  
\end{gather*}
A noi ci interessa risolverla in modo tale che si osservi che
l'angolo $\phi$ decresca continuamente. Matematicamente
si traduce nella condizione 
\begin{gather*}
    \frac{b}{2m} < \omega_0
\end{gather*}
Se questa è soddisfatta allora la soluzione generale è data dalla seguente
\begin{gather*}
    \phi(t) = Ce^{-\frac{b}{2m}t} \cdot  \cos(\omega_0't - \gamma) 
\end{gather*}
Dove
\begin{itemize}
    \item $\omega_0'$ è la frequenza angolare dell'oscillazione smorzata:
    determinata sostituendo nell'equazione di moto l'equazione sopra;
    \item C, $\gamma$ costanti da determinare dalle condizioni iniziali
\end{itemize}
Per poter calcolare queste occorre conoscere $\ddot{\phi}$ e $\dot{\phi}$:
\begin{gather*}
    \left\{\begin{array}{l}
        \ddot{\phi} + \frac{b}{m}\dot{\phi} + \omega_0^{2} \phi = 0 \\
        \phi(t) = C\exp\left(-\frac{b}{2m}t\right) \cdot  \cos(\omega_0't - \gamma) 
    \end{array}
    \right.
\end{gather*}
Allora derivo due volte la seconda equazione ottenendo un espressione di $\ddot{\phi}$:
\begin{gather*}
    \dot{\phi} = C\exp\left(-\frac{b}{2m}t\right)\left(-\frac{b}{2m}\cos(\omega_0't - \gamma) - \omega_0'\sin(\omega_0't - \gamma)\right)\\
    \ddot{\phi} = C\exp\left(-\frac{b}{2m}t\right)\left(\left(-\frac{b}{2m}-\omega_0^{'2}\right) \cos(\omega_0't - \gamma) + \omega_0'\sin(\omega_0' t - \gamma)\right)
\end{gather*}
Sostituendo nell'equazione di moto si ottiene
l'espressione per lo smorzamento di $\phi$. 


\section{Procedure operative in laboratorio per lo smorzamento}
Come si effettuano le misure in laboratorio correttamente:
\begin{enumerate}
    \item Scegliamo l'ampiezza angolare $\phi^{in}$ come valore
    intermedio tra quelli misurati di massimo e minimo per la validità
    delle considerazioni fisiche e quindi una distanza $x_0$ dalla verticale
    associata;
    \item Rilasciamo il pendolo col centro di massa a distanza $x_0^{in}$ dalla 
    verticale locale e quindi $(x_0')^{in} = x_0^{in} + R$ e si conta
    al rilascio delle sferetta le oscillazioni complete e ogni
    $5$ oscillazioni si valuta la posizione $x_0(n)$ raggiunta dal centro di massa
    e si definisce allora $\delta x_0'(n)$ la diminuzione della distanza 
    dal bordo esterno della sferetta e possiamo misurare ed ottenere indirettamente
    la distanza $x_0'(n)$ raggiunta dalla massima elongazione n-esima come
    $x_0'(n) = x_0'(n = 0) - \delta x_0'(n)$. La misura allora di $\delta x_0'(n)$ può essere ripetuta più volte rilasciando
    altrettante volte il pendolo e seguendo la sua evoluzione in modo da poter
    valutare $\delta x_0'(n)$ con una stima migliore così come determinare
    il suo errore $\Delta (\delta x_0'(n))$ e confrontarlo con $0.5 cm$, ossia il valore
    considerabile come la minima variazione ragionevolmente apprezzabile e quindi 
    \begin{gather*}
        \left\{\begin{array}{l}
            \text{scarto max } > 0.5 cm \Rightarrow \Delta(\delta x_0'(n)) = \text{ scarto max rispetto alla media} \\
            \text{scarto max } < 0.5 cm \Rightarrow \Delta(\delta x_0'(n)) = 0.5cm
        \end{array}\right.
    \end{gather*}
    Avendo allora ottenuto $\delta x_0'(n) \pm \Delta (\delta x_0'(n))$ possiamo
    valutare ora $x_0'(n) \pm \Delta x_0'(n)$  come
    \begin{gather*}
        \left\{\begin{array}{l}
            x_0'(n) = x_0'(n = 0) - \delta x_0'(n) \\
            \Delta x_0'(n) = \Delta (x_0'(n = 0)) + \Delta (\delta x_0'(n)) = 0.5cm + \Delta (\delta x_0 '(n))
        \end{array}\right.
    \end{gather*}
    Per la relazione che intercorre tra $x_0'$ e $x_0$ i $\delta x_0'(n)$ misurati rappresentando direttamente
    anche di quanto al chiudersi della n-esima oscillazione è diminuita la massima
    elongazione dal centro ossia di quanto è diminuita $x_0$ 
    e dunque è $\delta x_0'(n) = \delta x_0(n)$ e quindi
    \begin{gather*}
        x_0(n) = x_0'(n) - R = (x_0'(n = 0) - \delta x_0'(n)) - R  = x_0(n = 0) - \delta x_0'(n)
    \end{gather*}
    E quindi l'incertezza associata è
    \begin{gather*}
        \Delta x_0(n) = \Delta x_0(n = 0) + \Delta (\delta x_0'(n)) = 0.5 cm + \Delta (\delta x_0'(n))
    \end{gather*}
    \item $x_0(n) \pm \Delta x_0(n)$ ricaviamo i valori $\phi_0 \pm \Delta \phi_0(n)$ e 
    si riporta in grafico i valori in funzione dell'n corrispondente e ci si aspetta che
    i punti del grafico mostrino un andamento ben descrivibile da una relazione lineare
    $\phi_0(n) = A + Bn$ e dove $A = \phi^{in}$ e $B$ negativo. Si può concludere
    che nel limite delle piccole oscillazioni basandoci sulla descrizione del moto 
    come ideale data la presenza delle forze di attrito, il cui contributo 
    energetico è piccolo rispetto all'energia totale del moto. Nella
    nostra specifica configurazione sperimentale la differenza sul periodo
    in questo limite delle piccole oscillazioni risulta assolutamente non rilevabile 
    rispetto alle incertezze di misura sul periodo $T_0$.
\end{enumerate}

\subsection{Analisi dello smorzamento}
Se siamo sicuri che $\frac{b}{2m\omega_0} <<1$ allora posso dare un'occhiata migliore
alla nostra $\phi_0(n) = \phi_0^{in} \exp\left(-\frac{bT_0' n}{2m}\right)$. Infatti se siamo sicuri 
di questo, allora si avrà lo stesso ragionamento per $\frac{b}{2m\omega_0'}$ poiché
\begin{gather*}
    \frac{b}{2m\omega_0'} = \frac{b}{2m\omega_0}\left(1 - \left(\frac{b}{2m\omega_0}\right)^{2} \right)^{\frac{1}{2}} 
\end{gather*}
Si può allora risolvere con lo sviluppo di Taylor ed ottenere che
sono circa uguali. Se questo è vero allora si ha che, sostituendo:
\begin{gather*}
   \frac{b}{2m\omega_0'} = \frac{bT_0'}{2m2\pi} << 1 \\
   \frac{bT_0'}{2m} << 2\pi \ \Longrightarrow \ \frac{bT_0'}{2m} << 1
\end{gather*}
Supponendo che la nostra situazione sperimentale ci consenta questo allora
di definire la funzione rispetto a $\phi(n)$ come
\begin{gather*}
    \phi(n) \approx \phi_0^{in} - \frac{\phi_0^{in}bT_0' }{2m}n 
\end{gather*}
\begin{wrapfigure}{r}{0.4\textwidth}
    \centering
    \caption{Il grafico dello smorzamento}
    \begin{tikzpicture}
        \draw[->](0, 0) -- (4, 0) node[at end, below] {$n$};
        \draw[->](0, 0) -- (0, 4) node[at end, left] {$\phi_0(n)$};
        \draw(0.75, 0.2) -- (0.75, -0.2) node[at end, below] {$5$};
        \draw(1.5, 0.2) -- (1.5, -0.2) node[at end, below] {$10$};
        \draw(2.25, 0.2) -- (2.25, -0.2) node[at end, below] {$15$};
        \draw(3, 0.2) -- (3, -0.2) node[at end, below] {$20$};
        \draw(3.75, 0.2) -- (3.75, -0.2) node[at end, below] {$25$};
        \draw[|-|](0, 3) -- (0, 3.75);
        \draw[|-|](0.75, 2.75) -- (0.75, 3.5);
        \draw[|-|](1.5,  2.5) -- (1.5, 3.25);
        \draw[|-|](2.25, 2.25) -- (2.25, 3);
        \draw[|-|](3, 2) -- (3, 2.75);
        \draw[|-|](3.75, 1.75) -- (3.75, 2.5);
        \filldraw(0, 0) circle (0pt) node[anchor = north] {$0$};
    \end{tikzpicture}    
\end{wrapfigure}
Posso allora determinare $\frac{b}{2m\omega_0}$ attraverso il metodo grafico:
posso infatti calcolare ogni $5$ misure (ossia per $n \in \{5, \dots, 25\}$) il grafico
con con l'ampiezza $\phi_0(n)$ sulle ordinate ed il numero di oscillazioni sulle ascisse. 
Se vedo che i punti indentificano in modo chiaro un andamento lineare, allora posso 
capire che il mio termine $\frac{bT_0'}{2m}n << 1$. E allora il nostro risultato è verificato.
Inoltre posso anche verificare che $T_0' \approx T_0$ e che quindi la differenza
tra i due periodi non è qualitativamente misurabile nel nostro apparato sperimentale. 
Possiamo allora costruirci l'errore di schematizzazione relativo per avere una quantità
che ci permetta di confrontare le due schematizzazioni in maniera quantitativa.
\begin{gather*}
    \frac{\delta T_0}{T_0} = \frac{1}{2}\left(\frac{b}{2m\omega_0}\right)^{2} 
\end{gather*}
Se riusciamo allora a ricavare una stima di questo valore potremmo confrontarlo con il 
nostro errore sperimentale. Dato che a noi interessa però l'ordine di grandezza,
sul grafico si traccia solo la retta best e avremmo quindi $\phi_0(n) = A + Bn$. Dove 
allora $B = -\phi_0^{in} \frac{bT}{2m}$ e $A = \phi_0^{in}$.
\begin{gather*}
    \frac{bT}{2m} = \frac{|B|}{\phi_0^{in} }
\end{gather*}
Che come ordine di grandezza è approssimativo a $\frac{bT_0}{2m}$ e dunque 
possiamo dire che
\begin{gather*}
    \frac{b}{2m\omega_0} \approx \frac{1}{2\pi}\frac{|B|}{\phi_0^{in} }
\end{gather*}
Allora la nostra incertezza sperimentale è proprio uguale all'incertezza su $T$. Ci si aspetta allora
una relazione del tipo:
\begin{gather*}
    \frac{\delta T_0}{T_0} << \frac{\Delta T}{T}
\end{gather*}


\end{document}
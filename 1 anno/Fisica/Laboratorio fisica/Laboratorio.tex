\documentclass[a4paper, oneside]{book}
\usepackage{graphicx}
\usepackage{amsthm}
\usepackage{amsmath}
\usepackage[a4paper,
            bindingoffset=0.2in,
            left=2cm,
            right=2cm,
            top=2cm,
            bottom=2cm,
            footskip=.25in]{geometry}
\usepackage[italian]{babel}
\usepackage{pgfplots}
\usepackage{tabularx}
\usepackage{wrapfig} %wrapping images in text
\usepackage{color}
\definecolor{page}{rgb}{0.129,0.157,0.212}
\pagecolor{page}
\color{white}
\graphicspath{ {./images/} }
\usetikzlibrary{datavisualization}
\usetikzlibrary{datavisualization.formats.functions}
\pgfplotsset{width=10cm,compat=1.9}

\title{Laboratorio I}
\author{Tommaso Miliani}
\date{2024/2025}

\begin{document}
\theoremstyle{definition}
\theoremstyle{theorem}
\theoremstyle{lemma}

\newtheorem{definition}{Definizione}[chapter]
\newtheorem{theorem}{Teorema}[chapter]
\newtheorem{lemma}{Proposizione}[theorem]

\maketitle

\tableofcontents

\chapter{Le grandezze fisiche}
\section{Introduzione alle grandezze fisiche}
In fisica alle grandezze con cui si deve operare si da una
\textbf{definizione operativa}, ovvero si specifica se
due grandezze descrivono la stessa quantità. Le grandezze per cui si
da questa definizione operativa prendono il nome di \textbf{fondamentali}
mentre quelle che derivano da queste grandezze e per le quali non diamo alcuna definizione
operativa prendono il nome di \textbf{derivate}. \\
Una volta stabilita una definizione operativa per le
grandezze fondamentali si procede con lo scegliere un \textbf{valore campione}
e usarlo per misurare tutti gli oggetti. La meccanica ci permette di descrivere tutte
le grandezze a partire da tre grandezze fondamentali: Lunghezza, tempo e massa ($L, T, M$).

\section{Le tre grandezze fondamentali}
\subsection{Lunghezza}
Costruendo un piano di prova che sia geometricamente
astratto (ossia che non presenti alcuna imperfezione) ed
una retta fisica astratta, ossia una retta che ottengo 
levigando tra di loro tre blocchi di ferro in modo tale che
si levighino tra di loro ed incidendo su quello che è stato levigato
con gli altri due blocchi. Incrociando due rette fisiche astratte si ottiene
un punto fisico astratto. Attraverso una riga, che suppongo perfetta, posso determinare
la relazione tra i due segmenti (che si creano collegando due punti fisici astratti). \\
QUesto tuttavia mi darà informazioni solo se un segmento è più lungo
dell'altro ma non mi dice se è più lungo di altri segmenti; posso allora
suddividere la riga con dei segmenti piccoli in modo tale da poter
ottenere una \textbf{riga graduata} che mi permette di confrontare
un segmento con l'unità della mia riga.

\section{Tempo}
IL concetto di intervallo di tempo  fa uso del moto periodico
e quindi due eventi hanno la stessa durata se passano lo stesso numero di oscillazioni.

\subsection{Massa}
Definiamo la \textbf{bilancia delle masse} il carrello che quando lasciato
inizia ad oscillare in modo periodico e quindi
la massa è quella quantità che modifica il periodo di oscillazione del carrellino
(sto definendo la massa inerziale) per cui aumentando
la massa aumenterà il periodo di oscillazione del carrello. \\
Posso inserire un oggetto di volume maggiore e se aumenta il
periodo allora deve essere aumentata anche la massa: si osserva allora
che in qualche modo massa e volume sono collegati. Due corpi hanno la 
stessa massa se e solo se nel sistema del carrello hanno lo stesso
periodo di oscillazione.

\subsection{Forza}
La forza è una grandezza vettoriale che descrive la fatica che faccio per spostare
qualcosa, l'effetto dell'applicazione di una forza è quello di cambiare lo stato di moto. \\
Se le forze applicate sono più di uno non è sempre detto che mutino lo stato di moto:
potrebbero provocare delle deformazioni come nel caso delle molle, la cui forza
applicata è ricavata proprio a partire dalle deformazioni. \\
Posso misurare quindi quanta forza è applicata ad un oggetto attraverso un dinamometro,
ossia una molla attaccata al soffitto con una scala graduata di lato per
misurare l'estensione della molla. 

\section{Le grandezze derivate}
\subsection{Superficie}
Se, avendo una superficie piana, volessi misurarne l'estensione 
posso misurare la lunghezza del lato e posso allora definire
un quadratino di lato $1u$ e definisco l'unità fondamentale della superficie 
come 
\begin{gather*}
    S = l^{2} 
\end{gather*} 
Per altre figure ho
\begin{gather*}
    S = \pi r^{2}, \text{cerchio} \qquad S = bh, \text{Rettangolo} \qquad S = 2\pi r h, \text{cilindro} \\
    S = \frac{1}{2}bh, \text{triangolo}, \qquad S = 4 \pi r^{2}, \text{sfera}, \qquad S = l^{2}, \text{quadrato}   
\end{gather*}
Qualsiasi superficie è dunque il prodotto di due lunghezze per un fattore numerico
variabile a seconda della superficie da calcolare (che si ricava mediante l'integrazione),
le dimensioni fisiche, espresse mediante le equazioni dimensionali allora è proprio
\begin{gather*}
    [S] = [L^{2} \ M^{0}  \ T^{0} ]
\end{gather*}

\subsection{Volume}
Applico lo stesso ragionamento fatto per la superficie, il volume del piccolo
cubetto di lato $u$ è dato proprio da $V = l^{3}$.
\begin{gather*}
    V = abc, \text{cubo} \qquad V = \pi r^{2}h, \text{cilindro} \qquad V = \frac{4}{3}\pi r^{2}, \text{sfera} 
\end{gather*}  
Allora le dimensioni del volume sono proprio
\begin{gather*}
    [V] = [L^{3} \ M ^{0} \ T^{0}  ]
\end{gather*}

\subsection{Angolo}
Prendendo un cerchio e due semirette che si incrociano in un punto
$O$ di origine del cerchio, posso prendere arbitrariamente due punti 
sulla circonferenza intercettati dalle due semirette e definisco allora 
l'angolo come
\begin{gather*}
    \alpha = \frac{\overline{AB}}{r}
\end{gather*}
Da notare come l'angolo, dato che è un rapporto tra due dimensioni di lunghezza
è adimensionale. Esiste anche la definizione di angolo sessagesimale: prese le due semirette, le faccio
girare fino a che non coincidono e le posso poi dividere ulteriormente
per cui si ottiene la seguente conversione tra radianti e angoli sessagesimali:
\begin{gather*}
    \pi RAD = 180
\end{gather*}

\subsection{Angolo solido}
Posso passare ad una dimensione superiore rispetto all'angolo
piano e quindi posso, a partire da $O$ costruire un insieme di semirette che
coincidono con una superficie nello spazio. In
base alla superficie nello spazio. In base alla superficie 
scelta possiamo ottenere piramidi o coni. Posso definire allora l'ampiezza
dell'angolo solido, prendendo una sfera con
centro in $O$. \\
Facendo il rapporto tra l'area della superficie intersecata ed il raggio al
quadrato della sfera ottengo allora
l'ampiezza dell'angolo solido.
\begin{gather*}
    \Omega = \frac{S}{r^{2} }
\end{gather*}
Definito come steradiante; se prendessi una superficie coincidenti con 
quella di una sfera, avrei una sola retta e quindi 
\begin{gather*}
    \Omega = \frac{4\pi r^{2} }{r^{2} }
\end{gather*}
Ovviamente adimensionale.

\subsection{Densità media assoluta}
Si ottiene facendo il rapporto tra le masse del corpo ed
il suo volume
\begin{gather*}
    \delta = \frac{m}{V}, \qquad [\delta] = [L^{-3} \ M \ T^{0} ]
\end{gather*}

\subsection{Velocità scalare media}
Si può definire come il rapporto tra la traiettoria percorsa da un punto 
e il tempo impiegato a percorrerla 
\begin{gather*}
    <v> = \frac{\Delta S}{\Delta t}
\end{gather*}
E quindi dimensionalmente
\begin{gather*}
    [<v>] = [L \ M^{0} \ T^{-1}  ]
\end{gather*}

\subsection{Accelerazione scalare media}
L'accelerazione scalare media  è definita come
\begin{gather*}
    <a > = \frac{\Delta v}{\Delta t}, \qquad [<a>] = [L \ M^{0} \ T^{-2}  ]
\end{gather*}

\subsection{Frequenza}
Se c'è un fenomeno che si ripete possiamo usare la frequenza ovvero
l'inversa del periodo dell'evento.
\begin{gather*}
    f = \frac{1}{T}, \qquad [f] = [L^{0} \ M^{0} \ T^{-1}  ]
\end{gather*}

\subsection{Forza}
Facciamo riferimento al secondo principio della dinamica:
una forza applicata ad una massa modifica lo stato di moto 
\begin{gather*}
    \vec{F} = m\vec{a}, \qquad [F] = [L \ M \ T^{-2} ]  
\end{gather*}

\subsection{Pressione}
LA posso definire nel caso particolare di una forza agente su di una
superficie. Chiamo allora pressione il rapporto tra queste due grandezze e dunque
secondo l'analisi dimensionale posso andare a definire come
\begin{gather*}
    P = \frac{F }{S}, \qquad [P] = [L^{-1}  \ M \ T^{-2} ]
\end{gather*}

\subsection{Lavoro}
Si parla di lavoro quando il punto di applicazione della forza
si muove nello spazio. Posso definire il lavoro come 
\begin{gather*}
    L = F \Delta S \cos\theta
\end{gather*}
Facendo allora l'analisi dimensionale si ottiene

\begin{gather*}
    [L] = [L^{2} \ M \ T^{-2}  ]
\end{gather*}

\subsection{Potenza media}
\begin{gather*}
    <W> = \frac{L}{\Delta t}, [W] = [M \ L^{2} \ T^{-2}  ]
\end{gather*}

\section{Unità di misura}
\subsection{Sistema MKS}
La definizione ufficiale è l'acronimo di metro, kilogrammo e secondo. Le unità di misura associate alle
3 grandezze fondamentali lunghezza, massa e tempo. 
\begin{enumerate}
    \item Intervallo di tempo: per definirlo abbiamo visto che serve un fenomeno periodico
    Si potrebbe usare i moti periodici celesti come rotazione, rivoluzione... Per indicare il tempo 
    si definisce il secondo solare medio che è esattamente \begin{gather*}
        1s = \frac{d}{86400}
    \end{gather*}
    Ossia il giorno solare medio. Successivamente si è sostituito questa definizione con una più precisa che utilizza la propagazione delle
    onde elettromagnetiche da parte degli atomi. Si usò l'orologio atomico sfruttando la frequenza di oscillazione delle
    energie degli elettroni negli atomi e dunque si 
    è arrivati a descrivere il secondo come l'intervallo di tempo entro il
    quae avvengono $9'192 '632 '770$ oscillazioni
    \item Lunghezza: Si utilizza la radiazione elettromagnetica del Kripton
    per definire il metro come un particolare multiplo della sua onda elettromagnetica,
    poi si è deciso di definire il metro come 
    \begin{gather*}
        1m = \frac{c}{299'792'458}s
    \end{gather*}
    \item Massa: Si può definite come una certa frazione della costante di Plank:
    \begin{gather*}
        1 kg = \frac{h}{6.63607015 \cdot  10^{-34} } \cdot \frac{s}{m^{2} }
    \end{gather*}
\end{enumerate}

\subsection{IL sistema internazionale}
Il sistema internazionale è nato specificamente per poter 
utilizzare delle unità di misura ricavate con metodi più precisi dei
metodi antichi:
\begin{enumerate}
    \item Temperatura: Si può misurare in gradi Kelvin o anche in gradi Celsius: 
    \begin{gather*}
        T(C)e = T(K) - 273.15
    \end{gather*} 
    \item Mole:
    E' un modo per rappresentare un numero di oggetti utilizzando il numero
    di avogadro, molto utilizzato se si parla di particelle.
    \item Quantità di carica: si misura in ampere.
    \item Intensità luminosa: Rappresenta una certa intensità luminosa definita ad una certa distanza da una certa sorgente;
\end{enumerate}
Da queste grandezze (ci sono anche lunghezza, massa e tempo come detto prima) si definiscono poi 
le grandezze derivate in modo tale da non dover dare un altra definizione da zero.  

\subsection{Sistema CGS}
E' un sistema che utilizza centimetri, grammi e secondi.

\subsection{Sistema pratico}
Si scelgono come grandezze fondamentali lunghezza, tempo e forza.  \\
SI definisce allora la lunghezza ed il tempo come nei precedenti sistemi.\\
La definizione della forza è presa sfruttando la definizione di massa e quella di
forza peso arrivando a definirla come kilogrammo forza peso. Il peso di un oggetto può variare
al variare dell'altezza e quindi per definire il kilogrammo forza peso devo necessariamente affidarmi
ad una definizione di $g$ accetta come media. e quindi la massa 
diventa una grandezza derivata con unità di misura $kgf \frac{s^{2} }{m}$.

\chapter{L'analisi dell'errore in generale}
\section{Errori o incertezze}
Sebbene i termini "errore" e "incertezza" indichino concetti diversi, spesso per i fisici
la distinzione è irrilevante. L'\textbf{errore} sulla misura di una grandezza è la differenza
tra il valore della misura ed il valore vero (spesso non si riesce ad ottenerlo poiché non
si ottiene il valore reale) mentre l'\textbf{incertezza} di una misura è invece l'ipotesi 
sulla stima dell'errore.

\section{Inevitabilità dell'incertezza}
Nessuna grandezza può essere misurata con assoluta precisione: si può comunque operare
con cura per limitare tutti i possibili errori ed incertezze durante le misure anche
se eliminarle del tutto è virtualmente impossibile.

\section{Stima delle incertezze nella lettura su una scala graduata}
Supponiamo di avere un righello e di misurare una matita e risulta che essa
si possa trovare più vicina a $36mm$. In questo caso possiamo affermare:
\begin{eqnarray*}
    migliore \ stima \ della \ lunghezza = 36 mm \\
    intervallo \ probabile: \ da \ 35,5mm \ a \ 36,5 mm.
\end{eqnarray*}
La valutazione della posizione tra le incisioni su di una scala numerata è chiamata
\emph{interpolazione}.

\section{Stime delle incertezze nelle misure ripetibili}
E' possibile fare stime ragionevoli dell'incertezza in una serie di misure quando
queste misure hanno lo stesso numero di cifre significative: inoltre è sempre ragionevole
dire che il valore tra cui è compresa la misurazione è il minimo misurato ed il massimo
misurato durante questa serie di misure. \\
Tuttavia non sempre questo metodo di ottenere incertezze è completamente attendibile:
se ottengo due forze diverse di rottura per due fili che, per ipotesi, sono identici ma
ottengo valori diversi, allora è probabile che vi sia un errore di misura dovuto allo
strumento per misurare la forza di rottura o al fatto che magari i due cavi non sono
completamente identici. \\
Questo tipo di errori prende il nome di \textbf{errori sistematici} e possono essere
difficili da rilevare, inoltre affliggono tutte le misure.

\chapter{Rappresentare le incertezze}
\section{Migliore stima $\pm$ incertezza}
Generalmente una serie di misure può essere rappresentata nel seguente modo:
\begin{equation}
    l = l_m + \delta l
\end{equation}
Dove $l_m$ è la migliore stima di una misura e $\delta l$ l'\textbf{incertezza} o
\textbf{margine di errore} nella misura di $l$.

\section{Cifre significative}
Per indicare le incertezze esistono delle regole fondamentali.
\begin{enumerate}
    \item \textbf{Regola per scrivere le incertezze}: le incertezze sperimentali
    dovrebbero di regola essere arrotondate ad una cifra significativa. Incertezze come
    $\pm 0,02385$ non hanno senso e si arrotondano a $0,02$. Una velocità misurata come:
    $6051,78 \pm 30 m/s$ è ridicola ed è meglio esprimerla come $6050 \pm 30 m/s$ in quanto
    le cifre 1, 7 ed 8 sono totalmente irrilevanti rispetto all'incertezza.
    \item \textbf{Regola per scrivere i risultati}: l'ultima cifra significativa in qualunque
    risultato dovrebbe essere dello stesso ordine di grandezza dell'incertezza:
    $92,81 \pm 0,3$ dovrebbe essere arrotondata a $92,8 \pm 0,3$.
    \item \textbf{Regola dell'1}: se la prima cifra significativa è un uno, allora si conserva
    almeno un'altra cifra: $112 \pm 10$ diventa $110 \pm 10$ e non $100 \pm 10$, mentre
    $0.002 \pm 0.0001167$ diventa $ 0.002 \pm 0.00012$ e non $0.0001$ poiché c'è un uno davanti.
    \item \textbf{Regola dell'arrotondamento}: quando si arrotonda un numero , se si ha un 5
    si può arrotondare per eccesso se e solo se il numero prima è dispari:
    \begin{gather*}
        12.346 \to 12.3 \quad 
        12.350 \to 12.4 \quad
        12.250 \to 12.2 \\
        12.251 \to 12.3 \quad
        12.351 \to 12.4 \quad
        12.349 \to 12.3
    \end{gather*}
\end{enumerate}
Per ridurre le inaccuratezze dovute all'arrotondamento, \emph{tutti i numeri usati in
calcoli successivi dovrebbero tenere almeno una cifra significativa in più di
quanto richiesto nel risultato finale. Alla fine del calcolo, il risultato finale 
dovrebbe essere arrotondato per rimuovere queste cifre extra, non significative}.
Se inoltre si richiede l'uso della notazione scientifica, è bene esprimere valore
e incertezza con la stessa forma: 
\[
    (1,61 \pm 0,05) \cdot 10^{-19}
\]

\section{Discrepanza}
Prima di affrontare la questione di come utilizzare le incertezze, bisogna capire cosa è
la \textbf{discrepanza}, ossia la differenza tra due valori misurati della stessa grandezza.
La discrepanza può essere sia \emph{significativa} o \emph{non significativa}. Una discrepanza significativa
sta a dire che i margini di incertezza delle due misure non si sovrappongono, altrimenti è
non significativa. Spesso la discrepanza rispetto ad una misura già determinata con
un'alta accuratezza è anche chiamato \emph{errore vero}.

\section{Confronto di valori misurati e accettati}
Determinare un singolo valore misurato è completamente privo di interesse, infatti
cercare di misurare l'incertezza si ottiene solo dal confronto di almeno due valori.
Il tipo più semplice di esperimento è la misura di una grandezza di valore accettato noto.
Per esempio ottenere un valore misurato di $329 \pm 5 m/s$ rispetto ad un valore noto accettato
di $331 m/s$ è coerente, così come è coerente un valore misurato $325 \pm 5 m/s$ in quanto
questo valore ci dice che probabilmente il valore cade tra $330$ e $320 m/s$, e che è possibile
che possa cadere anche poco al di fuori.

\section{Confronto di due misure}
Molti esperimenti consistono nella misura di due valori, e spesso il loro confronto
è cruciale per poter determinare l'accuratezza delle misure:
presi due valori come $ p = p_m \pm \delta p$ e $q = q_m \pm \delta q$. Se volessimo
confrontarli, potremmo dire che l'incertezza in una differenza è data da:
\begin{equation}
    \delta x = \delta p + \delta q.
\end{equation}
Sarà poi definita diversamente.

\section{Grafici di relazioni tra grandezze}
Generalmente una serie di misure si può rappresentare graficamente come una funzione
continua basata su dati distanziati tra loro in modo discreto. Ogni punto di cui è stata
ottenuta una misura ha anche una \textbf{barra di errore} associata, ossia una barra che rappresenta
l'incertezza e l'intervallo in cui probabilmente si trova quella misura. La funzione
continua deve però passare per le barre di errore di ogni dato. Attraverso l'interpolazione
è possibile ottenere una funzione che possa meglio rappresentare le misure sul grafico.

\section{Incertezze relative}
La bontà di una misura dipende non solo dall'incertezza, ma anche dal suo rapporto
con ciò che è stato misurato questo rapporto è l'\textbf{incertezza relativa}:
\begin{equation}
    incertezza \ relativa = \frac{\delta x}{|x_m|}
\end{equation}
questo rapporto, anche chiamato \emph{precisione}, moltiplicando per cento si ottiene
l'\emph{incertezza percentuale}. E' importante sottolineare che l'incertezza relativa è
adimensionale ed è diversa dall'incertezza assoluta.

\section{Cifre significative e incertezze relative}
Il numero di cifre significative è strettamente collegato alla nozione di incertezza
relativa ed indica approssimativamente l'incertezza relativa di una misura.
Per uno scienziato sperimentale dire che $x = 21$ vuol dire che x può essere sia
$21 \pm 0,5$ (come per i matematici) oppure $21 \pm 1$ ma anche $21 \pm 5$. Dire quindi
che una misura ha solo due cifre significative è solo un'indicazione grossolana della
sua precisione. Generalmente se una misura ha N-cifre significative, allora la sua incertezza
è circa 1 sulla N-esima cifra:
\[
    x = 21 \qquad y = 0,21
\]
Le loro incertezze sono quindi:
\[
    x = 21 \pm 1 \qquad t = 0,21 \pm 0,01
\]
La loro incertezza relativa è del $5\%$:
\[
    \frac{\delta x}{x} = \frac{\delta y}{y} = \frac{1}{21} = \frac{0,01}{0,21} = 5\%
\]
Tuttavia questa relazione non è universale: in casi come: $t=99$, l'incertezza è dell'$1\%$,
mentre per $s = 10$ è del $10\%$. Questo ci porta a dire che questa correlazione 
incertezza-cifre significative è molto approssimativa.

\section{Prodotto tra due valori misurati}
L'aspetto più importante degli errori relativi emerge quando si iniziano a moltiplicare tra
loro valori numerici di misure, prendendo come esempio la quantità di moto:
\begin{eqnarray*}
    (m \ misurato) = m_m\left(1 \pm \frac{\delta m}{m_m}\right) \\
    (v \ misurato) = v_m\left(1 \pm \frac{\delta v}{v_m}\right)
\end{eqnarray*}
La migliore stima della quantità di moto è:
\[
    (migliore \ stima \ per \ p) = p_m = m_mv_m.
\]
E' quindi logico pensare che il massimo valore di $p_m$ è dato dal segno più delle
incertezze delle singole misure ed il valore minimo è dato da dal segno meno.
\[
    p = m_m v_m\left(1 \pm \frac{\delta m}{m_m}\right)\left(1 \pm \frac{\delta v}{v_m}\right)
\]
Si noti che svolgendo i conti comparirà un prodotto del tipo: $\frac{\delta m}{m_m}
\frac{\delta v}{v}$, questo prodotto, essendo molto piccolo, è quasi irrilevante ai fini
della precisione di p, per cui può essere eliminato. Il valore di p sarà quindi:
\[
    p = m_m v_m \left(1 \pm \left(\frac{\delta m}{m_m} + \frac{\delta v}{v_m}\right)\right)
\]
L'incertezza di p sarà quindi la somma delle incertezze di m e di v.
Anche questa regola è provvisoria e sarà sostituita con una più completa.

\section{Approssimazione delle funzioni}
Data una funzione possiamo calcolarne una approssimazione
attraverso il suo sviluppo di Taylor e per cui l'errore che si commette
nell'approssimazione di questa funzione è proprio i termini
che non ho considerato:
\begin{align}
    \epsilon = \left| \frac{R_N(x, x_0)}{f(x)} \right| 
\end{align}

\chapter{Strumenti e misure}
\section{Metodi per misure delle grandezze fisiche}
\begin{enumerate}
    \item Misura diretta: confrontare direttamente una grandezza fisica con un campione
    ed uno omogeneo preso come unità di misura.
    \item misura indiretta: viene effettuata quando la grandezza da misurare
    è una funzione di altre grandezze fisiche a essa non omogenea, ossia
    $\psi = f(x, y, z, \dots)$ misurando separatamente le grandezze tramite la relazione funzionale.
    \item misura con apparecchi tarati che viene effettuata con uno strumento apposito che fornisce 
    facilmente tramite un indicatore il valore della misura in questione senza bisogno di campioni.
\end{enumerate}

\section{Gli strumenti del laboratorio e le loro caratteristiche}
\subsection{Caratteristiche generali}
\begin{enumerate}
    \item Prontezza: rapidità con cui lo strumento si mette in equilibrio . Viene
    misurata dal tempo caratteristico dello strumento che rappresenta
    il tempo necessario affinché lo strumento si porti in una nuova situazione di 
    equilibrio dopo che la grandezza $\psi$ abbia subito una variazione $\Delta \psi$ in un intervallo di tempo $<< T_c$. 
    Uno strumento è tanto pronto quanto $T_c$ è piccolo. Per misurare correttamente variazioni successive
    che aumento in $\Delta T_i$ occorre uno strumento $T_c << \Delta T_i$. 
    \item Portata: E' la massima quantità di grandezza che può essere misurata con quello strumento.
    \item Soglia: la minima quantità di grandezza che può essere rivelata dallo strumento
    \item Intervallo di funzionamento: intervallo soglia-portata 
\end{enumerate} 

\subsection{La sensibilità}
Uno strumento di misura fornisce una risposta $R$ quando viene sollecitato dal valore della grandezza fisica che 
va a misurare. $R(y)$ è la curva della sensibilità dello strumento che si ricava misurando
le varie $R_i$ quando lo strumento è sollecitato dai valori noti $\psi_i$ . \\
Generalmente la taratura dello strumento è fornita ma è buona pratica rieseguirla spesso
per assicurarsi che i valori sulla curva non siano cambiati. \\
La sensibilità di uno strumento è data dal valore della pendenza
della curva di taratura nell'intorno del valore medio $R(\psi)$ della grandezza. Sensibilità 
media
\begin{gather*}
    <\sigma> = \frac{|R_2 - R_1}{|\psi_2 - \psi_1|}
\end{gather*}
dove $R_2$ e $R_1$ sono le risposte dello strumento considerato quando misura
rispettivamente i valori $\psi_2$ e $\psi_1$. La quantità $<\sigma>$ è definita nella ipotesi
che la sensibilità dello strumento rimanga costante
nell'intervallo $|\psi_2 - \psi_1|$. Se la relazione $R(\psi)$ non è lineare
allora $\sigma$ non sarà costante ma varierà da punto a punto della curva di taratura. 
Esiste allora un valore minimo $\Delta R_m$ che ogni strumento ha la capacità di 
indicare con un valore della sua risposta diverso dai valori contigui. Generalmente
$\Delta R_m$ è uguale all'errore di lettura, ossia il minimo valore della variazione della risposta.
\begin{gather*}
    \Delta \psi_M = \frac{\Delta R_m}{\sigma(\psi)}
\end{gather*}
rappresenta l'intorno dei valori di $\psi$ nel quale lo strumento fornisce sempre una
risposta e per tutti i valori compresi nell'intervallo lo strumento indica la stessa risposta 
$R(\psi)$. \\ 
$\Delta \psi_m$ prende il nome di errore di sensibilità. In alcuni strumenti
$R(\psi)$ è di tipo continuo mentre in altri di tipo discreto (digitale). Per gli strumenti digitali si ha solo
per dare l'errore di sensibilità. In alcuni strumenti esistono poi effetti che provocano
una variazione continua di $R(\psi)$. \\
Il risultato di tutte queste cose di natura aleatoria è che
in corrispondenza di un valore costante $\psi$ alterniamo per gli
$n$ valori di $R_i(\psi)$ una distribuzione attorno ad un valore medio 
$<R>$ con una larghezza $\delta R$ definibile tramite
lo scarto quadratico medio:
\begin{gather*}
    \delta R = \sqrt{\frac{\sum (R_i - <R>)^{2} }{n - 1}} 
\end{gather*}
a tale fluttuazione nella risposta corrisponderà un'incertezza
sulla misura di $\psi$ detta errore di riproducibilità ossia
$\delta \psi = \delta R / \sigma(\psi)$. Si chiama precisione
\begin{gather*}
    \frac{1}{\delta \psi}
\end{gather*}
E precisione relativa
\begin{gather*}
    \frac{\delta \psi}{<\psi>}
\end{gather*}
In uno strumento ben progettato il reciproco della precisione
e l'errore di sensibilità dovrebbero essere confrontabili 
l'uno con l'altro. Se l'errore di sensibilità è $<\delta \psi$ 
occorrerà ripetere varie volte la valutazione del valore della 
risposta $R$ in modo da ottenere una stima del suo valore medio. \\ 
Se l'errore di sensibilità $> \delta \psi$ non vengono usate le prestazioni
possibili dello strumento e risulta che l'errore di
sensibilità è la maggior causa nell'incertezza della determinazione
di $R(\psi)$ le varie caratteristiche di uno strumento non sono fra
di loro indipendenti. \\
L'errore di sensibilità di una misura è data dalla minima 
variazione della quantità da misurare che può essere rivelata
della misura scelta. L'errore di sensibilità di una misura sarà inferiore alla somma
degli errori di sensibilità dei singoli strumenti usati. 

\section{Gli strumenti} 
\subsection{Il nonio}
Un metodo per vedere l'errore di sensibilità di una scala, consiste nel tracciare
una graduazione ausiliaria  concorde con la principale sulla parte
scorrevole. L'indice fa da zero per il nonio e la lunghezza pari a $n$ tratti
della scala principale divide in $n + 1$ parti uguali. Una misura
si legge vedendo quale tacchetta coincide meglio fra nonio e scala principale
il valore $x = l / (n + 1)$, dove $l $ è la tacca che corrisponde. Ogni divisione del nonio
vale 
\begin{gather*}
    a' = a \frac{n}{n + 1} \Rightarrow l \cdot  a = x + l \cdot a' = x + l \cdot  a \frac{n}{n + 1}
\end{gather*}

\subsection{Calibro}
Il calibro ha errore di sensibilità di anche $0.01 mm$. Per misurare
spazi interni ed esterni e la vite micrometrica, costituita
da una vite, ha un passo breve di $0.5 mm$, l'esterno è fissato
ad un tamburo graduato che ha un diametro maggiore così da mettere 
in evidenza un piccolo spostamento lungo l'asse. La vite ha sempre un po' di gioco per questo è
consigliato ripetere sempre la misura da effettuare in modo tale che l'errore raggiungibile
di sensibilità raggiunge il $1 \mu m$.

\subsection{Palmer}
Tamburo solidale con vite micrometrica simile allo
però alla madrevite sono connessi 3 piedi i cui estremi individuano i vertici di un 
triangolo equilatero sui piedi vi è una scala verticale laminata dell'orlo affilato del disco
connesso alla vite micrometrica. La scala verticale ha una graduazione uguale al passo della
vite (di solito $0.5 mm$). Sul disco c'è una scala che
permette di determinare le frazioni di giro della vite
prendendo come riferimento il filo della scala.

\subsection{Lo sferometro}
Lo sferometro è uno strumento che si pone su di un piano e
si determina la lettura del disco quando la parte della vite
micrometrica è in contatto con il piano (usando la frizione) . 
LA stessa operazione va fatta sul corpo di 
cui si vuole misurare lo spessore il cui valore si ottiene per la loro
differenza. COn lo sferometro si piò misurare il raggio di 
curvatura di una calotta sferica. \\
Appoggiando lo strumento alla calotta si delimita una nuova calotta più
piccola ma con lo stesso raggio di curvatura $R$
e $h$. 
\begin{gather*}
    R = \sqrt{r^{2} + (R - h)^{2} } , \qquad R = \frac{r^{2} + h^{2} }{2h}
\end{gather*} 

\section{Gli errori degli strumenti}
Quando si misura una grandezza otteniamo un valore della misura
che è sempre afflitto da degli errori. Gli errori possono essere di due categorie
principali \textbf{sistematici} e \textbf{accidentali}.

\subsection{Sistematici}
Errori attesi che si ripercuotono nella misura sempre nello stesso
modo alterando il risultato stesso. Questi errori sono sempre presenti
ma si dividono in diversi tipi: 
\begin{enumerate}
    \item schematizzazione: dovuti a cause o fenomeni che vengono trascurati;
    \item Strumentali: legati allo strumento;
    \item Personali: sottostime o sovrastima, etc...
\end{enumerate}
Possono essere eliminati con l'esperienza

\subsection{Accidentali}
Si presentano in modo aleatori per cose varie ed indipendenti
e non si possono eliminare.


\chapter{Propagazione delle incertezze}
La maggior parte delle grandezze fisiche non si ottengono con misure dirette: la velocità
o l'accelerazione richiedono più misure e quindi bisogna non solo combinare la stima
di queste misure ma anche i loro errori.

\section{La valutazione dell'errore}
Avviene in due fasi distinte
\begin{enumerate}
    \item Stima a priori dell'errore
    \item Valutazione a posteriori dell'errore
\end{enumerate}
Quando si eseguono delle valutazioni allora dobbiamo tenere conto che a priori
si valutano le misure dirette ed i loro errori sistematici, mentre 
a posteriori le misure indirette ed i loro errori accidentali.

\section{Stima a priori degli errori in una misura diretta}
Essa è importante quando vogliamo organizzare una esperienza in laboratorio.
E' importante infatti pre-selezionare una strumentazione che permetta di mantenere
l'errore prestabilito di una data misura, ed è importante valutare
le fasi ell'esperimento più critiche in modo tale da dedicarvi più tempo.

\subsection{Errori sistematici}
Sappiamo che essi manterranno permanentemente le nostre informazioni per eccesso
o per difetto e dovremmo allora esaminare preliminarmente gli strumenti. 
Gli errori sistematici di norma si manifestano sottoforma di \textbf{offset}
e/o \textbf{parallasse}. L'offset è la misura che si ottiene da uno strumento
in condizioni di riposo (ad esempio una bilancia tarata male) oppure da uno strumento come
il compasso di Palmer che, data la natura dello strumento, da sempre una
lettura anche a riposo. L'errore sulla parallasse invece si ha quando si legge
una misura su di una scala graduata ad un angolo rispetto a stare perfettamente davanti
allo strumento (come nel caso di un termometro). Per potersi
accorgere di questi errori si confrontano con strumenti più performanti
oppure diversi.

\subsection{Errori accidentali}
La stima di questo tipo di errore dovrà essere sempre ricavata da un analisi
delle modalità con cui viene eseguita una misura. La maggior parte delle volte 
si può ricorrere all'errore di sensibilità, ma nel caso di un cronometro questo non è
vero in quanto l'essere umano introduce un errore relativo al suo
tempo di reazione. 

\section{Stima a priori degli errori di una misura indiretta e propagazione degli errori semplici}
Data una misura indiretta $g$ ottenuta da una qualche relazione funzionale
$g = f(x, y, z, \dots)$ come si può determinare l'errore su questa misura?
Si deve utilizzare la \textbf{formula di propagazione dell'errore}.

\subsection{Proporzionalità diretta $g = f(x) = kx$}
In questo caso la misura $g$ dipende dalla misura diretta
della misura $x$ e da quante volte è stata presa questa misura. \\
L'errore su $x_m$ è esattamente il suo intorno e dunque posso 
ricavare l'incertezza su $g$ come 
\begin{gather*}
    g_m + \Delta g = f(x_m + \Delta x) \Rightarrow \Delta g = f(x_m + \Delta x) - f(x_m)
\end{gather*}
Posso allora generalizzare l'errore sulla misura come
\begin{align}
    \Delta g = |k|\Delta x
\end{align}
E quindi posso determinare l'\textbf{errore relativo}, che non è altro che
la percentuale dell'errore sulla misura. 
\begin{align}
    \frac{\Delta g}{|g_m|} = \frac{\Delta x}{x_m}
\end{align}

\subsection{Proporzionalità quadratica $g = f(x) = x^{2}$}
Devo distinguere gli intervalli $\Delta g_-$ e $\Delta g_+$ 
e non sono distribuiti simmetricamente rispetto alla misura e quindi
\begin{gather*}
    \Delta g_+ = f(x_m + \Delta x) - f(x_m) = 2x_m\Delta x + \Delta x^{2} = 2x_m \Delta x 
\end{gather*}
L'altra
\begin{gather*}
    \Delta g_- = f(x_m - \Delta x) - f(x_m) = 2x_m\Delta x + \Delta x^{2} = 2x_m \Delta x 
\end{gather*}
Allora posso dire che l'incertezza sarà
\begin{align}
    \Delta g = |2x_m| \Delta x
\end{align}
E l'errore relativo
\begin{gather*}
    \frac{\Delta g}{|g_m|} = 2 \frac{\Delta x}{x_m}
\end{gather*}

\subsection{Proporzionalità $g = f(x) = x^{3}$} 
Analoga alla quadratica:
\begin{align}
    \Delta g = |3x_m^{2}| \Delta x
\end{align}
\begin{align}
    \frac{\Delta g}{|g_m|} = 3\frac{\Delta x}{|x_m|}
\end{align}

\subsection{Proporzionalità inversa $g = f(x) = \frac{1}{x}$}
Procediamo come per le altre (usando Taylor)
\begin{gather*}
    \Delta g_+ = \frac{1}{x_m + \Delta x} - \frac{1}{x_m} \approx \frac{1}{x}\left(1 - \frac{\Delta x}{x_m}\right) - \frac{1}{x_m} \approx - \frac{\Delta x}{x^{2} }
\end{gather*}
Lo stesso vale per l'altra
\begin{gather*}
    \Delta g_- = \dots = \frac{\Delta}{x^{2} }
\end{gather*}
\begin{align}
    \Delta g = \frac{\Delta x}{|x_m|^{2} }
\end{align}
\begin{align}
    \frac{\Delta g}{|g_m|} = \frac{\Delta x}{|x_m|}
\end{align}


\subsection{$g = f(x)  = \sin x$ e $g = f(x) = \cos x$}
\begin{gather*}
    \Delta g_+ = \sin(x_m + \Delta c) - \sin(x_m)
\end{gather*}
Usando allora Taylor e svolgendo la somma
\begin{gather*}
    \sin \Delta x \approx \Delta x \\
    \cos \Delta x \approx 1 
\end{gather*}
E allora diventa
\begin{gather*}
    \cos x_m \cdot  \Delta x
\end{gather*}
\begin{align}
    g = f(x)  = \sin x &\ \Rightarrow \ & \Delta g \approx |\cos x_m| \Delta x \\
    g = f(x)  = \cos x &\ \Rightarrow \ & \Delta g \approx |\sin x_m| \Delta x
\end{align}

\subsection{$g = f(x) = \ln x$}
Usando allora Taylor si ottiene 
\begin{gather*}
    \Delta g_+ = \ln\left(x_m \left(1 + \frac{\Delta x}{x_m}\right)\right) - \ln x_m = \frac{\Delta x}{x_m} = \Delta g_-
\end{gather*}
\begin{align}
    \Delta g = \frac{\Delta x}{|x_m|}
\end{align}
Dato che l'incertezza assoluta è uguale a quella relativa si può introdurre
la seguente formulazione che risulterà essere fondamentale per il caso generale della
propagazione degli errori.
\begin{align}
    \Delta ln x = \frac{\Delta x}{x _m}
\end{align}

\section{Il caso generale della propagazione dell'errore per una funzione che ci da una misura indiretta}
\begin{wrapfigure}{r}{0.50\textwidth} 
    \centering
    \label{Fig. 3.1}
    \caption{Rappresentazione grafica di un valore dato da una certa
    funzione q(x)}
    \includegraphics[width=0.50\textwidth]{funzione-errore.png}
\end{wrapfigure}

In generale, presa una qualunque funzione $q(x)$, il suo valore più accurato
sarà rappresentato da $q_m = q(x_{m})$. L'incertezza calcolabile diventa quindi:
\[
    \delta q = q(x_m + \delta x) - q(x_m).
\] 
Un'approssimazione asserisce che, per qualunque funzione, e qualunque incremento 
piccolo $u$, si ottiene:
\[
    q(x + u) - q(x) \approx \frac{dq}{dx}u
\]
Se $\delta x$ è molto piccola, si può dire che (preceduta dal segno meno se q(x) è
decrescente):
\[
    \delta q = \frac{dq}{dx}\delta x
\] 
Per una qualunque funzione diventa quindi:
\begin{equation}
    \delta q = \left|\frac{dq}{dx}\right| \delta x
\end{equation}
L'incertezza per una potenza diventa quindi ($q(x) = x^n$):
\[
    \delta q = \left|\frac{dq}{dx}\right|\delta x = |nx^{n-1}|\delta x
\]
dividendo per $|q| = |x^n|$ si ottiene:
\[
    \frac{\delta q}{|q|} = |n|\frac{\delta x}{|x|}
\]

\section{Caso di una funzione a più variabili}
La relazione funzionale con più di una variabile diventa più complesso
del caso ad una variabile sola: si deve infatti considerare il contributo all'errore
di tutte le variabili. Procediamo per casi prima di arrivare alla formulazione
generale

\subsection{Somma di due variabili o più variabili}
\begin{gather*}
    g = x + y
\end{gather*}
\begin{gather*}
    x_m \pm \Delta x 
    y_m \pm \Delta y
\end{gather*}
L'incertezza della misura sarà allora 
\begin{gather*}
    \Delta g_+ = \Delta x + \Delta y \\
    \Delta g_- = - \Delta x - \Delta y
\end{gather*}
Allora
\begin{align}
    \Delta g = \Delta x + \Delta y
\end{align}
Scegliendo queste incertezze mi sto mettendo nel caso peggiore 
e sto assumendo di aver sbagliato sia $x$ che $y$ di tutta la loro
incertezza. C'è anche la possibilità di ridurre però l'errore
con la formulazione generale.

\subsection{Prodotto di due o più variabili}
\begin{gather*}
    g = xy
\end{gather*}
Il ragionamento è lo stesso della somma e posso allora
scrivere
\begin{gather*}
    x_m \pm \Delta x = x_m \left(1 \pm \frac{\Delta x}{x_m}\right) \\
    y_m \pm \Delta y = y_m \left(1 + \frac{\Delta y}{y_m}\right)
\end{gather*}
Con ovviamente l'ipotesi che $\Delta x, \Delta y << x_m, y_m$ 
E quindi 
\begin{gather*}
    \Delta g_+ = x_m \left(1 \pm \frac{\Delta x}{x_m}\right)y_m \left(1 + \frac{\Delta y}{y_m}\right) = x_m y_m \left(\frac{\Delta x}{x_m} + \frac{\Delta y}{y_m}\right)
\end{gather*}
L'errore relativo diventerà allora
\begin{align}
    \frac{\Delta g}{|g_m|} = \frac{\Delta x}{|x_m|} + \frac{\Delta y}{|y_m|}
\end{align}
Che è lo stesso per il quoziente di funzioni

\section{Formula di propagazione dell'errore}
Avendo una grandezza $g$ ottenuta dalla relazione funzionale $f$ di più grandezze 
$x, y , \dots$, per una grandezza di questo tipo posso dire che la derivata di questa
funzione non è altro che la somma delle derivate parziali rispetto a tutte le variabili
e si può allora dimostrare che (sostituendo i differenziali con le grandezze)
\begin{align}
    \Delta g \approx \left| \frac{\partial g}{\partial x}  \right|_{x = x_m, y = y_m, \dots} \Delta x +  \left| \frac{\partial g}{\partial y}  \right|_{x = x_m, y = y_m, \dots} \Delta y + \dots = \sum \left| \frac{\partial g}{\partial x_i}  \right|_{x_i = x_{im}} \Delta x_I
\end{align}
Anche in questo caso la formula generale ci da una stima molto conservativa dell'errore
e si può dimostrare che è valida la relazione 
\begin{align}
    \Delta g = \sqrt{\sum \left(\frac{\partial f}{\partial x_i} \Delta x_{im } \right)^{2} } 
\end{align}
La formula di propagazione dell'errore è facile se ci sono solo prodotti, nei casi più complessi ci si deve rifare
alla risoluzione con le derivate logaritmiche

\section{La risoluzione con i logaritmi}
Ricordando che
\begin{gather*}
    d \ln g = \frac{dg}{g}
\end{gather*}
Possiamo allora utilizzare il differenziale per ricavare l'errore relativo
attraverso il logaritmo da entrambi i lati dell'uguaglianza:
\begin{gather*}
    \ln g = \ln(\dots)
\end{gather*}
con le proprietà dei logaritmi si possono cercare di isolare le variabili 
e poi si deriva da entrambi i lati per ottenere l'errore relativo. Una cosa 
importante è ricordare che il passaggio a valori finiti va fatto mettendo
i valori assoluti ai coefficienti dei differenziali delle grandezze.


\chapter{Introduzione all'analisi statistica delle incertezze casuali}
\section{Valutazioni a posteriori degli errori: deviazione standard e media}
Il modo in cui si compiono le valutazioni a posteriori dipende da
come sono state effettuate le misurazioni in laboratorio. Nel caso di una sola misura
di grandezze si valuta l'incertezza con la stima a priori, ma questo è un caso limite in quanto 
solitamente si possono ripetere le misure.
Supponendo di voler misurare una grandezza $x$ e di aver identificato tutte le fonti 
di errore sistematico e di averle ridotte a livello trascurabile, poiché tutte le altri
fonti di errore sono casuali, dovremmo riuscire ad identificarle facilmente.
Supponendo di fare $N$ misure di una grandezza $x$, la migliore stima 
per la misura di $x$ è data da:
\begin{align}
    \bar{x} = \frac{\sum_{i = 1}^{N}x_i}{N}.
\end{align}
La \textbf{deviazione standard} o \textbf{scarto quadratico medio} invece consente di ottenere una stima
dell'incertezza delle $N$ misure ottenute e si ottiene nella seguente maniera.
Per ogni $x_i$ calcoliamo lo \textbf{scarto} di $x_i$ rispetto a $\bar{x}$:
$x_i - \bar{x} = d_i$  che ci dice quanto ogni misura differisce dalla media.
Se lo scarto è alto, allora la misura è stata poco precisa, altrimenti molto
precisa.\\
Elevando ora al quadrato ogni scarto si ottiene la \emph{deviazione standard} e
mediando queste misure ottenute non si otterrà più zero (come si otterrebbe
invece facendo la media degli scarti):
\begin{align}
    \sigma_x = \sqrt{\frac{1}{N}\sum_{i = 0}^{N}(d_i)^2} = \sqrt{\frac{1}{N}\sum_{i = 0}^{N}(x_i - \bar{x})^2}
\end{align} 
La deviazione standard è quindi lo \emph{scarto quadratico medio} delle misure.
La quantità $\sigma^2$ è chiamata invece \textbf{varianza}. \\
Si può ottenere una correzione della deviazione standard attraverso la sostituzione
di $N$ con $N-1$: infatti, nel caso assurdo in cui si abbia solo una misura, con $N$
avremmo che la deviazione standard sia zero mentre con $N - 1$si arriva ad una forma indeterminata
di $0/0$ che è totalmente in linea con la nostra ignoranza nel non conoscere
se vi è incertezza (poiché si ha solo una misura). Questa sostituzione inoltre riduce la tendenza
a sottostimare l'errore. La \textbf{deviazione standard} diventa quindi:
\begin{align}
    \sigma_x = \sqrt{\frac{1}{N-1}\sum_{i = 0}^{N}(d_i)^2} = \sqrt{\frac{1}{N-1}\sum_{i = 0}^{N}(x_i - \bar{x})^2}
\end{align} 
In ogni caso, la prima definizione prende il nome di \emph{deviazione standard
della popolazione} e la seconda \emph{deviazione standard del campione}. \\
Generalmente si può ottenere l'errore in modo semplice semplicemente considerando
lo \textbf{scarto massimo} su un campione di poche misure (meno di dieci)
anche se esistono alcune eccezioni: il caso di misure tutte uguali o il caso in cui lo scarto massimo è
minore dell'errore di sensibilità: in questi casi si sceglie l'errore di sensibilità come
errore per la grandezza. Nel caso di più di dieci misure invece si utilizzerà la deviazione
standard come migliore stima dell'incertezza su di una misura.

\section{Valori anomali e confronti tra misure}
Durante i calcoli può capitare di incontrare dei valori che si potrebbero
definire \textbf{anomali} in quanto molto diversi da altri valori. In queste
situazioni è consigliabile ripetere la misura se si è ancora in laboratorio piuttosto 
che scartarla (una nuova scoperta è sempre dietro l'angolo!). Si può anche aumentare
il numero di misure di un campione in modo tale che risulti un evento particolarmente sfortunato. \\
E' possibile confrontare le proprie misure con altre per vedere se \textbf{sono in accordo}
attraverso due metodi:
\begin{enumerate}
    \item \textbf{Confronto con valore di riferimento}: Compiuta una certa misura e ottenuto
    come risultato un certo $\overline{x}_1 \pm \Delta x_1$, se dobbiamo confrontarla con un
    valore di riferimento $\hat{x}$ dobbiamo allora verificare che si trovi almeno entro
    l'intervallo dell'errore oppure vedere se graficamente risultano compatibili
    tra di loro. Generalmente anche se una misura sembra cadere poco al di fuori della banda
    di errore di riferimento non è un segno che sia una misura anomala. In seguito si vedrà che
    esiste una sorta di limite di confidenza entro al quale questo può accadere. 
    \item \textbf{Confronto tra più valori diversi}: Posso utilizzare questo approccio anche
    per relazionare valori diversi tra di loro. Nel caso in cui le misure siano compatibili
    tra di loro si può utilizzare la media pesata per trovare la misura
    matematica corretta. Quando ciò non è possibile allora si può semplicemente scegliere il valore più preciso 
    tra quelli analizzati. 
\end{enumerate}


\section{La deviazione standard nella media}
Se $x_1, \dots, x_n$ sono i risultati di $N$ misure della grandezza $x$, allora come
abbiamo visto, si dice che la migliore stima di questa grandezza sia proprio $\bar{x}$.
La Dev. Standard invece misura le incertezze sulle singole misure, la deviazione standard
della media invece si chiama \textbf{SDOM} e si indica con $\sigma_{\bar{x}}$:
\begin{align}
    \sigma_{\bar{x}} = \frac{\sigma_x}{\sqrt{N}}
\end{align}
Così il risultato finale diventa:
\begin{align*}
    x = x_{best} \pm \sigma_{\bar{x}}
\end{align*}
Questo ci garantisce che l'incertezza possa diminuire sensibilmente (anche solo di
un fattore $\sqrt{N}$ rispetto alle N misure extra) all'aumentare delle misure compiute
anche se questo non ci garantisce l'eliminazione degli errori sistematici, che rimangono
all'interno delle misure anche all'aumentare di $N$.

\section{Errori sistematici}
Se non è chiaro l'errore sistematico che hanno gli strumenti, si possono fare
le seguenti assunzioni per la somma degli errori: se sappiamo che l'errore non è 
oltre l'$1\%$ con buona confidenza, allora si può ottenere la somma diretta, altrimenti
se è migliore di questo valore ed il $68\%$ delle bilance mostra questo valore
migliore allora si può utilizzare con una confidenza del $68\%$ la somma in quadratura:
\begin{align}
    \delta x_{tot} = \sqrt{(\delta x_{cas})^2 + (\delta x_{sis})^2}
\end{align}


\chapter{La distribuzione normale per studiare la distribuzione delle misure di una certa grandezza}
\section{Media pesata, frequenza ed istogramma}
\subsection{L'istogramma}
Per stimare il valore di una certa grandezza si può utilizzare la media aritmetica,
tuttavia esiste anche un metodo più veloce di fare la media ossia la \textbf{media pesata}:
\begin{align}
    \bar{x} = \frac{\sum_{i = 1}^{k}x_kn_k}{N}
\end{align}
Dove il "peso" $n_k$ non è altro che il numero di occorrenze di una certa misura nel nostro set di dati.
Possiamo allora considerare la \textbf{frequenza} con cui appare una certa misura
nel nostro set di dati e la posso definire come
\begin{align}
    F_k = \frac{n_k}{N}
\end{align}
Così si può scrivere il valore vero di $x$ come:
\begin{align}
    \bar{x} = \sum_{i = 1}^{k}x_kF_k
\end{align}
Implicando che:
\begin{align}
    \sum_{i = 1}^{k}F_k = 1
\end{align}
Poiché tutte le frazioni $F_k$ ottenute si ottiene 1. Qualunque insieme di numeri
la cui somma è $1$ è chiamato \textbf{normalizzato} e la precedente prende il nome
di \textbf{condizione di normalizzazione}.
Un'\textbf{istogramma a barre} è un istogramma con delle barre che
rappresentano la distribuzione dei valori è attraverso dalle altezze delle stesse. Tuttavia
questo istogramma ha un problema in quanto spesso le grandezze non sono numeri
interi ma variano in maniera continua all'interno di un intervallo.
Queste considerazioni implicano che diversi grafici possono avere la stessa media, allora
dobbiamo introdurre qualche parametro che ci permetta di differenziare due grafici
diversi anche se hanno la stessa media: un parametro importante è la precisione delle misure. 
Si potrebbe allora calcolare l'errore attraverso la deviazione standard,
la quale, purtroppo, è computazionalmente scomoda.

\subsection{Istogramma a intervalli o classi}
Dato che gli istogrammi a barre non sono ancora soddisfacenti in quanto ogni misura potrebbe
avere molte cifre decimali per cui ci sarebbero molte barre da un solo elemento. Possiamo allora
costruire degli intervalli in modo tale che siano proporzionali al numero di misure
contenute: un'\textbf{istogramma ad intervalli} è un istogramma a barre 
che racchiude (rappresenta la frazione dentro ogni barra) un certo intervallo di misure: l'area di ciascuna barra
 di misure contenute in ciascun
intervallo di misure.
La larghezza dell'intervallo $k$-esimo si chiama \textbf{bin}, e possiamo chiamare 
$f_k$ come la \textbf{densità di frequenza} e utilizzare questa per rappresentare l'istogramma
\begin{align}
    f_k = \frac{F_k}{\Delta BIN}
\end{align}
L'analisi dimensionale allora ci porta a dir che $f_k$ ha la dimensione
inversa della misura per cui lo stiamo calcolando. Con questa
formulazione, la frequenza di una certa misura $F_k$ rappresenta
l'area di una certa barra mentre $\Delta BIN$ rappresenta la classe o intervallo
del nostro grafico. I bin tuttavia devono essere scelti con un certo giudizio: infatti
bin troppo grandi possono raggruppare troppe misure in un unico intervallo "falsando"
la frequenza dei valori mentre bin troppo piccoli tendono ad avere l'effetto opposto
spaziando troppo le misure e risultando in istogrammi spezzati che rendono inutile
l'applicazione del metodo.

\section{Distribuzione limite}
Nella maggior parte degli esperimenti al crescere dei dati $N$ il grafico a barre
inizia ad assumere una forma ben precisa: quando $N \to \infty$ il grafico prende
il nome di \textbf{distribuzione limite} o \textbf{asintotica} proprio perché
non può essere raggiunta. Utilizzando le derivate si ottiene che la 
frazione di dati in un piccolo intervallo è data da:
\begin{align*}
    f(x)dx
\end{align*}
\begin{center}
    \definecolor{grey}{rgb}{0.5, 0.5, 0.5}
    \begin{tikzpicture}
        \draw[->] (0,0) -- (4, 0) node[midway, below] {$x, x+dx$};
        \draw[-, very thick] (2, 0.75) edge[out=180,in=0,-] (0,0);
        \draw[-, very thick] (2, 0.75) edge[out=0,in=180,-] (4,0);
        \draw[<->] (2.2, 0.6) -- (2.2, 0.1) node[midway, left] {$f(x)$};
        \draw[-] (2.3, 0.73) -- (2.3, 0.0);
        \draw[-] (2.4, 0.68) -- (2.4, 0.0);
        \fill[grey](2.3, 0.73) rectangle (2.4, 0);
    \end{tikzpicture}
\end{center}
di conseguenza sotto una distribuzione limite, l'area della funzione si calcola
proprio utilizzando l'integrale da un punto a ad un punto b della funzione per
ottenere la frazione di misure che cadono tra $a$ e $b$:
\begin{align}
    F_k = \lim_{\Delta bin \to 0} f_k \Delta bin = \int_{a}^{b} f(x)dx 
\end{align}
La funzione $f(x)$ diventa quindi la \textbf{funzione densità di probabilità}.
Di conseguenza:
\begin{align}
    \int_{-\infty}^{+\infty}f(x)dx = 1
\end{align}
Posso allora definire se 
\begin{gather*}
    \bar{x} = \sum x_k F_k = \sum x_k f_k \Delta bin 
\end{gather*}
Allora posso dire che
\begin{gather*}
    \bar{x} = \int_{-\infty}^{+\infty} xf(x) \ dx 
\end{gather*}
Per la deviazione standard posso applicare lo stesso ragionamento ed ottenere allora
\begin{gather*}
    \sigma_x^{2} = \frac{\sum (x_i - \bar{x} )^{2} }{N} 
\end{gather*}
Questa non è altro che la stessa deviazione standard
ma con i pesi associati e dunque, si ottiene che
\begin{gather*}
    \sigma_x^{2} = \sum(x_i - \bar{x} ) ^{2} F_k \Rightarrow \sum (x_i - \bar{x} )^{2}f_k \Delta bin \Rightarrow \int_{-\infty}^{+\infty} (x_i - \bar{x} )^2 f(x) \ dx 
\end{gather*}

\section{La distribuzione normale}
\begin{wrapfigure}{r}{0.4\textwidth}
    \label{Fig. 5.2}
    \caption{Istogramma}
    \begin{tikzpicture}
        \draw[->] (0, 0) -- (4, 0) node[at end, right] {$x$};
        \draw[->] (0, 0) -- (0, 2) node[at end, above] {$f(x)$};
        \draw[-, very thick] (2, 2) edge[out=180,in=0,-] (1,0);
        \draw[-, very thick] (2, 2) edge[out=0,in=180,-] (3,0);
        \draw[-, dashed] (2,2) -- (2, 0) node[at end, below] {valore vero di $x$};
        \draw[<->] (2.05, 1.5) -- (2.4, 1.5) node[at end, right] {$\sigma$}; 
    \end{tikzpicture}
\end{wrapfigure}
La distribuzione qui in Figura \ref{Fig. 5.2} è una curva a campana o distribuzione
limite dei valori trovati di $x$. Al centro c'è il valore probabile di $x$ (anche chiamato
valore vero), tuttavia diverse misure possono avere delle campane più o meno ripide a
seconda della deviazione standard: più cresce $\sigma$ più la campana risulta schiacciata.
Tuttavia un errore sistematico può spingere la mia campana verso una direzione definita
e così il centro della distribuzione sarà spostato rispetto al valore vero atteso. \\
Cosa è dunque il \emph{valore vero}? Si può supporre che il valore vero di una misura
sia il valore attorno al quale si concentra la distribuzione del grafico a campana
dopo un grande numero di misure. La funzione che descrive questa tendenza è anche
chiamata \textbf{funzione di Gauss} o \textbf{Distribuzione normale} ed il suo
prototipo è dato da:
\begin{align}
    f(x) = e^{-\frac{x^2}{2\sigma^2}}
\end{align}
Dove $\sigma$ è il parametro di larghezza (ossia la deviazione standard). 
La funzione di Gauss è una curva a campana centrata su zero ma può essere centrata attorno
al valore vero $X$ (anche chiamato $X$ vero o migliore stima) semplicemente sostituendo $x - X$ al posto di $x$.
Possiamo normalizzare la funzione in modo tale che il suo integrale mi dia $1$ nella seguente maniera:
\begin{align*}
    f(x) &= Ne^{-\frac{(x-X)^2}{2\sigma^2}} \\
    \int_{-\infty}^{+\infty} f(x)dx &= \int_{-\infty}^{+\infty} Ne^{-\frac{x-X}{2\sigma^2}}\\
\end{align*}
Per calcolare questo integrale si può sostituire $x-X = y$ e quindi $dx = dy$ e si ottiene:
\begin{align*}
    N\int_{-\infty}^{+\infty}e^{-\frac{y^2}{2\sigma^2}}dy
\end{align*}
E quindi porre $y/\sigma = z$ e quindi $dy = \sigma dz$ ed ottenere:
\begin{align*}
    N\sigma\int_{-\infty}^{+\infty}e^{-\frac{z^2}{2}}dz
\end{align*} 
Abbiamo allora ottenuto l'integrale di Gauss, il cui risultato è proprio $\sqrt{2\pi}$: 
\begin{align*}
    \int_{-\infty}^{+\infty} f(x)dx = N\sigma \sqrt{2\pi} = 1
\end{align*}
Infine dal momento che deve essere uguale ad uno, il fattore di normalizzazione
sarà:
\begin{align*}
    N = \frac{1}{\sigma \sqrt{2 \pi}}
\end{align*}
La distribuzione di Gauss sarà quindi nella sua forma finale:
\begin{align}
    G_{X, \sigma}(x) = \frac{1}{\sigma \sqrt{2\pi}}e^{-\frac{(x-X)^2}{2\sigma^2}}
\end{align}
Come abbiamo visto, il valore medio si ottiene attraverso l'integrale 
$\bar{x} = \int_{-\infty}^{+\infty}xf(x)dx.$ per cui per la distribuzione di Gauss si ha che:
\begin{align}
    \bar{x} = \int_{-\infty}^{+\infty}xG_{X, \sigma}(x)dx
\end{align}
Se le misure di una certa grandezza $x$ sono distribuite normalmente allora si ha che la migliore
stima della grandezza $X$ è proprio la media aritmetica delle misure. Si può dimostrare come segue:
\begin{align*}
    \bar{x} = \int_{-\infty}^{+\infty}xG_{X, \sigma}(x) \ dx =  \frac{1}{\sigma \sqrt{2\pi}}\int_{-\infty}^{+\infty} xe^{-\frac{(x-X)^2}{\sigma^2}}dx
\end{align*}
Cambiando variabile $y = x - X$ allora $dx = dy$ e $x = y + X$ l'integrale diventa:
\begin{align*}
    \bar{x} = \frac{1}{\sigma \sqrt{2\pi}} \left(
        \int_{-\infty}^{+\infty}ye^{-\frac{y^2}{2\sigma^2}}dy
        + X\int_{-\infty}^{+\infty} e^{-\frac{y^2}{2\sigma^2}}dy
    \right)
\end{align*}
Il primo integrale è zero poiché il contributo di ogni punto $y$ è cancellato
da quello del punto $-y$. Il secondo è l'integrale di normalizzazione e quindi si semplifica 
con la costante prima della parentesi dando come risultato:
\begin{align*}
    \bar{x} = X.
\end{align*}
Questo risultato è vero però se e solo se si eseguono moltissime prove e se
le misure che ho ottenuto sono distribuite normalmente; date queste condizioni
il valore vero attorno al quale è centrata la Gaussiana si avvicinerà sempre di più a $X$.
Interessante è trovare la deviazione standard ($\sigma$) della Gaussiana che si trova come segue:
\begin{align*}
    \sigma_x^2 = \int_{-\infty}^{+\infty}(x- \bar{x})^2f(x)dx
\end{align*}
Sostituendo $\bar{x} = X$ e ponendo $x - X = y$ ed integrando per parti
si ottiene il risultato:
\begin{align*}
    \sigma_x^2 = \sigma^2
\end{align*}
Ovviamente ha senso che sia la corretta deviazione standard solo dopo un numero infinito
di prove.

\section{La deviazione standard come limite di confidenza del 68\%}
Eseguendo lo studio di funzione sulla Gaussiana si ottiene che
la sua derivata prima si azzera proprio nel valore vero di $x$,
la sua derivata seconda inoltre, si azzera nei punti
$x \pm \sigma$. Se si volesse conoscere invece in corrispondenza
di quali valori si trovano i punti $\frac{1}{2}p(x)$, questi si trovano esattamente
nei punti $x \pm 2\sigma \sqrt{2 \ln 2}$. \\
Possiamo quindi calcolare la probabilità che un certo dato si
trovi entro $t$ deviazioni standard rispetto alla migliore
stima del valore $x$.
\begin{gather*}
    P(x - t \sigma, x + t \sigma) = \frac{1}{\sqrt{2\pi} \sigma}\int_{x - t\sigma}^{x + t\sigma} \exp\left(-\frac{(x - X)^{2} }{2\sigma^{2} }\right) \ dx = erf(t)
\end{gather*}
Si ottiene la funzione errore di $t$ e che assume valori a seconda
del valore di $t$. Secondo le tabelle già calcolate di probabilità
si ha che il $68\%$ delle misure ricade entro una deviazione
standard dal valore vero e ben il $95.45\%$ entro due deviazioni standard;
tuttavia scegliere valori molto alti per $t$ non ha senso in quanto 
già $3\sigma$ esprimono una probabilità del $99.73\%$. 
Dati questi risultati possiamo allora prendere
$\sigma$ come limite di confidenza della nostra Gaussiana in quanto
entro un $\sigma$ dal centro della curva si ha il $68\%$ dell'area
della curva, ossia i due terzi dei dati sperimentali. Posso anche dire che,
preso un certo elemento $x_i$ delle mie misure, esso ha il
$68\%$ di possibilità di rientrare entro una deviazione standard
rispetto al valore vero della curva. 

\section{Giustificazione della media come migliore stima di $X$ e di $\sigma$ come migliore stima per l'incertezza}
Partendo dai dati e ricavandoci i parametri della gaussiana, e supponendo che $X$ e $\sigma$ non sono noti,
tutta questa equazione dipende proprio dal
loro valore, per cui dalle letture $x_1, \dots, x_N$ 
le cui probabilità di essere ottenute sono date proprio da:
\begin{gather*}
    P(x_1) \ dx \Rightarrow P(x_1, x_1 + dx) = p(x_1) dx
\end{gather*}
Si può facilmente trovare con la formula della gaussiana che
la probabilità di trovare un valore tra $x_i$ e $x_i + dx$ non è altro che
trovare la probabilità tra $x_i, x_i$ (dato la continuità della funzione)
che tuttavia non posso calcolare in quanto non conosco né $X$ né $\sigma$. 
Si può dimostrare intanto che $P_{TOT} = P_1 \cdot \dots \cdot  P_n$
ossia
\begin{align}
    P_{TOT} = \left(\frac{dx}{\sqrt{2\pi} \sigma}\right)^{n} \prod \exp\left(-\frac{(x_i - X)^{2} }{2\sigma ^{2} }\right) 
\end{align}
Mi chiedo infatti quali siano i valori più plausibili per
$X$ e $\sigma$ che giustifichino la forma della Gaussiana. \\
Per fare ciò mi devo ricondurre al principio di Massima 
verosimiglianza. Dati gli $N$ valori misurati, le migliori stime
per $\sigma$ e $X$ sono quei valori per i quali le misure
assumano la massima probabilità: possiamo ragionare sui teoremi delle funzioni
continue e pensare ai valori e calcolare quando si azzera la derivata
\begin{gather*}
    \frac{\partial P_{TOT}}{\partial X} = \left(\frac{dx}{\sqrt{2\pi} \sigma}\right)^{n} \frac{\partial }{\partial X}  \exp\left(-\frac{\sum x_i - X }{2\sigma ^{2} }\right)
\end{gather*}
Risolvendo la derivata, per far si che si azzeri bisogna che
\begin{gather*}
    \sum (x_i - X) = 0 \Rightarrow X = \frac{\sum x_i}{N}
\end{gather*}
Potrei anche vedere se dopo $X$ la funzione decresca (si) e che prima
cresca (si) allora ho trovato il valore giusto per $X$. \\
Consideriamo la stessa procedura per $\sigma$
\begin{gather*}
        \frac{\partial P_{TOT}}{\partial \sigma} = \left(\frac{dx}{\sqrt{2\pi} \sigma}\right)^{n} e^{()}\frac{1}{\sigma^{N + 1} }\left(-N + \frac{\sum (x_i - X)^{2} }{\sigma^{2} }\right) = 0
\end{gather*}
Si ottiene allora il seguente caso: dato che $\bar{x}$ è un valore che io
costruisco e che, per definizione, risiede molto vicino al valore
vero $X$, io so che
\begin{gather*}
    \sum (x_i - \bar{x} )^{2} \leq \sum (x_i - X)^{2}
\end{gather*} 
Perché, non avendo infinite misure, io so che la distanza tra i valori $x_i$ e $\bar{x}$
è una sottostima rispetto agli scarti quadratici rispetto al valore vero $X$ anche se la media
$\overline{x}$ risiede molto vicino al valore vero. Posso allora trovare la deviazione
standard utilizzando la migliore stima per $X$, ossia la media: 
\begin{gather*}
    \sigma = \sqrt{\frac{\sum (x_i - \overline{x})^{2} }{N}}
\end{gather*}
In questo modo sto sottostimando il valore vero e otterrò un valore
sempre minore di quello effettivo: per aggiustarlo basta sottostimare
anche il denominatore con la cosiddetta \textbf{correzione di Bessel},
la quale è valida perché più è grande il numero di misure e più la differenza tra le
due è minore formulazioni è minore. In questo modo introduco un grado di libertà al denominatore 
per cui ho bisogno di almeno due misure per poter determinare l'incertezza su $\overline{x}$
\begin{align}
    \sigma = \sqrt{\frac{\sum (x_i - \overline{x})^{2} }{N - 1}} 
\end{align}

\section{Giustificazione della somma in quadratura}
Se consideriamo una misura del tipo $g  = f(x, y, z, \dots)$ possiamo
calcolare come
\begin{gather*}
    \Delta g = \left| \frac{\partial f}{\partial x}  \right|\Delta x + \left| \frac{\partial f}{\partial y}  \right| \Delta y + \dots  
\end{gather*}
Ma se l'errore fosse di natura accidentale la migliore
stima di questo valore diventa la somma in quadratura
e dunque
\begin{gather*}
    \Delta g \approx \sqrt{\sum\left(\frac{\partial f}{\partial x_i} \right)^{2} \Delta x_i} 
\end{gather*}
La dimostrazione di questa si articola in quattro punti: \\
I) Prendiamo un caso in cui $g = x + A$ dove $x$ presenta
una certa incertezza e $A$ è invece un numero puro
allora anche $g$ si può scrivere come una gaussiana traslata
e con lo stesso $\sigma$. Possiamo anche verificare che
$\sigma_x = \sigma_g$.
\begin{gather*}
    P(x, x + d x) = \frac{1}{\sqrt{2\pi}\sigma_x} \exp\left(-\frac{(x - X)^{2} }{2\sigma_x^{2} }\right)\\
    P(g) = P(g, g + dg) = \frac{1}{\sqrt{2\pi}\sigma_x} \exp\left(-\frac{(g - (x + A))^{2} }{2\sigma_x^{2} }\right) dg
\end{gather*}
Mi aspetto che $G = X + A$, il che è vero, ed è dunque verificata. \\
II) Se invece prendessimo $g = Bx$ allora per lo stesso procedimento mi aspetto
di ottenere
\begin{gather*}
    P(g) = P(g, g + dg) = \frac{1}{\sqrt{2\pi}\sigma_x}\exp\left(-\frac{\left(\frac{g}{B} - X\right)^{2} }{2\sigma_x^{2} }\right) \frac{dg}{B} = \\
    \frac{1}{\sqrt{2\pi}\sigma_xB}\exp\left(-\frac{\left(g - Bx\right)^{2} }{2\sigma_x^{2}B^{2} }\right) dg
\end{gather*}
Si vede subito che $G = Bx$ traslata e il parametro di larghezza è deformato: $\sigma_g = \sigma_xB$  e quindi
fare misure indirette ha ripercussioni anche sull'incertezza. \\
III) Prendiamo il caso in cui la relazione sia $z = x + y$, le misure $x$ e $y$ saranno
normalmente distribuite intorno al loro valore vero $X$ e $Y$ e avranno
allora larghezza $\sigma_x$ e $\sigma_y$. Le probabilità di ottenere
un particolare valore di $x_i$ e $y_i$ è
\begin{gather*}
    P_x(x_i) = P(x, x + dx) = \frac{1}{\sqrt{2\pi}\sigma_x}\exp\left(-\frac{(x_i - X)^{2} }{2\sigma_x^{2} }\right)dx \\
    P_y(y_i) = P(y, y + dy) = \frac{1}{\sqrt{2\pi}\sigma_x} \exp\left(-\frac{(y_i - Y)^{2} }{2\sigma_y^{2} }\right)dy
\end{gather*}
La probabilità di ottenere una certa coppia di valori
$x_i$ e $y_i$ da misurare $x$ e $y$ indipendenti è data da
\begin{gather*}
    P_{x + y}(x_i, y_i) = P_x(x_i)P_y(y_i)= \frac{1}{2\pi \sigma_x\sigma_y}\exp\left(-\frac{(x_i - X)^{2} }{2\sigma_x^{2}} - \frac{(y_i - Y)^{2} }{2\sigma_y^{2} }\right) dx\ dy
\end{gather*}
Vogliamo ora dimostrare che $P(z)$ sia una gaussiana utilizzando la seguente identità:
\begin{gather*}
    \frac{x^{2} }{A} + \frac{y^{2} }{B} = \frac{(x + y)^{2} }{A + B} + \frac{(Bx + Ay)^{2} }{AB(A + B)}
\end{gather*}
Ottenendo all'interno dell'esponente:
\begin{gather*}
    \frac{(x_i - X)^{2} }{\sigma_x^{2} } + \frac{(y_i - Y)^{2} }{\sigma_y^{2} } = \frac{(x_i - X + y_i - Y)^{2} }{\sigma_x^{2}+\sigma_y^{2} } + \frac{\left(\sigma_y^{2} (x_i - X) - \sigma_y^{2}(y_i - Y)\right)^{2} }{\sigma_x^{2}\sigma_y^{2}(\sigma_x^{2} + \sigma_y^{2} )} \\
\end{gather*}
Posso allora chiamare il secondo termine del secondo membro come 
\begin{gather*}
    \alpha = \frac{\sigma_y^{2}(x_i - X)-\sigma_y^{2}(y_i - Y)}{\sigma_x\sigma_y\sqrt{\sigma_x^{2}+\sigma_y^{2} } }
\end{gather*}
Ottenendo
\begin{gather*}
    P_{x+y}(x_i, y_i) = \frac{1}{2\pi \sigma_x \sigma_y}\exp\left(-\frac{1}{2}\left(\frac{(z_i - X - Y)^{2} }{\sigma_x^{2}+\sigma_y^{2} } \right)\right) \exp\left(-\frac{\alpha^{2} }{2}\right)dx \ dy = P(x_i + y_i , \alpha)
\end{gather*}
Essa è l'espressione che corrisponde alla probabilità
di trovare il valore $z_i$ con una particolare coppia di valori $x_i$ e $y_i$. 
Se vogliamo la probabilità $P_z(z_i)$ di ottenere un valore
$z_i$ indipendentemente dalla particolare coppia scelta: dovremmo
sommare i contributi di tutte le coppie di valori tali che
$z_i  = x_j + y_j$ ossia i contributi $P_{x + y}(x_i, y_i)$ 
in cui il primo termine esponenziale è sempre lo stesso 
mentre il secondo differisce a seconda della coppia scelta e quindi
l'integrazione di questi termini è un risultato del 
tipo
\begin{gather*}
    P(x + y) = P_z(z_i) = \int_{-\infty}^{+\infty} P(x + y, \alpha) \ d\alpha = \frac{1}{\sqrt{2\pi}\sqrt{\sigma_x^{2} + \sigma_y^{2} }  }\exp\left(-\frac{(z_i - x - y)^{2} }{2(\sigma_x^{2} + \sigma_y^{2} )}\right)
\end{gather*}
Il che ci porta alla seguente osservazione
\begin{gather*}
    Z = X + Y \\
    \sigma_z = \sqrt{\sigma_x^{2} + \sigma_y^{2} } 
\end{gather*}
Il fatto che si possa sommare in quadratura e ridurre l'errore
rispetto alla somma lineare è dovuto al fatto che è estremamente improbabile
una fluttuazione statistica concorde ad entrambe le grandezze misurate
direttamente. 

IV) Il caso generale \\
Prendiamo il caso generale di una certa $g = f(x_1, \dots, x_n)$ 
arriveremo a dimostrare che 
\begin{gather*}
    \sigma_g = \sqrt{\sum \left(\frac{\partial f}{\partial x_i} \sigma_{x_i}\right)^{2} } 
\end{gather*}
Per dimostrarla prendiamo il caso più semplice di
$g = f(x, y)$ e prendendo come ipotesi $\frac{\sigma_x}{|x|}<<1$ e $\frac{\sigma_y}{|y|}<<1$ in tali
condizioni possiamo sviluppare con Taylor ed ottenere
\begin{gather*}
    g_i = f(x_i, y_i) = f(X, Y) + \frac{\partial f}{\partial x}\left|_{x = X, y = Y} (x - X) + \frac{\partial f}{\partial y}  \right|_{x = X, y = Y} (y - Y)  
\end{gather*}
Tale relazione esprime il valore della grandezza $g$ misurata indirettamente con tre termini
\begin{enumerate}
    \item E' la costante $f(x, y)$ dello stesso tipo della
    costante $A$ del caso $1$;
    \item E' costituita dal fattore costante $\left.\frac{\partial f}{\partial x} \right|_{x = X, y = Y}$ moltiplicato
    per la variabile $x - X$ che, come si nota dal caso uno è distribuita
    attorno al valore vero $0$ con parametro di larghezza $\sigma_x$. Secondo il caso $2$ questo
    termine sarà distribuito attorno al valore vero $0$ e parametro di
    larghezza $\left.\frac{\partial f}{\partial x} \right|_{x = X, y = Y}\sigma_x$. 
    \item E' analoga alla alla precedente distribuita attorno
    a $0$ con distanza $\left.\frac{\partial f}{\partial x} \right|_{x = X, y = Y}\sigma_y$.
\end{enumerate}
Combinando questi $3$ termini e ricordando la terza dimostrazione si conclude
che i valori $g_i$ sono distribuiti normalmente 
intorno al valore vero $f(X, Y)$ con parametro di larghezza
dato da
\begin{align}
    \sigma_g = \sqrt{\left(\left.\frac{\partial f}{\partial x}\right|_{x = X, y = Y}\right)^2\sigma_x^{2} + \left(\left.\frac{\partial f}{\partial y}\right|_{x = X, y = Y} \right)^{2}\sigma_y^{2} } 
\end{align}


\section{Deviazione standard della media}
Ricorda che la stima di $\sigma$ abbiamo dimostrato essere
\begin{gather*}
    \sigma_x = \sqrt{\frac{\sum (x_i - \bar{x} )^{2} }{N - 1}} 
\end{gather*}
Dato $\overline{x} = \frac{\sum x_i}{N}$ come migliore stima di $X$, dimostriamo che $\sigma_{\bar{x} } = \frac{\sigma}{\sqrt{N} }$
facendo una serie di $N$ misure avrò una certa media $\bar{x}$; rifacendo l'esperimento $N$ volte, troverò
che $\bar{x}_2 \neq \bar{x}_1$ e così per le altre medie che ripeto. Posso allora definire
le misure associate a ciascuna media nella seguente maniera.  
\begin{gather*}
    m = 1, \qquad x_{1, 1}, x_{1, 2}, \dots\qquad \to \overline{x_1}\\
    \vdots \\
    m  \qquad x_{m, 1}, x_{m, 2}, \dots \qquad \to \overline{x_m} 
\end{gather*}  
Dove $m$ indica l'indice univoco di ciascuna media 
Potrò allora definire la \textbf{distribuzione di media} $P(\bar{x} )$
e osservare come si distribuiscono le misure e ricordando che ciascuna media è calcolata come
\begin{gather*}
    \bar{x} = \frac{\sum x_i}{N}
\end{gather*}
Posso affermare che nei casi $g = x+ A, g = Bx, g = x_1 + x_2$, queste
medie si distribuiscono secondo una gaussiana e anche che $\overline{x}$ è il
valore centrato di una gaussiana, della quale posso ricavare la sua
ampiezza attraverso la propagazione dell'errore:
\begin{gather*}
    g = \frac{\sum x_i}{N} \ \Longrightarrow \ \sigma_g = \sqrt{\sum \left(\frac{\partial f}{\partial x_i} \sigma_{x_i}\right)^{2} } 
\end{gather*} 
Con la somma in quadratura si può ricavare allora che
\begin{gather*}
    \sigma_g = \sqrt{\sum\left(\frac{1}{N}\sigma_x\right)^{2} } = \frac{\sigma_x}{\sqrt{N} }
\end{gather*}
Anche questa Gaussiana sarà centrata attorno al valore vero
$X$ e avrà una larghezza minore di 
\begin{gather*}
    \sigma_{\bar{x} } = \frac{\sigma_x}{\sqrt{N} }
\end{gather*}
La Gaussiana dei singoli valori invece risulterà più larga
e più bassa
\begin{gather*}
    \bar{x} \pm \frac{\sigma_x}{\sqrt{N} }
\end{gather*} e possiamo stimare anche l'incertezza
sul valore di $\sigma_x$ come
\begin{gather*}
    \sigma_x \pm \frac{\sigma_x}{\sqrt{2(N - 1)} }
\end{gather*}
L'incertezza relativa rispetto a $\sigma$ diminuisce notevolmente 
al crescere del campione di misure. Se volessimo migliorare
la precisione di un fattore $10$ dovremmo aumentare $N$ di un
fattore $100$. \\ 
Questi strumenti non tengono conto di opportune componenti
sistematiche e si possono usare se gli errori sistematici sono
trascurabili ma se $N \to +\infty$ arriverò comunque ad un certo punto in
cui l'errore accidentale sarà comunque minore dell'errore sistematico
e di conseguenza le ipotesi per l'utilizzo di questi strumenti verrebbero meno.

\chapter{Rigetto di dati}
\section{Il problema del rigetto di dati}
A volte può capitare che dei dati possano sembrare \emph{sbagliati} anche se in realtà,
dato un numero di misurazioni molto alto, è perfettamente normale trovarli anche
se molto improbabili. In questo caso la conoscenza della distribuzione
di Gauss si rivela molto utile in quanto ci permette di \emph{scartare} dei dati che
potremmo definire \emph{scomodi}. Decidere quali dati scartare il più delle volte è soggettivo in quanto
si potrebbe star scartando un dato magari misurato male (strumento tarato male o
lettura errata) ma, potenzialmente, si potrebbe star scartando la parte più
interessante del nostro set.

\section{Criterio di Chauvenet}
Si introduce allora un criterio che ci permette
di scartare misure improbabili dal nostro set di dati: 
utilizzando la probabilità della Gaussiana posso determinare
se un dato è da scartare perché poco probabile
oppure è da tenere. Si inizia calcolando innanzitutto $\bar{x}$ e $\sigma_x$
e prendendo il valore $x_s$ di cui vogliamo valutare la probabilità 
di ottenerlo. Ipotizzando una Gaussiana ideale posso ricavare
la probabilità di ottenere un certo risultato semplicemente utilizzando 
il numero di deviazioni standard dal centro della campana:
\begin{gather*}
    t_s = \frac{|x_s - \bar{x}| }{\sigma_x}
\end{gather*}
Si può allora calcolare la relativa probabilità di ottenere quel valore
come 
\begin{align}
    P(|x - \bar{x}| \geq t_S \sigma_x) = 1 - erf(t_s)
\end{align}
Che prende il nome di probabilità a due code della Gaussiana;
nel caso in cui questa misura sia molto distante dal valore
attorno al quale è centrata la distribuzione allora è possibile
eliminarla giustificando che la probabilità di ottenere questo dato rispetto
al nostro set di dati è molto bassa. Per farlo è possibile definire
il \textbf{numero di misure atteso}:
\begin{gather*}
    N_{att} = P \cdot N 
\end{gather*}
Il \textbf{Criterio di Chauvenet} ci dice che se questo numero di misure
attese è $> 0.5$ allora la misura non è da scartare, altrimenti è da
scartare. Questo criterio rappresenta una convenzione e per questa ragione è
da considerare come tale e non come una regola univocamente accettata. Inoltre, ogni volta che si
scarta una misura bisogna fare attenzione e ricalcolare
i valori della gaussiana e per questo è consigliabile eliminare al
più un valore con questo criterio. E' possibile anche
scartare un gruppo di misure: in questo caso dobbiamo moltiplicare il valore
convenzionale di misure atteso $0.5$ per il numero di misure che
si vuole scartare. Se questo gruppo di misure è abbastanza
numeroso rispetto al totale questo potrebbe essere sintomo di un
\textbf{errore sistematico}.


\chapter{Covarianza e livello di confidenza}
\section{La covarianza nella propagazione degli errori}
Supponiamo di avere delle misure ricavate dalla relazioni funzionali
$g = f(x_1, \dots, x_n)$, sappiamo che l'errore conosciuto sia
\begin{gather*}
    \sigma_g \approx \sqrt{\sum \left(\frac{\partial f}{\partial x_i} \sigma_{x_i}\right)^{2} } 
\end{gather*}
Possiamo allora ricavare una formula più generale e partendo dal
caso particolare due misure del tipo $g = f(x, y)$, esse sono sempre misurate
a coppia ed in un certo senso sono associate. Per esempio potremmo
misurare il valore $g$ attraverso un pendolo semplice e ricavare i
diversi valori e ottenere la media come la sommatoria dei valori $g_i$ oppure
ottenere
\begin{gather*}
    g = f(x, y) = f(\bar{x}, \bar{y}  ) + \left.\frac{\partial f}{\partial x}\right|_{x = \bar{x}, y = \bar{y}  } (x - \bar{x} ) +   \left.\frac{\partial f}{\partial y}\right|_{x = \bar{x}, y = \bar{y}  } (y - \bar{y} )
\end{gather*} 
E questo vale ovviamente se l'errore relativo rispetto ad entrambe $x$ e $y$
è molto minore di uno. Si ottiene che la media di questa misura $g$ è
data dallo sviluppo al primo grado della funzione media ottenendo
\begin{align}
    \bar{g} \approx \frac{1}{N} \sum \left(f(\bar{x}, \bar{y}  ) + \left.\frac{\partial f}{\partial x}\right| (x - \bar{x} ) +   \left.\frac{\partial f}{\partial y}\right|(y - \bar{y} )\right) 
\end{align}
E dato che la sommatoria degli scarti è zero, allora
\begin{gather*}
    \bar{g} = f(\bar{x}, \bar{y}  ) 
\end{gather*}
Possiamo allora calcolare la \textbf{varianza} per la misura $g$ (sviluppando al primo grado
con Taylor )
\begin{gather*}
    \sigma_g^{2} = \frac{\sum (g_i - \bar{g} )^{2} }{N - 1} \approx  \frac{1}{N - 1}\sum \left(f(\bar{x}, \bar{y}  ) + \frac{\partial f}{\partial x} (x_i - \bar{x} )  + \frac{\partial f}{\partial y}(y_i - \bar{y} ) - f(\bar{x}, \bar{y}  )\right)^{2} = \\
    \frac{1}{N - 1} \sum \left(\left(\frac{\partial f}{\partial x} \right)^{2} (x_i - \bar{x}  )^{2} + \left(\frac{\partial f}{\partial y} \right)^{2} (y_i - \bar{y} )^{2} + 2 \frac{\partial f}{\partial x}\frac{\partial f}{\partial y} (x_i - \bar{x} ) (y_i - \bar{y} )\right)  
\end{gather*}
Posso allora scomporre la parentesi in tre sommatorie ed osservare che
\begin{align}
    \sigma_g^{2} \approx \left(\frac{\partial f}{\partial x} \right)^{2}\sigma_x^{2} + \left(\frac{\partial f}{\partial y} \right)^{2}\sigma_y^{2} + 2\frac{\partial f}{\partial x}\frac{\partial f}{\partial y} \sigma_{x, y}        
\end{align}
Dove l'ultimo termine prende il nome di \textbf{covarianza}, ossia dà informazione
sulla dipendenza l'una dall'altra delle due variabili $x, y$. Le due variabili sono
indipendenti tra di loro se e solo se  $\sigma_{x, y} \to 0$.
dobbiamo tenere conto della covarianza quando le variabili sono correlate
tra di loro 
\begin{itemize}
    \item \textbf{correlate positivamente}: $\sigma_{x, y} > 0$;
    \item \textbf{correlate negativamente} o \textbf{anticorrelate}: $\sigma_{x, y} < 0$;
    \item \textbf{non correlate}: $\sigma_{x, y} = 0$.
\end{itemize}
Nel caso di covarianza nulla allora avrò per ogni sovrastima di $x$ una
sottostima di $y$ e quindi in una relazione del tipo $g = f(x, y)$ so ch
se le misure sono indipendenti tra di loro allora la covarianza sarà sempre zero
poiché i contributi saranno mediamente ugualmente positivi e negativi. Anche 
se le misure non sono correlate allora non è detto che siano indipendenti,
mentre, è vero che, per $N$ molto grande, grandezze dipendenti
siano correlate. E' vero però che indipendenti $\Rightarrow$ non correlate.

\section{Dimostrazione della conservatività della somma delle incertezze assolute}
Sappiamo che se $g = x + y$, allora $\Delta g = \Delta x + \Delta y$.
Voglio allora dimostrare che questa è una stima conservativa utilizzando la 
\textbf{disuguaglianza di Cauchy-Schwartz} la cui tesi è che
\begin{gather*}
    |\sigma_{x, y}| \leq \sigma_x \sigma_y
\end{gather*}
Sappiamo allora che
\begin{gather*}
    \sigma_x \sigma_y \geq |\sigma_{x, y}| \geq \sigma_{x, y}
\end{gather*}
Il che ci porta a giustificare che la somma lineare delle due incertezze assolute
è una stima conservativa dell'errore infatti possiamo esprimere 
\begin{gather*}
    \sigma_g^{2} \approx \left(\frac{\partial f}{\partial x} \sigma_x\right)^{2} + \left(\frac{\partial f}{\partial y}\sigma_{y} \right)^{2} + 2\frac{\partial f}{\partial x}\frac{\partial f}{\partial y}\sigma_{x, y } \leq \\
    \left(\frac{\partial f}{\partial x} \right)^{2}\sigma_x^{2} + \left(\frac{\partial f}{\partial y} \right)\sigma_{y}^2 + 2 \left| \frac{\partial f}{\partial x} \right| \left| \frac{\partial f}{\partial y}  \right| \left| \sigma_{x, y} \right|      \leq\\
        \left(\frac{\partial f}{\partial x} \right)^{2}\sigma_x^{2} + \left(\frac{\partial f}{\partial y} \right)\sigma_{y}^{2} + 2 \left| \frac{\partial f}{\partial x} \right| \left| \frac{\partial f}{\partial y}  \right| \sigma_{x} \sigma_y\\
        = \left(\frac{\partial f}{\partial x}\sigma_x + \frac{\partial f}{\partial y} \sigma_y \right)^{2} 
\end{gather*}   
Allora si ottiene la tesi
\begin{align}
    \sigma_g \leq   \frac{\partial f}{\partial x}\sigma_x + \frac{\partial f}{\partial y} \sigma_y
\end{align}
Si vede allora che il massimo valore per la deviazione standard sulla misura di $g$ è quello
che si ottiene è quello che si ottiene allora sommando linearmente i valori assoluti delle
incertezze misurate direttamente. \\
Dimostriamo ora la disuguaglianza di Cauchy-Schwartz. \\
Considero innanzitutto una funzione ausiliaria come
\begin{gather*}
    A(t) = \frac{1}{N - 1} \sum \left((x_i - \bar{x} ) + t(y_i - \bar{y} )\right)^{2}  
\end{gather*}
Che è sempre maggiore di zero e quindi il suo minimo è anch'esso
maggiore di zero. Ora posso derivare questa funzione e trovare questo minimo:
\begin{gather*}
    \frac{1}{N - 1} \sum \left(2\left((x_i - \bar{x} ) + t(y_i - \bar{y} )\right)(y_i - \bar{y} )\right) = 0
\end{gather*}
Per avere il minimo posso portare dentro le sommatorie e quindi spezzarle nel seguente modo:
\begin{gather*}
    2\left(\frac{1}{N - 1}\sum(x_i - \bar{x} )(y_i - \bar{y} ) + t\sum \frac{(y_i - \bar{y} )^{2} }{N - 1}\right) = 0
\end{gather*}
E quindi
\begin{gather*}
    2(\sigma_{x, y} + t\sigma_y^{2} ) = 0
\end{gather*}
Avremmo allora che per trovare il minimo
\begin{gather*}
    t_{min} = - \frac{\sigma_{x, y}}{\sigma_y^{2} }
\end{gather*}
E quindi si ottiene la tesi che cercavamo nella dimostrazione precedente.


\section{Livello di confidenza}
Può accadere spesso di verificare la compatibilità di due 
o più misure in modo statistico: chiamiamo $x_m$ la misura effettuata di una grandezza 
$x$ che soddisfa due ipotesi principali:
\begin{itemize}
    \item Errori non sistematici;
    \item Un alto numero di misure.
\end{itemize}
Stimerò la misura allora come $x_m \pm \sigma$ e ci chiediamo se questo valore
sia o meno compatibile con un certo $x_e$: prima di tutto si era detto che
un valore (non statisticamente) è compatibile con un altro se il suo valore
vero si trova all'interno della barra d'errore dell'altro valore. \\
Con un approccio un pochino più statistico supponiamo che
all'interno di quell'intervallo si possa trovare
$x_e$ con una probabilità del $68\%$. Nel caso in cui
$x_e$ caschi fuori dall'intervallo allora possiamo utilizzare lo
stesso approccio utilizzato per il criterio di Chauvenet: allora chiamo $t$
la distanza in deviazioni standard in modo da relazionare
la discrepanza delle due misure con la deviazione standard. Ovviamente mi aspetterei 
$t = 0$ anche se è statisticamente impossibile che due misure siano perfettamente uguali
tra di loro. 
\begin{gather*}
    t = \frac{|x_m - x_e|}{\sigma}
\end{gather*}
Posso allora ricavarmi la probabilità di ottenere un certo valore $x$
generico che disti dal valore ottenuto più di
$t\sigma$.
\begin{gather*}
    P(|x - x_e| \geq t\sigma) = 1 - \int_{x_e - t\sigma}^{x_e + t\sigma}p(x) \ dx = 1 - erf(t)
\end{gather*}
Dobbiamo allora decidere quando una probabilità è tale che da ritenere ragionevole
la distanza trovata e definire la compatibilità. Convenzionalmente 
si considera accettabile un valore di probabilità di almeno il $5\%$.
Si dice il \textbf{livello di confidenza} la probabilità opposta del $95\%$ e possiamo allora
dire che una probabilità del $5\%$ corrisponde a $t = 1.96$. \\
Se allora $x_e$ dista più di questo valore in deviazioni standard allora è ragionevole pensare
che tale distanza non sia dovuta a fluttuazioni statistiche a che invece
ci siano stati degli errori nelle misure. Possiamo allora indicare il livello di confidenza
quando si svolgono delle misure proprio secondo questo criterio. Se una misura anomala è in genere
oltre i $5\sigma$ di solito è una misura da attribuire ad una nuova scoperta come negli
esperimenti del CERN.

\section{Confronto tra due misure}
Se volessimo confrontare due misure tipo 
\begin{gather*}
    x_1 \pm \sigma_1 \qquad \qquad x_2 \pm \sigma_2
\end{gather*}
Possiamo fare un ragionamento un base alla \textbf{discrepanza} tra le due misure:
\begin{gather*}
    \tau_m = |x_1  - x_2|
\end{gather*}
Il valore ideale per la discrepanza tra due misure è ovviamente zero, l'idea di utilizzare
questa proprietà è analoga a quella del caso del confronto con un valore dato.
Propago allora l'errore per ricavarmi l'errore sulla discrepanza come
errore in quadratura $\sigma_{\tau} = \sqrt{\sigma_1^{2} + \sigma_2^{2}  } $ e posso allora 
calcolarmi la \textbf{discrepanza normalizzata} in termini di $\tau$ e quindi
\begin{gather*}
    t = \frac{|\tau_m - \tau_e|}{\sigma_{\tau}} = \frac{\tau_m}{\sigma_{\tau}} = \frac{|x_1 - x_2|}{\sigma_{\tau}}
\end{gather*}
E quindi la probabilità di ottenere un valore al di fuori di questo
intervallo è allora 
\begin{gather*}
    P(|\tau - \tau_e| \geq t\sigma_{\tau}) = 1 - erf(t)
\end{gather*}
Se si ragionasse con un livello di confidenza analogo 
dovremmo trovare ad una distanza di circa $2\sigma$.


\chapter{Medie pesate}
\section{La media pesata}
Avendo fatto varie misure di una certa grandezza la \textbf{media pesata}
ci permette di combinarle in modo da ottenere un unico valore. L'unico
prerequisito della media pesata è, ovviamente, la compatibilità tra le misure.
Supponendo di aver fatto determinate misure si può, dato che queste sono
compatibili, combinarle assegnando un certo peso ad ognuna di esse 
in modo che alcune risultino più "importanti" di altre; si pesano inoltre anche
le incertezze delle singole misure in modo che alcune incertezze siano più significative
di altre.
Possiamo allora scrivere la probabilità di ottenere un certo valore arbitrariamente
vicino a tre valori misurati e ricavarne la probabilità massima:
posso pensare questa probabilità come una funzione del valore vero e trovare 
quel valore vero che rende la probabilità maggiore possibile 
\begin{gather*}
    P(x_1, x_1 + dx ) = \frac{1}{\sqrt{2\pi} \sigma_1} \exp\left(-\frac{(x_1 - x)^{2} }{2\sigma_1^{2} }\right)  = P_1
\end{gather*}
La probabilità complessiva sarà proporzionale a
\begin{gather*}
    P_{tot} = P(x_1)P(x_2)P(x_3) \propto \frac{1}{\sigma_1 \sigma_2 \sigma_3} \exp\left(-\sum\left(\frac{(x_i - x)^{2} }{\sigma_i^{2} }\right)\right)
\end{gather*}
Avrò allora che le probabilità delle altre due misure posso indicarle con $P_2$ d $P_3$ e allora
la probabilità complessiva di ottenere una misura tra $x_1 + dx$ e le altre due misure
considerate, (se sono ovviamente indipendenti e quindi non correlate) sarà data dalla minimizzazione dei quadrati:
\begin{gather*}
    P_{tot} = \prod P_i = \frac{1}{\sigma_1^{2} \sigma_2^{2} \sigma_3^{2}} \exp\left(-\frac{\chi^{2} }{2}\right) \ (dx)^{3} \\
    \chi^{2} = \frac{(x_1 - x)^{2} }{\sigma_1^{2} } + \frac{(x_2 - x)^{2} }{\sigma_2^{2} } + \frac{(x_3 - x)^{2} }{\sigma_3^{2} }  = \sum\left(\frac{(x_i - x)^{2} }{\sigma_i^{2} }\right)
\end{gather*}
Allora derivando rispetto al valore vero $x$ ottengo l'espressione della minimizzazione
del $chi$ quadrato:
\begin{gather*}
    x = \frac{\frac{x_1}{\sigma_1^{2} } + \frac{x_2}{\sigma_2^{2} } + \frac{x_3}{\sigma_3^{2} }}{\frac{1}{\sigma_1^{2} } + \frac{1}{\sigma_2^{2} } + \frac{1}{\sigma_3^{2} }}
\end{gather*}
In termini generali posso affermare che
\begin{align}
    x_{MP} = \frac{\sum \frac{x_i}{\sigma_i^{2} }}{\sum \frac{1}{\sigma_i^{2} }}
\end{align}
Possiamo chiamare i pesi $\omega_i = \frac{1}{\sigma_i^{2} }$ 
e ottenere una formulazione più generale attraverso i pesi
\begin{align}
     x_{MP} = \frac{\sum x_i \omega_i}{\sum w_j}
\end{align}
L'incertezza da associare a questo valore si può ottenere allora dalla propagazione
dell'errore.


\section{L'incertezza associata alla media pesata}
Come posso allora ricavare l'incertezza sulla media pesata $\sigma_{MP}$?
Per fare questo posso riscrivere l'incertezza secondo la definizione e ottenere:
\begin{gather*}
    \sigma_{MP}^{2} = \sum\left(\frac{\partial x_{MP}}{\partial x_i} \sigma_i\right)^{2}\\
        \frac{\partial x_{MP}}{\partial x_i} = \frac{\partial }{\partial x_i} \frac{x_1 \omega_1 + \dots + x_n \omega_n}{\sum \omega_j} = \\
    \frac{1}{\sum \omega_s} \left(\frac{\partial }{\partial x_i}(x_1 \omega_1) + \dots + \frac{\partial }{\partial x_i} (x_i \omega_i) + \dots + \frac{\partial }{\partial x_n} \frac{\partial }{\partial x_i} (x_n \omega_n)    \right) =\frac{1}{\sum \omega_j} \omega_i  \\
    \ \Longrightarrow \ \sigma_{MP}^{2} = \sum \left(\frac{\omega_i}{\sum \omega_j}\right)^{2} \frac{1}{\omega_i} = \frac{1}{\left(\sum \omega_j\right)^{2} } \sum \omega_i = \frac{1}{\sum \omega_i} 
\end{gather*}
Si ottiene allora l'incertezza sulla media pesata come
\begin{align}
    \sigma_{MP} = \sqrt{\frac{1}{\sum \frac{1}{\sigma_i^{2} }}} 
\end{align}


\section{Applicazioni della media pesata}
Nel caso di misure ripetute della stessa grandezza e sapendo che tutti i pesi sono identici, viene meno l'utilizzo
delle formule complesse della media pesata e favorendo dunque l'utilizzo delle formule semplici
viste in precedenza. La media pesata infatti si può utilizzare
se e solo se gli errori sono di natura accidentale e non di natura sistematica
altrimenti si potrebbero assegnare pesi grandi a misure fatte con errori
sistematici grandi.


\chapter{Relazioni tra grandezze fisiche}
\section{Il metodo grafico}
\begin{wrapfigure}{r}{0.5\textwidth}
    \centering
    \label{Fig 1.1}
    \caption{Misure rispetto al modello lineare}
    \begin{tikzpicture}
        \draw[->] (0, 0) -- (4, 0) node[at end, below] {$x$};
        \draw[->] (0, 0) -- (0, 4) node[at end, left] {$y$};
        \draw[-, dashed] (0, 1) -- (4, 3);
        \filldraw (1, 1.2) circle (1pt);
        \filldraw (1.5, 2)circle (1pt);
        \filldraw (2, 1.8) circle (1pt);
        \filldraw (2.5, 2.4) circle (1pt);
        \draw[-] (1, 0.1) -- (1, -0.1) node[at end, below] {$x_i$};
        \draw[-] (-0.1, 1.2) -- (0.1, 1.2) node[at start, left] {$y_i$};
        \draw[|-|, red] (1, 0.6) -- (1, 1.8) node[at end, left] {$\sigma_y$};
        \draw[|-|, red] (0.7, 1.2) -- (1.3, 1.2) node[at start, below] {$\sigma_x$};
    \end{tikzpicture}    
\end{wrapfigure}
Ipotizziamo di aver acquisito delle grandezze $x$ e $y$, vogliamo trovare la relazione che
lega le due grandezze come $y = f(x)$. Iniziando
dalla relazione $y = A + Bx$, si può utilizzare il metodo grafico, ossia
possiamo utilizzare un grafico su carta millimetrata per poter studiare la distribuzione trovata
e verificare che sia compatibile con una distribuzione lineare. \\
In generale i punti possiedono però anche delle barre di errore e quindi
dopo aver rappresentato anche quelle è possibile determinare
se un punto appartiene al modello lineare oppure se è troppo distante per
potere essere ragionevolmente considerato. La statistica ci dice che si 
aspetta una compatibilità del $68\%$ e ricordandoci che nel caso
ci fossero punti molto distanti dagli altri essi possono essere
scartati con opportune considerazioni.

\section{Determinazione di $A$ e $B$}
Per ricavare i coefficienti $A$ e $B$ nel modello di relazione lineare $y = A + Bx$
possiamo ricorrere ai seguenti metodi: posso ricavare $B$, ovvero il \textbf{coefficiente angolare},
prendendo una coppia di punti e utilizzando la relazione 
\begin{gather*}
    B = \frac{y_2 - y_1}{x_2 - x_1}
\end{gather*}
E' consigliabile prendere punti più distanti possibili
in quanto la quadrettatura del foglio millimetrato ci darà una 
incertezza di $1mm$.
In generale si prendono i punti in cui la retta interseca 
i bordi del foglio (ovviamente sempre che siano i più comodi). \\
Per ricavare $A$ posso seguire la seguente procedura:
\begin{gather*}
    A = y_0 - B x_0
\end{gather*}
Dove $(x_0, y_0)$ è il punto maggiormente vicino al bordo di
sinistra del grafico sempre per una questione di approssimazione
e una volta determinati tali valori dovremmo ricavare le
\textbf{incertezze} associate. In generale non ci sarà solo una retta 
a soddisfare la nostra relazione ma ne avremo molte di più. \\
Posso adattare \textbf{criteri soggettivi} per determinare la retta
da considerare; normalmente si prende quella che passa più vicino possibile a
più punti possibili. Dopodiché cerchiamo gli estremi ovvero la
retta più pendente e quella meno pendente che soddisfano ancora i
nostri prerequisiti e scrivere allora
\begin{align*}
    A_{best}& \quad B_{best} \to \text{retta probabile} \\
    A_{min}& \quad B_{min} \to \text{retta meno inclinata} \\
    A_{max}& \quad B_{max} \to \text{retta più inclinata}
\end{align*}
Possiamo ricavare allora l'errore sulla migliore stima di $A$ e $B$ come
lo scarto massimo tra la migliore stima e il loro valore massimo e minimo
\begin{gather*}
    A_{best} \pm \Delta A \qquad B_{best} \pm \Delta B
\end{gather*}
Ovviamente bisogna approssimare le cifre significative
a causa della limitatezza dei quadretti.

\section{Relazione non lineare}
Nel caso di una relazione non lineare il metodo grafico diventa molto
più complicato: prendiamo per esempio il caso $y = A + Bx^{2}$. 
Un metodo potrebbe essere prendere in relazione una grandezza e la funzione
associata alla seconda grandezza e allora la relazione diventa 
lineare nella seguente maniera:
\begin{gather*}
    y = A + Bg(x) \qquad g(x)  = x^{2} 
\end{gather*} 
Un altro modo per ottenere una relazione lineare è quella di poter 
"sistemare" l'asse delle ascisse in modo tale che diventi lineare.

\section{Metodo dei minimi quadrati}
Con il metodo dei minimi quadrati posso ottenere un \emph{fit grafico} molto più preciso
che con il metodo grafico visto fino ad ora: questo metodo infatti si basa univocamente sul
nostro set di dati e ci permette di determinare la retta migliore dal punto di vista
statistico. Partiamo da una situazione di una
coppia di misure e supponiamo che le incertezze su $y$ abbiano tutte la
stessa grandezza, altrimenti andrebbero separate. Possiamo allora studiare,
grazie alla statistica, la probabilità
\begin{gather*}
    P(y_1, y_1 + dy), \dots, P(y_i, \dots, y_i + dy)
\end{gather*}
Per ipotesi possiamo considerare che ogni valore $y_i$ sia
legato ad un certo $x_i$ e quindi possiamo dire che la 
probabilità di ottenere un certo valore $y_i$ sia data da
\begin{gather*}
    P(y_i, y_i + dy) \approx \frac{1}{\sqrt{2\pi} \sigma_i} \exp \left(-\frac{\left(y_i - (A + Bx_i)\right)^{2} }{2\sigma_i^{2} }\right) \ dy
\end{gather*}
Possiamo inoltre dire che le incertezze sulle singole $y_i$ sono tutte
uguali in quanto sono state ricavate attraverso il medesimo strumento
e dunque è ragionevole pensare che $\sigma_{y_1} = \dots = \sigma_{y_n}$. Inoltre
posso utilizzare il principio di Massima verosimiglianza e quindi esprimere la
probabilità totale come
\begin{gather*}
    P_{tot} =  \left(\frac{dy}{\sqrt{2\pi} }\right)^{n} \frac{1}{\prod \sigma_i} \exp\left(-\frac{1}{2}\sum\frac{(y_i - A - Bx_i)^{2} }{\sigma_i^{2} }\right) = \left(\frac{dy}{\sqrt{2\pi} }\right)^{n} \frac{1}{\prod \sigma_i} \exp\left(-\frac{\chi^{2} }{2}\right) \\
    \text{dove } \chi = \sum \frac{(y_i - A - Bx_i)^{2} }{\sigma^{2} }
\end{gather*}
Il valore $y_i - A - Bx_i$ è lo scarto lungo l'asse $y$ del valore misurato
e del punto in cui la retta interseca l'intervallo (idealmente sarebbe zero). Applicando
ora il principio di massima verosimiglianza le migliori stime di $A$ e $B$ sono quei valori
per i quali la loro probabilità è massima. \\
Il valore massimo si ha in corrispondenza del minimo esponente dell'esponenziale; questo
si può osservare sempre con il principio di Massima verosimiglianza che facendo la derivata di questa funzione a due variabili e imponendo che sia uguale a zero,
l'unico termine che devo porre uguale a zero è
il termine all'interno dell'esponenziale e dunque posso dire che
\begin{gather*}
    \left\{\begin{array}{l}
        \frac{\partial \chi^{2} }{\partial A} = \sum\left(-\frac{1}{\sigma^{2} }2(y_i - A - Bx_i)\right) = 0 \\
        \frac{\partial \chi^{2} }{\partial B} = \sum\left(-\frac{1}{\sigma^{2} }2(y_i - A - Bx_i)x_i\right) = 0 \\
    \end{array}
    \right. \\
    \left\{\begin{array}{l}
        \sum A + B\sum x_i = \sum y_i \\
        A \sum x_i + B \sum x_i^{2} = \sum x_i y_i 
    \end{array}\right.
\end{gather*}
Possiamo risolvere il sistema in forma matriciale ed ottenere
\begin{gather*}
    \begin{pmatrix}
        n & \sum x_i \\
        \sum x_i & \sum x_i^{2}  
    \end{pmatrix} \begin{pmatrix}
        A \\
        B
    \end{pmatrix} = \begin{pmatrix}
        \sum y_i \\
        \sum x_i y_i
    \end{pmatrix}
\end{gather*}
E quindi trovare le costanti $A$ e $B$ coe
\begin{align}
    A &= \frac{\sum x_i^{2}\sum y_i - \sum x_i \sum x_i y_i}{n\sum x_i^{2} - (\sum x_i)^{2} } \\
    B &= \frac{n\sum x_i y_i - \sum x_i \sum y_i}{n\sum x_I^{2} - (\sum x_i)^{2}  }
\end{align}


\section{Incertezze tra $A$ e $B$}
Possiamo considerare le grandezze $A$ e $B$ come ottenute indirettamente 
e quindi posso propagare l'errore ed ottenere $\sigma_A$ nell'ipotesi che gli errori siano
stati tutti accidentali.
\begin{gather*}
    \sigma_A^{2} = \sum \left(\frac{\partial f}{\partial x_i} \sigma_{x_i} \right)^{2} + \sum \left(\frac{\partial f}{\partial y_i} \sigma_{y_i}\right)^{2}   
\end{gather*}
Dato che si è specificato che l'incertezza sulle $x$ sia molto più piccola di quella 
sulle $y$, allora posso ignorare il primo termine del secondo membro e scrivere
semplicemente (ovviamente considerando che tutte le incertezze su tutti
gli $y_i$ siano uguali) che
\begin{gather*}
    \sigma_A^{2} = \sum \left(\frac{\partial f}{\partial y_i} \sigma_{y_i}\right)^{2}   
\end{gather*}
Ora, posso svolgere la parziale dentro le parentesi e ottenere:
\begin{gather*}
\frac{\partial A}{\partial y_i} = \frac{\partial }{\partial y_i}  \frac{\sum x_i^{2}\sum y_i - \sum x_i \sum x_i y_i}{n\sum x_i^{2} - (\sum x_i)^{2} }  \Rightarrow \\
\frac{\partial }{\partial y_i} = \frac{1}{n \sum x_i^{2} - (\sum x_i)^{2} } \cdot \left(\sum x_i^{2} - x_j \sum x_i \right) = \frac{1}{\Delta}\left(\sum x_i^{2} - x_j \sum x_i \right)\\
\Delta = n \sum x_i^{2} - (\sum x_i)^{2}
\end{gather*}
E quindi posso risolvere la seguente:
\begin{gather*}
    \sigma_A^{2} =  \sum\left(\frac{1}{\Delta}\left(\sum x_i^{2} - x_j \sum x_i \right)\right)^{2}  \sigma_y^{2} = \\
    \sum \left(\frac{1}{\Delta^{2} }\left(\left(\sum x_i^{2} \right)^{2} + x_j^{2} \left(\sum x_i\right)^{2}  - 2x_j \sum x_i \sum x_i^{2} \right) \right)\sigma_y^{2} = \\
    \frac{\sigma_y^{2} }{\Delta^{2} } \left(n\left(\sum x_i^{2} \right)^{2} + \sum x_j^{2} \left(\sum x_i\right)^{2} -2 \sum x_i \sum x_i^{2} \sum x_j\right) = \\
    \frac{\sigma_y^{2} }{\Delta^{2} } \left(n\left(\sum x_i^{2} \right)^{2} - \sum x_i^{2} \left(\sum x_i\right)^{2}   \right) = \\
    \frac{\sigma_y^{2} }{\Delta^{2} } \left(\sum x_i^{2} \right)\left(n \sum x_i^{2} - \left(\sum x_i \right)^{2} \right)
\end{gather*}
Si ottiene allora, dato che l'ultima parentesi è uguale a $\Delta$
\begin{align}
    \sigma_A = \sigma_y \sqrt{\frac{\sum x_i^{2} }{\Delta}} 
\end{align}
Possiamo allora applicare lo stesso ragionamento per $B$ e quindi ottenere
\begin{gather*}
    \sigma_B^{2} = \sum \left(\frac{\partial B}{\partial y_i} \right)^{2}\sigma_y^{2}   
\end{gather*}
E quindi dato
\begin{gather*}
    \frac{\partial B}{\partial y_i} =  \frac{1}{\Delta} \left(n x_j  - \sum x_i \right) \\
    \Delta = n \sum x_i^{2} - (\sum x_i)^{2}
\end{gather*}
Posso risolvere la seguente:
\begin{gather*}
    \sigma_B^{2} = \sum \left(\frac{1}{\Delta} \left(n x_j - \sum x_i\right)\right)^{2} \sigma_y^{2} = \\
    \frac{1}{\Delta^{2} } \sum \left(n^{2} x_j^{2} + \left(\sum x_i\right)^{2} - 2nx_j \sum x_i  \right)   = \\
    \frac{\sigma_y^{2} }{\Delta^{2} } \left(n^{2} \sum x_i^{2} + n\left(\sum x_i\right)^{2} -  2n \sum x_i \sum x_j \right) = \\
    \frac{\sigma_y^{2} }{\Delta^{2} } \left(n^{2} \sum x_i^{2} - n\left(\sum x_i\right)^{2}   \right)^{n}  
\end{gather*}
Allora dato che nell'ultimo termine c'è nuovamente $\Delta$ si ottiene
\begin{align}
    \sigma_B = \sigma_y \sqrt{\frac{n}{\Delta}} 
\end{align}


\section{Linearità e coefficiente di correlazione lineare}
\begin{wrapfigure}{r}{0.4\textwidth}
    \centering
    \label{Fig 4}
    \caption{Grafico per il fit}
    \begin{tikzpicture}
        \draw[->] (0, 0) -- (4, 0) node[at end, below] {$x$};
        \draw[->] (0, 0) -- (0, 4) node[at end, left] {$y$};
        \draw[-, dashed] (0, 0.3) -- (4, 4);
        \draw[-, red] (0, 0.5) -- (4, 3.6);
        \draw[-, cyan] (0, 0.1) -- (4, 4.2);
        \filldraw (1, 1) circle (1pt);
        \filldraw (2, 2.2) circle (1pt);
        \filldraw (3, 2.8) circle (1 pt);
        \filldraw (4, 3.8) circle (1pt);
        \draw[|-|] (1, 1.3) -- (1, 0.7);
        \draw[|-|] (2, 2.5) --  (2, 1.9);
        \draw[|-|] (3, 3.2) -- (3, 2.4);
        \draw[|-|] (4, 4.2) -- (4, 3.4);
        \filldraw[red] (1.5, 1) circle (1pt);
        \draw[|-|, red] (1.5, 1.4) -- (1.5, 0.6) node[midway, right] {da scartare};
    \end{tikzpicture}    
\end{wrapfigure}
Osservando il grafico possiamo ricavare la distanza dei valori
dalla retta e, con la minimizzazione dei quadrati, possiamo allora 
determinare la Linearità del campione di dati che abbiamo e verificare quindi
che la distribuzione dei dati sia compatibile con l'errore
accidentale e che sia effettivamente una distribuzione lineare di dati. \\
Se la relazione è davvero lineare allora mi aspetterò che
\begin{gather*}
    Y_i = A + B X_o \approx A + B x_i
\end{gather*}
Dovremmo avere che $y_i - A - B x_i = 0$ sia il più piccolo
possibile. Nonostante si conosca solo dati che
non mi permettano di determinare questa distribuzione di scarti, si sa che
tutti i valori $y_i$ sono distribuiti normalmente intorno
al valore vero $A + Bx_i$ e le deviazioni $y_i - A - Bx_i$ sono anch'esse
distribuite normalmente attorno al valore vero $0$ e con lo
stesso parametro di larghezza $\sigma_y$.
\begin{gather*}
    \overline{\sigma_y} = \frac{\sqrt{\sum (y_i - A - Bx_i)^{2} } }{N - 2}
\end{gather*} 
Questo mi da il valore dello sparpagliamento dei valori rispetto alla retta
vera (che tuttavia non ho), la quale è ottenuta secondo il metodo dei minimi quadrati:
la quale ha due parametri, e quindi due gradi di libertà e da qui segue il fattore $N - 2$
al denominatore. \\
Se graficamente trovo che
\begin{itemize}
    \item $\overline{\sigma_y} \approx \sigma_y$: La \textbf{fluttuazione media} dei valori $y$
    rispetto alla retta dei minimi quadrati è confrontabile con
    l'incertezza sperimentale;
    \item $\overline{\sigma_y} >> \sigma_y$: La fluttuazione è molto maggiore
    di quanto ci si potrebbe aspettare dall'incertezza misurata e quindi non è
    molto normale, questo implica che abbiamo sottostimato l'incertezze oppure che
    i dati potrebbero non avere una distribuzione lineare;
    \item  $\overline{\sigma_y} << \sigma_y$: se le barre di errore
    fossero state valutate meglio, avremmo avuto una minoranza di punti che non contiene
    la retta: probabilmente gli errori sperimentali non sono stati sovrastimati e quindi, se 
    fossero stati calcolati meglio, alcuni punti potrebbero non appartenere alla retta trovata.
\end{itemize} 
Possiamo anche ragionare con grandezze non fisiche con il metodo
dei minimi quadrati e trovare quindi una sequenza di valori (se è possibile ovviamente)
rappresentarla come relazione lineare utilizzando il
coefficiente di correlazione lineare
\begin{align}
    r = \frac{\sum (x_i - \bar{x} )(y_i - \bar{y} )}{\sqrt{\sum (x_i - \bar{x} )^{2} \sum (y_i - \bar{y} )^{2} } } \qquad \left( = \frac{\sigma_{x, y}}{\sigma_x \sigma_y}\right)
\end{align}
Data la tesi del teorema di Cauchy-Schwartz, posso dire che $-1 \leq r \leq 1$. Se il modello
fosse lineare, allora $r$ assumerebbe o $-1$ oppure $1$. Possiamo allora ottenere da queste ipotesi che
\begin{gather*}
    \bar{y} = \frac{\sum y_i}{N} = A + \frac{B\sum x_i}{N} = A + B\bar{x}  
\end{gather*}
E quindi sostituendo si ottiene che
\begin{gather*}
    y_i - \bar{y} = B(x_i - \bar{x} ) 
\end{gather*}
E allora 
\begin{gather*}
    r = \frac{B}{|B|} = \pm 1
\end{gather*}
Avendo un certo set di misure possiamo ricavare la retta coi minimi quadrati e quindi con
$A$ e $B$ possiamo interpolare e scoprire i valori $y'$ ai quali corrisponderebbero determinati
valori di $x'$. In questo modo posso anche ricavarmi la covarianza nel modello lineare ottenendo allora:
\begin{gather*}
    \sigma_y^{2} = \left(\frac{\partial y}{\partial A} \right)^{2}\sigma_A^{2} + \left(\frac{\partial y}{\partial B} \right)^{2}\sigma_B^{2} 2\frac{\partial y}{\partial A} \frac{\partial y}{\partial B}\sigma_{A, B}  \\
    \sigma_{A, B} = -\sigma_y^{2} \frac{\sum x_i}{\Delta} 
\end{gather*}


\section{Metodo dei minimi quadrati pesati}
Studiamo il caso in cui le incertezze dei valori non siano simili
tra di loro e assumendo come sempre errori puramente accidentali ci fermiamo
a definire intanto
\begin{gather*}
    \chi^{2} = \frac{\sum (y_i - A - Bx_i)^{2} }{\sigma_y^{2} } = \sum \omega_i (y_i - A - Bx_i) ^{2} 
\end{gather*}
E quindi posso risolvere il sistema per trovare il punto di minimo
\begin{gather*}
    \left\{\begin{array}{l}
        0 = \frac{\partial \chi^{2} }{\partial A} = -2\sum \omega_i (y_i - A - Bx_i) \\
        0 = \frac{\partial \chi^{2} }{\partial B} = -2\sum \omega_i(y_i - A  -B x_i) 
    \end{array}\right.
\end{gather*}

Ottenendo allora 
\begin{align}
    A &= \frac{\sum \omega_i x_i^{2} \sum \omega_i y_i - \sum \omega_i x_i - \sum \omega_i x_i y_i}{\sum \omega_i \sum \omega_i x_i^{2} - \left(\sum \omega_i x_i\right)^{2} }\\
    B &= \frac{\sum \omega_i \sum \omega_i x_i y_i - \sum \omega_i x_i \sum\omega_i y_i}{\sum \omega_i \sum \omega_i x_i^{2} - \left(\sum \omega_i x_i\right)^{2}  }
\end{align}
Con il metodo dei quadrati pesati si tende a dare più importanza ai valori con
la minore incertezza, altrimenti adattando il metodo standard avremo un risultato diverso. Propagando
l'errore in maniera analoga si ottiene allora
\begin{align}
    \sigma_A = \sqrt{\frac{\sum \omega_i x_i^{2} }{\Delta \omega}}\\
    \sigma_B = \sqrt{\frac{\sum \omega_i}{\Delta \omega}}  
\end{align}
E anche in questo caso posso definire $\overline{\sigma}$ che mi permette di verificare
se i miei valori abbiano senso oppure no. Infatti
se $y_i - A - Bx_i \approx \sigma_{y_i}$ allora mi aspetto che
\begin{gather*}
    \chi^{2} = \sum\frac{(y_i - A - Bx_i)^{2} }{\sigma_{y_i}^{2} }  
\end{gather*} 
Dati i due gradi di libertà di questo calcolo, allora ottengo
che
\begin{gather*}
    \frac{\chi^{2} }{2} \sim 1. 
\end{gather*}

\section{Varianza efficace e relazioni non lineari}
Possiamo anche studiare il caso in cui le incertezze su $x$ non siano approssimabili 
e quindi data la relazione tra i due errori, la banda di errore
su $x$ è legata alla banda di errore su $y$ secondo la seguente relazione
\begin{gather*}
    \sigma_{y_i} = \tan\alpha \sigma_{x_i} = \frac{dy}{dx}  \sigma_{x_i}
\end{gather*}
Posso allora definire la \textbf{varianza efficace} l'errore totale e quindi,
utilizzando la somma in quadratura e ricordando che la derivata rispetto ad $x$ di $y$
è proprio $B^{2}$ e che se la relazione fosse lineare avrei $\tan^{2}\alpha = B^{2}$, allora posso
definire
\begin{gather*}
    \sigma_{y_i} = \sqrt{\sigma_{y_i}^{2} \left(\frac{\partial y}{\partial x} \right)^{2}\sigma_{x_i}^{2}   }  = \sqrt{\sigma_{y_i}^{2} + B^{2} \sigma_{x_i}^{2}}    
\end{gather*}  
 Un qualunque polinomio del tipo $y = A + Bx + \dots$ 
sono relazioni non lineari per $x$ e $y$ ma lineari 
per $A$ e $B$ se provassimo a rifare i calcoli otterrei
\begin{gather*}
    \chi^{2} = \sum \frac{(y_i - A - Bx \dots)}{\sigma_{y_i}^{2} }  
\end{gather*}
Procedendo a fare la derivata prima rispetto ad ogni variabile $A$, $B$, $\dots$ 
allora è possibile risolverlo . \\
Se le funzioni non fossero lineari nemmeno nei parametri allora dovrei provare
a linearizzare tutta la funzione attraverso il metodo grafico riscrivendo
parti della funzione in funzione di $Y$ e $X$. 
\begin{gather*}
    f(x) = \alpha e^{\beta x} \\
    \ln y = \ln \alpha + \beta x \\
    Y = \ln y, A = \ln \alpha, B = \beta, X = x \\
    Y = A + BX.
\end{gather*}
Se le incertezze sono uguali allora si utilizza il metodo dei minimi quadrati:
\begin{gather*}
    \sigma_{y_i} = \left|\frac{d \ \ln y_i}{d y_i} \right| \sigma_{y_i} = \frac{1}{y_i}\sigma_{y_i} 
\end{gather*}

\chapter{Distribuzione binomiale}
La distribuzione binomiale spesso è associata con la probabilità
di trovare l'asso in un dado, ossia avere un certo numero di misure ripetute.
La distribuzione di Poisson è quella che viene fuori da eventi casuali con
tempo caratteristico ben definito (per esempio l'emissione spontanea
di fotoni da parte di atomi). Nel caso di un dado a sei facce si 
elencano alcuni casi:
\begin{enumerate}
    \item La probabilità di trovare una faccia specifica è proprio $\frac{1}{6}$ e che non
        esca quella faccia è la complementare $p' = \frac{5}{6}$. 
    \item La probabilità di trovare tre facce uguali in tre lanci ripetuti e consecutivi
    è proprio $\frac{1}{6^{3}}$.
    \item La probabilità di trovare 2 facce uguali in tre lanci è $ppp' + pp'p + p'pp = 3p^{2}p' $.
    Poiché ho tre disposizioni possibili di risultati. 
\end{enumerate}
Nel caso generale io chiamo $n$ il numero complessivo di prove ed il numero di
successi come $\nu$ e quindi:
\begin{gather*}
    P(\nu \ \text{successi in n prove}) =
\end{gather*}
\begin{align}
   b_{n, p}(\nu) = \frac{n!}{\nu! (n - v)!} \cdot  p^{\nu} q^{n - \nu}  
\end{align}
Ho una distribuzione binomiale in funzione di $\nu$ con n e p fissati. (dove $q = 1 -p$). Questa
funzione non è molto diversa da una gaussiana, ma è una funzione discreta
perché è definita nei naturali. Il disaccordo tra la binomiale e la gaussiana
si riduce al crescere di n e quando $n \to \infty $ è proprio una gaussiana.
Verifico:
\begin{align}
    \bar{\nu} &= \frac{1}{N}\sum_{i = 0}^{n} \nu_i = np\\
    \sigma_{\nu} &= \sqrt{\sum_{i = 0}^{n} \frac{(\nu_i - \bar{\nu} )^{2} }{N-1}} = \sqrt{np (1 - p)} 
\end{align} 

\section{Approssimazione della binomiale con la gaussiana}
La distribuzione gaussiana ci permette di approssimare la funzione binomiale. A livello teorico
sono la stessa curva ma la binomiale tratta valori discreti mentre la Gaussiana tratta valori continui. 
La distribuzione binomiale $B_{n, p}(\nu)$ è ben approssimata per valori di n molto grande dalla
Gaussiana con la stessa media e la stessa deviazione standard:
\begin{align}
    B_{n, p}(\nu) &\approx G_{X, \sigma} (\nu) \\
    X = np \qquad& \sigma = \sqrt{np(1 - p)} 
\end{align}

\section{Distribuzione di Gauss per gli errori casuali}
Supponendo di avere solo errori causali (i sistematici sono trascurabili) della
stessa entità $\epsilon$, allora il nostro valore fluttuerà tra: $x = X - n\epsilon$
e $x = X + n\epsilon$. Se capita che $\nu$ sorgenti diano errori positivi e $(n  - \nu)$
errori negativi allora il risultato sarà:
\begin{gather*}
    x = X + \nu \epsilon - (n - \nu)\epsilon \\
    = X + (2\nu - n) \epsilon
\end{gather*} 
La probabilità di questo risultato ora è proprio la binomiale:
\begin{gather*}
    P(\nu \ errori \ positivi) = B_{n, 1/2} (\nu)
\end{gather*}
Se il numero di sorgenti n è grande e $\epsilon$ è sufficientemente piccola, allora
le nostre misure sono distribuite normalmente e quindi per essere più precisi $\sigma_\nu
= \sqrt{np(-1p)} = \sqrt{\frac{n}{4}}$. La deviazione standard è quindi $\sigma_x = 2\epsilon\sigma_\nu$.
Se ora $n \to \infty$ e $\epsilon \to \infty $ allora la binomiale diventa la gaussiana.

\chapter{Distribuzione di Poisson}
La distribuzione di Poisson è una distribuzione che si utilizza
molto per i decadimenti radioattivi in quanto si basa su eventi casuali
con un certo intervallo di tempo ben determinato. Ci permette di determinare
come fluttuano i successi in un dato bin quando ripeto le stesse misure.
Quando si esegue un plot di dati con $n$ molto grande e $p$ molto piccola, allora
una distribuzione binomiale è approssimabile ad una funzione più semplice chiamata distribuzione
di Poisson, la quale è definita come:
\begin{align}
    P_{\mu}(\nu) = e^{-\mu} \cdot  \frac{\mu^{\nu} }{\nu!} 
\end{align}
Per cui si ottiene: 
\begin{gather*}
    \bar{\nu} = \mu , \qquad \sigma_{\nu} = \sqrt{\mu} = \sqrt{\mu}   
\end{gather*}

$\bar{\nu}$ è proprio il valore della Gaussiana teorica che io costruisco
seguendo proprio la distribuzione di Poisson che ottengo misurandolo
dai dati. L'errore sul valore di $\bar{\nu}$ è $\sqrt{\bar{\nu}  }$.

\section{$\mu$ come conteggio medio atteso}
Perché $\mu = \bar{\nu}$? Questo perché 
\begin{gather*}
    \bar{\nu} = \sum_{\nu = 0}^{\infty } \nu P_{\mu}(\nu) = \sum_{\nu = 0}^{\infty } \nu e^{-\mu} \frac{\mu^{\nu} }{\nu!} 
\end{gather*}
Che diventa:
\begin{gather*}
    \bar{\nu} = \mu e^{-\mu} \sum_{\nu = 1}^{\infty } \frac{\mu^{\nu - 1} }{(\nu - 1)!}    
\end{gather*}
Si nota che nella serie vi è proprio lo sviluppo di taylor di $e^{\mu}$ per cui:
\begin{align}
    \bar{\nu} = \mu 
\end{align} 
Il parametro $\mu$ è proprio il numero medio di conteggi atteso. Talvolta si può ottenere 
il tasso medio R con cui si verificano gli eventi:
\begin{gather*}
    \mu = \text{tasso} \times \text{tempo} = RT
\end{gather*}

\section{Deviazione standard}
La distribuzione di Poisson fornisce la probabilità di ottenere un risultato $\nu$ in un esperimento
in cui si conteggiano eventi che avvengono casualmente ma ad un tasso medio definito. La
deviazione standard di qualunque distribuzione è proprio la radice quadrat della media degli scarti:
\begin{align}
    \sigma_\nu^{2} = \bar{\nu^{2}} - \bar{\nu}^{2}    
\end{align}
Quindi si ottiene che:
\begin{align}
    \sigma_\nu = \sqrt{\mu} 
\end{align}

\section{Approssimazione di Poisson con la Gaussiana}
Così come la binomiale, la distribuzione di Poisson dà risultati per valori discreti
ed è definita con un solo parametro, non è simmetrica rispetto ad un valore e non ha una
forma a campana. Per valori di $\mu \to \infty$ la distribuzione di Poisson tende
ad assumere la forma di una Gaussiana così come la binomiale per grandi valori dei parametri.

\section{Rimuovere il rumore di fondo}
Nel caso del conteggio delle radiazioni da un certo campione, spesso il rumore di fondo
(in questo caso la radiazione di altri campioni vicini o dei raggi cosmici), potrebbe alterare
i risultati del nostro esperimento. Supponendo di contare $\nu_{tot}$ di eventi in un tempo
$T_{tot}$ e poi $\nu_{fondo}$ eventi di sottofondo in un tempo $T_{fondo}$. Ovviamente non
si sottrae direttamente ma dobbiamo prima calcolare i tassi:
\begin{gather*}
    R_{tot} = \frac{\nu_{tot}}{T_{tot}} \qquad R_{fondo} = \frac{\nu_{fondo}}{T_{fondo}}
\end{gather*}
e poi calcolare il tasso come differenza:
\begin{gather*}
    R_{sorg} = R_{tot} - R_{fondo}
\end{gather*}
Le incertezze nei tassi di errori vanno calcolate con la propagazione degli errori e non
con il metodo della radice quadrata: assumendo $\nu_{tot} = \mu \pm \sqrt{\mu}$ allora l'errore
sui tassi è:
\begin{gather*}
    R_{tot} = \frac{\nu_{tot}}{T_{tot}} \pm \frac{\sqrt{\mu}}{T_{tot}} \\
    R_{fondo} = \frac{\nu_{fondo}}{T_{fondo}} \pm \frac{\sqrt{\mu}}{T_{fondo}}
\end{gather*} 
Nella sottrazione si combinano in quadratura in quanto certamente casuali ed indipendenti.

\chapter{Il test del chi-quadrato per una distribuzione}
\section{Cosa è il Chi-quadrato}
Come determino se la distribuzione risultante è consistente con quella teorica
attesa? Utilizzo il chi-quadrato che mi permette di determinare la consistenza
dei risultati attesi. Se facessi $n$ misure per le quali conosco $\sigma$ posso
calcolare i valori attesi e la deviazione standard e quindi definisco:
\begin{align}
    \chi^{2} = \sum\left(\frac{\text{valore osservato - valore atteso}}{deviazione}\right)^{2}  
\end{align}

Se la distribuzione di $x$ è $\chi^{2} + n$, se conosco allora
$\chi^{2}$ dovrebbe allora essere dell'ordine di $n$. Se $\chi^{2} >> n$
allora è molto probabilmente sbagliata. Il procedimento per la 
scelta degli intervalli intermedi viene calcolato dipende dal fatto che la grandezza
sia continua o lineare. \\
Misure continue: se la funzione è continua allora la probabilità che un valore cada nella curva è
esattamente l'integrale tra un certo valore $a$ e $b$. IN generale l'area totale dell'integrale
non sarai mai superiore ad $1$.
Misure di una variabile continua: si  può verificare l'accordo tra una distribuzione reale ed una attesa attraverso
l'utilizzo del $\chi^{2}$.

\section{I gradi di libertà}
In generale il numero di gradi di libertà ci permette di definire
il numero di dati "Non impiegati" nei calcoli e si calcola come
il numero di dati osservato meno quello dei dati calcolati dai dati per determinare
i numeri attesi. 

\end{document}
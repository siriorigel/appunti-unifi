\documentclass[a4paper, oneside]{article}
\usepackage{graphicx}
\usepackage{amsthm}
\usepackage{amsmath}
\usepackage{amssymb}
\usepackage[a4paper,
            bindingoffset=0.2in,
            left=2cm,
            right=2cm,
            top=2cm,
            bottom=2cm,
            footskip=.25in]{geometry}
\usepackage[italian]{babel}
\usepackage{pgfplots}
\usepackage{tabularx}
\usepackage{tikz}
\usepackage{wrapfig}
\usepackage{color}
\usepackage[d]{esvect}
\usepackage{chemfig}
\usepackage{mhchem}
\definecolor{page}{rgb}{0.129,0.157,0.212}
\pagecolor{page}
\color{white}
\graphicspath{ {./images/} }
\usetikzlibrary{shapes.geometric}
\usetikzlibrary{datavisualization}
\usetikzlibrary{datavisualization.formats.functions}
\usetikzlibrary{patterns}
\pgfplotsset{width=10cm,compat=1.18}

\title{}
\author{Tommaso Miliani}
\date{}

\begin{document}
\newtheoremstyle{theoremEnv}
                {}          % Space above
                {}          % Space below
                {\slshape}  % Body font
                {}          % Indent amount
                {\bfseries} % Head font
                {.}         % Punctuation after head
                {\newline}  % Space after theorem head
                {}          % Theorem head spec
\theoremstyle{theoremEnv}

\newtheorem{definition}{Definizione}[section]
\newtheorem{theorem}{Teorema}[section]
\newtheorem{lemma}{Proposizione}[section]
\newtheorem{observation}{Osservazione}[section]
\newtheorem{corollary}{Corollario}[theorem]
\newtheorem{example}{Esempio}[section]
\newtheorem{remark}{Enunciato}[section]

\maketitle

\section*{Compito Novembre}
\section{Esercizio 1}
\subsection{Punto 1}
Si può imporre la condizione di statica, ma si prova che le reazioni 
vincolari che agiscono sul blocco sono
\begin{gather*}
    m_1g\cos\alpha \qquad m_2g\cos\alpha
\end{gather*}
Si trova dunque $F_a$, imponendo dunque la condizione di attrito statico.


\subsection{Punto 2}
Si studiano tutte le componenti del sistema in modo indipendente,
dunque si impostano le equazioni di moto per tutti i corpi del sistema,
dunque i due gradi di libertà $x_1$ e $x_2$ si evolvono in modo 
indipendente ma il centro di massa ha un solo grado di libertà. A questo punto 
il centro di massa non si muove di moto uniformemente accelerato, anche se non 
sono in grado di dire che tipo di moto hanno i due corpi da soli. Il centro di massa compie dunque 
un moto armonico.
Scrivere $x_{cm}$ e scrivere allora in funzione di $x_{cm}$ il moto 
di $x_1$ con due termini: uno dell'oscillazione del moto e si verifica 
anche che $x_1$ non riesce a salire. Trattare questo corpo come un 
corpo unico in modo tale che la molla non oscilli è un errore. I due 
gradi di libertà infatti possono essere scomposti in due gradi di libertà indipendenti tra di loro: 
la stessa cosa accade anche alla variabiel $\Delta$ e si risolvono 
dopo in modo scollegato i due e infine si unisce il risultato. 

\subsection{Punto 3}
Nel sistema di riferimento il blocco $M$ si vuole fisso

\section{Esercizio 2}
\subsection{Punto 1}
Se si ignora il corpo di sotto che trasla, i due rulli fanno rotolamento puro rispetto
alla tavola sopra, e dunque con la formula di rotolamento puro, il punto di contatto sulla
tavola che scorre sotto ha velocità doppia, e dunque istantaneamente il punto di
contatto è fisso e dunque la velocità della sbarretta è esattamente due volte la velocità
del centro di massa del disco. Si trova dunque
\begin{gather*}
    V_E = 2R\omega \qquad  V_A = V_B = R\omega
\end{gather*}


\subsection{Punto 2}
Si studia il moto capendo le condizioni iniziali: 
\begin{gather*}
    x_C(0) = x_C(A) = \frac{L}{2} 
\end{gather*}
SI scrive dunque l'energia in funzione al pezzo che scende, alla sbarretta
che trasla e la rotazione dei due dischi (l'energia del centro di masssa no 
è inclusa in quanto la sto calcolando rispetto ad un polo fisso).
Trovata la $x$ di equilibrio che coincide con $E$ derivando l'energia meccanica,
si utilizzano le condizioni iniziali: integrando le espressioni della velocità si 
ottengono le condizioni per le quali durante l'oscillazione si abbia velocità
massima per $A$ imponendo che A non vada mai oltre il punto $E$. 
\begin{gather*}
    x_A \geq x_E \qquad x = x_{MAX}
\end{gather*}
Se lo spostamento è massimo, tutto ritorna allo stato di quiete e si poteva ugualiare
l'energia potenenziale iniziale a quella finale. 

\subsection{Punto 3}
IL modulo finale della velocità dei rulli si trova attraverso la 
soluzione del moto armonico semplice, dunque facendo i conti 
si ha che 
\begin{gather*}
    \omega = \frac{2}{5}\sqrt{\frac{g}{L}} 
\end{gather*}
Infine il
\begin{gather*}
    t = \frac{5}{4}\pi\sqrt{\frac{L}{g}} 
\end{gather*}

\subsection{Punto 4}
In funzione del tempo le tensioni si trovano attraverso le forze che 
agiscono sulle tavole superiori: ossia le forze di tensione, la reazione 
vincolare dei due dischi e la forza di attrito che permette il rotolamento
e la forza peso. Dato che sono vincoli unilateri, i loro moduli devono essere
positivi: si studia dunque le due forze di attrito:
\begin{gather*}
    T = f_A + f_B \qquad T > 0 \left\{\begin{array}{l}
        T_C = 0 \\
        T_D = \left| T \right| 
    \end{array}\right.\qquad T < 0 \left\{\begin{array}{l}
            T_C = \left| T \right| \\
            T_D = 0 
        \end{array}\right. 
\end{gather*}
Ossia sono i due casi in cui una delle due tira e l'altra no (ossia
sono impercettibilmente tese solamente per tenere ferma la tavola sopra).
Sui singoli dischi adesso agiscono: le forze peso e quelle di attrito. 

\subsection{Punto 5}
La condizione di non strisciamento deriva esattamente dalla condizione di 
rotolamento puro sui dischi:
\begin{gather*}
    F_{A_1} = \frac{3}{4}mR\dot{\omega}
\end{gather*}
Le due $\vv{N}$ dei dischi non sono uguali e sono dirette verso il
basso rispetto ai punti di contatti. Studiando l'equazione cardinale
della tavola sopra:
\begin{gather*}
    \vv{N_A} + \vv{N_B} = M\vv{g}   
\end{gather*} 
Per capire se sono diversi si studiano i moemnti rispetto ad un polo qualsiasi
imponendo la condizione di non rotolamento: dato che un disco 
si avvicina e l'altro si allontana, le due reazioni vincolari devono necessariamente essere
diverse



\end{document}

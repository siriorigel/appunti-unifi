\documentclass[a4paper, oneside]{book}
\usepackage{graphicx}
\usepackage{amsthm}
\usepackage{amsmath}
\usepackage[a4paper,
            bindingoffset=0.2in,
            left=2cm,
            right=2cm,
            top=2cm,
            bottom=2cm,
            footskip=.25in]{geometry}
\usepackage[italian]{babel}
\usepackage{pgfplots}
\usepackage{tabularx}
\usepackage{tikz}
\usepackage{wrapfig}
\usepackage{color}
\usepackage[h]{esvect}
\definecolor{page}{rgb}{0.129,0.157,0.212}
\pagecolor{page}
\color{white}
\graphicspath{ {./images/} }
\usetikzlibrary{shapes.geometric}
\usetikzlibrary{datavisualization}
\usetikzlibrary{datavisualization.formats.functions}
\pgfplotsset{width=10cm,compat=1.9}

\title{Appunti di Fisica I}
\author{Tommaso Miliani}
\date{2024/2025}

\begin{document}

\maketitle

\tableofcontents

\chapter*{Le grandezze fisiche}
\section*{Grandezze e loro misurazione}
L'insieme delle operazioni volte ad associare un numero ad una grandezza
fisica si chiama \textbf{misurazione} e rende possibile e non ambigua
ogni valutazione di uguaglianza o disuguaglianza tra due grandezze
della stessa specie.
Ogni misura, seppur ottenuta con strumenti diversi, ci consente di ottenere
un rapporto con la misura di un altra scelta come \emph{unità di misura}. 
Le misure in fisica si definiscono in \textbf{modo operativo} ossia attraverso
l'utilizzo di molteplici sistemi di misura volti a dare la medesima misura.

\section*{Metodi di misurazione}
Il modo per ottenere un numero associato ad una grandezza è quello di
confrontare \emph{direttamente} la grandezza con l'unità di misura, in questo
modo si ottiene il rapporto tra le due che definisce la misura effettiva.
La misurazione diretta richiede innanzitutto l'introduzione di un criterio di
confronto per definire l'uguaglianza, un criterio per la def. della somma
e la scelta di un campione come unità di misura. Un'altro tipo di misurazione
è quella \textbf{strumentale}, che utilizza strumenti già precedentemente
tarati. Infine c'è anche una misurazione \textbf{indiretta}, attraverso
le relazioni tra grandezze fisiche.

\section*{Dimensioni delle grandezze fisiche}
Le grandezze fisiche si possono classificare come \textbf{fondamentali}
o \textbf{derivate}. Tradizionalmente nella meccanica le grandezze di
lunghezza (L), massa (M) e tempo (T)  sono definite come fondamentali e 
le altre derivate da esse. \\
Indicando con $[L], [M], [T]$ le dimensioni delle grandezze fisiche,
si ottiene che esse sono in relazione nella seguente maniera:
\[
    [L] = [L^1M^0T^0]; \qquad [M] = [L^0M^1T^0]; \qquad [T] = [L^0M^0T^1].
\]
Le grandezze con potenza zero non contribuiscono alle dimensioni e possono 
essere tralasciate. Alla stessa maniera si definiscono quelle derivate
come superficie e volume:
\[
    [S] = [L^2M^0T^0] = [L^2]; \qquad [V] =[L^3M^0T^0] = [L^3].
\]
Le grandezze nelle quali gli esponenti sono tutti nulli prendono il nome di 
\textbf{adimensionali}, un esempio sono gli angoli:
\[
    \alpha = \frac{s}{R} \ (arco \ e \ Raggio).
\]
In ogni legge le dimensioni di entrambi i membri devono sempre essere uguali.

\section*{Sistemi di unità di misura}
Multipli e sottomultipli dei sistemi decimali:
\begin{center}
    \begin{tabular}{|c|c|c|}
    \hline
    Ordine & Nome & Simbolo \\
    \hline
    $10$ & deca & da  \\
    \hline
    $10^2$ & etto & h \\
    \hline
    $10^{3}$ & kilo & k \\
    \hline
    $10^{6}$ & mega & M \\
    \hline
    $10^{9}$ & giga & G \\
    \hline
    $10^{12}$ & tera & T \\
    \hline
    $10^{15}$ & peta & P\\ 
    \hline
    $10^{18}$ & exa & E \\
    \hline 
    $10^{21}$ & zetta & Z\\
    \hline
    $10^{24}$ & yotta & Y\\
    \hline
\end{tabular}
\begin{tabular}{|c|c|c|}
    \hline
    Ordine & Nome & Simbolo \\
    \hline
    $10^{-1}$ & deci & d \\
    \hline
$10^{-2}$ & centi & c \\
\hline
$10^{-3}$ & milli & m \\
\hline
$10^{-6}$ & micro & $\mu$ \\
\hline
$10^{-9}$ & nano & n \\
\hline
$10^{-12}$ & pico & p \\
\hline
$10^{-15}$ & femto & f \\
\hline
$10^{-18}$ & atto & a\\
\hline
 $10^{-21}$ & zepto & z \\
 \hline
 $10^{-24}$ & yocto & y\\
 \hline
\end{tabular}
\end{center}

\section*{Tempo}
La definizione di tempo è spiegata negli appunti di chimica,
e grazie a tale definizione si è in grado di distinguere intervalli molto piccoli
di tempo da $10^{-26} -- 10^{17}s$. Tutto ciò che va sopra o sotto tali intervalli
non ha senso.
Si potrebbe supporre che il tempo sia continuo, ma la meccanica quantistica ci
spiega che in realtà molte grandezze che sembravano continue, in realtà sono
quantizzate, come la carica dell'elettrone o l'energia associata ad una radiazione.
Il tempo in fisica classica è tuttavia assoluto ed uniforme, mentre nella fisica
reale non è così.

\section*{Lunghezza}
Nella fisica classica lo spazio è \emph{euclideo}, ed il metro è definito tra
ordini di grandezza $10^{-18}--10^{26}m$, sebbene recentemente si è associato
la misura più piccola possibile nell'universo come la lunghezza
di Planck.

\section*{Massa}
Definita tra gli intervalli $10^{-31}--10^{55}kg$, la massa è in realtà doppiamente definita,
da una parte si definisce massa \emph{inerziale} e \emph{gravitazionale}.

\section*{Misure ed indeterminazione}
A livello macroscopico potremmo accontentarci di una imprecisione
dell'ordine di $10^{-6}$ metri, ed il che potrebbe anche essere
sufficiente per la maggior parte delle misure, tuttavia nel mondo
microscopico tale precisione risulterebbe essere un errore macroscopico.
Se si tentasse di misurare direttamente un elettrone, la natura
ci mostrerebbe uno strano comportamento: se si trovasse accuratamente
la sua posizione, avremmo una forte incertezza sulla sua
velocità e così anche all'inverso. Questo comportamento segue la legge di
Heisenberg:
\[
    \Delta x \Delta q_x \geq \frac{h}{4\pi}.
\]
dove $h = 6.62606896\cdot 10^{-34}Js$. A livello macroscopico l'incertezza
sulla quantità di moto è irrilevante e in moltissimi casi nemmeno misurabile
neanche dagli strumenti più sensibili.

\part{Vettori e cinematica}
\chapter{Calcolo vettoriale}
\section{Grandezze scalari e vettoriali}
I numeri come la massa, il tempo o la temperatura in fisica sono chiamate
grandezze \textbf{scalari} e si differenziano dalle grandezze \textbf{vettoriali}
poiché quest'ultime non possono solo essere descritte da un numero ma anche 
da una direzione e da un verso.
\begin{center}
   \begin{tikzpicture}
    \draw[->](0,0) -- (1,0) node[midway, above] {$\vv{v}$};
\end{tikzpicture} 
\end{center}
Un vettore ha tre caratteristiche fondamentali:
\begin{enumerate}
    \item \textbf{modulo}: ossia l'intensità;
    \item \textbf{direzione}: ossia la direzione nello spazio verso cui è orientato;
    \item \textbf{verso}: ossia il verso di tale direzione.
\end{enumerate}

I vettori possono essere sia \textbf{liberi} se non sono fissati su una loro
origine, altrimenti \textbf{applicati} quando hanno un origine ben definita.

\section{Definizioni vettoriali e operazioni base}
Due vettori sono uguali quando hanno stesso modulo, direzione e verso, sono \textbf{opposti}
se $\vv{b} = -\vv{a}$.
La somma tra due vettori gode della proprietà \textbf{commutativa} e
\textbf{associativa}, inoltre la somma avviene sommando le componenti 
dei vettori, la sottrazione è uguale alla somma di un vettore con 
l'inverso del secondo:
\begin{align*}
    &\vv{c} = \vv{a} + \vv{b} &\vv{d} = \vv{e} - \vv{f}\\
    &\begin{tikzpicture}
        \draw[->](0, 0) -- (2,0) node[midway, below] {$\vv{a}$};
        \draw[->](2, 0) -- (3, 2) node[midway, below] {$\vv{b}$};
        \draw[->, red](0,0) --(3,2) node[midway, above, red] {$\vv{c}$};
    \end{tikzpicture} 
    &\begin{tikzpicture}
        \draw[->](0, 0) -- (2, 0.5) node[midway, below] {$\vv{f}$};
        \draw[->](2, 0.5) -- (3, 3.5) node[midway, right] {$\vv{e}$};
        \draw[->](3, 3.5) -- (1, 3) node[midway, above] {-$\vv{f}$};
        \draw[->](0, 0) -- (1, 3) node[midway, left] {$\vv{e}$};
        \draw[->, red](2, 0.5) -- (1, 3) node[midway, right, red] {$\vv{e} - \vv{f}$};
    \end{tikzpicture}
\end{align*}
Il prodotto tra un vettore ed uno scalare è così definito ed ha 
le seguenti proprietà:
\begin{align*}
    &\vv{a} = \lambda \vv{v}. \\ 
    &\begin{tikzpicture}
        \draw[->](0, 0) -- (1, 0.5) node[midway, below] {$\vv{v}$};
        \draw[->](0, 0.5) -- (3, 2) node[midway, above] {3$\vv{v}$};
    \end{tikzpicture}
\end{align*}
\begin{enumerate}
    \item $\lambda (\mu) \vv{a} = \mu (\lambda \vv{a}) = (\mu \lambda) \vv{a}$;
    \item $(\lambda + \mu)\vv{a} = \lambda \vv{a} + \mu \vv{a}$;
    \item $\lambda (\vv{a} + \vv{b}) = \lambda \vv{a} + \lambda \vv{b}$;
    \item $\lambda \vv{a} = 0 \Rightarrow \lambda = 0 \ oppure \ \vv{a} = 0$.
\end{enumerate}

\section{Versori}
Il rapporto tra un vettore qualsiasi $\vv{a}$ ed il suo modulo è per definizione
un vettore adimensionale e prende il nome di \textbf{versore}:
\begin{align}
        \hat{u}_a = \frac{\vv{a}}{|a|}
\end{align}
e quindi, dato che il vettore è sia vincolato al modulo che ad una direzione
ed un verso, allora si può esprimere  un vettore in relazione al proprio versore,
il quale, indica esattamente la direzione ed il verso del vettore
\begin{align}
        \vv{a} = a \hat{u}_a
\end{align}
Così come per un vettore, possiamo esprimere 
una retta orientata come un vettore, anche una retta orientata
è caratterizzata da un versore $\hat{u}$ e coincide con il 
versore di un qualsiasi vettore parallelo alla retta stessa e con verso
concorde e quindi tutti i vettori che hanno la direzione di
tale retta si può esprimere come
\begin{gather*}
    \vv{a} = \lambda \hat{u}  
\end{gather*}
Il numero $\lambda$ è chiamato \textbf{parte scalare} di $\vv{a}$. 


\section{Scomposizione di vettori}
\begin{wrapfigure}{r}{0.30\textwidth}
    \begin{tikzpicture}
    \draw[-, dashed] (0, 0) -- (3, 4);
    \draw[-, dashed] (0, 1.5) -- (4, 0);
    \draw[->, very thick](0.87, 1.16) -- (1.9, 2.5) node[midway, left] {$\vv{a}_1$};
    \draw[->, very thick](0.87, 1.16) -- (3.3, 0.25) node[midway, below] {$\vv{a}_2$};
    \draw[->, red](0.87, 1.16) -- (4.15, 1.65) node[midway, above, red] {$\vv{a}$};
    \draw[-, dashed](1.9, 2.5) -- (4.15, 1.65);
    \draw[-, dashed](3.3, 0.25) -- (4.15, 1.65);
    \end{tikzpicture}
\end{wrapfigure}
Un qualsiasi vettore $\vv{a}$ può essere espresso come somma dei versori
$\hat{u}_1$ e $\hat{u}_2$ se è complanare al piano $\sigma$ individuato dai
versori stessi, per cui:
\[
  \vv{a} = \vv{a}_1 + \vv{a}_2 = a_1 \hat{u}_1 + a_2 \hat{u}_2
\]
Da notare che i versori si chiamano \emph{i componenti} e non 
\emph{le componenti}, in quanto le componenti di un vettore sono le parti
scalari che lo compongono, mentre i componenti sono i versori
del vettore. Se $\vv{a}$ non giacesse sul piano, allora sarebbe necessario un versore
non complanare. 

\section{Prodotto scalare}
Il prodotto scalare è anche definito come prodotto interno tra due vettori
poiché opera sullo stesso piano di appartenenza dei vettori utilizzando l'angolo
tra di essi:
\begin{align}
        \vv{a} \cdot \vv{b} = ab \cos(\theta).
\end{align}
Le proprietà del prodotto scalare sono le seguenti:
\begin{enumerate}
    \item commutativa (simmetrica) ($\vv{a} \cdot \vv{b} = \vv{b} \cdot  \vv{a}$  );
    \item distributiva bilineare ($\vv{a} \cdot  (\vv{b} + \vv{c}  ) = \vv{a} \cdot  \vv{b} + \vv{a} \cdot \vv{c}$   );
    \item $\vv{a} \cdot (\lambda \vv{b}) = \lambda \vv{a} \cdot \vv{b}$.
\end{enumerate}
Il prodotto scalare tra due vettori è quindi considerato come il prodotto del modulo di 
uno qualsiasi di essi per la proiezione ortogonale dell'altro sul primo. Dimostrazioni
delle tre proprietà:
\begin{enumerate}
    \item La prima è ovvia in quanto 
    \begin{gather*}
        \vv{a} \cdot  \vv{b} = ab\cos \theta \\
        \vv{b} \cdot  \vv{a} = ba\cos\theta \\     
    \end{gather*}
    Che sono ovviamente uguali.
    \item Indicando con $\hat{u}_a$ il versore del vettore $\vv{a}$ 
    dobbiamo dimostrare allora che
    \begin{gather*}
        a((\vv{b} + \vv{c}  ) \cdot  \hat{u}_a ) = ab_a + ac_a \ \Longrightarrow \ (\vv{b} + \vv{c}  ) \cdot \hat{u}_a = b_a + c_a 
    \end{gather*} 
    ossia è dimostrata.
    \item \begin{gather*}
        \vv{a} \cdot  \lambda \vv{b} = a \cdot \lambda b \cos\theta \\
        \lambda \vv{a} \cdot  \vv{b} = \lambda a b \cos\theta    
    \end{gather*}
    Che sono ovviamente uguali.
\end{enumerate}

\section{Prodotto vettoriale}
Il prodotto vettoriale invece è il prodotto di un vettore per un altro e dà come
risultato un terzo vettore ortogonale al piano individuato tra i due.
\begin{align}
        \vv{c} = \vv{a} \times \vv{b}.
\end{align}
Ha le seguenti caratteristiche:
\begin{enumerate}
    \item Il modulo del nuovo vettore è $c = ab|\sin(\theta)|$;
    \item Direzione $\perp$ al piano di $\vv{a}, \vv{b}$;
    \item Verso nel quale disporsi determinato dalla rotazione antioraria
    del primo vettore sul secondo (se $< \pi$), determinato con la regola della mano destra.
\end{enumerate}
Questa definizione di prodotto vettoriale fissa una direzione convenzionale
per il nuovo vettore $\vv{c}$: tale convenzione è chiamata \textbf{regola della mano destra}
e si può affermare allora che il verso del prodotto vettoriale (se l'incide è $\vv{a}$ e $\vv{b}$ è il medio )
è dato dal pollice. Adesso, dalla definizione di prodotto vettoriale è implica
\begin{gather*}
    \vv{a} \times \vv{a} = 0; \qquad \vv{a} \times (\lambda \vv{a}  ) = 0   
\end{gather*}
E quindi valgono le seguenti proprietà:
\begin{enumerate}
    \item anticommutativa: $\vv{a} \times \vv{b} = -\vv{b} \times \vv{a}$;
    \item distributiva: $\vv{a} \times (\vv{b} + \vv{c}) = \vv{a} \times \vv{b} + \vv{a} \times \vv{c}$;
    \item $\vv{a} \times (\lambda \vv{b}) = \lambda (\vv{a} \times \vv{b})$.
\end{enumerate}
Le cui dimostrazioni sono le seguenti:
\begin{enumerate}
    \item La prima deriva dalla convenzione della rotazione imposta:
    infatti, scambiando l'ordine dei vettori, per ottenere un risultato
    uguale dobbiamo invertire la rotazione della mano e quindi si inverte il verso. 
    \item E' ovvia
    \item \begin{gather*}
        c = a\lambda b|\sin\theta| \\
        c = \lambda (ab \sin\theta)
    \end{gather*}
    che sono uguali. 
\end{enumerate}

\section{Rappresentazione cartesiana ortogonale}
Il sistema di coordinate più utilizzato consiste nell'utilizzo di tre rette
orientate passanti per uno stesso punto O chiamato origine e $\perp$ tra loro
$x, y, z$ ed i cui versori sono $\hat{i}, \hat{j}, \hat{k}$: ossia la terna 
cartesiana destra. Una retta orientata sulla 
quale sia stato definito un punto specifico detto \emph{origine} prende il nome di
\textbf{asse}, e quindi le tre rette e lo spazio che identificano prende il
nome di \emph{\textbf{terna cartesiana ortogonale}}. 
\[
    \vv{v} = v_x \hat{i} + v_y \hat{j} + v_z \hat{k}     \qquad
    \vv{v} = \left\{\begin{array}{l}
        v_x = \vv{v} \cdot \hat{i} = v \cos(\alpha) \\
        v_y = \vv{v} \cdot \hat{j} = v \cos(\beta) \\
        v_z = \vv{v} \cdot \hat{k} = v \cos(\gamma)
    \end{array}\right.
\]
Dove $\alpha, \beta, \gamma$ sono gli angoli che il vettore forma rispettivamente
con l'asse $x, y, z$.

\section{Equivalenza tra rappresentazioni vettoriali}
Col teorema di Pitagora si ricava che:
\begin{align}
        v = \sqrt{v_x^2 + v_y^2 + v_z^2}
\end{align}
e quindi il versore di questo vettore sarà
\[
    \hat{u}_v = \cos(\alpha)\hat{i} + \cos(\beta)\hat{j} + \cos(\gamma)\hat{k}.
\]
Si può ricavare, dalle espressioni cartesiana ortogonale le espressioni
dei \textbf{coseni direttori}, ossia le espressioni per il coseno degli
angoli che forma ogni componente con ogni asse. 
\begin{align}
    \cos\alpha = \frac{v_x}{\sqrt{v_x^2 + v_y^2 + v_z^2}} \qquad     \cos\beta = \frac{v_y}{\sqrt{v_x^2 + v_y^2 + v_z^2}} \qquad     \cos\gamma = \frac{v_z}{\sqrt{v_x^2 + v_y^2 + v_z^2}}
\end{align}
e quindi si ricava
\begin{gather*}
    \cos^{2}\alpha + \cos^{2}\beta + \cos^{2}\gamma = 1   
\end{gather*}

\section{Espressioni cartesiane delle operazioni tra vettori}
Dati due vettori, espressi in componenti cartesiane ortogonali, $\vv{v}
= v_x \hat{i} + v_y \hat{j} + v_z \hat{k}$ e $\vv{w} = w_x \hat{i} +
w_y \hat{j} + w_z \hat{k}$, valgono le seguenti:
\begin{enumerate}
    \item $\vv{v} + \vv{w} = (v_x+ w_x)\hat{i} + (v_y+ w_y)\hat{j} + (v_z+ w_z)\hat{k}$;
    \item $\vv{v} = \vv{w} \Leftrightarrow v_x = w_x, \quad v_y = w_y, \quad v_z = w_z$;
    \item $\lambda\vv{v} = \lambda v_x \hat{i} + \lambda v_y \hat{j} + \lambda v_z \hat{k}$;
    \item $\vv{v} \cdot \vv{w} = v_x w_x + v_y w_y + v_z w_z$;
    \item  \begin{gather*}
        \vv{v} \cdot  \vv{v} = v^{2} = v_x^{2} + v_y^{2} + v_z^{2}      
    \end{gather*}
    \item \[
        \vv{v} \times \vv{w} = \det\left(\begin{array}{c c c }
            \hat{i} & \hat{j} & \hat{k} \\
            v_x & v_y & v_z \\
            w_x & w_y & w_z
        \end{array}\right) = (v_yw_z - v_zw_y)\hat{i} + (v_zw_x-v_xw_z)\hat{j} +(v_xw_y -v_yw_x)\hat{k}.
    \]
\end{enumerate}

\section{Derivazione di vettori}
Prendendo un vettore in funzione di qualche variabile tipo: $\vv{w} = \vv{w}(t)$, la sua
derivata sarà: 
\begin{align}
        \frac{d\vv{w}}{dt} = \lim_{\Delta t \to 0}\frac{\vv{w}(t + \Delta t) - \vv{w}(t)}{\Delta t}
    = \lim_{\Delta t \to 0} \frac{\Delta \vv{w}}{\Delta t}.
\end{align}
Se si utilizza un sistema di coordinate cartesiane ortogonali i cui versori
non dipendono dal tempo, la funzione vettoriale
\begin{gather*}
    \vv{w} (t) = w_x\hat{i} + w_y \hat{j} + w_z\hat{k}    
\end{gather*}
dove la dipendenza dalla variabile tempo è univocamente determinata
da quelle delle tre componenti scalari. In questo caso la derivazione della
funzione vettoriale si riduce alla derivata rispetto al tempo delle componenti. 
Si hanno allora le seguenti proprietà con relative dimostrazioni
\begin{enumerate}
    \item \begin{gather*}
        \frac{d}{dt} (\vv{w} (t) + \vv{v} (t) ) = \frac{d\vv{w}(t) }{dt} + \frac{d\vv{v}(t) }{dt};
    \end{gather*}
    \item \begin{gather*}
        \frac{d}{dt}(\lambda (t) \vv{w}(t) ) = \frac{d\lambda(t)}{dt}\vv{w}(t) + \lambda(t) \frac{d\vv{w}(t) }{dt} ;
    \end{gather*}
    \item \begin{gather*}
        \frac{d}{dt}(\vv{w}(t) \cdot  \vv{v}(t)  ) = \frac{d\vv{w}(t) }{dt} \cdot  \vv{v}(t) + \vv{w}(t)\cdot \frac{d\vv{v}(t) }{dt}  ;
    \end{gather*}
    \item \begin{gather*}
        \frac{d}{dt}(\vv{w}(t) \times \vv{v}(t)  ) = \frac{d\vv{w}(t) }{dt} \times \vv{v}(t) + \vv{w}(t) \times \frac{d\vv{v}(t) }{dt}  ;
    \end{gather*}
    \item \begin{gather*}
        \vv{w}(t) = \vv{w}(s(t)) \ \Longrightarrow \ \frac{d\vv{w}(t) }{dt} = \frac{d\vv{w}(s) }{ds} \frac{ds}{dt}  ;
    \end{gather*}
    \item \begin{gather*}
        d\vv{w}(t) = \frac{d\vv{w}(t) }{dt}dt 
    \end{gather*}
    E' il differenziale di $\vv{w}$ il quale rappresenta l'incremento della funzione
    nell'intervallo infinitesimo $dt$. 
\end{enumerate}


\section{Derivate di versori e di vettori}
\begin{wrapfigure}{r}{0.35\textwidth}
    \begin{tikzpicture}
        \draw[->](0, 0) -- (4, 0) node[midway, below] {$\hat{u}(t + \Delta t)$};
        \draw[->](0, 0) -- (2, 3.4) node[midway, left] {$\hat{u}(t)$};
        \draw[-, dashed] (0, 0) -- (3, 1.7);
        \draw[->](0.55, 0.95).. controls (0.9, 0.5).. (1, 0) node[midway, right] {$\Delta \phi$};
        \draw[->, red](2, 3.4) -- (4, 0) node[midway, right] {$\Delta \hat{u}$};
        \draw[->, red](2, 3.4) -- (5, 2) node[midway, above] {$\hat{n}$};
    \end{tikzpicture}
\end{wrapfigure}
Sia $\Delta\phi$ l'angolo di rotazione corrispondente ad un $\Delta t$ positivo
e quindi, la derivata del versore $\hat{u}$ di un vettore $\vv{v}$, per $\Delta t \to 0$
e $\Delta \phi \to 0$, si ha che $\Delta \hat{u}$ tende ad essere $\perp$ rispetto
a $\hat{u}(t)$ e come verso il verso dalla parte in cui ruota $\hat{u}$, ossia la direzione di $\hat{n}$
e per quanto riguarda il modulo si osserva che, dato che $u = 1$, $|\Delta \hat{u} | = 2u\sin\frac{\Delta \phi}{2}$:
\[
    \frac{|\Delta \hat{u}|}{\Delta t} = \frac{|\Delta \hat{u}|}{\Delta \phi}
    \frac{\Delta \phi}{\Delta t} = \frac{\sin\frac{\Delta \phi}{2}}{\frac{\Delta \phi}{2}}
    \frac{\Delta \phi}{\Delta t}.
\]
Passando ora al limite per $\Delta t \to 0$ si ha:
\[  
    \lim_{\Delta t \to 0} \frac{|\Delta \hat{u}|}{\Delta t} =
    \lim_{\Delta \phi \to 0} \frac{\sin\frac{\Delta \phi}{2}}{\frac{\Delta \phi}{2}}
    \lim_{\Delta t \to 0} \frac{\Delta \phi}{\Delta t} = \frac{d\phi}{dt}.
\]

Si ottiene dunque la seguente relazione, in quanto il versore $\hat{n}$ è il versore
perpendicolare al versore $\hat{u}$ e che dà il verso al versore e giace sul piano
$\sigma$ contenente $\hat{u}(t)$ e $\hat{u}(t + dt)$ perpendicolare a $\hat{u}$ 
dalla parte di $\hat{u}(t + dt)$ e $\frac{d\phi}{dt}$ è una quantità positiva.
$\Delta \hat{u}$:
\[
    \frac{d\hat{u}}{dt} = \frac{d\phi}{dt}\hat{n}.
\]

\begin{wrapfigure}{r}{0.4\textwidth}
    \centering
    \caption{Direzione del vettore $\vec{w}$ }
\begin{tikzpicture}
    \draw[-] (0, 0) -- (1, 3);
    \draw[-] (0, 0) -- (4, 0);
    \draw[-] (1, 3) -- (5, 3);
    \draw[-] (4, 0) -- (5, 3);
    \draw[->] (1.5,1.5) -- (3, 1.5) node[midway, below] {$\hat{u}(t)$};
    \draw[->] (1.5,1.5) -- (2.8,2.5) node [midway, right] {$\hat{u}(t + \Delta t)$};
    \draw[->, red] (1.5, 1.5)--(1.5, 3.2) node[midway, right] {$\vv{w}$};
\end{tikzpicture}\end{wrapfigure}



Per caratterizzare la rotazione di $\hat{u}$ si utilizza invece un'altro vettore
$\vv{w}$ che ha modulo $\frac{d\phi}{dt}$, direzione $\perp$ al piano dei due vettori
e verso tale da individuare la rotazione del versore di partenza: $\vv{w} =
\hat{u} \times \hat{n}$. Si dimostra quindi che:
\begin{align}
    \frac{d\hat{u}}{dt} = \frac{\Delta \phi}{dt}\hat{n} = \vv{w} \times \hat{u}.
\end{align}
Questa è vera perché per le definizioni date $\hat{w}$ ed il versore $\vv{w}$ si ha che
\begin{gather*}
    \vv{w} = \frac{d\phi}{dt}\hat{u}_{w} \qquad  \hat{u}_{w} = \hat{u} \times \hat{n}   
\end{gather*}  
I versori $\hat{u}, \hat{n}, \hat{u}_{w}$ costituiscono una terna ortogonale destra. Di conseguenza 
si ha che $\hat{u}_{w} \times \hat{u} = \hat{n}$ e quindi si ha che
\begin{gather*}
    \vv{w} \times \hat{u} = \left(\frac{d\phi}{dt}\right) \hat{u}_w \times \hat{u} = \left(\frac{d\phi}{dt}\right) \hat{n} = \frac{d\hat{u} }{dt}     
\end{gather*}      
Si osservi che moltiplicando vettorialmente tutti i membri per $\hat{u}$ si ottiene facilmente che
\begin{gather*}
    \vv{w} = \hat{u} \times \frac{d\hat{u} }{dt}  
\end{gather*}
Preso però un generico vettore $\vv{z}(t)$ anche il modulo può dipendere dal tempo e
dunque purché sia definito il suo versore ed il suo modulo non sia nullo allora
\[
    \frac{d\vv{z}}{dt} = \frac{d}{dt}(z\hat{u}_z) = \frac{dz}{dt} \hat{u}_z +
    z\frac{d\hat{u}_z}{dt} = \frac{dz}{dt}\hat{u}_z + z\vv{w}\times \hat{u}_z
\]
quindi:
\begin{align}
    \frac{d\vv{z}}{dt} = \frac{dw}{dt}\hat{u}_w + \vv{w} \times \vv{z}
\end{align}


\section{Momento di un vettore applicato}
\begin{wrapfigure}{r}{0.35\textwidth}
    \caption{Il momento di un vettore applicato in $A$ rispetto ad un certo polo $\Omega$}
    \begin{tikzpicture}
        \draw[-] (0, 0) -- (1, 2.5);
        \draw[-] (0, 0) -- (4, 0);
        \draw[-] (1, 2.5) -- (5, 2.5);
        \draw[-] (4, 0) -- (5, 2.5);
        \draw[-, dashed] (1, 0.5) -- (3, 0.5);
        \draw[-, dashed] (1.5, 0.5) -- (2, 2) node[midway, left] {$d$};
        \filldraw (2, 2) circle (1pt) node[anchor=west]{$\Omega$};
        \draw[->, red] (2,2) -- (2, 3.5) node[midway, right] {$\vv{m}$};
        \draw[->] (2, 2) -- (3, 0.5) node[midway, right] {$\vv{r}$};
        \filldraw (3, 0.5) circle (1pt) node[anchor=north]{B};
        \filldraw (2.25, 0.5) circle (1pt) node[anchor=north]{A};
        \draw(2, 2) -- (2.25, 0.5);
        \draw[-] (2.8, 0.8) .. controls (2.65, 0.7) .. (2.6, 0.5) node[midway, left] {$\theta$};
        \draw[-, dashed] (3, 0.5) -- (3.3, 0);
        \draw[->, red] (3, 0.5) -- (4, 0.5) node[midway, above] {$\vv{v}$};
    \end{tikzpicture}
\end{wrapfigure}
Conoscere il punto di applicazione di un vettore è fondamentale per alcune grandezza
come il lavoro oppure il momento torcente etc...
Un vettore $\vv{v}$ applicato (in A) rispetto ad un punto $\Omega$ chiamato \emph{polo},
è il vettore libero definito come:
\begin{align}
    \vv{m} = \vv{r} \times \vv{v}.
\end{align}
Dove $\vv{r} = \vv{\Omega A}$, il momento è $\perp$ al piano e individuato dai due vettori
del prodotto vettoriale e ha come modulo $m = rv \sin(\theta)$.
Mostriamo ora come il momento non dipenda dal polo, ma dalla posizione della
retta di applicazione rispetto al polo stesso: presi i due vettori
$\vv{r} = \vv{\Omega A}, \vv{v}$, consideriamo un punto B che appartiene alla 
retta su cui giace $\vv{v}$, il quale è traslato in modo da essere applicato in B
invece che in A. Il momento risultante sarà dunque:
\[
    \vv{m}' = \vv{\Omega B} \times \vv{v} = (\vv{\Omega A} + \vv{AB}) \times \vv{v} =
    \vv{\Omega A} \times \vv{v} = \vv{m}.
\]
in quanto i due vettori $\vv{AB}$ e $\vv{v}$ sono paralleli. Ne consegue che il momento
non cambia se spostiamo il punto di applicazione del vettore lungo la sua retta di azione.\\
Si definisce \textbf{momento assiale} di $\vv{v}$ rispetto ad un asse di versore $\hat{u}$
la grandezza scalare:
\[
    m_u = (\vv{r} \times \vv{v}) \cdot \hat{u} \quad (con \ \vv{r} = \vv{\Omega A}).
\] 
La posizione di $\Omega$ sulla retta d'azione non cambia il momento assiale
del vettore.
Il momento assiale (se il vettore $\vv{m}$) è parallelo all'asse z, diventa:
\[
    m_z = \vv{m} \cdot \hat{k} = (\vv{r} \times \vv{v}) \cdot \hat{k}.
\]
Per esplicitare $m_z$ possiamo utilizzare il determinante del prodotto vettoriale, ossia
moltiplicare il versore $\hat{k}$ con ciascuno degli elementi della prima riga del
determinante e quindi
\begin{gather*}
    m_z = \det \begin{pmatrix}
        \hat{i} & \hat{j} & \hat{k} \\
        r_x & r_y & r_z \\
        v_x & v_y & v_z
    \end{pmatrix} = \det \begin{pmatrix}
        0 & 0 & 1\\
        r_x & r_y & r_z \\
        v_x & v_y & v_z
    \end{pmatrix} = r_x v_y - r_y v_x
\end{gather*} 
In pratica per calcolare il momento assiale, si considerano solo i contributi
su quell'asse e si ignorano quelli sugli altri assi, dunque possiamo concludere che il momento assiale
di un vettore rispetto ad una retta assegnata si annulla se e solo se il vettore
è parallelo alla retta o la retta d'azione del vettore incontra la retta. 

\section{Vettore posizione e sistemi di coordinate}
Il vettore posizione è un vettore particolare poiché è definito a partire
da un origine nella quale il suo modulo è pari a zero. Nei sistemi
di coordinate curvilinee ortogonali il vettore posizione può essere
scomposto nelle sue tre componenti:
\begin{align}
    \vv{r} = x\hat{i} + y \hat{j} + z\hat{k}
\end{align}
Le sue componenti rispetto ai versori degli assi si chiamano \textbf{coordinate cartesiane}.


\subsection{coordinate polari piane}
\begin{wrapfigure}{r}{0.3\textwidth}
    \caption{Relazione tra versori polari piani e cartesiani}
    \begin{tikzpicture}[scale=0.75]
        \draw[->] (0, 0) -- (0, 4) node[at end, left] {$\hat{j}$};
        \draw[->] (0, 0) -- (4, 0) node[at end, below] {$\hat{i}$};
        \draw[-] (0, 0) -- (3, 2) node[midway, above] {$r$};
        \filldraw(0, 0) node[anchor = north] {$O$};
        \filldraw(3, 2) node[anchor = east] {$P$};
        \draw[-] (1.5, 1) .. controls (1.8, 0.8) and (2, 0.5).. (2, 0) node[midway, right] {$\theta$};
        \draw[->, red] (3, 2) -- (2.3, 3) node[at end, left] {$\hat{u}_{\theta}$};
        \draw[->, red] (3,2) -- (3.85, 2.7) node[at end, above] {$\hat{u}_r$};
        \draw[very thin] (3, 2) -- (3, 4);
        \draw[-, ultra thick] (3,2) -- (3.85,2);
        \draw[very thin] (3.85, 2) -- (5, 2);
        \draw[-] (3.4, 2.3) .. controls (3.5, 2.15) ..  (3.5, 2) node[midway, right] {$\theta$};
        \draw[-, dashed] (3.85, 2.7) -- (3.85, 2) node[midway, right] {$\sin(\theta)$};
        \draw[->] (3.40, 1.5) -- (3.40, 1.8) node[at start, below] {$\cos(\theta)$};
    \end{tikzpicture}
\end{wrapfigure}
Su un piano cartesiano, oltre all'utilizzo delle coordinate cartesiane,
si può utilizzare anche un sistema di coordinate che fa utilizzo di una
coppia di scalari che esprimono la distanza dall'origine, anche
chiamata \textbf{polo}, e l'angolo del vettore posizione $\vv{r} = \vv{OP}$.
Ogni punto sul piano possiede una coppia di versori chiamati \emph{versore radiale}
che è il versore del vettore posizione ed il \emph{versore trasverso} che è il versore
della tangente in P alla circonferenza di raggio OP.
\begin{align}
    \vv{r} =& r\hat{u}_r \\
    \left\{\begin{array}{l}
        x = r \cos(\theta) \\
        y = r \sin(\theta)
    \end{array}\right. \qquad &\qquad \left\{\begin{array}{l}
        r = \sqrt{x^2 + y^2} \\
        \theta = \arctan\left(\frac{y}{x}\right)
    \end{array}\right.
\end{align}
La dipendenza da $\hat{u}_r$ e da $\hat{u}_{\theta}$ da $\theta$ è data dalla geometria della relazione
tra i versori polari:
\begin{align}
    \left\{\begin{array}{l}
        \hat{u}_r = (\hat{u}_r \cdot \hat{i})\hat{i} + (\hat{u}_r \cdot \hat{j})\hat{j} =
        \cos(\theta) \hat{i} + \sin(\theta) \hat{j} \\
        \hat{u}_{\theta} = (\hat{u}_{\theta} \cdot \hat{i})\hat{i} + (\hat{u}_{\theta} \cdot \hat{j})\hat{j} =
        -\sin(\theta) \hat{i} + \cos(\theta) \hat{j}
    \end{array}\right.
\end{align}
Si ottengono anche le derivate di questi versori:
\begin{align}
    \frac{d \hat{u}_r}{d\theta} = \hat{u}_{\theta} \qquad \qquad
    \frac{d \hat{u}_{\theta}}{d \theta} = -\hat{u}_r
\end{align}
Che si dimostra esprimendo tale vettore in termini di versori che non dipendano dalla variabile rispetto a cui
si deriva. Nel nostro caso si utilizzano le relazioni per i versori e dato che
$\hat{i}$ e $\hat{j}$ che non sono funzioni dell'angolo. 
\begin{gather*}
    \left\{\begin{array}{l}
        \frac{d\hat{u}_r }{d\theta} = -\sin\theta \hat{i} + \cos\theta \hat{j} = \hat{u}_{\theta} \\
        \frac{d\hat{u}_{\theta} }{d\theta} = -\cos\theta \hat{i} - \sin\theta \hat{j} = -\hat{u}_r      
    \end{array}\right.
\end{gather*}  

\section{Coordinate polari sferiche}
\begin{wrapfigure}{r}{0.4\textwidth}
    \centering
    \label{Fig 2.2}
    \caption{Versori delle coordinate polari sferiche}
    \includegraphics[width=0.4\textwidth]{polari.jpg}
\end{wrapfigure}
E' un sistema di coordinate che si basa sulla scelta di un origine O, una asse che
prende il nome di \textbf{asse polare} e di un
semipiano $\sigma$ di riferimento contenente tale asse. Le superfici individuate sono:
\begin{enumerate}
    \item Le sfere individuate dal loro raggio $\vv{r}$;
    \item Il semicono con vertice O e base il cerchio su cui è poggiato il punto P e caratterizzato dalla colatitudine $\theta$;
    \item I semipiani passanti per l'asse polare e caratterizzati dalla longitudine $\phi$ 
\end{enumerate} 
Le coordinate del punto P sono quindi esprimibili attraverso il $\vv{r}$, $\theta$ e $\phi$, mentre il vettore
nel punto è scomponibile in tre componenti: $\vv{u}_r$, $\vv{u}_{\theta}$ e $\vv{u}_{\phi}$, che sono
rispettivamente perpendicolari alle superfici sferica, conica e al semipiano. Questa terna di vettori
prende il nome di \textbf{terna locale} in quanto dipende dal punto P scelto.
Nel caso in cui l'asse polare coincida con z ed il semipiano passi per $y = 0$, le coordinate di P sono:
\begin{align}
    P = \left\{\begin{array}{l}
        x = r \cdot  \sin\theta \cdot \cos \phi \\
        y = r \cdot  \sin\theta \cdot \sin \phi \\
        z = r \cdot  \cos\theta
    \end{array}\right.
\end{align}
Si ottiene allora che il versore $\hat{u}_r$ sarà dato da:
\begin{align}
    \hat{u}_r = \sin\theta \cdot \cos\phi \hat{i} + \sin\theta \cdot \sin\phi \hat{j} + \cos\theta\hat{k}    
\end{align} 
E che le espressioni cartesiane degli altri versori sono:
\begin{align}
    \hat{u}_{\theta} &=  \cos\theta \cdot \cos\phi \hat{i} + \cos\theta \cdot \sin\phi \hat{j} - \sin\theta\hat{k} \\
    \hat{u}_{\phi} &= -\sin\theta\hat{i} + \cos\phi\hat{j}    
\end{align}

\section{Coordinate polari cilindriche}
\begin{wrapfigure}{r}{0.4\textwidth}
    \centering
    \label{Fig 2.3}
    \caption{Coordinate polari cilindriche}
    \includegraphics[width=0.4\textwidth]{polari-cilindriche.jpg}
\end{wrapfigure}
Analogamente alle coordinate polari sferiche, le coordinate polari cilindriche si basano
sulla scelta di una retta orientata O$z$ di riferimento e di un semipiano $\sigma$ di riferimento contenente tale asse.
Si chiama ora $\pi$ il semipiano passante per O e $\perp Oz$, le superfici sono:
\begin{enumerate}
    \item I piani paralleli al piano $\pi$ individuati dalla quota $z$ dove il
    vettore $\vv{P}_z$ è $\perp$;
    \item Le superfici cilindriche con asse di simmetria $Oz$ definite dal raggio  $\vv{r}$
    della sezione normale del cilindro; alla superficie cilindrica per P è $\perp \vv{P}_r$;
    \item I semipiani passanti per $Oz$ caratterizzati dall'angolo $\theta$ formati con $\sigma$
    ai quali sono $\perp \vv{P}_\theta$    
\end{enumerate}
Le coordinate cilindriche di un punto generico sono date dagli scalari appena visti che individuano
le coordinate che si intersecano in P.  Le relazioni sono dunque:
\begin{align}
    P = \left\{\begin{array}{l}
        x = r \cos\theta \\
        y = r \sin\theta \\
        z = z
    \end{array}\right.
\end{align}
Allora la rappresentazione dei tre versori è dunque:
\begin{align}
    \hat{P}_z &= \hat{k} \\
    \hat{P}_r &=  \cos\theta \hat{i} + \sin\theta \hat{j} \\
    \hat{P}_{\theta} &= -\sin\theta\hat{i} + \cos\theta \hat{j}     
\end{align}

\chapter{Cinematica}
\section{I fondamenti della cinematica}
\subsection{Sistemi di riferimento}
Il concetto di moto presuppone l'esistenza di qualcos'altro su cui fare riferimento. La scelta
di un sistema di riferimento è arbitraria e sistemi di riferimento più semplici e comodi
permettono di giungere a conclusioni in modo semplice e veloce. Un sistema di riferimento è composto da
corpi, osservatori, regoli e orologi tutti fissi gli uni rispetto agli altri. \\
I corpi possono avere le più varie forme dimensioni e caratteristiche anche se in molte
situazioni è possibile adottare una schematizzazione che consiste nel trascurare le dimensioni
geometriche e di ridurre un corpo ad un punto geometrico in cui è concentrata tutta la massa
che prende il nome di \textbf{punto materiale}. Questa approssimazione è accettabile quando
le dimensioni sono trascurabili rispetto alle distanze che vengono percorse e non influiscono
in maniera rilevante sulle caratteristiche del moto. 

\subsection{Equazioni vettoriali di un moto: traiettoria e legge oraria}
Si dice che un corpo è in moto rispetto ad un dato sistema di riferimento $S$ quando la sua posizione 
in $S$ cambia nel tempo. Nello schema di un punto materiale le caratteristiche sono fornite dalla 
conoscenza del vettore posizione $\vv{r}$ in funzione del tempo. \\
Nell'ipotesi che il tempo sia continuo, si può affermare che:
\begin{gather*}
    |\vv{r}(t) - \vv{r}(t + \Delta t)| \to 0, \quad \text{per} \quad \Delta t \to 0   
\end{gather*}
E questo ci porta a definire moti complessi continui esprimendo il vettore posizione
in funzione del tempo:
\begin{align}
    \vv{r} = \vv{r}(t)  
\end{align}
\begin{wrapfigure}{r}{0.4\textwidth}
    \centering
    \caption{L'Ascissa curvilinea}
    \begin{tikzpicture}
        \draw[red](0, 0) .. controls (0.5, -1.5) and (2, -0.25) .. (3, 0) node[at start, right] {$\gamma$};
        \filldraw(3, 0) circle (1pt) node[anchor = south] {$P_2$};
        \draw[red, ->](3, 0) .. controls (3.5, 0.2) .. (4, 0.2);
        \filldraw(0.3, -0.5) circle (1pt) node[anchor = south west] {$P_1$};
        \filldraw(1.6, -0.5) circle (1pt) node[anchor = north] {$\Omega$};
        \filldraw(0.5, -0.75) circle (0pt) node[anchor = north] {$s_1 < 0$};
        \filldraw(2, -0.5) circle (0pt) node[anchor = west] {$s_2 > 0$};
    \end{tikzpicture}    
\end{wrapfigure}
La funzione vettoriale è anche rappresentabile mediante le tre funzioni scalari:
\begin{align}
    \vv{r}(t) = \left\{\begin{array}{l}
        x = x(t) \\
        y = y(t) \\
        z = z(t)
    \end{array}\right. 
\end{align}
Con queste informazioni si può costruire la \textbf{traiettoria} del corpo, ossia
l'insieme delle posizioni occupate in certi istanti di tempo. A livello geometrico
questa è un caso particolare della rappresentazione in forma parametrica di una curva
nello spazio. Per separare l'aspetto geometrico da quello cinematico si usa la
\textbf{rappresentazione intrinseca di un moto}: supponiamo di conoscere la traiettoria $\gamma$ di un punto materiale e supponiamo di creare
una successione di segmenti infinitesimi definendo un'origine $\Omega$, un verso e un'unità di misura. 
Ad ogni punto P si fa corrispondere $s$ ossia un numero reale chiamato \textbf{ascissa curvilinea} 
e sarà positivo o negativo a seconda se si trova dopo o prima di $\Omega$. Grazie ad $s$
possiamo ricavare l'\textbf{equazione della traiettoria} e l'equazione $s = s(t)$ ci
fornisce la \textbf{legge oraria}. 

\section{La velocità}
\subsection{Il concetto di velocità ed il suo vettore}
\begin{wrapfigure}{r}{0.4\textwidth}
    \centering
    \begin{tikzpicture}
        \filldraw(0, 0) circle (2pt) node[anchor=west]{$O$};
        \filldraw[red] (1, 4) circle (2pt) ;
        \filldraw[red] (1.5, 3.9) circle (2pt) ;
        \filldraw[red] (2, 3.75) circle (2pt) ;
        \filldraw[red] (2.5, 3.5) circle (2pt);
        \filldraw[red] (3, 3) circle (2pt) ;
        \draw[->] (0, 0) -- (1, 4) node[left] {$\vv{r}(t)$};
        \draw[->] (0, 0) -- (3, 3) node[midway, right]{$\vv{r}(t' + \Delta t)$};
        \draw[-] (1, 4) -- (1.5,  3.9);
        \draw[-] (1.5, 3.9) -- (2,  3.75);
        \draw[-] (2, 3.75) -- (2.5,  3.5);
        \draw[-] (2.5, 3.5) -- (3,  3);
        \draw[->] (4, 4) -- (3, 3.5) node[at end, above] {$\gamma$};
    \end{tikzpicture}    
\end{wrapfigure}
Osservando un dato moto con una macchina fotografica che scatta delle foto
ogni $\Delta \tau$ secondi, si osserva una certa regolarità che è più evidente nei casi in cui
si diminuisca $\Delta \tau$; ci si aspetta dunque che la curva formata dal moto goda di certe
proprietà di regolarità.
Con riferimento ad un esperimento di lancio di una pallina orizzontalmente, si osserva che la pallina
sembra compiere due moti distinti, uno verticale (quadratico) ed uno orizzontale (lineare), con queste assunzioni
possiamo considerare due istanti: $t' = t + \Delta t$. Poste come $P$ e $P'$ le posizioni
occupate dalla pallina negli istanti $t$ e $t'$ e posti  $\vv{r}(t)$ e $\vv{r}(t')$ i corrispondenti vettori posizione
rispetto all'origine O del sistema di riferimento, possiamo ricavare una prima informazione sul moto come
\begin{gather*}
    \Delta \vv{r} = \vv{r}(t') - \vv{r}(t) 
\end{gather*}
Naturalmente si tratta di un'informazione sulla velocità media su quanto è accaduto tra i due istanti di
tempo e non ci da informazioni su come si sia mossa la pallina tra i due istanti, se abbia seguito
una traiettoria retta oppure una parabola...
\begin{gather*}
    \vv{v}_m = \frac{\vv{r}(t + \Delta t) - \vv{r}(t)}{\Delta t}
\end{gather*}
La \textbf{velocità media} non dipende dal percorso ma dalle posizioni iniziali e finali. Si
ottengono più informazioni però dal vettore \textbf{velocità istantanea} al ridursi di $\Delta t$:
\begin{gather*}
    \vv{v} = \lim_{\Delta t \to 0} \vv{v}_m = \lim_{\Delta t \to 0} \frac{\Delta \vv{r}}{\Delta t}
\end{gather*}
Si conclude, tenendo a mente la derivata di un vettore, che $\vv{v}$ è:
\begin{gather*}
    \vv{v}(t) = \frac{d \vv{r}(t)}{dt}
\end{gather*}
Le caratteristiche generali di questo vettore sono ottenute dall'analisi del moto di un corpo,
in modo da ottenere la curva $\gamma$ con opportune approssimazioni ed interpolazioni.
Si osserva per ipotesi, che la velocità istantanea è esattamente parallela allo spostamento
$\vv{PP'}$ e ha quindi la direzione della retta che interseca $\gamma$ in P e P'. In altri termini:
\begin{gather*}
    \lim_{\Delta s \to 0} \frac{|\vv{PP'}|}{|\Delta s |} = 1
\end{gather*}

\subsection{Rappresentazione della velocità}
\begin{wrapfigure}{r}{0.4\textwidth}
    \centering
    \caption{Costruzione del versore tangente alla curva}
    \begin{tikzpicture}
        \draw(0, 0) .. controls (1, 0) and (1.25, 0) .. (1.5, 0.5);
        \draw(1.5, 0.5) .. controls (2 , 2.75) and (3.25, 2.75).. (4, 1.25);
        \filldraw(0, 0) circle (1pt) node[anchor = south ] {$\Omega$};
        \filldraw(1.5, 0.5) circle (1pt) node[anchor = east] {P''};
        \filldraw(1.75, 1) circle (0pt) node[anchor = west] {$\Delta s < 0$};
        \filldraw(2.1, 1.9) circle (1pt) node[anchor = east] {P};
        \filldraw(4, 1.25) circle (1pt) node[anchor = west] {P'};
        \draw[->](2.1, 1.9) -- (4, 1.25) node[midway, above] {$\Delta \vv{r}$ };
        \filldraw(3, 2.5) circle (0pt) node[anchor = south] {$\Delta s > 0$};
        \draw[red, ->](2.1, 1.9)-- (2.4, 2.4) node[at end, left] {$\hat{u}_t$ };
    \end{tikzpicture}     
\end{wrapfigure}
Dati due punti $P$ e $P'$ della curva individuate dalle ascisse curvilinee $s$ ed $s' = s + \Delta s$
e detto $\Delta \vv{r}$ il vettore $\vv{PP'}$, si considera il rapporto $\Delta \vv{r}/ \Delta s$
e il suo limite per $\Delta s \to 0$. Questo vettore ha la direzione della secante e il verso concorde con
gli archi crescenti. Il limite tende ad assumere la direzione della tangente alla curva in P ed ad
avere modulo unitario con verso concorde con quello dell'orientamento della curva. Esso 
è dunque il \textbf{versore tangente} alla curva orientata e si esprime come:
\begin{gather*}
    \hat{u}_t = \lim_{\Delta s \to 0} \frac{\Delta \vv{r} }{\Delta s} = \frac{d\vv{r}}{ds}
\end{gather*}
Si può stabilire il legame tra della velocità con le equazioni della traiettoria e con la legge oraria
tenendo conto del ruolo di variabile intermedia fra $\vv{r}$ e t giocato da s:
\begin{align}
    \vv{v} = \frac{d\vv{r}}{dt} = \frac{d\vv{r}}{ds} \frac{ds}{dt} = \frac{ds}{dt} \hat{u}_t  
\end{align}  
Questa mostra che il modulo è dato da $\left| \frac{ds}{dt} \right|$ ed il verso di $\vv{v}$
coincide con $\hat{u}_t$ e la grandezza:
\begin{gather*}
    v_s = \frac{ds}{dt} 
\end{gather*}
è chiamata \textbf{parte scalare} della velocità rispetto al versore $\hat{u}_t$ ed è chiamata anche
\textbf{velocità scalare}:
\begin{gather*}
    \vv{v} = v_s \qquad \hat{u}_t = \dot{s}\hat{u}_t
\end{gather*}
Essendo a conoscenza della dell'equazione oraria del moto si può determinare l'equazione della
velocità scalare ad ogni istante con la derivazione rispetto al tempo. \\
Si è detto come il moto possa considerarsi la successione di di spostamenti infinitesimi
$d\vv{r} = \vv{v} dt$ in intervalli temporali $dt$ e tali spostamenti hanno direzione 
e verso uguale alla velocità istantanea e modulo proporzionale a dt. Lo \textbf{spazio percorso}
è quindi la somma delle lunghezze degli archi infinitesimi percorsi $|ds| = |\vv{v}| dt = vdt$,
ossia l'integrale della velocità:
\begin{align}
    \int_{t_1}^{t_2} |\vv{v}(t)|dt = \int_{t_1}^{t_2} v(t) dt \ \ (\text{spazio percorso})
\end{align}
Essendo lo spazio percorso per definizione non negativo (può essere nullo tuttavia), questa quantità è in genere diversa
dalla somma degli archi infinitesimi $ds$ ossia dall'integrale definito della velocità scalare,
che dà la differenza fra i valori iniziali e finali dell'ascissa curvilinea:
\begin{align}
    \Delta s = s_2 - s_1 = \int_{t_1}^{t_2} v_s(t) \ dt = \int_{t_1}^{t_2} \frac{ds}{dt} \ dt = \int_{s_1}^{s_2} ds
\end{align}
La quale potrebbe essere nulla anche se il corpo è in movimento. \\
Oltre che alla forma intrinseca, la velocità si rappresenta anche in forma cartesiana
a seconda del sistema di coordinate usato per il vettore posizione nel sistema scelto. In un sistema di riferimento
con origine O, $\vv{r}$ ha la rappresentazione cartesiana e quindi dato che i versori $\hat{i}, \hat{j}, \hat{k}$
degli assi coordinati non dipendono da t, si può scrivere:
\begin{align}
    \vv{v} = \frac{d\vv{r} }{dt} = \frac{dx}{dt} \hat{i} + \frac{dy}{dt} \hat{j} + \frac{dz}{dt} \hat{k}    
\end{align}    
Le componenti cartesiane di $\vv{v}$ sono dunque le derivate rispetto al tempo delle corrispondenti componenti
del vettore posizione.
\begin{gather*}
    v_x = \frac{dx}{dt}; \qquad v_y = \frac{dy}{dt}; \qquad v_z = \frac{dz}{dt}
\end{gather*}
e quindi:
\begin{gather*}
    \vv{v} = \frac{d\vv{r} }{dt} = \dot{x} \hat{i} + \dot{y} \hat{j} + \dot{z} \hat{k} 
\end{gather*}
Il modulo è quindi calcolabile come: $\sqrt{v_x^{2} + v_y^{2} + v_z^{2}}$ 

\section{L'accelerazione}
Durante il moto il vettore $\vv{v}$ non resta costante al trascorrere del tempo ma cambia poiché
variano il suo modulo, la sua direzione orientata o entrambi. Se ad un dato istante $t$ la velocità è $\vv{v}(t)$ e
all'istante $t' = t + \Delta t$ la velocità è $\vv{v}(t')$ si definisce \textbf{accelerazione media} 
nell'intervallo $\Delta t$ il vettore:
\begin{gather*}
    \vv{a}_m = \frac{\vv{v}(t + \Delta t) - \vv{v}(t)}{\Delta t}
\end{gather*}
Questa grandezza vettoriale ci da informazioni sull'accelerazione media
e sul cambiamento del vettore velocità in un certo intervallo di tempo. 
Quando questo intervallo di tempo si riduce, allora si considera l'accelerazione istantanea
\begin{gather*}
    \vv{a} = \lim_{\Delta t \to 0} \vv{a}_m = \lim_{\Delta t \to 0} \frac{\Delta \vv{v}}{\Delta t}
\end{gather*}
Questo essendo il limite del rapporto incrementale della velocità è la \textbf{derivata prima del vettore velocità}:
\begin{align}
    \vv{a}(t) = \frac{d\vv{v}(t)}{dt} = \frac{d^{2}\vv{r}(t)}{dt^{2}} 
\end{align}
L'espressione cartesiana dell'accelerazione è dunque:
\begin{gather*}
    \vv{a} = \dot{\vv{v} } = \ddot{\vv{r}}= \ddot{x} \hat{i} + \ddot{y} \hat{j} + \ddot{z} \hat{k}
\end{gather*}
Con le parti scalari dell'accelerazione che si esprimono come:
\begin{gather*}
    \left\{\begin{array}{l}
        a_x = \frac{dv_x}{dt} = \frac{d^{2} x}{dt^{2}} = \ddot{x} \\
        a_y = \frac{dv_y}{dt} = \frac{d^{2} y}{dt^{2}} = \ddot{y} \\
        a_z = \frac{dv_z}{dt} = \frac{d^{2} z}{dt^{2}} = \ddot{z} 
    \end{array}\right.
\end{gather*}


\subsection{Rappresentazione dell'accelerazione}
Il vettore accelerazione riflette le diverse possibili variazioni elementari del vettore velocità;
in generale il vettore $\vv{a}$ è esprimibile come la somma di due componenti distinte:
una parallela alla velocità e collegata alla rapidità di variazione della sua parte scalare e un'altra 
perpendicolare alla velocità dipendente dalla rapidità di variazione della sua direzione. 
\begin{align}
    \vv{a} = \frac{d\vv{v}}{dt} = \frac{d}{dt}(v_s \hat{u}_t) = \frac{dv_s}{dt}\hat{u}_t + v_s \frac{d\hat{u}_t}{dt}  
\end{align} 
Allora dato che $v_s = \frac{ds}{dt}$ si ha che la componente tangenziale:
\begin{align}
    \vv{a}_t = \frac{d^{2}s}{dt^{2}}\hat{u}_t = \ddot{s}\hat{u}_t    
\end{align}
Tale componente ha lo stesso verso di $\hat{u}_t$ se la velocità scalare cresce.
Esso è il componente di $\vv{a}$   che riflette le variazioni del modulo e/o del verso
di $\vv{v}$ e viene anche detto componente tangenziale di $\vv{a}$ chiamata \textbf{accelerazione tangenziale}. \\
Per ottenere un'espressione più significativa del secondo componente di $\vv{a}$ si esplicita
la derivata di $\hat{u}_t$ rispetto al tempo: dato che il versore tangente $\hat{u}_t = \frac{d\vv{r}}{ds}$ dipende dalla scelta
del verso positivo per le ascisse curvilinee sulla traiettoria, e non dalle effettive caratteristiche
istantanee del moto, conviene esprimere la dipendenza di $\hat{u}_t$ dal tempo
attraverso la variazione di $\hat{u}_t$ al cambiare di $s$: 
\begin{gather*}
    \frac{d\hat{u}_t}{dt} = \frac{d\hat{u}_t }{ds} \frac{ds}{dt} = \dot{s} \frac{d\hat{u}_t }{ds} 
\end{gather*}   
\begin{wrapfigure}{r}{0.4\textwidth}
    \centering
    \caption{Cerchio osculatore}
    \begin{tikzpicture}
        \draw[-, dashed] (0, 0) circle (1) node[anchor=west]{$O$};
        \filldraw (0, 0) circle (1pt);
        \draw[-] (0, 0) -- (0, 1) node[midway, left] {$\rho$};
        \draw[-] (0, -1.5) -- (2, 0.5);
        \filldraw (0.7, -0.7) circle (1pt) node[anchor=north]{$P$};
        \draw[->, red, thick] (0.7, -0.7) -- (0.3, -0.3) node[at end, below] {$\hat{u}_n$};
        \draw[->, red, thick] (0.7, -0.7) -- (1.1, -0.3) node[midway, below] {$\hat{u}_t$};
        \draw[->, thin] (-1,-1) .. controls (0.15,-1.3) and (0.85, -0.9) .. (1.5,0.5);
    \end{tikzpicture}
\end{wrapfigure}
La derivata del versore $\hat{u}_t$ rispetto as s è la caratteristica intrinseca della traiettoria
dipendente dalle sue proprietà locali in P; con il medesimo procedimento si trova, data la definizione
di derivata di un versore:
\begin{gather*}
    \frac{d\hat{u}_t }{ds} = \frac{d\phi}{ds} \hat{u}_n \ (\text{vettore} \ \perp)
\end{gather*} 
Dalla geometria un elemento di curva attorno ad un generico punto P può essere approssimato
con un elemento di arco di circonferenza,il cui cerchio associato è detto \emph{cerchio osculatore}
e ha il centro nel \textbf{centro di curvatura} della curva P e raggio $\rho$ mentre la curvatura
è data da $1/\rho$. 
La retta $\perp$ alla tg in P della curva è chiamata normale principale e il versore della sua direzione
è proprio $\hat{u}_n$. Tutte queste proprietà sono locali poiché dipendono dal punto P. Poiché
$d\phi$ è uguale all'angolo $\frac{ds}{\rho}$ sotto il quale viene visto l'elemento di arco di curva
dal centro di curvatura si ha allora che:
\begin{align}
    \frac{d\hat{u}_t}{ds} = \frac{1}{\rho}\dot{s}\hat{u}_n 
\end{align} 

\begin{wrapfigure}{r}{0.4\textwidth}
    \centering
    \caption{L'angolo $d\phi$ fra le tangenti in P e P' è uguale all'angolo
    al centro formato da CP e CP'}
    \begin{tikzpicture}
        \draw[-, dashed] (0, 0) circle (1) node[anchor=west]{$O$};
        \filldraw (0, 0) circle (1pt);
        \filldraw (0.25, -0.9) circle (1pt) node[anchor=north]{$P$};
        \filldraw (0.7, -0.7) circle (1pt) node[anchor=north] {$P'$};
        \draw[-, line width=0.1 pt] (0, 0) -- (0.7, -0.7);
        \draw[-, line width=0.1pt] (0, 0) -- (0.25, -0.9) node[midway, left]{$\rho$};
        \draw[-, line width=0.1pt] (-1, -1.3) -- (2, -0.5);
        \draw[-, line width=0.1pt] (0, -1.6) -- (1.5, 0.4);
        \draw (1.2, 0) arc (45:15:1.4) node[midway, right] {$d\phi$};
    \end{tikzpicture}  
\end{wrapfigure}
Allora sostituendo nell'espressione di $\vv{a}_t$  si ottiene:
\begin{align}
    \vv{a} &= \vv{a}_t + \vv{a}_n = a_t\hat{u}_t + a_n \hat{u}_n = \ddot{s}\hat{u}_t + \frac{\dot{s}^{2}}{\rho} \hat{u}_n       
\end{align} 
La componente normale dell'accelerazione è $\vv{a}_n$ ed è sempre non negativa per cui $\vv{a}_n$ punta sempre 
al centro di curvatura; essa è chiamata allora \textbf{accelerazione centripeta} ed è esprimibile
come: 
\begin{align}
    \vv{a}_n = \frac{v^{2}}{\rho}\hat{u}_n  
\end{align}  
Quindi $\vv{a}_n$ è presente in ogni moto su traiettoria non rettilinea e ogni moto di curva è dunque
accelerato. Nella sua rappresentazione intrinseca (ossia con i versori), l'accelerazione ha modulo:
\begin{gather*}
    a = \sqrt{a_t^{2} +a_n^{2}} 
\end{gather*} 
Non si potrebbe introdurre un terzo vettore, chiamato per esempio strappo che determina la velocità
di variazione dell'accelerazione? Si, potremmo indicarlo come: $\vv{b} = \frac{d\vv{a}}{dt} = \frac{d^{3 }\vv{r}}{dt^{3}}$ ma
come si vedrà nel capitolo della dinamica, l'accelerazione è l'unica direttamente legata
ai fenomeni e alle interazioni con l'ambiente che determinano il moto di un corpo.

\section{I moti elementari}
Le due relazioni: 
\begin{align}
    \left\{\begin{array}{l}
        \vv{v} = \dot{s}\hat{u}_t \\
        \vv{a} = \ddot{s}\hat{u}_t + \frac{\dot{s}^2}{\rho}\hat{u}_n      
    \end{array}\right.
\end{align}
ci permettono di definire in modo semplice i moti elementari su traiettoria nota poiché 
riescono a separare l'aspetto cinematico da quello geometrico della traiettoria nel piano. \\
Fissando l'attenzione sull'equazione oraria si possono definire diverse classi di moti:
\begin{enumerate}
    \item Moti uniformi ($\dot{s}$ costante);
    \item Moti uniformemente vari (con $\ddot{s}$ costante);
    \item Moti rettilinei ($\rho \to \infty $);
    \item Moti circolari ($\rho$ costante).  
\end{enumerate}

\subsection{Moti con $\dot{s}$ costante}
\begin{wrapfigure}{r}{0.4\textwidth}
    \centering
    \label{Fig 3.4}
    \caption{Diagramma orario del moto uniforme}
    \begin{tikzpicture}
        \draw[->, line width=0.025mm] (0, 0) -- (3, 0) node[at end, below] {$t$};
        \draw[->, line width=0.025mm] (0, 0) -- (0, 3) node[at end, left] {$s$};
        \draw[-, red] (-1, 0) -- (3, 2);
        \draw[line width=0.025mm, -, dashed] (-0.5, 0.5) -- (2.5, 0.5);
        \draw[line width=0.025mm, -, dashed] (0, 1.5) -- (2, 1.5) node[at start, left] {$s_0$};
        \draw[line width=0.025mm, -, dashed] (2, 0) -- (2, 1.5) node[at start, below] {$t_0$};
        \draw (1, 1) arc (35:0:0.9) node[midway, right] {$\alpha$};
        \filldraw (1, 2.5) circle (0pt) node[anchor=west] {$s = \dot{s}_0(t- t_0) +s_0$};
        \filldraw (2, 1) circle (0pt) node[anchor=west] {$\tan\alpha = \dot{s}_0$};
    \end{tikzpicture}    
\end{wrapfigure}
La ricerca della funzione $s(t)$ di cui è nota la derivata, equivale a trovare una primitiva
di $\dot{s}(t)$ e farne l'integrale indefinito:
\begin{align}
    s(t) = \int \dot{s}(t)dt + c 
\end{align} 
nel caso in cui $\dot{s}$ sia costante si ha che:
\begin{gather*}
    s(t) = \dot{s}_0 t + c 
\end{gather*}
Il valore $c$ si determina a conoscenza dell'ascissa curvilinea $s_0 = s(t_0)$ ad un
certo istante $t_0$ e quindi:
\begin{gather*}
    s(t) - s(t_0) = \dot{s}_0(t- t_0) \Rightarrow s(t) = \dot{s}_0(t - t_0) + s_0  
\end{gather*}
Ossia l'equazione oraria del moto uniforme. Se invece si ha che $t_0 = 0$ allora:
\begin{align}
    s = \dot{s}_0t + s_0 
\end{align}
da cui si deduce che nel moto uniforme il punto materiale percorre archi di traiettoria di uguale
lunghezza in tempi uguali. Si ottiene lo stesso risultato con il processo di separazione di variabili:
\begin{gather*}
    ds = \dot{s}_0 \ dt \\
    \int ds = \int \dot{s}_0 dt = \dot{s}_0 \int dt
\end{gather*}
E quindi l'ultima equazione ci da nuovamente l'equazione di moto.

\subsection{Moti con $\ddot{s}$ costante}
In questo caso si ha che $\frac{d\dot{s}}{dt} = \ddot{s}_0$ e separando le variabili $\dot{s}$ e $t$
si ha che $d\dot{s}= \ddot{s}_0 dt$. Scelto allora $t_0 = 0$ ed integrando per separazione di variabili si ottiene:
\begin{gather*}
    \dot{s} = \ddot{s}_0t + \dot{s}_0
\end{gather*}
Ossia la legge oraria con cui varia nel tempo la velocità scalare e che può essere scritta
come 
\begin{gather*}
    \frac{ds}{dt} = \ddot{s_0}t + \dot{s}_0 
\end{gather*}
per cui separando le variabili ed integrando si ottiene:
\begin{align*}
    ds = \ddot{s}_0t dt + \dot{s}_0dt \ \Longrightarrow \ & \int ds = \int \ddot{s}_0 t dt + \int \dot{s}_0 dt = \ddot{s}_0 \int t dt + \dot{s}_0 \int dt \\
    \ \Longrightarrow \ & s = \frac{1}{2}\ddot{s}_0t^{2} + \dot{s}_0 t + s_0     
\end{align*}
Ossia la legge oraria dei moti uniformemente vari. Estendendo il caso generale con $(t - t_0)$
si ottiene quindi le due relazioni:
\begin{align}
    \left\{\begin{array}{l}
        \dot{s} = \ddot{s}_0 (t - t_0) + \dot{s}_0   \\
        s = \frac{1}{2}\ddot{s}_0(t - t_0)^{2} + \dot{s}_0 (t - t_0) + s_0
    \end{array}\right.
\end{align}
Eliminando dunque t si ottiene l'utile relazione:
\begin{gather*}
    v^{2} = v_0^{2} + 2\ddot{s}_0(s - s_0) = v_0^{2} + 2a_t(s - s_0)   
\end{gather*}

\section{I moti rettilinei}
Il moto rettilineo è di gran lunga il moto più importante poiché qualsiasi moto in tre dimensioni
può essere visto come la composizione dei moti rettilinei rispetto ai tre assi cartesiani con  $\hat{u}_t$
parallelo alla velocità e parallelo alla accelerazione e spostamenti. 
Due casi particolari sono il \textbf{moto rettilineo}, con lo scalare $v_0$ costante
\begin{align}
    x(t) = v_0t + x_0
\end{align}
e il moto \textbf{uniformemente accelerato}, il quale, partendo dalla
formulazione generale 
\begin{align}
    x(t) = \frac{1}{2}a_0t^{2} + v_0t + x_0 
\end{align}
si può ottenere l'equazione che esprime la dipendenza temporale dalla parte scalare della velocità:
\begin{gather*}
    \dot{x}(t) = a_0t + v_0 
\end{gather*}


\section{Moti circolari}
\subsection{Geometria e versori intrinseci della traiettoria}
\begin{wrapfigure}{r}{0.4\textwidth}
    \centering
    \label{Fig 3.5}
    \caption{Ascissa curvilinea $s$ ed angolo di rotazione $\theta$ nel moto circolare}
    \begin{tikzpicture}
        \draw (0, 0) circle (2) node[midway, left] {$O$};
        \draw[->] (0, 0) -- (3, 0) node[at end, below] {$x$};
        \draw[->] (0, 0) -- (0, 3) node[at end, left] {$y$};
        \draw[dashed, -] (0, 0) -- (0.82, 0.82);
        \draw[->] (0.8, 0) arc (0:45: 0.8) node[midway, right] {$\theta$} node[at start, below] {$R$};
        \draw[->, very thick] (2, 0) arc (0:45:2) node[at start, below] {$\Omega$} node[midway, right] {$s = R\theta$} node[at end, below] {$P$};
        \draw[->, red] (1.41, 1.41) -- (2, 2) node[at end, right] {$\hat{u}_r$};
        \draw[->, red] (1.41, 1.41) -- (0.82, 0.82) node[at end, above] {$\hat{u}_n$};
        \draw[->, red] (1.41, 1.41) -- (0.82, 2) node[at end, above] {$\hat{u}_t$};
        \filldraw(2, 0) circle (1pt);
        \filldraw(1.41, 1.41) circle(1pt);
    \end{tikzpicture}    
\end{wrapfigure}
Scegliendo un asse cartesiano con origine centrata al centro del cerchio che descrive la traiettoria
del moto circolare, si hanno le equazioni parametriche del moto circolare in termini di $\theta$
nella forma:
\begin{gather*}
    x = R\cos\theta; \qquad y = R\sin\theta; \qquad z = 0.
\end{gather*}
Durante il moto il vettore posizione ha modulo costante $R$ e come versore:
\begin{gather*}
    \hat{u}_r = \frac{\vv{r}}{R} = \cos\theta\hat{i} + \sin\theta\hat{j}   
\end{gather*}
Assunto sulla circonferenza un sistema di coordinate curvilinee con origine $\Omega$ e
orientazione \emph{antioraria} si ha che lo spostamento $s$ è dato da $R\theta$
se si considera un sistema di riferimento con origine in $\Omega$, 
mentre la traiettoria è rappresentata da:
\begin{align}
    \vv{r}(s) = R\cos\frac{s}{R}\hat{i} + R\sin\frac{s}{R}\hat{j} = R\cos\theta\hat{i} + R\sin\theta\hat{j}     
\end{align}
In ogni punto della traiettoria i \emph{versori tangente $\hat{u}_t$} e \emph{normale $\hat{u}_n$} sono diretti come 
in figura. Dato che il centro O coincide con il centro della circonferenza, e che
il raggio $R$ sia il raggio delle circonferenza in questione, si ha che:
\begin{align*}
    \hat{u}_t =& \ \frac{d\vv{r} }{ds} = \frac{d\hat{u}_r}{d\theta} = -\sin\theta\hat{i} + \cos\theta\hat{j} \\
    \hat{u}_n =& \ R\frac{d\hat{u}_t}{ds} = \frac{d\hat{u}_t}{d\theta} = -\cos\theta\hat{i} -\sin\theta\hat{j} = -\hat{u}_r    
\end{align*}

\subsection{Moto circolare uniforme}
Se la velocità scalare e anche $\dot{\theta}=v_0 / R$ è costante allora il moto è uniforme e vale:
\begin{align}
    \theta(t) = \frac{v_0}{R}(t - t_0) + \theta_0
\end{align} 
Essendo allora $\ddot{s} = 0$ l'accelerazione ha solo la componente normale:
\begin{gather*}
    \vv{a} = \frac{\dot{s}^{2}}{R}\hat{u}_n = \frac{v_0^{2} }{R}\hat{u}_n   
\end{gather*}
Così come per i moti vari, l'accelerazione è solo centripeta per cui in questo moto è costante 
in modulo ma non in direzione. Considerato il caso $\theta_0 = 0$ e sia $\dot{\theta}_0 = \omega > 0$ allora
il punto materiale all'istante $t = 0$ si trova in $\Omega$ e si muove sulla circonferenza con
velocità scalare $v_0 = \omega R$. Essendo allora $\theta = \omega t$ e $s = \omega R t$ si trovano
le seguenti espressioni per $\vv{r}(t)$ derivando rispetto al tempo per $\vv{v}(t )$ e $\vv{a}(t)$:     
\begin{align}
    \left\{\begin{array}{l}
        \vv{r} = R \cos(\omega t) + R\sin(\omega t) \hat{j} \\
        \vv{v} = -\omega R \sin(\omega t) \hat{i} + \omega R \cos(\omega t) = \omega R \hat{u}_t \\
        \vv{a} = -\omega^{2} R\cos (\omega t) \hat{i} - \omega^{2}R \sin(\omega t) \hat{j} = -\omega^{2} \vv{r}             
    \end{array}\right.
\end{align}

\subsection{Moto circolare uniformemente vario}
In questo caso la velocità scalare varia linearmente nel tempo e l'equazione oraria si trasforma come:
\begin{align}
    s(t) = \frac{1}{2}a_0(t - t_0)^{2} + v_0(t - t_0) + s_0
\end{align}
Oppure l'analoga
\begin{align}
    \theta(t) = \frac{1}{2}\frac{a_0}{R}(t - t_0)^{2} + \frac{v_0}{R}(t - t_0) + \theta_0  
\end{align}


\subsection{Grandezze angolari}
\begin{wrapfigure}{r}{0.4\textwidth}
    \centering
    \caption{Relazioni tra i vettori nel moto circolare con centro in $O$}
    \begin{tikzpicture}
         \draw(0,0) ellipse (2 and 1);
         \draw[red, ->](0, 0) -- (0, 1.5) node[at end, left] {$\vv{\omega}$};
         \draw[red, ->](0, 0) -- (1, 0.8) node[midway, below] {$\vv{r}$ };
         \draw[red, ->](1, 0.8) -- (0.25, 1.25) node[midway, above] {$\vv{v}$};
         \filldraw (0, 0) circle (1pt) node[anchor = north] {$O$};
    \end{tikzpicture}    
\end{wrapfigure}
Il moto circolare introduce nuove grandezze cinematiche e la sua comodità è quella di avere il vettore
posizione $\vv{r}$ costante nel tempo per cui la sua derivata:
\begin{align}
    \vv{v}(t) = \vv{\omega}(t) \times \vv{r}(t) \\
    \vv{\omega}(t) = \frac{d\theta(t)}{dt}\hat{k} = \dot{\theta}\hat{k}.       
\end{align} 
Dove $\hat{k}$ è un versore $\perp$ al piano del moto: il vettore $\vv{\omega} (t)$ è chiamato
\textbf{velocità angolare} e caratterizza la rapidità di variazione nel tempo della direzione del vettore
$\vv{r}$. Dalla definizione discende che la velocità angolare ha il verso di $\hat{k}$ se la variazione è positiva
altrimenti il verso opposto. \\ 
In un moto circolare generico il modulo ed il verso di $\omega$ cambiano col tempo ma la direzione è
sempre la stessa, il vettore \textbf{accelerazione angolare} è definito come:
\begin{align}
    \vv{\alpha}(t) = \frac{d\vv{\omega}}{dt} = \frac{d\omega}{dt}\hat{k} = \ddot{\theta}\hat{k}   
\end{align}    
che è $\perp$ al piano del moto. Anche l'accelerazione $\vv{a }$ si esprime in funzione delle
grandezze angolari:
\begin{gather*}
    \vv{a} = \frac{d}{dt}(\vv{\omega} \times \vv{r}) = \frac{d \vv{\omega}  }{dt} \times \vv{r} + \vv{\omega} \times \vv{v}   
\end{gather*}
Per cui si ottiene l'espressione dell'accelerazione mediante la velocità e l'accelerazione angolari:
\begin{align}
    \vv{a} = \vv{\alpha} \times \vv{r} + \vv{\omega} \times(\vv{\omega} \times \vv{r})    
\end{align}
Se l'origine del sistema di riferimento è proprio il centro della circonferenza allora si può esprimere come:
\begin{align}
    \vv{a} = \vv{\alpha} \times \vv{r} - \omega^{2}\vv{r}     
\end{align}
E nel moto circolare uniforme si riduce a $\vv{a} = -\omega^{2}\vv{r}$. \\
Le definizioni della velocità e dell'accelerazione angolari possono essere estese a traiettorie
non circolari e eventualmente anche non piane assumendo in tal caso che il punto materiale
si muova lungo un arco infinitesimo di un cerchio osculatore.

\subsection{Periodicità del moto circolare uniforme}
Dopo aver ruotato di $2\pi$ i vettori $\vv{r}$, $\vv{v}$ e $\vv{a}$ si ripetono
periodicamente, e proprio il moto circolare uniforme è l'esempio perfetto del \textbf{moto periodico}. 
Un moto è definito periodico se, in un dato sistema di riferimento a partire da un 
qualsiasi istante t le sue caratteristiche si ripresentano inalterate dopo un intervallo di
tempo $T$ chiamato \textbf{periodo}. Un moto periodico è quindi tale che $\forall t, n$:
\begin{gather*}
    \vv{r}(t + nT) = \vv{r}(t)  
\end{gather*}
Il periodo del moto circolare è:
\begin{align}
    T = \frac{2\pi}{|\omega|}
\end{align}
Infatti, passando da $t$ a $t + T$ i valori delle funzioni trigonometriche non cambiano
in quanto il loro argomento $\theta$ varia di $\omega T = \pm 2 \pi$. \\
La \textbf{frequenza} è il numero di periodi in un unità di tempo (in genere il secondo):
\begin{align}
    \nu = \frac{1}{T}
\end{align} 
mentre la \textbf{pulsazione} coincide con la velocità angolare nel moto uniforme e quindi
non ha validità generale:
\begin{align}
    \omega_0 = 2\pi\nu = \frac{2\pi}{T}
\end{align}

\subsection{Equazioni differenziali del moto circolare uniforme}
La relazione 
\begin{gather*}
    \vv{a} = \vv{a}_n = \vv{\omega} \times \vv{v} = -\omega^{2}\vv{r}      
\end{gather*}
si può anche esprimere come: 
\begin{gather*}
    \frac{d^{2}\vv{r} }{dt^{2} } = -\omega_0^{2}\vv{r}
\end{gather*}
dove la pulsazione corrisponde alle due equazioni:
\begin{gather*}
    \frac{d^{2}x(t) }{dt^{2} } = -\omega^{2}x(t); \qquad \frac{d^{2}y(t)}{dt^{2} } = -\omega_0^{2} y(t).
\end{gather*}
Tali equazioni hanno la medesima struttura del tipo:
\begin{gather*}
    \frac{d^{2}f(t)}{dt^{2}} + \omega^{2}_0 f(t) = 0
\end{gather*}
una funzione del tipo:
\begin{align}
    f(t) = A\cos(\omega_0t + \phi_0)
\end{align}
è l'integrale generale dell'\textbf{equazione differenziale}.

\section{Moto oscillatorio armonico}
Considerato un punto materiale che si muove lungo una retta, che scegliamo come asse x,
secondo la legge oraria:
\begin{align}
    x(t) = A\cos(\omega_0 t + \phi_0)
\end{align}
Dato che il coseno è un numero puro, $A$ ha le dimensioni di una lunghezza e prende
il nome di \textbf{ampiezza} (o \emph{elongazione}) , l'argomento del coseno $\phi$ prende il nome
di \textbf{fase}; $\phi_0$ è il valore all'istante iniziale ed è dunque la \textbf{fase iniziale},
la terza costante è invece la pulsazione.\\
Da questo, essendo il coseno periodico e dipendente da t, allora anche la funzione risultante
sarà periodica e quindi si ripete regolarmente da un qualsiasi istante t ad ogni periodo T e
poiché un cambiamento di fase di $2\pi$ corrisponde ad un intervallo di tempo $T$, allora 
\begin{gather*}
    T = \frac{2\pi}{\omega_0}
\end{gather*}
\begin{wrapfigure}{r}{0.4\textwidth}
    \centering
    \caption{Traiettoria del moto oscillatorio armonico}
    \begin{tikzpicture}[scale=1.3]
        \draw[-, dashed] (0, 0) -- (0.5, 0);
        \draw[->, dashed] (2.5, 0) -- (3, 0) node[at end, below] {$x$};
        \draw[-, red, very thick] (0.5, 0) -- (2.5, 0) node[at start, below] {$-A$} node[at end, below] {$A$};
        \filldraw (1.5, 0) circle (1pt) node[anchor = north] {$O$};
        \draw[->, thick] (0.6, 0.25) -- (1, 0.25) node[midway, above] {$a_x > 0$}; 
        \draw[<-, thick] (2, 0.25) -- (2.4, 0.25) node[above, midway] {$a_x < 0$}; 
    \end{tikzpicture}    
\end{wrapfigure}
Il moto in questione ha come legge oraria una funzione coseno di ampiezza A, ma il moto 
vero e proprio avviene su una traiettoria rettilinea. Per ogni valore di t si può calcolare:
\begin{align}
    \left\{\begin{array}{l}
        v_x = \dot{x}= -\omega_0 A \sin(\omega t + \phi_0) \\
        a_x = \ddot{x} = -\omega_0^2 A \cos(\omega_0 t + \phi_0) = - \omega_0^{2}x 
    \end{array}\right.
\end{align}
Velocità ed accelerazione hanno lo stessa Periodicità del moto ma sono sfasate in anticipo
rispettivamente di $\pi/2$ e $\pi$. A partire dalla soluzione generale dell'equazione differenziale
$\ddot{x} + \omega_0^{2} x = 0$ si può ricavare la soluzione particolare corrispondente alle assegnate condizioni iniziali 
attraverso le seguenti:
\begin{gather*}
    \left\{\begin{array}{l}
        x(0) = x_0 = A \cos\phi_0 \\
        v_x(0) = v_{0x} = - \omega_0 A \sin\phi_0
    \end{array}\right. \\
    \left\{\begin{array}{l}
        \tan\phi_0 = -\frac{v_{0x}}{\omega_0 x_0} \\
        A = \sqrt{x_0^{2} + \frac{v_{0x}^{2} }{\omega_0^{2} }} 
    \end{array}\right.
\end{gather*}  
Si parla di moto oscillatorio armonico anche su traiettoria qualunque, quando la funzione $s(t)$
soddisfa la corrispondente differenziale:
\begin{align}
    \ddot{s} + \omega_0^{2} s = 0
\end{align}
Ogni funzione periodica secondo il teorema di Fourier può essere espressa attraverso la serie:
\begin{align}
    f(t) = \frac{c_0}{2} + \sum_{n = 1}^{N} c_n \sin(n \omega_0 t + \phi_n)
\end{align}
E dunque ogni moto armonico complesso diventa la somma finita o infinita di moti armonici semplici.

\section{Moto piano in coordinate polari}
In questo paragrafo si vedrà come i sistemi di coordinate polari sono applicabili ai moti 
nel piano, in quanto, spesso, il sistema di coordinate polari è un metodo di gran lunga migliore per potere determinare
posizioni, velocità ed accelerazioni dei corpi. \\
Ogni vettore parallelo al piano può essere espresso in termini dei suoi versori $\hat{u}_r$ e $\hat{u}_\theta$ relativi al generico punto P chiamati 
\emph{componente radiale e trasversa}. Essendo il vettore posizione:
\begin{gather*}
    \vv{r} = r\hat{u}_r  
\end{gather*}  
e essendo $\vv{v} = \frac{d\vv{r} }{dt}$ allora:
\begin{gather*}
    \vv{v} = \frac{dr}{dt}\hat{u}_r + r\frac{d\hat{u}_r}{dt} = \frac{dr}{dt} \hat{u}_r + r\frac{d\theta}{dt}\hat{u}_\theta    
\end{gather*} 
$\vv{v}$ è esprimibile quindi come la somma vettoriale della componente radiale e trasversa:
\begin{gather*}
    \vv{v} = v_r \hat{u}_r + v_\theta\hat{u}_\theta   = \dot{r} \hat{u}_r + (r\dot{\theta}) u_{\theta}
\end{gather*}
Il modulo è quindi:
\begin{gather*}
    v = \sqrt{(\dot{r})^2 +(r\dot{\theta})^{2}  } 
\end{gather*}
Il significato delle componenti polari della velocità è il seguente: la componente
radiale dà la rapidità istantanea con cui cambia la distanza del punto dal polo mentre quella trasversa
è dovuta alla variazione di direzione del vettore posizione. Derivando la velocità come
componente radiale e trasversa si ottiene:
\begin{gather*}
    \frac{d\vv{v} }{dt} = \frac{dv_r}{dt}\hat{u}_r + v_r\frac{d\hat{u}_r }{dt} + \frac{dv_{\theta}}{dt}\hat{u}_{\theta} + v_{\theta}\frac{d\hat{u}_{\theta} }{dt} \\
    \frac{d\hat{u}_r }{dt} = \dot{\theta} \hat{u}_{\theta} \\
    \frac{d\hat{u}_{\theta} }{dt} = -\dot{\theta}\hat{u}_{r} \\
    \frac{dv_r}{dt} = \ddot{r} \\
    \frac{dv_{\theta}}{dt} = \ddot{\theta} \\
    \ \Longrightarrow \ \vv{a} = (\ddot{r} - r\dot{\theta}^{2} ) \hat{u}_r + (2\dot{r}\dot{\theta} + r\ddot{\theta}) \hat{u}_{\theta} 
\end{gather*}
Possiamo quindi indicare l'accelerazione come:
\begin{gather*}
    \vv{a} = a_r \hat{u}_r + a_\theta \hat{u}_\theta    
\end{gather*}
E quindi le sue componenti radiali e trasverse sono:
\begin{align}
    a_r =& \ \ddot{r} - r\dot{\theta}^{2} \\
    a_\theta =& \ 2\dot{r}\dot{\theta} + r \ddot{\theta} = \frac{1}{r}\frac{d}{dt}(r^{2}\dot{\theta} )   
\end{align}
Il modulo dell'accelerazione sarà dato dalla seguente
\begin{align}
    a = \sqrt{(\ddot{r} - r\dot{\theta}^{2} )^{2} + (2\dot{r}\dot{\theta} + r\ddot{\theta})^{2}} 
\end{align}

\section{Moto con accelerazione costante: La caduta dei gravi}
\begin{wrapfigure}{r}{0.4\textwidth}
    \centering
    \label{Fig 3.7}
    \caption{Parabola nel piano e traiettoria di un moto di grave generico}
    \begin{tikzpicture}
        \draw[->] (0, 0) -- (4, 0) node[at end, below] {$x$} node[at start, left] {$O$};
        \draw[->] (0, 0) -- (0, 4) node[at end, left] {$y$};
        \draw[red, very thick] (0, 0) .. controls (1.25, 4) and (1.75, 4) .. (3, 0);
        \draw[-, thin] (0.25, 0.75) arc (75:0:0.8) node[midway, right] {$\alpha$};
        \draw[->] (0, 0) -- (0.3, 1.5) node[at end, above] {$\vv{v}_0$};
        \filldraw (1, 2.55) circle (1pt) node[anchor=east] {$P$};
        \draw[->] (1, 2.55) -- (1, 1.55) node[at end, right] {$\vv{g}$};
        \draw[->] (1, 2.55) -- (1.5, 3.5) node[at end, left] {$\vv{v}$}; 
    \end{tikzpicture}    
\end{wrapfigure}
Tutti i corpi nella cinematica in caduta libera si muovono sotto l'effetto di una accelerazione,
ossia l'accelerazione di gravità che agisce solo sulla componente verticale della velocità di un corpo
(ovviamente trascurando la resistenza dell'aria). Si ha quindi che:
\begin{align}
    \vv{v}(t) &= \vv{v}(t_0) + \int_{t_0}^{t}\vv{g}\ dt' = \vv{v}_0 + \vv{g} \int_{t_0}^{t} dt' = \vv{v}_0 + \vv{g}(t - t_0) \\
    \vv{r}(t) &= \vv{r}(t_0) + \int_{t_0}^{t}\vv{v}(t')\ dt' = \vv{r}_0 + \vv{v}_0 (t - t_0) + \frac{1}{2}\vv{g}(t - t_0)^{2}.               
\end{align}
Si osservi che sia $\vv{v}$ sia il vettore $\vv{r} - \vv{r}_0$ si ottengono sommando due vettori che hanno
le direzioni di $\vv{v}_0$ e di $\vv{g}$. La traiettoria è dunque una curva che giace sul piano verticale. 
Nello studio del moto dei gravi è particolarmente convenitene utilizzare un sistema di coordinate
cartesiane e scrivere:
\begin{gather*}
    \vv{a} = \vv{g} = -g\hat{j}   
\end{gather*}     
All'istante generico t, se $\alpha$ è l'angolo della velocità all'istante iniziale:
\begin{gather*}
    \vv{v}(t) = (v_0 \cos\alpha)\hat{i} + (v_0 \sin\alpha - gt) \hat{j} \\
    \vv{r}(t) = (v_0 \cos\alpha)t \hat{i} + \left((v_0 \sin\alpha)t - \frac{1}{2}gt^{2}\right)\hat{j}       
\end{gather*}
Per le componenti cartesiane si ottiene la proiezione del moto del punto sull'asse delle ascisse
che si muove di moto rettilineo uniforme e la proiezione lungo l'asse y che è un moto uniformemente accelerato. 
Si rappresenta dunque il moto come \textbf{composizione di moti rettilinei indipendenti}:
\begin{align}
    \left\{\begin{array}{l}
        v_x = v_{0x} = v_0 \cos\alpha \\
        v_y = v_{0y} -gt = v_0\sin\alpha - gt\\
        v_z = 0
    \end{array}
    \right.; \left\{ \begin{array}{l}
        x = (v_0\cos\alpha)t \\
        y = (v_0\sin\alpha)t - \frac{1}{2}gt^{2} \\
        z = 0 
    \end{array}\right.
\end{align}

Eliminando ora il parametro t si ottiene invece la traiettoria in forma esplicita:
\begin{gather*}
    \left\{\begin{array}{l}
        t = \frac{x}{v_0\cos\alpha} \\
        y = (v_0 \sin\alpha)t - \frac{1}{2}gt^{2} 
    \end{array}\right.
\end{gather*}
\begin{align}
    y = (\tan\alpha)x - \left(\frac{g}{2v_0^{2}\cos^{2}\alpha}\right)x^{2} = \frac{v_{0y}}{v_{0x}}x - \frac{1}{2} \frac{g}{v_{0x}^{2}}x^{2}  
\end{align}

\section{Leggi di trasformazione della velocità ed accelerazione}
Cosa succede se le velocità e le posizioni sono in sistemi di riferimento diversi? 
In questo si deve operare una trasformazione che ci consente di trasformare le velocità 
e le accelerazioni tra sistemi di riferimento diversi. Preso un sistema di riferimento
S ed un sistema di riferimento S' in moto con una certa velocità rispetto a S, allora
\begin{gather*}
    \vv{r} = \vv{r'} + \vv{R}   
\end{gather*}
dove $\vv{R} = \vv{OO'}$, ossia la distanza tra i due sistemi di riferimento
e $\vv{r}'$ (e tutti gli altri con l'apice) il vettore posizione del
punto rispetto all'origine $O'$. supponendo lo spazio ed il tempo assoluti (non vero):
\begin{gather*}
    \vv{v} = \vv{v}' + \vv{v}_\tau   
\end{gather*}  
Dove $\vv{v}_\tau $ è chiamata \textbf{velocità di trascinamento}, e dipende esattamente
dalla posizione del punto materiale. Il significato è chiaro se preso un punto P solidale con
S' dove $\vv{v}' = 0$ e quindi la sua velocità è $\vv{v}_\tau$: il sistema in quiete $S$
vede il punto $P$ muoversi con una certa velocità rispetto al sistema $S'$ . La relazione
della velocità è chiamata \emph{legge di trasformazione classica delle velocità}.
\begin{gather*}
    \vv{v} = \left( \frac{d\vv{r} }{dt} \right)_S \quad \text{velocità del punto materiale in S}; \\
    \vv{v}' = \left( \frac{d\vv{r}' }{dt} \right)_S' \quad \text{velocità del punto materiale in S'}.
\end{gather*}  
Posto allora $\vv{V}(t)= \left( \frac{d\vv{R}}{dt}\right)_S$ allora la velocità
di trascinamento sarà:
\begin{align}
    \vv{v}_\tau = \vv{V}(t) + \vv{\omega}(t) \times (\vv{r}(t) -\vv{R}(t))   
\end{align} 
Dipende dal moto relativo dei due sistemi di riferimento e dalla posizione particolare
occupata dal punto mobile nell'istante considerato. Inoltre $\vv{v}_\tau$ non è la stessa per tutti i punti di S'.\\
Consideriamo due casi particolari:
\begin{enumerate}
    \item $\vv{\omega} = 0$: 
    In questo caso si ha che $\vv{v}_\tau$ è indipendente dalla posizione e tutti i punti di S'
    si muovono con la stessa velocità rispetto a S.
    \item $\vv{\omega} \neq 0$: 
    In questo caso $\vv{v}_\tau$ cambia da punto a punto ed è uguale alla somma vettoriale della velocità
    di $O'$ e del vettore $\vv{\omega} \times (\vv{r} - \vv{R})$. 
    \item $\vv{V}(t) = 0$: in questo caso $O'$ è fisso in S,
    per cui $\vv{v}_\tau = \vv{\omega} \times (\vv{r} -\vv{R})$. 
\end{enumerate}
Il caso più generale di quando si ha un moto di rotazione di S' rispetto ad S è il caso del
\textbf{moto di rototraslazione}, in questo caso l'asse lungo cui avviene la rotazione è considerabile
come luogo dei punti solidali con S' la cui velocità di trascinamento è uguale a $\vv{V}(t)$. \\
Per quanto riguarda invece l'accelerazione si ha che:
\begin{align}
    \vv{a} = \vv{a}' + \vv{a}_\tau + \vv{a}_{co}    
\end{align}
Dove $\vv{a}_\tau$ dipende dal moto relativo dei sistemi di riferimento e dalla posizione del punto materiale.  
mentre il terzo termine è presente se e solo se $\vv{v}'$ è diversa da zero.
\begin{align}
    \vv{a}_\tau = \vv{A} + \vv{\alpha} \times (\vv{r} - \vv{R} ) + \vv{\omega} \times (\vv{\omega} \times (\vv{r} - \vv{R} ))
\end{align} 
e
\begin{align}
    \vv{a}_{co} = 2\vv{\omega } \times \vv{v}'.       
\end{align}
Dove:
\begin{gather*}
    \vv{A} = \left( \frac{d^{2} \vv{R} }{dt^{2} } \right)_S \qquad \vv{\alpha} = \left( \frac{d\vv{\omega} }{dt} \right)_S  
\end{gather*}
Anche qui il termine $\vv{a_{\tau}}$ ha un significato simile a quello della velocità di trascinamento:
infatti rappresenta l'accelerazione relativa solidale al sistema di riferimento $S'$ se osservata dal sistema di riferimento $S$. L'accelerazione
ha tuttavia un'altro termine, ossia l'accelerazione di Coriolis, che è presente se e solo se un oggetto è 
in rotazione rispetto al sistema di riferimento inerziale $S$.
 

\section{Giustificazione delle relazioni della velocità ed accelerazione nei moti relativi}
Dati due sistemi di riferimento $S$ ed $S'$, in cui $S$ è il sistema di riferimento inerziale,
possiamo individuare il vettore $\vv{R}$ come il vettore $\vv{OO'}$, adesso
posso dire che un punto di $S'$ è individuato rispetto all'origine $O$ dal vettore
\begin{align}
    \vv{r} \equiv (\vv{r} - \vv{R}  )  +\vv{R} 
\end{align} 
In $S'$ si denota con $\vv{r'}$ il vettore che individua la posizione di un punto
rispetto all'origine $O'$; implicando due ipotesi fondamentali:
\begin{enumerate}
    \item Lo spazio sia \textbf{assoluto}: ossia indipendente dal sistema di riferimento
    \item Il tempo sia \textbf{assoluto}: ossia gli osservatori misurino $\Delta t \equiv \Delta t'$.
\end{enumerate}
E quindi possiamo definire il vettore $\vv{r'} = \vv{r} - \vv{R}  $.

\subsection{Derivate dei vettori nei moti relativi} 
Per poter dimostrare le relazioni della velocità e dell'accelerazione è
necessario prima dimostrare che i vettori possono essere derivati e che tipo di
derivata si ottiene. 
Considerato il generico vettore nei due sistemi di riferimento:
\begin{gather*}
    \vv{w} = w_x \hat{i} + w_y \hat{j} + w_z \hat{k} = w_x'  \hat{i'}  + w_y' \hat{j'} + w_z' \hat{k'}  
\end{gather*}
Per valutare adesso la derivata di questo vettore rispetto al tempo, possiamo partire dalla sua rappresentazione
in $S'$: ossia derivo rispetto al tempo i versori ed i moduli in $S'$ all'interno
del sistema di riferimento $S$:
\begin{gather*}
    \left(\frac{dw_x'}{dt}\right)_S = \left(\frac{dw_x'}{dt}\right)_S\hat{i'} + w_x'\left(\frac{\hat{i'} }{dt}\right)_S + \left(\frac{dw_y'}{dt}\right)_S \hat{j'} + w_y' \left(\frac{\hat{j'} }{dt}\right)_S + \left(\frac{dw_z'}{dt}\right)_S \hat{k'} + w_z' \left(\frac{\hat{k'} }{dt}\right)_S   
\end{gather*} 
avendo però la definizione di tempo assoluto:
\begin{gather*}
    \left(\frac{dw_x'}{dt}\right)_S = \left(\frac{dw_x'}{dt}\right)_{S'}
\end{gather*}
Dobbiamo ora valutare la derivata rispetto a
\begin{gather*}
    \left(\frac{d\vv{w} }{dt}\right)_{S'}
\end{gather*}
Posso ora analizzare cosa succede nelle derivate dei versori di $S'$ rispetto
a $S$, ossia come
\begin{gather*}
    w_x' \left(\frac{d\hat{i} }{dt}\right)_S
\end{gather*}
Adesso, se i versori non cambiano nel tempo (ossia non ruotano) e
dunque la loro derivata è nulla. Se i versori invece ruotassero rispetto nel
tempo allora le loro derivate sarebbe espresse come
\begin{gather*}
    \left(\frac{d\hat{i'}(t) }{dt}\right)_S = \vv{\omega_1} (t) \times \hat{i'}(t) \qquad \left(\frac{d\hat{j'}(t) }{dt}\right)_S = \vv{\omega_2}(t) \times \hat{j'}(t) \qquad \left(\frac{d\hat{k'}(t) }{dt}\right) = \vv{\omega_3}\times \hat{k'}(t)      
\end{gather*}
Esistendo allora questo vettore $\vv{\omega}(t)$ e quindi è possibile scrivere
per ciascuno dei versori
\begin{gather*}
    \left(\frac{d\hat{u}(t) }{dt}\right)_S = \vv{\omega}(t) \times \hat{u}(t)  
\end{gather*} 
Ciò discende dalla permanenza nel tempo della ortogonalità della terna
di versori di $S'$ e quindi $\vv{\omega}(t)$ caratterizza il moto relativo delle
due terne di assi associato al moto relativo dei due sistemi di riferimento. 
Allora si può dire che 
\begin{gather*}
    w_x'\left(\frac{d\hat{i'} }{dt}\right)_S = \vv{\omega} \times w_x' \hat{i'}  
\end{gather*}
E analoghe per gli altri termini dello stesso tipo, allora si ottiene che
la loro somma è esattamente uguale a $\vv{\omega} \times \vv{w}$ per le regole del
prodotto vettoriale ottenendo allora 
\begin{align}
    \left(\frac{d\vv{w} }{dt}\right)_S  = \left(\frac{d\vv{w} }{dt}\right)_{S'} + \vv{\omega} \times \vv{w}  
\end{align}


\subsection{Dimostrazione formula velocità relativa}
Per il vettore velocità nel riferimento $S$ possiamo scrivere che il vettore
velocità sarà allora dato dalla derivata di $\vv{r}$ e dunque
\begin{gather*}
    \vv{v} = \left(\frac{d\vv{r} }{dt}\right)_S = \left(\frac{d\vv{R} }{dt}\right)_{S} + \left(\frac{d\vv{r'} }{dt}\right)_S  
\end{gather*} 
Posso allora applicare la formula generale nel caso di $\vv{w} = \vv{r}$ come abbiamo dimostrato prima,
ottenendo che la velocità rispetto all'origine di $S'$ è
\begin{gather*}
    \left(\frac{d\vv{r'} }{dt}\right)_S = \left(\frac{d\vv{r'} }{dt}\right)_{S'} + \vv{\omega} \times \vv{r'}  
\end{gather*}  
Per cui si ottiene, dato $\vv{r}' = \vv{r} - \vv{R}$:   
\begin{gather*}
    \vv{v} = \left(\frac{d\vv{R} }{dt}\right)_S + \left(\frac{d\vv{r'} }{dt}\right)_S = \left(\frac{d\vv{r'} }{dt}\right)_{S'} + \vv{\omega} \times \vv{r'}   \\
    \vv{v} = \vv{v'} + \left(\frac{d\vv{R} }{dt}\right)_S + \vv{\omega} \times (\vv{r} - \vv{R}  )   
    \ \Longrightarrow \ \vv{v} = \vv{v'} + \vv{v}_{\tau}   
\end{gather*}
L'espressione della velocità di trascinamento dipenderà dal moto 
relativo dei due sistemi di riferimento e dalla particolare
posizione occupata dal punto mobile all'istante considerato.


\subsection{dimostrazione formula accelerazione relativa}
Per ottenere l'accelerazione nei sistemi di riferimento $S$ e $S'$, allora si 
deriva rispetto al tempo e si ottiene
\begin{gather*}
    \vv{a} = \left(\frac{d\vv{v} }{dt}\right)_S = \left(\frac{d\vv{v}' }{dt}\right)_S + \left(\frac{d^{2} \vv{R} }{dt^{2} }\right)_S + \left(\frac{d\vv{\omega} }{dt}\right)_S \times \vv{r}' + \vv{\omega} \times \left(\frac{d\vv{r}' }{dt}\right)_S    
\end{gather*}
Possiamo allora utilizzare la derivate dei vettori posso esprimere 
la derivata rispetto a $\vv{v}'$ come 
\begin{gather*}
    \left(\frac{d\vv{v}' }{dt}\right)_S = \left(\frac{d\vv{v}' }{dt}\right)_{S'}+ \vv{\omega} \times \vv{v}' = \vv{a}' + \vv{\omega} \times \vv{v}'     
\end{gather*} 
Si ottiene allora il secondo temine della derivata di $\vv{v}'$ come
\begin{gather*}
    \vv{\omega} \times \left(\left(\frac{d\vv{r}' }{dt}\right)_S + \vv{\omega} \times \vv{r}'  \right) = \vv{\omega} \times \vv{v}' + \vv{\omega} \times (\vv{\omega} \times \vv{r}'  )    \\
    \vv{a} = \vv{a'} + 2\vv{\omega} \times \vv{v'} + \left(\frac{d^{2} \vv{R} }{dt^{2} }\right)_S + \left(\frac{d\vv{\omega}}{dt} \right)_s \times (\vv{r} - \vv{R}  ) +  \vv{\omega} \times (\vv{\omega} \times (\vv{r} - \vv{R} )) \\
    \vv{a} = \vv{a'} + \left(\frac{d^{2} \vv{R} }{dt^{2} }\right)_S + \vv{\alpha}  \times (\vv{r} - \vv{R}  ) +  \vv{\omega} \times (\vv{\omega} \times (\vv{r} - \vv{R} )) + 2\vv{\omega} \times \vv{v'}      
\end{gather*}.

\part{Dinamica e lavoro}
\chapter{Principi della dinamica}
\section{Forza e definizione operativa}
Per definire operativamente una forza si introduce un dinamometro: una molla collegata
ad un estremo fissato ed una scala graduata a fianco che
misura la deformazione della molla dovuta dall'azione di una forza. \\
Per dare una definizione operativa devo dire quando due
grandezze sono uguali e come si sommano. $F_1$ e $F_2$ sono uguali
se producono la stessa deformazione della molla (che sia allungamento
o accorciamento). Dal momento che la forza ha una direzione (quella della
molla) e un verso ossia quello che determina l'accorciamento o
l'allungamento. Se io ho due forze, per far sì che queste siano vettori
io dico che la loro somma è un'altra forza che produce una deformazione
sulla molla pari alla somma delle loro deformazioni.

\section{Principi della dinamica}
Si può definire il sistema di riferimento inerziale come un sistema di riferimento orientato
secondo una terna cartesiana destrorsa di versori $\hat{i}, \hat{j}, \hat{k}$ nel quale valgono i
seguenti principi della dinamica:   
\begin{enumerate}
    \item \textbf{Primo principio}: Se la forza totale è uguale a zero, il corpo persevera nel suo stato di quiete
    o di moto rettilineo uniforme. Principio di esistenza non riconducibile al secondo
    valido in sistemi di riferimento inerziali. 
    \item \textbf{Secondo principio}: $\vv{F} = m\vv{a}$: per ogni corpo vale per forze diverse
    tra di loro \begin{gather*}
        \frac{|\vv{F_1}|}{|\vv{a_1}| } = \dots = \frac{|\vv{F_i}|}{|\vv{a_i}|} = m
    \end{gather*}  
    Ossia ogni forza applicata è proporzionale all'accelerazione che imprime su di un corpo di
    massa $m$; ne segue che non è universale in quanto dipende dal corpo (massa inerziale). 
    \item \textbf{Terzo principio} (Azione e reazione): Se vediamo $\vv{F}$ applicata ad un corpo allora esisterà una
    una forza uguale ed opposta (con stesso modulo e direzione ma verso opposto) 
\end{enumerate}


\section{Dinamica di Fletcher}
\begin{wrapfigure}{r}{0.4\textwidth}
    \centering
    \label{Fig 1.1}
    \caption{Dinamica di due corpi ed una carrucola.}
    \begin{tikzpicture}
        \draw (-1, 0) --(2, 0);
        \draw (2, 0) -- (2, -2);
        \draw (2, 0) -- (2.25, 0.25);
        \draw (2.35, 0.35) circle (0.15);
        \draw (0, 0) rectangle (1, 1) node[midway] {$M_1$};
        \draw (1, 0.5) -- (2.4, 0.5);
        \draw (2.5, 0.4) -- (2.5, -1.5);
        \draw (2.2, -1.5) rectangle (2.8, -2.1) node[midway] {$M_2$};
        \draw[->] (-0.5, 0) -- (-0.5, 0.5) node[at end, left] {$y$};
        \draw[->] (-0.5, 0) -- (0, 0) node[at end, below] {$x$};
        \draw[->] (2.2, -0.2) -- (2.7, -0.2) node[at end, right] {$y$}; 
        \draw[->] (2.2, -0.2) -- (2.2, -0.7) node[at end, below] {$x$};
    \end{tikzpicture}    
\end{wrapfigure}
Per lo studio di questo sistema prima di tutto si individua un sistema di riferimento 
inerziale: in questo caso si centra le assi del nostro sistema di riferimento
nel primo oggetto. Così si può approssimare il sistema di riferimento come se fosse
inerziale e solidale con la Terra entro certe approssimazioni. Disegniamo ora lo schema delle forze che agiscono sui corpi.
A questo punto possiamo impostare il sistema con tutte le forze (considerato che non si ha attrito) 
fissando un SdR inerziale vicino all'oggetto e ottenendo il seguente schema della forze:
\begin{gather*}
    \left\{ \begin{array}{l}
        \vv{N} = N\vv{j} \\
        m_1 \vv{g} = -m_1 g \vv{j} \\
        \vv{T} = T \vv{i}       
    \end{array}\right. 
\end{gather*}

\begin{wrapfigure}{r}{0.4\textwidth}
    \centering
    \caption{Schema delle forze corpo uno}
    \begin{tikzpicture}
    \draw (0, 0) -- (2, 0);
    \draw (0.5, 0) rectangle (1.5, 1) node[midway] {$M_1$};
    \draw[->] (1, 1) -- (1, 2) node[at end, left] {$\vv{N} $};
    \draw[->] (1.5, 0.5) -- (2.5, 0.5) node[at end, above] {$\vv{T} $};
    \draw[->] (1, 0) -- (1, -1) node[midway, right] {$m_1 \vv{g}$};
\end{tikzpicture}   
\end{wrapfigure}
Data la definizione delle forze, si può ora disegnare lo schema delle forze
e isolare i contributi lungo l'asse $x$ e lungo l'asse $y$:
\begin{gather*}
    \left\{\begin{array}{l}
        (y) \quad N - m_1 g = 0 \\
        (x) \quad T = m_1 \ddot{x_1}
    \end{array}\right. 
\end{gather*}
Adesso si scrive lo schema delle forze per quanto riguarda il secondo
corpo con il sistema di riferimento sotto la carrucola:
\begin{gather*}
    \left\{\begin{array}{l}
        \vv{T'} = -T'\hat{i} \\
        m_2 \vv{g} = m_2 g \hat{i}    
    \end{array}\right.\\
    \vv{T'} + m_2 \vv{g} = m_2 \vv{a}_2 \\
    -T' + m_2g = m_2 \ddot{x}_2    
\end{gather*}


\begin{wrapfigure}{r}{0.4\textwidth}
    \centering
    \caption{Schema delle forze corpo due}
    \begin{tikzpicture}
            \draw (5, 0) rectangle (6, 1) node[midway] {$M_2$};
    \draw[->] (5.5, 1) -- (5.5, 2) node[at end, right] {$\vv{T'}$};
    \draw[->] (5.5, 0) -- (5.5,  -1) node[midway, right] {$m_2 \vv{g} $};

    \end{tikzpicture}    
\end{wrapfigure}
A questo punto si fanno delle assunzioni per risolvere il problema: prima di tutto
il filo è ideale (non si estende né si deforma); inoltre ho considerato
le tensioni come se fossero uguali e da questo si ottiene che:
\begin{gather*}
    \ddot{x_1} = \ddot{x_2} \\
    |\vv{T}| = |\vv{T'}|\\
    T = T'  
\end{gather*}
Imposto ora il sistema tra le tensioni ottenendo:
\begin{gather*}
    \left\{ \begin{array}{l}
        T = m_2 \ddot{x_1} \\
        -T + m_2 g = m_2 \ddot{x_1}
    \end{array}\right.
\end{gather*}
Sommando si ottiene:
\begin{gather*}
    m_2 g = (m_1 + m_2) \ddot{x_1} \\
\end{gather*}
\begin{align}
    \ddot{x_1} = \frac{m_2 g}{m_1 + m_2}
\end{align}
Questo ci torna poiché l'inerzia è la somma delle masse e la forza
che agisce su questo oggetto è solo la forza peso del secondo oggetto.
Se ora tolgo $m_1$ allora $m_2$ è in caduta libera. Altrimenti se non
ci fosse $m_2$ allora $\ddot{x_1} = 0$ Si ottiene allora la tensione
moltiplicando per $m_1$:
\begin{align}
    \boxed{T = m_1 \ddot{x_1} = \frac{m_1 m_2 g}{m_1 + m_2}}
\end{align}
Adesso integriamo quella di prima:
\begin{gather*}
    \dot{x}_1 = \frac{m_2 g}{m_1 + m_2} t + a \\
    x_1 = \frac{m_2g}{2(m_1 + m_2)}t^{2} + at + b  
\end{gather*}
La posizione fisicamente è descritta con queste formule ma sono io che
decido dove mettere l'origine degli assi cartesiani e se io decido che
l'origine di essi è dove si trovava $m_1$ al tempo $ t = 0$ allora $b = 0$
altrimenti b diventa la distanza dall'origine della massa al tempo $t = 0$.
La macchina di Fletcher permette di eguagliare le tensioni ai due capi di una 
fune passata in una carrucola (entrambe però prive di massa). 

\subsection{Il filo ideale}
\begin{wrapfigure}{r}{0.4\textwidth}
    \centering
    \caption{Filo ideale}
    \begin{tikzpicture}
        \draw[<-] (0, 0) -- (1, 0) node[midway, above] {$\vv{T}$};
        \draw[-, ultra thick] (1, 0) -- (3, 0);
        \draw[->] (3, 0) -- (4, 0) node[midway, above] {$\vv{T}'$};
    \end{tikzpicture}    
\end{wrapfigure}
Il concetto di filo ideale in fisica presuppone che se si applica una forza
da entrambe le parti del filo di pari intensità allora non si ha alcuna deformazione.
Nel caso del filo da solo allora, assumendo:
\begin{gather*}
    m_{filo} \sim 0 \qquad \vv{T}  + \vv{T}' = m_{filo} \vv{a} = 0  
\end{gather*}
Nel caso della dinamica di Fletcher invece si ha che lo stesso filo ideale e
le stesse tensioni hanno però somma diversa da zero in quanto la carrucola riesce
in qualche modo ad "agire" sul filo con una forza diversa da zero
dal punto di vista fisico il sistema si esprime quindi con le seguenti:
\begin{gather*}
    d\vv{N} + \vv{T}_{\phi} + \vv{T}'_{\phi} = m_{filo}\vv{ a} \sim 0    
    \left\{\begin{array}{l}
        dN - T_{\phi} \sin\left(\frac{d\phi}{2}\right) - T_{\phi}
    \end{array}\right. 
\end{gather*}


\subsection{Macchina di Atwood}
\begin{wrapfigure}{r}{0.4\textwidth}
    \centering
    \caption{Macchina di Atwood}
    \begin{tikzpicture}
        \draw (0, 0) circle (1);
        \draw (-1, 0) -- (-1, -1.5);
        \draw (1, 0) -- (1, -2.5);
        \draw[->] (0, 0) -- (0, 0.5) node[at end, left] {$y$};
        \draw[->] (0, 0) -- (0.5, 0) node[at end, below] {$x$};
        \draw (-1.25, -1.5) rectangle (-0.75, -2) node[midway] {$M_1$};
        \draw (0.75, -2.5) rectangle (1.25, -3) node[midway] {$M_2$};
        \draw[->] (-1, -1.5) -- (-1, - 1) node[at end, left] {$\vv{T}$};
        \draw[->] (-1, -2) -- (-1, -2.5) node[at end, right] {$m_1\vv{g}$};
        \draw[->] (1, -2.5) -- (1, -2) node[at end, left] {$\vv{T}'$};
        \draw[->] (1, -3) -- (1, -3.5) node[at end, right] {$m_2\vv{g}$};
    \end{tikzpicture}    
\end{wrapfigure}
La macchina di Atwood è forse una delle macchine più importanti della fisica
classica poiché è riuscita a dare una spiegazione al secondo principio della
dinamica. Considerando una semplice macchina di Atwood composta da una
sola carrucola e due masse per ogni estremità della fune, si hanno 2 gradi di libertà
anche se, complessivamente, se ne
hanno quattro in quanto si ha anche la possibilità di oscillare. Il filo
in questo caso si appoggia per metà giro alla carrucola; attraverso un'accorta
preparazione dell'esperimento si possono ridurre i gradi di libertà a due eliminando 
le oscillazioni e fissando la carrucola il più stabilmente possibile
e cercando di non far oscillare le masse.
Utilizzando un filo ideale, le posizioni delle masse sono legate in 
modo tale che il filo possa scendere dalla parte della massa più grande riducendo
quindi il grado di libertà a 1. In poche parole sappiamo a priori che il sistema si muoverà
solamente verso la massa più pesante. Posso allora applicare i principi della dinamica 
per risolvere il problema:
\begin{gather*}
    1: \quad \vv{T} + m_1 \vv{g} = m_1 \vv{a}_1 \\
    2: \quad \vv{T}' + m_2 \vv{g} = m_2 \vv{a}_2      \\
    \left\{\begin{array}{l}
        T - m_1 g = m_1 \ddot{y}_1 \\
        T' - m_2g = m_2 \ddot{y}_2 
    \end{array}\right. \\
    \text{Posto} \ T = T' \qquad  \ddot{y}_2 = -\ddot{y}_1 
\end{gather*}
Si ha l'accelerazione per il primo corpo:
\begin{align}
    \boxed{\ddot{y_1} = \frac{(m_2 - m_1)g}{m_1 + m_2}}
\end{align}
Se $m_2 >> m_1$ allora $\ddot{y}_1 = g$. Se invece $m_1 >> m_2$ allora
$\ddot{y}_1 = -g$. Si trova ora la tensione come
\begin{align}
    T = m_1 g + m_1 \ddot{y}_1 \ \Longrightarrow \ \boxed{T =\frac{2m_1m_2}{m_1 + m_2}g}
\end{align}

\section{La molla ideale}
\begin{wrapfigure}{r}{0.4\textwidth}
    \centering
    \caption{La molla ideale}
    \begin{tikzpicture}
        \draw (0, 0) -- (0.5, 0);
        \draw (0.5, 0) -- (0.5, -1);
        \draw[decoration={aspect=0.3, segment length=4mm, amplitude=1mm,coil},decorate,opacity=0.9] (0.5, -0.5) -- (2.5,-0.5);
        \draw (2.5, -1) rectangle (3.25, -0.25) node[midway] {$M$};
        \draw (0.5, -1) -- (4, -1);
        \draw[<->] (0.75, -1.25) -- (1.75, -1.25) node[midway, below] {$l_r$};
        \draw[->, thick] (2.5, -0.6) -- (2, -0.6) node[at end, below] {$\vv{F}_d$};
        \draw[->, thick] (2.85, -0.25) -- (2.85, 0.25) node[at end, right] {$\vv{N} $};
        \draw[->, thick] (2.85, -1) -- (2.85, -1.5) node[at end, right] {$m \vv{g}$};
        \filldraw(0.5, -0.5) circle (1pt) node[anchor = east] {$O$};
        \filldraw(1.75, -0.5) circle (1pt) node[anchor = south] {$O'$};
    \end{tikzpicture}  
\end{wrapfigure}
Una molla è in posizione di riposo quando su di essa non agiscono forze esterne
(si dimostrerà più avanti che la somma delle forze interne è pari a
zero in qualsiasi sistema), pertanto se è sollecitata da forze esterne,
la molla imprimerà una forza di \textbf{richiamo} (perché tende a riportare
la molla stessa alla posizione di lunghezza di equilibrio) definita come:
\begin{align}
    \vv{F} = -k(x - l_r)\hat{i}  
\end{align}
Dove $k$ è chiamata \textbf{costante di elasticità}, ed è una caratteristica propria
della molla, $x$ è la posizione dell'oggetto rispetto all'origine della molla centrato in
$O$ e $l_r$ è la \textbf{lunghezza a riposo della molla}, ossia la lunghezza che ha la molla 
rispetto all'origine quando non è sollecitata da alcuna forza.
La molla è quindi  un apparato che tende ad esercitare sempre una forza che  la fa tornare sempre
alla posizione di riposo. Lo schema delle forze del sistema si rappresenta
come segue, ponendo ora il sistema di riferimento inerziale proprio nel punto in cui la molla
è a riposo $O'$. Dal punto di vista delle forze quindi:
\begin{gather*}
    \vv{F} = -kx\hat{i} \ \Longrightarrow \  -kx = m \ddot{x} 
\end{gather*}
Di conseguenza si ottiene la legge oraria di un moto attaccato ad una molla:
\begin{align}
    \boxed{m\ddot{x} + kx = 0}  
\end{align}
Ossia l'equazione del moto armonico. Questa equazione differenziale del secondo
ordine omogenea non è integrabile facilmente, comunque si ottiene: 
\begin{gather*}
    \ddot{x} + \frac{k}{m}x = 0 \ \Longrightarrow \ \ddot{x} + w^{2}x = 0   
\end{gather*}
Con $\omega^{2} = \frac{k}{m}$. Allora possiamo, data la soluzione generale di questa
equazione differenziale, ottenere a sistema le espressioni per la velocità 
e l'accelerazione:
\begin{gather*}
    \left\{\begin{array}{l}
        x = A \cos(\omega t + \phi) \\
    \dot{x} = -A\sin(\omega t + \phi)\omega \\
    \ddot{x} = -A \cos(wt + \phi)\omega^{2} 
    \end{array}\right.
\end{gather*}
Ottenendo ora l'equazione di moto come
\begin{gather*}
    -A \cos(\omega t + \phi)\omega^{2} + \omega^{2} A \cos(\omega t + \phi) = 0
\end{gather*}


\subsection{Il caso specifico: una molla attaccata al soffitto}
\begin{wrapfigure}{r}{0.45\textwidth}
    \centering
    \caption{Sistema con molla attaccata al soffitto}
    \begin{tikzpicture}
        \draw(0, 0) -- (3, 0);
        \draw[decoration={aspect=0.3, segment length=4mm, amplitude=1mm,coil},decorate] (1.5, 0) -- (1.5, -2);
        \draw (1, -2) rectangle (2, -3);
        \draw[->, very thick] (1.5, -2) -- (1.5, -1) node[midway, right] {$\vv{F}_e$};
        \draw[->] (1.5, -3) -- (1.5, -4) node[at end, right] {$m\vv{g} $};
        \draw[decoration={aspect=0.3, segment length=1mm, amplitude=1mm,coil},decorate] (1, 0) -- (1, -1);
        \draw[|-|] (0.5, -1) -- (0.5, -2) node[midway, right] {$\Delta l$};
    \end{tikzpicture}    
\end{wrapfigure}
Scegliendo un sistema di riferimento nel punto in cui la molla è
a riposo, posso definire i versori dei vari vettori e ottenere la seguente relazione
(dalla definizione) per la forza elastica:
\begin{gather*}
    \vv{F_e} = k \Delta l \hat{j}   
\end{gather*}
Gli altri vettori posso esprimerli come segue: 
\begin{gather*}
    \vv{g} = -g\hat{j} \qquad \vv{a} = \ddot{y} \hat{j}
\end{gather*}
E quindi posso ricavare dall'equazione differenziale della molla ideale
\begin{gather*}
    k\Delta l = - mg = m \ddot{y}   
\end{gather*}
Se si chiama allora $\Delta l = -y$ come l'abbassamento di quota dovuto alla massa
che è attaccata alla molla; si ottiene l'equazione del moto armonico non omogeneo:
\begin{gather*}
    -k y -mg = m\ddot{y} \\
    m\ddot{y} + ky = -mg
\end{gather*}
Per risolvere questa equazione differenziale posso operare attraverso una
sostituzione di variabile nella seguente maniera:
\begin{gather*}
    y(t) = \zeta (t) + \hat{y} \Rightarrow \ddot{y} (t) = \ddot{\zeta}(t)  \\
    m\ddot{\zeta} + k\zeta + k \hat{y} = -mg \\
    \text{Posto} \ \hat{z} = -\frac{mg}{k} \\
    m \ddot{\zeta} + k \zeta = 0   
\end{gather*}
Quando $\zeta$ è zero, allora si ritorna alla situazione in cui l'origine del sistema
di riferimento è proprio la posizione di equilibrio. 
\begin{align}
    y(t) = A\cos(\Omega t + \phi)- \frac{mg}{k}
\end{align}
Derivando si ottiene :
\begin{align}
    \dot{y} (t) = -\Omega A \sin(\Omega t + \phi) 
\end{align}
E quindi si hanno le impostazioni delle soluzioni:
\begin{align}
    \left\{\begin{array}{l}
        A \cos\phi = \frac{mg}{k} \\
        A \sin\phi = 0
    \end{array}\right. \Rightarrow \left\{\begin{array}{l}
        A = \frac{mg}{k} \\
        \phi = 0
    \end{array}\right.
\end{align}
Poiché $A = 0$ non ci da alcuna informazione sul moto .
\begin{gather*}
    y(t) = \frac{mg}{k} \left( \cos(\Omega t)  - 1\right)
\end{gather*}
\begin{align}
    y_{max} &= -2\frac{mg}{k} \\
    \dot{y}_{max} &= \frac{mg}{k} \Omega = g \sqrt{\frac{m}{g}}  
\end{align}
Dopo quanto  tempo si raggiunge la quota massima:
\begin{gather*}
    \Omega t_{min} = \pi \\
    t_{min} = \frac{\pi}{\Omega} = \frac{T}{2}
\end{gather*}
Quando si raggiunge il massimo modulo per la prima volta:
\begin{gather*}
    t_{max} = \frac{\pi}{2\Omega} = \frac{\text{Periodo}}{4} 
\end{gather*}


\section{Moto del pendolo semplice (o pendolo matematico)}
\begin{wrapfigure}{r}{0.4\textwidth}
    \centering
    \caption{Un pendolo semplice}
    \begin{tikzpicture}
        \draw(0, 0) -- (1, -2);
        \filldraw (1, -2) circle (1pt);
        \draw[dashed] (0, 0) -- (0, -2.5) node[midway, left] {$l$};
        \draw[dashed](1, -2) arc (-60: -90: 2);
        \draw[->](1, -2) -- (1, -3) node[at end, below] {$m\vv{g}$ };
        \draw[->] (1,-2) -- (1.5, -1.75) node[at end, above] {$\hat{u}_t$};
        \draw[dashed] (0.5, -1) arc (-60:-90:1) node[midway, below] {$\theta$};
        \draw[->, thick](1, -2) -- (0.5, -1) node[at end, above] {$\vv{N} $};
        \draw[->] (1, -2) -- (1.5, -3) node[at end, right] {$\vv{g}_r$};
        \draw[->] (1, -2) -- (0.5, -2.25) node[at end, below] {$\vv{g}_t$}; 
        \filldraw (0, - 2.25) circle (1pt) node[anchor = north] {$A$};
    \end{tikzpicture}    
\end{wrapfigure}
Un pendolo è costruito con un filo ed una massa sufficientemente piccola
da poter essere approssimata ad un punto materiale. Dal punto di vista matematico
qualunque punto materiale su di una traiettoria circolare su di un piano
verticale si comporta come la massa nel pendolo.  Questo accadrà se e solo se il filo è completamente teso e
se il punto materiale è vincolato in qualche modo tale da poter solo
ruotare sul piano verticale senza che cada verticalmente (ovviamente senza attrito).
La risultante in questo caso sarà: 
\begin{gather*}
     \vv{ N} + m\vv{g } = m\vv{a}   
\end{gather*}
Chiamiamo i versori $\hat{u}_n$ il versore di $\vv{N}$ e $\hat{u_t}$ il versore 
della velocità tangenziale. Se il punto è attaccato ad un filo rigido allora si ha che $N \geq 0$,
inoltre se è una guida o un asticella allora $N \equiv R$ ossia la reazione
vincolare della guida. 
Si decompone ora la forza peso nella componente radiale e tangenziale
ottenendo la seguente relazione
\begin{gather*}
    m \vv{g} = -mg\cos\theta \hat{u}_n -mg\sin\theta \hat{u}_t  
\end{gather*}
Data l'accelerazione
\begin{gather*}
    \vv{a} = \ddot{s}\hat{u}_t + \frac{\dot{s}^{2} }{l}\hat{u}_n   
\end{gather*}
Dividendo le componenti tangenziali e normali si ottengo le relazioni:
\begin{gather*}
        \left\{\begin{array}{l}
        -mg \sin\theta = m \ddot{s} \\
        -mg\cos\theta + N = \frac{\dot{s}^{2}  }{l} 
    \end{array}\right.
\end{gather*}
Noi però sappiamo che dalla definizione degli archi con gli angoli in radianti:
\begin{gather*}
        s = \theta l  \ \Longrightarrow \  \dot{s} = \dot{\theta}l \ \Longrightarrow \   \ddot{s} = \ddot{\theta} l  
\end{gather*}
Quindi possiamo riscrivere le due equazioni con le nuove sostituzioni:
\begin{gather*}
        \left\{\begin{array}{l}
        ml \ddot{\theta} = -mg\sin\theta \\
        N =ml\dot{\theta}^{2} + mg\cos\theta    
    \end{array}\right.
\end{gather*}
Il primo membro della seconda è sicuramente positiva e quindi sarà
soddisfatta se e solo se $-\frac{\pi}{2} < \theta < \frac{\pi}{2}$ altrimenti
sarà negativo. Se l'oggetto nel movimento si ferma, allora non arriverà mai sopra il corpo
del pendolo poiché si azzererebbe $\dot{\theta}$: fisicamente si spiega poiché il filo si affloscia.
Posso allora risolvere la prima equazione ed ottenere l'equazione differenziale
che mi rappresenta il moto di un pendolo semplice in assenza di attrito:
\begin{gather*}
    \ddot{\theta} + \frac{g}{l}\sin\theta = 0
\end{gather*} 
Per piccole oscillazioni, ossia $\theta(t) \leq \frac{\pi}{20}$ si può applicare lo sviluppo di Taylor
cosicché $\sin\theta \approx \theta$ e quindi si ottiene la differenziale di un moto
armonico:
\begin{align}
    \ddot{\theta} + \frac{g}{l}\theta = 0.
\end{align}
Il moto è armonico solo per questa approssimazione, altrimenti non è
armonico. Posto
\begin{gather*}
     \Omega = \sqrt{\frac{g}{l}} 
\end{gather*}
il periodo sarà dato da
\begin{align}
    T = \frac{2\pi}{\Omega} = 2\pi \sqrt{\frac{l}{g}} 
\end{align}
L'isocronismo del pendolo funziona se e solo se si compiono piccole oscillazioni,
altrimenti il periodo è un'altra cosa. Se l'oscillazione è più grande si trova una formula
in sviluppo in serie (senza dimostrazione poiché si fa a lab) che è:
\begin{align}
    T = 2\pi \sqrt{\frac{l}{g}}\left( 1 + \left( \frac{1}{2} \right)^{2} \sin^{2} \left( \frac{\theta}{2} \right) 
    + \left( \frac{13}{24} \right)^{2} \sin^{4} \left( \frac{\theta}{2} + ..\right) \right) 
\end{align}


\section{Caduta di un  filo con massa}
\begin{wrapfigure}{r}{0.4\textwidth}
    \centering
    \caption{Filo con massa}
    \begin{tikzpicture}
        \draw(0, 0) -- (3, 0);
        \draw(3, 0) -- (3,-3);
        \draw(1, 0) -- (1, 0.5);
        \draw(1, 0.5) -- (3.5, 0.5);
        \draw(3.5, 0.5) -- (3.5, -1.5);
        \draw(3.5, -1.5) -- (3, -1.5);
        \draw[->](3.25, -1.5) -- (3.25, -2.5) node[at end, right] {$\lambda x \vv{g}$};
        \draw[->](3.25, 0) -- (3.25, -0.5) node[at end, below] {$x$};
        \draw[->](3.25, 0) -- (3.75, 0) node[at end, right] {$y$};
    \end{tikzpicture}    
\end{wrapfigure}
Assumendo che il filo abbia una densità costante, fissato un sistema di riferimento
allora noi sappiamo che i due pezzi di filo hanno una certa dimensione.
La massa del filo su cui agisce la forza peso sarà densità per lunghezza, ossia:
\begin{gather*}
    x \lambda \ \Longrightarrow \ \lambda = \frac{M}{L}
\end{gather*}
Per cui la forza peso sul filo che ciondola è:
\begin{gather*}
    \vv{F} = x\lambda\vv{g} 
\end{gather*}
E quindi si ottiene per il secondo principio:
\begin{gather*}
    M\ddot{x} = \lambda x g 
\end{gather*}
Si ottiene allora l'espressione dell'accelerazione del filo come:
\begin{gather*}
    \ddot{x} = - \frac{g}{L}x = 0
\end{gather*}
Chiamato allora $\beta^{2} = \frac{g}{L}$, si ottiene una equazione simile
a quella della molla anche se non si può applicare seno e coseno come risoluzione dell'
equazione
\begin{gather*}
    \ddot{x}-\beta^{2} x = 0 \\
    x = Ae^{\beta t} + Be^{-\beta t} \\
    \dot{x} = Ae^{\beta t} \beta - Be^{-\beta t} \beta \\
    \ddot{x} = Ae^{\beta t}\beta^{2} + Be^{-\beta t} \beta^{2} = \beta^{2} x           
\end{gather*} 
Il solo segno meno ha cambiato completamente l'equazione. Ora ricavo $A$ e $B$ dalle condizioni
iniziali:
\begin{gather*}
    \left\{ \begin{array}{l}
        x = x_0 \\
        \dot{x} = 0 
    \end{array}\right.
\end{gather*}
Allora si ha che:
\begin{gather*}
    x(0) = x_0 = A + B \\
    \dot{x}(0) = 0 = \beta(A - B)
\end{gather*}
Quindi le nostre soluzioni sono proprio il coseno e seno iperbolici:
\begin{gather*}
    x = x_0 \frac{e^{\beta t} + e^{-\beta t}}{2} = x_0 \cosh\beta t
\end{gather*}
Se $\beta t << 1$, allora si ha che in $ t = 0$ e $ x = x_0$ si può sviluppare
con Taylor i membri del coseno iperbolico:
\begin{gather*}
    e^{\beta t} = 1 + \beta t + \frac{1}{2}\beta^{2}t^{2} \\
    e^{-\beta t} = 1 - \beta t + \frac{1}{2}\beta^{2}t^{2}
\end{gather*}
\begin{gather*}
    x = x_0 \left(1 + \frac{1}{2}\beta^{2}t^{2}\right)
\end{gather*}
In $t = 0$ allora il moto è uniformemente accelerato, se invece 
$t >> 0$, diventa un moto esponenziale, ossia le leggi che vincolano il moto
fanno crescere la velocità esponenzialmente. Questo vuol dire che il corpo continua ad aumentare la sua velocità?
No poiché una volta che il filo è finito allora continuerà a cadere come
tutti i corpi e quindi:
\begin{gather*}
    x = \frac{x_0}{2}e^{\beta t} 
\end{gather*}

\subsection{Focus su equazioni differenziali lineari omogenee con coefficienti costanti}
Un'equazione del genere è:
\begin{align}
    a_1\ddot{x} + a_2 \dot{x} + a_3 x = 0  
\end{align}
Un'equazione tipo:
\begin{gather*}
    x = Ae^{\alpha t} 
\end{gather*}
se si sostituisce nella differenziale si ottiene
\begin{gather*}
    Ae^{\alpha t} (a_1\alpha^{2} + a_2\alpha + a_3) = 0
\end{gather*}
La soluzione è data dalla risoluzione dell'equazione di secondo grado,
il che ci porta a dare la soluzione della differenziale di prima che contiene però due soluzioni:
una parte reale data dal coseno ed una parte immaginaria data dal seno:
\begin{gather*}
    \cos\omega t = \frac{e^{\omega t} + e^{-\omega t} }{2} \\
    \sin\omega t = \frac{e^{\omega t} + e^{-\omega t} }{2i}
\end{gather*}
Se il determinante dell'equazione di secondo grado si annulla allora si ottiene
la soluzione.
\begin{gather*}
    x = (A + Bt)e^{\beta t} 
\end{gather*}

\section{SDR non inerziali e esempi}
Accade spesso che la scelta di un SDR non inerziale sia fisicamente
più chiara e semplice per descrivere alcune situazioni fisiche e vale quindi la pena di discuterlo. Bisogna cambiare
però il modo in cui si utilizzano le leggi della dinamica:
con le leggi della trasformazione si era già visto la trasformazione
dell'accelerazione e della velocità dei SDR non inerziali, definite quindi
l'accelerazione di trascinamento e di Coriolis per cui si esprime ( si chiama SDR mobile il SDR non inerziale):
\begin{gather*}
    \vv{a} = \vv{a'} + \vv{a}_t + \vv{a}_{co}    
\end{gather*}
Posso esprimere la risultante delle forze nel sistema di riferimento 
$S'$ trovando quindi l'espressione per la prima cardinale nei SdR non inerziali: 
\begin{gather*}
    m\vv{a'} =m(\vv{a} - \vv{a}_t - \vv{a}_{co}   )  
\end{gather*} 
L'osservatore $S$ osserva che non gli torna l'accelerazione e dunque
ipotizza che ci siano delle accelerazioni (e quindi delle forze) immaginarie
che non riesce per qualche motivo a vedere ma che agiscono e modificano
il moto. 
La forza di trascinamento e la forza di Coriolis possono essere espresse come
\begin{gather*}
    \vv{F}_t = -m\vv{a}_t, \qquad \vv{F}_{co} = -m \vv{a}_{co}    
\end{gather*}
L'osservatore di $S$ riprendendo la definizione operativa di forza col
dinamometro osserva che quelle forze apparenti si comportano come
delle forze vere anche se non sono forze dovute ad interazione e non
rispettano dunque il primo né tantomeno il terzo principio. La forza di trascinamento in particolare
dipende proprio dal sistema di riferimento e sostanzialmente questa
formula diventa il nuovo secondo principio della dinamica:
\begin{align}
    \boxed{m\vv{a}' = \vv{F} + \vv{F}_t + \vv{F}_{co} }   
\end{align}
Dati i riferimenti di $S$ e $S'$ allora considerate le componenti dell'accelerazioni
e la distanza dal punto P che vogliamo misurare si ottengono le espressioni
per le forze di trascinamento e di Coriolis del punto $P$ rispetto alle origini dei sistemi
di riferimento:
\begin{gather*}
    \vv{F}_t = -m\vv{a}_t = -m \vv{a}_{O} -m\dot{\vv{\omega} } \times (P - O') -m\vv{\omega} \times(\vv{\omega} \times (P - O) ) \\
    \vv{F}_{co} = - m\vv{a}_{co} = -2m\vv{\omega} \times \vv{v}'  
\end{gather*}

\subsection{Il caso del pendolo sul treno}
\begin{wrapfigure}{r}{0.4\textwidth}
    \centering
    \caption{Pendolo nel treno}
    \begin{tikzpicture}
        \filldraw(0, 0) circle (1pt);
        \draw(0, 0) -- (1, 2);
        \draw[->, very thick](0, 0)  -- (0.5, 1) node[at end, left] {$\vv{T}$};
        \draw(0, 0) -- (-0.5 , -1);
        \draw[->, very thick](0, 0) -- (-1, 0) node[at end, above] {$-m\vv{a}_t$};
        \draw[->, very thick](0, 0) -- (0, -1) node[at end, right] {$m\vv{g}$};
        \draw[dashed](1, 2) -- (1, -1);
        \draw(1, 1) arc (270: 245: 1) node[midway, below] {$\alpha$};
    \end{tikzpicture}    
\end{wrapfigure} 
Con $\vv{\omega} = 0$ e $\vv{a_{O}}$ costante, l'osservatore su di un treno osserva che il dinamometro che utilizza per 
misurare l'accelerazione di un corpo risulta modificato: l'accelerazione del treno
produce una forza contraria al verso dell'accelerazione che prende il nome di
\textbf{forza di trascinamento}.
Se al posto del dinamometro mettessi un pendolo, nel caso in cui il treno 
abbia la stessa $\vv{a}_t$ e lasciassi oscillare il pendolo otterrei lo stesso effetto:
il pendolo sembra oscillare maggiormente da una parte (quella con verso opposto al moto
del treno) rispetto all'altra. Delle diverse forze in gioco solo quella di trascinamento fa inclinare il pendolo
all'indietro.
Impostando il problema dal punto di vista trigonometrico si ottiene:
\begin{gather*}
    \sin\alpha = \frac{\left| \vv{a}_t  \right| }{\sqrt{g^{2} + a_t^{2} } }
\end{gather*}
Sono in grado di definire il periodo di questo pendolo con accelerazione costante?
Utilizzo $\vv{g}'$ dato dalla composizione della forza peso e dall'accelerazione
di trascinamento, il cui modulo sarà:
\begin{align}
    \left| \vv{g}'  \right| = \sqrt{g^{2} + a_t^{2}}  
\end{align}


\subsection{Ascensore accelerato}
\begin{wrapfigure}{r}{0.4\textwidth}
    \centering
    \caption{L'ascensore accelerato}
    \begin{tikzpicture}
        \draw(0, 0) rectangle (2, 3);
        \draw[->, thick](0, 0) -- (0.5, 0);
        \draw[->, thick](0, 0) -- (0, 0.5);
        \draw[->] (-0.5, 0) -- (-0.5, 1) node[at end, left] {$\vv{a}_{O'}$};
        \filldraw (1, 1) circle (1pt) node[anchor = west] {$P$};
        \draw[->](1, 1) -- (1, 0.5) node[at end, left] {$\vv{F}_t$} node[at end, right] {$m\vv{g}$};
    \end{tikzpicture}    
\end{wrapfigure}
Anche nel caso dell'ascensore accelerato dovrò  considerare i contributi della
forza di trascinamento del sistema di riferimento $S'$ (ossia il sistema solidale con la base
dell'ascensore); data allora la formula della forza di trascinamento che si è
vista prima:
\begin{gather*}
        \vv{F}_t = -m\vv{a}_{O'} - m\vv{\omega} \times (\vv{\omega} \times (P - O)) - m\vv{\omega} \times (P - O)
\end{gather*}
Se l'ascensore sta accelerando verso l'alto, allora accade che io mi senta
schiacciato verso il basso a causa del segno della forza di trascinamento e,
poiché non c'è componente rotatoria, si ottiene:
\begin{gather*}
    \vv{F}_t = -m\vv{a}_{O'} 
\end{gather*}
In questo caso la forza di trascinamento si somma alla forza peso e quindi
si ottiene una forza peso nuova data da:
\begin{align}
    m\vv{g}' = \vv{F}_t + m\vv{g} = m(a_{O'} + g)\hat{g} 
\end{align}
Lungo il versore $\hat{g}$ che è comune ad entrambe e le forze.
Nel caso in cui l'ascensore stia invece scendendo (rispetto ad un SdR inerziale) con una certa accelerazione,
il nuovo peso effettivo diminuisce in quanto è dato 
dalla relazione:
\begin{align}
    m\vv{g}' = m(-|a_{O'}| + g) 
\end{align}
Se invece fosse in caduta libera allora dato che $\vv{a}_{O'} = \vv{g}$, 
non c'è nessuna forza peso "nuova" l'accelerazione è proprio solo quella di gravità. 
Questa situazione è un sistema di riferimento non inerziale in cui siamo
in una situazione in caduta libera e sperimentalmente si può realizzare
con un aereo (l'esperimento zero gravity). Si deduce dunque che le accelerazioni 
di trascinamento (posto che non vi sia rotazione), producono sempre una forza
apparente contraria al verso dell'accelerazione stessa per un SdR inerziale.

\subsection{Il caso della piattaforma ruotante con $\dot{\vec{\omega}} = 0$}
\begin{wrapfigure}{r}{0.4\textwidth}
    \centering
    \caption{Piattaforma ruotante}
    \begin{tikzpicture}
        \node[ellipse,
        draw,
	    minimum width = 4cm, 
	    minimum height = 2.2cm] (e) at (0,0) {};
        \filldraw (0, 0)circle  (1pt) node[anchor = east]{$O$};
        \filldraw(1.25, -0.25) circle (2pt) node[anchor = south east ] {$P$};
        \draw(0, 0)[->] -- (4, 0) node[at end, below] {$x' $};
        \draw(0, 0)[->] -- (0, 2) node[at end, left] {$z' $};
        \draw(0, 0)[->] -- (-1, -2) node[at end, below] {$y' $};
        \draw(1.25, -0.25)[->] -- (2, -0.45) node[at end, right] {$\vv{F}_{cf}$};
        \draw(1.25, -0.25)[->] -- (1.75, 0.25) node[at end, right] {$\hat{u}_t$};
        \draw[->](1.25, -0.25) -- (0.75, -0.75) node[at end, below] {$\vv{v}'$}; 
        \draw[->](1.25, -0.25) -- (1.25, 0.75) node[at end, right] {$\vv{N}$};
        \draw[->](1.25, -0.25) -- (1.25, -1.25) node[at end, right] {$m\vv{g}$};
        \draw[->, very thick](1.25, -0.25) -- (0.5, -0.123) node[at end, below] {$\vv{F}_{co}$};
        \draw[dashed, thin] (0, 0) -- (1.25, -0.25) node[midway, above] {$\rho$};
    \end{tikzpicture}    
\end{wrapfigure}
Preso $\vv{\omega}$ costante nel caso della piattaforma rotante non potremmo ignorare i termini
delle forze apparenti dati dalla velocità di rotazione della piattaforma. Posso considerare $x, y, z$ gli assi per il sistema di riferimento inerziale 
e gli assi $x', y', z'$ per il SDR non inerziale, in modo tale che l'asse $z$ sia ortogonale al piano
di rotazione e coincida con $z'$. Posso impostare ora la forza di trascinamento e la forza complementare (Coriolis):
\begin{align*}
    \vv{F}_t &= -m\vv{\omega} \times (\vv{\omega} \times (P - O)) \\
    \vv{F}_{co} &= -2m \vv{\omega} \times \vv{v'}  = -2m \vv{\omega} \times (\vv{\omega} \times r \hat{u_r}  )    
\end{align*}
La forza di trascinamento è semplificata in quanto i vettori sono paralleli e dunque
si annulla il prodotto vettoriale. 
Nel piano possiamo anche ottenere una rappresentazione dall'alto che ci
consente di esprimere le forze con le coordinate cilindriche,
la distanza $P - O$ è proprio:
\begin{gather*}
        P - O = \rho \hat{u}_{\rho}  + z\hat{k} 
\end{gather*}


\begin{wrapfigure}{r}{0.4\textwidth}
    \centering
    \caption{Schematizzazione del moto dall'alto}    
    \begin{tikzpicture}
        \draw(0, 0) circle (2.5);
        \draw[->](0, 0) -- (3, 0) node[at end, below] {$x'$};
        \draw[->] (0, 0) -- (0, 3) node[at end, left] {$y'$};
        \draw[dashed](0, 0) -- (1.41, 1.41) node[midway, above] {$\rho$};
        \filldraw (1.41, 1.41) node[anchor = west] {$P$};
        \draw[->] (1, 0) arc (0: 45: 1) node[midway, right] {$\phi$};
        \draw[->] (1.41, 1.41) -- (2, 2) node[at end, right] {$\hat{u}_{\rho}$};
        \draw[->] (1.41, 1.41) -- (1, 1.82) node[at end, right] {$\hat{u}_{\phi}$ }; 
    \end{tikzpicture} 
\end{wrapfigure}
E quindi la forza di trascinamento, essendo $\vv{\omega} = \omega \hat{k}$
\begin{align}
    \vv{F}_t = + m\omega^{2} \rho \hat{u}_{\rho}  
\end{align}
Dato che ha segno positivo, questa tende ad andare verso l'esterno rispetto all'origine
e prende quindi il nome di forza centrifuga. La velocità a questo punto
può essere espressa come:
\begin{gather*}
    \vv{v'} = \frac{d}{dt} (P- O) = \dot{\rho}\hat{u}_{\rho} + \rho\dot{\phi}\hat{u}_{\phi} + \dot{z}\hat{k}
\end{gather*}
E quindi possiamo trovare la forza di Coriolis che sarà, vista l'impostazione
precedente delle forze, data da:
\begin{align}
    \vv{F}_{co} =  -2m\omega^{2}\rho\hat{u}_{\rho}      
\end{align}
Che relazione c'è tra $\dot{\phi}$ e $\omega$? Ad una prima analisi potrebbero sembrare la stessa cosa, ma
in realtà sono due cose complementare diverse. $\omega$ è costante sia in modulo che direzione ed è la velocità angolare
con cui ruota la piattaforma mentre $\phi$ è la posizione nel sistema di riferimento ruotante che l'osservatore sta
guardando e quindi $\dot{\phi}$ è la variazione della posizione del punto $P$ rispetto all'osservatore $S$. \\
Ci possiamo chiedere adesso che  moto avrebbe l'osservatore rispetto al SDR non inerziale? L'osservatore appare
ruotare rispetto al SDR non inerziale ma in senso opposto e quindi
\begin{gather*}
    \dot{\phi} = -\omega \qquad \dot{\rho} = 0 
\end{gather*}
La forza centrifuga resta inalterata e non cambia mentre
$\rho$ non cambia e quindi posso esprimere la forza centrifuga e di Coriolis(che diventa la forza complementare):
\begin{gather*}
    \vv{F}_t = m\omega^{2}\rho\hat{u}_{\rho}   \\
    \vv{F}_{co} = 2m \omega^{2}\rho\hat{u}_{\rho}   
\end{gather*}
Il corpo rispetto al centro del SdR non inerziale appare muoversi verso l'esterno e verso 
destra se la giostra ruota verso destra e verso sinistra se la giostra ruota verso
sinistra. Rispetto all'SdR inerziale invece appare spinto verso l'esterno. 

\subsection{Il caso della guida}
\begin{wrapfigure}{r}{0.4\textwidth}
    \centering
    \caption{Guida circolare}
    \begin{tikzpicture}
        \draw(0, 0) circle (2);
        \draw[->](0, 0) -- (2.41, -2.41) node[at end, right] {$x'$};
        \draw[->](0, 0) -- (2.41, 2.41) node[at end, right] {$y'$};
        \draw (0, 0) -- (2, 0);
        \draw[->] (0.41, -0.41) arc(-45:0:0.5) node[midway, right] {$\phi$};
        \filldraw (1, 0) circle (1pt) node[anchor = south east] {$P$};
        \draw[->] (1, 0) -- (1.5, 0) node[at end, above] {$\vv{F}_t$};
        \draw[->] (1, 0) -- (1, 0.5) node[at end, above] {$\vv{F}_{co}$};
        \draw[->] (1, 0) -- (1, -0.5) node[at end, right] {$\vv{N}$};
    \end{tikzpicture}    
\end{wrapfigure}
Un punto materiale che scorre su di una guida senza attrito ha solo un grado di libertà in quanto può solo scorrere
lungo la traiettoria forzata dalla guida.
Dal momento che il sistema $S'$ è solidale
con l'oggetto, allora $\dot{\phi} = 0$, allora essendo che ho la componente radiale
\begin{align*}
    \hat{u}_{\rho} :& \  m\omega^{2}\rho =  m a_{\rho} \\
    \hat{u}_{\phi} :& -2m\omega \dot{\rho} + N = 0  
\end{align*}
La guida quindi deve produrre una reazione vincolare per tenere fermo l'oggetto dentro sé stessa. 
Quindi possiamo definire l'accelerazione (Ricordando la definizione di accelerazione di Coriolis) come:
\begin{align*}
    \vv{a'} = \frac{d}{dt} \vv{v'}  = \dot{\rho}\hat{u}_{\rho} + \rho\dot{\phi}\hat{u}_{\phi} + \dot{z}\hat{k}    \\ 
    \vv{a'} = (\ddot{\rho} - \rho\dot{\phi}^{2})\hat{u}_{\rho} + (2\dot{\rho}\dot{\phi} + \rho\ddot{\phi})\hat{u}_{\rho}   
\end{align*}
Allora si ottengono le descrizioni per i versori:
\begin{align*}
    \hat{u}_{\rho} :& \ \omega^{2}\rho = \ddot{\rho} \\
    \hat{u}_{\phi} :& \ N = 2m\omega \dot{\rho}   
\end{align*}
E allora si ottiene l'equazione del moto armonico modificata (quella per il filo con massa
che cade):
\begin{gather*}
    \ddot{\rho} - \omega^{2}\rho = 0 \ \Longrightarrow \ 
    \rho = \rho_0 \cosh(\omega t)  \ \Longrightarrow \ 
    \dot{\rho} = \rho_0 \omega\sinh(\omega t) 
\end{gather*}
Allora la reazione vincolare della guida è proprio:
\begin{align}
    N = 2m \rho_0 \omega \sinh(\omega t)
\end{align}
che è proprio la forza che accelera il corpo che scorre nella guida
per cui un osservatore in un SDR inerziale vede il corpo accelerato.

\section{Il sistema di riferimento terrestre}
\begin{wrapfigure}{r}{0.4\textwidth}
    \centering
    \caption{Terra e riferimento}
    \begin{tikzpicture}
        \draw (0, 0) circle (2);
        \draw[dashed, thin] (0, 2.35) -- (0, -2);
        \draw[->] (-0.5 ,2.35) arc (200:340:0.5) node[at end, right] {$\vv{\omega}_T$};
        \draw[dashed, thin](-2, 0) arc(240: 300: 4);
        \draw[dashed, thin](2, 0) arc(60: 120: 4);
        \filldraw (0, 0) circle (1pt) node[anchor = east] {$O$};
        \filldraw(1.41, 1.41) circle (1pt) node[anchor = south] {$P$};
        \draw[->, thick] (1.41, 1.41) -- (0.5, 0.5) node [midway, left] {$m\vv{g}$};
        \draw[->, thick] (1.41, 1.41) -- (0.7, 0.5) node [at end, below] {$\vv{F}_G$};
        \draw[->, thick] (1.41, 1.41) -- (2, 1.41) node[at end, right] {$\vv{F}_{cf}$};
        \draw[->] (1.5, 0) arc (0:45:1.5) node[midway, right] {$\theta$};
        \draw(0, 1.41) -- (1.41, 1.41) node[midway, above] {$\rho$};
        \draw[dashed, thin] (0, 0) -- (1.41, 1.41);
        \draw[dashed, thin] (0, 0) -- (2, 0) node[midway, below] {$R_T$};
    \end{tikzpicture}    
\end{wrapfigure}
Se facessimo degli esperimenti sulla Terra dovremmo riuscire a mettere in luce le
forze apparenti di un SDR non inerziale. La Terra ruota intorno al proprio asse
rispetto alle stelle fisse da Ovest verso Est e quindi mentre ruota su sé stessa con velocità
angolare $\left| \vv{\omega}_T  \right| = \frac{2\pi}{86'164}$ essa dovrebbe ritrovare il Sole nel medesimo punto
ad ogni rivoluzione anche se questo non accade. Questo perché il giorno solare medio è più lungo del giorno sidereo: rispetto
alle stelle fisse, infatti, la Terra ha compiuto un giro in più. Che conseguenza ha il fatto che
la Terra non è un SDR inerziale?
Questo vuol dire che ci sono delle forze apparenti in tutti gli esperimenti e quindi
rispetto alla Terra nei moti su larga scala intervengono queste forze apparenti. 
All'equatore per l'osservatore inerziale, un oggetto vicino alla Terra è soggetto
alla forza di gravità dovuta all'interazione con la Terra (che si assume sferica),
mentre per l'osservatore sulla Terra mi aspetto che ci sia una forza centrifuga. Usando un filo a piombo, 
la sua verticale non passerà per il centro della Terra ma sarà leggermente inclinata rispetto
alla forza peso ideale a causa della forza centrifuga. 
Voglio calcolare allora il vero peso degli oggetti e misuro oltre che
la direzione del pendolo anche la forza peso. Associo $\hat{u}_{\rho}$ il versore $\perp \hat{k}$
che giace nello stesso piano del versore  $\hat{u}_r$ (il versore parallelo all'asse $\vv{OP}$ )
e che è il versore della centrifuga. Allora:
\begin{gather*}
        \hat{u}_r = \left( \cos\theta\vv{u}_{\rho} + \sin\theta\hat{k} \right) 
\end{gather*}
La forza peso allora si esprimerà come:
\begin{gather*}
        m\vv{g} = \vv{F}_{G} + \vv{F}_{cf} = -mg_0 \hat{u}_r + m\omega^{2}_T \rho \hat{u}_{\rho} 
\end{gather*}
Dove $g_0$ è $g$ senza la correzione della forza centrifuga, sostituendo allora quanto ottenuto prima si ottiene.
\begin{gather*}
        m\vv{g} = m \left( \cos\theta\left( -g_o + \omega^{2}_T R_T\right) \hat{u}_{\rho} + g_0 \sin\theta \hat{k} \right)
\end{gather*}
Volendo ora calcolarne il modulo, si ottiene la seguente:
\begin{gather*}
        \left| g \right| = \sqrt{\cos^{2}\theta\left(g_0^{2}  + \omega^{4}_T R_T^{2} - 2g_0 \omega^{2}_T R_T\right) + g_0^{2}\sin^{2}\theta }
\end{gather*}
Che sostituendo, raccogliendo e semplificando si ottiene:
\begin{gather*}
        \left| g \right| = \sqrt{g_0^{2} \left( 1 - \frac{2\omega^{2}_T R_T}{g_0}\cos^{2}\theta + \left( \frac{\omega^{2}_T R_T }{g_0} \right)^{2} \cos^{2}\theta  \right) }  
\end{gather*}
Usando lo sviluppo di Taylor allora si può semplificare i coseni (in particolare il primo)
poiché hanno un contributo molto piccolo e quindi:
\begin{gather*}
        \left| g \right| \approx g_0 \left(1 - \frac{\omega^{2}_T R_T}{g_0}\cos^{2}\theta\right) 
\end{gather*}
L'accelerazione al polo è $g_{polo} = 9.823 m/s^{2} $ e quella all'equatore $g_{eq} = 9.789 m/s^{2} $ poiché
la Terra è un ellissoide di rotazione e la forza di Coriolis ha fatto schiacciare la Terra in modo tale che
all'equatore il raggio terrestre sia più grande che ai poli.
Preso un triangolo i cui lati sono la forza centripeta, uno la forza di gravità
vera e una quella ideale, applicando la trigonometria si ottiene l'espressione dell'angolo
tra le due forze ($\delta$):
\begin{gather*}
    \frac{F_{cf} }{\sin\delta} = \frac{g}{\sin\theta} \\
    \sin\delta \approx  \frac{\omega^{2}_T R_T }{g_0} \cos\theta\sin\theta
\end{gather*} 

\chapter{L'Energia ed il lavoro}
\section{Il lavoro}
\begin{wrapfigure}{r}{0.4\textwidth}
    \centering
    \label{Fig 5.1}
    \caption{Il lavoro di una forza su di una superficie inclinata}
    \begin{tikzpicture}
        \draw(0, 0) -- (5, 0);
        \draw(0, 0) -- (0, 2);
        \draw(0, 2) -- (5, 0);
        \draw(5, 0.2) circle (0.12);
        \draw[->](4.88, 0.25) -- (4, 0.65) node[at end, above] {$\vv{F}_0$};
        \filldraw(5, 0) circle (1pt) node[anchor = north] {$O$};
        \filldraw(0, 2) circle (1pt) node[anchor = south] {$F$};
    \end{tikzpicture}    
\end{wrapfigure}
Se avessi un grande oggetto da spostare molto massiccio,
dovrei avere una forza tale che possa equilibrare e vincere la forza peso 
dell'oggetto per poterlo sollevare, oppure vincere la resistenza della forza di
attrito statica per poterlo muovere orizzontalmente.
Come hanno quindi fatto gli egiziani a portare i blocchi delle piramidi sopra le piramidi stesse
per la loro costruzione? Hanno creato delle impalcature
che permettessero di far strisciare i blocchi su di un piano inclinato con
angolo molto piccolo. Come misuro quindi lo "sforzo" necessario per compiere
questo spostamento? La grandezza che abbiamo bisogno per
poter spingere il blocco fino alla fine è proprio il lavoro. Si definisce
il \textbf{lavoro} come la quantità che descrive gli scambi di energia in un sistema.
In particolare, il \textbf{lavoro di una forza}, è definito come  il prodotto scalare
tra il modulo della forza per lo spostamento che fa compiere all'oggetto su cui è applicata. Il lavoro è sempre di una forza
e si indica con il simbolo $\delta L$, ossia il lavoro infinitesimo definito
come il prodotto scalare
\begin{align}
    \delta L = \vv{F} \cdot  \vv{\delta S} 
\end{align}
E' quindi lo spostamento della forza lungo un certo cammino, ma il suo
valore non cambia poiché dipende solo dagli istanti iniziali e finali;
si definisce quindi il lavoro lungo un certo cammino. Per trovare il
valore devo fare l'integrale di linea, ossia l'integrale della traiettoria
con il limite che mi fa tendere lo spostamento a $d x$. Per ognuno di questi
divido il mio percorso in tanti pezzi per cui si ottiene il lavoro di una forza $\vec{F}$ lungo
uno specifico tratto $AB$ su di una curva $\Gamma$: 
\begin{align}
    L_{AB, \Gamma} = \oint_{A}^{B} \vv{F} \cdot  \vv{\delta S} \ ds  
\end{align}
Definiamo il lavoro di una forza come:
\begin{align}
    \delta L = \vv{F} \cdot \vv{dr}  
\end{align}
Dove $dr$ rappresenta lo spostamento infinitesimale della forza $\vv{F}$. $\delta L$
non è un differenziale esatto in quanto il differenziale esatto in fisica presuppone la
scomposizione lungo le componenti di una data grandezza; essendo però il lavoro
una quantità scalare, allora il suo differenziale non è esatto. Allora possiamo
definire il lavoro totale come la somma di tutti gli infinitesimi spostamenti:
\begin{align}
    L_{AB, \vv{F}} = \sum_{i = 0}^{n}\vv{F}_i \vv{dr}_i = \int_{A, \Gamma}^{B}\vv{F}\cdot \vv{dr}   
\end{align}
Nel disegno, il lavoro compiuto dalla forza dipende dall'angolo del piano inclinato e quindi:
\begin{gather*}
    L_{\vv{F}_0} = \int_{OF}^{}\vv{F}_0 \cdot  \vv{dr} = Mg \sin\alpha L = Mgh   
\end{gather*}
Se volessi esplicitare l'integrale allora posso dividerlo lungo tutte le direzioni e quindi risolvere
la somma degli integrali. Con il concetto di integrale possiamo anche dimostrare il teorema
delle forze vive.

\section{Il teorema delle forze vive}
Data l'espressione del lavoro per una forza che produce un certo spostamento
infinitesimo $\vv{dr}$: 
\begin{gather*}
    \delta L = \vv{F} \cdot \vv{dr} 
\end{gather*}
Se avessi più forze, ognuna di esse dovrebbe compiere un certo lavoro
che dipende dallo spostamento che causano: posso considerare quindi
la risultante delle forze applicate su di un corpo lungo una traiettoria
istante per istante come se le forze fossero tutte applicate insieme 
come somma in un unica forza:
\begin{gather*}
    \vv{F}_{tot} = \sum_{i = 0}^{n} \vv{F}_i   
\end{gather*}
e definire quindi il lavoro totale:
\begin{gather*}
    \delta L_{tot} =  \sum_{i = 0}^{n} \vv{F}_i  \cdot \vv{dr} = \sum_{i = 0}^{n} \delta L_{\vv{F}_i}  
\end{gather*}
Il secondo principio della dinamica ci dice che in un sistema di riferimento la risultante delle forze
produce sempre una certa accelerazione su di un corpo di massa $m$: 
\begin{gather*}
    \vv{F}_{tot} = m\vv{a}.  
\end{gather*}
Si può allora riscrivere il lavoro iniziale come:
\begin{gather*}
    \delta L_{\vv{F}_{tot}} = m\vv{a} \cdot  \vv{dr}   
\end{gather*}
Essendo la velocità la derivata rispetto al tempo del vettore posizione, allora
è proprio $\frac{dr}{dt}$ e quindi al posto di $\vv{F}$ posso mettere:
\begin{gather*}
    \delta L_{\vv{F}_{tot}} = m\frac{d\vv{v} }{dt} \cdot \vv{v}dt  
\end{gather*}
Eseguendo allora il prodotto scalare si ottiene
\begin{gather*}
    \frac{d}{dt}(\vv{v} \cdot  \vv{v}) = \left(\frac{d\vv{v}}{dt}\right)\vv{v} + \vv{v}\left(\frac{d\vv{v}}{dt}\right) = 2\vv{v}\cdot \left(\frac{d\vv{v}}{dt}\right)
\end{gather*}
Dato che questo prodotto ha un 2 davanti, allora devo mettere un mezzo 
nell'espressione del lavoro. A questo punto posso integrare ed ottenere l'energia cinetica come
\begin{gather*}
    \int \frac{1}{2}m\vv{v} d\vv{v} = \frac{1}{2}m \int \vv{v}d\vv{v} = dK    
\end{gather*}
Se io avessi da compiere un lavoro continuo e non infinitesimo? Allora su di una traiettoria il lavoro
diventerà:
\begin{gather*}
    L_{AB, \Gamma} = \int_{A, \Gamma}^{B} \vv{F} \cdot  \vv{dr} = \int_{A, \Gamma}^{B} dK = \frac{1}{2}mv_{B}^{2} - \frac{1}{2}mv_{A}^{2}     
\end{gather*}
E allora possiamo concludere con il teorema delle forze vive che il lavoro
è proprio la variazione dell'energia cinetica:
\begin{align}
    L_{AB, \Gamma} = \Delta K
\end{align}
Questo ci dice inoltre che se io compio lavoro su di un oggetto
allora sto provocando una variazione dell'energia cinetica. Il differenziale
di $K$ dipende solo dalle condizioni iniziali e finali e non è quindi una
funzione della posizione. 

\clearpage
\subsection{Teorema delle forze vive nel piano inclinato}
\begin{wrapfigure}{r}{0.3\textwidth}
    \centering
    \label{Fig 5.2}
    \caption{Un corpo su di un piano inclinato}
    \begin{tikzpicture}
        \draw(0, 0) -- (3, 0);
        \draw(0, 2) -- (3, 0);
        \draw(2, 0) arc(180:150: 1) node[midway, left] {$\alpha$};
        \draw(0, 2.12) circle (0.12);
    \end{tikzpicture}    
\end{wrapfigure}
Posso impostare il problema nella seguente maniera: sapendo che devo trovare la
funzione della discesa dell'oggetto, questa è proprio:
\begin{gather*}
    mg\sin\alpha = m\ddot{x} \\
    x = \frac{1}{2}g\sin\alpha t^{2} 
\end{gather*}
E poi si ricava la velocità finale combinando le varie derivate.
Il metodo due è quello dell'energia cinetica con un vincolo semplice:
la forza normale è ortogonale allo spostamento e quindi non produce alcun tipo
di lavoro poiché non impedisce lo spostamento lungo la traiettoria permessa. Allora il lavoro è proprio:
\begin{gather*}
    L_{AB} = Mg\sin\alpha L \qquad \text{con } v_B = v_f \qquad v_A = 0
\end{gather*}
Utilizzando ora il teorema delle forze vive, per cui ho solo $v_B$
\begin{gather*}
    Mg\sin\alpha L = \frac{1}{2}Mv_{f}^{2} 
\end{gather*}
E allora:
\begin{gather*}
    v_f = \sqrt{2g\sin\alpha L} 
\end{gather*}
Questo non ci dà il tempo di arrivo dell'oggetto (come invece avrei con la risoluzione
mediante la dinamica) ma solamente la velocità finale. E' importante notare (come si vedrà più 
avanti, che la derivate delle'energia totale rispetto al vincolo mi permetta di ottenere
l'equazione di moto). 

\subsection{L'attrito nel teorema delle forze vive}
Considerando lo stesso piano inclinato, su cui però è presente attrito, 
la forza di attrito su di un corpo nel piano inclinato sarà data da
\begin{gather*}
    |\vv{F}_{at}| = \mu_d Mg\cos\alpha 
\end{gather*}
Posso quindi ottenere l'espressione dell'energia cinetica come la
differenza tra l'energia potenziale del corpo e l'energia dissipata 
dall'attrito:
\begin{gather*}
    Mg\sin\alpha L -\mu_d Mg\cos\alpha= \frac{1}{2}Mv_f^{2} 
\end{gather*}
La velocità finale diventerà allora:
\begin{gather*}
    v_f = \sqrt{2gL(\sin\alpha - \mu_d \cos\alpha)} 
\end{gather*}
Mentre la forza peso è costante, la forza di attrito non è costante
poiché dipende dal verso in cui sta andando l'oggetto. Se analizzassimo il caso
di un oggetto che scorre giù da un piano lungo una guida allora la forza
peso non fa lavoro sui tratti orizzontali ma la forza di attrito, quando è presente,
ha sempre verso contrario al moto. 

\section{Definizione di campo conservativo}
Una forza posizionale è una forza che dipende dalla posizione di un oggetto,
come la forza elastica; forze come l'attrito non sono posizionali poiché dipendono dal 
verso di applicazione della forza; possiamo allora scrivere la definizione di forza
posizionale nella seguente maniera:
\begin{align}
    \vv{F} = \vv{F}(x, y, z)  
\end{align}
Si distinguono le forze conservative (sottoclasse) che sono quelle forze
posizionali in cui il lavoro non dipende dal cammino seguito:
\begin{gather*}
    L_{AB, \Gamma_1} = L_{AB, \Gamma_2}, \forall \Gamma_1, \Gamma_2
\end{gather*}
La seconda proprietà di queste forze è che il lavoro in un percorso
chiuso è zero (chiamata anche circuitazione):
\begin{align}
    \oint \vv{F} \cdot  \vv{dr} = 0  
\end{align}
Lungo una traiettoria chiusa, che io la percorra in un senso
o nell'altro, non cambia il lavoro, cambia solo di segno per cui
la sua circuitazione sarà sempre 0. La prima implica la seconda e
viceversa. \\
Terza proprietà: $\exists V$ energia potenziale tale che:
\begin{align}
    \delta L &= -dV \\
    L_{AB, \Gamma} &= V(A) - V(B), \qquad \forall \Gamma
\end{align}
Abbiamo un differenziale non esatto eguagliato ad un differenziale esatto:
questo vuol dire che il differenziale di $V$ è quello totale della funzione
(che considera dunque tutte le variabili). 
Si dimostra che esiste questa funzione energetica $V$, ossia l'energia
potenziale, considerando lungo una traiettoria un punto di riferimento $P_0$ 
da cui facciamo passare il percorso $A, P_0, B$ e quindi si ottiene che la
circuitazione di questo percorso sia:
\begin{gather*}
    L_{AP_0} + L_{P_0B} + L_{BA} = 0 
\end{gather*}
Fissato il punto $P_0$ allora questa è una funzione di $A$ e posso esprimerla
secondo le seguenti relazioni definendo una funzione energetica che possa variare a
seconda del punto considerato $V(x, y, z)$:
\begin{align*}
    L_{AP_0} &= -V(x_A, y_A, z_A) \\
    L_{P_0B} &= -V(x_B, y_B, z_B)
\end{align*}
E per cui si ottiene che il lavoro da $A$ a $B$ è dato dalla seguente relazione:
\begin{gather*}
    L_{AB} = V(A) - V(B)
\end{gather*}
L'energia potenziale è sempre definita a meno di una costante; possiamo allora
esprimere il lavoro mediante l'energia potenziale (che è posizionale e quindi in funzione dello spazio)
e allora scrivere:
\begin{gather*}
    \delta L = -dV = - (V(x + dx, y+dy, z + dz) - V(x, y, z))
\end{gather*}
Matematicamente questo si può scrivere come:
\begin{gather*}
    - (V(x + dx, y+dy, z + dz) - V(x, y +dy, z +dz) + V(x, y +dy, z +dz)\\
    - V(x , y , z +dz) + V(x , y , z +dz) - V(x, y, z))
\end{gather*}
Facendo la derivata parziale per ogni variabile che compare
nell'espressione dell'energia potenziale ottengo il suo differenziale
esatto espresso come
\begin{gather*}
    -\left(\frac{\partial V}{\partial x}dx + \frac{\partial V}{\partial y}dy  + \frac{\partial V}{\partial z}dz\right) 
\end{gather*}
Esprimendo la forza nella sue componenti posso ottenere l'espressione
del lavoro rispetto al differenziale di tutte le componenti:
\begin{gather*}
    \delta L = \vv{F} \cdot  \vv{dr} = F_x dx + F_y dy + F_z dz  
\end{gather*}
questa relazione vale per qualunque sia $dx, dy, dz$. Eguagliando allora lo stesso 
incremento infinitesimo dell'energia potenziale ad ogni componente della forza, si ottiene 
proprio che ogni componente della forza è la derivata parziale rispetto a quella
componente della funzione di energia $V$ che ho considerato. 
\begin{gather*}
    \left\{\begin{array}{l}
    F_x = -\frac{\partial V}{\partial x}\\
    F_y = -\frac{\partial V}{\partial y}\\
    F_z = -\frac{\partial V}{\partial z}
    \end{array}\right. \Rightarrow \vv{F} = - \vv{\nabla} V 
\end{gather*}
Dove $\nabla$ è l'operatore differenziale (\textbf{gradiente}) che esprime la derivata parziale rispetto
a tutte le componenti di un certo vettore. Posso esprimere il lavoro con
il gradiente attraverso la seguente trasformazione e eguagliando al differenziale
della mia funzione energetica $V$:
\begin{gather*}
    \delta L = -\vv{\nabla} V \cdot \vv{dr} = -dV 
\end{gather*}
Con questa si hanno le seguenti relazioni quando $\vv{F} = -\vv{\nabla} V$
\begin{align}
    \vv{\nabla} \times \vv{F} &= 0 \\
    \vv{\nabla} \times \vv{\nabla} V &= 0 
\end{align}
la prima prende il nome di \textbf{rotore della forza} che è
zero se la forza è conservativa. Svolgendo il prodotto vettoriale si ha
\begin{align}
    \det\left| \begin{tabular}{c c c}
         $\hat{i} $ & $\hat{j} $&  $\hat{k}$ \\
         $\frac{\partial }{\partial x}$ &  $\frac{\partial }{\partial y}$ & $\frac{\partial }{\partial z}$ \\
         $\frac{\partial V}{\partial x}$  & $\frac{\partial V}{\partial y}$  &$\frac{\partial V}{\partial z} $
    \end{tabular} \right| = \hat{i} \left( \frac{\partial^{2} V }{\partial y \partial z} - \frac{\partial^{2} V }{\partial y \partial z}\right) = 0. 
\end{align}
Dal momento che le derivate parziali sono le stesse, allora il determinante è zero
ed il rotore della forza è zero.
Si riassumono le proprietà dei campi conservativi:
\begin{enumerate}
    \item \begin{gather*}
        L_{AB, \Gamma_1} = L_{AB, \Gamma_2}
    \end{gather*}
    \item Per ogni possibile linea chiusa nel campo si ha
    \begin{gather*}
        \oint \vv{F} \cdot d\vv{r} = 0  
    \end{gather*} 
    \item Esiste una funzione scalare $V(\vv{r} )$ detta energia
    potenziale tale che si può calcolare indipendentemente dalla
    curva $\Gamma$ per cui si è calcolato il lavoro:
    \begin{gather*}
        L_{AB} = V(\vv{r_A} ) - V(\vv{r_B} )
    \end{gather*}
    \item \begin{gather*}
        \vv{F} = -\vv{\nabla}V \ \Longrightarrow \ \vv{\nabla} \times \vv{F} = 0 \ \Longrightarrow \ \text{rot}\vv{F} = 0     
    \end{gather*}
\end{enumerate}

\section{Energia meccanica}
Un dominio si dice semplicemente connesso se, presa una curva chiusa,
la posso deformare fino a farla diventare un punto. Considerato un toroide, 
se prendo una curva all'interno del dominio, non posso farla collassare in un punto. Se abbiamo un sistema
su cui agiscono solo forze conservative, (anche forze non conservative che non
fanno lavoro), allora il teorema delle forze vive mi dice che:
\begin{gather*}
    \delta L = - dV = dK
\end{gather*}
E quindi isolando poso dire che
\begin{gather*}
    dK = - dV
\end{gather*}
E allora
\begin{gather*}
    dK + dV = 0
\end{gather*}
Se avessi solo forze conservative la somma dell'energia potenziale e cinetica
non varierebbero: quella somma rimarrebbe costante nel tempo. La somma di questi
due contributi energetici prende il nome di \textbf{energia meccanica}:  
\begin{align}
    E = K + V 
\end{align}
Lo stesso vale per due forze conservative qualsiasi: se avessi solo forze conservative allora avrei
che la derivata dell'energia cinetica è esattamente zero in quanto è una funzione costante:
\begin{gather*}
    dE = 0
\end{gather*}
Se invece ho una forza non conservativa, l'energia meccanica non si
conserva: se si avesse una forza non conservativa allora $\delta L = -dV$ non varrebbe più.
Considerati i contributi energetici delle forze conservative possiamo esprimere il
lavoro totale come la somma dei contributi conservativi e non conservativi
\begin{gather*}
    \delta L_{tot} = \delta L_{cons} + \delta L_{ncons}\\
    dK = -\sum dV_i + \delta L_{ncons} 
\end{gather*}
Quindi se lo porto dall'altra parte ottengo;
\begin{gather*}
    d\left(K + \sum dV_i \right) = \delta L_{ncons}
\end{gather*}
Chiamato allora
\begin{gather*}
    E = K + \sum dV_i
\end{gather*}
L'energia meccanica non sarà più uguale ad una costante 
ma uguale al lavoro che compiono le forze
non conservative:
\begin{gather*}
    dE = \delta L_{ncons}
\end{gather*}
Se il lavoro dell'attrito è negativo allora l'energia meccanica tenderà a
diminuire nel sistema considerato, in quanto l'energia cinetica viene \textbf{dissipata}
in calore mediante l'attrito; questo si osserva nel mondo reale ed è il motivo
per il quale le macchine perpetue non esistono. 

\section{Forze conservative e loro esempi}
\subsection{La forza peso}
\begin{wrapfigure}{r}{0.4\textwidth}
    \centering
    \caption{Forza peso}
    \begin{tikzpicture}
        \filldraw(0, 0) circle (1pt) node[anchor = south] {$m$};
        \draw[->](0, 0) -- (0, -1) node[at end, right] {$m\vv{g}$ };
        \draw[->](0, -3) -- (1, -3) node[at end, below] {$x$};
        \draw[->](0, -3) -- (0, -2) node[at end, left] {$y$};
    \end{tikzpicture}    
\end{wrapfigure}
La forza peso è una forza conservativa in quanto rispetta le proprietà di forza
conservativa: il suo rotore è zero in quanto è una forza costante e il suo lavoro 
lungo una curva qualsiasi chiusa è esattamente zero. Si può dimostrare considerando
un lavoro infinitesimo compiuto dalla forza peso: la forza peso è un \textbf{campo uniforme}: ossia
caratterizzata da modulo $mg$ costante e indipendente da $\vv{r}$. Se volessimo
allora calcolare l'energia potenziale attraverso il gradiente in un certo percorso $AB$, otterrei la
seguente espressione:  
\begin{gather*}
    \delta L = (m\vv{g})\cdot d\vv{r}  
\end{gather*}
Allora:
\begin{gather*}
    \delta L = -mg\hat{j}\cdot dy \hat{j} 
\end{gather*}
Il lavoro finito allora:
\begin{gather*}
    \int_{A}^{B} \delta L = -mg(y_B -y_A) = V(A) - V(B)
\end{gather*}
E' del tutto ragionevole, che per un punto generico P,
l'energia potenziale sia data dalla seguente: 
\begin{align}
    V(P) = mgy_P
\end{align}
Ossia l'energia potenziale è direttamente proporzionale
alla quota rispetto al SdR scelto e, dato che è definita a meno di una costante, il lavoro
è dato dalla differenza delle potenziali; anche se avessi scelto
come riferimento dell'asse y un punto qualunque, non sarebbe cambiato
nulla rispetto al risultato ottenuto: se aggiungessi la stessa quantità sia a $V(A)$ che a
$V(B)$ allora questa costante che viene aggiunta si cancella in quanto per
definizione l'energia potenziale è definita a meno di una costante.
\begin{gather*}
    \nabla V = \frac{\partial V}{\partial x}\hat{i} +  \frac{\partial V}{\partial y}\hat{j} +\frac{\partial V}{\partial z}\hat{k}
\end{gather*}
Ottenendo proprio
\begin{gather*}
    \nabla V = mg \hat{j} 
\end{gather*}
Dato che io per definizione di gradiente:
\begin{gather*}
    \vv{F} = -\nabla  V \Rightarrow -mg\hat{j}  
\end{gather*}
La direzione dell'asse y è un'altra scelta arbitraria che ho fatto a priori
nell'esperimento così come l'orientazione degli assi. Ma non cambia niente in quanto
fin tanto che seguo le leggi della mano destra posso comunque scegliere
qualsiasi sistema di riferimento in qualsiasi verso lo voglia orientare.


\begin{wrapfigure}{r}{0.4\textwidth}
    \centering
    \caption{Moto verticale}
    \begin{tikzpicture}
        \draw(0, 0) -- (3, 0);
        \draw(0, 0) -- (0, 3);
        \draw[<->](1, 0) -- (1, 2) node[midway, left] {$h$};
        \filldraw(1.5, 2) circle(1pt) node[anchor = west] {$m$};
        \draw[->, thick](1.5, 2) -- (1.5, 1) node[at end, right] {$\vv{g}$ };
    \end{tikzpicture}    
\end{wrapfigure}
Nel caso di un corpo in caduta, se si ponesse come riferimento l'asse
$z$ rivolto verso l'alto, allora l'energia meccanica si potrebbe esprimere come:
\begin{gather*}
    E = \frac{1}{2}mv^{2} + mgz 
\end{gather*}
Quando $t = 0$ si ottiene:
\begin{gather*}
    z = h \qquad v = 0 \qquad  E = mgh 
\end{gather*}
Al tempo finale $t = t_f$:
\begin{gather*}
    z= 0 \qquad v = v_f \qquad E = \frac{1}{2}mv_f^{2}  
\end{gather*}
Si ha allora l'utile espressione della velocità finale per un corpo in caduta libera:
\begin{align}
    v_F = \sqrt{2gh} 
\end{align}
Se volessi determinare l'equazione di moto, potrei derivare rispetto al vincolo 
l'energia meccanica e ottenere le seguenti espressioni:
\begin{gather*}
     E = \frac{1}{2}m \dot{z}^{2} + mgz \\
     \frac{\partial E}{\partial z} =  \dot{z}(m\ddot{z} + mg) = 0   
\end{gather*}
Questa è l'equazione di moto di un corpo in caduta libera. Dato che la velocità
velocità iniziale è nulla, allora si ottiene lo stesso caso che si avrebbe con 
l'applicazione del teorema delle forze vive:
\begin{gather*}
    \dot{z} = 0 \\
    m\ddot{z} = -mg 
\end{gather*}
Che vale solo se i vincoli sono ideali e se non si muovono: infatti per poter
applicare la conservazione dell'energia è necessario avere solamente un vincolo.
Nel caso di pochi vincoli è possibile cercare approssimazioni o modellizazioni che mi
consentano di tornare ad avere un vincolo solo (quando possibile) per poter
utilizzare la conservazione dell'energia meccanica. Quando non è possibile
applicare l'energia meccanica devo risolvere le situazioni fisiche attraverso
la dinamica.

\subsection{Forza elastica}
\begin{wrapfigure}{r}{0.4\textwidth}
    \centering
    \label{Fig 4.1}
    \caption{La molla ideale}
    \begin{tikzpicture}
        \draw (0, 0) -- (0.5, 0);
        \draw (0.5, 0) -- (0.5, -1);
        \draw[decoration={aspect=0.3, segment length=1.5mm, amplitude=1.3mm,coil},decorate,opacity=0.9] (0.5, -0.5) -- (2,-0.5);
        \draw (0.5, -1) -- (2.5, -1);
        \draw (2, -1) rectangle (3, 0) node[midway] {$M$};
    \end{tikzpicture}    
\end{wrapfigure}
Dalla legge della forza elastica ottengo la forza che imprime una molla
(ideale) come risposta ad una forza che provochi un certo allungamento $x$ (con
allungamento si intende anche una contrazione della molla). 
\begin{gather*}
    \vv{F}_d = -kx\dot{i}  
\end{gather*}
Si può allora prendere in considerazione un lavoro infinitesimo per determinare 
se è un campo conservativo (oppure farne il rotore) e ottenere la seguente espressione: 
\begin{gather*}
    \delta L = \vv{F}_d \cdot \vv{dr} = -kx\hat{i}\cdot dx\hat{i}     
\end{gather*}
L'espressione del lavoro lungo la traiettoria $AB$, dove $A$ e $B$ sono due punti scelti arbitrariamente:
\begin{gather*}
    L_{AB} = \int_{A}^{B} -kx \ dx = -\frac{1}{2}kx_B^{2} + \frac{1}{2} kx_A^{2} = V(A) - V(B)  
\end{gather*}
Questa è proprio la definizione di energia potenziale, dunque il campo è
conservativo e posso dunque considerare l'energia
potenziale per un punto $P$ arbitrario ad distanza $x_P$ dalla lunghezza di riposo
della molla come
\begin{gather*}
    V(P) = \frac{1}{2}kx_P^{2} 
\end{gather*}
Si può allora determinare il gradiente per la forza elastica
\begin{gather*}
    \nabla V = kx\hat{i} 
\end{gather*}
Quindi essendo:
\begin{gather*}
    \vv{F} = \nabla V  = -kx\hat{i} 
\end{gather*}
Allora abbiamo dimostrato che la forza elastica è proprio una forza conservativa. 
Dato che è definita a meno di una costante, allora posso cambiare l'origine
dell'asse $x$ e cambierebbe solo la forma dell'energia potenziale.
Se si tirasse la molla nell'istante $t = 0$ allora l'energia sarebbe conservata
e quindi in ogni istante posso ricavarne la velocità:
\begin{gather*}
    t = 0 \ \Longrightarrow \  x = x_0 \qquad v = 0 \qquad E = \frac{1}{2}kx_0^{2} \\
    t = t_f \ \Longrightarrow \  x = 0 \qquad v = v_f \qquad E = \frac{1}{2}mv_f^{2}  
\end{gather*}
E quindi si trova l'utile relazione per la velocità finale di un oggetto
richiamato da una molla ad una distanza di $x_0$ rispetto alla posizione di riposo
della molla:
\begin{align}
    v_f = \sqrt{\frac{k}{m}}x_0 
\end{align}
Con la relazione $F = ma$ posso ottenere una dimostrazione diversa al posto
del rotore:
\begin{gather*}
    E = \frac{1}{2}m\dot{x}^{2} + \frac{1}{2}kx^{2} \\
    \frac{\partial E}{\partial x} =  \dot{x}(m\ddot{x} + kx) = 0   
\end{gather*}
Se $\dot{x}$ è zero si ottiene esattamente l'equazione del moto armonico:
\begin{gather*}
    m\ddot{x} + kx = 0
\end{gather*} 


\subsection{Molla attaccata al soffitto}
\begin{wrapfigure}{r}{0.3\textwidth}
    \centering
    \begin{tikzpicture}[scale = 1.4]
        \draw(0, 0) -- (2, 0);
        \draw[decoration={aspect=0.3, segment length=1.5mm, amplitude=1.3mm,coil},decorate,opacity=0.9] (1, 0) -- (1, -2);
        \draw[->](0, -1) -- (0.5, -1) node[at end, below] {$y$};
        \draw[->](0, -1) -- (0, -1.5) node[at end, left] {$x$};
        \draw(0.75, -2) rectangle (1.25, -2.5) node[midway] {$M$};
        \draw[->](1, -2.5)  -- (1, -3) node[at end, right] {$m\vv{g}$};
        \draw[->] (1, -2) -- (1, -1.5) node[at end, right] {$\vv{F}_e$};
    \end{tikzpicture}    
\end{wrapfigure}
Nel caso di una molla attaccata al soffitto con una massa che pende da essa, è possibile definire un sistema
di riferimento (e considerare ovviamente una molla ideale che non oscilli
lungo la direzione perpendicolare rispetto alla forza gravitazionale). 
Si ha dunque un solo grado di libertà nel mio sistema e posso applicare le 
definizioni di energia meccanica.
\begin{gather*}
    E = \frac{1}{2}mv^{2} + \frac{1}{2}kx^{2} - mgx\\
    v = \dot{x} \\
    \frac{\partial E}{\partial x} =  \dot{x}(m\ddot{x} + kx - mg) = 0    
\end{gather*}
Una soluzione è sicuramente $v = \dot{x}$ mentre l'altra soluzione è data dalla
differenziale non omogenea di forma:
\begin{gather*}
    \ddot{x} + \frac{k}{m}x = g
\end{gather*}
La cui soluzione generale è esattamente:
\begin{align}
    x(t) = A\cos(\omega t) + B\sin (\omega t) + \frac{mg}{k}
\end{align}

\clearpage
\section{Le forze centrali a simmetria sferica}
\begin{wrapfigure}{r}{0.25\textwidth}
    \centering
    \begin{tikzpicture}
        \filldraw(0, 0) circle (1pt) node[anchor = east] {$C$};
        \filldraw(1.5, 1.5) circle (1pt) node[anchor = west] {$P$};
        \draw[dashed, thin] (0, 0) -- (1.5, 1.5) node[midway, above] {$r$};
        \draw[->, red, very thick ] (1.5, 1.5) -- (0.9, 0.9) node[midway, above] {$\vv{F} $};
        \draw[->, red, very thick ] (0, 0) -- (0.6, 0.6) node[midway, below] {$\hat{u}_r$};
    \end{tikzpicture}    
\end{wrapfigure}
Le forze centrali sono delle forze sempre dirette verso un stesso punto,
come la gravità nella gravitazione universale o la forza elettrostatica
nell'elettromagnetismo; si dicono inoltre a simmetria sferica poiché il loro modulo 
è in funzione solamente del raggio vettore. Posso dimostrare che il
lavoro di queste forze sia conservativo nella seguente maniera: 
\begin{gather*}
    \delta L = \vv{F} \cdot \vv{dr}   \qquad \vv{r}= r\hat{u}_r
\end{gather*}
Per cui:
\begin{gather*}
    \vv{dr} = dr\hat{u}_r + rd\hat{u}_r   
\end{gather*}
Dato che il prodotto scalare tra lo stesso vettore è uguale
ad 1 ma la somma dei contributi $dr \cdot \hat{u} _r + \hat{u}_r dr  = 0$
allora  $ d\vv{r}  \perp \hat{u}_r $, quindi il lavoro diventa (considerato che il primo termine è parallelo al vettore
$\vv{F}$ mentre il secondo termine è perpendicolare ):
\begin{gather*}
    \delta L = F(r)\hat{u}_r  \cdot  (dr \hat{u}_r + r d\hat{u}_r) = F(r) dr.
\end{gather*}
Per cui preso l'integrale per uno spostamento continuo e finito
\begin{gather*}
    L_{AB} = \int_{A}^{B}F(r) dr = G(r_B) - G(r_A) 
\end{gather*}
Dove $G(r)$ è una primitiva ausiliaria che ho definito come primitiva 
dell'integrale generica, dunque, posta uguale all'energia potenziale
\begin{gather*}
    V(r) = -G(r)
\end{gather*}
Posso risalire alla definizione di energia potenziale e quindi questo tipo di forze
sono conservative.

\subsection{Forze a simmetria sferica non centrali}
\begin{wrapfigure}{r}{0.2\textwidth}
    \centering
    \begin{tikzpicture}
        \draw[->](0, 0) -- (2, 0) node[at end, below] {$x$};
        \draw[->](0, 0) -- (0, 2) node[at end, left] {$y$};
        \draw[dashed, thin](0, 0) -- (1.41, 1.41) node[midway, below] {$r$};
        \draw (0.5, 0) arc (0:45:0.5) node[midway, right] {$\theta$};
        \draw[->] (1.41, 1.41) -- (1, 1.82) node[midway, above] {$\vv{F}$};
        \draw[->] (1.41, 1.41) -- (1.82, 1.82) node[at end, right] {$\hat{u}_r$};
    \end{tikzpicture}    
\end{wrapfigure}
Le forze non centrali a simmetria sferica sono delle forze il cui modulo dipende dalla distanza
e dal fatto che non è centrale (per cui la direzione non è radiale e quindi diretta
verso un punto specifico). Queste forze non conservano il momento angolare.
Prendendo in considerazione una curva qualsiasi posso ottenere la circuitazione
come
\begin{gather*}
    \oint \vv{F}\vv{dr} = F(r) 2\pi R   
\end{gather*}
Essendo questo integrale non uguale a zero, allora la forza non è conservativa.
Per cui tutte le forze non centrali a simmetria sferica non sono conservative. 

\subsection{Forze centrali non a simmetria sferica}
\begin{wrapfigure}{r}{0.2\textwidth}
    \centering
    \begin{tikzpicture}
        \draw[->](0, 0) -- (2, 0) node[at end, below] {$x$};
        \draw[->](0, 0) -- (0, 2) node[at end, left] {$y$};
        \draw[->, thick] (0, 0) -- (1, 0) node[at start, left] {$C$} node[midway, below] {$R$} node[at end, below] {$A$};
        \draw[->, thick] (0, 1) -- (0, 0) node[at start, left] {$B$};
        \draw[dashed] (0, 0) -- (0.7, 0.7);
        \draw[->, thick](0.7, 0.7) -- (1, 1) node[at end, right] {$\vv{F}$};
        \draw[->, thick] (1, 0) arc (0: 90: 1); 
    \end{tikzpicture}    
\end{wrapfigure}
Queste forze sono centrali ma non hanno una simmetria sferica per cui il loro modulo cambia
se cambia la loro posizione sulla sfera (ma non la distanza dal centro).
In questo caso la forza sarà data dalla seguente relazione:
\begin{gather*}
    \vv{F} = F(r)\cos\theta\hat{u}_r  
\end{gather*}
Allora l'integrale di linea:
\begin{gather*}
    \oint \vv{F}\vv{dr} = \int_{C}^{A} \vv{F}\vv{dr} + \int_{A}^{B} \vv{F}\vv{dr} + \int_{B}^{C}\vv{F}\vv{dr}           
\end{gather*}
Quindi l'integrale di linea, risolvendo tutti gli integrali spezzati diventa:
\begin{gather*}
    \oint \vv{F}\vv{dr} = \int_{r = 0}^{r_A = R} F(r)dr + 0 + 0   
\end{gather*}
Così come nel disegno, la forza non è simmetrica rispetto ad una sfera poiché 
non dipende solo ed esclusivamente dal modulo del raggio vettore
che congiunge la forza al centro della sfera ideale ma anche dall'orientazione
del vettore forza. 

\clearpage
\subsection{La forza di richiamo di una molla vista come forza centrale a simmetria sferica}
\begin{wrapfigure}{r}{0.3\textwidth}
    \centering
    \caption{Molla tridimensionale}
    \begin{tikzpicture}
        \draw[decoration={aspect=0.3, segment length=2mm, amplitude=1.3mm,coil},decorate,opacity=0.9] (0, 0) -- (2, 1);
        \filldraw(0, 0) circle (1pt) node[anchor = east] {$C$};
        \filldraw(2, 1) circle (1pt) node[anchor = south] {$m$};
        \draw[->](2, 1) -- (2.5, 1.25) node[at end, above] {$\vv{F}_d$};
        \draw[|-|](0.1, 0.6) -- (1, 1) node[midway, above] {$l_0$};
        \draw[->](0, 0)  -- (1, 0) node[at end, below] {$x$};
        \draw[->](0, 0) -- (0, 1) node[at end, left] {$y$};
        \draw(0.75, 0.4) arc (30:0:0.8) node[midway, right] {$\theta$};
    \end{tikzpicture}    
\end{wrapfigure}
Attaccando una molla ad vincolo qualsiasi, quando si tira la molla
si ha una forza di richiamo verso il centro della molla: la forza
è di tipo centrale a simmetria sferica e avrà quindi una certa 
energia potenziale e sarà anche conservativa .
\begin{gather*}
    \vv{F}_d = -k(r - l_0)\hat{u}_r \\
    V = \frac{1}{2}k(r - l_0)^{2}   
\end{gather*}
Allora l'energia meccanica diventa nel caso di un certo angolo rispetto da uno degli
assi nel piano di giacenza della molla:
\begin{gather*}
    E = \frac{1}{2}mv^{2} + \frac{1}{2}kr^{2} \\
    \vv{r} = r\hat{u}_r \\
    \vv{v} = \dot{r}\hat{u}_r + r\dot{\theta } \hat{u}_{\theta}        \\
    v^{2} = \dot{r}^{2} + r^{2}\dot{\theta}^{2}      
\end{gather*}
La sola energia meccanica non basta per risolvere il problema
poiché si hanno due gradi di libertà.

\subsection{Piattaforma ruotante con $\dot{\vec{\omega}} = 0$. Risoluzione mediante il teorema
delle forze vive}
\begin{wrapfigure}{r}{0.4\textwidth}
    \centering
    \caption{La piattaforma ruotante}
    \begin{tikzpicture}
        \node[ellipse,
        draw,
	    minimum width = 4cm, 
	    minimum height = 2cm] (e) at (0,0) {};
        \draw[dashed, very thin, ->](0, -1.5) -- (0, 1.5) node[at end, above] {$z$};
        \draw[->](0.5, 1.3) -- (0.5, 1.6) node[midway, right] {$\vv{\omega}$};
        \draw[|-|](0, 0) -- (1, 0) node[midway, below] {$\rho$};
        \filldraw(1, 0) circle(1pt) node[anchor = south] {$m$} node[anchor = north] {$P$};
    \end{tikzpicture}    
\end{wrapfigure}
Considerata una piattaforma ruotante ed un sistema di riferimento $S'$ solidale
con la piattaforma e avente origine nel centro ed asse $z$ coincidente con l'asse
di simmetria. Rispetto ad un osservatore inerziale posto fuori dalla piattaforma
il moto non è inerziale e risulta accelerato da delle forze che non 
sembrano rispettare le leggi della dinamica.
Un punto $P$ sulla piattaforma di massa $m$ a distanza $\rho$ dal centro
della piattaforma è soggetto allora ad una certa forza di trascinamento
\begin{gather*}
    \vv{F_t} = -m\vv{\omega}\times(\vv{\omega} \times (\rho\hat{u}_\rho  + z\hat{k})  )  = +m\omega^{2}\rho\hat{u}_\rho  
\end{gather*}
Vale solo il contributo di $\rho$ poiché l'oggetto non si muove lungo la verticale.
Nel caso di un piccolo spostamento $d\vv{r}$ allora si ottiene:
\begin{gather*}
    d\vv{r} = d\rho \hat{u}_\rho + \rho d\hat{u}_\rho + dz\hat{k}      
\end{gather*} 
Allora si ottiene proprio il lavoro come:
\begin{gather*}
    \delta L = \vv{F}_t d\vv{r} = (m\omega^{2}\rho\hat{u}_\rho  )\cdot (d\rho\hat{u}_\rho + \rho \vv{dr} + dz\hat{k}   )  = m\omega^{2}\rho d\rho  
\end{gather*}
L'energia potenziale centrifuga è data da  
\begin{gather*}
    V = -\frac{1}{2}m\omega^{2}\rho^{2}  
\end{gather*}
La forza di coriolis
\begin{gather*}
    \vv{F}_{co} -2\vv{\omega}\times \vv{v}   
\end{gather*}
La forza di Coriolis non fa lavoro poiché è sempre ortogonale
allo spostamento infatti:
\begin{gather*}
    \delta L = \vv{F}\cdot \vv{dr} = (-2m\vv{\omega} \times \vv{v}  ) \cdot  \vv{v} \cdot dt = 0   
\end{gather*}
Fa sempre lavoro zero e quindi non altera il bilancio energetico
mentre se si muove allora si cambiano le condizioni e quindi potrebbero
non valere più le condizioni di vincolo liscio. \\
Nel SDR non inerziale in assenza di attriti e altre forze, si conserva
l'energia meccanica:
\begin{gather*}
    E = \frac{1}{2}mv^{2} - \frac{1}{2}m\omega^{2}\rho^{2}   
\end{gather*} 
Per un osservatore esterno alla piattaforma in un SdR inerziale, l'energia meccanica
non si conserva. Infatti un'osservatore esterno vede che l'energia cinetica
cambia e non vede altre forze conservative(la centrifuga non c'è) tuttavia l'energia
meccanica non si conserva poiché c'è $\omega$ costante data dal motore
della piattaforma ruotante che compie lavoro per tenere un certo regime di rotazione della piattaforma. 
Nel caso della pallina che è su questa piattaforma:
\begin{gather*}
    t = 0 \qquad \rho = \rho_0 \qquad \dot{\rho} = 0 \qquad E = -\frac{1}{2}m\omega^{2}\rho_0^{2}   \\
    t = t_f \qquad \rho = R \qquad \dot{\rho} = ? \qquad E = \frac{1}{2}m\dot{\rho}^{2}- \frac{1}{2}m\omega^{2}R^{2}      
\end{gather*} 
Dato che l'energia si conserva allora posso uguagliarle e ottenere:
\begin{gather*}
    \dot{\rho} = \pm \omega\sqrt{R^{2} - \rho_0^{2}}  
\end{gather*}
Ha senso? Si perché R è il bordo della giostra e quindi quella radice
è ben definita e quindi è vero sia per la conservazione dell'energia
ed in tutti i casi del movimento da $\rho_0 $ a $R$ e viceversa. 
Derivando l'espressione dell'energia posso ottenere l'equazione di moto
\begin{gather*}
    \dot{\rho}(m\ddot{\rho} - m\omega^{2}\rho) = 0 
\end{gather*}

\subsection{Il sistema dell'antimolla}
Se una molla è fissata al centro di una piattaforma
ruotante con una massa, se vince il contributo della rotazione 
allora si ha una antimolla poiché la massa attaccata alla molla tende ad uscire
dalla piattaforma ruotante vincendo la costante della molla; altrimenti
si ha un moto armonico semplice se $k >> \omega$. Posso descrivere
l'energia meccanica come il contributo rotatorio tangenziale più il contributo
rotatorio puro e la forza di richiamo della molla: 
\begin{gather*}
    E = \frac{1}{2}m\dot{\rho^{2} } - \frac{1}{2}m\omega\rho^{2} + \frac{1}{2}k \rho^{2}   
\end{gather*}

\section{Studio dell'equilibrio del sistema}
Nello studio di un sistema con vincoli ideali e forze conservative, allora
lo studio della posizione di equilibrio è relativamente facile.
In un qualsiasi SdR, tenendo conto delle forze apparenti e riconducendoci
dunque al caso ideale, un corpo rimane fermo (equilibrio più semplice) quando
la risultante è zer. Se le forze sono conservative allora se la risultante è nulla
vuol dire che si ha un solo grado di libertà e dunque si può esprimere la
forza risultante come il gradiente dell'energia potenziale:
\begin{gather*}
    \vv{F} = -\vv{\nabla} V = 0  
\end{gather*}
Nei campi di forze conservativi, è spesso utili individuare i punti in cui l'energia 
potenziale assume lo stesso valore. Il luogo geometrico di tali punti
è detta \textbf{superficie equipotenziale}. Per ogni spostamento
lungo questa superficie il lavoro infinitesimo $\delta L$ è nullo e da ciò segue che
la forza è sempre perpendicolare rispetto alla superficie. Questa è proprio
la proprietà generale del gradiente di una funzione scalare. L'energia
potenziale allora avrà un minimo se
\begin{gather*}
    \frac{dV}{dx} = 0
\end{gather*}
Posso poi valutare la derivata seconda rispetto al grado di libertà per determinare la stabilità dell'equilibrio:
se la derivata seconda è maggiore di zero, allora l'equilibrio è \textbf{stabile},
altrimenti \textbf{instabile}.

\subsection{L'equilibrio nel pendolo}
\begin{wrapfigure}{r}{0.4\textwidth}
    \centering
    \caption{Pendolo}
    \begin{tikzpicture}[scale = 1.2]
        \draw(0, 0) -- (2, 0);
        \draw(1, 0) -- (1, -2) node[midway, left] {$L$};
        \filldraw(1, -2) circle (1pt)node[anchor = east] {$m$};
        \draw[->, thick](1, 0) -- (1.5, 0) node[at end, below] {$x$};
        \draw[->, thick](1, 0) -- (1, -0.5) node[at end , right] {$y$};
    \end{tikzpicture}    
\end{wrapfigure}
Considerato un pendolo semplice, è possibile impostare il sistema di riferimento in modo tale che
l'unico grado di libertà sia l'angolo $\theta$: si identifica col SdR centrato nella massa l'energia potenziale 
gravitazionale in base alla quota assunta dalla massa del pendolo rispetto
al SdR centrato sul perno del pendolo:
\begin{gather*}
    V = mgy
\end{gather*}
Le energie potenziali per un punto $V_P$ si riferisce alla quota rispetto alla 
massa $m$ per cui quando $\theta = 0$ allora l'energia potenziale nel sistema di riferimento scelto
varierà di un certo $\Delta V = mgL\cos\theta$:
\begin{gather*}
    V_P = (L - L\cos\theta)mg 
\end{gather*}
Derivando:
\begin{gather*}
    \frac{dV_P}{d\theta} = -mgL(-\sin\theta) = mgL \sin\theta
\end{gather*}
Allora quando $\theta = 0$ la derivata è zero. Se avessi voluto considerare invece
la posizione della massa $m$ non in funzione dell'angolo rispetto alla verticale passante per 
un dato sistema di riferimento, ma secondo la sua posizione $(x, y)$ rispetto 
ad un dato SdR fisso, allora non avrei potuto trovare l'equazione di moto 
dall'energia potenziale. La scelta intelligente del grado di libertà si rivela
essere dunque cruciale per poter analizzare le situazioni fisiche.  
Se il pendolo fosse composto da un filo resistente alla trazione, allora 
$\theta = 0$ sarebbe l'unico punto di equilibrio, altrimenti, se si considerasse
il pendolo costituito da una sbarra, i punti di equilibrio sarebbero due:  
\begin{gather*}
    \theta = \left\{\begin{array}{l}
        0 \\
        \pi
    \end{array}\right.
\end{gather*}
È possibile ora valutare le condizioni di equilibrio e determinare
la loro stabilità attraverso la derivata del secondo ordine:
\begin{gather*}
    \frac{d^{2} V}{d\theta^{2} } = mgL\cos\theta
\end{gather*}
Quindi:
\begin{gather*}
    V^{''}(0) = mgL > 0, \qquad  V^{''}(\pi) = -mgL < 0,
\end{gather*}
Quindi in $\theta = \pi$ c'è proprio un massimo (equilibrio instabile) per l'energia potenziale
ed un minimo per $\theta = 0$ (equilibrio stabile).
In più dimensioni la stazionarietà si ottiene con le derivate parziali di
tutte le variabili:
\begin{gather*}
    \frac{\partial V}{\partial x} = 0, \dots 
\end{gather*}
E quindi facendo la derivata parziale due volte rispetto
ad x, y... allora la sua forma può essere o un massimo o un minimo
in funzione proprio delle variabili si ottengono o paraboloidi, oppure una
sella: una condizione di minimo in una variabile ed una di massimo in un'altra.
Sarebbe stabile in una direzione ma sostanzialmente è instabile: nel punto di sella
infatti non c'è una vera stabilità (analisi due).

\subsection{L'equilibrio di una molla sul soffitto}
Nella situazione di una molla appesa al soffitto, scelto come SdR il punto
di riposo della molla (e considerato che la molla non oscilli rispetto alla
direzione della forza peso) posso impostare l'energia potenziale di una massa
$m$ a distanza $x$ dalla posizione di riposo:
\begin{gather*}
    V = \frac{1}2{kx^{2} } - mgx
\end{gather*}
Per trovare la condizione di equilibrio faccio allora la derivata
rispetto ad $x$:
\begin{gather*}
    \frac{dV}{dx} = kx -mg
\end{gather*}
E quindi si annulla proprio per $kx = mg$ ossia dove le due
forze si eguagliano in funzione dell'unica coordinata $x$ e quindi
un solo punto di equilibrio. Troviamo ora se è instabile o stabile
attraverso la derivata seconda:
\begin{gather*}
    \frac{d^{2} V}{dx^{2} } = k > 0 \ \Longrightarrow \  \text{equilibrio stabile}
\end{gather*} 

\clearpage
\subsection{L'equilibrio nell'antimolla (Non da fare)}
\begin{wrapfigure}{r}{0.4\textwidth}
    \centering
    \label{FIg}
    \caption{Antimolla}
    \begin{tikzpicture}
        \draw[decoration={aspect=0.3, segment length=1.2mm, amplitude=1mm,coil},decorate,opacity=0.9] (0, 2) -- (0.5, 1.8);
        \draw[dashed](0, 0) -- (0, 2) node[midway, left] {$L$};
        \draw(0, 0) -- (0.5, 1.8) node[midway, right] {$L$};
        \draw(0, 0.5) arc(90:75:0.5) node[midway, above] {$\theta$};
        \filldraw(0.5, 1.8) node[anchor = west] {$m$};2
    \end{tikzpicture}    
\end{wrapfigure}
Nell'antimolla l'asta ha massa zero e vincola il punto materiale ad avere solo un grado di libertà
facendo sì che l'energia potenziale della massa diventi proprio:
\begin{gather*}
    V = \frac{1}{2}kL^{2}\theta^{2} + mgl\cos\theta  
\end{gather*}
E allora si ottiene l'espressione, utilizzando taylor per il coseno nel caso in cui
si ha $\theta << 1$:
\begin{gather*}
    \cos\theta \approx 1 - \frac{\theta^{2} }{2} \\
    V \approx \frac{1}{2}kL^{2}\theta^{2} + mgl\left(1 - \frac{\theta^{2} }{2}\right)  
\end{gather*}
Facendo la derivata prima si ottiene la soluzione:
\begin{gather*}
    \frac{dV}{d\theta} = L(kL - mg)\theta \\
    \theta = 0
\end{gather*}
C'è anche il caso in cui $kL = mg$; facendo invece la derivata seconda si ha:
\begin{gather*}
    \frac{d^{2} V}{d\theta^{2} } = L(kL -mg)
\end{gather*}
Se $kL > mg$ allora la derivata seconda è maggiore di zero
ed ho un equilibrio stabile, altrimenti, se la derivata seconda è minore
di zero,ho un equilibrio instabile; se sono uguali allora fisicamente
la derivata seconda è zero e significa che è un equilibrio "indifferente".

\section{Studio del moto con un piccolo spostamento dall'equilibrio}
In funzione di una certa distanza $x$ rispetto alla posizione di equilibrio $x_{eq}$
è dato dallo sviluppo di taylor della funzione dell'energia potenziale:
\begin{gather*}
    V(x) = V(x_{eq}) + \frac{dV}{dx}|_{x = x_{eq}}(x - x_{eq}) + \frac{1}{2}\frac{d^{2} V}{dx^{2} }|_{x = x_{eq}}(x - x_{eq})^{2} + \dots 
\end{gather*}
Allora date le definizioni di prima con le derivate prime e seconde si ha proprio:
\begin{gather*}
    V(x) = c + \frac{1}{2} k(x - x_{eq})^{2}   
\end{gather*}
Dove $V_{eq} = c$ e $k$ è un parametro della derivata seconda che può 
essere positivo o negativo.
\begin{gather*}
    k = \left\{\begin{array}{l l}
        k > 0 & \text{moto armonico}\\
        k < 0 & \text{moto esponenziale}
    \end{array}\right.
\end{gather*}
\begin{wrapfigure}{r}{0.4\textwidth}
    \centering
    \caption{Il grafico dell'energia potenziale}
    \begin{tikzpicture}
        \draw[->](0, 0) -- (0, 3) node[at end, left] {$V(x)$};
        \draw[->](0, 0) -- (3, 0) node[at end, below] {$x$};
        \draw( 0, 2) ..controls(1, -1) and (1.5, 4) .. (3, 0);
        \draw[thin, dashed] (0, 1.25) -- (3, 1.25) node[at start, left] {$E_1$}; 
    \end{tikzpicture}    
\end{wrapfigure}
Questo vuol dire che possiamo sapere, discostandosi di poco
rispetto all'equilibrio, il tipo di moto che accadrà. 
Spesso si trova il grafico rispetto al parametro e quindi, quando
si ha un vincolo liscio, si conserva l'energia e quindi essendo l'energia
la somma tra la potenziale e la cinetica, l'energia cinetica è sempre
positiva per definizione. Questo però ci pone dei limiti poiché
una è sempre positiva o zero e l'altra cambia sempre di segno e l'energia
meccanica è sempre costante. Si ha quindi una situazione in cui
ci sono alcune configurazioni  permesse mentre altre non sono proprio possibili
a causa della natura dell'energia cinetica: nel caso in figura le configurazioni
permesse sono sempre quelle sotto l'energia meccanica, altrimenti l'energia cinetica dovrebbe essere
negativa.
\begin{gather*}
    E = \frac{1}{2}mv^{2} + V(x) \\
    v = \pm \sqrt{2\frac{E - V(x)}{m}}   
\end{gather*}
Per costruzione, dipendendo da $x$ cambierà il segno. Si può allora
fare qualcosa di più e quindi essendo la velocità
\begin{gather*}
    v = \frac{ds}{dt} , \quad v = \dot{s}, \quad s = s(x). 
\end{gather*}
Posso separare le variabili $s, t$ e quindi:
\begin{gather*}
    \frac{ds}{ \pm \sqrt{2\frac{E - V(s)}{m}} } = dt
\end{gather*}
Prendendo allora l'integrale da entrambe le parti tra due intervalli
di tempo $t_1, t_2$ e $s_1, s_2$ allora:
\begin{gather*}
    \int_{s_1}^{s_2} \frac{ds}{ \pm \sqrt{2\frac{E - V(s)}{m}} } = \int_{t_1}^{t_2}dt \\
    2(t_2 - t_1) = \int_{s_1}^{s_2} \sqrt{\frac{2m}{E - V(s)}} \ ds.   
\end{gather*}
In linea di principio posso calcolarmi questo integrale nelle piccole
oscillazioni. Si introduce allora il concetto di \textbf{barriera di potenziale}:
ossia l'altezza oltre la quale la pallina non può muoversi
tra due cunette, non può allora andare nell'altra cunetta e
classicamente non può essere attraversato. (fluidi)


\part{Dinamica dei sistemi e corpo rigido}
\chapter{Dinamica dei sistemi}
\section{Sistemi non approssimabili ad un punto materiale}
Nella dinamica dei sistemi non si hanno più sistemi formati da un solo punto materiale ma da degli
oggetti estesi  che possono essere divisi in tanti punti materiali 
per studiarne la dinamica. In generale posso scegliere quali corpi posso integrare nel mio sistema
a seconda di come è più comodo.
Per ciascun punto conosco massa e vettore posizione e posso allora determinare il centro di massa che,
in un sistema discreto, è il un punto ideale definito dal vettore
\begin{align}
    \vv{r_c} = \frac{\sum_{i = 1}^{N}m_i \vv{r_i}}{\sum_{i = 1}^{N} m_i}   
\end{align}
dove $m_i$ è la massa di ogni punto materiale. Posso definire nella stessa
maniera la massa del centro di massa come la somma delle masse totali e quindi
ottenere l'espressione di prima come:
\begin{gather*}
    \vv{r_c} = \frac{1}{M}\sum_{i = 1}^{N}m_i \vv{r_i} .
\end{gather*}
In un sistema continuo non posso semplicemente fare l'integrale di questa espressione
ma devo invece considerare un volumetto elementare di massa molto piccola
$dm$ e volume $dV$ e, se sono sufficientemente piccole, potrò utilizzare la relazione $dm = \rho dV$, che è
funzione della posizione $\vv{r}$; posso sostituire questa
espressione con un integrale come:
\begin{gather*}
    \vv{r} = \frac{\int_{V} \vv{r} \rho dV}{\int_{V} \rho dV} 
\end{gather*}
Possiamo definire la \textbf{densità volumetrica di massa} (o semplicemente densità) come
il limite a cui tende il suo valore medio quando tende a zero il volume:
\begin{align}
    \rho = \lim_{\Delta V \to 0}\frac{\Delta m}{\Delta V} = \frac{dm}{dV} 
\end{align}
Possiamo anche definire la \textbf{densità superficiale} se il sistema è 
schematizzabile come una superficie:
\begin{align}
    \sigma = \lim_{\Delta S \to 0}\frac{\Delta m}{\Delta S} = \frac{dm}{dS} 
\end{align}
Nel caso in cui il sistema sia schematizzabile con una linea posso definire
la \textbf{densità lineare}:
\begin{align}
    \lambda = \lim_{\Delta l \to 0}\frac{\Delta m}{\Delta l} = \frac{dm}{dl} 
\end{align}

\clearpage
\section{Centro di massa nelle figure complesse}
\subsection{Le figure piane}
Nel caso di due masse, il centro di massa si trova tra le due masse. 
Nel caso in cui una delle due masse sia molto maggiore dell'altra allora
il centro di massa sarà spostato verso la massa maggiore. Ma se ci fossero molte più masse?
Se io avessi delle figure geometriche semplici, allora il centro di massa risiede 
sull'asse di simmetria delle masse. Si chiama allora asse di simmetria
un asse sul quale ogni massa è tale per cui c'è sempre un'altra massa a distanza uguale.
Un sistema si dice \textbf{omogeneo} se la sua densità è costante, altrimenti è \textbf{non omogeneo}:
nel caso del vettore del centro di massa:
\begin{gather*}
    \vv{r_C} = \frac{\int_{V}\vv{r} \rho dV  }{\int_{V}\rho dV} \ \Longrightarrow \  \frac{\int_{V}\vv{r} dV  }{\int_{V} dV} 
\end{gather*}
\begin{wrapfigure}{r}{0.4\textwidth}
    \centering
    \begin{tikzpicture}
        \draw(0, 0) -- (3, 0) node[at start, below] {$A$} node[at end, below] {$B$};
        \draw(0, 0) -- (1, 2);
        \draw(1, 2) -- (3, 0);
        \draw(0.05, 0.1) -- (2.95, 0.1);
        \draw(1, 2) -- (1.5, 0);
    \end{tikzpicture}    
\end{wrapfigure}
Nel caso di simmetrie o strutture note, come nel caso di una figura piana di cui
conosco gli assi di simmetria tipo il quadrato, il centro degli assi di simmetria
sarà proprio il centro del quadrato. Nel caso di figure meno semplici, come
già il triangolo, la posizione del centro di massa non è intuitiva
poiché il centro del triangolo non è così evidente. Scegliendo uno dei lati
possiamo prenderne una fetta parallela al lato scelto ed indicarne
il centro e ripetere il procedimento fino a tracciare la mediana
rispetto all'angolo opposto, faccio lo stesso per tutti gli altri lati
e quindi, essendo che tutte e tre le mediane si incontrano esattamente a due terzi
della lunghezza delle mediane, allora quello è il centro di massa del triangolo. 

\subsection{Il centro di massa nelle figure solide: l'esempio del cono}
\begin{wrapfigure}{r}{0.4\textwidth}
    \centering
    \caption{Il cono. $r$ è funzione della posizione $x$.}
    \begin{tikzpicture}
        \draw[->](-1, 0) -- (5, 0) node[at end, below] {$x$};
        \draw(0, 0) -- (3, 1.5);
        \draw(0, 0) -- (3, -1.5);
        \node[ellipse,
        draw,
	    minimum width = 1cm, 
	    minimum height = 3cm] (e) at (3,0) {};
        \draw[dashed](3, -1.5) -- (3, 1.5) node[midway, anchor = south east] {$R$};
        \node[ellipse,
        draw,
	    minimum width = 0.7cm, 
	    minimum height = 1.45cm] (e) at (1.5,0) {};
        \draw[dashed](1.5, -0.72) -- (1.5, 0.72) node[midway, anchor = south east] {$r$};
    \end{tikzpicture}    
\end{wrapfigure}
Nel caso di una figura solida come un cono, lo scopo è cercare di ridurre
gli integrali bi-tridimensionali in integrali unidimensionali con certe approssimazioni
e con certi procedimenti. Nel caso in cui il cono considerato sia omogeneo,
posso considerare delle piccole circonferenze e calcolarne il volume infinitesimo
per poter determinare il centro di massa:
\begin{gather*}
    x_c = \frac{\int_{V} xdV}{\int_{V} dV} 
\end{gather*}
considerando che il cono abbia densità uniforme e che sia
simmetrico rispetto all'asse $x$ sul quale risiede la sua altezza,
il volume infinitesimo da trovare di una sezione $dx$:
\begin{gather*}
    dV = 2\pi rdx
\end{gather*}
Dato che conosco la distanza dalla cima ($h$)
posso ricavare il raggio  $r$ in funzione del raggio della base $R$ e
della sua quota $x$ rispetto alla punta del cono. 
\begin{gather*}
    r = \frac{R}{h}x \ \Longrightarrow \ dV = \pi \frac{R^{2} }{h^{2} }x^{2}dx 
\end{gather*} 
Sono passato da un integrale tridimensionale fino ad un integrale
unidimensionale:
\begin{gather*}
    \vv{r_c} = \frac{\int x\left(\pi \frac{R^{2} }{h^{2} }x^{2}\right)}{\int \left( \pi \frac{R^{2} }{h^{2} }x^{2} \right) } dx
\end{gather*}
Allora risolvendo:
\begin{gather*}
    x_c = \frac{\left. \frac{x^{4} }{4} \right|^{h}_{0}  }{\left. \frac{x^{3} }{3} \right|^{h}_{0}  } = \frac{3}{4}h
\end{gather*}

\section{Quantità di moto}
Possiamo definire la quantità di moto come il \emph{vettore libero}:
\begin{gather*}
    \vv{q} = m\vv{v}  
\end{gather*}
la quantità di moto del centro di massa è proprio la somma dei contributi
di tutte le quantità di moto all'interno del sistema dato dalla seguente: 
\begin{gather*}
    \vv{Q} = \sum_{i = 1}^{n} m_i \vv{v_i}   
\end{gather*}
Analogamente posso considerare la quantità di moto di un sistema finito come l'integrale dei
singoli prodotti della velocità per il differenziale della massa:
\begin{align}
    \vv{Q} = \int \vv{v} \ dm  
\end{align}
Se volessi ottenere la velocità del centro di massa allora
dovrei derivare rispetto al tempo la posizione del centro di massa:
\begin{gather*}
    \vv{v_c} = \dot{\vv{ {r}_c}} = \frac{\sum_{i = 1}^{n} m_i \vv{v_i} }{M}  
\end{gather*}
Se si deriva invece la quantità di moto del sistema si ottiene la derivata
delle quantità di moto dei singoli punti materiali oppure la somma delle masse per l'accelerazione
dei singoli punti:
\begin{gather*}
    \dot{\vv{Q} } = \sum_{i = 1}^{n}\dot{\vv{q}_i } =  \sum_{i = 1}^{n}m_i \vv{a}_i     
\end{gather*}
ossia la risultante della forza che agisce sul corpo:
\begin{align}
        \dot{\vv{Q} } = \sum_{i = 1}^{n}\vv{F}_i    
\end{align}
Possiamo scomporre la forza totale come la somma tra le forze esterne al sistema e le
forze interne (dovute all'interazione tra i vari corpi del sistema stesso):
\begin{gather*}
    \dot{\vv{Q} } = \sum_{i = 1}^{n}\vv{F}_i   = \sum_{i = 1}^{n}\vv{F}_i^{(INT)}  + \sum_{i = 1}^{n}\vv{F}_i^{(EXT)} 
\end{gather*}
Per le forze interne devo considerare il terzo principio della dinamica per ogni punto, che mi porta, 
per ogni forza applicata ad un punto, a dire che ce ne sia una uguale e contraria per ogni punto del sistema;
di conseguenza la sommatoria delle forze interne è equivalente a zero:
\begin{gather*}
    \boxed{\sum_{i = 1}^{n}\vv{F}_i^{(INT)}  =  \sum_{i = 1}^{n}\sum_{j = 1, j \neq i}^{n}\vv{F}_{i, j} = 0 }   
\end{gather*}
Dalla prima cardinale si ha che la derivata della quantità di moto non è altro che
la sommatoria delle forze esterne applicate al corpo:
\begin{gather*}
    \dot{\vv{Q} } = \sum_{i = 1}^{n}\vv{F}^{(EXT)}_i = \vv{F}^{(EXT)} 
\end{gather*}
che permette di descrivere i sistemi più complessi con descrizioni molto semplici. 
Si arriva anche alla seconda equazione del centro di massa:
\begin{align}
        \boxed{\vv{F} = M\vv{a}_c} 
\end{align}
Se la risultante delle forze esterne che agisce su di un corpo è
zero, allora la quantità di moto è conservata poiché
\begin{gather*}
   \vv{F}^{EXT} = 0 \ \Longrightarrow \    \dot{\vv{Q}} = 0 \ \Longrightarrow \  \vv{Q} = \text{costante}  
\end{gather*}
che prende il nome di \textbf{prima equazione cardinale della dinamica}.
Quando un sistema non è soggetto a forze esterne, e dunque la sua quantità di
moto rimane costante (cioè si conserva), allora tale sistema prende il nome di \textbf{isolato}.

\clearpage
\subsection{Esempio di sistema isolato}
\begin{wrapfigure}{r}{0.4\textwidth}
    \centering
    \label{}
    \caption{}
    \begin{tikzpicture}
        \draw(0, 0) rectangle (1, 1) node[midway] {$m_1$};
        \draw(3, 0) rectangle (4, 1) node[midway] {$m_2$};
        \draw[decoration={aspect=0.3, segment length=1.5mm, amplitude=1.3mm,coil},decorate,opacity=0.9] (1, 0.5) -- (3,0.5);
        \draw[->](-1, 0) -- (-0.5, 0) node[at end, below] {$x$};
        \draw[->](-1, 0 ) -- (-1, 0.5) node[at end, left] {$y$};
    \end{tikzpicture}    
\end{wrapfigure}
Considerando un sistema composto da due masse attaccate ad una molla, posso dire che questo sistema
è isolato in quanto non agiscono forze esterne (la forza peso sulle $y$ è
bilanciata dalla reazione vincolare del piano).
Questo sistema non riesco ad esprimerlo mediante $F = ma$ e nemmeno attraverso
la conservazione dell'energia poiché ho due gradi di libertà. Essendo 
che la forza elastica è una forza interna, lungo l'orizzontale si conserva la quantità
di moto; dunque, impostando la quantità di moto, posso ridurre il grado di libertà del sistema ad uno solo: è
quindi possibile applicare il principio della conservazione dell'energia
meccanica: 
\begin{gather*}
    E  = \frac{1}{2}m_1v_1^{2} + \frac{1}{2}m_2v_2^{2} + \frac{1}{2}k(x_2 - x_1 - l_0)^{2}  
\end{gather*}
e metterlo a sistema con la conservazione della quantità di moto lungo l'asse $x$:
\begin{gather*}
    Q_x = m_1v_1 + m_2v_2
\end{gather*}
La quantità di moto mi permette dunque, sotto specifiche condizioni, di prendersi
in carico un grado di libertà. 

\section{Il Momento angolare}
Dato che un sistema di $n$ punti ha $n \cdot 3$ gradi di libertà, 
allora dobbiamo trovare un'analoga per le forze esterne in modo tale che possa descrivere tutti i gradi di libertà.
Si definisce allora il momento angolare per un sistema di punti come
\begin{align}
    \vv{L}_{\Omega} =  \sum_{i = 1}^{n} (P_i - \Omega) \times m_i \vv{v_i}  = \sum_{i = 1}^{n} (P_i - \Omega) \times \vv{q_i}  = \sum_{i = 1}^{n}\vv{l_i} 
\end{align}
Il momento angolare è una quantità vettoriale applicata rispetto ad un punto
qualsiasi $\Omega$ che prende il nome di \textbf{polo}, il quale non deve per forza
coincidere con l'origine degli assi cartesiani (è totalmente
arbitrario), che descrive il momento rotatorio di un certo sistema di punti
rispetto al polo di riferimento. Per ciascun punto del sistema posso definire il vettore
posizione e spostamento come ($P_i - \Omega$) e dunque tutte le definizioni del momento angolare sono 
ugualmente valide, così come la somma dei singoli momenti angolari ed il momento
angolare del sistema o del singolo punto è una quantità vettoriale rispetto ad
un punto qualsiasi $\Omega$. Scegliendo un sistema di riferimento
qualsiasi, posso allora definire la distanza rispetto all'origine delle coordinate come:
\begin{gather*}
    \vv{r}_{i, \Omega} = \vv{r}_i - \vv{r}_{\Omega}  
\end{gather*}

\section{Seconda equazione cardinale della dinamica}
Derivando rispetto al tempo il momento angolare, ottengo la seguente espressione:
\begin{gather*}
    \dot{\vv{L} }_{\Omega} = \sum_{i = 1}^{n}(P_i - \Omega) \times m_i \vv{a}_i + \sum_{i = 1}^{n} (\vv{v}_i - \vv{v}_{\Omega}) \times m_i \vv{v}_i     
\end{gather*}
Noi sappiamo che per la seconda di Newton la risultante delle forze è proprio
la somma delle masse per accelerazione, inoltre, dato che $\vv{v}_i \times \vv{v}_i = 0$, 
posso fare la somma solo sul secondo prodotto vettoriale della seconda somma ottenendo allora
una nuova espressione della derivata del momento angolare:
\begin{gather*}
    \dot{\vv{L} } = \sum_{i = 1}^{n} (P_i - \Omega) \times \vv{F}_i - \vv{v}_{\Omega}\times \sum_{i = 1}^{n}m_i\vv{v}_i      
\end{gather*}
Considerando allora la definizione di momento di una forza rispetto ad un polo
$\Omega$ qualsiasi, come:
\begin{align}
    \vv{M} = (P - \Omega) \times \vv{F}  
\end{align}
e dato  che le forze sono la somma delle forze interne ed esterne, posso esprimere la 
singola forza agente su di un punto $P_i$ come:
\begin{gather*}
    \vv{F}_i = \sum_{j \neq  i}^{n}\vv{F}_{i, j}^{(INT)} + \vv{F}_i^{(EXT)}      
\end{gather*}
Con queste considerazioni posso esprimere la somma dei momenti torcenti come:
\begin{gather*}
    \sum_{i = 1}^{n}(P - \Omega) \times \vv{F}_i = \sum_{i = 1}^{n}\sum_{j \neq i}^{n}  (P_i - \Omega) \times F_{i, j}^{(INT)} + \sum_{i = 1}^{n}(P_i - \Omega) \times \vv{F}_i^{(EXT)}   
\end{gather*}
Dato che la sommatoria delle forze interne è zero e non contribuisce
al momento, l'espressione della derivata del momento angolare prende il nome di 
\textbf{seconda equazione cardinale della dinamica}:
\begin{align}
\boxed{    \dot{L}_{\Omega} = \vv{M}_{\Omega}^{(EXT)} - \vv{v}_{\Omega}\times \vv{Q}     
}\end{align}
Il secondo termine è nullo quando si verifica una di queste tre condizioni:
\begin{enumerate}
    \item $\vv{v}_{\Omega} = 0$: ossia il polo è fisso rispetto al sistema di riferimento 
    inerziale considerato;
    \item $\Omega = C_M$: ossia il polo coincide con il centro di massa;
    \item $\vv{v}_{\Omega} \times \vv{v}_{CM} = 0$: la velocità del polo e del centro di 
    massa sono parallele tra di loro.    
\end{enumerate} 
Questa equazione presenta, così come la prima cardinale, 3 gradi di libertà.


\subsection{Il cambio di polo(rinuncia agli studi)} 
Se invece di determinare il momento angolare rispetto al polo $\Omega$, decidessi di utilizzassi 
il polo $\Omega'$, l'espressione del momento angolare non cambierebbe assolutamente. Infatti,
partendo dall'espressione del momento angolare:
\begin{gather*}
    \vv{L}_{\Omega} = \sum_{i = 1}^{n}(P_i - \Omega) \times m_i \vv{v}_i = \sum_{i = 1}^{n} ((P_i - \Omega') - (\Omega' - \Omega)) \times m_i \vv{v}_i     
\end{gather*}
Tornerei a quella di partenza con $\Omega \equiv \Omega'$. Se volessi spezzare
invece la sommatoria si otterrebbe:
\begin{gather*}
    \vv{L}_{\Omega} = \vv{L}_{\Omega'} + (\Omega' - \Omega) \times \vv{Q}   
\end{gather*}
Se la quantità di moto è zero allora il momento angolare non dipenderebbe dal polo scelto:
questo vuol dire che il momento angolare, se la quantità di moto è nulla, è 
indipendente dal polo scelto. Dato che si può esprimere anche il momento
delle forze esterne rispetto a questo polo nuovo $\Omega'$:
\begin{gather*}
    \vv{M}_{TOT}^{EXT} = \sum_{i = 1}^{n}(P_i - \Omega)\times \vv{F}_i = \sum_{i = 1}^{n}(P_i - \Omega') \times \vv{F} + \sum_{i = 1}^{n} (\Omega' - \Omega) \times \vv{F}_i  = \\
    \vv{M}_{TOT, \Omega'}^{EXT} + (\Omega' - \Omega) \times \vv{F}_i^{EXT}        
\end{gather*}
Se cambio quindi il polo il momento cambia, se però la risultante delle forze esterne è uguale
a zero, allora il momento risultante delle forze esterne è indipendente dal polo.
Se $\Omega' = C_M$ allora il momento angolare rispetto ad un polo generico 
è il momento angolare rispetto al centro di massa + un contributo:
\begin{align}
    \vv{L}_{\Omega} = \vv{L}_C + (C - \Omega) \times M\vv{v_C}   
\end{align}
Il contributo $(C - \Omega) \times M\vv{v_C}$, è il momento angolare che si deve aggiungere al 
centro di massa quando il corpo ruota intorno ad un asse che non passa per il centro di massa.
L'espressione dunque del momento angolare rispetto ad un polo $\Omega$ qualsiasi prende il nome di 
\textbf{terzo teorema del centro di massa}. 
Si vede che il momento angolare è sempre riferito ad un certo asse:
dato che il momento delle forze esterne non dipende dal punto di applicazione
su di un asse, allora anche il momento angolare non dipende dal punto di
applicazione su di un asse. Se invece si avesse $\Omega = C_M$ e $\vv{v}_{\Omega} = 0$?
si ottiene il seguente sistema:
\begin{align}
        \left\{\begin{array}{l}
       \vv{F}^{TOT} = \dot{\vv{Q} } \ \Longrightarrow \  \vv{F}^{EXT} = 0 \ \Longrightarrow \ \vv{Q} \text{ const}      \\
       \vv{M}_{\Omega}^{EXT} = \dot{\vv{L}}_{\Omega}  \ \Longrightarrow \ \vv{M}_{\Omega}^{EXT} = 0 \ \Longrightarrow \ \vv{L}_{\Omega} \text{ const}   
    \end{array}\right.
\end{align}
Che è proprio la \textbf{terza cardinale}: questo sistema di due equazioni equivale
ad un sistema di sei equazioni scalari e mi consente di determinare il comportamento
di sistemi che hanno al massimo $6$ gradi di libertà: questo è il motivo 
per il quale il problema dei tre corpi in astrofisica non si può risolvere
attraverso la dinamica. 
\clearpage


\subsection{Esempio per un sistema di due punti}
\begin{wrapfigure}{r}{0.2\textwidth}
    \centering
    \begin{tikzpicture}[scale =0.75]
        \draw[dashed](0, 0) -- (3, 0);
        \filldraw(0, 0) circle (1pt) node[anchor = east] {$m_1$} ;
        \filldraw(3, 0) circle (1pt) node[anchor = west] {$m_2$} ;
        \draw[->](0, 0) -- (0, -1) node[at end, left] {$\vv{v}_1$};
        \draw[->](3, 0) -- (3, 1) node[at end, right] {$\vv{v}_2$};
        \draw[->](2, 1) arc(0:180:0.5) node[midway, above] {$\vv{\omega}$};
        \filldraw(1.5, 0) circle (1pt) node[anchor = south] {$C$};
    \end{tikzpicture}    
\end{wrapfigure}
Per un sistema di due punti:
\begin{gather*}
    \vv{Q} = \vv{q}_1 + \vv{q}_2 = const \\
    \dot{\vv{Q}} = 0 = \dot{\vv{q}}_1 + \dot{\vv{q}}_2 \\
    \vv{F}_{1, 2} = -\vv{F}_{2, 1}        
\end{gather*}  
Ottenendo metà del secondo principio della dinamica. 
Posso esprimere il momento angolare come:
\begin{gather*}
    \vv{L}_C = (P_1 - C)\times m_1 \vv{v}_1 + (P_2 - C)\times m_2 \vv{v}_2 \\
    m_1 = m_2 = m \\
    |P_1 - C| = |P_2 - C| = d   \\
    |\vv{v}_1| = \omega d, |\vv{v}_2| = \omega d  
\end{gather*}


\begin{wrapfigure}{r}{0.2\textwidth}
    \centering
    \begin{tikzpicture}[scale = 0.8]
        \draw(0, 0) -- (0, 4);
        \draw(-1, 1) -- (1, 3);
        \draw[dashed](-2, 2) -- (2, 2);
        \filldraw(-1, 1) circle (1pt) node[anchor = east] {$m_2$};
        \filldraw(1, 3) circle (1pt) node[anchor = west] {$m_2$};
        \draw(0.5, 2.5) arc(45:0:0.7) node[midway, right] {$\phi$};
        \draw[->](-0.25, 3.5) arc(180: 360: 0.25) node[at end, above] {$\vv{\omega}$};
    \end{tikzpicture}    
\end{wrapfigure}
Si può ora esprimere tutto in coordinate polari attraverso l'utilizzo
di un angolo $\phi$ e quindi:
\begin{gather*}
    \left\{\begin{array}{l}
        (P_1 - C) = d\hat{u}_r  \Rightarrow  \vv{v}_1 = d\dot{\hat{u}}_r = d \dot{\phi} \hat{u}_{\phi}   \\
        (P_2 - C) = d\hat{u}_r   \Rightarrow  \vv{v}_2 = d\dot{\hat{u}}_r = d \dot{\phi} \hat{u}_{\phi}
    \end{array}\right.
\end{gather*}
In un moto di sola rotazione(non c'è traslazione del sistema che sta ruotando)
e quindi la scelta del polo non è obbligatoria. In generale il momento angolare
aiuta quando si hanno fenomeni di rotazione.

\section{Le forze parallele}
\subsection{Il centro delle forze parallele}
Il momento angolare non è sempre parallelo alla velocità angolare,
infatti date le equazioni cardinali si possono ottenere:
\begin{gather*}
    \left\{\begin{array}{l}
        \vv{F}^{(EXT)} = \dot{\vv{Q}}_{TOT} \\
        \vv{M}_{\Omega}  = \dot{\vv{L}}_{\Omega} + (\vv{r}  \times \vv{Q}_{TOT} )  
    \end{array}\right.
\end{gather*}
Se tutte le forze applicate su di un corpo sono parallele, allora posso dimostrare che questo
sistema di forze è equivalente al semplice sistema costituito solo dalla forza:
\begin{gather*}
   \sum_{i = 1}^{n} \vv{F}_i = \left(\sum_{i = 1}^{n} F_i\right) \hat{ u}    
\end{gather*}
applicata nel punto individuato dal vettore posizione 
\begin{gather*}
        \vv{r}_F = \frac{\sum_{i = 1}^{n} F_i \vv{r}_i }{\sum_{i = 1}^{n} F_i} 
\end{gather*}
Ossia la media pesata delle forze totali. Questi due sistemi di forza hanno la medesima
risultante, dunque bisogna dimostrare che abbiano lo stesso momento rispetto ad un polo qualsiasi (per esempio l'origine):
\begin{gather*}
    \vv{M} = \sum_{i} (\vv{r}_i \times F_i \hat{u}) = \left(\sum_{i} F_i \vv{r}_i \right) \times \hat{u} = \vv{r}_F \times \vv{F}   
\end{gather*}
A questo punto, un sistema di forze parallele contribuisce al momento del
sistema come se si considerasse una sola forza risultante posta alla distanza
$\vv{r_F}$ (che è esattamente la media delle distanze) il cui punto nello spazio
di applicazione prende il nome di \textbf{centro delle forze parallele}. 

\subsection{Il caso della forza peso}
Nel caso della forza peso, considerata la forza peso come:
\begin{gather*}
    \vv{g} = -g\hat{k}  
\end{gather*}
posso definire la singola forza peso agente sul singolo punto di massa $m_i$ del sistema
come
\begin{gather*}
    \vv{F_i} = -m_ig\hat{k}  
\end{gather*}
dalle relazioni precedenti possiamo definire
un punto $G$ che prende il nome di \textbf{baricentro}: ossia il punto 
nel quale si può applicare la risultante delle tante forze peso (dato che sono parallele);
per corpi non troppo estesi coincide con il centro di massa:
\begin{align}
    \vv{r_G} = \frac{1}{M}\sum_{i = 1}^{n}m_i \vv{r_i} = \vv{r_C}    
\end{align} 
Questo punto prende anche il nome di \textbf{centro di gravità}.
E' utile ricondursi a sistemi di forze paralleli in situazioni complesse
come nel caso di reazioni vincolari esercitate da piani lisci su di un
corpo esteso oppure nel caso in cui si voglia considerare le forze
di trascinamento in un un SdR non inerziale:
\begin{gather*}
    \vv{F}_{\tau, i} = -m_i \vv{a}_{O'}  
\end{gather*}

\section{Lavoro ed energia nei sistemi. Teoremi di Konig}
Per un sistema di punti è ragionevole pensare
che l'energia cinetica sia la somma dell'energia cinetica
di tutti i punti. Dato quindi un sistema di riferimento $S$
esterno al sistema ed un sistema di riferimento $S'$ interno al sistema
(con terna di assi sempre parallele a quella del sistema fisso  $S$)
posso definire il vettore $\vv{r} = (O' - O)$ e le distanze dei punti 
rispetto al sistema di riferimento $S$ come: 
\begin{gather*}
    \vv{r}_i = \vv{r}_i' + \vv{r}_C 
\end{gather*} 
Dove $C$ è il centro del sistema di riferimento $S'$, $O$ è il centro 
del sistema di riferimento $S$ e $\vv{r_i'}$ il vettore posizione 
di ogni punto rispetto al sistema di riferimento $S'$.  
Data allora la definizione del baricentro
come media pesata delle forze, si può ottenere:
\begin{gather*}
    \vv{r_C} = \frac{\sum_{i = 1}^{n} m_i (\vv{r_i}' + \vv{r_C})}{\sum_{i = 1}^{n}m_i} 
\end{gather*}
e quindi per definizione di centro di massa
\begin{gather*}
    \sum m_i \vv{r_i'} = 0 
\end{gather*}
Derivando allora rispetto al tempo si ha la quantità di
moto che si conserva:
\begin{gather*}
    \sum m_i \vv{v_i'} = \vv{Q'} = 0  
\end{gather*}
Questo dimostra che in un sistema di riferimento con origine nel centro
di massa, la quantità di moto del sistema materiale è sempre nulla. Derivando
rispetto al tempo nuovamente si ha che la sommatoria delle forze è zero,
ossia la prima cardinale. Se il sistema $S'$ è in traslazione rispetto 
al sistema $S$, allora la sua velocità di trascinamento sarà data da
\begin{gather*}
    \vv{v}_i = \vv{v}_i' + \vv{v}_C   \ \Longrightarrow \  \vv{v_\tau} = \vv{v_C} 
\end{gather*}
Le relazioni che intercorrono tra un sistema fisso $S$ ed il sistema di corpi
$S'$ in moto relativo di pura traslazione prendono il nome di \textbf{Teoremi di Konig}.

\clearpage
\subsection{Teorema di Konig per l'energia cinetica}
Possiamo studiare e dimostrare il teorema di Konig per l'energia cinetica,
questo ci dice che: l'energia cinetica di un sistema di punti è esattamente la somma
delle energie cinetiche di tutti i punti del sistema sommato al contributo
dell'energia cinetica del centro di massa:
\begin{align*}
    K = &\sum_{i = 1}^{n}\frac{1}{2}m_i v_i^{2} = \sum_{i = 1}^{n}\frac{1}{2}m_i \vv{v}_i \cdot  \vv{v}_i = \\
    &\sum_{i = 1}^{n}\frac{1}{2}m_i (\vv{v}_i' + \vv{v}_C) \cdot  (\vv{v}_i' + \vv{v}_C )  = \\
    &\sum_{i = 1}^{n} \frac{1}{2}m_i v_i^{2} + \sum_{i = 1}^{n}m_i \vv{v}_i' \vv{v}_C  + \sum_{i = 1}^{n}\frac{1}{2}m_i \vv{v}_C^{2}   
\end{align*}
I singoli termini allora indicano grandezze diverse: il primo termine è proprio
l'energia cinetica rispetto al centro di massa, il secondo termine è sempre zero mentre
la terza grandezza è l'energia cinetica del centro di massa.
Il teorema di Konig per l'energia cinetica ci dice proprio che l'energia cinetica
di un sistema è sempre esprimibile come la somma dell'energia cinetica del centro di massa
come se avesse la massa in un punto e velocità del centro di massa, sommato
ad un termine che è l'energia cinetica rispetto al centro di massa per
ogni punto. \\
Per quanto riguarda il lavoro delle forze interne posso definire, con la generalizzazione
del teorema delle forze vive, che 
\begin{gather*}
    \delta L_i = dK_i
\end{gather*}
Quindi posso sommare successivamente tutte le particelle del sistema come
\begin{gather*}
        \sum_{i = 1}^{n}\delta L_i  = \sum_{i = 1}^{n}dK_i = dK
\end{gather*}
Quindi il lavoro complessivo sia delle forze interne che esterne equivale proprio
alla differenza di energia cinetica tra due configurazioni possibili $A$ e $B$. 
Il lavoro delle forze interne, come si vedrà per il corpo rigido, è sempre uguale
a zero, dunque il teorema di Konig e delle forze vive ci consentono di esprimere il lavoro
delle forze totali come solo il lavoro delle forze esterne. 

\subsection{Teorema di Konig per il momento angolare}
Possiamo definire il momento angolare rispetto al polo $\Omega$ di un sistema di corpi come:
\begin{gather*}
    \vv{L}_{\Omega} = \sum_{i = 1}^{n}(P_i - \Omega)\times m_i \vv{v}_i    
\end{gather*} 
Allora, siccome il momento angolare è uguale anche cambiando polo, posso decidere di utilizzare
un "polo intermedio":
\begin{gather*}
    \sum_{i = 1}^{n}((P_i - C) + (C - \Omega))\times m_i (\vv{v}_i + \vv{v}_C)
\end{gather*}
Spezzando tutto allora:
\begin{gather*}
    \sum_{i = 1}^{n}(P_i - C) \times m_i \vv{v}_i' + \sum_{i = 1}^{n}(C - \Omega) \times m_i \vv{v}_i ' + \sum_{i = 1}^{n} (P_i - C) \times m_i \vv{v}_C + \sum_{i = 1}^{n}(C - \Omega) \times m_i \vv{v}_C    
\end{gather*}
Se scegliessi il polo $\Omega$ come il centro del sistema di riferimento (non cambia nulla
per definizione), allora il secondo ed il terzo termine sono zero poiché uno è esattamente la distanza dal centro
meno la distanza dal centro; l'altro con $P_i - C$ è esattamente (quando sommo tutte le masse) la distanza dal centro di massa dal centro di massa
. Gli altri due pezzi sono combinati ottenendo:
\begin{gather*}
    \vv{L}_{\Omega} = \vv{L}_C' + (C - \Omega) \times M\vv{v}_C   \ \Longrightarrow \ \vv{L_\Omega} = \vv{L_C'} + \vv{r_C} \times \vv{Q}    
\end{gather*} 
Il primo termine è proprio il momento angolare rispetto al centro di massa:
\begin{gather*}
    \vv{L}_C' = \sum_{i = 1}^{n}(P_i - C) \times m_i \vv{v}_i   
\end{gather*}
mentre il secondo termine è il momento angolare del centro di massa 
rispetto ad un dato asse di rotazione che non passa per il centro 
di massa. 
\begin{gather*}
    (C - \Omega) \times M\vv{v}_C 
\end{gather*}
Questa relazione è l'analoga che si è trovata per il momento angolare quando si ha 
cambiamento di polo.
In meccanica quantistica il primo termine è proprio lo spin
di una particella  che è intrinseco della particella. 
Per il corpo rigido sarà molto utile per definire il momento angolare.

\subsection{Il momento angolare del moto circolare uniforme}
\begin{wrapfigure}{r}{0.4\textwidth}
    \centering
    \caption{Mom. angolare cerchi}
    \begin{tikzpicture}
        \draw(0, 0) circle(2) node[anchor = east] {$O$};
        \draw(0, 0) -- (1.41, 1.41) node[midway, above] {$\vv{r}$};
        \draw[->](1.41, 1.41) -- (1, 1.82) node[at end, right] {$\vv{v}$};
        \draw[->](0, 0) -- (2, 0) node[at end, right] {$x$};
        \draw[->](0, 0) --(0, 2) node[at end, above] {$y$};
        \draw[->] (0.4, 0) arc (0:45:0.4) node[midway, right] {$\phi$};
    \end{tikzpicture}    
\end{wrapfigure}
Anche nel moto circolare uniforme è presente una accelerazione:
l'accelerazione centripeta infatti fà si che, per un osservatore inerziale, 
il moto risulti accelerato. Imponendo il suo momento angolare dalla definizione:
\begin{gather*}
    \vv{L}_O  = \vv{r}\times m\vv{v} = (mr^{2})\dot{\phi}\hat{k}   
\end{gather*}
Con le seguenti sostituzioni:
\begin{gather*}
    v = \omega r = \dot{\phi}r \\
    \omega = \dot{\phi}
\end{gather*}
Il teorema delle forze vive mi dice che è costante l'energia cinetica
e la forza centripeta non compie lavoro sulla traiettoria poiché 
cambia solo la direzione della velocità, non compiendo lavoro sul corpo. 
Ne segue che l'energia cinetica è conservata.


\chapter{Il corpo rigido}
\section{Introduzione al concetto di corpo rigido}
Si definisce \textbf{corpo rigido} un sistema di almeno tre punti
assolutamente indeformabile e nel quale le distanze tra ogni coppia
di punti è inalterata rispetto al tempo indipendentemente dalle
forze esterne che gli vengono applicate. Questo è dimostrabile
scegliendo tre punti non allineati all'interno del corpo rigido: 
ciò consente di associare ad ogni sistema rigido una terna cartesiana ortogonale
che lo segue fedelmente scegliendo l'origine in uno dei suoi punti materiali.  
Essendo fissa la posizione di ogni punto rispetto al sistema di riferimento
considerato, la posizione del corpo rigido è conosciuta se e solo se
sono noti le coordinate dell'origine ed i $9$ coseni direttori
di ognuno delle coordinate (che si riducono a tre in quanto devono soddisfare
le condizioni di ortonormalità) e dunque il corpo rigido nello spazio ha 6
gradi di libertà. Se i punti fossero in moto l'uno con l'altro allora
si avrebbero $N \cdot 3$ gradi di libertà se il corpo non ruotasse. \\
Possiamo mettere in relazione la velocità di un punto materiale $P$
rispetto a due sistemi di riferimento con la seguente
\begin{gather*}
    \vv{v_P} = \vv{v_C} + \vv{\omega} \times (\vv{r} - \vv{r_C}  ) \\
\end{gather*}
Dove $\vv{r}$ indica la distanza di un punto qualsiasi $P$ dall'origine del
sistema di riferimento $S$ e $\vv{r_C}$ indica la distanza dall'origine del
sistema di riferimento $S'$ centrato nel centro di massa del corpo rigido.  
Se il corpo traslasse senza ruotare, si avrebbe che
\begin{align*}
    \vv{v_P} = \vv{v_C}   
\end{align*}

\section{Statica e dinamica del corpo rigido}
\subsection{Moto rotatorio di un corpo rigido}
\begin{wrapfigure}{r}{0.4\textwidth}
    \centering
    \caption{La rotazione di un punto del corpo rigido rispetto
    ad un asse passante per il centro di massa}
    \begin{tikzpicture}[scale = 1.2]
        \draw[dashed](0, -1) -- (0, 2);
        \filldraw (0, 0) circle (1pt) node[anchor = west] {$O'$};
        \draw(0, 1) ellipse (1 and 0.3);
        \draw[->](0, 0) -- (-1, 1) node[at end, left] {$\vv{r} $};
        \draw[->](0, 0) -- (0, 0.5) node[at end, right] {$\vv{\omega}$};
        \draw(0, 0.4) arc (90:135:0.4) node[midway, above] {$\phi$};
        \filldraw(-1, 1) circle (1pt) node[anchor = south] {$P$};
        \draw[dashed, thin] (-1, 1) -- (0, 1) node[midway, above] {$d$};
    \end{tikzpicture}    
\end{wrapfigure}
Se prendessimo un sistema di riferimento $S$ con origine $O$ e una terna cartesiana
di versori $\hat{i}, \hat{j}, \hat{k}$ corrispondenti agli assi $x, y, z$ ed un
sistema $S'$ con le medesime caratteristiche, e solidale con il corpo rigido, il quale è in rotazione
rispetto ad un asse che coincide con l'asse di rotazione $\vv{\omega}$. Dato che rispetto
al centro $O'$ il punto $P$ non trasla , allora la velocità del punto è
\begin{gather*}
    \vv{v_P} = \vv{\omega} \times \vv{O'P}   
\end{gather*}
Da questa si evince che tutti i punti il cui prodotto vettoriale è
pari a zero hanno velocità nulla; chiamo questi punti che hanno velocità nulla
in un corpo rigido che sta ruotando come \textbf{asse di rotazione}, ossia
l'asse intorno al quale sta ruotando l'oggetto stesso. La velocità tra 
due punti $P_1$ e $P_2$ del corpo rigido sarà la sottrazione vettoriale
delle velocità:
\begin{gather*}
    \vv{v_1} - \vv{v_2} = \vv{\omega} \times \vv{P_2P_1}    
\end{gather*}
I punti fuori dall'asse di rotazione si muovono di moto rotatorio
semplice su di una circonferenza con distanza $d = |\vv{r} | \cdot  \sin\phi$. 
Possiamo quindi stimare lo spostamento infinitesimo del punto $P$ come
\begin{gather*}
    ds = r d\phi \ \Longrightarrow \ \frac{ds}{dt} = r \frac{d\phi}{dt} \ \Longrightarrow \ \dot{\phi} = \dot{r}\phi
\end{gather*}
Quindi il modulo della velocità sarà dato dalla seguente
\begin{gather*}
    |\vv{v_P}| = \omega r \sin\phi = \omega d 
\end{gather*}
Adesso se l'asse non è fisso, possiamo sempre utilizzare le relazioni precedenti
e semplicemente la velocità varierà in base alla variazione dell'asse
di rotazione in funzione del tempo. 


\subsection{Moto di rotolamento puro}
\begin{wrapfigure}{r}{0.4\textwidth}
    \centering
    \caption{Il moto di rotolamento puro}    
    \begin{tikzpicture}[scale = 0.8]
        \draw[-] (0, 0) circle (2.5);
        \draw[->, thick] (-3.5,-2.5) -- (3.5, -2.5) node[at end, below] {$x$};
        \filldraw (0, 0) circle (1pt) node[above] {$C$};
        \filldraw (0, -2.5) circle (1pt) node[above] {$Q$};
        \filldraw(0, -2.5) circle (1pt) node[below] {$M$};
        \filldraw(2.5, 0) circle (1pt) node[right] {$P_1$};
        \filldraw (0, 2.5) circle (1pt) node[above] {$P_2$};
        \filldraw (-2.5, 0) circle (1pt) node[left] {$P_3$};
        \draw[->, thick] (0, 2.5) -- (2, 2.5) node[midway, above] {$\vv{\omega} + \dot{x}$};
        \draw[->, thick] (0, 0) -- (1, 0) node[midway, above] {$ \dot{x}$};
        \draw[->, thick] (0, -2.5) -- (1, -2.5) node[midway, above] {$\dot{x}$};
        \draw[->, thick] (0, -2.5) -- (-1, -2.5) node[midway, above] {$\vv{\omega}$};
        \draw[->, thick] (-2.5, 0) -- (-2.5, 1) node[midway, left] {$\vv{\omega}$};
        \draw[->, thick] (-2.5, 0) -- (-1.5, 0) node[midway, below] {$\dot{x}$};
        \draw[->, thick] (-2.5, 0) -- (-1.5, 1) node[at end, above] {$\dot{x} + \vv{\omega}$};
    \end{tikzpicture}   
\end{wrapfigure}
E' il moto di rotazione di un corpo su di un altro i quali hanno istante
per istante la stessa velocità nel punto di contatto come nel caso di una ruota su di una strada. Un corpo rigido che compie
rototraslazione su di un'altro corpo rigido rimanendoci in contatto prende il nome di moto di \textbf{rotolamento generico} o \textbf{rotolamento
puro} senza  strisciamento.
Quando si ha solo moto di rotazione senza velocità traslazionale
allora la ruota rimane a contatto col pavimento e si ha rotolamento
puro. 
\begin{gather*}
    \vv{v_Q} = \vv{v_M} = 0 \\
    Q \ \text{costante alla ruota} \\
    M \ \text{costante alla strada}     
\end{gather*}
Il moto di rotazione per un punto generico in un corpo rigido è dato dalla:
\begin{gather*}
    \vv{v_P} = \vv{v_C} + \vv{\omega} \times \vv{CP}    
\end{gather*}
Dove $P$ è un punto generico nella ruota. Possiamo esprimere 
la velocità del punto in relazione alla velocità di rotazione della ruota come: 
\begin{gather*}
    \vv{\omega} = \omega \hat{j} = \dot{\phi}\hat{j} \ \Longrightarrow \ 
    \vv{v_P} = \dot{x_C} \hat{i} +    \dot{\phi}\hat{j} \times (P-C) 
\end{gather*}
Dato che il corpo è libero di muoversi lungo i tre assi ma può solamente 
ruotare lungo l'asse $x$, allora posso dire che il corpo ha 4 gradi di libertà anche se,
in realtà, essendo $C$ vincolato, il corpo ruota e trasla insieme e dunque
i veri gradi di libertà si riducono a solo uno: la ruota può solo rototraslare 
nella direzione positiva dell'asse $x$. Questo si può dimostrare con le seguenti 
\begin{align*}
    \vv{v}_Q = \dot{x}_c\hat{i}  + \dot{\phi}\hat{j} \times (Q-C) \ &\Longrightarrow \  \vv{v}_Q =  \dot{x}_c \hat{i} + \dot{\phi} \hat{j} \times (-R\hat{k} ) 
\end{align*}
Dato che $Q - C$ è il raggio $R$ e che, la velocità nel punto $Q$ deve essere nulla,
si ha
\begin{gather*}
    \vv{v}_Q = \dot{x}_c \hat{i} + (-R\dot{\phi} \hat{i} ) = 0 \ \Longrightarrow \     \dot{x}_c = R\dot{\phi}       
\end{gather*}
Quindi esiste solo un grado di libertà.
La parte a contatto della ruota col terreno è immobile ed ha un asse
che cambia nel tempo ma rimane sempre parallelo a sé stesso ed ad omega.
\clearpage

\subsection{Il moto dei punti interni alla circonferenza: la cicloide}
\begin{wrapfigure}{r}{0.4\textwidth}
    \centering
    \caption{Individuazione di un punto sulla circonferenza del moto rotatorio}
    \begin{tikzpicture}
        \draw(0, 0) circle (2);
        \draw(0, -2) -- (1, 1);
        \filldraw(0, 0) circle (1pt);
        \filldraw(0, -2) circle (1pt) node[anchor = north] {$C$};
        \filldraw(1, 1) circle (1pt) node[anchor = south] {$P$};
        \draw(0, 0) -- (1, 1) node[midway, below] {$\mu R$};
        \draw[dashed](0, 0) -- (0, -2) node[midway, left] {$R$};
        \draw[->](0, 0.5) arc (90:45:0.5) node[midway, above] {$\phi$}; 
        \draw[dashed] (0, 0) -- (0, 2);
    \end{tikzpicture}    
\end{wrapfigure}
Il moto di tutti i punti interni alla ruota eseguono un moto che prende il nome di \textbf{cicloide allungata},
la quale, ha la seguente espressione (si può poi integrare per $t$ tenendo conto delle condizioni iniziali
$t = 0 \ \Longrightarrow \ x(0) = 0$ e $t = 0 \ \Longrightarrow \ z(0) = R\mu + R$ ):
\begin{gather*}
    \left\{ \begin{array}{l}
        \dot{x} = \dot{\phi}(R + \mu R \cos\phi) \\
        \dot{y} = -\dot{\phi}\mu R \sin\phi 
    \end{array} \right. \ \Longrightarrow \ 
    \left\{ \begin{array}{l}
        x_P = R \phi + \mu R \sin \phi \\
        y_P = R + \mu R \cos \phi 
    \end{array} \right. 
\end{gather*}
Mentre il punto sulla sommità della ruota esegue un moto che prende il nome di  \textbf{cicloide ordinaria} che
ha la stessa espressione ma con $\mu = 1$:
\begin{gather*}
    \left\{ \begin{array}{l}
        x_p = R \phi + R \sin \phi \\
        z_p = R + R \cos \phi
    \end{array} \right. 
\end{gather*}
La formula generale di una cicloide è data dalla seguente:
\begin{align}
    (x - \phi R)^{2} + (z - R)^{2} = R  
\end{align}


\subsection{Esempio di statica del corpo rigido}
\begin{wrapfigure}{r}{0.4\textwidth}
    \centering
    \caption{parallelepipedo}
    \begin{tikzpicture}
        \draw(0, 0) -- (3, 0);
        \draw(0.5, 0) -- (2, 2);
        \draw(2, 2) -- (3, 2);
        \draw(3, 2) -- (1.5, 0);
        \draw[->](0, 0) -- (0.5, 0) node[at end, below] {$x$};
        \draw[->](0, 0) -- (0, 0.5) node[at end, left] {$y$};
        \draw[->](1, 0) -- (1, 0.5) node[at end, above] {$\vv{N}$};
        \draw[->](1.75, 1) -- (1.75, 0.5) node[at end, right] {$M\vv{g}$};
        \filldraw(0, 0) circle (1pt) node[anchor = north] {$O$};
    \end{tikzpicture}    
\end{wrapfigure}
Il corpo rigido rimane in equilibrio quando non cambia
la quantità di moto del sistema, dalle equazioni cardinali della dinamica si ha 
necessariamente che 
\begin{gather*}
    \vv{F}^{(EXT)} = 0 \\
    \vv{M}^{(EXT)} = 0    
\end{gather*}
Immaginando di avere un piano di appoggio e un parallelepipedo storto:
questo oggetto è in equilibrio oppure no? Devo vedere con le cardinali
se esiste un momento torcente per il quale il corpo cade.
Se le dimensioni del corpo sono piccole rispetto alle dimensioni della Terra
allora posso dire, con buona approssimazione, che l'accelerazione locale
di gravità è costante e parallela per tutti i corpi. Posso allora
considerare la forza peso come concentrata nel centro di massa (baricentro)
e applicare la prima cardinale:
\begin{gather*}
    M\vv{g} + \vv{N} = 0 \ \Longrightarrow \ N = Mg  
\end{gather*}
Applicare ora la seconda cardinale posso prendere un polo opportuno, e chiamate
$x$ la distanza di applicazione della normale e $x'$ la distanza dell'applicazione
della forza peso rispetto al SdR $S$ con centro $O$:
\begin{gather*}
    M_{z}^{(EXT)} = Nx - Mgx' 
\end{gather*}
Per far si che $Nx = Mgx'$ deve necessariamente risultare che $x = x'$.
La risultante deve essere sempre applicata sulla base di appoggio: ne segue che
il centro di massa deve trovarsi lungo la verticale passante per la base di appoggio
poiché, altrimenti, non c'è stabilità ed il corpo si ribalta.

\subsection{La carrucola fissa}
\begin{wrapfigure}{r}{0.4\textwidth}
    \centering
    \caption{Carrucola fissa}
    \begin{tikzpicture}
        \draw(0, 0) circle (1);
        \draw(-1, -1.5) -- (-1, 0);
        \draw(1, -1.5) -- (1, 0);
        \draw[->, thick, red](1, -1.5) -- (1.5, -2.1) node[at end, right] {$\vv{F}_2$};
        \draw[dashed, thin](1, -1.5) -- (1, -2.5);
        \draw(1, -2) arc (-90:-50:0.5) node[midway, below] {$\beta$};
        \draw[->, thick, red](-1, -1.5) -- (-1.3, -2.4) node[at end, left] {$\vv{F}_1$};
        \draw[dashed, thin](-1, -1.5) -- (-1, -2.5);
        \draw(-1, -2) arc (270:250:0.5) node[midway, below] {$\alpha$};
        \draw[->](0, 0) -- (0, -0.5) node[at end, right] {$M\vv{g}$};
        \draw[->](0, 0) -- (0, 0.5) node[at end, right] {$\vv{N}$};
        \filldraw(1, -1.5) circle (1pt);
        \filldraw(-1, -1.5) circle (1pt);
        \filldraw(0, 0) circle (1pt) node[anchor = east] {$C$};
    \end{tikzpicture}    
\end{wrapfigure}
Nel caso ideale la carrucola segue la fune, ma nel caso 
reale il movimento della fune non sempre è seguito da quello
della carrucola a meno che l'attrito non sia molto debole.
Perché siano in equilibrio (trascurando la massa della fune)
la risultante delle forze si dovrebbe eguagliare:
\begin{gather*}
    \vv{N} + M\vv{g} + \vv{F}_1 + \vv{F}_2 = 0   
\end{gather*}
Scomponendo lungo i due assi:
\begin{gather*}
    Nx - F_1 \sin\alpha + F_2 \sin\beta = 0 \\
    N_y + Mg - F_1 \cos\alpha - F_2 \cos\beta = 0
\end{gather*}
Dove $\alpha$ e $\beta$ sono gli angoli rispetto alla verticale
delle due forze. Se scegliessi il polo di applicazione
del momento nel centro della carrucola, avrei che i momenti sono rappresentati
dalle forze e, poiché la fune non ha massa e la carrucola è fissa, non ho bisogno
di conoscere i moduli delle forze né il loro braccio. Questo deriva dal fatto
che anche se le forze avessero angoli diversi, nel momento in cui io tiro la fune,
essa cambierebbe il suo avvolgimento sulla carrucola. Allora per far si
che si abbia un momento risultante nullo (e che quindi il sistema sia in equilibrio),
avrò solamente una condizione:
\begin{gather*}
    F_1 = F_2 
\end{gather*}

\subsection{Il problema della scala}
\begin{wrapfigure}{r}{0.4\textwidth}
    \centering
    \caption{Il problema della scala}
    \begin{tikzpicture}[scale = 1.2]
        \draw[->](0, 0) -- (2, 0) node[at end, below] {$x$};
        \draw[->](0, 0) -- (0, 2.5) node[at end, left] {$y$};
        \draw(0, 2) -- (1, 0) -- (1.3, 0.15) -- (0.3, 2.15) -- (0, 2);
        \draw[->, thick](0, 2) -- (0.75, 2) node[at end, above] {$\vv{N}_1$};
        \draw[->, thick](1, 0) -- (1, 0.75) node[at end, right] {$\vv{N}_2$};
        \draw[->, thick](1, 0) -- (0.25, 0) node[at end, below] {$\vv{F}_a$};
        \draw[->, thick](0.5, 1) -- (0.5, 0.25) node[midway, left] {$M\vv{g}$};
        \draw(0.8, 0.4) arc (120:180:0.42) node[at end, below] {$\alpha$};
        \filldraw(1, 0) circle (1pt) node[anchor = north] {$B$};
        \filldraw(0, 2) circle (1pt) node[anchor = east] {$A$};
        \filldraw(0.5, 1) circle (1pt) node[anchor = east] {$C$};
    \end{tikzpicture}    
\end{wrapfigure}
Il problema della scala è un problema della statica dei corpi rigidi
con la forza di attrito che è data dalla seguente formulazione:
\begin{gather*}
    |\vv{F}_a| \leq \mu_s|\vv{N}_2|  
\end{gather*}
Per poter essere in equilibrio deve risultare che
\begin{gather*}
    \vv{N}_1 + \vv{N}_2 + M\vv{g} + \vv{F}_a  = 0 \\
    N_1 - F_a = 0  \ \Longrightarrow \  N_1 = F_a \\
    N_2 - Mg = 0 \ \Longrightarrow \ N_2 = Mg
\end{gather*}
Con la prima cardinale non sono in grado di determinare se la scala
rimane in equilibrio oppure no, devo scegliere allora un polo di riduzione
per calcolare il momento e quindi dirmi a che angolo $\alpha$ è in equilibrio.
Scegliendo il polo nel punto in cui la scala si appoggia per terra, e 
date le distanze come:
\begin{align*}
    (C - B) &= \frac{L}{2}(-\sin\alpha \hat{i} + \cos\alpha \hat{j}) \\
    (A - B) &= 2(C - B)
\end{align*}
Posso esprimere ora il momento delle forze esterne nel polo $B$
con la seguente:
\begin{align*}
    \vv{M}_B^{EXT} &= (C - B)\times M\vv{g} + (A - B)\times \vv{N}_1  \\
    &=  \frac{L}{2}(-\sin\alpha  \hat{i} + \cos\alpha \hat{j} ) \times (-Mg\hat{j} )  + L (-\sin\alpha \hat{i} + \cos\alpha \hat{j}) \times N_1 \hat{i} \\
    &= + \frac{L}{2}\sin\alpha Mg\hat{k} - L \cos\alpha N_1 \hat{k} = 0   
\end{align*}
Da questa si ottiene l'espressione della reazione vincolare 
alla parete come:
\begin{gather*}
    N_1 = \frac{1}{2} \tan\alpha Mg
\end{gather*}
Ricordando l'espressione della forza di attrito si ha l'angolo 
di equilibrio in funzione del coefficiente di attrito:
\begin{gather*}
    N_1 = F_a \leq \mu_s N_2 \ \Longrightarrow \  \tan\alpha \leq 2\mu_s
\end{gather*}

\subsection{Momenti angolari per casi particolari}
Ricordando l'espressione per la velocità di un certo punto $P$ nel corpo rigido
con sistema di riferimento $S'$ rispetto ad un sistema di riferimento $S$ fisso:  
\begin{gather*}
    \vv{v}_P = \vv{v}_{O'} + \omega \times(O - O')  
\end{gather*}
Nel caso di moti puramente traslatori, il momento angolare è dato da
\begin{gather*}
    \vv{L}_{\Omega} = \sum_{i = 1}^{n}(P_i - \Omega) \times m_i \vv{v}_i = \sum_{i = 1}^{n}(P_i - \Omega) \times m_i \vv{v}_{O'} = (C - \Omega)\times M \vv{v}_{O'}      
\end{gather*}
Nel caso del moto rototraslatorio il momento angolare è dato da:
\begin{gather*}
    \vv{L}_{\Omega} = \sum_{i = 1}^{n}(P_i - \Omega)\times m_i\vv{v}_i = \sum_{i = 1}^{n}m_i(P_i - \Omega)\times \vv{v}_{O'} + \sum_{i = 1}^{n}m_i(P_i - \Omega) \times (\vv{\omega} \times (P_i - O') )     
\end{gather*}
Nel caso in cui il momento angolare sia applicato sull'origine del 
sistema di riferimento e che questo coincida esattamente con il centro
del corpo rigido
\begin{gather*}
    \vv{L}_{O'} = \sum_{i = 1}^{n}m_i(P_i - O')\times (\vv{\omega} \times (P_i - O'))  
\end{gather*}
Nel caso in cui $\vv{\omega}$ sia costante, allora essendo il suo versore
costante, il momento si semplifica ulteriormente.

\section{Il momento assiale e il momento di inerzia}
Dall'espressione del momento angolare in funzione del polo $O'$ e della
velocità angolare $\vv{\omega}$ posso ottenere una quantità scalare che prende
il nome di \textbf{momento assiale}, ossia il momento di inerzia riferito ad un certo
asse (ossia la direzione sulla quale giace il versore $\hat{u}_\omega$ ):
\begin{gather*}
    L_{\omega, O'}  = \vv{L}_{O'} \cdot  \hat{u}_{\omega} = \sum_{i = 1}^{n}m_i \cdot \hat{u}_{\omega} (P_i - O') \times (\hat{u}_{\omega} \times (P_i - O')) \omega    
\end{gather*}
Posso ottenere, con le regole del prodotto misto, una sua espressione diversa:
\begin{gather*}
    L_{\omega, O'}  =  \left(\sum_{i = 1}^{n} m_i \left| \hat{u}_{\omega} \times (P_i - O')  \right|^{2}  \right) \omega
\end{gather*}
Se chiamassi ora la distanza dall'asse di rotazione di ogni punto $P_i$ del
corpo rigido come 
\begin{gather*}
    |\hat{u}_{\omega} \times (P_i - O')| = d_i 
\end{gather*}
Si otterrebbe un'altra analoga per il momento assiale:
\begin{gather*}
    L_{\omega, O'} = \omega \sum m_id_i^{2}   
\end{gather*}
La sommatoria è una grandezza scalare che prende il nome di \textbf{momento di
inerzia} che dipende dall'asse di rotazione 
scelto e ha (da qui il nome) lo stesso ruolo della \textbf{massa inerziale} e si esprime come:
\begin{align}
    I = \sum m_id_i^{2}  = \int d^{2} dm 
\end{align}
Allora posso esprimere il momento assiale con la semplice formulazione:
\begin{gather*}
    L_{\vv{\omega} } = I \omega
\end{gather*}
Posso ottenere dunque dalla prima cardinale il \textbf{momento assiale delle forze}, ossia una 
quantità scalare che si ottiene proiettando sull'asse di rotazione il  momento angolare
delle forze esterne rispetto ad un polo qualunque scelto sull'asse di rotazione:
\begin{align}
    \vv{M}_{\Omega} \cdot  \hat{u}_{\omega} = I_{\hat{\omega} } \dot{\omega}  \ \Longrightarrow \ M_\Omega = I\frac{d^{2}\theta }{dt^{2} }
\end{align}
\clearpage

\section{Il teorema di Huygens-Steiner e calcolo di vari momenti di inerzia}
\begin{wrapfigure}{r}{0.4\textwidth}
    \centering
    \caption{Momento di inerzia rispetto ad assi paralleli}
    \begin{tikzpicture}
        \draw(0, 0) circle (2);
        \filldraw(-1, 0) circle (1pt) node[anchor = east] {$C$};
        \filldraw(0.5, 0) circle (1pt) node[anchor = west] {$A$};
        \draw[->](-1, -2.5) -- (-1, 2.5) node[at end, left] {$z$};
        \draw[->](-1, 0) -- (-0.5, 0.5) node[at end, above] {$y$};
        \draw[->](-1, 0) -- (-0.5, -0.5) node[at end, below] {$x$};
        \filldraw(1, 1) circle (1pt) node[anchor = west] {$P_i$};
        \draw(0.5, -2.5) -- (0.5, 2.5);
        \draw[|-|](-0.9, 2.25) -- (0.4, 2.25) node[midway, above] {$D$}; 
        \draw[dashed, thin](1, 1) -- (-1,1) node[midway, above] {$d$};
    \end{tikzpicture}    
\end{wrapfigure}
Il teorema di Huygens-Steiner stabilisce che il momento di inerzia rispetto ad
un asse qualsiasi può essere definito come la somma di due termini:
\begin{enumerate}
    \item Il momento di inerzia calcolato rispetto all'asse passante per il centro di massa del sistema;
    \item Il prodotto della massa totale del sistema con il quadrato della distanza tra i due assi.
\end{enumerate}
Si ottiene dunque la seguente relazione:
\begin{align}
    I = I_C + md^{2} 
\end{align}
Il teorema può essere dimostrato secondo i seguenti procedimenti.
Considerato un corpo rigido con un dato centro di massa e scelto un asse lungo
tale centro di massa, posso allora esprimere il momento di inerzia di qualsiasi
punto $P_i$ a distanza $d_i$ dall'asse $z$ che passa per il centro di massa come
\begin{gather*}
    I_C = \sum_{i = 1}^{n}m_id_i^{2}  
\end{gather*}
Con la seguente notazione posso esprimere un asse che passa lungo il centro
di massa  come $ \hat{C}$. Scelto ora il polo $A$ che si trova a distanza $D$ dall'asse
passante per C voglio allora calcolarmi il momento di inerzia e ottengo che per
un punto qualsiasi rispetto all'asse è proprio:
\begin{gather*}
    I_{\hat{C} } = \sum_{i = 1}^{n}m_i (x_i^{2} + y_i^{2} ) 
\end{gather*}
Proprio perché il punto $P_i$ giace su un piano perpendicolare all'asse $\hat{C}$; posso
dire che il momento di inerzia rispetto al polo $A$ sull'asse $\hat{A}$ parallelo
all'asse $\hat{C}$ è dato da:   
\begin{gather*}
    I_{\hat{A} } = \sum m_i((x_i - D)^{2} + y_i^{2} ) = \sum m_i(x_i^{2} + y_i^{2} ) + \sum m_iD^{2} - 2\sum m_ix_iD
\end{gather*}
Il primo termine è proprio il momento di inerzia rispetto al centro di massa, mentre il secondo è
il momento di inerzia totale della massa ed il terzo è il momento di inerzia uguale a zero.
Allora il momento di inerzia tra due assi paralleli è proprio il momento di inerzia del primo
asse più la distanza per la massa del corpo:
\begin{gather*}
    I_{\hat{A} } = I_{\hat{C} } + MD^{2} 
\end{gather*}
Con questo teorema si è anche dimostrato che il momento di inerzia minimo si ha
rispetto ad un asse che passa per il centro di massa.

\subsection{Il momento di inerzia di una sbarra omogenea sottile}
\begin{wrapfigure}{r}{0.4\textwidth}
    \centering
    \caption{Sbarra omogenea}
    \begin{tikzpicture}
        \draw(0, 0) rectangle (3, 1);
        \draw[|-|](0, -0.5) -- (3, -0.5) node[midway, below] {$L$};
        \draw[->](1.5, 0.5) -- (4, 0.5) node[at end, below] {$x$};
        \draw[->](1.5, 0.5) -- (1.5, 2) node[at end, left] {$y$};
        \filldraw(1.9, 1) circle (0pt) node[anchor = south] {$dx$};
        \draw(1.85, 1) -- (1.85, 0);
        \draw(1.95, 1) -- (1.95, 0);
    \end{tikzpicture}    
\end{wrapfigure}
Per calcolare il momento di inerzia di una sbarretta posso suddividerla in
tanti piccoli rettangoli in modo tale da ottenere l'integrale
del momento di inerzia come:
\begin{gather*}
    I = \int dm d^{2} 
\end{gather*}
La massa infinitesima è data dalla densità lineare delle singole
fette per la loro lunghezza infinitesima:
\begin{gather*}
    dm = dx \lambda
\end{gather*}
Sostituendo nell'integrale, posso integrare rispetto a tutta la lunghezza
della sbarretta:
\begin{gather*}
    I = \int_{-\frac{L}{2}}^{\frac{L}{2}}  dx \lambda x^{2} 
\end{gather*}
Sapendo che la massa è $M = \lambda L$ e che il centro di massa
è localizzato nell'origine del sistema di riferimento $S$, posso ottenere
il momento di inerzia rispetto al centro di massa come 
\begin{align}
    I_{\hat{C} } = \frac{1}{12}ML^{2}
\end{align}
Rispetto ad un asse passante per il bordo a distanza $\pm \frac{L}{2}$ il momento di inerzia diventa, secondo
Huygens-Steiner è dato da: 
\begin{align}
    I_{\hat{A} } = \lambda \int_{0}^{L} x^{2}dx = \frac{1}{3}ML^{2}  
\end{align}

\subsection{Il rettangolo omogeneo}
\begin{wrapfigure}{r}{0.4\textwidth}
    \centering
    \caption{}
    \begin{tikzpicture}
        \draw(0, 0) rectangle (3, 2);
        \draw[|-|](-0.5, 0) -- (-0.5, 2) node[midway, left] {$b$};
        \draw[|-|](0, -0.5) -- (3, -0.5) node[midway, below] {$a$};
        \draw[dashed, ->] (0, 1) -- (3.5, 1) node[at end, below] {$x$};
        \filldraw(2, 0) rectangle (2.1, 2);
        \draw[dashed, ->](1.5, -0.5) --(1.5, 2.5) node[at end, right] {$y$};
        \filldraw (1.5, 1) circle (1pt) node[anchor = east] {$C$};
    \end{tikzpicture}    
\end{wrapfigure}
Posso ricavare il momento di inerzia di un rettangolo omogeneo: questo momento
di inerzia è analogo al caso della sbarretta infinitesima con lunghezza $a$ e
massa totale $M$ per quanto riguarda l'asse $x$.  Adesso devo anche
calcolare il momento di inerzia rispetto all'asse $Y$: allora posso dire che 
considero piccole sbarrette (analogamente al caso della sbarretta) di  lunghezza
$a$ e con massa nuovamente $M$. Dato che il momento di inerzia è additivo, 
e che entrambi i momenti di inerzia sono calcolati rispetto a
degli assi che passano per il centro, posso esprimere il momento di
inerzia totale del rettangolo omogeneo come: 
\begin{align}
    I_{\hat{z} } = \frac{1}{12}M(a^{2} +b^{2} )
\end{align}
Ossia esattamente la somma dei momenti di inerzia che ha come asse l'asse
perpendicolare al rettangolo stesso.

\subsection{Il parallelepipedo omogeneo}
Dato che il parallelepipedo è omogeneo, posso calcolare il momento di inerzia
come se fosse tutto schiacciato e dunque ricondurmi nel caso del rettangolo omogeneo:
il suo momento di inerzia rispetto all'asse $x$ è dunque:
\begin{align}
    I_{\hat{z} } = \frac{1}{12}M(a^{2} + b^{2} )
\end{align}
Posso ottenere il momento di inerzia anche rispetto agli altri due assi $x$ e $y$
utilizzando come lati i lati paralleli all'asse considerato.
\hfill

\subsection{Il cerchio omogeneo}
\begin{wrapfigure}{r}{0.4\textwidth}
    \centering
    \caption{Il momento del cerchio}
    \begin{tikzpicture}
        \draw(0, 0) circle (2);
        \draw(-3, 0) -- (3, 0) node[at end, below] {$x$};
        \draw(0, -3) -- (0, 3) node[at end, right] {$y$}; 
        \draw[|-|](-2, -2.1) -- (-0.05, -2.1) node[midway, below] {$R$};
        \draw(0, 0) circle (1.5);
        \draw(0, 0) circle (1.4);
        \filldraw (0.7, 0) circle (0pt) node[anchor = north] {$r$};
    \end{tikzpicture}    
\end{wrapfigure}
Per trovare il momento di inerzia di un cerchio posso ricondurmi allo studio
del momento di inerzia per una corona corona spessa $dr$.
Posso ottenere la superficie della corona come $dS = 2\pi r dr. $, allora
la massa infinitesima della corona è data da
\begin{gather*}
    dm  = \sigma dS \ \Longrightarrow \ dm = \sigma 2\pi r dr
\end{gather*}
Dato che la corona ha un certo spessore $dr$, allora  posso calcolare il momento 
di inerzia per una corona e fare l'integrale unidimensionale per considerare
tutte le corone e ricavare il momento totale rispetto all'asse $z$, dato che 
la distanza rispetto all'asse $z$ di quella corona è $r$, allora 
\begin{gather*}
    I_z = \int_{0}^{R}\sigma 2 \pi r^{3}dr = \sigma 2\pi \frac{R^{4} }{4}    
\end{gather*}
E dato che la massa totale è proprio $M = \pi R^{2}\sigma$, allora il momento di inerzia
del cerchio omogeneo sarà:
\begin{align}
    I_z = \frac{1}{2}MR^{2} 
\end{align} 
Allora dato che il momento di inerzia è la somma dei momenti di inerzia
rispetto agli altri assi, posso dire che il momento rispetto all'asse $z$
è esattamente la somma dei momenti di inerzia rispetto agli assi $x$ ed $y$.
Data la simmetria del cerchio allora i momenti rispetto ai due assi devono
necessariamente essere uguali:
\begin{gather*}
    I_x = I_y = \frac{1}{4}MR^{2}
\end{gather*}

\subsection{Il momento rispetto al cilindro}
\begin{wrapfigure}{r}{0.4\textwidth}
    \centering
    \caption{Il cilindro}
    \begin{tikzpicture}
        \draw[->, dashed](-0.5, -1) -- (3.5, 1) node[at end, above] {$y$};
        \draw(0, 1) -- (3, 1);
        \draw(0, -1) -- (3, -1);
        \draw(0, 1) arc(135:225:1.42);
        \draw(3, 0) ellipse (0.5 and 1);
        \draw[dashed](0, 1) -- (0, 0) node[midway, right] {$R$};
        \draw[->, dashed](-1, 0) -- (4, 0) node[at end, below] {$x$};
        \draw[->, dashed](1.5, -1.5) -- (1.5, 1.5) node[at end, right] {$z$};
    \end{tikzpicture}    
\end{wrapfigure}
Per determinare il momento di inerzia di un cilindro posso considerare il momento
del cilindro come l'integrale da $-\frac{L}{2}$ a $\frac{L}{2}$ del momento di inerzia
di un cerchio di raggio $R$. Rispetto allora all'asse
$x$ il momento di inerzia:
\begin{gather*}
    I_x = \frac{1}{2}MR^{2} 
\end{gather*}
Rispetto all'asse $z$ invece la cosa si complica: il momento di inerzia rispetto
a quell'asse assi è dato da quello del cerchio più il termine di Huygens-Steiner:
\begin{gather*}
    dI = \frac{1}{4}dmR^{2}+ dmx^{2}  
\end{gather*}
Con $x$ la distanza rispetto all'origine del cilindro infinitesimo considerato; dato
che la massa infinitesima del cilindretto è data da:  
\begin{gather*}
    dm = \rho dV = \rho \pi R^{2} dx  
\end{gather*}
E allora il momento di inerzia è esprimibile come integrale unidimensionale
\begin{gather*}
    I_z = \int_{-\frac{h}{2}}^{\frac{h}{2}}\left(\rho \pi R^{2} dx\left(\frac{R^{2} }{4} + x^{2} \right)\right) 
\end{gather*}
Dato che il cilindro è simmetrico rispetto all'al piano $zy$ 
e rispetto ai piani $xy$ e $xz$ si ha che 
\begin{align}
    I_z = I_y = M\left(\frac{R^{2} }{4} + \frac{h^{2} }{12}\right)
\end{align}
\hfill
 
\subsection{La sfera omogenea}
\begin{wrapfigure}{r}{0.4\textwidth}
    \centering
    \caption{}
    \begin{tikzpicture}
        \draw(0, 0) circle (2);
        \draw[->](-3, 0) -- (3, 0) node[at end, below] {$x$};
        \draw[->](0, -3) -- (0, 3) node[at end, left] {$y$};
        \filldraw(0, 0) circle (1pt) node[anchor = south east] {$O$};
        \node[ellipse,
        draw,
	    minimum width = 0.1cm, 
	    minimum height = 2.6cm] (e) at (1.5,0) {};
        \draw(0, 0) -- (1.5, 1.25) node[midway, above] {$R$};
        \filldraw(1, 0) circle(0pt) node[anchor = north] {$x$};
        \draw[dashed] (1.5, 0) -- (1.5, 1.25) node[midway, left] {$r$};
    \end{tikzpicture}    
\end{wrapfigure}
Per calcolare il momento di inerzia di una sfera (omogenea), posso utilizzare
$r$ come il raggio dell'anello infinitesimo di spessore $dr$  e distanza
$x$ dall'origine. Posso ottenere il raggio $r$ attraverso il teorema di
Pitagora:
\begin{gather*}
    r^{2} = R^{2} - x^{2}
\end{gather*}
Dato che la sfera è omogenea posso esprimere la 
massa infinitesima dell'anello come
\begin{gather*}
    dm = \rho d V    
\end{gather*}
Posso esprimere la superficie infinitesima e dunque il volume infinitesimo
come:
\begin{gather*}
    dS = \pi r^{2} = \pi(R^{2} - x^{2}  )  \ \Longrightarrow \ dV = dx dS = \pi (R^{2} - x^{2}  ) dx 
\end{gather*}
Allora la massa infinitesima è
\begin{gather*}
    dm = \rho  \pi (R^{2} - x^{2}  ) dx
\end{gather*}
Il momento di inerzia rispetto ad un dato asse è allora dato dall'integrale:
\begin{gather*}
    I_x = \int \frac{1}{2}dmr^{2} \ \Longrightarrow \ I_x = \int_{-R}^{R} \frac{1}{2}\rho \pi (R^{2} -x^{2} )^{2} dx
\end{gather*}
Per cui si ottiene il momento di inerzia rispetto a tutti gli assi della sfera come 
\begin{align}
    I = \frac{2}{5}MR^{2} 
\end{align}

\section{Utilizzo dei principi del corpo rigido a situazioni fisiche}
\subsection{Estensione della macchina di Fletcher con la carrucola reale}
\begin{wrapfigure}{r}{0.4\textwidth}
    \centering
    \caption{Macchina di Fletcher reale}
    \begin{tikzpicture}
        \draw(0, 0) -- (3, 0);
        \draw(3, 0) -- (3, -2);
        \draw(0, 0) rectangle (1, 1) node[midway] {$M_1$};
        \draw(1, 0.5) -- (3.3, 0.5);
        \draw(3.25, 0.25) circle (0.25);
        \draw(3.5, 0.3) -- (3.5, -1);
        \draw(3, -1) rectangle (4, -2) node[midway] {$M_2$};
        \draw(3, 0)-- (3.05, 0.05);
    \end{tikzpicture}    
\end{wrapfigure}
Nella macchina di Fletcher con la carrucola reale (posto che non ci siano
attriti) ho tre corpi con massa collegati con una fune inestensibile e priva
di massa. Il sistema è vincolato in modo tale da avere un solo grado di libertà 
per ogni oggetto ma non sono indipendenti: la rotazione della carrucola ed il 
movimento delle masse dipendono l'uno dall'altro e dunque i corpi si possono
muovere solamente lungo una direzione. Si suppone inoltre che la 
carrucola non oscilli (per ridurre i gradi di libertà). Preso allora un sistema di riferimento nell'origine della prima massa, posso allora
applicare il primo principio per cui per il primo oggetto:
\begin{gather*}
    \vv{T} = T\hat{i} \ \Longrightarrow \ m_1 \ddot{x}_1 = T  
\end{gather*}
Mentre per il secondo oggetto, dato che la carrucola non è ideale, avrà una
tensione diversa rispetto alla massa $1$:
\begin{gather*}
    \vv{T}' = T'\hat{j} \ \Longrightarrow \ m_2\ddot{y}_2 = T'-m_2g   
\end{gather*}
Scegliendo ora un polo di riduzione per la carrucola, su di essa agiscono 
le tensioni e la forza di gravità che è controbilanciata dalla reazione
vincolare a cui è fissata la carrucola stessa. 

Si possono schematizzare le masse e la  carrucola come:
\begin{gather*}
    \begin{tikzpicture}
        \draw(0, 0) -- (2, 0);
        \draw(0.5, 0) rectangle (1.5, 1) node[midway] {$M_1$};
        \draw[->](1.5, 0.5) -- (2.5, 0.5) node[at end, above] {$\vv{T}$};
        \draw[->](1, 0) -- (1, -1) node[at end, below] {$m_1\vv{g}$};
        \draw[->](1, 1) -- (1, 2) node[at end, right] {$\vv{N}$};
        \draw(5, 0) circle (1);
        \draw[->](5, 1) -- (4, 1) node[at end, above] {$\vv{T}$};
        \draw[->](6, 0) -- (6, -1) node[at end, right] {$-\vv{T}'$};
        \filldraw (5, 0) circle (1pt) node[anchor = south west] {$C$};
        \filldraw (5, 1) circle (1pt) node[anchor = south] {$A$};
        \filldraw (6, 0) circle (1pt) node[anchor = west] {$B$};
        \draw[->](5, 0) -- (5, -0.5) node[at end, right] {$M\vv{g}$};
        \draw[dashed] (4, 0) -- (5, 0) node[midway, above] {$R$};
        \draw(8, 0) rectangle (9, 1) node[midway] {$M_2$};
        \draw[->](8.5, 1) -- (8.5, 2) node[at end, left] {$\vv{T'}$};
        \draw[->](8.5, 0) -- (8.5, -1) node[at end, right] {$m_2\vv{g}$};
    \end{tikzpicture}    
\end{gather*}
Applicando ora la seconda cardinale alla carrucola, e conosciute le distanze tra
i punti di contatto della corda ed il centro, posso dire che i punti di contatto
delle corde e del centro sono date da  
\begin{gather*}
    (A - C) = R\hat{j} \qquad
    (B - C) = R\hat{i}  
\end{gather*}
E allora il momento torcente delle forze è dato dalla seguente relazione:
\begin{gather*}
    ((A - C) \times \vv{T} + (B - C) \times (-\vv{T}' )) \cdot  \vv{R} = I_C \dot{\omega} 
\end{gather*}
Allora dato che $\vv{\omega} = \omega\hat{k}$:
\begin{align}
    TR - T'R = I_C \dot{\omega}
\end{align}   
Definito $\phi$ come l'angolo di rotazione della carrucola si ha la seguente relazione
\begin{gather*}
    \omega = \dot{\phi} \qquad \dot{\omega} = \ddot{\phi} \ \Longrightarrow \ R\ddot{\phi} = -\ddot{x}_1
\end{gather*}
Non ci resta ora che sostituire nelle relazioni:
\begin{gather*}
    m_1 \ddot{x}_1 = T \\
    -m_2 \ddot{x_1} = T' - m_2 g \\
    TR - T'R = -I_c \frac{\ddot{x}_1}{R}
\end{gather*}
Inoltre, sapendo che il momento di inerzia di un disco è 
$\frac{1}{2}MR^{2} $, si può mettere tutto nella terza relazione:
\begin{gather*}
    \left(m_1 + m_2 + \frac{I}{R^{2} }\right)\ddot{x}_1 = m_2g \ \Longrightarrow \ \ddot{x_1} = g\frac{m_2}{m_1 + m_2 + \frac{I}{R^{2} }}
\end{gather*}
Si ricavano ora le tensioni:
\begin{align}
    T &= \frac{m_1m_2g}{m_1 + m_2 + \frac{I}{R^{2} }} \\
    T' &= \frac{m_1m_2 + \frac{m_2I}{R^{2} }}{m_1 + m_2 + \frac{I}{R^{2} }}
\end{align}


\subsection{La macchina di atwood reale attaccata al soffitto}
\begin{wrapfigure}{r}{0.4\textwidth}
    \centering
    \caption{MAcchina di atwood reale}
    \begin{tikzpicture}
        \draw(0, 0) circle (2);
        \filldraw(0, 0) circle (1pt) node[anchor = south east] {$C$};
        \filldraw(-2, 0) circle (1pt) node[anchor = east] {$A$};
        \filldraw(2, 0) circle (1pt) node[anchor = west] {$B$};
        \draw[->, thick](2, 0) -- (2, -1) node[at end, right] {$\vv{T_2}$ };
        \draw[dashed](-2, 0) -- (0, 0) node[midway, below] {$R$};
        \draw[->](0, 0) -- (0, 1) node[at end, left] {$y$};
        \draw[->](0, 0) -- (1, 0) node[at end, below] {$x$};
        \draw[->](0, 0) -- (0, -1) node[at end, right] {$M\vv{g}$};
        \draw[->](0, 0) -- (0.7, 1.3) node[at end, right] {$\vv{N}$};
        \draw(-2, 0) -- (-2, -2.5);
        \draw(-1.75, -2.5) rectangle (-2.25, -3) node[midway] {$m_1$};
        \draw[->, thick](-2, 0) -- (-2, -1) node[midway, left] {$-\vv{T}_1$};
        \draw[->, thick](-2, -2.5) -- (-2, -1.5) node[midway, left] {$\vv{T}_1$};
        \draw[->, thick](-2, -3) -- (-2, -4) node[at end, right] {$m_1\vv{g}$};
        \draw(2, 0) -- (2, -3);
        \draw(1.75, -3) rectangle (2.25, -3.5) node[midway] {$m_2$};
        \draw[->, thick](2, -3.5) -- (2, -4.5) node[at end, right] {$m_2\vv{g}$ };
        \draw[->, thick](2, -3) -- (2, -2) node[at end, right] {$\vv{T_2}$ };
        \draw[->](0.5, 0) arc (0:60:0.5) node[midway, right] {$\phi$};
    \end{tikzpicture}    
\end{wrapfigure}
Un'altra situazione di corpo rigido è considerare due masse attaccate ad una
carrucola fissata al soffitto: se la carrucola è fissa, allora il sistema
ha solamente un grado di libertà (se la fune è anche inestensibile).
Posso allora applicare per le due masse la prima cardinale:
\begin{gather*}
    m_1 \ddot{y}_1 = T_1 - m_1 g \\
    m_2 \ddot{y}_2 = T_2 - m_2 g
\end{gather*}
A questo punto la prima cardinale non mi basta per poter determinare il moto 
delle masse: se si considera che $m_1 > m_2$, allora la carrucola gira in senso
antiorario (e dunque nel verso giusto rispetto a $\phi$), allora
il contributo al momento delle forze può essere espresso come:
\begin{gather*}
    T_1 R - T_2 R = I_C \ddot{\phi}
\end{gather*}
E, data la definizione di $\phi$, posso imporre $R\ddot{\phi} = -\ddot{y}_2$, 
e dunque si può sostituire nell'equazione (come per la macchina di Atwood):
\begin{align}
    T_1 &= m_1 g\frac{2m_2 + \frac{M}{2}}{m_1 + m_2 + \frac{M}{2}}\\
    T_2 & = m_2g\frac{2m_1 + \frac{M}{2}}{m_1 + m_2 + \frac{M}{2}}
\end{align}


\subsection{Il pendolo fisico (o composto)}
\begin{wrapfigure}{r}{0.4\textwidth}
    \centering
    \caption{Il pendolo composto}
    \begin{tikzpicture}
        \filldraw(0, 0) circle (1pt) node[anchor = west] {$O$};
        \filldraw(1, -2) circle (1pt) node[anchor = south] {$C$};
        \draw(0, 0) -- (1, -2) node[midway, right] {$h$};
        \draw[->](1, -2) -- (1, -3) node[at end, left] {$M\vv{g}$};
        \draw[dashed](0, 1) -- (0, -3);
        \draw[->](0, -1) arc (-90: -60: 1) node[midway, below] {$\phi$}; 
        \draw[dashed](0, -2.2) arc (-90: 0:2.5);
        \draw[dashed](1, -2) -- (1.5, -3) node[midway,above] {$\hat{n}$ };
        \draw[dashed](1, -2) -- (3, -0.8) node[midway, below] {$\hat{t}$}; 
        \draw[->](1, -3) arc(-90:-60:1) node[midway, below] {$\phi$};
        \draw[->](0, 0) -- (1, 0.8) node[at end, above] {$\vv{N}$};
    \end{tikzpicture}    
\end{wrapfigure}
Il pendolo fisico (o composto) è un corpo rigido che compie il moto
di un pendolo; in questo caso il perno di rotazione non coincide con il centro di massa.
Mettendoci nel caso di piccole oscillazioni, posso
dire che questo moto è essenzialmente un moto armonico nel pendolo semplice
con periodo:
\begin{gather*}
    T = 2\pi \sqrt{\frac{L}{g}} 
\end{gather*}
Dato che non so dove è diretta la forza vincolare e dunque
la metto dove voglio; adesso dato che ho solo un grado di libertà 
(che ho parametrizzato con $\phi$), posso utilizzare solo un'equazione 
per risolvere il problema però io non conosco né in modulo né in 
direzione la forza vincolare. Posso dunque prendere il perno come polo di riduzione. 
Allora posso esprimere il momento di inerzia come e la prima cardinale come:
\begin{gather*}
    I_0 = I_C + Mh^{2}  \ \Longrightarrow \ -Mgh\sin\phi = I_0 \ddot{\phi}
\end{gather*}
Per cui l'equazione di moto per un pendolo assume la seguente forma:
\begin{align}
    \ddot{\phi} + \frac{Mgh}{I_0}\sin\phi = 0 
\end{align}
Per cui il suo periodo diventa:
\begin{align}
    T = 2\pi \sqrt{\frac{I_0}{Mgh}} 
\end{align}
Possiamo esprimere la normale come
\begin{gather*}
    \vv{N} + N_t \hat{t} + N_n \hat{n}   
\end{gather*}
Dove $\hat{t}$ è il versore della tangente mentre $\hat{n}$ è il versore
normale rispetto alla congiungente. Possiamo allora esprimere la prima
cardinale per le due come
\begin{gather*}
    N_t - Mg\sin\phi = Mh\ddot{\phi} \\
    N_n- Mg\cos\phi = Mh\dot{\phi}^{2} 
\end{gather*}  
Posso ottenere allora la derivata secondo di $\phi$ e quindi posso determinarmi
la componente tangenziale e ottenere:
\begin{gather*}
    \ddot{\phi} = -\frac{Mgh}{I_0}\sin\phi
\end{gather*}
Per la componente tangenziale:
\begin{align}
    N_t = Mg \sin\phi - Mh\frac{Mgh}{I_0}\sin\phi = Mg\sin\phi\left(1 - \frac{Mh^{2} }{I_0}\right)
\end{align}
Nel caso del pendolo semplice la tangenziale è zero. Moltiplicando per entrambe le parti
di $\ddot{\phi}$ per integrare e rimuovere $dt$ si ottiene per le condizioni
iniziali in $t = 0$, $\phi = 0$ e $\dot{\phi} = \Omega$:
\begin{gather*}
    \dot{\phi}^{2} = \Omega_0^{2} - \frac{2Mgh}{I_0}(1 - \cos\phi)  
\end{gather*} 
sostituendo nell'espressione per la normale:
\begin{align}
    N_n = Mg\cos\phi + Mh\left(\Omega_0^{2} - \frac{2Mgh}{I_0}(1 - \cos\phi)\right)
\end{align}
Nel caso in cui l'angolo iniziale fosse noto per esempio $\phi = \phi_0$:
\begin{align}
    N_n = Mg\cos\phi + Mh\frac{2Mgh}{I_0}(\cos\phi - \cos\phi_0) = Mg\cos\phi \left(1 + \frac{2Mh^{2} }{I_0}\right) - \frac{2M^{2}h^{2}g}{I_0}\cos\phi_0
\end{align}

\section{Il moto di rotolamento}
\begin{wrapfigure}{r}{0.4\textwidth}
    \centering
    \caption{Il moto di rotolamento}
    \begin{tikzpicture}
        \draw(0, 0) -- (4, 0);
        \draw(0, 0) -- (0, 2);
        \draw(0, 2) -- (4, 0);
        \draw[->](0, 2) -- (1, 1.5) node[at end, below] {$x$};
        \draw[->](0, 2) -- (0.5, 3) node[at end, left] {$z$};
        \draw(3.5, 0.25) arc (150:180:0.5) node[midway, left] {$\alpha$};
        \draw(2, 1.55) circle (0.5);
    \end{tikzpicture}    
\end{wrapfigure}
Il moto di rotolamento è il moto di rotazione di un corpo rigido
che possa rotolare (come una sfera etc..), scelto un sistema di riferimento
in modo tale che l'asse $x$ sia parallela al piano inclinato e
l'asse $z$ perpendicolare al piano. Per avere ora moto di rotolamento
puro, io devo avere un corpo con un attrito "specifico" che posso
ricavarmi in funzione dell'angolo:
\begin{gather*}
    \vv{N} + \vv{F}_a + m\vv{g} = m\vv{a}_C    
\end{gather*}
Scomponendo posso dire
\begin{gather*}
    x) \quad mg\sin\alpha - F_a = ma \\
    z) \quad N - mg\cos\alpha = 0
\end{gather*}
Posso utilizzare la seconda cardinale nella forma semplice (considerando
allora la proiezione lungo la velocità angolare dei momenti). La velocità angolare
lungo $y$ non cambia e mi trovo nelle condizioni
nelle quali posso utilizzare la forma assiale della seconda cardinale lungo
l'asse $y$. Posso scegliere un polo di riduzione in modo da utilizzare le formule
semplici: per farlo il polo deve essere fisso o coincidere con il centro di massa. 
C'è anche la possibilità di scegliere un polo la cui velocità è parallela a quella del
centro di massa. Scelto come polo di riduzione $\Omega = C$, allora
posso schematizzare il corpo rigido come:
\begin{gather*}
    \begin{tikzpicture}
        \draw(0, 0) circle (2);
        \filldraw(0, 0) node[anchor = south] {$C$};
        \draw[dashed](-1.41, -1.41) -- (1.41, 1.41);
        \draw[->](0, 0) -- (0, -1) node[at end, right] {$m\vv{g}$}; 
        \draw[thick, ->](-1.41, -1.41) -- (-0.8, -0.8) node[at end, right] {$\vv{N}$};
        \draw[thick, ->](-1.41, -1.41) -- (-2, -0.82) node[at end, below] {$\vv{F}_a$};
        \draw[->](1, 0) arc (0:-45:1) node[midway, right] {$\dot{\phi}$};
    \end{tikzpicture}
\end{gather*}
Nelle condizioni di rotolamento puro c'è una relazione semplice tra $\ddot{x}$ e $\ddot{\phi}$;
dato che $\dot{\phi}$ diventa positiva, anche la sua derivata diventa positiva,
allora anche $\ddot{x}$ aumenta nello stesso senso, posso esprimere
allora la prima cardinale come:
\begin{gather*}
    \dot{\phi} = \frac{\ddot{x}}{R} \ \Longrightarrow \ F_a R = I_C \ddot{\phi} 
\end{gather*}
Posso riscrivere le equazioni sugli assi come:
\begin{gather*}
    \left\{\begin{array}{l}
        mg\sin\alpha - F_a = m\ddot{x} \\
        F_a = \frac{I_c}{R^{2}}\ddot{x}
    \end{array}\right.
\end{gather*}
L'accelerazione del centro di massa è data da:
\begin{gather*}
    \ddot{x} = \frac{g\sin\alpha}{1 + \frac{I_C}{mR^{2} }}
\end{gather*}
Che è minore di quella che ha se scivolasse senza attrito; posso
esprimere il momento di inerzia per alcuni corpi rigidi come:
\begin{align*}
    &\text{disco omogeneo}: & I_C = \frac{1}{2}mR^{2} \ \Longrightarrow \   \ddot{x} = \frac{2}{3} g\sin\alpha \\
    &\text{sfera omogenea}: &I_c = \frac{2}{5}mR^{2} \ \Longrightarrow \   \ddot{x} = \frac{5}{7}g\sin\alpha \\
    &\text{cerchione}: &I_C = mR^{2} \ \Longrightarrow \  \ddot{x} = \frac{1}{2}g\sin\alpha 
\end{align*}
Se volessimo considerare il caso del disco si otterrebbe:
\begin{gather*}
    F_a = \frac{1}{3}mg\sin\alpha
\end{gather*}
Il caso limite dell'attrito è dunque
\begin{gather*}
    F_a \leq \mu_s N  \ \Longrightarrow \ \tan\alpha \leq 3\mu_s
\end{gather*}

\section{Carrucole mobili}
\begin{wrapfigure}{r}{0.4\textwidth}
    \centering
    \caption{Carrucola mobile}
    \begin{tikzpicture}
        \draw(0, 0) -- (1, 0);
        \draw(0.5, 0) -- (0.5, -2);
        \draw(1.5, -2) circle(1);
        \draw[->, thick](2.5, -2) -- (2.5, -1) node[at end, right] {$\vv{F}$};
        \draw(2.5, -1) -- (2.5, -0.5);
        \draw[->, thick](0.5, -2) -- (0.5, -1) node[at end, left] {$\vv{T}$};
        \draw[->, thick](1.5, -2) -- (1.5, -2.5) node[at end, right] {$m\vv{g}$};
    \end{tikzpicture}    
\end{wrapfigure}
La carrucola mobile è una carrucola che non è fissata in alcun modo: in questo modo
non c'è alcuna forza vincolare da bilanciare. Se si passasse una fune fissata da una
parte che si avvolge sulla carrucola e che viene tirata dall'altra parte, allora la carrucola
compie effettivamente un moto di rotolamento puro:
\begin{gather*}
    \vv{T} + \vv{F} + M\vv{g} = M\vv{a}_C \\
    T + F - Mg = M\ddot{z}    
\end{gather*}
Preso come centro di riduzione il centro di massa, allora posso decidere
come prendere l'angolo (in senso orario) e quindi 
\begin{gather*}
    \ddot{\phi} = -\frac{\ddot{z}}{R} \ \Longrightarrow \ (T - F)R = I_C \ddot{\phi}
\end{gather*}
E allora, ricavandomi le espressioni di $F$ e $T$, posso dire che l'equazione
prende la forma di
\begin{gather*}
    2F - Mg = \left(M + \frac{I_C}{R^{2} }\right)\ddot{z}
\end{gather*}
La condizione di equilibrio è proprio quella per cui
\begin{gather*}
    F = \frac{1}{2}Mg 
\end{gather*}
e dunque posso ricavare la tensione come 
\begin{gather*}
    T = F - \frac{I_c}{R^{2} }\frac{2F - Mg}{M + \frac{I_c}{R^{2} }}
\end{gather*}
Vedo che la carrucola mobile può dimezzare la forza necessaria per poter tirare oggetti. 

\section{L'espressione del tensore di inerzia }
Dato che ruota con velocità $\omega$ allora posso scegliere un
polo ossia il centro di massa 
\begin{gather*}
    \vv{L}_C = \sum_{i = 1}^{n}  (P_i - C) \times (m_i\vv{v_i} ) = \sum_{i = 1}^{n}  (P_i - C) \times m_i(\vv{v}_i + \vv{\omega} \times (P_i - C)  )
\end{gather*}
Dato che la velocità del centro di massa è zero, allora posso dire che $\vv{v_i} $ sono tutte zero
e quindi, con le proprietà del prodotto vettoriale, posso semplificare
\begin{gather*}
    \vv{L}_C = \sum_{i = 1}^{n}m_i(P_i - C)(P_i - C)\vv{\omega} - \sum_{i = 1}^{n}m_i(P_i - C)\vv{\omega}(P_i - C)     
\end{gather*}
Dato che ora $P_i - C$ è il sistema di riferimento per ogni punto
del corpo rigido, posso allora dire che
\begin{gather*}
    (P_i - C) = x_i \hat{i} + y_i \hat{j} + z_i \hat{k}, \qquad \vv{\omega} = \omega_x \hat{i} + \omega_y \hat{j} + \omega_z \hat{k}       
\end{gather*}
Quindi il momento angolare diventa semplificando e risolvendo
\begin{gather*}
    \vv{L}_C = \sum_{i = 1}^{n}m_i\left\{\begin{array}{l}
        (y_i^{2}\omega_x + z_i^{2}\omega_x - x_iy_i\omega_y - x_iz_i\omega_z)\hat{i} \\
        (x_i^{2}\omega_y + z_i^{2}\omega_y - x_iy_i\omega_x - y_iz_i\omega_z)\hat{j} \\
        (x_i^{2}\omega_z + y_i^{2}\omega_z - x_iz_i\omega_x - y_iz_i\omega_y)\hat{k}   
    \end{array}\right.  
\end{gather*}
POsso allora esprimere il momento angolare in forma matriciale
\begin{align}
    \begin{pmatrix}
        L_x \\
        L_y \\
        L_z
    \end{pmatrix} = \begin{pmatrix}
        \sum_{i}^{}m_i(y_i^{2} + z_i^{2}) & -\sum_{i}^{} m_i x_iy_i & -\sum_{i}^{} m_ix_i z_i \\
        -\sum_{i}^{}m_ix_iy_i & \sum_{i}^{} m_i(x_i^{2} + z_i^{2} ) & -\sum_{i}^{} m_iy_iz_i \\
        -\sum_{i  }^{}m_ix_iz_I & -\sum_{i }^{}m_iy_iz_i & \sum_{i }^{}   m_i(x_i^{2} + y_i^{2} )
    \end{pmatrix}\cdot \begin{pmatrix}
        \omega_x \\
        \omega_y \\
        \omega_z
    \end{pmatrix}
\end{align}
Posso esprimere questa matrice come un \textbf{tensore di inerzia} ossia
come
\begin{gather*}
    \begin{pmatrix}
        L_x \\
        L_y \\
        L_z
    \end{pmatrix} = \begin{pmatrix}
        I_{xx} & I_{xy} & I_{xz} \\
        I_{xy} & I_{yy} & I_{yz} \\
        I_{xz} & I_{yz} & I_{zz} 
    \end{pmatrix} \cdot \begin{pmatrix}
        \omega_x \\
        \omega_y \\
        \omega_z
    \end{pmatrix}
\end{gather*}
Se la matrice è diagonalizzabile allora posso trovare degli autovalori
per cui posso rendere la matrice diagonale attraverso gli \textbf{assi di principali
di inerzia} ossia gli assi $x', y', z'$:
\begin{gather*}
    \begin{pmatrix}
        L_x \\
        L_y \\
        L_z
    \end{pmatrix} = \begin{pmatrix}
        I_{x'x'} & 0 & 0 \\
        0 & I_{y'y'} & 0 \\
        0 & 0 & I_{z'z'} 
    \end{pmatrix} \cdot \begin{pmatrix}
        \omega_{x'} \\
        \omega_{y'} \\
        \omega_{z'}
    \end{pmatrix}
\end{gather*}
Se  $\vv{\omega}$ è parallelo ad un asse principale di inerzia allora
posso trovare che si annullano le altre due componenti e quindi il momento
sarà presente solo rispetto a quell'asse. 

\section{Momento del corpo rigido rispetto ad un asse }
Dato un asse fisso ed un sistema di coordinate cilindriche
posso dire che la distanza di un qualsiasi punto dal centro
del corpo rigido è data da:
\begin{gather*}
    (P_i - O) = z_C\hat{k} + \vv{\rho}_i  
\end{gather*}
Dato che il sistema di riferimento è solidale con il corpo rigido in
rotazione, posso dire che sono valide le seguenti:
\begin{gather*}
    \rho_i = \rho_i \hat{u}_{\rho} \\
    \left| \rho_i \right| = CONST  \\
    \vv{v} = \vv{\omega} (P - O) \\
    \vv{v}_O = 0   
\end{gather*}
Posso esprimere il momento angolare come
\begin{gather*}
    \vv{L}_0 = \sum_{i = 1}^{n} (P_i - O) \times m_i \vv{v}_i   
\end{gather*}
Allora posso dire, con tutte le sostituzioni
\begin{gather*}
    \sum_{ = }^{} (z_o \hat{k} + \vv{\rho}_i ) \times m_i(\omega \hat{k} \times (z_i \hat{k} + \vv{\rho}_i ) )  
\end{gather*}
E quindi, svolgendo i prodotti vettoriali, e dato che è una terna destrorsa
locale (che cambia dunque da punto a punto), posso sapere già che
\begin{gather*}
    \vv{u}_{\rho}, \vv{u}_{T_i}, \hat{k} \\
    \hat{k} \times \hat{u}_{\rho} = \hat{u}_{T_i}      
\end{gather*}
Posso allora esprimere il momento angolare come
\begin{gather*}
    L_O = \sum m_i \rho_i^{2}\vv{\omega} - \omega \sum   m_i z_i \vv{\rho}_i
\end{gather*}
Il primo termine è il momento angolare parallelo alla velocità angolare
mentre il secondo è il momento angolare ortogonale alla velocità angolare.
Chi sono questi due contributi? Il primo è proprio il momento di inerzia
dell'i-esimo punto rispetto all'asse di rotazione considerato.
Se il mio asse di rotazione è parallelo ad un asse di simmetria,
allora  il secondo termine potrà essere eliminato
\begin{gather*}
    \vv{L}_O = \vv{L}_{\parallel} + \vv{L}_{\perp}   
\end{gather*}
E quindi loro sono proprio:
\begin{gather*}
    \vv{L}_{\parallel} = I_{O}\vv{\omega} \\
    \vv{L}_{\perp} = - \omega \sum m_iz_i \vv{\rho}_i 
\end{gather*}
Derivando l'espressione del momento angolare
posso vedere che si ottiene
\begin{gather*}
    \dot{\vv{L} }_0 = \sum m_i \rho_i^{2}\dot{\vv{\omega} } - \dot{\omega} \sum   m_i z_i \vv{\rho}_i - \omega \sum  m_i z_i \dot{\vv{\rho} }_i
\end{gather*}
Dato che $\rho_i$ ha modulo costante e ho una velocità angolare,
allora la sua derivata è proprio (dalle formule di Poisson) 
\begin{gather*}
    \dot{\vv{\rho}}_i = \vv{\omega} \times \vv{\rho}_i  
\end{gather*}
Quindi si ottiene
\begin{gather*}
    \dot{\vv{L} }_O = I_{O}\dot{\vv{\omega} } - \frac{\dot{\omega}}{\omega}\omega \sum m_i z_i \vv{\rho}_i - \omega \sum  m_i z_i \vv{\omega} \times \vv{\rho}_i   
\end{gather*}
E allora si ottiene proprio
\begin{gather*}
    \dot{\vv{L} }_O = I_O \dot{\vv{\omega} } + \frac{\dot{\omega}}{\omega} \vv{L}_{\perp} + \vv{\omega} \times \vv{L}_{\perp}   
\end{gather*}
Il primo termine è parallelo rispetto al $\dot{\vv{L}}_O$ e quindi al
versore $\hat{k}$,  il secondo termine è ortogonale a $k$ mentre il terzo è un prodotto vettoriale e quindi è 
ortogonale a tutti e due i termini del prodotto vettoriale. SI può ora
riscrivere la seconda cardinale come
\begin{align}
    \vv{M}_O^{(ext)} = \dot{\vv{L} }_{\parallel} + \frac{\dot{\omega}}{\omega}\vv{L}_{\perp} + \vv{\omega} \times \vv{L}_{\perp}     
\end{align}
La seconda cardinale allora è espressa per il corpo rigido che ruota rispetto
ad un asse fisso; si vede allora che nella direzione di $\omega$ allora rimane solo il primo termine
come
\begin{gather*}
    \vv{M}^{ext}_{O} \hat{k} = I_O \dot{\omega}   
\end{gather*}
Se io metto in rotazione un oggetto, allora essendo che deve per forza
ruotare lungo l'asse considerato, esso continua a girare e quindi 
il momento angolare sta cambiando e quindi il terzo termine sta cambiando 
e non  trascurabile, da dove viene allora? E' una forza vincolare
che viene scaturita dalla rotazione impressa alle condizioni iniziali .

\section{Conservazione dell'energia e lavoro nei corpi rigidi}
\begin{wrapfigure}{r}{0.4\textwidth}
    \centering
    \begin{tikzpicture}
        \filldraw (1, 1) circle (1pt) node[anchor = south] {$P_2$};
        \filldraw(3, 2) circle (1pt) node[anchor = south] {$P_1$};
        \draw[->](0, 0) -- (0.5, 0) node[at end, below] {$x$};
        \draw[->](0, 0) -- (-0.7, -0.45) node[at end, above] {$y$};
        \draw[->](0, 0) -- (0, 0.5) node[at end, left] {$z$};
        \draw[->](1, 1) -- (1.9, 1.45) node[at end, below] {$\vv{F_{1,2}} $};
        \draw[->](3, 2) -- (2.1, 1.55) node[at end, above] {$\vv{F_{2,1}} $};
    \end{tikzpicture}    
\end{wrapfigure}
Si può definire il lavoro e l'energia nei corpi rigidi attraverso un semplice sistema di due
punti: rispetto ad un sistema di riferimento $S$ posso allora chiedermi
quale sia il lavoro compiuto infinitesimamente che, per ogni punto, è dato da:
\begin{gather*}
    \delta L_1 + \delta L_2 = \vv{F}_{1, 2} \cdot  d\vv{r}_1 + \vv{F}_{2, 1} \cdot d \vv{r}_2     
\end{gather*}
Dove $\vv{r_1}$ indica il vettore posizione del punto $P_1$ rispetto all'origine
di $S$ e $\vv{r_2}$ indica il vettore posizione del vettore, mentre il vettore
$d\vv{r_1}$ ed il vettore $d\vv{r_2}$ indicano lo spostamento infinitesimo dei due punti.   
Dato che il corpo è rigido allora la distanza tra i due punti è
costante e quindi
\begin{gather*}
    (d\vv{r}_1 - d\vv{r}_2  )\cdot (\vv{r}_1 - \vv{r}_2  ) = 0
\end{gather*}
Dato che la forza è parallela alla distanza tra il primo ed il secondo punto
allora posso dire che il lavoro è nullo; dunque il lavoro delle forze di
rigidità è nullo (che ha senso dato che il corpo è rigido).
Il lavoro totale infinitesimo sulle forze interne del corpo 
rigido è dato dalla somma di ciascun effetto reciproco:
\begin{gather*}
    \delta L = \sum \delta L_i = \sum \vv{F}_i 
\end{gather*}
Preso un punto a caso $A$ nel corpo rigido la velocità di un dato punto rispetto
a quel punto sarà
\begin{gather*}
    \vv{v}_i = \vv{v}_A + \vv{\omega} \times (P_i - A)   \ \Longrightarrow \ \delta L = \sum \vv{F}_i (\vv{v}_A + \vv{\omega} \times (P_i - A)  )dt 
\end{gather*}
Posso allora dire che il lavoro infinitesimo sarà dato dalla seguente formulazione:
\begin{gather*}
    \delta L = \vv{F}^{ext}\cdot  d\vv{r}_A + \omega dt\sum (P_i - A) \times \vv{F}_i 
\end{gather*}
Il primo termine non è altro che la risultante delle forze esterne mentre il secondo termine 
è proprio il risultato dei momenti delle singole forze esterne moltiplicate per $\vv{\omega}$ e 
quindi dato che
\begin{gather*}
    \vv{\omega} = \dot{\phi} \hat{u}_{\phi} \ \Longrightarrow \ \dot{\phi} dt = d\phi  
\end{gather*} 
Dunque 
\begin{gather*}
    \delta L = \vv{F}^{ext}\cdot d\vv{r}_A + \vv{M}^{ext} _A \cdot  \hat{u}_{\phi}d\phi     
\end{gather*}
Il lavoro allora dipende dalle risultanti delle forze esterne e dallo spostamento
infinitesimo angolare; se si avesse un solo moto di rotazione, potrei scegliere $A$ sull'asse
ed il lavoro sarebbe solo dato dal momento assiale delle forze esterne per l'angolo.
Se si considerassero le tre componenti del momento angolare conterebbe solo
il momento assiale lungo la velocità angolare. Allora se ruota in modo caotico il vincolo
che gli viene applicato può essere liscio e può non compiere alcun lavoro.
Dato che posso esprimere l'energia cinetica come
\begin{gather*}
    K = \frac{1}{2}Mv_C^{2} + \sum_{i = 1}^{n}m_i v_i^{2'}   
\end{gather*}
Come conseguenza del teorema di Konig
\begin{gather*}
    \vv{v}_i' = \vv{\omega} \times (P_i - C)  
\end{gather*}
Allora la componente rotazionale dell'energia cinetica del corpo rigido è
data (considerata $d_i$ come la distanza di ogni punto dall'asse di rotazione
considerato)
\begin{gather*}
    \frac{1}{2}\sum_{i = 1}^{n} m_i d_i \omega^{2}  
\end{gather*}
Allora l'energia cinetica del corpo rigido diventa proprio la somma tra
l'energia cinetica della velocità del centro di massa più la componente rotazionale
data dalla precedente. L'energia cinetica totale si esprimerebbe come
\begin{align}
    K = \frac{1}{2}Mv_c^{2} + \frac{1}{2} I_C \omega^{2}  
\end{align}
Dove $I_C$ non è altro che il momento di inerzia del corpo rigido.

\subsection{Pendolo fisico}
Nel pendolo composto con vincolo ideale, si può applicare la conservazione
dell'energia scegliendo quindi l'energia 
\begin{gather*}
    E = \frac{1}{2}I_O \omega ^{2} - Mgh\cos\phi 
\end{gather*}
Dato che la velocità angolare è la derivata dell'angolo, posso
esprimere la derivata dell'energia
\begin{gather*}
    \dot{E}  =0 = I_O \dot{\phi} \ddot{\phi} + Mgh\sin\phi\dot{\phi}
\end{gather*} 
Il periodo del pendolo è allora dato da
\begin{gather*}
    T = 2\pi \sqrt{\frac{I_O}{Mgh}} 
\end{gather*}

\subsection{Moto di rotolamento}
L'energia per il moto di rotolamento è data dalla seguente:
\begin{gather*}
    E = \frac{1}{2}Mv_{C}^{2} + \frac{1}{2}I_C \omega^{2} - Mgx_C\sin\alpha
\end{gather*}
La velocità è esprimibile come la derivata della posizione e dunque
dato che $\omega = \frac{\dot{x}}{R}$:
\begin{gather*}
    E = \frac{1}{2}(MR^{2} + I_C )\dot{\phi}^{2} - MgR\dot{\phi} \sin\alpha
\end{gather*}
Derivando rispetto a $\dot{\phi}$ e ponendo uguale a zero
si ha la conservazione dell'energia con la seguente espressione:
\begin{gather*}
    (MR^{2} + I_C)\ddot{\phi}-  MgR\sin\alpha
\end{gather*}
Potrò esprimere l'accelerazione come
\begin{gather*}
    \ddot{x} = \frac{MgR\sin\alpha}{(MR^{2} + I_C)\frac{1}{R}} = \frac{g\sin\alpha}{1 + \frac{I_C}{MR^{2} }}
\end{gather*}

\chapter{Gli urti}
\section{L'impulso e gli urti}
\begin{wrapfigure}{r}{0.4\textwidth}
    \centering
    \caption{Il grafico della forza}
    \begin{tikzpicture}
        \draw[->](0, 0) -- (3, 0) node[at end, below] {$t$};
        \draw[->](0, 0) -- (0, 3) node[at end, left] {$F $};
        \draw(0.25, 0.5) .. controls (0.8, 0.5) and (0.9, 1) .. (1, 1.5);
        \draw(2, 1.5) .. controls (2.1, 1) and (2.2, 0.5) .. (2.75, 0.5);
        \draw(1, 1.5) .. controls (1.35, 3.5) and (1.65, 3.5) .. (2, 1.5); 
        \draw[red](0, 0.75) -- (3, 0.75);
    \end{tikzpicture}    
\end{wrapfigure}
Possiamo considerare l'impulso come 
\begin{align}
    \vv{I} = \int \vv{F} \ dt  
\end{align}
Ossia l'impulso è l'integrale della forza rispetto al tempo. Fisicamente
si può definire come l'\textbf{impatto} che ha una forza su un dato corpo:
più una forza riesce ad accrescere il momento di una forza in breve termine e più
impatto ha. Si definiscono due tipi di forze:
\begin{itemize}
    \item \textbf{Forze impulsive}: sono forze che imprimono una grande variazione
    della quantità di moto in un $\delta t$ molto piccolo.
    \item \textbf{Forze non impulsive}: sono forze che imprimono piccola
    variazione della quantità di moto in un $\Delta t$ molto grande.
\end{itemize}
Si definisce allora il \textbf{Teorema dell'impulso} come:
\begin{align}
    \vv{I} = \int \vv{F} \ dt = \Delta \vv{Q} 
\end{align}
Ossia l'impulso corrisponde alla variazione della quantità di moto.

\subsection{Definizione di urto e forze in gioco}
\begin{wrapfigure}{r}{0.4\textwidth}
    \centering
    \caption{L'urto tra due palline}
    \begin{tikzpicture}
        \draw(0, 0) -- (3, -3);
        \draw(0.6, -0.25) circle (0.25);
        \draw(2.15, -1.75) circle (0.25);
        \draw[->](0.6, -0.25) -- (1.1, -0.75);
        \draw[->](2.15, -1.75) -- (1.65, -1.25);
    \end{tikzpicture}    
\end{wrapfigure}
Nel caso generale degli urti le forze che prendono parte agli urti 
sono delle grandi forze che agiscono in poco tempo. Se due palline si urtano, allora si sviluppano delle forze molto
intense in tempi molto brevi che cambiano drasticamente la quantità di moto del sistema.
Considerando l'intervallo di tempo durante l'urto posso trascurare l'impulso delle
forze non impulsive in quanto in quel lasso di tempo le forze impulsive sono
molto maggiori in modulo delle forze non impulsive e quindi nell'integrale
dell'impulso posso solo considerare il contributo delle forze impulsive. \\
Se si considera l'insieme formato dai due corpi, le forze impulsive sono
delle forze interne in quanto l'urto di un corpo all'altro è esattamente la forza
che imprime uno dei due oggetti dall'altro. Per quanto riguarda quindi 
la variazione della quantità di moto si può dire che il sistema è isolato in quanto
anche se c'è la forza peso, etc .., le forze d'urto sono
le uniche che considero e allora la quantità di moto del sistema
si conserva in quanto considero solo l'effetto delle forze interne.
Possiamo allora considerare il sistema \textbf{isolato} e dunque possiamo
utilizzare la conservazione della quantità di moto, del momento angolare e 
dell'energia meccanica del sistema.

\subsection{E' sempre possibile trascurare le forze non impulsive?}
Nonostante quanto detto finora, non è possibile sempre trascurare le forze
non impulsive in quanto nell'urto di una pallina dall'alto con un certo angolo rispetto all'orizzontale
su di una pallina ferma, questa potrebbe voler sfondare il piano su cui è appoggiata
ma questo non accade se si considera anche il contributo del vincolo.In generale
bisogna sempre chiedersi se il vincolo ha una certa forza impulsiva che può scaturire
a causa delle forze impulsive. La geometria del sistema di riferimento diventa 
dunque essenziale per determinare o meno l'impulsività delle forze vincolari.

\subsection{Dimostrazioni}
Si può dimostare la conservazione della quantità di moto durante gli urti: se si considerano solamente
le forze impulsive e due corpi che si scontrano, il teorema dell'impulso ci permette
di calcolare la variazione della quantità di moto durante un breve intervallo di tempo
(ossia durante un intervallo di tempo in cui è possibile considerare solamente le forze
impulsive):
\begin{gather*}
    \vv{I} = \Delta \vv{q}  = \Delta \vv{q^{(i)} }  + \Delta \vv{q^{e} }  = \int_{t_i}^{t_f} \vv{F} \ dt 
\end{gather*}
Se si applica il teorema a due corpi in urto:
\begin{gather*}
    \vv{I_1} = \Delta \vv{q_1} = \int_{t_i}^{t_f} \vv{F_{1, 2}^{(i)} }  \ dt   + \int_{t_i}^{t_f} \vv{F_{2, 1}^{(e)} } \ dt 
    \vv{I_2} = \Delta \vv{q_2} = \int_{t_i}^{t_f} \vv{F_{2, 1}^{(i)} }  \ dt   + \int_{t_i}^{t_f} \vv{F_{1, 2}^{(e)} } \ dt 
\end{gather*}
Dunque se si hanno solamente forze impulsive:
\begin{gather*}
    \vv{I_1} = \Delta \vv{q_1} = \int_{t_i}^{t_f} \vv{F_{1, 2}^{(i)} }  \ dt 
    \vv{I_2} = \Delta \vv{q_2} = \int_{t_i}^{t_f} \vv{F_{2, 1}^{(i)} }  \ dt 
\end{gather*}
Per la terza equazione della dinamica si ha che 
\begin{gather*}
    \vv{I_1} = \int_{t_i }^{t_f} \vv{F_{1, 2}^{(i)} } \ dt = \Delta \vv{q_i} = - \Delta \vv{q_2}    \\
    \vv{I_2} = \int_{t_i}^{t_f} \vv{F_{2, 1}^{(i)} } \ dt = \Delta \vv{q_2} 
\end{gather*}
Se si considera  la variazione della quantità di moto totale del sistema:
\begin{gather*}
    \Delta \vv{Q} = \Delta \vv{q_1} + \Delta \vv{q_2} \approx -\Delta \vv{q_2} + \Delta \vv{q_2} \approx 0     
\end{gather*}
E dunque la quantità di moto si conserva. Con lo stesso procedimento si dimostra come
è possibile dimostrare la conservazione del momento angolare. \\
Per determianre la conservazione dell'energia, si deve ricorrere alla definizione
di sistema isolato: in termodinamica un sistema si dice \textbf{isolato} se non scambia
energia con l'ambiente esterno. In altre parole il lavoro delle forze esterne si può 
non consderare e dunque l'energia propria del sistema è conservata. 
\begin{gather*}
    E = U + K
\end{gather*}
Ossia l'energia meccanica interna al sistema si conserva. Si considera inoltre
che le forze interne siano nulle o non compiano alcun lavoro. Non è detto infatti
che si possa conservare l'energia cinetica (anche se si conserva l'energia totale)
poiché dipende dalla tipologia di urto. 

\section{Urto unidimensionale tra due corpi}
\begin{wrapfigure}{r}{0.4\textwidth}
    \centering
    \caption{}
    \begin{tikzpicture}
        \draw(0, 0) -- (5, 0);
        \draw(1, 0.5) circle (0.5);
        \draw(4, 0.5) circle (0.5);
        \draw[->](1.5, 0.5) -- (2.25, 0.5) node[midway, above] {$\vv{v_1}$ };
        \draw[->](3.5, 0.5) -- (2.75, 0.5) node[midway, above] {$\vv{v_2}$ };
    \end{tikzpicture}    
\end{wrapfigure}
Dato che si considerano solamente i contributi delle forze impulsive
negli urti, allora posso utilizzare la quantità di moto nell'urto
unidimensionale tra due corpi
\begin{gather*}
    m_1 \vv{v_{1f}} + m_2 \vv{v_{2i}} = m_1 \vv{v_{1f}} + m_2 \vv{v_{2f}}    
\end{gather*}
Dato che ho sei incognite, posso ricondurmi al caso unidimensionale
in modo tale da poter ridurre le incognite della mia equazione
a due sole. Posso farlo con buona
approssimazione se e solo se l'urto è centrato (ossia la traiettoria contiene entrambi i centri
delle palline). Adesso ciò che succederà dipende dalla forza di interazione
e si potrebbe allora dissipare energia (quello che succede all'energia cinetica
dipende dai dettagli dell'urto). Possiamo definire allora
due tipologie di urto: 
\begin{itemize}
    \item \textbf{Urto elastico}, ossia un urto in cui 
    l'energia cinetica subito prima dell'urto e subito dopo l'urto è uguale.
    \item \textbf{Urto anelastico}: si definisce urto anelastico un 
    urto in cui non si conserva l'energia cinetica e dove il
    lavoro delle forze interne è nullo.
    \item \textbf{Urto completamente anelastico}: si verifica quando dissipo 
    tutta l'energia relativa al centro di massa e dunque i due corpi si attaccano.
\end{itemize}
Nel caso di urto elastico posso utilizzare anche la conservazione dell'energia cinetica per
determinare la velocità finale:
\begin{gather*}
    \left\{\begin{array}{l}
        m_1 \vv{v}_{1f} + m_2 \vv{v}_{2i} = m_1 \vv{v}_{1f} + m_2 \vv{v}_{2f}     \\
        \frac{1}{2}m_1 v_{1i}^{2} + \frac{1}{2}m_2v_{2i}^{2} =  \frac{1}{2}m_1 v_{1f}^{2} + \frac{1}{2}m_2v_{2f}^{2} 
    \end{array}\right.
\end{gather*}
Posso allora risolvere
\begin{gather*}
    m_1(v_{1i} - v_{1f}) =  m_2(v_{2i} - v_{2f}) \\
    m_1(v_{1i}^{2}  - v_{1f}^{2} ) =  m_2(v_{2i}^{2}  - v_{2f}^{2} )
\end{gather*}
Le velocità finali allora diventano
\begin{align}
    v_{1f} &= \frac{m_1 - m_2}{m_1 + m_2}v_{1i} + \frac{2m_1}{m_1 + m_2}v_{2i} \\
    v_{2f} &= \frac{2m_1}{m_1 + m_2}v_{1i} + \frac{m_1 -m_2}{m_1 + m_2}v_{2i}
\end{align}


\subsection{Casi particolari dell'urto unidimensionale}
1. $m_1 = m_2$:
\begin{gather*}
    v_{1f}  = v_{2i} \\
    v_{1i} = v_{2f}
\end{gather*}
Le particelle si scambiano le velocità. In particolare nelle centrali
nucleari a fissione per rallentare i neutroni lenti bisogna usare come moderatore
l'acqua pesante in quanto è ricca di neutroni. \\
2. $v_{2i} = 0$
\begin{gather*}
    v_{1f} = \frac{m_1 - m_2}{m_1 + m_2}v_{1i} \\
    v_{2f} = \frac{2m_1}{m_1 + m_2}v_{1i}
\end{gather*}
Nell'urto elastico unidimensionale con una particella ferma, la seconda 
particella si mette in moto nella stessa direzione della particella
in movimento se e solo se l'urto è centrale. La prima invece può andare 
o avanti o indietro (se la massa è superiore va avanti altrimenti indietro). \\
2.1 $m_1 = m_2$ e $v_{2i} = 0$:
\begin{gather*}
    \left\{\begin{array}{l}
        v_{1f} = 0 \\
        v_{2f} = v_{1i}
    \end{array}\right.
\end{gather*}
2.b. $m_2 >> m_1$ 
\begin{gather*}
    v_{1f} = -v_{1i} \\
    v_{2f} = 0
\end{gather*}

\section{Urto bidimensionale}
\begin{wrapfigure}{r}{0.4\textwidth}
    \centering
    \caption{Schematizzazione dell'urto bidimensionale}
    \begin{tikzpicture}
        \draw(0, 2) circle (0.25);
        \draw(0, 0) -- (5, 0);
        \draw[->](0, 2) -- (0.5, 1.5) node[at end, above] {$v_{1i}$};
        \draw(2, 0.25) circle(0.25);
        \draw[->](2, 0.5) -- (2, 1) node[at end, right] {$\vv{N}$};
        \draw[dashed, thin] (0, 2) -- (2, 0.25) -- (4, 2);
    \end{tikzpicture}    
\end{wrapfigure}
Nel caso dell'urto bidimensionale abbiamo due gradi di libertà: di conseguenza non si
può applicare la conservazione dell'energia ma bisogna considerare l'impulso
delle forze e la quantità di moto per ridurre i gradi di libertà. 
In questo caso nell'urto tra due corpi, in modo tale che
\begin{gather*}
    v_{1i} \neq 0  \qquad v_{2i} = 0
\end{gather*} 
Il corpo $1$ colpirà la parete e finirà  per compiere un moto di riflessione rispetto
al vincolo e tenderà a rimbalzare con lo stesso angolo
con cui ha urtato il vincolo (ovviamente se e solo se l'urto è elastico
e lungo l'asse $x$ non ho alcuna forza perché così riduco i gradi di libertà). 
Dal punto di vista sostanziale tutti gli urti coi raggi cosmici e nella fisica 
delle particelle seguono tutti questo modello anche se per corpi relativistici dovrò stare
attento alle velocità. \\
Per risolvere questo tipo di urti possiamo utilizzare due sistemi di
riferimento
\begin{itemize}
    \item \textbf{SdR centro di massa}: il sistema di riferimento centro di massa
    è un sistema di riferimento in cui il centro di massa dell'intero sistema rimane fermo;
    \item \textbf{SdR laboratorio}: Ossia il sistema di riferimento nel quale uno
    dei due corpi rimane fermo e l'altro si muove rispetto a lei.
\end{itemize}
\paragraph{Sistema di riferimento centro di massa}
Nel sistema di riferimento centro di massa si ha che la quantità
di moto è vincolata dal teorema di Konig: dunque si conserva. Dato che
è possibile applicare al conservatività della quantità di moto e dato che
l'urto che supponiamo è elastico, allora si dovrà avere:
\begin{gather*}
    \vv{Q} = 0 \qquad \vv{q_1} = -\vv{q_2}   
\end{gather*}
Inoltre, l'energia cinetica dovrà essere, dato che le singole quantità di moto
si possono esprimere in relazione al sistema di riferimento $S$ come il contributo
del teorema di Konig più il contributo rispetto al sistema di riferimento $S'$:
\begin{gather*}
    \vv{q_1} = \vv{q_1'} + m\vv{v_C} \\
    \vv{q_2} = \vv{q_2'} + m\vv{v_C}      
\end{gather*}
L'energia cinetica si esprime allora come
\begin{gather*}
    K = \frac{1}{2}m_1 v_1^{2} + \frac{1}{2}m_2 v_2^{2}  = \frac{1}{2}(v_1 + v_1')^{2} + \frac{1}{2}(v_2 + v_2')^{2}   
\end{gather*}
Dato che deve risultare per il sistema di punti
\begin{gather*}
    \frac{\sum m_i v_i}{\sum m_i} = \vv{v_C} 
\end{gather*}

\begin{gather*}
    K = \frac{1}{2}(v_1)
\end{gather*}





\part{Gravitazione universale}
\chapter{La gravitazione universale}
\section{La gravitazione}
Dai dati sperimentali e dallo studio dei fenomeni gravitazionali della
totalità del mondo e dei corpi astronomici che lo circondano Keplero formulò
delle leggi empiriche per lo studio dei fenomeni di attrazione gravitazionale portando
anche alla nascita della teoria eliocentrica.
\begin{enumerate}
    \item Orbite ellittiche: I corpi celesti compiono un'orbita assumibile ad un ellisse.
    \item I corpi spazzano aree uguali in tempi uguali: la velocità areolare lungo un orbita di un determinato 
    pianeta è costante e non è uguale per le orbite.
    \item $\frac{\alpha^{3} }{T^{2} } = C$. Ossia il rapporto tra il semiasse maggiore dell'ellisse e
    del periodo di rivoluzione è costante.  
\end{enumerate}
Le leggi impongono dell ipotesi, non che il centro di massa del Sole con
il suo centro geometrico e che le orbite siano piane.
\subsection{La prima di Keplero}
Considerando il momento angolre e preso come polo 
calcolato rispetto al polo (il fuoco dell'ellisse dove sta il sole)
allora si ottiene che questo è costante lungo tutta l'orbita. \\
Si ha allora un moto periodico con 
le masse della Terra e del Sole approssimabili a puntiformi
date le enormi distanze. Inoltre il momento angolare è
costante in direzione poiché l'orbita è regolare.

\subsection{Seconda di Keplero}
\begin{wrapfigure}{r}{0.4\textwidth}
    \centering
    \caption{}
    \begin{tikzpicture}
                        \node[ellipse,
        draw,
	    minimum width = 4cm, 
	    minimum height = 2.2cm] (e) at (0,0) {};
        \filldraw(-1, 0) circle(1pt) node[anchor = south] {$F_1$};
        \filldraw(1, 0) circle(1pt) node[anchor = south] {$F_2$};
        \filldraw(0, 0) circle(1pt);
        \draw[->](-1, 0) -- (0, 1.1) node[midway, right] {$\vv{r}$};
        \draw[->, red] (0, 1.1) -- (-1, 1.1) node[midway, above] {$\vv{v}_P$ };
        \draw(0, 0) arc (0:60:0.4) node[midway, right] {$\phi$};
    \end{tikzpicture}    
\end{wrapfigure}
In coordinate polari si ha la seconda legge di Keplero in modo tale che
\begin{gather*}
    \vv{r} = (P - S) = r\vv{v}_0 \\
    \vv{v} = \dot{\vv{r} } = \dot{r}\hat{u}_m + r\dot{\hat{u} }_n = \dot{r}\hat{u}_n + r\dot{\phi}\hat{u}_{\phi}      
\end{gather*}

\begin{align}
    \vv{L}_{O} = \vv{r} \times m_p\vv{v}(\dot{r}\hat{u}_r + r\dot{\phi} \hat{u}_{\phi} )  = m_P r^{2}\dot{\phi}\hat{k}    
\end{align}
Per uno spostamento infinitesimo $dt$: si può definire
allora la \textbf{velocità areolare} ossia
\begin{align}
    \sigma = \frac{dA}{dt} = \frac{1}{2}r^{2}\frac{d\phi}{dt} = \frac{1}{2}r^{2}\dot{\phi} 
\end{align}
A livello vettoriale la velocità angolare ha la stessa direzione
del vettore momento angolare ottenendo allora la relazione
\begin{gather*}
    \vv{L} = 2m_P \vv{\sigma}  
\end{gather*}
Mettendo allora in relazione le due leggi si ha che si conserva il momento
angolare e dalla seconda legge cardinale si ottiene proprio che
il momento risultante delle forze esterne è zero. \\
Possiamo allora definire l'accelerazione come la derivata della velocità
angolare 
\begin{gather*}
    \vv{a} = \dot{\vv{v} } = (\ddot{r} - r\dot{\phi}^{2} )\hat{u}_r + (2\dot{r}\dot{\phi} + r\ddot{\phi})\vv{\phi}_{\phi}    \\
    \vv{\sigma} = \frac{1}{2}r^{2}\dot{\phi}   
\end{gather*}
Allora si ottiene che 
\begin{gather*}
    \vv{a} = (\ddot{r} - r\dot{\phi}^{2})\hat{u}_r + (\frac{2}{r}\frac{d\phi}{dt})  
\end{gather*}
E allora per la seconda legge si ottiene che la derivata
della velocità areolare è zero e quindi $\vv{a} // \vv{u}_r$.  

\subsection{La terza di Keplero}
Nell'ipotesi di orbita circolare si assume per la maggior parte dei pianeti 
si ha che
\begin{gather*}
    r = \alpha, \qquad \sigma = \frac{1}{2}r^{2}\dot{\phi}, \ddot \dot{\phi} \ const 
\end{gather*}
Allora possiamo definire il coefficiente di eccentricità come
\begin{align}
    \epsilon = \frac{r_A - r_P}{r_a + r_P}, \qquad 0 \leq \epsilon \leq 1
\end{align}
Se l'orbita è ragionevolmente circolare allora possiamo utilizzare le
seguenti approssimazioni:
\begin{gather*}
    \sigma = \frac{1}{2}r^{2}\dot{\phi} \\
    \dot{\phi}^{2} = \frac{4\pi^{2} }{T^{2} } \\
    \dot{\phi}^{2} = \frac{4\pi^{2}C}{r^{3} }   
\end{gather*}
E quindi possiamo definire la forza come
\begin{gather*}
    \vv{F} = -m_P r \dot{\phi}^{2}\hat{u}_r = -m_P 4\pi^{2}C \frac{1}{r^{2} }\hat{u}_R     
\end{gather*}
L'ellisse è il luogo geometrico dei punti la 
cui distanza dei punti su di essa rispetto ai due fuochi è uguale.
e allora si ha che
\begin{gather*}
    r_1 + r_2 = 2\alpha
\end{gather*}


\section{In coordinate polari}
\begin{wrapfigure}{r}{0.4\textwidth}
    \centering
    \caption{L'orbita come luogo geometrico}
    \begin{tikzpicture}
        \node[ellipse,
        draw,
	    minimum width = 4cm, 
	    minimum height = 2.2cm] (e) at (0,0) {};
        \filldraw(0, 0) circle (1pt);
        \filldraw(-1, 0) circle (1pt);
        \filldraw(1, 0) circle(1pt);
        \draw(-1, 0) -- (0.8, 1) node[midway, left] {$r \hat{u}_r$};
        \draw(1, 0) -- (0.8, 1);
        \draw(-0.6, 0) arc(0: 30: 0.4);
    \end{tikzpicture}    
\end{wrapfigure}
Dal teorema di Carnot del coseno si ha che
\begin{gather*}
    PF = \sqrt{r^{2} + 4d^{2} - 4rd\cos\phi } \\
    r + PF = 2\alpha
\end{gather*}
E allora si ottiene che
\begin{gather*}
    r = \frac{\alpha^{2}-d^{2}  }{\alpha -d\cos\phi} = \frac{b^{2} }{\alpha -d\cos\phi}
\end{gather*}
Possiamo allora, definito 
\begin{gather*}
    \phi = \frac{d}{b}, \qquad \rho = \frac{b^{2} }{\alpha} \Rightarrow  r = \frac{\rho}{1 - \epsilon\cos\phi}
\end{gather*}
Dal caso particolare dell'approssimazione "circolare" mi posso portare al 
caso generale attraverso l'accelerazione
\begin{gather*}
    \vv{a} = \ddot{\vv{r} } = (\ddot{r} -r\dot{\phi}^{2} ) \hat{u}_r   
\end{gather*}
utilizzando l'approssimazione di $\dot{\phi} \neq  const$ posso
ottenere l'equazione del raggio come
\begin{gather*}
    \frac{1}{r} = \frac{1}{\rho} - \frac{\epsilon}{\rho} \cos\phi
\end{gather*}
E allora posso dire che che
\begin{gather*}
    \dot{r} = -\frac{2\sigma \epsilon}{\rho} \sin\phi\\
    \ddot{r} = -\frac{2\sigma \epsilon}{\rho} \cos\phi \dot{\phi} = -\frac{4\sigma^{2} \epsilon}{\rho r^{2} }\cos\phi
\end{gather*}
Posso allora esprimere l'accelerazione come
\begin{gather*}
    \ddot{r } - r\dot{\phi}^{2} = -\frac{4\sigma^{2} }{r^{2} }\left(\frac{\epsilon}{\rho}\cos\phi + \frac{1}{r}\right) \Rightarrow \ddot{r} - r\dot{\phi } = -\frac{4\sigma^{2} }{\rho r^{2} }
\end{gather*}
Allora la forza diventa
\begin{gather*}
    \vv{F}_P = m_P \vv{a} = -m_P \frac{4\sigma^{2} }{\rho r^{2} }\hat{u}_r
\end{gather*}

\section{Momento angolare e l'inerzia delle forze}
\begin{wrapfigure}{r}{0.4\textwidth}
    \centering
    \caption{IL momento angolare}
    \begin{tikzpicture}
        \filldraw(0, 0) circle (1pt) node[anchor = south ] {$m_1$};
        \filldraw(1, 2) circle (1pt) node[anchor= west] {$m_2$};
    \end{tikzpicture}    
\end{wrapfigure}
La forza gravitazionale che agisce tra due corpi è data dalla legge di gravitazione
\begin{align}
    f = G\frac{m_1m_2}{r^{2} }
\end{align}
la quale esprime il modulo della forza attrattiva esercitata tra due masse e 
vale per i corpi schematizzabili come puntiformi e anche per corpi sferici. Se i corpi
non fossero schematizzabili a livello puntiforme allora l'attrazione
gravitazionale è calcolata come
\begin{align}
    f = - \int_{m_1} \int_{m_2} G \frac{dm_1dm_2}{r^{3} } \vv{r} 
\end{align}


\section{La forza di gravità di un guscio sferico}
\begin{wrapfigure}{r}{0.5\textwidth}
    \centering
    \caption{}
    \begin{tikzpicture}
        \draw(0, 0) circle (3);
        \draw(0, 0) circle (2.9);
        \filldraw (5, 0) circle (1pt) node[anchor = south] {$m$};
        \draw[->](0, 0) -- (6, 0) node[at end, below] {$x$};
        \draw(0, 0) -- (1.7, 2.45) node[midway, above] {$r$};
        \draw[dashed](1.7, 2.45) -- (5, 0) node[midway, above] {$\rho$};
        \draw(1, 0) arc (0:55:1) node[midway, right] {$\theta$};
        \filldraw(0, 0) circle(1pt) node[anchor = east] {$C$};
        \draw[|-|](-3, -0.75) -- (5, -0.75) node[midway, below] {$r_B$};
        \draw[|-|](3, -0.25) -- (5, -0.25) node[midway, below] {$r_A$};
    \end{tikzpicture}    
\end{wrapfigure}
Per trovare la forza di gravità agente su di un punto esterno
od interno rispetto ad un guscio sferico di spessore infinitesimo si ottiene
considerando la densità superficiale $\sigma$:
\begin{gather*}
    \sigma = \frac{M}{4\pi r^{2} }
\end{gather*} 
Data solo la legge di gravitazione universale dobbiamo allora
ridurre il problema e cercare di ridurre il problema ad una situazione in cui
è facile integrare e quindi scelgo un punto sulla superficie che per definizione avrà
distanza $r$ dal centro e quindi data l'energia potenziale del punto $m$, 
\begin{gather*}
    V = -\frac{Gm_1m_2}{r_{1, 2}}
\end{gather*}
Tutti i punti che hanno la stessa distanza dal punto $P$ contribuiranno allo stesso modo all'energia potenziale:
quindi danno un contributo all'energia potenziale pari a:
\begin{gather*}
    dV = -\frac{Gm \ dM}{\rho}, \qquad dM = \sigma \ dS, \qquad dS = 2\pi r \sin\theta \cdot  r d\theta
\end{gather*}
Posso approssimare lo spessore come un cilindretto e quindi posso calcolare 
la superficie. Introducendo l'angolo $\theta$ posso allora definire il differenziale
della massa come
\begin{gather*}
    d M = 2\sigma r^{2} \sin\theta \ d\theta \Rightarrow d M = \frac{M}{2} \cdot  \sin\theta \ d\theta
\end{gather*}
In pratica il $d\theta$ è esattamente lo spostamento infinitesimo del punto sulla superficie e quindi
ottengo il volume del cilindretto con altezza $r \ d\theta$. Posso allora dire che
l'energia potenziale di ogni punto sulla stessa circonferenza sarà:
\begin{gather*}
    d V = - G m \frac{M}{2} \frac{\sin\theta \ d\theta}{\rho}, \rho^{2} = r^{2} + x^{2} - 2rx\cos\theta   
\end{gather*}
Dato che $x$ è fissato, allora se differenziamo l'espressione di $\rho$ viene che 
\begin{gather*}
    2\rho \ d\rho = 2 r x \sin\theta \ d\theta 
\end{gather*}
Si ottiene allora che 
\begin{gather*}
    \frac{d\rho}{rx} = \frac{\sin\theta \ d\theta}{\rho}
\end{gather*}
Se ora facessi l'integrale dell'energia potenziale posso allora ottenere che:
\begin{gather*}
    V = -\frac{G m M}{2rx} \int d\rho = -\frac{G m M}{2rx} (r_B - r_A)
\end{gather*}
Si possono distinguere i due casi in cui il punto materiale è esterno rispetto alla superficie ma anche
quando $P'$ è interno: si danno allora due casi: un caso in cui $P$ sia esterno 
alla superficie:
\begin{gather*}
    \text{Esterno}: \left\{\begin{array}{l}
        r_A = x - r \\
        r_B = x + r
    \end{array}\right. \\
    \text{Interno}: \left\{\begin{array}{l}
        r_A = r - x \\
        r_B = x + r
    \end{array}\right.
\end{gather*}
Per un punto esterno alla superficie allora si può dire che
\begin{gather*}
    V = -\frac{G m M}{2rx} 2r = -\frac{G m M}{x}
\end{gather*}
E questa è esattamente l'energia potenziale che risente un oggetto di 
massa $m$ da un oggetto di massa $M$ che si trova esattamente sul punto $C$. Se quindi
è esterno alla superficie allora questa superficie posso considerarla come se fosse
concentrata nel suo centro.  Per un punto interno:
\begin{gather*}
    V = -\frac{G m M}{2rx} 2x = -\frac{G m M}{r}
\end{gather*}
In questo caso è posta alla distanza $r$, ossia come se il punto fosse sulla superficie
e non al suo interno. La forza gravitazionale che risente dunque al suo interno
è nulla.
\section{La gravità di una sfera piena}
\begin{wrapfigure}{r}{0.4\textwidth}
    \centering
    \caption{}
    \begin{tikzpicture}
        \draw(0, 0) circle (2);
        \draw(1, 1.68) -- (1, -1.68);
        \draw(1.2, 1.55) -- (1.2, -1.55);
        \draw(1.1, 1.65) circle (1pt) node[anchor = west] {$m$};
        \draw(1.1, 1) circle (1pt);
        \draw[->](1.1, 1) -- (1.6, 1) node[at end, above] {$\vv{N} $};
        \draw[->](0, 0) -- (0.5, 0.45) node[at end, below] {$\hat{u}_r$};
        \draw[dashed] (0, 0) -- (1.1, 1);
        \draw[->](1.1, 1) -- (0.6, 0.55) node[midway , above] {$\vv{F_g}$};
        \draw[dashed](0, 0) circle(1.45);
    \end{tikzpicture}    
\end{wrapfigure}
Divido allora in tante bucce ma ora tutte le bucce si comportano 
in modo diverso tra di loro: quelle vicine alla massas non contribuiscono 
mentre quelle più interne contribuiscono alla forza gravitazionale. La forza è dovuta
solo alla massa che si trova allora all'interno.  \\
All'interno non agisce tutta la sfera ma solo la parte della massa interna:
\begin{gather*}
    \vv{F}_g = - \frac{G m M_i}{r^{2} } \hat{u}_r, \qquad M_i = \frac{4}{3}\pi r^{3} \rho 
\end{gather*}
E quindi il rapporto tra la massa interna e quella della Terra è
\begin{gather*}
    \frac{M_i}{M} = \frac{r^{3} }{R^{3} }
\end{gather*}
Posso allora dire che che essendo un punto materiale in una guida:
\begin{gather*}
    \left\{\begin{array}{l}
        \vv{F}_g = - \frac{G m M_T}{R^{3} } (x \hat{i} + y \hat{j}  ) \\
        \vv{N} = N \hat{i}  
    \end{array}\right. \\
    \left\{\begin{array}{l}
        x) N - \frac{G m M_T}{R^{3} }x = 0 \\
        y) - \frac{G m M_T}{R^{3} }y = m\ddot{y}
    \end{array}\right.
\end{gather*}
Allora l'equazione che abbiamo per il corpo è proprio un moto armonico:
\begin{gather*}
    \ddot{y} + \frac{g}{R}y = 0
\end{gather*}
Questo si chiama pozzo di Gauss e si vede che la massa esterna non conta.

\section{Il moto dei razzi}
\begin{wrapfigure}{r}{0.4\textwidth}
    \centering
    \caption{}
    \begin{tikzpicture}
        \draw(0, 0) rectangle (2, 1);
        \draw[->](0, 0.5) -- (-1, 0.5) node[at end, above] {$\vv{u}$};
    \end{tikzpicture}    
\end{wrapfigure}
Con gli strumenti sviluppati fino ad ora si è in grado di descrivere il moto di un razzo:
In un dato istante di tempo il razzo espelle una certa quantità di massa
che genera una spinta. 
\begin{gather*}
    Q(t) = m(t) \cdot \vv{v}(t)  \\
    \vv{Q} (t + dt) = M(t + dt) \vv{v}(t + dt) + dm (\vv{u} + \vv{v}(t + dt) )  
\end{gather*}
Dove $\vv{u}$ è il vettore della velocità di espulsione del gas rispetto al razzo. Esplicitandola
di nuovo posso allora dire che rispetto al missile. 
\begin{gather*}
    \Delta \vv{Q} = \vv{Q}(t + dt) - \vv{Q}(t) = \vv{F}^{ext} \ dt    
\end{gather*} 
E quindi risolvendo si ottiene dato che $dM = -dm$:
\begin{gather*}
    \vv{F}^{ext}  \ dt = M d\vv{v} + dm \vv{u}   
\end{gather*}
Dividendo ora tutto per $dt$:
\begin{gather*}
    M\frac{d\vv{v} }{d t} = \vv{u}\frac{dM}{dt} + \vv{F}^{ext}   
\end{gather*}
Nel caso in cui il missile sia nello spazio e quindi che le forze esterne siano
nulle la nostra espressione ci dice che
\begin{gather*}
    M dv = u dM \Rightarrow \int_{t_i}^{t_f} dv = u \int_{t_i}^{t_f}\frac{dM}{M}
\end{gather*}
Allora
\begin{gather*}
    v(t_f) = v(t_i) - u\ln\frac{M(t_i)}{M(t_f)} \\
    v_f = v_i + |u| \ln \left(\frac{M_{mis} + M_{carb}(t_i)}{M_{mis} + M_{carb}(t_f)}\right)
\end{gather*}

Nel secondo caso si ha la partenza da Terra verso l'alto:
\begin{gather*}
    \vv{F}^{ext} = M\vv{g}   
\end{gather*}
Possiamo usare sempre gli stessi principi:
\begin{gather*}
    M \ dv = u \ dM - Mg \ dt \\
    dv = u \frac{dM}{M} - g \ dt \\
    \int_{t_i}^{t_f} dv = v(t_f) = \int_{t_i}^{t_f} \left(u \frac{dM}{M} - g \ dt \right)
\end{gather*}
Allora si ottiene che
\begin{gather*}
    v(t_f) = |u| \ln \frac{M_i}{M_f} - g(t_f - t_i)
\end{gather*}
Più carburante brucio e meno importante diventa l'ultimo termine in modo
tale da poter approssimare alla forma vista nel primo caso. Per avere più spinta ci conviene
avere più velocità di espulsione possibile.

























\appendix
\chapter{Esercizi}
\section{Esercizio su moodle (DPM)}
\begin{wrapfigure}{r}{0.4\textwidth}
    \centering
    \label{Figsnd0i}
    \caption{culo}
    \begin{tikzpicture}
        \draw(0, 0) -- (4, 0);
        \draw(2.5, 0) rectangle (3, 0.5) node[midway] {$B$};
        \draw(2.75, 0.5) -- (2.75, 2) node[midway, right] {$l$};
        \draw(2.75, 2) -- (1.25, 1) node[midway, above] {$l$};
        \filldraw(1.25, 1) circle (1pt) node[anchor= east] {$A$};
        \draw[->](3.5, 0) -- (4, 0) node[at end, below] {$x$};
        \draw[->](3.5, 0) -- (3.5, 0.5) node[at end, left] {$y$};
        \draw[->, very thick, red](1.25, 1) -- (2, 1.5) node[midway, above] {$\vv{T}_A$};
        \draw[->, very thick, red](1.25, 1) -- (1.25, 0.25) node[at end, right] {$m_A\vv{g}$};
        \draw[->, very thick, red](2.75, 0) -- (2.75, -0.5) node[at end, right] {$m_B\vv{g}$};
        \draw[->, very thick, red](2.75, 0.5) -- (2.75, 1) node[at end, right] {$m_B\vv{g}$};
    \end{tikzpicture}    
\end{wrapfigure}
Presi due punti materiali e con l'approssimazione
che il filo è teso e che i due oggetti appesi sono due
punti materiali, si assume anche che B sia in quiete e che
A parta da fermo quando $\theta = \frac{\pi}{2}$.\\
Posto il sistema di riferimento, allora posso misurare l'accelerazione
con un sistema di riferimento ideale e il filo inizialmente
uguale e potrebbe scorrere (si assume quindi un solo grado di libertà 
dicendo che B è in quitee e che il filo non scorra).
Studiando B si ottiene la relazione:
\begin{gather*}
     \vv{T}_B  + \vv{N} + m_b \vv{g} = 0   
\end{gather*}
Studiano invece A, il quale si muove all'istante zero e quindi
posso dier:
\begin{gather*}
    m_A\vv{a}_A = m_A \vv{g} + \vv{T}_A   
\end{gather*}
Ora dato che A si muove, la sua traiettoria sarà una parabola 
e quindi posso dire dall'equazione vettoriale devo capire quando si alza:
con la schematizzazione del problema posso dire che B si alza quando A inizia
a muoversi, definiti allora i versori di A, posso dire che la proiezione
lungo $\hat{n}$ è dato da: 
\begin{gather*}
    -m_A l \dot{\theta}^{2} = m_A g \cos\theta - T_A 
\end{gather*}
ALlora posso dire che:
\begin{gather*}
    T_A = m_A g \cos\theta + m_A l \dot{\theta}^{2} 
\end{gather*}
E quindi
\begin{gather*}
    N = m_B g - T_B \geq 0
\end{gather*}
E allora posso riscrovere, dato che la velocità di A posso esprimerla
come $l \dot{\theta}$,
\begin{gather*}
    m_A(g \cos\theta + \frac{v_A^{2} }{l}) \leq m_B g 
\end{gather*}
DAto che tutte le forze sono conservative, allora posso utilizzare
la conservazione dell'energia meccanica anche perché c'è solo un vincolo:
\begin{gather*}
    E = \frac{1}{2}m_A v_A^{2} + m_A gy_A 
\end{gather*} 
Allora essendo che tutta questa per deginizione è una costatne,
allora devo eguagliarla all'energia al tempo zero:
\begin{gather*}
    E_i  = E(v_A = 0, y_A = l) \Rightarrow = m_A g l \\
\end{gather*}
Imponendo le due uguaglianze ottengo:
\begin{gather*}
    v_A^{2} = 2g(l -y_A) 
\end{gather*}
Sostituendo la velocità nella formula trovata prima posso ottenere,
posto
\begin{gather*}
    \Delta y = l - y_A = l\cos\theta \geq 0
\end{gather*}
Da qui trovo il coseno di teta e quindi Sostituendo tutto 
nell'espressione sopra
\begin{gather*}
    m_A(\frac{\Delta y}{l} + 2\frac{\Delta y}{l}) \leq m_B
\end{gather*}
Allora ottengo l'espressione per il delta y come:
\begin{gather*}
    \Delta y_{max} = \frac{m_B l}{3m_A}
\end{gather*}
Ossia il valore massimo oltre il quale la masssa $B$ inizia
ad alzarsi e quindi fa scorrere il filo. 

\subsection{Esercizio di esame febbraio 2021}
\begin{wrapfigure}{r}{0.4\textwidth}
    \centering
    \label{asd}
    \caption{IMMAGINE IDS DS MDOOODLE DE D}
    \begin{tikzpicture}
        
    \end{tikzpicture}    
\end{wrapfigure}
DAto il testo (che si spera l'utente abbia letto) si fanno le seguenti
assunzioni: essendo la guida fissa, essa è descritta dall'equazione
$y = kx^{2}$, avendo a che fare con un punto materiale non essendoci
attrito, ho anche a che fare con un vincolo bilatero, e ho un SDR inerziale
poiché la guida non si sposta ed è fissa. Quello che posso fare è
scrivere le forze in gioco e la reazione vincolare della guida. Per
ipotesi di mancanza di attrito so anche che $N$ è ortogonale alla
traiettoria. Tutte le forze sono conservative e quindi l'energia meccanic
si conserva ed il vincolo è bilatero 
\begin{gather*}
    E = \frac{1}{2}mv^{2} - mgy =  const = 0. 
\end{gather*} 
Allora posso ricavare la velocità:
\begin{gather*}
    v = \sqrt{2gy} 
\end{gather*}
Essendo $y = kx^{2}$, la $y$ finale, ossia quella priam che tocchi terra
è qurlla della x massima e quindi la $x_{MAX}$ è semplicemente L e allora 
\begin{gather*}
    v_F = \sqrt{2gkL^{2} } 
\end{gather*} 
PUNTO B: \\
\begin{wrapfigure}{r}{0.4\textwidth}
    \centering
    \label{dfs}
    \caption{fdas}
    \begin{tikzpicture}
        
    \end{tikzpicture}    
\end{wrapfigure}
Mentre $\vv{N}_x$ è concorde con l'asse di riferimento, allora
il segno di $\vv{N}_y$ non lo è e per come abbiamo scelto il sistema
di riferimento non è possibile che sia concorde.
DAto che la guida può solo spingere ma non tirare, allora
\begin{gather*}
    \vv{N}_x, \vv{N}_y \geq 0  
\end{gather*}
Ora
\begin{gather*}
    \vv{N} + m\vv{g} = m\vv{a}   
\end{gather*}
Che posso proiettare lungo la direzione $x$ ed $y$ ed ottengo
che 
\begin{gather*}
    N_x = m\dot{x} \\
    mg - m\dot{y} = N_y
\end{gather*}
DAto che si hanno due gradi di libertà, dobbiamo utilizzare delle
approssimazioni: fin tanto che il corpo è attaccato alla guida allora
questo soddisfa le equazioni della guida:
\begin{gather*}
    \vv{r} = P - O = x\hat{i} + kx^{2}\hat{j}    
\end{gather*}
Derivando, si ottiene l'eqauzione della velocità:
\begin{gather*}
    \vv{v} = \dot{x}\hat{i} + 2kx\dot{x} \hat{j}   
\end{gather*}
Derivando ancora:
\begin{gather*}
    \vv{a} = \ddot{x}\hat{i} + 2k(\dot{x}^{2} + x\ddot{x})\hat{j}  
\end{gather*}
Il secondo metodo è utilizzare l'energia meccanica:
\begin{gather*}
    v^{2}  =2gy  
\end{gather*}
Sostituendo la relaszione ottenuta preimac di v:
\begin{gather*}
    v^{2} = \dot{x}^{2}(1 + 4k^{2}x^{2}) = 2gkx^{2}   
\end{gather*}
Allora:
\begin{gather*}
    \dot{x}^{2} = \frac{2gkx^{2} }{1 + 4k^{2}x^{2}  } 
\end{gather*}
METODO ALTERNATIVO: \\
COnsiderato che $\vv{N}_y$ e $\vv{N}_x$ non sono più generici
e io so che $N$ deve essere ortogonale alla traiettoria:
allora se vado a disegnare la traiettoria, avrò un versore tangente
ed un versore invece ortogonale ad $\hat{u}_t$:
\begin{gather*}
    \vv{N} \cdot  \hat{u}_C = 0 \\
    \vv{N} \cdot \vv{v} = 0 \\     
\end{gather*}   
Sostituendo quelle equzioni della velocità allora posso ottenere:
\begin{gather*}
    N_x \dot{x} - N_y2kx\dot{x} = \dot{x}(N_x - 2kxN_y)
\end{gather*}
Ci sono allora due soluzioni per questa equazione:
\begin{gather*}
    N_x = 2kxN_y \quad e \quad x = 0
\end{gather*}
Da l'equazione:
\begin{gather*}
    \left\{\begin{array}{l}
        N_x = m\ddot{x} \\
        N_Y = m(g - \ddot{y})
    \end{array}\right. \\
    \ddot{x} = \frac{N_x}{m} = \frac{2kxN_y}{m}
\end{gather*}
Sostituendo all'interno della equzione per $N_y$ allora:
\begin{gather*}
    mg - N_y = m2k(\dot{x} + \frac{2kxN_y}{m})
\end{gather*}
SO anche che:
\begin{gather*}
    \dot{x}^{2} = \frac{2gkx^{2} }{1 + 4k^{2}x^{2}  } 
\end{gather*}
ALlora sostituendo e raccogliendo $N_y$:
\begin{gather*}
    N_y(1  + 4k^{2}x^{2} ) = mg(1 - \frac{4k^{2} x^{2} }{1 + 4k^{2} x^{2}  })
\end{gather*}
E allora:
\begin{gather*}
    N_y = \frac{mg}{(1 + 4k^{2}x^{2} )^{2} }
\end{gather*}
E allora non si stacca mai dalla guida. 

\section{Esercizio esame novembre 24}
\begin{wrapfigure}{r}{0.4\textwidth}
    \centering
    \caption{Schematizzazione del problema}
    \begin{tikzpicture}
        \draw(0, 0) circle(3);
        \draw[dashed, very thin](-4, 0) -- (4, 0) node[at end, below] {$x$};
        \draw[dashed, very thin](0, -4) -- (0, 4) node[at end, left] {$y$};
        \filldraw (0, 2) circle(1pt) node[anchor = east] {$P$};
    \end{tikzpicture}    
\end{wrapfigure}
Sul filo agiscono diverse forze ma non essendoci la gravità,
la forza che la rimpiazza è quella centrifuga e tende a spostarlo
verso una direzione (è radiale). Le ipotesi sono:
\begin{enumerate}
    \item La piattaforma ruota con velocità $\omega_0$ costante (c'è un motore esterno che tiene in rotazione il disco).
    \item Il filo del pendolo è ideale;
    \item Si lavora con la schematizzazione del punto materiale;
    \item SIstema non inerziale $O_{xy}$.
\end{enumerate}
Punto a: \\
Per calcolare la distanza si può scrivere per le coordinate:
\begin{gather*}
    \left\{\begin{array}{l}
        x_P = l\sin\phi \\
        y_P = a + l\sin\phi
    \end{array}\right.
\end{gather*}
Adesso per calcolare la distanza dal centro $O$ basterà calcolare la distanza
dal filo come:
\begin{gather*}
    |P - O|^{2} = x_P^{2} + y_P^{2} = l^{2}\sin^{2}\phi + a^{2} + l^{2}\cos^{2}\phi + 2al\cos\phi        
\end{gather*}
E quindi
\begin{gather*}
    |P - O| = \sqrt{l^{2} + a^{2} +2al\cos\phi} 
\end{gather*}
PUNTO B: \\
Sul punto P ci sono diverse forze: prima di tutto la forza di trascinamento verso l'esterno,
la forza di Coriolis (che non so dove agisce) e la forza centrifuga. Di queste forze quella che compie
lavoro è quella di trascinamento mentre quelle che non compiono lavoro sono la forza di Coriolis e la tensione (poiché
è sempre ortogonale alla direzione tangente). \\
Il lavoro della forza di trascinamento è data da:
\begin{gather*}
    V = -\frac{1}{2}m\omega^{2}_0 |P - O|^{2}   = -\frac{1}{2}m\omega_0^{2}(l^{2} + a^{2} + 2al\cos\phi) 
\end{gather*}
I termini sono quasi tutti costanti, possiamo allora considerare solo
il coseno e riassumere quindi l'energia potenziale come una costante meno
il pezzo non costante:
\begin{gather*}
    V = cost - m\omega_0^{2}al\cos\phi 
\end{gather*}
Derivando si ottiene proprio l'espressione:
\begin{gather*}
    V' = -m\omega_0^{2}al(-sin\phi) = 0 
\end{gather*}
Le cui soluzioni per ottenere l'equilibrio sono proprio $0, \pi$. Verifichiamo
allora la stabilità di questo equilibrio, per cui la derivata seconda sarà proprio:
\begin{gather*}
    V''= m\omega_0^{2}al\cos\phi  \Rightarrow  \left\{\begin{array}{l}
        > 0, \quad \phi = 0 \text{ stabile}\\
        < 0, \quad \phi = \pi \text{ instabile}
    \end{array}\right.
\end{gather*}

PUNTO C: \\
L'energia non è conservata in un sistema di riferimento esterno a causa
del motore del disco che lo tiene in rotazione mentre si conserva nel sistema
di riferimento ruotante $O_{x, y}$. Il ruolo del motore allora è descritto
con la forza di trascinamento. Possiamo allora scrivere l'energia nel sistema di
riferimento ruotante con un termine cinetico e uno potenziale:
\begin{gather*}
    \left\{\begin{array}{l}
        \dot{x}_P = l\cos\phi \cdot  \dot{phi} \\
        \dot{y}_P = -l\sin\phi \cdot  \dot{phi}
    \end{array}\right.
\end{gather*} 
Da qui allora si ha che 
\begin{gather*}
    \dot{V}_P^{2} = \dot{x}_P^{2} + \dot{y}_P^{2}  = l^{2}\dot{\phi}^{2}  
\end{gather*}
Troviamo allora il classico risultato del moto circolare e l'energia allora è ora:
\begin{gather*}
    E = \frac{1}{2}ml^{2}\dot{\phi}^{2} - m\omega_0^{2}al\cos\phi   
\end{gather*}
Con la costante dell'energia potenziale che posso liberamente mettere uguale 
a zero poiché l'energia meccanica è definita a meno di una costante. Con il metodo
dell'energia allora posso derivare l'energia ottenendo:
\begin{gather*}
    \dot{E} = 0 = \frac{1}{2}ml^{2}2\ddot{\phi}\dot{\phi} - m\omega^{2}_0al(-\sin\phi)\dot{\phi} = 0   
\end{gather*}
Semplificando si ottiene la seguente equazione di moto:
\begin{gather*}
    l\ddot{\phi} + \omega_0^{2} a\sin\phi = 0
\end{gather*}
Diventa allora l'equazione del pendolo senza la forza peso
ma con la forza centrifuga. Nel limite allora delle piccole oscillazioni 
sappiamo allora che $\phi$ è vicino a quella di equilibrio. 
\begin{gather*}
    \ddot{\phi} + \Omega^{2}\phi = 0 , \quad \Omega^{2} = \frac{\omega_0^{2} a}{l} 
\end{gather*}
Allora il periodo delle oscillazioni si ha che è:
\begin{gather*}
    T = \frac{2\pi}{\Omega}, \Omega = \sqrt{\frac{l}{a}} 
\end{gather*}
PUNTO D: \\
Senza le approssimazioni delle piccole oscillazioni dobbiamo 
prima studiare il moto e trovare la tensione del filo. Si sa da Newton che:
\begin{gather*}
    m\vv{a} = \vv{T} + \vv{F}_{co} + \vv{F}_T    
\end{gather*}
Allora queste si esprimono come:
\begin{align*}
    \vv{F}_{co} &= -2m\omega_0 \times \vv{v}  \\
    &=-2m\omega_0l\dot{\phi}\hat{u}_n  
\end{align*}
Dato che la tensione è opposta a $\hat{u}_n$ si ha:
\begin{gather*}
    \vv{T} = -T\hat{u}_n   
\end{gather*} 
E la trascinamento:
\begin{gather*}
    \vv{F}_T = m\omega_0^{2}(P - O)  
\end{gather*} 
Proietto allora lungo la direzione N tutte le forze ottenendo che:
posso scrivere intanto la tensione:
\begin{gather*}
    \vv{T} = m\vv{a} - \vv{F}_{co} - \vv{F}_T    \\
    \vv{T} = - \hat{u}_n \vv{T} = -(m\vv{a}-\vv{F}_{co} - \vv{F}_T) \hat{u}_n      
\end{gather*} 
Allora scriviamo i vettori che compaiono come proiezioni sulla direzione n:
\begin{gather*}
    \vv{a}\cdot \hat{u}_n = -l\dot{\phi}^{2} \text{ centripeta} \\
    \vv{F}_T \cdot  \hat{u}_n = m\omega_0^{2} (P - O)\cdot \hat{u}_n      
\end{gather*}
Possiamo esprimere il versore lungo n come:
\begin{gather*}
    \hat{u}_n = \sin\phi\hat{i} + \cos\phi\hat{j}   
\end{gather*}
E si ottiene allora la forza di trascinamento come:
\begin{gather*}
    \vv{F}_T = m\omega_0^{2}(x_P\sin\phi + y_P\cos\phi)  
\end{gather*}
Allora date le sostituzioni delle coordinate di P, possiamo
allora dire che:
\begin{gather*}
    m\omega_0^{2}(l + a\cos\phi) 
\end{gather*}
Adesso la proiezione della forza di coriolis è proprio:
\begin{gather*}
    \vv{F}_{co} -2m\omega_0l\dot{\phi}
\end{gather*}
Sostituendo tutte le espressioni trovate ora nell'espressione
per la tensione si ottiene allora:
\begin{gather*}
    T = ml\dot{\phi}^{2}-2m\omega_0l\dot{\phi} - 2m\omega_0 l \dot{\phi} 
\end{gather*}
Dobbiamo allora trovare $\dot{\phi}$ per poter trovare l'espressione per la tensione.
L'energia la abbiamo scritta prima come:
\begin{gather*}
    E = \frac{1}{2}ml^{2}\dot{\phi}^{2} - m\omega_0^{2}al\cos\phi   
\end{gather*}
Parte da fermo $(\dot{\phi} = 0)$ e quindi:
\begin{gather*}
    E_0 = -m\omega _0^{2}al\cos\phi_0 
\end{gather*}
E quindi portando dall'altra parte possiamo ottenere:
\begin{gather*}
    \frac{1}{2}ml^{2} \dot{\phi}^{2} = -m\omega_0^{2}al(\cos\phi_= - \cos\phi)  \\
    \dot{\phi}^{2} = \frac{2\omega_0^{2}a}{l} (\cos\phi - \cos\phi_0) \geq 0 
\end{gather*} 
Dato che l'espressione ha senso se e solo se è positiva, allora devo (dato $\phi_0 = \frac{\pi}{6}$) che
\begin{gather*}
    -\phi_0 \leq \phi \leq \phi_0
\end{gather*}
Anche senza le piccole oscillazioni allora so che il pendolo può oscillare (in 
questa situazione) solo tra quei due valori. ALlora $\dot{\phi}$ diventa:
\begin{gather*}
    \dot{\phi} = -\sqrt{\frac{2\omega_0^{2} a}{l}(\cos\phi - \cos\phi_0)} 
\end{gather*}
Il segno meno è dovuto poiché nel primo periodo $\phi$ diminuisce 
e quindi la tensione diventa:
\begin{gather*}
    T = ml\left(\frac{2\omega_0^{2} a}{l}(\cos\phi - \cos\phi_0) - 2\omega_0 \cdot  \left( -\sqrt{\frac{2\omega_0^{2} a}{l}(\cos\phi - \cos\phi_0)} \right) + \frac{\omega_0^{2}}{l}(l + a \cos\phi)\right)
\end{gather*}
Raccogliendo allora:
\begin{gather*}
    T = ml\omega_0^{2}\left(\frac{3a}{l}\cos\phi - \frac{2a}{l}\cos\phi_0 + 1 + 2\sqrt{\frac{2a}{l}(\cos\phi - \cos\phi_0)} \right) 
\end{gather*}


\end{document}
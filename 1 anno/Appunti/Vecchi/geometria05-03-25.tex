\documentclass[a4paper, oneside]{article}
\usepackage{graphicx}
\usepackage{amsthm}
\usepackage{amsmath}
\usepackage[a4paper,
            bindingoffset=0.2in,
            left=2cm,
            right=2cm,
            top=2cm,
            bottom=2cm,
            footskip=.25in]{geometry}
\usepackage[italian]{babel}
\usepackage{pgfplots}
\usepackage{tabularx}
\usepackage{wrapfig}
\graphicspath{ {./images/} }
\usetikzlibrary{datavisualization}
\usetikzlibrary{datavisualization.formats.functions}
\pgfplotsset{width=10cm,compat=1.9}

\title{Geoemtria rubei}
\author{Tommaso Miliani}
\date{04-03-25}

\begin{document}
\theoremstyle{definition}
\theoremstyle{theorem}
\theoremstyle{lemma}

\newtheorem{definition}{Definizione}[section]
\newtheorem{theorem}{Teorema}[section]
\newtheorem{lemma}{Proposizione}[theorem]
\newtheorem{example}{Esempio}[section]

\maketitle

\section{ALtri affini}
In questa sezione si elencheranno altre proprietà degli spazi affini
\begin{definition}
    Due sottospazi affini di un campo $ K^{n}$ si dicono \textbf{paralleli}
    se uno dei due ha la direzione contenuta nella direzione dell'altro.
\end{definition}
\begin{definition}
    Sia Z un sottospazio di $R^{n}$. Definiamo allora:
    \begin{align}
        Z^{\perp} = \{v \in R^{n} : (v, z) = 0, \forall z \in Z\} 
    \end{align} 
    dove $(v, z)$ denota il prodotto scalare standard di $v, z$.
\end{definition}
\begin{lemma}
    Sia $Z$ un sottospazio vettoriale di $R^{n}$. Allora $Z^{\perp}$ è un sottospazio
    di $R^{n}$ e si ha:
    \begin{enumerate}
        \item \begin{align}
            \dim(Z^{\perp} ) = n - \dim(Z);
        \end{align}
        \item   \begin{align}
            R^{n} = Z sm Z^{\perp}  
        \end{align}
        \item  \begin{align}
            (Z^{\perp})^{\perp}   = Z;
        \end{align}
        \item \begin{align}
            W \subset Z, Z^{\perp} \subset W^{\perp}, W \in R^{n}   
        \end{align}
    \end{enumerate}   
\end{lemma}
\begin{proof}
    TODO
\end{proof}

\begin{definition}
    Siano $S_1$ e $S_2$ due sottospazi affini di $R^{n}$ siano allora
    \begin{gather*}
        z_1 = \dim(S_1) \qquad z_" = \dim(S_2)
    \end{gather*} 
    Dico che $S_1$ e $S_2$ sono perpendicolari se valgono le seguenti condizioni
    \begin{align}
        Z_1 \subset Z_2^{\perp} \qquad se \qquad \dim(S_1) + \dim(S_2) \leq n \\
        Z_1^{\perp}  \subset Z_2 \qquad se \qquad \dim(S_1) + \dim(S_2) \geq n
    \end{align}
\end{definition}
Esempio
\begin{wrapfigure}{r}{0.4\textwidth}
    \centering
    \label{Fig esempio}
    \caption{GLi spazi dio bono}
    \begin{tikzpicture}
        \draw[->](0, 0) -- (2, 0);
        \draw[->](0, 0) -- (-0.5, -1);
        \draw[->](0, 0) -- (0, 2);
        \filldraw(1, -0.5) node[anchor = east] {$\Pi_1$};
        \filldraw(-0.5, 1) node[anchor = east] {$\Pi_2$};
    \end{tikzpicture}    
\end{wrapfigure}
\begin{example}
Siano $\Pi_1 = \left< e_1, e_2 \right>$ e $\Pi_2 = \left< e_2, e_3 \right> \in R^{3}$
sono vettoriali e quindi sono anche affini, essendo che la loro somma però è maggiore
della dimensione di R, allora devo apllicare le condizioni e quindi ottengo che
\begin{gather*}
    Z_1^{\perp} = \{x \in R^{3} : (x, v) = 0 \forall v \in <e_1, e_2>\} = <e_3>  
\end{gather*}
Allora si ottiene che:
\begin{gather*}
    Z_1^{\perp} \subset Z_2 
\end{gather*}
QUindi sono perpendicolari.
\end{example}


\begin{lemma}
    Supponendo che $\dim(Z_1) + \dim(Z_2) = n$, allora 
    \begin{align}
        \dim(Z_2^{\perp} ) = n -\dim(Z_2)  =\dim(Z)
    \end{align}
    e quindi 
    \begin{align}
        Z_1 = Z_2^{\perp}, Z_1^{\perp} \subset Z_2  
    \end{align}
    Quindi se $\dim(Z_1) + \dim(Z_2) = n$ allora se vale una delle
    due vale anche l'altra.
\end{lemma}

\begin{lemma}
    Se la nostra direzione è simmetrica con $S_1, S_2$ allora se
    $Z_1 \subset  Z_2^{\perp}$ allora $(Z_2^{\perp} )^{\perp} \subset Z_1^{\perp}$
    e quindi $Z_2 \subset  Z_1^{\perp}$.      
\end{lemma}

\begin{definition}[Proiezione ortogonale di un punto su di un sottospazio affine]
    SIa $S$ un sottospazio affine di $R^{n}$ e sia $P$ un punto che
    $\in R^{n}$, definisco la proiezione ortogonale di P su di S e la definsico come
    l'unico punto di intersezione tra S ed il sottospazio affine 
    \begin{align}
        P + (dir(S))^{\perp} 
    \end{align}  
\end{definition}
\begin{lemma}
    \begin{align}
        S \cap (P + dir(S)^{\perp} ) 
    \end{align}
    è formato da un solo punto.
\end{lemma}
\begin{proof}
    Posto $S = Q + Z$, dove $Q$ è un punto in $R^{n}$ e $Z$ 
    un sottospazio affine di $R^{n}$ allora considero 
    \begin{gather*}
        S' = P + Z^{\perp} 
    \end{gather*}  e voglio dimostrare che
    \begin{gather*}
        S \cap S'
    \end{gather*}
    E' dato da un punto solo e quindi che $z \in Z, w \in Z^{\perp}$
    esistono unici e quindi
    \begin{gather*}
        Q + z = P + w.
    \end{gather*} 
    ossia
    \begin{gather*}
        Q - P = w -z
    \end{gather*}
    Il primo membro $\in R^{n}$ e quindi questo è vero perché
    abbiamo dimostrato che $R^{n}$ è la somma diretta di $Z + Z^{\perp}$
    e quindi ogni elemento è possibile scriverlo come somma diretta di elementi
    unici.    
\end{proof}

\begin{definition}
    Siano $P$ e $Q$ due punti di $R^{n}$ definisco allora la loro
    distanza come la norma ossia il numero reale:
    \begin{align}
        d(P, Q) = |P - Q|
    \end{align} 
\end{definition}
\begin{definition}
    Siano $A, B \subset R^{n}$, si definisce la distanza tra due insiemi
    come :
    \begin{align}
        d(A, B) = \inf\{d(P, Q) : P \in A, Q \in B\}
    \end{align}
    Ossia l'estremo inferiore di quella distanza.
\end{definition}

\end{document}
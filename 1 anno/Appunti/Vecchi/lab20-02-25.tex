\documentclass[a4paper, oneside]{article}
\usepackage{graphicx}
\usepackage{amsthm}
\usepackage{amsmath}
\usepackage[a4paper,
            bindingoffset=0.2in,
            left=2cm,
            right=2cm,
            top=2cm,
            bottom=2cm,
            footskip=.25in]{geometry}
\usepackage[italian]{babel}
\usepackage{pgfplots}
\usepackage{tabularx}
\usepackage{wrapfig}
\graphicspath{ {./images/} }
\usetikzlibrary{datavisualization}
\usetikzlibrary{datavisualization.formats.functions}
\pgfplotsset{width=10cm,compat=1.9}

\title{Lab}
\author{Tommaso Miliani}
\date{20-02-25}

\begin{document}
\theoremstyle{definition}
\theoremstyle{theorem}
\theoremstyle{lemma}

\newtheorem{definition}{Definizione}[section]
\newtheorem{theorem}{Teorema}[section]
\newtheorem{lemma}{Proposizione}[theorem]

\maketitle

\section{Isotropi}
\begin{wrapfigure}{r}{0.4\textwidth}
    \centering
    \label{FIg 1.1}
    \caption{Forza applicata ad un oggetto in orizzontale}
    \begin{tikzpicture}
        \draw (0, 0) -- (0, 3);
        \draw (0, 1) -- (2, 1);
        \draw (0, 2) -- (2, 2);
        \draw(2, 1) -- (2, 2);
        \draw(1.8, 1) -- (1.8, 2) node[at end, above] {$\Delta L$};
        \draw[->](2, 1.5) -- (3, 1.5) node[at end, below] {$\vec{F}$};
    \end{tikzpicture}    
\end{wrapfigure}
Con l'equazione di Young si ottiene le seguenti formule:
\begin{align}
    \sigma = \frac{\left|\vec{F}\right|}{S} = E\frac{\left| \Delta L \right| }{L}
\end{align}

Adesso con il coefficiente di Poisson ($\gamma$), complessivamente il corpo si espande o si contrae
quando allungo il lato in quel modo e i lati trasversi si stringono?
A livello fisico succede questo:
\begin{align}
    \frac{\Delta W}{W} = -\gamma \frac{\Delta L}{L} = \frac{\Delta h}{h}
\end{align}
Il rapporto tra allungamento del lato parallelo al cui si applica la forza
è proporzionale all'accorciamento o allungamento del lato perpendicolare all'applicazione
della forza di un fattore $\gamma$. Il volume dell'oggetto considerato è infatti:
\begin{gather*}
    V = whL
\end{gather*} 
Per cui si ottiene che il rapporto $\frac{\Delta V}{V}$ sarà:
\begin{align}
    \frac{\Delta V}{V} = \frac{\Delta L}{L} (1 - 2\gamma)
\end{align}
Nei corpi \textbf{isotropi} generalmente $\gamma > \frac{1}{2}$ questo vuol
dire che non aumenta il volume della sbarretta, se fosse minore, il suo volume
aumenterebbe. Il modulo di comprimibilità, ossia il fattore di comprimibilità
di un materiale (utile per i gas), è dato da:
\begin{align}
    \gamma = \frac{E}{3(1 - 2\gamma)}
\end{align}
Quindi il $\sigma$ è:
\begin{align}
    \sigma = -K \frac{\Delta V}{V}
\end{align}

\section{Il modulo di scorrimento}
\begin{wrapfigure}{r}{0.4\textwidth}
    \centering
    \label{FIg 2.1}
    \caption{Oggetto vincolato su una base}
    \begin{tikzpicture}
        \draw (0, 0) -- (0, 3);
        \draw (0, 0.5) -- (1, 0.5);
        \draw (0, 2.5) -- (1, 2.5);
        \draw(1, 0.5) -- (1, 2.5);
        \draw[dashed](0, 2.5) -- (1, 2.25);
        \draw[dashed](0, 0.5) -- (1, 0.25);
        \draw(0.8, 2.5) arc (0:-15:0.8) node[midway, right] {$\alpha$};
        \draw[->](1, 0.5) -- (1, -0.5) node[at end, below] {$\vec{F}$}; 
    \end{tikzpicture}    
\end{wrapfigure}
Vincoliamo un oggetto ad una base rigida ma l'altra base parallela la
vogliamo non vincolata. In questo caso la base che è sottoposta alla forza parallela rispetto
alla base esterna fa scivolare una base rispetto all'altra. Questo
è immaginabile come una sorta di scorrimento di tanti strati del materiale uno
rispetto all'altro che dà luogo alla variazione della forma dell'oggetto. 
Matematicamente si può esprimere lo sforzo specifico come:
\begin{align}
    \sigma = \frac{F}{S} = G \frac{\left| \Delta x \right| }{L}
\end{align}
Inoltre si ha che:
\begin{gather*}
    G \frac{\left| \Delta x \right| }{L} = G\tan\alpha
\end{gather*}
E si potrebbe approssimare $\tan\alpha \approx \alpha$.
Il modulo di Young è dato invece da:
\begin{align}
    E = 2(1 + \gamma)G
\end{align}
Da cui si può ottenere una delle tre date le altre due.

\section{Flessione}
\begin{wrapfigure}{r}{0.4\textwidth}
    \centering
    \label{Fig 2.2}
    \caption{La flessione}
    \begin{tikzpicture}
        \draw(0, 0) -- (0, 4);
        \draw (0, 2) -- (3, 2);
        \draw(0, 3) -- (3, 3);
        \draw(3,3) -- (3, 2);
        \draw[->](3, 3) -- (3.5, 3) node[at end, right] {$\vec{F}$};
        \draw[dashed](0, 3) .. controls (1.5, 3) .. (2.75, 2.25);
        \draw[dashed](0, 2) .. controls (1.5, 2) .. (2.25, 1.25);
        \draw[dashed](2.75, 2.25) -- (2.25, 1.25);
        \draw[dashed](0, 2.5) -- (1.75, 2.5);
        \draw(2.5, 1.75) -- (1.25, 2.5);
        \draw(1.75, 2.5) arc (0:-30: 0.5) node[midway, right] {$\phi$};
    \end{tikzpicture}    
\end{wrapfigure}
Le fibre superiori del materiale si allungano e quelle inferiori si accorciano
per permettere la flessione dell'oggetto. L'angolo $\phi$ ci permettere di determinare quanta
flessione ha l'oggetto rispetto alla posizione orizzontale iniziale: si ottiene:
\begin{align}
    \phi = \frac{1}{E} \frac{4}{\pi} \frac{L}{r^{4} } M_f
\end{align}
Dove $r$ è il raggio di flessione dell'oggetto  rispetto all'arco della flessione e
$M_f$ è il modulo del momento flettente applicato per cui si ottiene:
\begin{align}
    \frac{\sigma}{E} \approx \frac{\Delta L}{L}
\end{align}

\end{document}
\documentclass[a4paper, oneside]{article}
\usepackage{graphicx}
\usepackage{amsthm}
\usepackage{amsmath}
\usepackage[a4paper,
            bindingoffset=0.2in,
            left=2cm,
            right=2cm,
            top=2cm,
            bottom=2cm,
            footskip=.25in]{geometry}
\usepackage[italian]{babel}
\usepackage{pgfplots}
\usepackage{tabularx}
\usepackage{wrapfig}
\graphicspath{ {./images/} }
\usetikzlibrary{datavisualization}
\usetikzlibrary{datavisualization.formats.functions}
\pgfplotsset{width=10cm,compat=1.9}

\title{Lab}
\author{Tommaso Miliani}
\date{18-02-25}

\begin{document}
\theoremstyle{definition}
\theoremstyle{theorem}
\theoremstyle{lemma}

\newtheorem{definition}{Definizione}[section]
\newtheorem{theorem}{Teorema}[section]
\newtheorem{lemma}{Proposizione}[theorem]

\maketitle

\section{Risoluzione esercizi del compito}
Data a funzione:
\begin{gather*}
    f = \frac{5a^{2} -b}{ab}
\end{gather*}
SOno dati i valori di $a$ e $b$ con i rispettivi errori assoluti:
\begin{gather*}
    a = (50 \pm 1) \cdot cm \\
    b = (2500 \pm 18) \cdot cm^{2}  
\end{gather*}
Seguendo la strada delle derivate logaritmiche, allora si ottiene il
differenziale dei logaritmi:
\begin{gather*}
    \frac{\Delta f}{f} \approx df =  d\ln(5a^{2} -b) - d\ln(a) - d\ln(b) \\
    \Rightarrow \frac{10a - 1}{5a^{2} - b} - \frac{1}{a} - \frac{1}{b}
\end{gather*}
Si raccoglie ora e si mette il valore assoluto al posto delle parentesi
nei raccoglimenti e dunque si ottiene l'espressione per l'errore relativo
cecando di raccogliere a il più possibile ottendneod:
\begin{gather*}
    \frac{\Delta f}{f} \approx \left| \frac{10a}{5a^{2} - b} \right|\Delta a + \left| \frac{1}{5a^{2} - b} + \frac{1}{b}\right| \Delta b  
\end{gather*}

Sostituendo numericamente si ottiene la seguente espressione
\begin{gather*}
    \frac{\Delta f}{f} \approx 
\end{gather*}




\end{document}
\documentclass[a4paper, oneside]{article}
\usepackage{graphicx}
\usepackage{amsthm}
\usepackage{amsmath}
\usepackage[a4paper,
            bindingoffset=0.2in,
            left=2cm,
            right=2cm,
            top=2cm,
            bottom=2cm,
            footskip=.25in]{geometry}
\usepackage[italian]{babel}
\usepackage{pgfplots}
\usepackage{tabularx}
\usepackage{wrapfig}
\graphicspath{ {./images/} }
\usetikzlibrary{datavisualization}
\usetikzlibrary{datavisualization.formats.functions}
\pgfplotsset{width=10cm,compat=1.9}

\title{Analisi}
\author{Tommaso Miliani}
\date{03-12-24}

\begin{document}
\theoremstyle{definition}
\theoremstyle{theorem}
\theoremstyle{lemma}

\newtheorem{definition}{Definizione}[section]
\newtheorem{theorem}{Teorema}[section]
\newtheorem{lemma}{Proposizione}[theorem]

\maketitle

\section{Integrazione per trovare la derivata di composte}
\begin{gather}
    \frac{d}{dx}F(g(x)) = F'(g(x))\cdot g'(x) \\
    \int F'(g(x))\cdot g'(x)dx = F(g(x)) + c \\
\end{gather}
Posto 
\begin{gather*}
    f(x) = F(x) \\
    F(x) = \int f(x) dx
\end{gather*}
Allora si ha la formula di integrazione per sostituizione:
\begin{gather}
    \int f(g(t))\cdot g'(t)dt = \int f(x)dx \\
    con \ x = g(t) 
\end{gather}
Esempio:
\begin{gather*}
    \int \frac{\log t}{t} = \int x dx \Rightarrow  x = \frac{\log t }{t}  \Rightarrow \frac{\log^2 t}{2} + c
\end{gather*}
Formula dell'integrazione di potenze per sostituizione
\begin{gather}
    \int g(t)^{\alpha}g'(t) dt =  \frac{g(t)^{\alpha + 1} }{\alpha + 1} + c 
\end{gather}

\begin{gather*}
    \int \sin^{3}t \ dt = \int \sin^{2} t \cdot \sin t \ dt \\
    = \int (1-\cos^2 t) \sin t \ dt \\
    g(t) = -\cos t. \\
    g'(t) = \sin t \\
    f(x) = 1 - x^2 \\
    \int 1 - x^2  \ dx = \frac{\cos^3 t }{3} + c
\end{gather*}
Si può anche cambiare variabile all'interno dell'integrale e quindi si ottiene
che dentro l'integrale devo sostituire $g$ al posto di $x$ e $g'(t)dt$ al posto di $dx$.
\begin{gather*}
    \int \frac{\log t}{t} \ dt, \ pongo \ x = \log t \Rightarrow 1\ dx =  \frac{1}{t} \ dt \\
    \int x \ dx = \frac{x^2}{2} + c = \frac{\log^{2} t}{2} + c
\end{gather*}

\section{Integrazione di derivate composte in un intervallo definito}
\begin{gather}
    \int_{a}^{b} f'(g(t))\cdot g'(t)dt = \left|^{g(b)}_{g(a)}x\right. = F(g(b)) - F(g(a)) = \\
    \int_{a}^{b} f(x) dx. 
\end{gather}
Esempio:
\begin{gather*}
    \int \frac{1}{\sqrt{1 - x^2}} \ dx \\
    posto \ x = \sin t \\
    \sqrt{1-x^2 } = \sqrt{1- \sin^2 t} = \sqrt{\cos^2 t}
\end{gather*}
Integrata ora per $-1 < x < 1$ allora diventa:
\begin{gather*}
    dx = \cos t \ dt. \\
    \int \frac{1}{\cos t} \cdot \cos t \ dt = t + c \\
    Quindi \ essendo \ x = \sin t \\
    \arcsin(x) + c
\end{gather*}

\section{Area del cerchio}
Data la formula della circonferenza: $x^2  + y^2  = R^2$
voglio calcolare:
\begin{gather*}
    4\int_{0}^{R} \sqrt{R^2 -x^2 } \ dx \\
    Posto \ ora \ x = R \sin t \\
    4\int_{0}^{\frac{\pi}{2}} R^2  \cos^2  t \ dt \\
    Posto \ \cos^2 t = \frac{1+\cos(2t)}{2} \Rightarrow  4R^2 \int_{0}^{\frac{\pi}{2}} \frac{1 + \cos(2t)}{2} \ dt  \\
    \Rightarrow  \left|^{\frac{\pi}{2}}_{0} \frac{x^2 }{4} + \frac{\sin(2t)}{4}\right. 
\end{gather*} 

\section{Integrali di fratte}
Posto inizialmente che il grado di $P(x) < Q(x)$, questi sono
due polinomi; se il grado non è minore allora dobbiamo eseguire
la divisione tra polinomi.
Divisione tra polinomi:
\begin{gather}
    \frac{P(x)}{Q(x)} = A(x) + \frac{R(x)}{Q(x)}
\end{gather}
Esempio:
\begin{gather*}
    \frac{x^{3} - 2x }{x^{2} + 3} \ diventa \\
    \begin{tabular}{c c c | c}
        $x^3$  &  & $-2x$ & $x^{2} +3$ \\
        \hline 
        $-x^3$  & & -3x & x \\
        &&-x&
    \end{tabular} =  \\
    -x + \frac{x}{x^2 +3}
\end{gather*}
Le cose importanti sono le seguenti:
\begin{enumerate}
    \item Il grado di $P(x) < Q(x)$;
    \item Il coefficiente della potenza più grande di Q sia 1.
\end{enumerate}

\begin{gather*}
    \int \frac{1}{x^2 +bx + c} \ dx  \\
    Completamento \ del \ quadrato: \ \left(x+\frac{b}{2}\right)^2 + c - \frac{b^2}{4} = \left( x + \frac{b}{2} \right)^2  -\frac{\Delta}{4}
    = -\frac{\Delta}{4} \left( \frac{\left(x+\frac{b}{2}\right)^2 }{\left(\frac{\sqrt{-\Delta}}{2}\right)^2 } + 1\right) 
\end{gather*}
Quindi l'integrale diventerà:
\begin{gather*}
    \frac{4}{-\Delta}\int \frac{1}{\left(\frac{2x + b}{\sqrt{-\Delta}}\right)+1} \ dx
\end{gather*}
Ponendo ora
\begin{align*}
    y = \frac{\left(x+\frac{b}{2}\right)^2 }{\left(\frac{\sqrt{-\Delta}}{2}\right)^2 }
\end{align*}
Diventa
\begin{align*}
    \frac{4}{-\Delta}\int \frac{1}{y^2  + 1} \cdot \frac{\sqrt{-\Delta}}{2} \ dy = \frac{2}{\sqrt{-\Delta}} \arctan\left(\frac{2x + b}{\sqrt{-\Delta}}\right) + c
\end{align*}
Un'altro fratto semplice è:
\begin{align}
    \int \frac{1}{(x-x_0)^{n}} \ dx = \int \frac{1}{y^{n} } \ dy = \int y^{-n} \ dy = \frac{y^{-n+1}}{-n+1} +c
\end{align}



\end{document}
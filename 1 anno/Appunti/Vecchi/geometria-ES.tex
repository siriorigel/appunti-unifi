\documentclass[a4paper, oneside]{article}
\usepackage{graphicx}
\usepackage{amsthm}
\usepackage{amsmath}
\usepackage[a4paper,
            bindingoffset=0.2in,
            left=2cm,
            right=2cm,
            top=2cm,
            bottom=2cm,
            footskip=.25in]{geometry}
\usepackage[italian]{babel}
\usepackage{pgfplots}
\usepackage{tabularx}
\usepackage{wrapfig}
\graphicspath{ {./images/} }
\usetikzlibrary{datavisualization}
\usetikzlibrary{datavisualization.formats.functions}
\pgfplotsset{width=10cm,compat=1.9}

\title{Esercizi geometria}
\author{Tommaso Miliani}
\date{03-12-24}

\begin{document}
\theoremstyle{definition}
\theoremstyle{theorem}
\theoremstyle{lemma}

\newtheorem{definition}{Definizione}[section]
\newtheorem{theorem}{Teorema}[section]
\newtheorem{lemma}{Proposizione}[theorem]

\maketitle

\section{Esercizio 15}
\begin{align*}
    A = \left\{ \left( \begin{array}{c}
        x_1 \\
        x_2 \\
        x_3 \\
        x_4
    \end{array} \right) \in R^{4} : x_1 - x_2 + x_3 -x_4 = 0 \right\},
    \qquad B = \left\{\left(\begin{array}{c}
        t + s \\
        2s \\
        t\\
        t+s
    \end{array}\right) : t, s \in R \right\}.
\end{align*}
\subsection{Forma cartesiana di B}
\begin{gather*}
    t\left( \begin{array}{c}
        1 \\
        0 \\
        1 \\
        1
    \end{array}\right) + s \left( \begin{array}{c}
        1 \\
        2 \\
        0 \\
        1
    \end{array}\right)  \\
    \left(\begin{tabular}{c c | c}
        1 & 1 & $x_1$ \\
        0 & 2 & $x_2$ \\
        1 & 0 & $x_3$ \\
        1 & 1 & $x_4$ 
    \end{tabular}\right) \quad \Rightarrow \quad
    \left(\begin{tabular}{c c | c}
        1 & 1 & $x_1$ \\
        0 & 2 & $x_2$ \\
        1 & 0 & $x_3$ \\
        0 & 0 & $x_4 - x_1$ 
    \end{tabular}\right) \Rightarrow 
    \left(\begin{tabular}{c c | c}
        1 & 1 & $x_1$ \\
        1 & 0 & $x_3$ \\
        0 & 2 & $x_2$ \\
        0 & 0 & $x_4 - x_1$ 
    \end{tabular}\right) \\
    \Rightarrow \left(\begin{tabular}{c c | c}
        1 & 1 & $x_1$ \\
        0 & -1 & $x_3 - x_1$ \\
        0 & 2 & $x_2$ \\
        0 & 0 & $x_4 - x_1$ 
    \end{tabular}\right)
    \Rightarrow  \left(\begin{tabular}{c c | c}
        1 & 1 & $x_1$ \\
        0 & -1 & $x_3 - x_1$ \\
        0 & 0 & $x_2 + 2x_3 -2x_1$ \\
        0 & 0 & $x_4 - x_1$ 
    \end{tabular}\right) \Rightarrow x_4 = x_1 \ per \ avere \ sol. \\
    \Rightarrow \left\{ \begin{array}{c}
        x_2 + 2x_3  -2x_1 \\
        x_4 - x_1 = 0
    \end{array}\right. \\
    La \ matrice \ associata \ e' \\
    \left( \begin{tabular}{c c c c | c}
        -2 & 1 & 2 & 0 & 0 \\
        -1 & 0 & 0 & 1 & 0
    \end{tabular}\right) \\
    \Rightarrow  \left(\begin{tabular}{c c c c | c}
        -1 & 0 & 0 & 1 & 0  \\
        0 & 1 & 2 & -2 & 0
    \end{tabular}\right) \Rightarrow x_1 = x_4, x_2 = 2x_4 - 2x_3 \Rightarrow \\
    x_3 \begin{pmatrix} 0 \\
    -2 \\
    1 \\
    0 \end{pmatrix} + x_4 \begin{pmatrix} 1 \\
    1 \\
    0 \\
    1 \end{pmatrix} \\
\end{gather*}

\subsection{Una base di A}
\begin{gather*}
    x_1 - x_2 + x_3 = x_4 \Rightarrow \begin{pmatrix} x_1 \\
    x_2 \\
    x_3 \\
    x_1 - x_2 + x_3 \end{pmatrix} \Rightarrow \\
    x_1 \begin{pmatrix} 1 \\
    0 \\
    0 \\
    1 \end{pmatrix} + x_2 \begin{pmatrix} 0 \\
    1 \\
    0 \\
    -1 \end{pmatrix}  x_3 \begin{pmatrix} 0 \\
    0 \\
    1 \\
    1 \end{pmatrix} 
\end{gather*}

\subsection{Base di B}
I vettori specificati per indicare s, t sono già la sua base per cui:
\begin{align*}
    <\begin{pmatrix} 1\\
    0\\
    1\\
    1 \end{pmatrix}, \begin{pmatrix} 1 \\
    2 \\
    0 \\
    1 \end{pmatrix} > 
\end{align*}

\subsection{Una base di $A \cap B$}
Da GrassMan so che $\dim(A) + \dim(B) = \dim(A \cap B) + \dim(A + B)$.
\begin{align*}
    
\end{align*}
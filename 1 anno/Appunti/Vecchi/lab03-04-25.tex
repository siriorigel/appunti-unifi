\documentclass[a4paper, oneside]{article}
\usepackage{graphicx}
\usepackage{amsthm}
\usepackage{amsmath}
\usepackage[a4paper,
            bindingoffset=0.2in,
            left=2cm,
            right=2cm,
            top=2cm,
            bottom=2cm,
            footskip=.25in]{geometry}
\usepackage[italian]{babel}
\usepackage{pgfplots}
\usepackage{tabularx}
\usepackage{tikz}
\usepackage{wrapfig}
\usepackage{color}
\definecolor{page}{rgb}{0.129,0.157,0.212}
\pagecolor{page}
\color{white}
\graphicspath{ {./images/} }
\usetikzlibrary{shapes.geometric}
\usetikzlibrary{datavisualization}
\usetikzlibrary{datavisualization.formats.functions}
\pgfplotsset{width=10cm,compat=1.9}

\title{Lab zoccols}
\author{Tommaso Miliani}
\date{03-04-25}

\begin{document}
\theoremstyle{definition}
\theoremstyle{theorem}
\theoremstyle{lemma}

\newtheorem{definition}{Definizione}[section]
\newtheorem{theorem}{Teorema}[section]
\newtheorem{lemma}{Proposizione}[theorem]
\newtheorem{example}{Esempio}[section]

\maketitle

\section{Roba già fatta altre cento volte}





\begin{gather*}
    \left\{\begin{array}{l}
        \star \ \phi_0 = \phi_0(0) \\
        \star \ T_0'
    \end{array}\right.
\end{gather*}
Nel limite delle piccole oscillazioni dunque vale la seguente:
\begin{gather*}
    \ddot{\phi} + \frac{b}{m}\dot{\phi} + \frac{g}{l}\phi = 0
\end{gather*}
E allora dato che
\begin{gather*}
    \frac{g}{l} = \omega_0^{2} 
\end{gather*}
Nel nostro caso sappiamo con certezza che il moto è di natura smorzata ma oscillatoria 
posso allora esprimere l'angolo in funzione del tempo come:
\begin{gather*}
    \phi(t) = Be^{kt} \Rightarrow B, k \in C 
\end{gather*}
Derivando l'espressione
\begin{gather*}
    \dot{\phi} = kBe^{kt} = k\phi 
\end{gather*}
E allora la derivata seconda rispetto al tempo è proprio
\begin{gather*}
    \ddot{\phi} = k^{2}\phi^{kt}  
\end{gather*}
SOstituendo tutto quanto nel limite delle piccole oscillazioni si ottiene
dunque la seguente espressione:
\begin{gather*}
    
\end{gather*}


\end{document}
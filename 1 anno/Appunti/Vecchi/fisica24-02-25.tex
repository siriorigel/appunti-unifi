\documentclass[a4paper, oneside]{article}
\usepackage{graphicx}
\usepackage{amsthm}
\usepackage{amsmath}
\usepackage[a4paper,
            bindingoffset=0.2in,
            left=2cm,
            right=2cm,
            top=2cm,
            bottom=2cm,
            footskip=.25in]{geometry}
\usepackage[italian]{babel}
\usepackage{pgfplots}
\usepackage{tabularx}
\usepackage{wrapfig}
\usepackage{color}
\pagecolor{black}
\color{white}
\graphicspath{ {./images/} }
\usetikzlibrary{datavisualization}
\usetikzlibrary{shapes.geometric}
\usetikzlibrary{datavisualization.formats.functions}
\pgfplotsset{width=10cm,compat=1.9}

\title{Appunti FIsica Cuccoli}
\author{Tommaso Miliani}
\date{24-02-25}

\begin{document}
\theoremstyle{definition}
\theoremstyle{theorem}
\theoremstyle{lemma}

\newtheorem{definition}{Definizione}[section]
\newtheorem{theorem}{Teorema}[section]
\newtheorem{lemma}{Proposizione}[theorem]

\maketitle

\section{Esercizio}
\begin{wrapfigure}{r}{0.4\textwidth}
    \centering
    \label{Fig 1.1}
    \caption{Giostra con oggetto sopra}
    \begin{tikzpicture}
        \node[ellipse,
        draw,
	    minimum width = 4cm, 
	    minimum height = 2.2cm] (e) at (0,0) {};
        \filldraw (0, 0)circle  (1pt) node[anchor = east]{$O$};
        \filldraw(1.25, -0.25) circle (2pt) node[anchor = south] {$P$};
        \draw(0, 0)[->] -- (4, 0) node[at end, below] {$y \ (\hat{j})$};
        \draw(0, 0)[->] -- (0, 4) node[at end, left] {$z \ (\hat{k})$};
        \draw(0, 0)[->] -- (-1, -2) node[at end, below] {$x \ (\hat{i})$};
        \draw(1.25, -0.25)[->] -- (2, -0.45) node[at end, right] {$\vec{F}_{cf}$};
        \draw(1.25, -0.25)[->] -- (1.75, 0.25) node[at end, right] {$\hat{u}_t$};
        \draw[->](1.25, -0.25) -- (0.75, -0.75) node[at end, below] {$\vec{v}'$}; 
        \draw[->](1.25, -0.25) -- (1.25, 0.75) node[at end, right] {$\vec{N}$};
        \draw[->](1.25, -0.25) -- (1.25, -1.25) node[at end, right] {$m\vec{g}$};
        \draw[->, very thick](1.25, -0.25) -- (0.5, -0.123) node[at end, below] {$\vec{F}_{co}$};
        \draw[dashed, thin] (0, 0) -- (1.25, -0.25) node[midway, above] {$\rho$};
    \end{tikzpicture}    
\end{wrapfigure}
Data una giostra con un oggetto sopra ( senza attrito), per un osservatore
inerziale (esterno) la forza risultante  sarà:
\begin{gather*}
    \vec{N} + m \vec{g} = m \vec{a}   
\end{gather*}
Per l'osservatore S fermo sul terreno l'accelerazione è dunque nulla
poiché ci sono solo forze verticali che si bilanciano perfettamente. All'istante $t = 0$ si ha per lui:
\begin{gather*}
    \left| \vec{OP} \right| = |\vec{d}| = \rho \\
    \vec{v} (t = 0) = 0  
\end{gather*} 
Come ragiona l'osservatore S' la cui origine O' coincide con quella di O?
L'osservatore è sulla giostra e quindi sugli oggetti agiscono anche altre forze che l'osservatore
S definirebbe immaginarie. La prima di queste è la forza centrifuga:
\begin{align}
    \vec{F}_{cf} = m \omega^{2} \rho \hat{u}_t  
\end{align}
Noi sappiamo che la velocità rispetto al SDR di $S$ si esprime come
(dato che all'istante $t = 0$ $\vec{v} = 0$):
\begin{align}
    \vec{v} =  \vec{v}' + \vec{v}_t + \vec{\omega} \times \vec{r}    
\end{align}
La componente $\vec{v}'$, che è propria del SDR di $S'$ sarà quindi data (per definizione): 
\begin{gather*}
    \vec{v}' = \vec{v} -   \vec{\omega} \times \vec{r}  = \vec{\omega} \times \vec{r}
\end{gather*}
A questo punto si esplicita :
\begin{gather*}
    \vec{v}'(t = 0) = -\omega \vec{d} \hat{u}_p  
\end{gather*}
Dato che nel sistema di riferimento S' si osserva la forza di Coriolis; all'istante $t= 0$
esplicitando $\vec{v}'$ e ricordando che:
\begin{gather*}
    \vec{N} + m\vec{g} + \vec{F}_{cf} + \vec{F}_{co}' = m\vec{a}     
\end{gather*} 
Esplicitando $\vec{F}_{co}$ questa diventa (svolti tutti i passaggi):
\begin{gather*}
    -2m\left( (\vec{\omega} \times \vec{r}  )\vec{\omega} - \omega'\vec{r}'  \right)
\end{gather*} 
Esplicitando generalmente attraverso i versori e le coordinate sull'asse $z$ si ottiene
la seguente relazione, che, una volta svolti tutti i passaggi, diventa:
\begin{align}
    \vec{F}_{co} = -2m\vec{\omega} \times (\vec{\omega} \times r\hat{u}_p)  = -2m \omega^{2} \rho \hat{u}_p  
\end{align}
La somma della forza centrifuga e di Coriolis ci permette di ottenere
\begin{align}
    \vec{F}_{co} + \vec{F}_{cf} = -m\omega^{2}\vec{d}\hat{u}_p     
\end{align}
Proiettando sul piano si ottiene che questa somma è uguale proprio a $m \vec{a}_{\parallel}'$ . Si continua
ad usare le notazioni vettoriali poiché nel piano i vettori hanno ancora due componenti
e non posso dunque utilizzare la notazione scalare. Questa equazione nuova mi dice che
\begin{align}
    -\omega^{2}\vec{d}\hat{u}_p = \vec{a}_{\parallel}'
\end{align}
Nell'istante $t = 0$ il mio corpo ha una sola accelerazione, ossia quella centripeta. Allora
vuol dire che non c'è accelerazione tangenziale e quindi la velocità di rotazione è costante, essendo
un moto circolare quell'accelerazione centripeta esiste facendo valere la relazione sopra
nell'istante $t= 0$, ma questa vale anche per tutti gli altri istanti poiché si tratta di un moto C.U. 

\section{Un sistema di riferimento inerziale quasi perfetto: le stelle fisse}
\begin{wrapfigure}{r}{0.4\textwidth}
    \centering
    \label{FIg 2.1}
    \caption{}
    \begin{tikzpicture}
        \draw (0, 0) circle (2);
        \draw[dashed, thin] (0, 3) -- (0, -3);
        \draw[->] (-0.5 ,3) arc (200:340:0.5) node[at end, right] {$\vec{\omega}_T$};
        \draw[dashed, thin](-2, 0) arc(240: 300: 4);
        \draw[dashed, thin](2, 0) arc(60: 120: 4);
        \filldraw (0, 0) circle (1pt) node[anchor = south] {$O$};
        \filldraw(1.41, 1.41) circle (1pt) node[anchor = south] {$P$};
        \draw[->] (1.41, 1.41) -- (0.5, 0.5) node [midway, left] {$\vec{F}_G$};
        \draw[->] (1.41, 1.41) -- (0.7, 0.5) node [at end, below] {$m\vec{g}$};
        \draw[->] (1.41, 1.41) -- (2, 1.41) node[at end, right] {$\vec{F}_{cf}$};
        \draw[->] (1.5, 0) arc (0:45:1.5) node[midway, right] {$\theta$};
        \draw(0, 1.41) -- (1.41, 1.41) node[midway, above] {$\rho$};
        \draw[dashed, thin] (0, 0) -- (1.41, 1.41);
        \draw[dashed, thin] (0, 0) -- (2, 0) node[midway, below] {$R_T$};
    \end{tikzpicture}    
\end{wrapfigure}
Se facessimo degli esperimenti sulla Terra dovremmo riuscire a mettere in luce le
forze apparenti di un SDR non inerziale. La Terra ruota intorno al proprio asse
rispetto alle stelle fisse da Ovest verso Est e quindi mentre ruota su sé stessa con velocità
angolare $\left| \vec{\omega}_T  \right| = \frac{2\pi}{86'164}$ essa dovrebbe ritrovare il Sole nel medesimo punto
ad ogni rivoluzione anche se questo non accade. Questo perché il giorno solare medio è più lungo del giorno sidereo, rispetto
alle stelle fisse, infatti, la Terra ha compiuto un giro in più. Che conseguenza ha il fatto che
la Terra non è un SDR inerziale?\\
Questo vuol dire che ci sono delle forze apparenti in tutti gli esperimenti e quindi
rispetto alla Terra nei moti su larga scala intervengono queste forze apparenti. 
All'equatore per l'osservatore inerziale, un oggetto vicino alla Terra è soggetto
alla forza di gravità dovuta all'interazione con la Terra (che si assume sferica),
mentre per l'osservatore sulla Terra mi aspetto che ci sia una forza centrifuga. Usando un filo a piombo, 
la sua verticale non passerà per il centro della Terra ma sarà leggermente inclinata rispetto
alla forza peso ideale a causa della forza centrifuga. 
Voglio calcolare allora il vero peso degli oggetti e misuro oltre che
la direzione del pendolo anche la forza peso. Associo $\hat{u}_{\rho}$ il versore $\perp \hat{k}$
che giace nello stesso piano del versore  $\hat{u}_r$ (il versore parallelo all'asse $\vec{OP}$ )
e che è il versore della centrifuga. Allora:
\begin{align}
    \hat{u}_r = \left( \cos\theta\vec{u}_{\rho} + \sin\theta\hat{k} \right) 
\end{align}
La forza peso allora si esprimerà come:
\begin{align}
    m\vec{g} = \vec{F}_{G} + \vec{F}_{cf} = -mg_0 \hat{u}_r + m\omega^{2}_T  + \rho \hat{u}_{\rho} 
\end{align}
Dove $g_0$ è $g$ senza la correzione della forza centrifuga, sostituendo allora quanto ottenuto prima si ottiene.
\begin{align}
    m\vec{g} = m \left( \cos\theta\left( -g_o + \omega^{2}_T R_T\right) \hat{u}_{\rho} + g_0 \sin\theta \hat{k} \right)
\end{align}
Volendo ora calcolarne il modulo, si ottiene la seguente:
\begin{align}
    \left| g \right| = \sqrt{\cos^{2}\theta\left(g_0^{2}  + \omega^{4}_T R_T^{2} - 2g_0 \omega^{2}_T R_T\right) + g_0^{2}\sin^{2}\theta }
\end{align}
Che sostituendo, raccogliendo e semplificando si ottiene:
\begin{align}
    \left| g \right| = \sqrt{g_0^{2} \left( 1 - \frac{2\omega^{2}_T R_T}{g_0}\cos^{2}\theta + \left( \frac{\omega^{2}_T R_T }{g_0} \right)^{2} \cos^{2}\theta  \right) }  
\end{align}
Usando lo sviluppo di Taylor allora si può semplificare i coseni (in particolare il primo)
poiché hanno un contributo molto piccolo e quindi:
\begin{align}
    \left| g \right| \approx g_0 \left(1 - \frac{\omega^{2}_T R_T}{g_0}\cos^{2}\theta\right) 
\end{align}
L'accelerazione al polo è $g_{polo} = 9.823 m/s^{2} $ e quella all'equatore $g_{eq} = 9.789 m/s^{2} $ poiché
la Terra è un ellissoide di rotazione e la forza di Coriolis ha fatto schiacciare la Terra in modo tale che
all'equatore il raggio terrestre sia più grande che ai poli.
Preso un triangolo i cui lati sono la forza centripeta, uno la forza di gravità
vera e una quella ideale, applicando la trigonometria si ottiene l'espressione dell'angolo
tra le due forze ($\delta$):
\begin{gather*}
    \frac{F_{cf} }{\sin\delta} = \frac{\sin g}{\sin\theta} \\
    \sin\delta \approx  \frac{\omega^{2}_T R_T }{g_0} \cos\theta\sin\theta
\end{gather*} 

\end{document}
\documentclass[a4paper, oneside]{article}
\usepackage{graphicx}
\usepackage{amsthm}
\usepackage{amsmath}
\usepackage[a4paper,
            bindingoffset=0.2in,
            left=2cm,
            right=2cm,
            top=2cm,
            bottom=2cm,
            footskip=.25in]{geometry}
\usepackage[italian]{babel}
\usepackage{pgfplots}
\usepackage{tabularx}
\usepackage{wrapfig}
\graphicspath{ {./images/} }
\usetikzlibrary{datavisualization}
\usetikzlibrary{datavisualization.formats.functions}
\pgfplotsset{width=10cm,compat=1.9}

\title{ESperienza elasticità}
\author{Tommaso Miliani}
\date{19-02-25}

\begin{document}
\theoremstyle{definition}
\theoremstyle{theorem}
\theoremstyle{lemma}

\newtheorem{definition}{Definizione}[section]
\newtheorem{theorem}{Teorema}[section]
\newtheorem{lemma}{Proposizione}[theorem]

\maketitle

\section{Elasticità}
L'esperienza dell'elasticità ha lo scopo di determinare se un oggetto
si è comportato come un corpo rigido durante tutta la durata dell'esperimento
mediante misurazioni indirette delle posizioni relativi tra il corpo rigido e 
gli altri oggetti. Esistono anche i corpi non rigidi, che sono più complicati
e che vanno analizzati caso per caso: un sistema di punti materiali con un corpo
esteso che ha una certa forma, allora se partiamo da una situazione di equilibrio 
e la sua accelerazione è nulla allora le forze:
\begin{gather*}
    \vec{a} = \frac{\sum \vec{F}_{int(i)} }{m_i} \\
\end{gather*}
Se un punto applica una forza esterna quello che avviene è che il corpo
si deforma e cambiano quindi le distanze tra gli oggetti e su di lui sono
applicate delle forze esterne oltre che a quelle interne .
\begin{gather*}
    \vec{a} = \frac{\sum \vec{F}_{int} + \vec{F}_{ext} }{m_i} 
\end{gather*}
In generale quando si ha un corpo esteso sottoposto a deformazioni e quando siamo 
in una situazione di equilibrio allora si bilanciano le forze esterne ed interne e 
se si smette di applicare delle forze esterne allora le forze interne agiranno
sul sistema deformandolo. Se il corpo torna però alle condizioni iniziali dopo
questo agire di forze, allora il corpo è elastico, altrimenti plastico.\\
Quando si applicano delle forze su di un corpo, esso può deformarsi in vari modi:
se la forza è verticale allora si parla di trazione o compressione, altrimenti
se la forza è orizzontale si parla di scorrimento o taglio. Se la risultante sulla
base libera è nulla ma il momento delle forze è $\neq 0$ ed è parallela alla base libera
allora il corpo è sottoposto a flessione se le forze sono nello stesso senso, altrimenti
si parla di torsione.

\section{Trazioni e compressioni}
\begin{wrapfigure}{r}{0.4\textwidth}
    \centering
    \label{FIg 1.1}
    \caption{Parallelepipedo}
    \begin{tikzpicture}
        \draw (0, 0) -- (1, 0);
        \draw (0, 0) -- (0, 2) node[at start, left] {$\Delta L$} node[midway, left] {$L$};
        \draw(1, 0) -- (1.5, 0.25);
        \draw(1.5, 0.25) -- (1.5, 2.25);
        \draw(0, 2) -- (1, 2);
        \draw (1, 2) -- (1.5, 2.25);
        \draw(0, 2) -- (0.5, 2.25) ;
        \draw(0.5, 2.25) -- (1.5, 2.25);
        \draw (1, 2) -- (1, 0);
        \draw[dashed] (0, 0.5) -- (1, 0.5);
        \draw[dashed] (1, 0.5) -- (1.5, 0.75);
        \draw[->] (0.75, 0) -- (0.75, -1) node[at end, right] {$\vec{F}$};
    \end{tikzpicture}    
\end{wrapfigure}
Nel caso in cui il Parallelepipedo poggi su di un piano indeformabile 
allora si comprimerà e quindi si modificherà di un certo $\Delta L$ e 
aumenterà la sua sezione, se sottoposto a trazione invece si allungherà causando
una riduzione la sua sezione. \\
Ogni materiale è dotato di una caratteristica chiamata \textbf{sforzo specifico(normale)}
che si calcola nella seguente maniera:
\begin{align}
    \sigma = \frac{|\vec{F} |}{S}
\end{align}
Si esprime in $Pa$, $\frac{N}{m^{2} }$. VAriando lo sforzo specifico si potrebbero causare
delle deformazioni differenti e finché questo sforzo si mantiene sotto un certo
limite chiamato \textbf{sforzo elastico} allora la deformazione è elastica e 
si indica come $\sigma_{e}$.
Dal grafico, dopo la fase di elasticità, ad una piccola variazione dello sforzo
corrisponde una grande variazione del rapporto $\Delta L /L$, in genere valido per
molti dei materiali.
\begin{wrapfigure}{r}{0.4\textwidth}
    \centering
    \label{FIg 2.1}
    \caption{Il grafico $\sigma $ deformazione relativa}
    \begin{tikzpicture}
        \draw(0, 0) -- (4, 0) node[at end, below] {$\frac{\Delta L}{L}$};
        \draw(0, 0) -- (0, 4) node[at end, left] {$\sigma$};
        \draw(0, 0) .. controls (0.8, 1.8) .. (4, 2);
        \draw[dashed](0, 1.5) -- (0.8, 1.5) node[at start, left] {$\sigma_p$};
        \draw[dashed](0.8, 0) -- (0.8, 1.5);
        \filldraw (0.5, -0.1) circle (0pt) node[anchor = north] {elasticità};
        \filldraw (3, -0.1) circle (0pt) node[anchor = north] {plasticità};
    \end{tikzpicture}    
\end{wrapfigure}
TUtti i materiali hanno una loro costante elastica per cui le deformazioni sono
proporzionali alla forza impiegata.  (Adottabile se e solo se il valore di $\sigma$ sta
dentro la curva) la forza sarà quindi:
\begin{align}
    F = E S \frac{\Delta L}{L}\\
\end{align}
$E = kg /mm^{2}$ ossia lo sforzo elastico. 

\begin{tabular}{c | c | c | c}
    & E & $\sigma$ & $\sigma$ \\
    \hline
    Acciaio & $ 20\cdot 10^{3}$ & 25 & 50 \\

\end{tabular}

\end{document}
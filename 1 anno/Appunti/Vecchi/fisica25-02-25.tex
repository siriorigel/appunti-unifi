\documentclass[a4paper, oneside]{article}
\usepackage{graphicx}
\usepackage{amsthm}
\usepackage{amsmath}
\usepackage[a4paper,
            bindingoffset=0.2in,
            left=2cm,
            right=2cm,
            top=2cm,
            bottom=2cm,
            footskip=.25in]{geometry}
\usepackage[italian]{babel}
\usepackage{pgfplots}
\usepackage{tabularx}
\usepackage{wrapfig}
\graphicspath{ {./images/} }
\usetikzlibrary{datavisualization}
\usetikzlibrary{datavisualization.formats.functions}
\pgfplotsset{width=10cm,compat=1.9}

\title{Fisica}
\author{Tommaso Miliani}
\date{25-02-25}

\begin{document}
\theoremstyle{definition}
\theoremstyle{theorem}
\theoremstyle{lemma}

\newtheorem{definition}{Definizione}[section]
\newtheorem{theorem}{Teorema}[section]
\newtheorem{lemma}{Proposizione}[theorem]

\maketitle

\section{La molla ideale}
\begin{wrapfigure}{r}{0.4\textwidth}
    \centering
    \label{FIg 1.1}
    \caption{La molla ideale}
    \begin{tikzpicture}    \draw (0, 0) -- (0.5, 0);
        \draw (0.5, 0) -- (0.5, -1);
        \draw[decoration={aspect=0.3, segment length=2.2mm, amplitude=2mm,coil},decorate,opacity=0.9] (0.5, -0.5) -- (2.5,-0.5);
        \draw (2.5, -1) rectangle (3.25, -0.25) node[midway] {$M$};
        \draw (0.5, -1) -- (4, -1);
    \end{tikzpicture}    
\end{wrapfigure}
La relazione fondamentale del moto armonico della molla è:
\begin{gather*}
    m\ddot{x} = - kx
\end{gather*}
Che diventa
\begin{gather*}
    m\ddot{x}\dot{x} = -kx\dot{x}  
\end{gather*}
Integrando parte per parte si ottiene quello che più avanti chiameremo lavoro:
\begin{gather*}
    \frac{1}{2}m\dot{x}^{2} + \frac{1}{2}kx^{2}   
\end{gather*}
Ossia il moto armonico:
\begin{align}
    \ddot{x} + \omega^{2}x = 0 
\end{align}
con $\omega^{2} = \frac{k}{m}$: 

\section{Il filo con massa}
\begin{wrapfigure}{r}{0.4\textwidth}
    \centering
    \label{Fig 1.2}
    \caption{FIlo con massa}
    \begin{tikzpicture}
        \draw(0, 0) -- (3, 0);
        \draw(3, 0) -- (3,-3);
        \draw(1, 0) -- (1, 0.5);
        \draw(1, 0.5) -- (3.5, 0.5);
        \draw(3.5, 0.5) -- (3.5, -1.5);
        \draw(3.5, -1.5) -- (3, -1.5);
        \draw[->](3.25, -1.5) -- (3.25, -2.5) node[at end, right] {$\lambda x \vec{g}$};
        \draw[->](3.25, -0.5) -- (3.25, -1) node[at end, below] {$x$};
        \draw[->](3.25, -0.5) -- (3.75, -0.5) node[at end, right] {$y$};
    \end{tikzpicture}    
\end{wrapfigure}
Assumendo che il filo abbia una densità costante, fissato un sistema di riferimento
allora noi sappiamo che i due pezzi di filo hanno una certa dimensione.
La massa del filo su cui agisce la forza peso sarà densità per lunghezza, ossia:
\begin{gather*}
    x \lambda, \lambda = \frac{M}{L}
\end{gather*}
Per cui la forza peso sul filo che ciondola è:
\begin{gather*}
    x\lambda\vec{g} 
\end{gather*}
E quindi si ottiene che:
\begin{gather*}
    M\ddot{x} = \lambda x g 
\end{gather*}
Si ottiene allora l'espressione dell'accelerazione del filo come:
\begin{gather*}
    \ddot{x} = - \frac{g}{L}x = 0
\end{gather*}
Chiamato allora $\beta^{2} = \frac{g}{L}$, si ottiene una equazione simile
a quella della molla anche se non si può applicare seno e coseno come risoluzione dell'
equazione
\begin{gather*}
    \ddot{x}-\beta^{2} x = 0 \\
    x = Ae^{\beta t} + Be^{-\beta t} \\
    \dot{x} = Ae^{\beta t} \beta - Be^{-\beta t} \beta \\
    \ddot{x} = Ae^{\beta t}\beta^{2} + Be^{-\beta t} \beta^{2} = \beta^{2} x           
\end{gather*} 
Il solo segno meno ha cambiato completamente l'equazione. Ora ricavo $A$ e $B$ dalle condizioni
iniziali:
\begin{gather*}
    \left\{ \begin{array}{c}
        x = x_0 \\
        \dot{x} = 0 
    \end{array}\right.
\end{gather*}
Allora si ha che:
\begin{gather*}
    x(0) = x_0 = A + B \\
    \dot{x}(0) = 0 = \beta(A - B)
\end{gather*}
Quindi le nostre soluzioni sono proprio il coseno e seno iperbolici:
\begin{gather*}
    x = x_0 \frac{e^{\beta t} + e^{-\beta t}}{2} = x_0 \cosh\beta t
\end{gather*}
Se $\beta t << 1$ allora per $ t = 0$ e $ x = x_0$ allora si ottiene
sviluppando con Taylor i membri del coseno iperbolico:
\begin{gather*}
    e^{\beta t} = 1 + \beta t + \frac{1}{2}\beta^{2}t^{2} \\
    e^{-\beta t} = 1 - \beta t + \frac{1}{2}\beta^{2}t^{2}
\end{gather*}
\begin{gather*}
    x = x_0 \left(1 + \frac{1}{2}\beta^{2}t^{2}\right)
\end{gather*}
In $t = 0$ allora il moto è uniformemente accelerato, se invece 
$t >> 0$, diventa un moto esponenziale, ossia le leggi che vincolano il moto
fanno crescere la velocità esponenzialmente. Questo vuol dire che il corpo continua ad aumentare la sua velocità?
No poiché una volta che il filo è finito allora continuerà a cadere come
tutti i corpi e quindi:
\begin{gather*}
    x = \frac{x_0}{2}e^{\beta t} 
\end{gather*}

\subsection{Focus su equazioni differenziali lineari omogenee con coefficienti costanti}
Un'equazione del genere è:
\begin{align}
    a_1\ddot{x} + a_2 \dot{x} + a_3 x = 0  
\end{align}
Un'equazione tipo:
\begin{gather*}
    x = Ae^{\alpha t} 
\end{gather*}
se si sostituisce nella differenziale si ottiene
\begin{gather*}
    Ae^{\alpha t} (a_1\alpha^{2} + a_2\alpha + a_3) = 0
\end{gather*}
La soluzione è data dalla risoluzione dell'equazione di secondo grado,
il che ci porta a dare la soluzione della differenziale di prima che contiene però due soluzioni:
una parte reale data dal coseno ed una parte immaginaria data dal seno:
\begin{gather*}
    \cos\omega t = \frac{e^{\omega t} + e^{-\omega t} }{2} \\
    \sin\omega t = \frac{e^{\omega t} + e^{-\omega t} }{2i}
\end{gather*}
Se il determinante dell'equazione di secondo grado si annulla allora si ottiene
la soluzione.
\begin{gather*}
    x = (A + Bt)e^{\beta t} 
\end{gather*}

\section{Cosa succede nei SDR non inerziali(Parte difficile)}
Accade spesso che la scelta di un SDR non inerziale sia fisicamente
più chiara e semplice e vale quindi la pena di discuterlo. Bisogna cambiare
però il modo in cui si utilizzano le leggi della dinamica.
Con le leggi della trasformazione si era già visto la trasformazione
dell'accelerazione e della velocità dei SDR non inerziali, definite quindi
l'accelerazione di trascinamento e di Coriolis per cui si esprime ( si chiama SDR mobile il SDR non inerziale):
\begin{gather*}
    \vec{a} = \vec{a}' + \vec{a}_t + \vec{a}_{co}    
\end{gather*}
Il sistema di riferimento S osserva quindi che le leggi della dinamica
sono valide e quindi ogni forza è associata ad una interazione a distanza o
di contatto per cui ottiene $\vec{F}_{tot} = m\vec{a}$. Il sistema di riferimento
$S'$ non riesce invece a determinare l'accelerazione. In $S'$ dunque si prende
la massa e l'accelerazione che è misurata dall'osservatore di $S'$ossia
$\vec{a}'$. Il sistema $S$ invece sa che:
\begin{gather*}
    m\vec{a}' =m(\vec{a} - \vec{a}_t - \vec{a}_{co}   )  
\end{gather*} 
L'osservatore $S'$ osserva che non gli torna l'accelerazione e dunque
ipotizza che ci siano delle accelerazioni (e quindi delle forze) immaginarie
che non riesce per qualche motivo a vedere ma che agiscono e modificano
il moto. 
La forza di trascinamento e la forza di Coriolis:
\begin{gather*}
    \vec{F}_t = -m\vec{a}_t, \qquad \vec{F}_{co} = -m \vec{a}_{co}    
\end{gather*}
L'osservatore di $S'$ riprendendo la definizione operativa di forza col
dinamometro osserva che quelle forze apparenti si comportano come
delle forze vera ma non sono forze che sono dovute ad interazione e non
rispettano dunque il terzo principio. La forza di trascinamento in particolare
dipende proprio dal sistema di riferimento e sostanzialmente questa
formula diventa il nuovo secondo principio della dinamica:
\begin{align}
    \boxed{m\vec{a}' = \vec{F} + \vec{F}_t + \vec{F}_{co} }   
\end{align}
Dati i riferimenti di $S$ e $S'$ allora considerate le componenti dell'accelerazioni
e la distanza dal punto P che vogliamo misurare si ottengono le espressioni
per le forze di trascinamento e di Coriolis:
\begin{gather*}
    \vec{F}_t = -m\vec{a}_t = -m \vec{a}_{O} -m\dot{\vec{\omega} } \times (P - O') -m\vec{\omega} \times(\vec{\omega} \times (P - O) ) \\
    \vec{F}_{co} = - m\vec{a}_{co} = -2m\vec{\omega} \times \vec{v}'  
\end{gather*}

\subsection{Il caso del pendolo sul treno}
\begin{wrapfigure}{r}{0.4\textwidth}
    \centering
    \label{FIg 34}
    \caption{Pendolo nel treno}
    \begin{tikzpicture}
        \filldraw(0, 0) circle (1pt);
        \draw(0, 0) -- (1, 2);
        \draw[->, very thick](0, 0)  -- (0.5, 1) node[at end, left] {$\vec{T}$};
        \draw(0, 0) -- (-0.5 , -1);
        \draw[->, very thick](0, 0) -- (-1, 0) node[at end, above] {$-m\vec{a}_t$};
        \draw[->, very thick](0, 0) -- (0, -1) node[at end, right] {$m\vec{g}$};
        \draw[dashed](1, 2) -- (1, -1);
        \draw(1, 1) arc (270: 245: 1) node[midway, below] {$\alpha$};
    \end{tikzpicture}    
\end{wrapfigure} 
Con $\vec{\omega} = 0$ e $\vec{a}_{O} const$, l'osservatore su di un treno osserva che il dinamometro che utilizza per 
misurare l'accelerazione osserva che la forza di trascinamento è opposta rispetto
all'accelerazione di trascinamento (ossia l'accelerazione del treno).
Se al posto del dinamometro metto un pendolo, nel caso in cui il treno 
abbia la stessa $\vec{a}_t$ e lascio oscillare il pendolo. Qui ci sono
diverse forze in gioco, una di queste è quella di trascinamento che fa inclinare il pendolo
all'indietro.
Impostando il problema dal punto di vista trigonometrico si ottiene:
\begin{gather*}
    \sin\alpha = \frac{\left| \vec{a}_t  \right| }{\sqrt{g^{2} + a_t^{2} } }
\end{gather*}
Sono in grado di definire il periodo di questo pendolo con accelerazione costante?
Utilizzo $\vec{g}'$ dato dalla composizione della forza peso e dall'accelerazione
di trascinamento, il cui modulo sarà:
\begin{align}
    \left| \vec{g}'  \right| = \sqrt{g^{2} + a_t^{2}}  
\end{align}


\end{document}
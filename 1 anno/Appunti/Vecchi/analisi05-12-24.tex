\documentclass[a4paper, oneside]{article}
\usepackage{graphicx}
\usepackage{amsthm}
\usepackage{amsmath}
\usepackage[a4paper,
            bindingoffset=0.2in,
            left=2cm,
            right=2cm,
            top=2cm,
            bottom=2cm,
            footskip=.25in]{geometry}
\usepackage[italian]{babel}
\usepackage{pgfplots}
\usepackage{tabularx}
\usepackage{wrapfig}
\graphicspath{ {./images/} }
\usetikzlibrary{datavisualization}
\usetikzlibrary{datavisualization.formats.functions}
\pgfplotsset{width=10cm,compat=1.9}

\title{Analisi}
\author{Tommaso Miliani}
\date{05-12-24}

\begin{document}
\theoremstyle{definition}
\theoremstyle{theorem}
\theoremstyle{lemma}

\newtheorem{definition}{Definizione}[section]
\newtheorem{theorem}{Teorema}[section]
\newtheorem{lemma}{Proposizione}[theorem]

\maketitle

\section{Integrali impropri}
Supponendo che io abbia una funzione continua, il suo integrale 
è esattamenta l'area del sottografico: il che è  ragionevolmente calcolabile.
SI potrebbe però pensare in linea teorica sottoinsiemi del piano
illimitati: ossia che non riesco a racchiudere in zone limitate di piano
\begin{align*}
    f: [1, +\infty] \to R, f(x) =\frac{1}{x}
\end{align*}
Posso considerarare il suo sottografico per cui $x \geq 1$ e $ 0 \leq y \leq \frac{1}{x}$.
La domanda è questa: l'area è finita o infinita?
Idealmente sto facendo:
\begin{align*}
    \int_{1}^{+\infty } \frac{1}{x} \ dx 
\end{align*}
Però non posso definirlo con Riemann poiché non posso fare partizioni
finite per un intervallo infinito poiché altrimenti dovrei fare una somma
con almeno una partizione di lunghezza infinita e questo perde di
significato. \\
Posso integrare fino ad un certo punto ed integrare fino a $c$ con $c > 1 $ e quindi
so fare:
\begin{align*}
    \int_{1}^{c} \frac{1}{x} \ dx
\end{align*}
Posso fare ora il $\lim_{c \to \infty}$ quindi:
\begin{align*}
    \lim_{c \to \infty } \int_{1}^{c} \frac{1}{x} \ dx. 
\end{align*} 
Per cui diventa:
\begin{gather*}
    \int_{1}^{c} \frac{1}{x} \ dx = F(c) - F(1) \\
    ossia \ \lim_{c \to \infty } \ln(c) - 0 = +\infty . 
\end{gather*}

Con un'altro esempio si vede invece che l'integrale di una funzione che
tende all'infinito non sempre ha un area infinita:
\begin{align*}
    f: [1, \infty ] \to R \\
    f(x) = \frac{1}{x^2} 
\end{align*}
Il suo integrale associato diventa quindi:
\begin{align*}
    \lim_{c \to \infty } \int_{1}^{c} \frac{1}{x^2 } \ dx.
\end{align*}
La mia primitiva è dunque $-\frac{1}{x}$ (è solo una delle primitive),
in ogni caso diventa:
\begin{align*}
    \int_{1}^{c} \frac{1}{x^2 } \ dx = \lim_{c \to \infty } -\frac{1}{c} + \frac{1}{1} = 1. 
\end{align*}

\begin{definition}[Integrale improprio]
    Sia $a \in R$ e sia $f: [a, +\infty ] \to R$ e supponiamo che $f$ sia
    tale che sia integrabile in $[a, c] \ \forall c > a$. Se esiste finito
    o infinito il limiti per $ c \to \infty $ dell'int:
    \begin{gather}
        \int_{a}^{c} f(x) \ dx
    \end{gather}  
    Allora $f(x)$ è integrabile in senso improprio nell'intervallo
    $[a, c]$ e si pone:
    \begin{gather}
        \int_{a}^{+\infty } f(x) \ dx = \lim_{c \to +\infty } \int_{a}^{c} f(x) \ dx   
    \end{gather}
    Inoltre se il limite è finito allora l'integrale è convergente
    altrimenti divergente. Se invece il limite non esiste allora l'integrale
    è indeterminato.
\end{definition}

Gli esercizi sugli integrali impropri solitamente chiedono di
il carattere dell'integrale improprio. Un esempio è:
\begin{align*}
    \int_{1}^{+\infty } \frac{1}{x^2  + \ln^2 (x)} \ dx 
\end{align*}

\section{Carattere degli integrali impropri di funzioni con segno costante}
\begin{lemma}[Carattere degli integrali impropri di funzioni con segno costante]
    Se  $f: [a, +\infty ] \to R$ è integrabile in $[a, c] \ \forall c > a$ e
    $f(x) \geq 0 \ \forall x geq a$ allora l'integrale di f(x) è ben
    definito ed esiste e può essere finito o infinito.
\end{lemma}
\begin{proof}
    La funzione che associa $c \to \int_{a}^{c} f(x) \ dx$ è costante e allora
    $c_1, c_2 > 0$ con $c_2 < c_1$  faccio:
    \begin{gather*}
        \int_{a}^{c_2} - \int_{a}^{c_1} f(x) \ dx = \int_{a}^{c_1} f(x) \ dx + \int_{a}^{c_2} f(x) \ dx - \int_{a}^{c_1} f(x) \ dx
    \end{gather*}
    Quindi si ha che:
    \begin{gather*}
        \int_{a}^{c_2} f(x) \ dx \geq \int_{a}^{c_1} f(x) \ dx \Rightarrow \int_{c_1}^{c_2} f(x) \ dx  \geq 0.  
    \end{gather*}
\end{proof}

La proposizione vale anche se $f(x) \leq 0$ in $[a, +\infty]$ ed in
questo caso l'integrale improprio può essere un numero oppure
divergere a $-\infty $.
La situazione del tutto speculare è quella per $[-\infty , a]$.
GLi integrali da $.\infty a +\infty $ è un integrale non ben definito
poiché si potrebbe trovare valori diversi dell'integrale se si fa crescere
più velocemente un interevallo rispetto all'altro.

\begin{theorem}
    Siano $f, g$ due funzioni integrabili e tali che: $0 \leq f(x) \leq g(x)$:
    allora:
    \begin{gather}
        Se \ \int_{a}^{+\infty } g(x) \ dx \ converge \\
        Allora \ \int_{a}^{+\infty } f(x) \ dx \ converge. 
    \end{gather}
    Analogamente se
    \begin{gather}
        \int_{a}^{+\infty } g(x) \ dx \ diverge \\
        Allora \ \int_{a}^{+\infty } f(x) \ dx \ diverge.  
    \end{gather}
\end{theorem}

\begin{proof}
    Riprendendo:
    \begin{align*}
        \int_{1}^{+\infty } \frac{1}{x^2  + \ln^2 (x)} \ dx 
    \end{align*}
Voglio dimostrare che questo integrale converge, allora prendo : $\frac{1}{x^2 }$ come
$g(x)$ e quindi, dal momento che
\begin{align*}
    \int_{1}^{+\infty } \frac{1}{x^2 } \ dx = 1 
\end{align*}    
Allora anche il primo integrale converge poiché io ho imposto
che $0 \leq f(x) \leq g(x)$.
\end{proof}


\end{document}
\documentclass[a4paper, oneside]{article}
\usepackage{graphicx}
\usepackage{amsthm}
\usepackage{amsmath}
\usepackage[a4paper,
            bindingoffset=0.2in,
            left=2cm,
            right=2cm,
            top=2cm,
            bottom=2cm,
            footskip=.25in]{geometry}
\usepackage[italian]{babel}
\usepackage{pgfplots}
\usepackage{tabularx}
\usepackage{wrapfig}
\graphicspath{ {./images/} }
\usetikzlibrary{shapes.geometric}
\usetikzlibrary{datavisualization}
\usetikzlibrary{datavisualization.formats.functions}
\pgfplotsset{width=10cm,compat=1.9}

\title{Fisica}
\author{Tommaso Miliani}
\date{11-03-25}

\begin{document}
\theoremstyle{definition}
\theoremstyle{theorem}
\theoremstyle{lemma}

\newtheorem{definition}{Definizione}[section]
\newtheorem{theorem}{Teorema}[section]
\newtheorem{lemma}{Proposizione}[theorem]

\maketitle

\section{La fi(si)ca che ci piace}
Nel caso di due masse, il cenetro di massa si trova tra le due masse. 
Nel caso in cui una delle due masse sia molto maggiore dell'altra allora
il centro di massa sarà spostato più verso la massa maggiore. Ma se ci fossero molte più masse?
Allora nel caso di più masse si ha che il centro di massa risiede 
sull'asse di simmetria delle masse. Si chiama allora asse di simmetria
un asse sul quale ogni massa è tale per cui c'è sempre un'altra massa a distanza uguale. \\
Un sistema si dice omogeneo se la sua densità è costante, altrimenti non è omogeneo:
nel caso del vettore del centro di massa allora essendo:
\begin{gather*}
    \vec{r}_C = \frac{\int_{V}\vec{r} \rho dV  }{\int_{V}\rho dV} \\
    \vec{r}_C = \frac{\int_{V}\vec{r} dV  }{\int_{V} dV} 
\end{gather*}
\begin{wrapfigure}{r}{0.4\textwidth}
    \centering
    \label{FIg}
    \caption{}
    \begin{tikzpicture}
        \draw(0, 0) -- (3, 0) node[at start, below] {$A$} node[at end, below] {$B$};
        \draw(0, 0) -- (1, 2);
        \draw(1, 2) -- (3, 0);
        \draw(0.05, 0.1) -- (2.95, 0.1);
        \draw(1, 2) -- (1.5, 0);
    \end{tikzpicture}    
\end{wrapfigure}
Nel caso di simmetrie o strutture note come nel caso di una figura piana di cui
conosco gli assi di simmetria come un quadrato, il centro degli assi di simmetria
del quadrato è proprio il centro del quadrato. Nel caso di figure meno semplici come
già il triangolo la posizione del centro di massa non è intuitivo
poiché il centro del triangolo non è così evidente. Scegliendo uno dei lati
possiamo prenderne una fetta parallela al lato scelto posso indicare
il centro di questo pezzetto e ripetere il procedimento fino a tracciare la mediana
rispetto all'angolo opposto, faccio lo stesso per tutti gli altri lati
e quindi, essendo che tutte e tre le mediane si incontrano esattamente a due terzi
della lunghezza delle mediane, allora quello è il entro di massa del traingolo. 

\subsection{Le figure solide}
\begin{wrapfigure}{r}{0.4\textwidth}
    \centering
    \label{}
    \caption{}
    \begin{tikzpicture}
        \draw[->](-1, 0) -- (5, 0) node[at end, below] {$x$};
        \draw(0, 0) -- (3, 1.5);
        \draw(0, 0) -- (3, -1.5);
        \node[ellipse,
        draw,
	    minimum width = 1cm, 
	    minimum height = 3cm] (e) at (3,0) {};
        \draw[dashed](3, -1.5) -- (3, 1.5);
        \node[ellipse,
        draw,
	    minimum width = 0.7cm, 
	    minimum height = 1.45cm] (e) at (1.5,0) {};
    \end{tikzpicture}    
\end{wrapfigure}
Nel caso di una figura solida come un cono, lo scopo è cercare di ridurre
gli integrali bi-tridimensionali in integrali unidimensionali con certe approssimazioni
e con certi procedimenti. Possiamo fare come nel caso del traingolo 
delle piccole fette attraverso i piani. ALlora l'integrale da risolvere
sarà (posto che ovviamente sia omogeneo):
\begin{gather*}
    x_c = \frac{\int_{V} xdV}{\int_{V} dV} 
\end{gather*}
Allora il volume da colcolare, data una sezione $dx$ del cono
\begin{gather*}
    dV = 2\pi rdx
\end{gather*}
E allora essendo che conosco la distanza dalla cima ($h$)
posso ricavare quel raggio come:
\begin{gather*}
    r = \frac{R}{h}x \\
    dV = \pi \frac{R^{2} }{h^{2} }x^{2}dx 
\end{gather*} 
SOno passato da un integrale tridimensionale fino ad un integrale
unidimensionale:
\begin{gather*}
    \vec{r}_c = \frac{\int x\left(\pi \frac{R^{2} }{h^{2} }x^{2}\right)}{\int \left( \pi \frac{R^{2} }{h^{2} }x^{2} \right) } dx
\end{gather*}
Allora risolvendo:
\begin{gather*}
    x_c = \frac{\left. \frac{x^{4} }{4} \right|^{h}_{0}  }{\left. \frac{x^{3} }{3} \right|^{h}_{0}  } = \frac{3}{4}h
\end{gather*}

\section{Quantità di moto}
\begin{wrapfigure}{r}{0.4\textwidth}
    \centering
    \label{}
    \caption{}
    \begin{tikzpicture}
        
    \end{tikzpicture}    
\end{wrapfigure}
La quantità di moto è data da:
\begin{gather*}
    \vec{q} = m\vec{v}  
\end{gather*}
la quantità di moto del centro di massa è proprio la somma dei contributi
di tutte le quantità di moto e quindi 
\begin{gather*}
    \vec{Q} = \sum_{i = 1}^{n} m_i \vec{v}_i   
\end{gather*}
Se volessi ottener la velocità del centro di massa allora
dovrei derivare rispetto al tempo la posizione del centro di massa:
\begin{gather*}
    \vec{v}_c = \dot{r}_c = \frac{\sum_{i = 1}^{n} m_i \vec{v}_i }{M}  
\end{gather*}
Se si deriva invece la quantità di moto del sistema si ottiene la detrivata
dei singoli punti materiali oppure la somma delle masse per l'accelerazione
dei singoli punit:
\begin{gather*}
    \dot{\vec{Q} } = \sum_{i = 1}^{n}\dot{\vec{q}_i } \sum_{i = 1}^{n}m_i \vec{a}_i     
\end{gather*}
si ottiene proprio la risultante della forza che agisce sul corpo:
\begin{gather*}
    \dot{\vec{Q} } = \sum_{i = 1}^{n}\vec{F}_i    
\end{gather*}
quindi dobbiamo metterla in una forma più semplice per poterla utilizzare: 
si distinguono allora forze interne e forze esterne: si chiamano forze interne
tutte le forze dovute all'interazione tra i corpi del sistema stesso
mentre quelle esterne sono quelle che provengono all'esterno e sia le forze apparenti.
\begin{gather*}
    \dot{\vec{Q} } = \sum_{i = 1}^{n}\vec{F}_i   = \sum_{i = 1}^{n}\vec{F}_i^{(INT)}  + \sum_{i = 1}^{n}\vec{F}_i^{(EXT)} 
\end{gather*}
Per le forze interne devo considerare anche il terzo principio per ogni punto, quindi 
per ogni forza applicata ad un punto ce ne è una uguale e contraria:
\begin{gather*}
    \boxed{\sum_{i = 1}^{n}\vec{F}_i^{(INT)}  =  \sum_{i = 1}^{n}\sum_{j = 1, j \neq i}^{n}\vec{F}_{i, j} = 0 }   
\end{gather*}
Si ottiene allora la prima equazione cardinale della dinamica:
\begin{gather*}
    \dot{\vec{Q} } = \sum_{i = 1}^{n}\vec{F}^{(EXT)}_i = \vec{F}^{(EXT)} 
\end{gather*}
che permette di descrivere i sistemi più complessi con descrizioni molto più semplici. Allora
si arriva anche alla seconda equazione del centro di massa:
\begin{gather*}
    \boxed{\vec{F} = M\vec{a}_c} 
\end{gather*}

\subsection{Il sistema isolato}
La prima equazione cardinale della dinamica mi dice che la forza esterna è 
zero e che la quantità di moto è conservata poiché
\begin{gather*}
    \dot{\vec{Q}} = 0, \qquad \vec{Q} = \text{costante}  
\end{gather*}
\begin{wrapfigure}{r}{0.4\textwidth}
    \centering
    \label{}
    \caption{}
    \begin{tikzpicture}
        \draw(0, 0) rectangle (1, 1) node[midway] {$m_1$};
        \draw(3, 0) rectangle (4, 1) node[midway] {$m_2$};
        \draw[decoration={aspect=0.3, segment length=1.5mm, amplitude=1.3mm,coil},decorate,opacity=0.9] (1, 0.5) -- (3,0.5);
        \draw[->](-1, 0) -- (-0.5, 0) node[at end, below] {$x$};
        \draw[->](-1, 0 ) -- (-1, 0.5) node[at end, left] {$y$};
    \end{tikzpicture}    
\end{wrapfigure}
Questo sistema non riesco ad esprimerlo mediante f = ma e nemmono attraverso
la conservazione dell'energia poiché ho due gradi di libertà. Essendo 
che la forza elastica adesso è una forza interna (e quelle esterne sono la
forza peso e la normale) e lungo l'orizzontale si conserva la quantità
di moto e la conservazione dell'energia che mi danno i due gradi di libertà.
\begin{gather*}
    E  = \frac{1}{2}m_1v_1^{2} + \frac{1}{2}m_2v_2^{2} + \frac{1}{2}k(x_2 - x_1 - l_0)^{2}  
\end{gather*}
Allora la quantità di moto:
\begin{gather*}
    Q_x = m_1v_1 + m_2v_2
\end{gather*}
Una vota che ho le condizioni iniziali posso risolvere il sistema. 

\end{document}
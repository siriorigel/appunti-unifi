\documentclass[a4paper, oneside]{article}
\usepackage{graphicx}
\usepackage{amsthm}
\usepackage{amsmath}
\usepackage[a4paper,
            bindingoffset=0.2in,
            left=2cm,
            right=2cm,
            top=2cm,
            bottom=2cm,
            footskip=.25in]{geometry}
\usepackage[italian]{babel}
\usepackage{pgfplots}
\usepackage{tabularx}
\usepackage{tikz}
\usepackage{wrapfig}
\usepackage{color}
\definecolor{page}{rgb}{0.129,0.157,0.212}
\pagecolor{page}
\color{white}
\graphicspath{ {./images/} }
\usetikzlibrary{shapes.geometric}
\usetikzlibrary{datavisualization}
\usetikzlibrary{datavisualization.formats.functions}
\pgfplotsset{width=10cm,compat=1.9}

\title{Fisica}
\author{Tommaso Miliani}
\date{03-04-25}

\begin{document}
\theoremstyle{definition}
\theoremstyle{theorem}
\theoremstyle{lemma}

\newtheorem{definition}{Definizione}[section]
\newtheorem{theorem}{Teorema}[section]
\newtheorem{lemma}{Proposizione}[theorem]
\newtheorem{example}{Esempio}[section]

\maketitle

\section{Estensione della macchina di Fletcher con la carrucola reale}
\begin{wrapfigure}{r}{0.4\textwidth}
    \centering
    \caption{Macchina di Fletcher reale}
    \begin{tikzpicture}
        \draw(0, 0) -- (3, 0);
        \draw(3, 0) -- (3, -2);
        \draw(0, 0) rectangle (1, 1) node[midway] {$M_1$};
        \draw(1, 0.5) -- (3.3, 0.5);
        \draw(3.25, 0.25) circle (0.25);
        \draw(3.5, 0.3) -- (3.5, -1);
        \draw(3, -1) rectangle (4, -2) node[midway] {$M_2$};
    \end{tikzpicture}    
\end{wrapfigure}
Nella macchina di Fletcher con la carrucola reale (posto che non ci siano
attriti) allora ho tre corpi con massa  collegati con una fune inestensibile e priva
di massa ed il sistema ha 1 grado di libertà per ogni oggetto ma non sono
indipendenti: la rotazione della carrucola ed il movimento delle masse dipendono l'uno
dall'altro. SI suppone inoltre che la carrucola non oscilli. \\
PReso allora un sistema di riferimento nell'origine della prima massa, posso allora
applicare il primo principio per cui per il primo oggetto:
\begin{gather*}
    \vec{T} = T\hat{i} \\
    m_1 \ddot{x}_1 = T  
\end{gather*}
Mentre per il secondo oggetto (dato che la carrucola non è ideale
la tensione non è la stessa e dunque la tensione che risente il secondo oggetto
è diversa rispetto a quella della massa 1):
\begin{gather*}
    \vec{T}' = T'\hat{j} \\
    m_2\vec{g} = -m_2g\hat{j} \\
    m_2\ddot{y}_2 = T'-m_2g   
\end{gather*}
Scegliendo ora un polo di riduzione per la carrucola, su di essa agiscono
le seguenti forze (la fune sollecita la carrucola e dunque se è inestensibile):
le due tensioni.

LA schematizzazione delle masse e della carrucola è data da:
\begin{gather*}
    \begin{tikzpicture}
        \draw(0, 0) -- (2, 0);
        \draw(0.5, 0) rectangle (1.5, 1);
        \draw[->](1.5, 0.5) -- (2.5, 0.5) node[at end, above] {$\vec{T}$};
        \draw[->](1, 0) -- (1, -1) node[at end, below] {$m\vec{g}$};
        \draw[->](1, 1) -- (1, 2) node[at end, right] {$\vec{N}$};
        \draw(8, 0) circle (1);
        \draw[->](8, 1) -- (7, 1) node[at end, above] {$\vec{T}$};
        \draw[->](9, 0) -- (9, -1) node[at end, right] {$-\vec{T}'$};
        \filldraw (8, 0) circle (1pt) node[anchor = north] {$C$};
        \filldraw (8, 1) circle (1pt) node[anchor = south] {$A$};
        \filldraw (9, 0) circle (1pt) node[anchor = west] {$B$};
        \draw[->](8, 0) -- (8, -0.5) node[at end, right] {$M\vec{g}$};
        \draw[dashed] (7, 0) -- (8, 0) node[midway, above] {$R$};
    \end{tikzpicture}    
\end{gather*}
Applicando ora la seconda cardinale alla carrucola e conosciute le distanze tra
i punti di contatto della corda ed il centro:
\begin{gather*}
    (A - C) = R\hat{j} \\
    (B - C) = R\hat{i}  
\end{gather*}
E allora
\begin{gather*}
    ((A - C) \times \vec{T} + (B - C) \times (-\vec{T}' )) \cdot  \vec{R} ? I_C \dot{\omega} 
\end{gather*}
Allora dato che $\vec{\omega} = \omega\hat{k}$:
\begin{align}
    TR - T'R = I_C \dot{\omega}
\end{align}   
Non ci resta che determinare allora chi è la derivata di $\omega$:
\begin{gather*}
    \omega = \dot{\phi} , \dot{\omega} = \ddot{\phi}
\end{gather*}
ALlora
\begin{gather*}
    R\dot{\phi} = -\dot{x}_1
\end{gather*}
Non ci resta ora che sostituire nelle relazioni:
\begin{gather*}
    m_1 \ddot{x}_1 = T \\
    -m_2 \ddot{x_1} = T' - m_2 g \\
    TR - T'R = -I_c \frac{\ddot{x}_1}{R}
\end{gather*}
Inoltre, sapendo che il momento di inerzia di un disco è 
$\frac{1}{2}MR^{2} $,
sostituendo tutto nella terza relazione si ottiene:
\begin{gather*}
    (m_1 + m_2 + \frac{M}{2})\ddot{x}_1 = m_2g
\end{gather*}
Si ricavano ora le tensioni:
\begin{align}
    T &= \frac{m_1m_2g}{m_1 + m_2 + \frac{M}{2}} \\
    T' &= \frac{m_1m_2 + \frac{m_2M}{2}}{m_1 + m_2 + \frac{M}{2}}
\end{align}
\section{La macchina di atwood reale attaccata al soffitto}
\begin{wrapfigure}{r}{0.4\textwidth}
    \centering
    \caption{MAcchina di atwood reale}
    \begin{tikzpicture}
        \draw(0, 0) circle (2);
        \filldraw(0, 0) circle (1pt) node[anchor = north] {$C$};
        \filldraw(-2, 0) circle (1pt) node[anchor = east] {$A$};
        \filldraw(2, 0) circle (1pt) node[anchor = west] {$B$};
        \draw[dashed](-2, 0) -- (0, 0) node[midway, below] {$R$};
        \draw[->](0, 0) -- (0, 1) node[at end, left] {$y$};
        \draw[->](0, 0) -- (1, 0) node[at end, below] {$x$};
        \draw[->](0, 0) -- (0, -1) node[at end, right] {$M\vec{g}$};
        \draw[->](0, 0) -- (0.7, 1.3) node[at end, right] {$\vec{N}$};
        \draw(-2, 0) -- (-2, -2.5);
        \draw(-1.75, -2.5) rectangle (-2.25, -3) node[midway] {$m_1$};
        \draw(-2, 0) -- (-2, -1) node[midway, left] {$-\vec{T}_1$};
        \draw(-2, -2.5) -- (-2, -1.5) node[midway, left] {$\vec{T}_1$};
        \draw(-2, -3) -- (-2, -4) node[at end, right] {$m_1\vec{g}$};
        \draw(2, 0) -- (2, -3);
        \draw(1.75, -3) rectangle (2.25, -3.5) node[midway] {$m_2$};
    \end{tikzpicture}    
\end{wrapfigure}
Per la prima massa posso applicare la prima cardinale e dunque scrivere
(considerata la carrucola imperfetta, il filo inestensibile e le tensioni diverse):
\begin{gather*}
    m_1 \ddot{y}_1 = T_1 - m_1 g \\
    m_2 \ddot{y}_2 = T_2 - m_2 g
\end{gather*}
Dalla prima cardinale so solo le condizioni di equilibrio ma se voglio
sapere il movimento della carrucola e trovare le tensioni non mi basta.
La prima massa mi fa girare la carrucola nel verso giusto (lungo l'angolo $\phi$ in senso antiorario)
e quindi il contributo al momento delle forze e al momento di inerzia è positivo
mentre l'altra forza la fa girare in senso opposto.
\begin{gather*}
    T_1 R - T_2 R = I_C \ddot{\phi} \\
    R\ddot{\phi} = -\ddot{y}_2
\end{gather*}
SI sostituisce come prima per la macchina di Atwood e si risolve il sistema.
\begin{align}
    T_1 &= m_1 g\frac{2m_2 + \frac{M}{2}}{m_1 + m_2 + \frac{M}{2}}\\
    T_2 & = m_2g\frac{2m_1 + \frac{M}{2}}{m_1 + m_2 + \frac{M}{2}}
\end{align}


\section{Il pendolo composto}
Per piccole oscillazioni è un moto armonico nel pendolo semplice
con periodo
\begin{gather*}
    T = 2\pi \sqrt{\frac{L}{g}} 
\end{gather*}
\begin{wrapfigure}{r}{0.4\textwidth}
    \centering
    \caption{Il pendolo composto}
    \begin{tikzpicture}
        \filldraw(0, 0) circle (1pt) node[anchor = west] {$O$};
        \filldraw(1, -2) circle (1pt) node[anchor = south] {$C$};
        \draw(0, 0) -- (1, -2) node[midway, right] {$h$};
        \draw[->](1, -2) -- (1, -3) node[at end, left] {$M\vec{g}$};
        \draw[dashed](0, 1) -- (0, -3);
        \draw[->](0, -1) arc (-90: -60: 1) node[midway, below] {$\phi$}; 
        \draw[dashed](0, -2.2) arc (-90: 0:2.5);
        \draw[dashed](1, -2) -- (1.5, -3); 
        \draw[->](1, -3) arc(-90:-60:1) node[midway, below] {$\phi$};
    \end{tikzpicture}    
\end{wrapfigure}
Il pendolo composto (o fisico) è un oggetto fisico ed esteso di forma qualsiasi
ed il suo centro di massa non può coincidere col perno di rotazione
poiché altrimenti non sarebbe un pendolo. ANche qui come nelle carrucole prima,
la forza vincolare non so dove è diretta e la metto un po' a caso. \\
Adesso dato che ho solo un grado di libertà (che ho parametrizzato con $\phi$),
posso allora utilizzare solo un'equazione per risolvere il problema però io non
conosco né in modulo né in direzione la forza vincolare. Posso allora
prendere allora il perno come polo di riduzione. 
Allora posso esprimere il momento di inerzia come
\begin{gather*}
    -Mgh\sin\phi = I_0 \ddot{\phi} \\
    I_0 = I_C + Mh^{2} 
\end{gather*}
Per cui l'equazione del moto:
\begin{align}
    \ddot{phi} + \frac{Mgh}{I_0}\sin\phi = 0 
\end{align}
Per piccole oscillazioni diventa un moto armonico con periodo calcolabile.
come
\begin{align}
    T = 2\pi \sqrt{\frac{I_0}{Mgh}} 
\end{align}
Possiamo esprimere la normale come
\begin{gather*}
    \vec{N} + N_t \hat{t} + N_n \hat{n}   
\end{gather*}
Dove $\hat{t}$ è il versore della tangente mentre $\hat{n}$ è il versore
normale rispetto alla congiungente. Possiamo allora esprimere la prima
cardinale per le due:
\begin{gather*}
    N_t - Mg\sin\phi = Mh\ddot{\phi} \\
    N_n- Mg\cos\phi = Mh\dot{\phi}^{2} 
\end{gather*}  
Posso ottenere allora la derivata secondo di $\phi$ e quindi posso determianremi
la componente tangenziale e ottenere:
\begin{gather*}
    \ddot{\phi} = -\frac{Mgh}{I_0}\sin\phi
\end{gather*}
Per la componente tangenziale:
\begin{align}
    N_t = Mg \sin\phi - Mh\frac{Mgh}{I_0}\sin\phi = Mg\sin\phi\left(1 - \frac{Mh^{2} }{I_0}\right)
\end{align}
Nel caso del pendolo semplice la tangenziale è zero. Moltiplicando per entrambe le parti
di $\ddot{\phi}$ per integrare e rimuovere $dt$ si ottiene per le condizioni
iniziali in $t = 0$, $\phi = 0$ e $\dot{\phi} = \Omega$:
\begin{gather*}
    \dot{\phi}^{2} = \Omega_0^{2} - \frac{2Mgh}{I_0}(1 - \cos\phi)  
\end{gather*} 
sostituendo nell'espressione per la normale:
\begin{align}
    N_n = Mg\cos\phi + Mh\left(\Omega_0^{2} - \frac{2Mgh}{I_0}(1 - \cos\phi)\right)
\end{align}
Nel caso in cui l'angolo iniziale fosse noto per esempio $\phi = \phi_0$:
\begin{align}
    N_n = Mg\cos\phi + Mh\frac{2Mgh}{I_0}(\cos\phi - \cos\phi_0) = Mg\cos\phi \left(1 + \frac{2Mh^{2} }{I_0}\right) - \frac{2M^{2}h^{2}g}{I_0}\cos\phi_0
\end{align}

\end{document}
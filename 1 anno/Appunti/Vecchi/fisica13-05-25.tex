\documentclass[a4paper, oneside]{article}
\usepackage{graphicx}
\usepackage{amsthm}
\usepackage{amsmath}
\usepackage[a4paper,
            bindingoffset=0.2in,
            left=2cm,
            right=2cm,
            top=2cm,
            bottom=2cm,
            footskip=.25in]{geometry}
\usepackage[italian]{babel}
\usepackage{pgfplots}
\usepackage{tabularx}
\usepackage{tikz}
\usepackage{wrapfig}
\usepackage{color}
\definecolor{page}{rgb}{0.129,0.157,0.212}
\pagecolor{page}
\color{white}
\graphicspath{ {./images/} }
\usetikzlibrary{shapes.geometric}
\usetikzlibrary{datavisualization}
\usetikzlibrary{datavisualization.formats.functions}
\pgfplotsset{width=10cm,compat=1.9}

\title{Fisica}
\author{Tommaso Miliani}
\date{13-05-25}

\begin{document}
\theoremstyle{definition}
\theoremstyle{theorem}
\theoremstyle{lemma}

\newtheorem{definition}{Definizione}[section]
\newtheorem{theorem}{Teorema}[section]
\newtheorem{lemma}{Proposizione}[theorem]
\newtheorem{example}{Esempio}[section]

\maketitle

\section{La forza di gravità di un guscio sferico}
\begin{wrapfigure}{r}{0.5\textwidth}
    \centering
    \caption{}
    \begin{tikzpicture}
        \draw(0, 0) circle (3);
        \draw(0, 0) circle (2.9);
        \filldraw (5, 0) circle (1pt) node[anchor = south] {$m$};
        \draw[->](0, 0) -- (6, 0) node[at end, below] {$x$};
        \draw(0, 0) -- (1.7, 2.45) node[midway, above] {$r$};
        \draw[dashed](1.7, 2.45) -- (5, 0) node[midway, above] {$\rho$};
        \draw(1, 0) arc (0:55:1) node[midway, right] {$\theta$};
        \filldraw(0, 0) circle(1pt) node[anchor = east] {$C$};
        \draw[|-|](-3, -0.75) -- (5, -0.75) node[midway, below] {$r_B$};
        \draw[|-|](3, -0.25) -- (5, -0.25) node[midway, below] {$r_A$};
    \end{tikzpicture}    
\end{wrapfigure}
Per trovare la forza di gravità agente su di un punto esterno
od interno rispetto ad un guscio sferico di spessore infinitesimo si ottiene
considerando la densità superficiale $\sigma$:
\begin{gather*}
    \sigma = \frac{M}{4\pi r^{2} }
\end{gather*} 
Data solo la legge di gravitazione universale dobbiamo allora
ridurre il problema e cercare di ridurre il problema ad una situazione in cui
è facile integrare e quindi scelgo un punto sulla superficie che per definizione avrà
distanza $r$ dal centro e quindi data l'energia potenziale del punto $m$, 
\begin{gather*}
    V = -\frac{Gm_1m_2}{r_{1, 2}}
\end{gather*}
Tutti i punti che hanno la stessa distanza dal punto $P$ contribuiranno allo stesso modo all'energia potenziale:
quindi danno un contributo all'energia potenziale pari a:
\begin{gather*}
    dV = -\frac{Gm \ dM}{\rho}, \qquad dM = \sigma \ dS, \qquad dS = 2\pi r \sin\theta \cdot  r d\theta
\end{gather*}
Posso approssimare lo spessore come un cilindretto e quindi posso calcolare 
la superficie. Introducendo l'angolo $\theta$ posso allora definire il differenziale
della massa come
\begin{gather*}
    d M = 2\sigma r^{2} \sin\theta \ d\theta \Rightarrow d M = \frac{M}{2} \cdot  \sin\theta \ d\theta
\end{gather*}
In pratica il $d\theta$ è esattamente lo spostamento infinitesimo del punto sulla superficie e quindi
ottengo il volume del cilindretto con altezza $r \ d\theta$. Posso allora dire che
l'energia potenziale di ogni punto sulla stessa circonferenza sarà:
\begin{gather*}
    d V = - G m \frac{M}{2} \frac{\sin\theta \ d\theta}{\rho}, \rho^{2} = r^{2} + x^{2} - 2rx\cos\theta   
\end{gather*}
Dato che $x$ è fissato, allora se differenziamo l'espressione di $\rho$ viene che 
\begin{gather*}
    2\rho \ d\rho = 2 r x \sin\theta \ d\theta 
\end{gather*}
Si ottiene allora che 
\begin{gather*}
    \frac{d\rho}{rx} = \frac{\sin\theta \ d\theta}{\rho}
\end{gather*}
Se ora facessi l'integrale dell'energia potenziale posso allora ottenere che:
\begin{gather*}
    V = -\frac{G m M}{2rx} \int d\rho = -\frac{G m M}{2rx} (r_B - r_A)
\end{gather*}
Si possono distinguere i due casi in cui il punto materiale è esterno rispetto alla superficie ma anche
quando $P'$ è interno: si danno allora due casi: un caso in cui $P$ sia esterno 
alla superficie:
\begin{gather*}
    \text{Esterno}: \left\{\begin{array}{l}
        r_A = x - r \\
        r_B = x + r
    \end{array}\right. \\
    \text{Interno}: \left\{\begin{array}{l}
        r_A = r - x \\
        r_B = x + r
    \end{array}\right.
\end{gather*}
Per un punto esterno alla superficie allora si può dire che
\begin{gather*}
    V = -\frac{G m M}{2rx} 2r = -\frac{G m M}{x}
\end{gather*}
E questa è esattamente l'energia potenziale che risente un oggetto di 
massa $m$ da un oggetto di massa $M$ che si trova esattamente sul punto $C$. Se quindi
è esterno alla superficie allora questa superficie posso considerarla come se fosse
concentrata nel suo centro.  Per un punto interno:
\begin{gather*}
    V = -\frac{G m M}{2rx} 2x = -\frac{G m M}{r}
\end{gather*}
In questo caso è posta alla distanza $r$, ossia come se il punto fosse sulla superficie
e non al suo interno. La forza gravitazionale che risente dunque al suo interno
è nulla.
\section{La gravità di una sfera piena}
\begin{wrapfigure}{r}{0.4\textwidth}
    \centering
    \caption{}
    \begin{tikzpicture}
        \draw(0, 0) circle (2);
        \draw(1, 1.68) -- (1, -1.68);
        \draw(1.2, 1.55) -- (1.2, -1.55);
        \draw(1.1, 1.65) circle (1pt) node[anchor = west] {$m$};
        \draw(1.1, 1) circle (1pt);
        \draw[->](1.1, 1) -- (1.6, 1) node[at end, above] {$\vec{N} $};
        \draw[->](0, 0) -- (0.5, 0.45) node[at end, below] {$\hat{u}_r$};
        \draw[dashed] (0, 0) -- (1.1, 1);
        \draw[->](1.1, 1) -- (0.6, 0.55) node[midway , above] {$\vec{F_g}$};
        \draw[dashed](0, 0) circle(1.45);
    \end{tikzpicture}    
\end{wrapfigure}
Divido allora in tante bucce ma ora tutte le bucce si comportano 
in modo diverso tra di loro: quelle vicine alla massas non contribuiscono 
mentre quelle più interne contribuiscono alla forza gravitazionale. La forza è dovuta
solo alla massa che si trova allora all'interno.  \\
All'interno non agisce tutta la sfera ma solo la parte della massa interna:
\begin{gather*}
    \vec{F}_g = - \frac{G m M_i}{r^{2} } \hat{u}_r, \qquad M_i = \frac{4}{3}\pi r^{3} \rho 
\end{gather*}
E quindi il rapporto tra la massa interna e quella della Terra è
\begin{gather*}
    \frac{M_i}{M} = \frac{r^{3} }{R^{3} }
\end{gather*}
Posso allora dire che che essendo un punto materiale in una guida:
\begin{gather*}
    \left\{\begin{array}{l}
        \vec{F}_g = - \frac{G m M_T}{R^{3} } (x \hat{i} + y \hat{j}  ) \\
        \vec{N} = N \hat{i}  
    \end{array}\right. \\
    \left\{\begin{array}{l}
        x) N - \frac{G m M_T}{R^{3} }x = 0 \\
        y) - \frac{G m M_T}{R^{3} }y = m\ddot{y}
    \end{array}\right.
\end{gather*}
Allora l'equazione che abbiamo per il corpo è proprio un moto armonico:
\begin{gather*}
    \ddot{y} + \frac{g}{R}y = 0
\end{gather*}
Questo si chiama pozzo di Gauss e si vede che la massa esterna non conta.

\section{Il moto dei razzi}
\begin{wrapfigure}{r}{0.4\textwidth}
    \centering
    \caption{}
    \begin{tikzpicture}
        \draw(0, 0) rectangle (2, 1);
        \draw[->](0, 0.5) -- (-1, 0.5) node[at end, above] {$\vec{u}$};
    \end{tikzpicture}    
\end{wrapfigure}
Con gli strumenti sviluppati fino ad ora si è in grado di descrivere il moto di un razzo:
In un dato istante di tempo il razzo espelle una certa quantità di massa
che genera una spinta. 
\begin{gather*}
    Q(t) = m(t) \cdot \vec{v}(t)  \\
    \vec{Q} (t + dt) = M(t + dt) \vec{v}(t + dt) + dm (\vec{u} + \vec{v}(t + dt) )  
\end{gather*}
Dove $\vec{u}$ è il vettore della velocità di espulsione del gas rispetto al razzo. Esplicitandola
di nuovo posso allora dire che rispetto al missile. 
\begin{gather*}
    \Delta \vec{Q} = \vec{Q}(t + dt) - \vec{Q}(t) = \vec{F}^{ext} \ dt    
\end{gather*} 
E quindi risolvendo si ottiene dato che $dM = -dm$:
\begin{gather*}
    \vec{F}^{ext}  \ dt = M d\vec{v} + dm \vec{u}   
\end{gather*}
Dividendo ora tutto per $dt$:
\begin{gather*}
    M\frac{d\vec{v} }{d t} = \vec{u}\frac{dM}{dt} + \vec{F}^{ext}   
\end{gather*}
Nel caso in cui il missile sia nello spazio e quindi che le forze esterne siano
nulle la nostra espressione ci dice che
\begin{gather*}
    M dv = u dM \Rightarrow \int_{t_i}^{t_f} dv = u \int_{t_i}^{t_f}\frac{dM}{M}
\end{gather*}
Allora
\begin{gather*}
    v(t_f) = v(t_i) - u\ln\frac{M(t_i)}{M(t_f)} \\
    v_f = v_i + |u| \ln \left(\frac{M_{mis} + M_{carb}(t_i)}{M_{mis} + M_{carb}(t_f)}\right)
\end{gather*}

Nel secondo caso si ha la partenza da Terra verso l'alto:
\begin{gather*}
    \vec{F}^{ext} = M\vec{g}   
\end{gather*}
Possiamo usare sempre gli stessi principi:
\begin{gather*}
    M \ dv = u \ dM - Mg \ dt \\
    dv = u \frac{dM}{M} - g \ dt \\
    \int_{t_i}^{t_f} dv = v(t_f) = \int_{t_i}^{t_f} \left(u \frac{dM}{M} - g \ dt \right)
\end{gather*}
Allora si ottiene che
\begin{gather*}
    v(t_f) = |u| \ln \frac{M_i}{M_f} - g(t_f - t_i)
\end{gather*}
Più carburante brucio e meno importante diventa l'ultimo termine in modo
tale da poter approssimare alla forma vista nel primo caso. Per avere più spinta ci conviene
avere più velocità di espulsione possibile.



\end{document}
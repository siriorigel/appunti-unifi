\documentclass[a4paper, oneside]{article}
\usepackage{graphicx}
\usepackage{amsthm}
\usepackage{amsmath}
\usepackage[a4paper,
            bindingoffset=0.2in,
            left=2cm,
            right=2cm,
            top=2cm,
            bottom=2cm,
            footskip=.25in]{geometry}
\usepackage[italian]{babel}
\usepackage{pgfplots}
\usepackage{tabularx}
\usepackage{wrapfig}
\graphicspath{ {./images/} }
\usetikzlibrary{datavisualization}
\usetikzlibrary{datavisualization.formats.functions}
\pgfplotsset{width=10cm,compat=1.9}

\title{FIsica}
\author{Tommaso Miliani}
\date{04-03-25}

\begin{document}
\theoremstyle{definition}
\theoremstyle{theorem}
\theoremstyle{lemma}

\newtheorem{definition}{Definizione}[section]
\newtheorem{theorem}{Teorema}[section]
\newtheorem{lemma}{Proposizione}[theorem]

\maketitle

\section{Il lavoro}
\begin{wrapfigure}{r}{0.4\textwidth}
    \centering
    \label{FIg 3r}
    \caption{Il lavoro di una forza su di una superficie inclinata}
    \begin{tikzpicture}
        \draw(0, 0) -- (5, 0);
        \draw(0, 0) -- (0, 2);
        \draw(0, 2) -- (5, 0);
        \draw(5, 0.2) circle (0.12);
        \draw[->](4.88, 0.2) -- (4, 0.6) node[at end, above] {$\vec{F}_0$};
    \end{tikzpicture}    
\end{wrapfigure}
Definiamo il lavoro di una forza come:
\begin{align}
    \delta L = \vec{F} \cdot \vec{dr}  
\end{align}
Dove $dr$ rappresenta lo spostamento infinitesimale mentre $\delta L$
non è un differenziale esatto. Allora possiamo
definire il lavoro totale come la somma di tutti gli infinitesimi spostamenti:
\begin{align}
    L_{AB, \vec{F}} = \sum_{i = 0}^{n}\vec{F}_i \vec{dr}_i = \int_{A, \Gamma}^{B}\vec{F}\cdot \vec{dr}   
\end{align}
Nel disegno, il lavoro compiuto dalla forza dipende dall'angolo del piano inclinato e quindi:
\begin{gather*}
    L_{\vec{F}_0} = \int_{OF}^{}\vec{F}_0 \cdot  \vec{dr} = Mg \sin\alpha L = Mgh   
\end{gather*}
Se volessi esplicitare l'integrale allora posso dividerlo lungo tutte le direzioni e quindi risolvere
la somma degli integrali. Con il concetto di integrale possiamo anche dimostrare il teorema
delle forze vive.

\subsection{IL teorema delle forze vive}
\begin{align}
    \delta L = \vec{F} \cdot \vec{dr}   
\end{align}
Se avessi più forze, ognuna di esse dovrebbe compiere un certo lavoro
che dipende dallo spostamento che causano: posso considerare quindi
la risultante delle forze applicate su di un corpo lungo una traiettoria
istante per istante come se le forze fossero tutte applicate insieme 
come somma in un unica forza:
\begin{gather*}
    \vec{F}_{tot} = \sum_{i = 0}^{n} \vec{F}_i   
\end{gather*}
Posso quindi definire il lavoro totale:
\begin{align}
    \delta L_{tot} =  \sum_{i = 0}^{n} \vec{F}_i  \cdot \vec{dr} = \sum_{i = 0}^{n} \delta L_{\vec{F}_i}  
\end{align}
Il secondo principio della dinamica ci dice che in un sistema di riferimento la risultante delle forze
produce un accelerazione e quindi:
\begin{gather*}
    \vec{F}_{tot} = m\vec{a}  
\end{gather*}
Si può allora riscrivere il lavoro iniziale come:
\begin{gather*}
    \delta L_{\vec{F}_{tot}} = m\vec{a} \cdot  dr  
\end{gather*}
Essendo la velocità la derivata rispetto al tempo del vettore posizione, allora
è proprio $\frac{dr}{dt}$ e quindi al posto di $\vec{F}$ posso mettere:
\begin{align}
    \delta L_{\vec{F}_{tot}} = m\frac{d\vec{v} }{dt} \cdot \vec{v}dt  = md\vec{v}\vec{v}  
\end{align} 
Nel differenziale però perdo l'informazione della direzione di $\vec{v}$ poiché posso 
riscrivere l'ultimo membro come
\begin{gather*}
    d\left( \frac{1}{2} m\vec{v} \cdot  \vec{v} \right) 
\end{gather*} 
Il prodotto scalare diventa allora:
\begin{gather*}
    d\left( mv^{2}  \right)
\end{gather*}
Un lavoro infinitesimo produce una variazione infinitesima di questa
quantità che prende il nome di energia cinetica. possiamo allora dire:
\begin{align}
    \delta L_{\vec{F}_{tot}} = d\left( \frac{1}{2}mv^{2} \right) = dK
\end{align}
Dove $K$ è proprio l'energia cinetica. Se io avessi da compiere
un lavoro continuo e non infinitesimo? Allora su di una traiettoria il lavoro
diventerà:
\begin{align}
    L_{AB, \Gamma} = \int_{A, \Gamma}^{B} \vec{F} \cdot  \vec{dr} = \int_{A, \Gamma}^{B} dK = \frac{1}{2}mv_{B}^{2} - \frac{1}{2}mv_{A}^{2}     
\end{align}
E allora possiamo concludere con il teorema delle forze vive che il lavoro
è proprio la variazione dell'energia cinetica:
\begin{gather*}
    L_{AB, \Gamma} = \Delta K
\end{gather*}
Questo ci dice inoltre che se io compio lavoro su di un oggetto
allora sto provocando una variazione dell'energia cinetica. Il differenziale
di $K$ dipende solo dalle condizioni iniziali e finali e non è quindi una
funzione della posizione. 

\section{Risoluzione dei problemi mediante il teorema delle forze vive}
\begin{wrapfigure}{r}{0.4\textwidth}
    \centering
    \label{Fig 2}
    \caption{Un corpo su di un piano inclinato}
    \begin{tikzpicture}
        \draw(0, 0) -- (3, 0);
        \draw(0, 2) -- (3, 0);
        \draw(2, 0) arc(180:150: 1) node[midway, left] {$\alpha$};
        \draw(0, 2.12) circle (0.12);

    \end{tikzpicture}    
\end{wrapfigure}
Posso impostare il problema nella seguente maniera: sapendo che devo trovare la
funzione della discesa dell'oggetto, questa è proprio:
\begin{gather*}
    mg\sin\alpha = m\ddot{x} \\
    x = \frac{1}{2}g\sin\alpha t^{2} 
\end{gather*}
E poi si ricava la velocità finale combinando le varie derivate.
Il metodo due è quello dell'energia cinetica con un vincolo semplice:
la forza normale è ortogonale allo spostamento e quindi non produce alcun tipo
di lavoro poiché non impedisce lo spostamento lungo la traiettoria permessa. Allora il lavoro è proprio:
\begin{gather*}
    L_{AB} = Mg\sin\alpha L \\
    v_B = v_f, \qquad v_A = 0 
\end{gather*}
Utilizzando ora il teorema delle forze vive, per cui ho solo $v_B$
\begin{gather*}
    Mg\sin\alpha L = \frac{1}{2}Mv_{f}^{2} 
\end{gather*}
E allora:
\begin{gather*}
    v_f = \sqrt{2g\sin\alpha L} 
\end{gather*}
Questo non ci da il tempo di arrivo dell'oggetto ma solo la velocità finale.

\subsection{L'attrito nel teorema delle forze vive}
La forza di attrito vale:
\begin{gather*}
    |\vec{F}_{at}| = \mu_d Mg \sin\alpha - \mu_d Mg\cos\alpha 
\end{gather*}
La velocità finale diventerà allora:
\begin{gather*}
    v_f = \sqrt{2gL(\sin\alpha - \mu_d \cos\alpha)} 
\end{gather*}
Mentre la forza peso è costante, la forza di attrito non è costante
poiché dipende dal verso in cui sta andando l'oggetto. Se analizzassimo il caso
di un oggetto che scorre giù da un piano lungo una guida allora la forza
peso non fa lavoro sui tratti orizzontali ma la forza di attrito è sempre presente
e con verso sempre contrario al moto. 

\section{Forze posizionali}
Una forza posizionale è una forza che dipende dalla posizione di un oggetto,
come la forza elastica; forze come l'attrito non so no posizionali poiché dipendono dal 
verso di applicazione della forza.
\begin{align}
    \vec{F} = \vec{F}(x, y, z)  
\end{align}

\subsection{Forze conservative}
Si distinguono le forze conservative (sottoclasse) che sono quelle forze
posizionali in cui il lavoro non dipende dal cammino seguito:
\begin{gather*}
    L_{AB, \Gamma_1} = L_{AB, \Gamma_2}, \forall \Gamma_1, \Gamma_2
\end{gather*}
La seconda definizione di queste forze è che il lavoro in un percorso
chiuso è zero (chiamata anche circuitazione):
\begin{align}
    \oint \vec{F} \cdot  \vec{dr} = 0  
\end{align}
Lungo una traiettoria chiusa, che io la percorra in un senso
o nell'altro, non cambia il lavoro, cambia solo di segno per cui
la sua circuitazione sarà sempre 0. e quindi la prima implica la seconda e
viceversa. \\
Terza definizione: $\exists V$ energia potenziale tale che:
\begin{align}
    \delta L &= -dV \\
    L_{AB, \Gamma} &= V(A) - V(B), \qquad \forall \Gamma
\end{align}
Abbiamo un differenziale non esatto eguagliato ad un differenziale esatto. 
Lungo una traiettoria possiamo identificare un punto di riferimento $P_0$ 
da cui facciamo passare il psorcorso $A, P_0, B$ e quindi si ottiene che la
ciruitazione di questo percoros sia:
\begin{gather*}
    L_{AP_0} + L_{P_0B} + L_{BA} = 0
\end{gather*}
Fissato il punto $P_0$ allora questa è una funzione di $A$ e nessuno
mi vieta di definirla in questo modo:
\begin{align}
    L_{P_0A} &= -V(x_A, y_A, z_A) \\
    L_{P_0B} &= -V(x_B, y_B, z_B)
\end{align}
E per cui si ottiene che
\begin{align}
    L_{AB} = V(A) - V(B)
\end{align}
L'energia potenziale è sempre definita a meno di una costante; possiamo allora
esprimere il lavoro mediante l'energia potenziale (che è posizionale e quindi in funzione dello spazio)
e allora scrivere:
\begin{gather*}
    \delta L = -dV = - (V(x + dx, y+dy, z + dz) - V(x, y, z))
\end{gather*}
Matematicamente questo si può scrivere come:
\begin{gather*}
    - (V(x + dx, y+dy, z + dz) - V(x, y +dy, z +dz) + V(x, y +dy, z +dz)\\
    - V(x , y , z +dz) + V(x , y , z +dz) - V(x, y, z))
\end{gather*}
Posso allora fare la derivata parziale per ogni variabile che compare
ottenendo allora 
\begin{gather*}
    -\left(\frac{\partial V}{\partial x}dx + \frac{\partial V}{\partial y}dy  + \frac{\partial V}{\partial z}dz\right) 
\end{gather*}
Esprimendo la forza nel lavoro mediante le sue componenti allora si ottiene:
\begin{gather*}
    \delta L = \vec{F} \cdot  \vec{dr} = F_x dx + F_y dy + F_Z dz  
\end{gather*}
E questo vale qualunque sia $dx, dy, dz$. Eguagliando allora lo stesso 
incremento infinitesimo dell'energia potenziale ad ogni componente della forza, si ottiene 
proprio che
\begin{gather*}
    \left\{\begin{array}{c}
    F_x = -\frac{\partial V}{\partial x}\\
    F_y = -\frac{\partial V}{\partial y}\\
    F_z = -\frac{\partial V}{\partial z}
    \end{array}\right. \Rightarrow \vec{F} = - \nabla V 
\end{gather*}
Dove $\nabla$ è l'operatore differenziale che esprime la derivata parziale rispetto
a tutte le componenti di un certo vettore 
\begin{gather*}
    \delta L = \vec{F}\cdot \vec{dr}  
\end{gather*}
Con l'operatore nabla attraverso:
\begin{gather*}
    \delta L = -\nabla V \cdot \vec{dr} = -dV 
\end{gather*}
Con questa si ottiene che:
\begin{align}
    \nabla \times \vec{F} &= 0 \\
     \vec{F} &= -\nabla V \\
    \nabla \times \nabla V &= 0 
\end{align}
ossia il rotore della forza è proprio il prodotto vettoriale.  
Si fa ora il determinante del prodotto vettoriale e quindi:
\begin{align}
    \det\left| \begin{tabular}{c c c}
         $\hat{i} $ & $\hat{j} $&  $\hat{k}$ \\
         $\frac{\partial }{\partial x}$ &  $\frac{\partial }{\partial y}$ & $\frac{\partial }{\partial z}$ \\
         $\frac{\partial V}{\partial x}$  & $\frac{\partial V}{\partial y}$  &$\frac{\partial V}{\partial z} $
    \end{tabular} \right| = \hat{i} \left( \frac{\partial^{2} V }{\partial y \partial z} - frac{\partial^{2} V }{\partial y \partial z}\right) = 0. 
\end{align}

e dal momento che le derivate parziali commutano, allora il determinante è zero
ed il rotore della forza è zero.

\end{document}


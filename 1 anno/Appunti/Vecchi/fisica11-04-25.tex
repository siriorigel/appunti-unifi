\documentclass[a4paper, oneside]{article}
\usepackage{graphicx}
\usepackage{amsthm}
\usepackage{amsmath}
\usepackage[a4paper,
            bindingoffset=0.2in,
            left=2cm,
            right=2cm,
            top=2cm,
            bottom=2cm,
            footskip=.25in]{geometry}
\usepackage[italian]{babel}
\usepackage{pgfplots}
\usepackage{tabularx}
\usepackage{tikz}
\usepackage{wrapfig}
\usepackage{color}
\definecolor{page}{rgb}{0.129,0.157,0.212}
\pagecolor{page}
\color{white}
\graphicspath{ {./images/} }
\usetikzlibrary{shapes.geometric}
\usetikzlibrary{datavisualization}
\usetikzlibrary{datavisualization.formats.functions}
\pgfplotsset{width=10cm,compat=1.9}

\title{Fisica}
\author{Tommaso Miliani}
\date{11-04-25}

\begin{document}
\theoremstyle{definition}
\theoremstyle{theorem}
\theoremstyle{lemma}

\newtheorem{definition}{Definizione}[section]
\newtheorem{theorem}{Teorema}[section]
\newtheorem{lemma}{Proposizione}[theorem]
\newtheorem{example}{Esempio}[section]

\maketitle


\section{Definizione di urto e forze in gioco}
\begin{wrapfigure}{r}{0.4\textwidth}
    \centering
    \caption{L'urto tra due palline}
    \begin{tikzpicture}
        \draw(0, 0) -- (3, -3);
        \draw(0.6, -0.25) circle (0.25);
        \draw(2.15, -1.75) circle (0.25);
        \draw[->](0.6, -0.25) -- (1.1, -0.75);
        \draw[->](2.15, -1.75) -- (1.65, -1.25);
    \end{tikzpicture}    
\end{wrapfigure}
Nel caso generale gli urti considerati nella fisica classica
durano poche frazioni di secondo. Riprendendo l'impulso possiamo 
esprimere queste grandi forze come
\begin{gather*}
    \vec{I} = \int \vec{F}\ dt = \Delta \vec{q}   
\end{gather*}
Se le due palline si urtano, allora si sviluppano delle forze molto
intense in tempi molto brevi che si posso rappresentare
come una curva molto stretta lungo l'asse del tempo e molto alta nell'asse
della forza del grafico tempo-spazio. \\
Cosa succede alle forze  peso
o quella per la forza vincolare? Possiamo distinguere due classi di forze:
le forze \textbf{impulsive} e quelle \textbf{non impulsive}. Non
sono intrinsecamente diverse ma semplicemente differiscono
nella loro intensità: quelle impulsive raggiungono forze molto grandi in tempi
molto piccoli mentre le altre sono spesso o costanti oppure forze
che non raggiungono intensità molto alte in tempi brevi. \\
Considerando l'intervallo di tempo durante l'urto posso trascurare l'impulso delle
forze non impulsive in quanto in quel lasso di tempo le forze impulsive sono
molto maggiori in modulo delle forze non impulsive e quindi nell'integrale
dell'impulso posso solo considerare il contributo delle forze impulsive. \\
Se si considera l'insieme formato dai due corpi, le forze impulsive sono
delle forze interne in quanto l'urto di un corpo all'altro è esattamente la forza
che imprime uno dei due oggetti dall'altro. Per quanto riguarda quindi 
la variazione della quantità di moto si può dire che il sistema è isolato in quanto
anche se c'è la forza peso, etc .., le forze d'urto sono
le uniche che considero e allora la quantità di moto del sistema
si conserva in quanto considero solo l'effetto delle forze interne.

\subsection{E' sempre possibile trascurare le forze non impulsive?}
Nonostante quanto detto finora, non è possibile sempre trascurare le forze
non impulsive in quanto nell'urto di una pallina dall'alto con un certo angolo rispetto all'orizzontale
su di una pallina ferma, questa potrebbe voler sfondare il piano su cui è appoggiata
ma questo non accade se si considera anche il contributo del vincolo.In generale
bisogna sempre chiedersi se il vincolo ha una certa forza impulsiva che può scaturire
a causa delle forze impulsive. La geometria del sistema di riferimento diventa 
dunque essenziale per determinare o meno l'impulsività delle forze vincolari.

\section{Urto elastico ed anelastico}
\begin{wrapfigure}{r}{0.4\textwidth}
    \centering
    \caption{Urto elastico}
    \begin{tikzpicture}
        \draw(0, 0) circle (0.25);
        \draw(3, 0) circle (0.25);
        \draw[->](0, 0) -- (0.75, 0);
        \draw[->](3, 0) -- (3.5, 0);
    \end{tikzpicture}    
\end{wrapfigure}
Nel caso in cui il sistema sia isolato, allora posso dire che
si conserva (nell'istante iniziale e in quello finale) si conserva
la quantità di moto totale
\begin{gather*}
    m_1 \vec{v}_{1f} + m_2 \vec{v}_{2i} = m_1 \vec{v}_{1f} + m_2 \vec{v}_{2f}    
\end{gather*}
Dato che ho sei incognite, posso ricondurmi al caso unidimensionale
in modo tale da poter ridurre le incognite della mia equazione
a due sole. Posso farlo con buona
approssimazione se e solo se l'urto è centrato (ossia la traiettoria contiene entrambi i centri
delle palline). Adesso ciò che succederà dipende dalla forza di interazione
e si potrebbe allora dissipare energia (quello che succede all'energia cinetica
dipende dai dettagli dell'urto). Possiamo definire allora
due tipologie di urto: l'\textbf{urto elastico} un urto in cui 
l'energia cinetica subito prima dell'urto e subito dopo l'urto è
uguale.  Si definisce \textbf{urto anelastico} un urto in cui non
si conserva l'energia cinetica. Quando dissipo tutta l'energia  relativa al centro di massa
allora i due corpi si attaccano e si parla allora di \textbf{urto completamente anelastico}. \\
Nel caso di urto elastico allora
posso utilizzare anche la conservazione dell'energia cinetica per
determinare la velocità finale:
\begin{gather*}
    \left\{\begin{array}{l}
        m_1 \vec{v}_{1f} + m_2 \vec{v}_{2i} = m_1 \vec{v}_{1f} + m_2 \vec{v}_{2f}     \\
        \frac{1}{2}m_1 v_{1i}^{2} + \frac{1}{2}m_2v_{2i}^{2} =  \frac{1}{2}m_1 v_{1f}^{2} + \frac{1}{2}m_2v_{2f}^{2} 
    \end{array}\right.
\end{gather*}
Posso allora risolvere
\begin{gather*}
    m_1(v_{1i} - v_{1f}) =  m_2(v_{2i} - v_{2f}) \\
    m_1(v_{1i}^{2}  - v_{1f}^{2} ) =  m_2(v_{2i}^{2}  - v_{2f}^{2} )
\end{gather*}
Le velocità finali allora diventano
\begin{align}
    v_{1f} &= \frac{m_1 - m_2}{m_1 + m_2}v_{1i} + \frac{2m_1}{m_1 + m_2}v_{2i} \\
    v_{2f} &= \frac{2m_1}{m_1 + m_2}v_{1i} + \frac{m_1 -m_2}{m_1 + m_2}v_{2i}
\end{align}

\subsection{Casi particolari}
1. $m_1 = m_2$:
\begin{gather*}
    v_{1f}  = v_{2i} \\
    v_{1i} = v_{2f}
\end{gather*}
Le particelle si scambiano le velocità. In particolare nelle centrali
nucleari a fissione per rallentare i neutroni lenti bisogna usare come moderatore
l'acqua pesante in quanto è ricca di neutroni. \\
2. $v_{2i} = 0$
\begin{gather*}
    v_{1f} = \frac{m_1 - m_2}{m_1 + m_2}v_{1i} \\
    v_{2f} = \frac{2m_1}{m_1 + m_2}v_{1i}
\end{gather*}
Nell'urto elastico unidimensionale con una particella ferma, la seconda 
particella si mette in moto nella stessa direzione della particella
in movimento se e solo se l'urto è centrale. La prima invece può andare 
o avanti o indietro (se la massa è superiore va avanti altrimenti indietro). \\
2.1 $m_1 = m_2$
\begin{gather*}
    \left\{\begin{array}{l}
        v_{1f} = 0 \\
        v_{2f} = v_{1i}
    \end{array}\right.
\end{gather*}
2.b. $m_2 >> m_1$ 
\begin{gather*}
    v_{1f} = -v_{1i} \\
    v_{2f} = 0
\end{gather*}

\subsection{La pallina che urta un vincolo}
\begin{wrapfigure}{r}{0.4\textwidth}
    \centering
    \caption{L'urto di un vincolo}
    \begin{tikzpicture}
        \draw(0, 2) circle (0.25);
        \draw(0, 0) -- (5, 0);
        \draw[->](0, 2) -- (0.5, 1.5) node[at end, above] {$v_{1i}$};
        \draw(2, 0.25) circle(0.25);
        \draw[->](2, 0.5) -- (2, 1) node[at end, right] {$\vec{N}$};
        \draw[->](2, 0.25) -- (2.5, 0.25) node[at end, above] {$v_{1xi}$};
        \draw[->](2, 0.25) -- (2, -0.25) node[at end, left] {$v_{1iy}$};
    \end{tikzpicture}    
\end{wrapfigure}
La pallina finirà per compiere un moto di riflessione rispetto
al vincolo e quindi tenderà a rimbalzare con lo stesso angolo
con cui ha urtato il vincolo (ovviamente se e solo se l'urto è elastico
e lungo l'asse $x$ non ho alcuna forza). Dal punto di vista sostanziale tutti
gli urti coi raggi cosmici e nella fisica delle particelle seguono
tutti questo modello anche se per corpi relativistici dovrò stare
attento alle velocità. 

\section{L'urto anelastico}
Si farà-....

\subsection{L'urto completamente anelastico}


\end{document}
\documentclass[a4paper, oneside]{article}
\usepackage{graphicx}
\usepackage{amsthm}
\usepackage{amsmath}
\usepackage[a4paper,
            bindingoffset=0.2in,
            left=2cm,
            right=2cm,
            top=2cm,
            bottom=2cm,
            footskip=.25in]{geometry}
\usepackage[italian]{babel}
\usepackage{pgfplots}
\usepackage{tabularx}
\usepackage{wrapfig}
\graphicspath{ {./images/} }
\usetikzlibrary{datavisualization}
\usetikzlibrary{datavisualization.formats.functions}
\pgfplotsset{width=10cm,compat=1.9}

\title{FIsica lenti}
\author{Tommaso Miliani}
\date{06-03-25}

\begin{document}
\theoremstyle{definition}
\theoremstyle{theorem}
\theoremstyle{lemma}

\newtheorem{definition}{Definizione}[section]
\newtheorem{theorem}{Teorema}[section]
\newtheorem{lemma}{Proposizione}[theorem]

\maketitle

\section{Energia meccanica}
Un dominio si dice semplicemente connesso se, presa una curva chiusa
la posso deformare fino a farla diventare un punto. Considerato un toroide, 
se prendo una curva all'interno del dominio, non posso farla collassare in un punto. Se abbiamo un sistema
su cui agiscono solo forze conservative, (anche forze non conservative che non
fanno lavoro), allora il teorema delle forze vive mi dice che:
\begin{gather*}
    \delta L = - dV = dK
\end{gather*}
E quindi isolando poso dire che
\begin{gather*}
    dK = - dV
\end{gather*}
E allora
\begin{gather*}
    dK + dV = 0
\end{gather*}
Se ho quindi solo forze conservative allora la somma dell'energia potenziale e cinetica
non variano: ossia quella somma è costante. La somma di due contributi energetici che danno una costante
è chiamata \textbf{energia meccanica}. 
\begin{align}
    E = K + V 
\end{align}
Che è uno dei concetti più importanti nella meccanica classica e nella fisica moderna
e diventa un concetto fondamentale della fisica in generale. Lo stesso vale se ho due forze conservative
considerando la somma delle forze conservative.
\begin{align}
    dE = 0
\end{align}
Se invece ho una forza non conservativa, l'energia meccanica non si
conserva: se si avesse una forza non conservativa allora $\delta L = -dV$ non vale più.
INfatti, considerati i contributi energetici delle forze conservative e non si ottiene
\begin{gather*}
    \delta L_{tot} = \delta L_{cons} + \delta L_{ncons}\\
    dK = -\sum dV_i + \delta L_{ncons} 
\end{gather*}
Quindi se lo porto dall'altra parte ottengo;
\begin{gather*}
    d(K + \sum dV_i ) = \delta L_{ncons}
\end{gather*}
Chiamato allora
\begin{gather*}
    E = K + \sum dV_i
\end{gather*}
L'energia meccanica non sarà più uguale ad una costante 
ma uguale al lavoro che compiono le forze
non conservative:
\begin{gather*}
    dE = \delta L_{ncons}
\end{gather*}
Se il lavoro dell'attrito è negativo allora tende a diminuire.

\section{Le forze conservative in particolare}
\subsection{la forza peso}
\begin{wrapfigure}{r}{0.4\textwidth}
    \centering
    \label{fig}
    \caption{Forza peso}
    \begin{tikzpicture}
        \filldraw(0, 0) circle (1pt) node[anchor = south] {$m$};
        \draw[->](0, 0) -- (0, -1) node[at end, right] {$m\vec{g}$ };
        \draw[->](-1, -3) -- (0, -3) node[at end, below] {$x$};
        \draw[->](-1, -3) -- (-1, -2) node[at end, left] {$y$};
    \end{tikzpicture}    
\end{wrapfigure}
Se la forza peso è costante allora il suo gradiente è zero 
poiché non c'è alcuna variazione ed il suo rotore è quindi anche zero.
Provando a fare un lavoro infinitesimo o finito provo a trovare
l'energia potenziale e provando a fare il gradiente devo trovare la
forza di partenza:
\begin{gather*}
    m\vec{g} = mg\hat{j} \\
    d\vec{r} = dy \hat{j} \\
    \delta L = (m\vec{g})\cdot d\vec{r}  
\end{gather*}
Allora:
\begin{gather*}
    \delta L = -mg\hat{j}\cdot dy \hat{j} 
\end{gather*}
Il lavoro finito allora:
\begin{gather*}
    \int_{A}^{B} \delta L = -mg(y_B -y_A) = V(A) - V(B)
\end{gather*}
E' del tutto ragionevole che per un punto generico P 
che l'energia potenziale sia:
\begin{gather*}
    V(P) = mgy_P
\end{gather*}
Dato che è definita a meno di una costante, il lavoro
è dato dalla differenza delle potenziali e potevo scegliere
come riferimento dell'asse y un punto qualunque e non sarebbe cambiato
nulla rispetto al risultato. Se aggiungo la stessa quantità sia a $V(A)$ che a
$V(B)$ allora questa costante che aggiungo si cancella in quanto per
definizione l'energia potenziale è definita a meno di una costante.
\begin{gather*}
    grad V = \frac{\partial V}{\partial x}\hat{i} +  \frac{\partial V}{\partial y}\hat{j} +\frac{\partial V}{\partial z}\hat{k}
\end{gather*}
Ottenendo proprio
\begin{gather*}
    grad V = mg \hat{j} 
\end{gather*}
Dato che io per definizione di gradiente:
\begin{gather*}
    \vec{F} = -grad V \Rightarrow -mg\hat{j}  
\end{gather*}
La direzione dell'asse y è un'altra scelta arbitraria che ho fatto a priori
nell'esperimento così come l'orientazione degli assi. Ma non cambia niente in quanto
fin tanto che seguo le leggi della mano destra posso comunque scegliere
qualsiasi sistema di riferimento in qualsiasi verso lo voglia orientare.
Nel caso di un corpo in caduta, se pongo come riferimento l'asse
z rivolto verso l'alto, allora l'energia meccanica la posso esprimere come:
\begin{gather*}
    E = \frac{1}{2}mv^{2} + mgz 
\end{gather*}
A: $t = 0$ ottengo:
\begin{gather*}
    z = h .v = 0, E = mgh 
\end{gather*}
Al tempo finale $t = t_F$:

\begin{gather*}
     z= 0, v = v_f, E = \frac{1}{2}mv_f^{2}  
\end{gather*}
Allora
\begin{gather*}
    v_F = \sqrt{2gh} 
\end{gather*}
\begin{wrapfigure}{r}{0.4\textwidth}
    \centering
    \label{grfg}
    \caption{dssd}
    \begin{tikzpicture}
        \draw(0, 0) -- (3, 0);
        \draw(0, 0) -- (0, 3);
        \draw[<->](1, 0) -- (1, 2) node[midway, left] {$h$};
        \filldraw(1.5, 2) circle(1pt) node[anchor = west] {$m$};
    \end{tikzpicture}    
\end{wrapfigure}
In un moto verticale se avessi usato $F = ma$ allora si ottiene:
\begin{gather*}
     E = \frac{1}{2}m \dot{z}^{2} + mgt \\
     \frac{dE}{dt} = 0 \\
     \dot{z}(m\ddot{z} + mg) = 0   
\end{gather*}
Per cui le soluzioni solo:
\begin{gather*}
    \dot{z} = 0 \\
    m\ddot{z} = -mg 
\end{gather*}
CHe vale solo se i vincoli sono onesti ed ideali e se non si muovono.
IN generale questa è molto comoda se ho pochi vincoli, altrimenti
con la conservazione dell'energia posso considerare sempre un grado di
libertà solo.

\subsection{Forza elastica}
\begin{wrapfigure}{r}{0.4\textwidth}
    \centering
    \label{Fig 4.1}
    \caption{La molla ideale}
    \begin{tikzpicture}
        \draw (0, 0) -- (0.5, 0);
        \draw (0.5, 0) -- (0.5, -1);
        \draw[decoration={aspect=0.3, segment length=1.5mm, amplitude=1.3mm,coil},decorate,opacity=0.9] (0.5, -0.5) -- (2,-0.5);
        \draw (0.5, -1) -- (2.5, -1);
        \draw (2, -1) rectangle (3, 0) node[midway] {$M$};
    \end{tikzpicture}    
\end{wrapfigure}
Dalla legge della forza elastica:
\begin{gather*}
    \vec{F}_d = -kx\dot{i}  
\end{gather*}
La forza elastica è conservativa e posso vederlo o facendo
il rotore oppure attraverso un lavoro infinitesimo:
\begin{gather*}
    \delta L = \vec{F}_d \cdot \vec{dr} = -kx\hat{i}\cdot dx\hat{i}     
\end{gather*}
E allora il lavoro:
\begin{gather*}
    L_{AB} = \int_{A}^{B} -kxdx = -\frac{1}{2}kx_b^{2} + \frac{1}{2} kx_A^{2} = V(A) - V(B)  
\end{gather*}
QUesta è proprio la definizione di energia potenziale, allora posso prendere
\begin{gather*}
    V(P) = \frac{1}{2}kx^{2} 
\end{gather*}
Posso allora determinare il gradiente e quindi
si ottiene proprio
\begin{gather*}
    grad V = kx\hat{i} 
\end{gather*}
QUindi essendo:
\begin{gather*}
    \vec{F} = -gradV  = -kx\hat{i} 
\end{gather*}
Allora abbiamo dimostrato che la forza elastica è proprio una forza conservativa. 
Dato che è definita a meno di una costante, allora posso cambiare l'origine
dell'asse x e cambierebbe solo la forma dell'energia potenziale ma pur sempre
compatibile con quella che abbiamo qui.
Se io tirassi la molla nell'istante $T = 0$allora l'energia è conservata
e quindi in ogni istante posso ricavarne la velocità:
\begin{gather*}
    t = 0, x = x_0, v = 0 , E = \frac{1}{2}kx_0^{2} \\
    t = t_f, x = 0, v = v_f, E = \frac{1}{2}mv_f^{2}  
\end{gather*}
E quindi la velocità:
\begin{gather*}
    v_F = \sqrt{\frac{k}{m}}x_0 
\end{gather*}
COn la relazione $ma = F$ allroa:
\begin{gather*}
    E = \frac{1}{2}m\dot{x}^{2} + \frac{1}{2}kx^{2} \\
    \frac{dE}{dt} = 0 \\
    \dot{x}(m\ddot{x} + kx) = 0   
\end{gather*}
Dato che $\dot{x}$ non è sempre zero, allora pongo solo:
\begin{gather*}
    m\ddot{x} = -kx
\end{gather*} 

\subsection{Molla attaccata al soffitto}
\begin{wrapfigure}{r}{0.3\textwidth}
    \centering
    \label{refesdf}
    \caption{fes}
    \begin{tikzpicture}
        \draw(0, 0) -- (2, 0);
        \draw[decoration={aspect=0.3, segment length=1.5mm, amplitude=1.3mm,coil},decorate,opacity=0.9] (1, 0) -- (1, -2);
        \draw[->](0, -1) -- (0.5, -1) node[at end, below] {$y$};
        \draw[->](0, -1) -- (0, -1.5) node[at end, left] {$x$};
        \draw(0.75, -2) rectangle (1.25, -2.5) node[midway] {$M$};
        \draw[->](1, -2.5)  -- (1, -3) node[at end, right] {$m\vec{g}$};
        \draw[->] (1, -2) -- (1, -1.5) node[at end, right] {$\vec{F}_e$};
    \end{tikzpicture}    
\end{wrapfigure}
ALlora con le equazioni dell'energia meccanica:
\begin{gather*}
    E = \frac{1}{2}mv^{2} + \frac{1}{2}kx^{2} - mgx\\
    v = \dot{x} \\
    \frac{dE}{dt} = 0 \\
    \dot{x}(m\ddot{x} + kx - mg) = 0    
\end{gather*}


\end{document}
\documentclass[a4paper, oneside]{article}
\usepackage{graphicx}
\usepackage{amsthm}
\usepackage{amsmath}
\usepackage[a4paper,
            bindingoffset=0.2in,
            left=2cm,
            right=2cm,
            top=2cm,
            bottom=2cm,
            footskip=.25in]{geometry}
\usepackage[italian]{babel}
\usepackage{pgfplots}
\usepackage{tabularx}
\usepackage{tikz}
\usepackage{wrapfig}
\usepackage{color}
\definecolor{page}{rgb}{0.129,0.157,0.212}
\pagecolor{page}
\color{white}
\graphicspath{ {./images/} }
\usetikzlibrary{shapes.geometric}
\usetikzlibrary{datavisualization}
\usetikzlibrary{datavisualization.formats.functions}
\pgfplotsset{width=10cm,compat=1.9}

\title{FisicA}
\author{Tommaso Miliani}
\date{08-04-25}

\begin{document}
\theoremstyle{definition}
\theoremstyle{theorem}
\theoremstyle{lemma}

\newtheorem{definition}{Definizione}[section]
\newtheorem{theorem}{Teorema}[section]
\newtheorem{lemma}{Proposizione}[theorem]
\newtheorem{example}{Esempio}[section]

\maketitle

\section{IL moto di rotolamento con le equazioni cardinali}
\begin{wrapfigure}{r}{0.4\textwidth}
    \centering
    \caption{Il moto di rotolamento}
    \begin{tikzpicture}
        \draw(0, 0) -- (4, 0);
        \draw(0, 0) -- (0, 2);
        \draw(0, 2) -- (4, 0);
        \draw[->](0, 2) -- (1, 1.5) node[at end, below] {$x$};
        \draw[->](0, 2) -- (0.5, 3) node[at end, left] {$z$};
        \draw(2, 1.5) circle (0.5);
    \end{tikzpicture}    
\end{wrapfigure}
Scelto un sistema di riferimento in modo tale che l'asse x sia parallela al piano inclinato,
allora posso scomporre la forza peso nelle due componenti e quindi posso
dire
\begin{gather*}
    \vec{N} + \vec{F}_a + m\vec{g} = m\vec{a}_C    
\end{gather*}
Scomponendo posso dire
\begin{gather*}
    x) \quad mg\sin\alpha - F_a = ma \\
    z) \quad N - mg\cos\alpha = 0
\end{gather*}
Posso utilizzare la seconda cardinale nella forma semplice (considerando
allora la proiezione lungo la velocità angolare dei momenti). La velocità angolare
dunque essendo lungo y non cambia e mi trovo allora nelle condizioni
nelle quali posso utilizzare la forma assiale della seconda cardinale lungo
l'asse $y$. Scelgo il polo di riduzione in modo da utilizzare le formule
semplici e quindi deve essere o fisso o coincida con il centro di massa. C'è anche
la possibilità di scegliere un polo la cui velocità è parallela a quella del
centro di massa. Scelto come polo di riduzione $\Omega = C$, allora
\begin{gather*}
    \begin{tikzpicture}
        \draw(0, 0) circle (2);
        \filldraw(0, 0) node[anchor = south] {$C$};
        \draw[dashed](-1.41, -1.41) -- (1.41, 1.41);
        \draw[->](0, 0) -- (0, -1) node[at end, right] {$m\vec{g}$}; 
        \draw[->](-1.41, -1.41) -- (-0.8, -0.8) node[at end, right] {$\vec{N}$};
        \draw[->](-1.41, -1.41) -- (-2, -0.82) node[at end, below] {$\vec{F}_a$};
        \draw[->](1, 0) arc (0:-45:1) node[midway, right] {$\dot{\phi}$};
    \end{tikzpicture}
\end{gather*}
Nelle condizioni di rotolamento puro c'è una relazione semplice tra $\ddot{x}$ e $\ddot{\phi}$
e l'unica cosa da stare attenti è il segno, dato che $\dot{\phi}$ aumenta e pure la sua
derivata, allora anche $\ddot{x}$ aumenta nello stesso senso.
\begin{gather*}
    F_a R = I_C \ddot{\phi} \\
    \dot{\phi} = \frac{\ddot{x}}{R}
\end{gather*}
Allora riscrivendo le forze lungo gli assi posso dire
\begin{gather*}
    \left\{\begin{array}{l}
        mg\sin\alpha - F_a = m\ddot{x} \\
        F_a = \frac{I_c}{R^{2}}\ddot{x}
    \end{array}\right.
\end{gather*}
Il centro di massa ha un accelerazione più piccola di quella che ha se
scivolasse senza attrito e quindi la sua accelerazione
\begin{gather*}
    \ddot{x} = \frac{g\sin\alpha}{1 + \frac{I_C}{mR^{2} }}
\end{gather*}
Il momento di inerzia per il disco omogeneo
\begin{gather*}
    I_C = \frac{1}{2}mR^{2} \Rightarrow  \ddot{x} = \frac{2}{3} g\sin\alpha
\end{gather*}
Per la sfera omogenea
\begin{gather*}
    I_c = \frac{2}{5}mR^{2} \Rightarrow  \ddot{x} = \frac{5}{7}g\sin\alpha 
\end{gather*}
Per il cerchione (auto)
\begin{gather*}
    I_C = mR^{2} \Rightarrow \frac{1}{2}g\sin\alpha 
\end{gather*}
Sostituendo allora la forza di attrito diventa:
\begin{gather*}
    F_a = \frac{1}{3}mg\sin\alpha
\end{gather*}
Il caso limite dell'attrito è dunque
\begin{gather*}
    F_a \leq \mu_s N 
\end{gather*}
E allora questo vuol dire che
\begin{gather*}
    \tan\alpha \leq 3\mu_s
\end{gather*}

\section{Carrucole mobili}
\begin{wrapfigure}{r}{0.4\textwidth}
    \centering
    \caption{Carrucola mobile}
    \begin{tikzpicture}
        \draw(0, 0) -- (1, 0);
        \draw(0.5, 0) -- (0.5, -2);
        \draw(1.5, -2) circle(1);
        \draw[->](2.5, -2) -- (2.5, -1) node[at end, right] {$\vec{F}$};
        \draw(2.5, -1) -- (2.5, -0.5);
        \draw(0.5, -2) -- (0.5, -1) node[at end, left] {$\vec{T}$};
        \draw(1.5, -2) -- (1.5, -2.5) node[at end, right] {$m\vec{g}$};
    \end{tikzpicture}    
\end{wrapfigure}
Nella carrucola mobile, questa compie effettivamente un moto
di rotolamento puro in quanto con l'attrito essa inizia a rotolare rispetto al filo
\begin{gather*}
    \vec{T} + \vec{F} + M\vec{g} = M\vec{a}_C \\
    T + F - Mg = M\ddot{z}    
\end{gather*}
Preso come centro di riduzione il centro di massa, allora posso decidere
come prendere l'angolo (in senso orario) e quindi 
\begin{gather*}
    (T - F)R = I_C \ddot{\phi} \\
    \ddot{\phi} = -\frac{\ddot{z}}{R}
\end{gather*}
E allora diventa
\begin{gather*}
    2F - Mg = \left(M + \frac{I_C}{R^{2} }\right)\ddot{z}
\end{gather*}
La condizione di equilibrio è proprio quella
\begin{gather*}
    F = \frac{1}{2}Mg 
\end{gather*}
Dinamicamente data $F$
\begin{gather*}
    T = F - \frac{I_c}{R^{2} }\frac{2F - Mg}{M + \frac{I_c}{R^{2} }}
\end{gather*}

\section{Corpo rigido ruotante }
Dato che ruota con velocità $\omega$ allora posso scegliere un
polo ossia il centro di massa 
\begin{gather*}
    \vec{L}_C = \sum_{i = 1}^{n}  (P_i - C) \times (m_i\vec{v_i} ) = \sum_{i = 1}^{n}  (P_i - C) \times m_i(\vec{v}_i + \vec{\omega} \times (P_i - C)  )
\end{gather*}
Dato che la velocità del centro di massa è zero, allora posso dire che $V_I$ è tutta zero
e quindi, con le proprietà del prodotto vettoriale posso semplificare così
\begin{gather*}
    \vec{L}_C = \sum_{i = 1}^{n}m_i(P_i - C)(P_i - C)\vec{\omega} - \sum_{i = 1}^{n}m_i(P_i - C)\vec{\omega}(P_i - C)     
\end{gather*}
Dato che ora $P_i - C$ è il sistema di riferimento per ogni punto
del corpo rigido, posso allora dire che
\begin{gather*}
    (P_i - C) = x_i \hat{i} + y_i \hat{j} + z_i \hat{k}, \qquad \vec{\omega} = \omega_x \hat{i} + \omega_y \hat{j} + \omega_z \hat{k}       
\end{gather*}
Quindi il momento angolare diventa semplificando e risolvendo
\begin{gather*}
    \vec{L}_C = \sum_{i = 1}^{n}m_i\left\{\begin{array}{l}
        (y_i^{2}\omega_x + z_i^{2}\omega_x - x_iy_i\omega_y - x_iz_i\omega_z)\hat{i} \\
        (x_i^{2}\omega_y + z_i^{2}\omega_y - x_iy_i\omega_x - y_iz_i\omega_z)\hat{j} \\
        (x_i^{2}\omega_z + y_i^{2}\omega_z - x_iz_i\omega_x - y_iz_i\omega_y)\hat{k}   
    \end{array}\right.  
\end{gather*}
POsso allora esprimere il momento angolare in forma matriciale
\begin{align}
    \begin{pmatrix}
        L_x \\
        L_y \\
        L_z
    \end{pmatrix} = \begin{pmatrix}
        \sum_{i}^{}m_i(y_i^{2} + z_i^{2}) & -\sum_{i}^{} m_i x_iy_i & -\sum_{i}^{} m_ix_i z_i \\
        -\sum_{i}^{}m_ix_iy_i & \sum_{i}^{} m_i(x_i^{2} + z_i^{2} ) & -\sum_{i}^{} m_iy_iz_i \\
        -\sum_{i  }^{}m_ix_iz_I & -\sum_{i }^{}m_iy_iz_i & \sum_{i }^{}   m_i(x_i^{2} + y_i^{2} )
    \end{pmatrix}\cdot \begin{pmatrix}
        \omega_x \\
        \omega_y \\
        \omega_z
    \end{pmatrix}
\end{align}
Posso esprimere questa matrice come un \textbf{tensore di inerzia} ossia
come
\begin{gather*}
    \begin{pmatrix}
        L_x \\
        L_y \\
        L_z
    \end{pmatrix} = \begin{pmatrix}
        I_{xx} & I_{xy} & I_{xz} \\
        I_{xy} & I_{yy} & I_{yz} \\
        I_{xz} & I_{yz} & I_{zz} 
    \end{pmatrix} \cdot \begin{pmatrix}
        \omega_x \\
        \omega_y \\
        \omega_z
    \end{pmatrix}
\end{gather*}
Se la matrice è diagonalizzabile allora posso trovare degli autovalori
per cui posso rendere la matrice diagonale attraverso gli \textbf{assi di principali
di inerzia} ossia gli assi $x', y', z'$:
\begin{gather*}
    \begin{pmatrix}
        L_x \\
        L_y \\
        L_z
    \end{pmatrix} = \begin{pmatrix}
        I_{x'x'} & 0 & 0 \\
        0 & I_{y'y'} & 0 \\
        0 & 0 & I_{z'z'} 
    \end{pmatrix} \cdot \begin{pmatrix}
        \omega_{x'} \\
        \omega_{y'} \\
        \omega_{z'}
    \end{pmatrix}
\end{gather*}
Se  $\vec{\omega}$ è parallelo ad un asse principale di inerzia allora
posso trovare che si annullano le altre due componenti e quindi il momento
sarà presente solo rispetto a quell'asse. 

\section{Momento del corpo rigido rispetto ad un asse }
Dato un asse fisso ed un sistema di coordinate cilindriche
posso dire che la distanza di un qualsiasi punto dal centro
del corpo rigido è data da:
\begin{gather*}
    (P_i - O) = z_C\hat{k} + \vec{\rho}_i  
\end{gather*}
Dato che il sistema di riferimento è solidale con il corpo rigido in
rotazione, posso dire che sono valide le seguenti:
\begin{gather*}
    \rho_i = \rho_i \hat{u}_{\rho} \\
    \left| \rho_i \right| = CONST  \\
    \vec{v} = \vec{\omega} (P - O) \\
    \vec{v}_O = 0   
\end{gather*}
Posso esprimere il momento angolare come
\begin{gather*}
    \vec{L}_0 = \sum_{i = 1}^{n} (P_i - O) \times m_i \vec{v}_i   
\end{gather*}
Allora posso dire, con tutte le sostituzioni
\begin{gather*}
    \sum_{ = }^{} (z_o \hat{k} + \vec{\rho}_i ) \times m_i(\omega \hat{k} \times (z_i \hat{k} + \vec{\rho}_i ) )  
\end{gather*}
E quindi, svolgendo i prodotti vettoriali, e dato che è una terna destrorsa
locale (che cambia dunque da punto a punto), posso sapere già che
\begin{gather*}
    \vec{u}_{\rho}, \vec{u}_{T_i}, \hat{k} \\
    \hat{k} \times \hat{u}_{\rho} = \hat{u}_{T_i}      
\end{gather*}

Posso allora esprimere il momento angolare come
\begin{gather*}
    L_O = \sum_{ = }^{} m_i \rho_i^{2}\vec{\omega} - \omega \sum_{  = }^{}   m_i z_i \vec{\rho}_i
\end{gather*}
Il primo termine è il momento angolare parallelo alla velocità angolare
mentre il secondo è il momento angolare ortogonale alla velocità angolare.
Chi sono questi due contributi? Il primo è proprio il momento di inerzia
dell'i-esimo punto rispetto all'asse di rotazione considerato.
Se il mio asse di rotazione è parallelo ad un asse di simmetria,
allora  il secondo termine potrà essere eliminato

\begin{gather*}
    \vec{L}_O = \vec{L}_{\parallel} + \vec{L}_{\perp}   
\end{gather*}
E quindi loro sono proprio:
\begin{gather*}
    \vec{L}_{\parallel} = I_{O}\vec{\omega} \\
    \vec{L}_{\perp} = - \omega \sum_{ = }^{} m_iz_i \vec{\rho}_i 
\end{gather*}

Derivando l'espressione del momento angolare
posso vedere che si ottiene
\begin{gather*}
    \dot{\vec{L} }_0 = \sum m_i \rho_i^{2}\dot{\vec{\omega} } - \dot{\omega} \sum_{ = }^{}   m_i z_i \vec{\rho}_i - \omega \sum_{ = }^{}  m_i z_i \dot{\vec{\rho} }_i
\end{gather*}
Dato che $\rho_i$ ha modulo costante e ho una velocità angolare,
allora la sua derivata è proprio (dalle formule di Poisson) 
\begin{gather*}
    \dot{\vec{\rho}}_i = \vec{\omega} \times \vec{\rho}_i  
\end{gather*}
Quindi si ottiene
\begin{gather*}
    \dot{\vec{L} }_O = I_{O}\dot{\vec{\omega} } - \frac{\dot{\omega}}{\omega}\omega \sum_{ = }^{} m_i z_i \vec{\rho}_i - \omega \sum_{ = }^{}  m_i z_i \vec{\omega} \times \vec{\rho}_i   
\end{gather*}
E allora si ottiene proprio
\begin{gather*}
    \dot{\vec{L} }_O = I_O \dot{\vec{\omega} } + \frac{\dot{\omega}}{\omega} \vec{L}_{\perp} + \vec{\omega} \times \vec{L}_{\perp}   
\end{gather*}
Il primo termine è parallelo rispetto al $\dot{\vec{L}}_O$ e quindi al
versore $\hat{k}$,  il secondo termine è ortogonale a $k$ mentre il terzo è un prodotto vettoriale e quindi è 
ortogonale a tutti e due i termini del prodotto vettoriale. SI può ora
riscrivere la seconda cardinale come
\begin{align}
    \vec{M}_O^{(ext)} = \dot{\vec{L} }_{\parallel} + \frac{\dot{\omega}}{\omega}\vec{L}_{\perp} + \vec{\omega} \times \vec{L}_{\perp}     
\end{align}
La seconda cardinale allora è espressa per il corpo rigido che ruota rispetto
ad un asse fisso; si vede allora che nella direzione di $\omega$ allora rimane solo il primo termine
come
\begin{gather*}
    \vec{M}^{ext}_{O} \hat{k} = I_O \dot{\omega}   
\end{gather*}
Se io metto in rotazione un oggetto, allora essendo che deve per forza
ruotare lungo l'asse considerato, esso continua a girare e quindi 
il momento angolare sta cambiando e quindi il terzo termine sta cambiando 
e non  trascurabile, da dove viene allora? E' una forza vincolare
che viene scaturita dalla rotazione impressa alle condizioni iniziali .

\subsection{Cosa succede al livello del lavoro}
\begin{wrapfigure}{r}{0.4\textwidth}
    \centering
    \begin{tikzpicture}
        \draw(0, 0) circle (2);
        \filldraw (1, 1) circle (1pt) node[anchor = south] {$P_2$};
        \filldraw(-1.5, -0.5) circle (1pt) node[anchor = south] {$P_1$};
    \end{tikzpicture}    
\end{wrapfigure}
L'unica condizione che chiedo è che il corpo sia rigido, e mi chiedo
quale sia il lavoro compiuto infinitesimamente, allora il lavoro per ogni punto
\begin{gather*}
    \delta L_1 + \delta L_2 = \vec{F}_{1, 2} \cdot  d\vec{r}_1 + \vec{F}_{2, 1} \cdot d \vec{r}_2     
\end{gather*}
Dato che il corpo è rigido allora la distanza tra i due punti è
costante e quindi
\begin{gather*}
    (d\vec{r}_1 - d\vec{r}_2  )\cdot (\vec{r}_1 - \vec{r}_2  ) = 0
\end{gather*}
Dato che la forza è parallela alla distanza tra il primo ed il secondo punto
allora posso dire che il lavoro e zero e dunque il lavoro delle forze di
rigidità è nullo. (Il che ha senso dato che il corpo è rigido).
Il lavoro totale infinitesimo sulle forze interne del corpo 
rigido è dato dalla somma di ciascun effetto reciproco:
\begin{gather*}
    \delta L = \sum_{ = }^{} \delta L_i = \sum_{ = }^{} \vec{F}_i 
\end{gather*}
Preso un punto a caso $A$ nel corpo rigido la velocità di un dato punto rispetto
a quel punto sarà
\begin{gather*}
    \vec{v}_i = \vec{v}_A + \vec{\omega} \times (P_i - A)   
\end{gather*}
\begin{gather*}
    \delta L = \sum_{ = }^{} \vec{F}_i (\vec{v}_A + \vec{\omega} \times (P_i - A)  )dt 
\end{gather*}
Posso allora dire che
\begin{gather*}
    \delta L = \vec{F}^{ext}\cdot  d\vec{r}_A + \omega dt\sum_{ = }^{} (P_i - A) \times \vec{F}_i 
\end{gather*}
Il primo termine non è altro che la risultante delle forze esterne mentre il secondo termine 
è proprio il risultato dei momenti delle singole forze esterne moltiplicate per $\vec{\omega}$ e 
quindi dato che
\begin{gather*}
    \vec{\omega} = \dot{\phi} \hat{u}_{\phi} \\
    \dot{\phi} dt = d\phi  
\end{gather*} 
Allora posso dire che il lavoro infinitesimo è
\begin{gather*}
    \delta L = \vec{F}^{ext}\cdot d\vec{r}_A + \vec{M}^{ext} _A \cdot  \hat{u}_{\phi}d\phi     
\end{gather*}
Il lavoro allora dipende dalle risultanti delle forze esterne e dallo spostamento
infinitesimo angolare e quindi se ho solo un moto di rotazione posso scegliere $A$ sull'asse
ed il lavoro è solo dato dal momento assiale delle forze esterne per l'angolo.
Se io considerassi le tre componenti del momento angolari allora conterà solo
il momento assiale lungo la velocità angolare. Allora se ruota in modo caotico il vincolo
che gli applico può essere liscio e può non compiere allora alcun lavoro.  \\
Dato che posso esprimere l'energia cinetica come
\begin{gather*}
    K = \frac{1}{2}Mv_C^{2} + \sum_{i = 1}^{n}m_i v_i^{2'}   
\end{gather*}
E allora dato che per il teorema di Konig
\begin{gather*}
    \vec{v}_i' = \vec{\omega} \times (P_i - C)  
\end{gather*}
Allora la componente rotazionale dell'energia cinetica del corpo rigido è
data (considerata $d_i$ come la distanza di ogni punto dall'asse di rotazione)
\begin{gather*}
    \frac{1}{2}\sum_{i = 1}^{n} m_i d_i \omega^{2}  
\end{gather*}
Allora l'energia cinetica del corpo rigido diventa proprio la somma tra
l'energia cinetica della velocità del centro di massa più la componente rotazionale
data da
\begin{align}
    K = \frac{1}{2}Mv_c^{2} + \frac{1}{2} I_C \omega^{2}  
\end{align}

\subsection{L'energia cinetica del pendolo fisico}
\begin{gather*}
    E = \frac{1}{2}I_O \omega ^{2} - Mgh\cos\phi 
\end{gather*}
Dato che la velocità angolare è la derivata dell'angolo, posso
esprimere la derivata dell'energia
\begin{gather*}
    \dot{E}  =0 = I_O \dot{\phi} \ddot{\phi} + Mgh\sin\phi\dot{\phi}
\end{gather*} 
Il periodo del pendolo è allora dato da
\begin{gather*}
    T = 2\pi \sqrt{\frac{I_O}{Mgh}} 
\end{gather*}

\subsection{Per il moto di rotolamento}
L'energia per il moto di rotolamento
\begin{gather*}
    E = \frac{1}{2}Mv_{C}^{2} + \frac{1}{2}I_C \omega^{2} - Mgx\sin\alpha = E(t = 0)  
\end{gather*}
La velocità è esprimibile come la derivata della posizione e dunque
dato che $\omega = \frac{\dot{x}}{R}$:
\begin{gather*}
    E = \frac{1}{2}(MR^{2} + I_C )\dot{\phi}^{2} - MgR\dot{\phi} \sin\alpha
\end{gather*}
Derivando rispetto a $\dot{\phi}$ e ponendo uguale a zero
allora
\begin{gather*}
    (MR^{2} + I_C)\ddot{\phi}-  MgR\sin\alpha
\end{gather*}

Allora posso esprimere l'accelerazione come
\begin{gather*}
    \ddot{x} = \frac{MgR\sin\alpha}{(MR^{2} + I_C)\frac{1}{R}} = \frac{g\sin\alpha}{\ + \frac{I_C}{MR^{2} }}
\end{gather*}

\end{document}
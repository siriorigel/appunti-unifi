\documentclass[a4paper, oneside]{article}
\usepackage{graphicx}
\usepackage{amsthm}
\usepackage{amsmath}
\usepackage[a4paper,
            bindingoffset=0.2in,
            left=2cm,
            right=2cm,
            top=2cm,
            bottom=2cm,
            footskip=.25in]{geometry}
\usepackage[italian]{babel}
\usepackage{pgfplots}
\usepackage{tabularx}
\usepackage{wrapfig}
\graphicspath{ {./images/} }
\usetikzlibrary{datavisualization}
\usetikzlibrary{datavisualization.formats.functions}
\pgfplotsset{width=10cm,compat=1.9}

\title{FIsica}
\author{Tommaso Miliani}
\date{10-3-25}

\begin{document}
\theoremstyle{definition}
\theoremstyle{theorem}
\theoremstyle{lemma}

\newtheorem{definition}{Definizione}[section]
\newtheorem{theorem}{Teorema}[section]
\newtheorem{lemma}{Proposizione}[theorem]

\maketitle

\section{Condizione di equilibrio}
La derivata dell'energia potenziale rispetto ad $x$
presuppone di avere solo un grado di libertà. 
\begin{gather*}
    \frac{dV}{dx}= 0
\end{gather*}
Ci da la condizione di equilibrio: infatti nel caso del pendolo
si ha che:
\begin{gather*}
    V = mg(L - L\cos\theta)
\end{gather*}
Facendo la derivata sul grado di libertà che sto considerando
allora posso fare la derivata su $\theta$ e questo automaticamente
tiene conto del vincolo poiché se parlo di $x, y. z$ non sto considerando
i vincoli del sistema, se invece li considero e lo vincolo nel piano allora 
l'unico grado di libertà è proprio $\theta$.
\begin{gather*}
    \frac{dV}{d\theta} = mgL\sin\theta
\end{gather*}
Ossia le soluzioni sono, considerato un filo che è resistente alla
trazione ma non alla compressione:
\begin{gather*}
    \theta = \left\{\begin{array}{l}
        0 \\
        \pi
    \end{array}\right.
\end{gather*}
Se invece considerassi il caso di una asta, allora se il pendolo è perpendicolare
l'asta impedisce al punto materiale di cadere e compensa il peso. Matematicamente
si esprime la piccola variazione di teta rispetto alla verticale se il punto materiale
e l'asta sono sopra con l'energia potenziale, nel caso della condizione di equilibrio
$\theta = 0$, l'energia potenziale è minima, altrimenti è massima, infatti facendo la derivata seconda
rispetto al grado di libertà, si ottiene proprio:
\begin{gather*}
    \frac{d^{2} V}{d\theta^{2} } = mgL\cos\theta
\end{gather*}
Quindi:
\begin{gather*}
    V^{''}(0) = mgL > 0, \qquad  V^{''}(\pi) = -mgL < 0,
\end{gather*}
Quindi in $\theta = \pi$ c'è proprio un massimo (equilibrio instabile) per l'energia potenziale
ed un minimo per $\theta = 0$ (equilibrio stabile).
IN più dimensioni la stazionarietà si ottiene con le derivate parziali di
tutte le variabili:
\begin{gather*}
    \frac{\partial V}{\partial x} = 0, \dots 
\end{gather*}
E quindi facendo la derivata parziale due volte rispetto
ad x, y... allora la sua forma può essere o un massimo o un minimo
in funzione proprio delle variabili si ottengono o paraboloidi, oppure una
sella: una condizione di minimo in una variabile ed una di massimo in un'altra.
Sarebbe stabile in una direzione ma sostanzialmente è instabile: nel punto di sella
infatti non c'è una vera stabilità.

\subsection{Molla ed equilibrio sul soffitto}
Scelto il punto di equilibrio dove la molla è a riposo:
\begin{gather*}
    V = \frac{1}2{kx^{2} }
\end{gather*}
messo ora il peso devo considerare anche l'energia potenziale della
forza peso che dipende dalla quota data dalla coordinata $x$ e quindi
sarà negativa. Per trovare la condizione di equilibrio faccio allora la derivata
rispetto ad $x$:
\begin{gather*}
    \frac{dV}{dx} = kx -mg
\end{gather*}
E quindi si annulla proprio per $kx = mg$ ossia dove le due
forze si eguagliano in funzione dell'unica coordinata $x$ e quindi
un solo punto di equilibrio. Troviamo ora se è instabile o stabile
attraverso la derivata seconda:
\begin{gather*}
    \frac{d^{2} V}{dx^{2} } = k > 0, \text{equilibrio stabile}
\end{gather*} 

\subsection{L'antimolla}
\begin{wrapfigure}{r}{0.4\textwidth}
    \centering
    \label{FIg}
    \caption{Antimolla}
    \begin{tikzpicture}
        \draw[decoration={aspect=0.3, segment length=1.2mm, amplitude=1mm,coil},decorate,opacity=0.9] (0, 2) -- (0.5, 1.8);
        \draw[dashed](0, 0) -- (0, 2) node[midway, left] {$L$};
        \draw(0, 0) -- (0.5, 1.8) node[midway, right] {$L$};
        \draw(0, 0.5) arc(90:75:0.5) node[midway, above] {$\theta$};
        \filldraw(0.5, 1.8) node[anchor = west] {$m$};2
    \end{tikzpicture}    
\end{wrapfigure}
Nell'antimolla l'asta ha massa zero e vincola il punto materiale ad avere solo un grado di libertà
facendo sì che l'energia potenziale della massa diventi proprio:
\begin{gather*}
    V = \frac{1}{2}kL^{2}\theta^{2} + mgl\cos\theta  
\end{gather*}
E allora si ottiene l'espressione, utilizzando taylor per il coseno:
\begin{gather*}
    \cos\theta \approx 1 - \frac{\theta^{2} }{2} \\
    V \approx \frac{1}{2}kL^{2}\theta^{2} + mgl\left(1 - \frac{\theta^{2} }{2}\right)  
\end{gather*}
Facendo la derivata prima si ottiene la soluzione::
\begin{gather*}
    \frac{dV}{d\theta} = L(kL - mg)\theta \\
    \theta = 0.
\end{gather*}
C'è anche il caso in cui $kL = mg$; facendo invece la derivata seconda si ha:
\begin{gather*}
    \frac{d^{2} V}{d\theta^{2} } = L(kL -mg)
\end{gather*}
Se $kL > mg$ allora la derivata seconda è maggiore di zero
ho un equilibrio stabile, altrimenti la derivata seconda è minore
di zero e ho un equilibrio instabile, se sono uguali allora fisicamente
la derivata seconda è zero e significa che è un equilibrio indifferente.

\section{Studio del moto con un piccolo spostamento dall'equilibrio}
In funzione di una certa distanza $x$ rispetto alla posizione di equilibrio $x_{eq}$
è dato dallo sviluppo di taylor della funzione dell'energia potenziale:
\begin{gather*}
    V(x) = V(x_eq) + \frac{dV}{dx}|_{x = x_{eq}}(x - x_{eq}) + \frac{1}{2}\frac{d^{2} V}{dx^{2} }|_{x = x_{eq}}(x - x_{eq})^{2} + \dots 
\end{gather*}
Allora date le definizioni di prima con le derivate prime e seconde si ha proprio:
\begin{gather*}
    V(x) = c + \frac{1}{2} k(x - x_{eq})^{2}   
\end{gather*}
Dove $V_{eq} = c$ e $k$ è un parametro della derivata seconda che può 
essere positivo o negativo.
\begin{gather*}
    k = \left\{\begin{array}{l l}
        k > 0 & \text{moto armonico}\\
        k < 0 & \text{moto esponenziale}
    \end{array}\right.
\end{gather*}
\begin{wrapfigure}{r}{0.4\textwidth}
    \centering
    \label{fidjs}
    \caption{dsf}
    \begin{tikzpicture}
        \draw[->](0, 0) -- (0, 3) node[at end, left] {$V(x)$};
        \draw[->](0, 0) -- (3, 0) node[at end, below] {$x$};
        \draw( 0, 2) ..controls(1, -1) and (1.5, 4) .. (3, 0);
        \draw[thin, dashed] (0, 1.25) -- (3, 1.25) node[at start, left] {$E_1$}; 
    \end{tikzpicture}    
\end{wrapfigure}
Questo vuol dire che possiamo sapere, discostandosi di poco
rispetto all'equilibrio, il tipo di moto che accadrà. 
Spesso si trova il grafico rispetto al parametro e quindi quando
si ha un vincolo si ha l'energia potenziale con un vincolo liscio e forze
conservative, allora si conserva l'energia e quindi essendo l'energia
la somma tra la potenziale e la cinetica, l'energia cinetica è sempre
positiva per definizione. Questo però ci pone dei limiti poiché
una è sempre positiva o zero e l'altra cambia sempre di segno e l'energia
meccanica è sempre costante. Si ha quindi una situazione in cui
ci sono alcune configurazioni  permesse mentre altre non sono proprio possibili
a causa della natura dell'energia cinetica: nel caso in figura le configurazioni
permesse sono sempre quelle sotto l'energia meccanica, altrimenti l'energia cinetica dovrebbe essere
negativa.
\begin{gather*}
    E = \frac{1}{2}mv^{2} + V(x) \\
    v = \pm \sqrt{2\frac{E - V(x)}{m}}   
\end{gather*}
Per costruzione, dipendendo da $x$ cambierà il segno. Si può allora
fare qualcosa di più e quindi essendo la velocità
\begin{gather*}
    v = \frac{ds}{dt} , \quad v = \dot{s}, \quad s = s(x). 
\end{gather*}
Posso separare le variabili $s, t$ e quindi:
\begin{gather*}
    \frac{ds}{ \pm \sqrt{2\frac{E - V(s)}{m}} } = dt
\end{gather*}
Prendendo allora l'integrale da entrambe le parti tra due intervalli
di tempo $t_1, t_2$ e $s_1, s_2$ allora:
\begin{gather*}
    \int_{s_1}^{s_2} \frac{ds}{ \pm \sqrt{2\frac{E - V(s)}{m}} } = \int_{t_1}^{t_2}dt \\
    2(t_2 - t_1) = \int_{s_1}^{s_2} \sqrt{\frac{2m}{E - V(s)}} \ ds.   
\end{gather*}
In linea di principio posso calcolarmi questo integrale nelle piccole
oscillazioni. Si introduce allora il concetto di \textbf{barriera di potenziale}:
ossia l'altezza oltre la quale la pallina non può muoversi
tra due cunette, non può allora andare nell'altra cunetta e
classicamente non può essere attraversato. 

\section{Sistemi non approssimabili ad un punto materiale (Dinamica dei sistemi)}
Il mio sistema non è più formato da un punto materiale ma da degli
oggetti estesi  che possono essere divisi in tanti punti materiali e studiarne
la dinamica. IN generale posso scegliere quali corpi posso ingegare nel mio sistema
a seconda di come è più comodo.
Per ciascun punto conosco massa e vettore posizione. Il centro di massa è un
punto ideale definito come (in un sistema discreto):
\begin{align}
    \vec{r}_c = \frac{\sum_{i = 1}^{N}m_i \vec{r}_i}{\sum_{i = 1}^{N} m_I}   
\end{align}
Dove $m_i$ è la massa propria di ogni misura, posso definire nella stessa
maniera la massa del centro di massa come la somma delle masse totali e quindi
ottenere l'espressione di prima come:
\begin{gather*}
    \vec{r}_c = \frac{1}{M}\sum_{i = 1}^{N}m_i \vec{r}_i   
\end{gather*}
IN un sistema continuo non posso semplicemente fare l'integrale di quella roba
ma devo invece considerare in volumetto elementare di massa molto piccola
$dm$ e volume $dV$ e, se sono sufficiente piccole, allora $dm = \rho dV$ tra di loro
ed in generale sono funzione della posizione $\vec{r}$, allora posso sostituire questa
espressione con un integrale come:
\begin{gather*}
    \vec{r} = \frac{\int_{V} \vec{r} \rho dV}{\int_{V} \rho dV} 
\end{gather*}

C'è una relazione se divido in due il corpo? Il centro di massa di tutto
l'oggetto è sempre il solito ma posso considerare le somme pe rle due partizioni
e quindi il centro di massa è proprio la somma tra le due masse delle partizioni.


\end{document}
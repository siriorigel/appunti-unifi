\documentclass[a4paper, oneside]{article}
\usepackage{graphicx}
\usepackage{amsthm}
\usepackage{amsmath}
\usepackage[a4paper,
            bindingoffset=0.2in,
            left=2cm,
            right=2cm,
            top=2cm,
            bottom=2cm,
            footskip=.25in]{geometry}
\usepackage[italian]{babel}
\usepackage{pgfplots}
\usepackage{tabularx}
\usepackage{wrapfig}
\graphicspath{ {./images/} }
\usetikzlibrary{datavisualization}
\usetikzlibrary{datavisualization.formats.functions}
\pgfplotsset{width=10cm,compat=1.9}

\title{Fisica}
\author{Tommaso Miliani}
\date{17-02-25}

\begin{document}
\theoremstyle{definition}
\theoremstyle{theorem}
\theoremstyle{lemma}

\newtheorem{definition}{Definizione}[section]
\newtheorem{theorem}{Teorema}[section]
\newtheorem{lemma}{Proposizione}[theorem]

\maketitle

\section{Esercizio sulla dinamica di fletcher}
\begin{wrapfigure}{r}{0.4\textwidth}
    \centering
    \label{Fig 1.2}
    \caption{Carrucola}
    \begin{tikzpicture}
        \draw (4, 3) -- (4, 0);
        \draw (4.35, 3.19) -- (4.35, 2.0);
        \draw (1.5, 3.35) -- (4.17, 3.35);
        \draw (4.1, 2.0) rectangle (4.6, 1.6);
        \draw (1, 3) rectangle (1.5, 3.5);
        \draw [->] (1.4, 3.25)--(2, 3.25);
        \draw [->] (1.25, 3.27)--(1.25, 3.75);
        \draw [->] (1.25, 3.23)--(1.25, 2.75);
        \draw (4.2, 3.2) circle (0.15) node[right] {}; 
        \draw (0, 3) -- (4, 3); 
    \end{tikzpicture}
\end{wrapfigure}
Per il primo oggetto si scrive:
\begin{align}
    \vec{N}_1 + m_1\vec{g} + \vec{T}_1 = m_1 \vec{a}_1 \\
    \vec{T}_2 + m_2 \vec{g} = m_2 \vec{a}_2        
\end{align}
Utilizzando due versori ($\hat{k}$ per il primo corpo) allora si ha che i vettori
hanno i seguenti versori:
\begin{gather*}
    \vec{N}_1 = N_1 \hat{k} \\
    \vec{g} = g \hat{k}    
\end{gather*}
Allora si hanno le seguenti equazioni per i corpi:
\begin{gather*}
    \vec{T}_1  = m_1 \ddot{x}_1 \\
    N_1 - m_1g = 0 
\end{gather*}
E per il secondo:
\begin{gather*}
    T_2 - m_2 g = m_2 \ddot{z}_2 \\
\end{gather*}
Se la fune rimane tesa per tutto l'esercizio e quindi si hanno le 
proprietà del filo ideale, allora questo comporta che dal punto di vista
geometrico  si ottiene:
\begin{gather*}
    \Delta \ddot{x}_1 = - \Delta\ddot{z}_2 \\
    \Rightarrow \dot{x}_1 = \dot{z}_2 \\  
    \Rightarrow a = \ddot{x}_1 = - \ddot{z}_2 
\end{gather*}
UN'ulteriore proprietà della funicella ideale è la sua assenza di massa:
allora la forza esercitata da una parte o dall'altra è la stessa e come
conseguenza si ha che le tensioni sono uguali in modulo:
\begin{gather*}
    T_1 = T_2 = T
\end{gather*}
Questo dipende anche dalle proprietà della Carrucola: essa deve essere 
priva di massa per poter avere la relazione scritta sopra. \\
Si può ora riscrivere le equazioni come:
\begin{align}
    T = m_1 a \\
    T - m_2 g = -m_2 a
\end{align}
Dalla prima si sostituisce nella seconda e quindi:
\begin{gather*}
    m_1 a - m_2 g = -m_2 a \\
    \Rightarrow a = g\frac{m_2}{m_1 + m_2}
\end{gather*}
Allora la tensione è esprimibile come:
\begin{align}
    T = g\frac{m_1 m_2}{m_1 + m_2}
\end{align}
L'esercizio di dinamica finisce qui ma se volessimo studiare il comportamento del sistema
nell'istante dopo, allora si muoverebbero di moto uniformemente accelerato.

\section{I vincoli non ideali: la forza di attrito}
\begin{wrapfigure}{r}{0.4\textwidth}
    \centering
    \label{FIg 2.1}
    \caption{La forza di attrito}
    \begin{tikzpicture}
        \draw (0, 0) -- (2, 0);
        \draw (0.5, 0) rectangle (1.5, 1);
        \draw[->] (1, 1) -- (1, 2) node[midway, left] {$\vec{N}$};
        \draw[->] (1.5, 0.5) -- (2, 0.5) node[midway, above] {$\vec{F}$};
        \draw[->] (0.5, 0.5 ) -- (0, 0.5) node[midway, above] {$\vec{F_s}$};
    \end{tikzpicture}    
\end{wrapfigure}
Se il vincolo fosse ideale, applicando una forza molto piccola orizzontalmente
rispetto alla verticale, l'oggetto non si muove. Ma questo perché?
IN una situazione ideale l'oggetto dovrebbe muoversi di moto uniformemente accelerato.
MA in una situazione reale il vincolo applica una forza orizzontale chiamata
\textbf{forza di attrito radente} che impedisce il movimento di un oggetto rispetto
all'altro. Applicando una forza sempre maggiore questa forza di resistenza
si adatta alla forza che viene impressa all'oggetto fino a che oltre una certa
intensità l'oggetto inizia a muoversi. Quindi:
\begin{align}
    \left| \vec{F}_s  \right| \leq \text{max}
\end{align}
Posto che il vincolo possa sopportare un livello di stress molto alto, allora 
$\vec{N}$ può crescere quanto si vuole, se si applica una forza verticale mentre si
applica una forza orizzontale per far muovere l'oggetto, ci si accorge che: 
\begin{align}
    \left| \vec{F}_s  \right| \leq \mu_s \left| \vec{N}\right| 
\end{align} 
Il \textbf{coefficiente di attrito statico} $\mu_s$ è un coefficiente proprio tra
un oggetto e la superficie su cui è in contratto. \\
Sempre una forza tangente al vincolo si oppone al movimento di un oggetto
in movimento sopra di esso poiché si osserva che un oggetto in movimento si ferma
dopo qualche istante come se ci fosse una forza opposta al movimento che si
oppone a quest'ultimo. Questa forza è proprio la \textbf{forza di attrito radente dinamico}
che è $\propto \vec{N}$ così come l'altra. Mentre quella statica si adatta in 
modo da compensare il movimento; questa forza invece ha come modulo sempre
uguale al suo massimo consentito: 
\begin{align}
    \vec{F}_s = -\mu_d \left| \vec{N} \right|\hat{u}_v    
\end{align} 
Dove 
\begin{gather*}
    \hat{u}_v = \frac{\vec{v} }{|\vec{v}|}
\end{gather*}

Essendo i vincoli dei dispositivi meccanici che impediscono dei movimenti 
ma ne lasciano liberi altri, il vincolo è un oggetto materiale e fisico
e le forze vincolari sono forze di interazioni tra due oggetti diversi e quindi
agiscono secondo il terzo principio della dinamica: ogni qualvolta che agisce 
una forza su di un oggetto allora esiste una forza opposta ad N che è agisce
sull'oggetto di partenza.

\subsection{Le superfici perfette}
Supponendo di avere due superfici a contatto e che siano fatti dello stesso identico
materiale; Feimann suppone che le superfici si possano lisciare nel modo 
più perfetto possibile (a dimensioni atomiche), allora quando mettiamo a contatto
queste due superfici si saldano diventando un unico blocco e quindi $\mu_s \to \infty$.

\subsection{Un esempio della schematizzazione delle forze su di un piano inclinato}
\begin{wrapfigure}{r}{0.4\textwidth}
    \centering
    \label{Fig 2.2}
    \caption{Esempio di attrito}
    \begin{tikzpicture}
        \draw (0, 0) -- (4, 0);
        \draw (0, 0) -- (0, 2);
        \draw (0,  2) -- (4, 0);
        \draw (1, 1.5) -- (1.5, 2);
        \draw (2, 1) -- (2.5, 1.5);
        \draw (1.5, 2) -- (2.5, 1.5);
        \draw[->] (2, 1.75) -- (2.5, 2.5) node[at end, left] {$\vec{N} $};
        \draw[->, very thick] (1, 1.5) -- (0, 2) node[midway, above] {$\vec{F_{at}}$}; 
        \draw[->] (1.5, 1.25) -- (1.5, 0.25) node[at end, left] {$m\vec{g} $};
        \draw (3.5, 0.25) arc (150:180:0.5) node[midway, left] {$\alpha$};
        \draw[->, red] (3, 1) -- (4, 0.5) node[at end, below] {$\hat{i}$};
        \draw[->, red] (3, 1) -- (3.5, 2) node[at end, left] {$\hat{j}$};
    \end{tikzpicture}    
\end{wrapfigure}
All'istante $t = 0$ la velocità del corpo sul piano è:
\begin{align}
    \vec{v}(t) = 0 
\end{align}
Le forze stanno in relazione tra di loro nella seguente maniera:
\begin{align}
    \vec{F}_{at} + \vec{N} m\vec{g} = m\vec{a}    
\end{align}
Ma dato il nostro sistema di riferimento noi sappiamo che: 
\begin{gather*}
    \vec{F}_{at} = -F_{at}\hat{i}  \\
    \vec{N} = N\hat{j}  
\end{gather*}
Il vettore forza di gravità invece è dato da:
\begin{gather*}
    \vec{g} = g\sin\alpha \hat{i} - g\cos\alpha \hat{j}   
\end{gather*}
L'accelerazione scomposta diventa allora:
\begin{align}
    a \hat{i } =& -F_{at} + mg \sin\alpha = ma  \\
    a \hat{j} =& N - mg\cos\alpha = 0  
\end{align}
Come ipotesi assumo che $F_{at} = mg\sin\alpha$ e quindi si ha che:
\begin{gather*}
    \left| \vec{F}_{at}  \right| = \left| F_{at} \right| \leq \mu_s \left| \vec{N}  \right|? 
\end{gather*}
Se sì, l'oggetto è fermo, se no, allora l'oggetto è in movimento. QUindi con l'angolo
del piano inclinato posso ricavare:
\begin{gather*}
    mg \sin\alpha \leq \mu_S mg\cos\alpha \Rightarrow \tan\alpha \leq \mu_s
\end{gather*}
Se $\tan\alpha \leq \mu_s$ è fermo , altrimenti non è fermo ma è in movimento e quindi si 
ottiene che:
\begin{gather*}
    F_{at} = \mu_d \left| \vec{N}  \right| = \mu_d mg\cos\alpha
\end{gather*}
E quindi si ottiene l'accelerazione che è data da:
\begin{gather*}
    a = g\left( \sin\alpha - \mu_d \cos\alpha \right)
\end{gather*}

\end{document}
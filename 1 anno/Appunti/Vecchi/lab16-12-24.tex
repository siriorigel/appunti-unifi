\documentclass[a4paper, oneside]{article}
\usepackage{graphicx}
\usepackage{amsthm}
\usepackage{amsmath}
\usepackage[a4paper,
            bindingoffset=0.2in,
            left=2cm,
            right=2cm,
            top=2cm,
            bottom=2cm,
            footskip=.25in]{geometry}
\usepackage[italian]{babel}
\usepackage{pgfplots}
\usepackage{tabularx}
\usepackage{wrapfig}
\graphicspath{ {./images/} }
\usetikzlibrary{datavisualization}
\usetikzlibrary{datavisualization.formats.functions}
\pgfplotsset{width=10cm,compat=1.9}

\title{Lab} 
\author{Tommaso Miliani}
\date{16-12-25}

\begin{document}
\theoremstyle{definition}
\theoremstyle{theorem}
\theoremstyle{lemma}

\newtheorem{definition}{Definizione}[section]
\newtheorem{theorem}{Teorema}[section]
\newtheorem{lemma}{Proposizione}[theorem]

\maketitle

\section{(cap 10, 11)}
La distribuzione binomiale spesso è associata con la probabilità
di trovare l'asso in un dado, ossia avere un certo numero di misure ripetute.
La distribuzione di Poisson è quella che viene fuori da eventi casuali con
tempo caratteristico ben definito (per esempio l'emissione spontanea
di fotoni da parte di atomi). Nel caso di un dado a sei facce si 
elencano alcuni casi:
\begin{enumerate}
    \item La probabilità di trovare una faccia specifica è proprio $\frac{1}{6}$ e che non
        esca quella faccia è la complementare $p' = \frac{5}{6}$. 
    \item La probabilità di trovare tre facce uguali in tre lanci ripetuti e consecutivi
    è proprio $\frac{1}{6^{3}}$.
    \item La probabilità di trovare 2 facce uguali in tre lanci è $ppp' + pp'p + p'pp = 3p^{2}p' $.
    Poiché ho tre disposizioni possibili di risultati. 
\end{enumerate}
Nel caso generale io chiamo $n$ il numero complessivo di prove ed il numero di
successi come $\nu$ e quindi:
\begin{gather*}
    P(\nu \ successi \ in \ n \ prove) =
\end{gather*}
\begin{align}
   b_{n, p}(\nu) = \frac{n!}{\nu! (n - v)!} \cdot  p^{\nu} q^{n - \nu}  
\end{align}
Ho una distribuzione binomiale in funzione di $\nu$ con n e p fissati. (dove $q = 1 -p$). Questa
funzione non è molto diversa da una gaussiana, ma è una funzione discreta
perché è definita nei naturali. Il disaccordo tra la binomiale e la gaussiana
si riduce al crescere di n e quando $n \to \infty $ è proprio una gaussiana.
Verifico:
\begin{gather*}
    \bar{\nu} = \frac{1}{N}\sum_{i = 0}^{n} \nu_i = np\\
    \sigma_{\nu} = \sqrt{\sum_{i = 0}^{n} \frac{(\nu_i - \bar{\nu} )^{2} }{N-1}} = \sqrt{np (1 - p)} 
\end{gather*} 
Con $n$ sorgenti di errore casuale allora $x = X \pm \epsilon$. 

\section{Distribuzione di Poisson}
LA dsistribuzione di Poisson è una distribuzione che si utilizza
molto per i decadimenti radioattivi in quanto si basa su eventi casuali
con un certo intervallo di tempo ben determinato. CI permette di determinare
come fluttuano i successi in un dato bin quando ripeto le stesse misure.
Quindi la probabilità di ottenere $\nu$ successi in un intervallo definito allora è:
\begin{align}
    P_{\mu}(\nu) = e^{-\mu} \cdot  \frac{\mu^{\nu} }{\nu!} 
\end{align}
Per cui si ottiene: 
\begin{gather*}
    \bar{\nu} = \mu , \qquad \sigma_{\nu} = \sqrt{\mu} = \sqrt{\mu}   
\end{gather*}
\begin{wrapfigure}{r}{0.4\textwidth}
    \centering
    \label{Fig 2.1}
    \caption{Poisson}
    \begin{tikzpicture}
        \draw[->] (0, 0) -- (3, 0) node[at end, below] {$x$};
        \draw[->] (0, 0) -- ( 0, 3) node[at end, left] {$\bar{\nu} $}; 
        \draw[|-|, ultra thick] (1.5, 0) -- (1.8, 0) node[midway, below] {$bin$};
    \end{tikzpicture}    
\end{wrapfigure}
$\bar{\nu}$ è proprio il valore della Gaussiana teorica che io costruisco
seguendo proprio la distribuzione di Poisson che ottengo misurandolo
dai dati. L'errore sul valore di $\bar{\nu}$ è $\sqrt{\bar{\nu}  }$.  

\end{document}
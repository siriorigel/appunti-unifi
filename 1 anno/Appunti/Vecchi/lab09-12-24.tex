\documentclass[a4paper, oneside]{article}
\usepackage{graphicx}
\usepackage{amsthm}
\usepackage{amsmath}
\usepackage[a4paper,
            bindingoffset=0.2in,
            left=2cm,
            right=2cm,
            top=2cm,
            bottom=2cm,
            footskip=.25in]{geometry}
\usepackage[italian]{babel}
\usepackage{pgfplots}
\usepackage{tabularx}
\usepackage{wrapfig}
\graphicspath{ {./images/} }
\usetikzlibrary{datavisualization}
\usetikzlibrary{datavisualization.formats.functions}
\pgfplotsset{width=10cm,compat=1.9}

\title{LAb.1}
\author{Tommaso Miliani}
\date{09-12-24}

\begin{document}
\theoremstyle{definition}
\theoremstyle{theorem}
\theoremstyle{lemma}

\newtheorem{definition}{Definizione}[section]
\newtheorem{theorem}{Teorema}[section]
\newtheorem{lemma}{Proposizione}[theorem]

\maketitle

\section{SOmma semplice o qudratica?}
Se ho degli errori non accidentali allora la propagazione degli errori
è data dalla seguente che è sviluppata da Taylor:
\begin{gather*}
    \delta q = \left| \frac{\partial q}{\partial x} \right| \delta x + \left| \frac{\partial q}{\partial y}  \right| \delta y.  
\end{gather*}
Se gli errori sono accidentali ed $x, y$ sono indipendenti allora non si utilizza
la somma delle derivate parziali ma si utilizza la standard deviation ossia:
\begin{gather*}
    \sigma_q = \sqrt{\left( \frac{\partial q}{\partial x}  \right)^{2} \sigma_x^{2} + \left( \frac{\partial q}{\partial y} \right)^{2} \sigma_y^{2} } 
\end{gather*}
QUesto caso è più immediato di quell'altro e molto più preciso. \\
Supponendo di avere errori accidentali dominanti allora ho le mie misure
attraverso le medie: avrò:
\begin{gather*}
    \bar{x} = \sum_{i = 1}^{n} \frac{x_i}{N} \\
    \sigma_x ^{2} = \frac{1}{N- 1} \sum_{i = 1}^{n}(x_i - \bar{x})^{2}   
\end{gather*}

\end{document}
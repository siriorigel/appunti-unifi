\documentclass[a4paper, oneside]{article}
\usepackage{graphicx}
\usepackage{amsthm}
\usepackage{amsmath}
\usepackage[a4paper,
            bindingoffset=0.2in,
            left=2cm,
            right=2cm,
            top=2cm,
            bottom=2cm,
            footskip=.25in]{geometry}
\usepackage[italian]{babel}
\usepackage{pgfplots}
\usepackage{tabularx}
\usepackage{wrapfig}
\graphicspath{ {./images/} }
\usetikzlibrary{datavisualization}
\usetikzlibrary{datavisualization.formats.functions}
\pgfplotsset{width=10cm,compat=1.9}

\title{Lab 1}
\author{Tommaso Miliani}
\date{25-02-25}

\begin{document}
\theoremstyle{definition}
\theoremstyle{theorem}
\theoremstyle{lemma}

\newtheorem{definition}{Definizione}[section]
\newtheorem{theorem}{Teorema}[section]
\newtheorem{lemma}{Proposizione}[theorem]

\maketitle

\section{L'esperienza dell'elasticità}
\begin{wrapfigure}{r}{0.4\textwidth}
    \centering
    \label{FIg werew}
    \caption{dsf}
    \begin{tikzpicture}
        \draw(0, 0) rectangle (2, 0.5);
        \draw(-0.5, 0) -- (0, 0);
        \draw(0, 0) -- (0, -0.5);
        \draw(2.5, 0) -- (2, 0);
        \draw(2, 0) -- (2, -0.5);
        \draw[->](1, 0) -- (1, -1) node[at end, right] {$\vec{F}$};
        \draw[->](0, 0.5) -- (0, 1.5) node[at end, right] {$\vec{N}$};
        \draw[->](2, 0.5) -- (2, 1.5) node[at end, left] {$\vec{N}$};
    \end{tikzpicture}    
\end{wrapfigure}
La situazione sperimentale è di questo tipo: si ha una barretta appoggiata
su dei supporti in modo che sia stabile e che non ci siano deformazioni: è un corpo
isotropo omogeneo approssimabile ad un parallelepipedo. Se si fa flettere
la sbarretta, allora la parte flessa è solo una parte (la zona centrale della
sbarretta). In un mondo ideale si applica una forza alla sbarretta che è appoggiata agli estremi
in modo da poter vedere che ci sono due vincolari a destra e a sinistra:
\begin{gather*}
    \vec{F} + 2\vec{N} = 0  
\end{gather*}
Per descrivere la deformazione della sbarretta possiamo spezzare la barretta
in modo tale da considerare la situazione di destra e di sinistra come se fossero
due flessioni distinte e quindi posso dire per la parte destra:
\begin{gather*}
    M = EI\frac{1}{R}
\end{gather*}
Dove $I$ è il momento di inerzia della sbarretta e $R$ è il raggio della
circonferenza che  approssima al meglio la curvatura e quindi considerata
la funzione $R(x)$ come la funzione del raggio che lo approssima al meglio,
\begin{gather*}
    \frac{1}{R(x)} \approx \left| \frac{d^{2} y }{dx^{2} } \right|  \\
    M(x) \approx  N\left(\frac{L}{2}-x\right)\sin\frac{\pi}{2}
\end{gather*}
COmbinando le equazioni si ottiene
\begin{gather*}
    \frac{F}{2}\left( \frac{L}{2} -x\right) \approx EI \frac{d^{2}y}{dx^{2}}
\end{gather*}
Si può quindi trovare l'integrale:
\begin{gather*}
   \int_{0}^{x} \left| \frac{d^{2} y }{dx^{2} } \right| =  \frac{F}{2EI} \int_{0}^{x} \left( \frac{L}{2} - x\right)  = \\
   \frac{F}{2EI} \left( \left( \frac{L}{2}x - \frac{x^{2} }{2} \right) - 0 \right)
\end{gather*}
Adesso per ottenere l'integrale del $\frac{dy}{dx}$ integro nuovamente
questo risultato ottenendo:
\begin{gather*}
    \int_{0}^{x}\frac{dy}{dx} =  \frac{F}{2EI} \int_{0}^{x} \left( \left( \frac{L}{2}x - \frac{x^{2} }{2} \right) - 0 \right) = \\
    \frac{F}{2EI} \left( \frac{Lx^{2} }{4} - \frac{x^{3} }{6}\right)
\end{gather*}
Nel piano delle fibre neutre calcoliamoci $y(\frac{L}{2})$ e ottenere a questo
punto dopo tutti i conti:
\begin{gather*}
    y\left( \frac{L}{2} \right) = \frac{FL^{3} }{48 E I}
\end{gather*}
Chiamando $A$ il Lato della faccia vincolata parallela all'asse x e B
l'altra parallela all'asse y, si ricava che 
\begin{gather*}
    I = \frac{1}{12}ab^{3} 
\end{gather*}
E quindi sostituendo questo risultato si ottiene:
\begin{gather*}
    y\left( \frac{L}{2} \right) = \frac{FL^{3} }{4 E ab^{3} }
\end{gather*}
A questo punto la variazione di quota della sbarretta risulta essere
\begin{gather*}
    f \approx cF
\end{gather*}
Dove
\begin{gather*}
    c = \frac{L^{3} }{4 E ab^{3} }
\end{gather*}
che è chiamata coefficiente di deformazione
\end{document}
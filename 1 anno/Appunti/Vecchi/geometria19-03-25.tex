\documentclass[a4paper, oneside]{article}
\usepackage{graphicx}
\usepackage{amsthm}
\usepackage{amsmath}
\usepackage[a4paper,
            bindingoffset=0.2in,
            left=2cm,
            right=2cm,
            top=2cm,
            bottom=2cm,
            footskip=.25in]{geometry}
\usepackage[italian]{babel}
\usepackage{pgfplots}
\usepackage{tabularx}
\usepackage{tikz}
\usepackage{wrapfig}
\usepackage{color}
\definecolor{page}{rgb}{0.129,0.157,0.212}
\pagecolor{page}
\color{white}
\graphicspath{ {./images/} }
\usetikzlibrary{shapes.geometric}
\usetikzlibrary{datavisualization}
\usetikzlibrary{datavisualization.formats.functions}
\pgfplotsset{width=10cm,compat=1.9}

\title{geometric}
\author{Tommaso Miliani}
\date{19-03-25}

\begin{document}
\theoremstyle{definition}
\theoremstyle{theorem}
\theoremstyle{lemma}

\newtheorem{definition}{Definizione}[section]
\newtheorem{theorem}{Teorema}[section]
\newtheorem{lemma}{Proposizione}[theorem]
\newtheorem{example}{Esempio}[section]

\maketitle

\section{Autovettori}
\begin{theorem}
    SIa $V$ uno spazio vettoriale su di un campo $K$, di
    dimensione finita $n$ e sia $f : V \to V$ lineari. Sia $\lambda \in K$.
    Affermo allora che $\lambda$ è radice del polinomio caratteristico se
    e solo se $\lambda$ è autovalore di $f$. 
\end{theorem}
\begin{proof}
    $\lambda$ è autovalore di f se e solo se 
    \begin{gather*}
        \exists v \in V - \{0\} : f(v) = \lambda v.
    \end{gather*}
    ossia equivale a dire che
    \begin{gather*}
        \exists v \in V - \{0\} : f(v) - \lambda I_V(v) = 0 
    \end{gather*}
    E quindi questo equivale a dire che sono linearmente indipendenti e quindi 
    equivale a dire che il rango meno $\lambda$ volte l'identità è minore di $n$.
    \begin{gather*}
        rk M_{B, B}(f - \lambda I_V) < n
    \end{gather*}
    Ossia è come dire
    \begin{gather*}
        \det(M_{B, B}(f) \lambda I_V) = 0
    \end{gather*}
    e quindi si ottiene la tesi:
    \begin{gather*}
        P_f(\lambda) = 0
    \end{gather*}
\end{proof}

\begin{lemma}
    Sia $A \in M(n \times n, K)$ dove $K$ è un campo e $n \in N - \{0\}$ 
    allora:
    \begin{enumerate}
        \item Il termine di grado zero di $P_A$ è $\det(A)$;
        \item $P_A$ ha grado $n$ e scrivendo i termini di tutti i gradi si ha:
        \begin{align}
            P_A(t) = (-1)^{n} t^{n} + (-1)^{n - 1}tr(A)t^{n - 1} \dots + \det(A)    
        \end{align}
    \end{enumerate}
\end{lemma}
\begin{proof}
    Per la prima per calcolare il termine di grado zero basta porre
    $x = 0$ e quindi il termine di grado zero è proprio il determinante poiché
    \begin{gather*}
        P_A(0) = \det(A - 0I_n) = \det(A) \\
        (P_A(t) = \det(A - tI_n))
    \end{gather*}
    Per la seconda si procede per induzione su n:
    \begin{gather*}
        n = 1, \qquad A = (a), \qquad  P_A(t) = \det((a) - tI_n)
    \end{gather*}
    Allora si ottiene, per la matrice $A, n \times n$:
    \begin{gather*}
        P_A(t) = \det(A - tI_n) = \det(\dots)
    \end{gather*}
    Sviluppando per la prima colonna e moltiplicando per 
    la matrice complementare si ottiene il determinante.
\end{proof}

\begin{lemma}
    Sia $A \in M(n \times n, K)$ se
    \begin{gather*}
        P_A(t) = (\lambda_1 - t)\dots
    \end{gather*}
    allora
    \begin{align}
        \det(A) &= \lambda_1 \cdot  \dots \cdot  \lambda_n \\
        tr(A) &= \lambda_1 + \dots + \lambda_ n
    \end{align}
\end{lemma}
\begin{proof}
    Per la proposizione appena vista il polinomio caratteristico di A
    è proprio:
    \begin{gather*}
        P_A(t) = (-1)^{n}t^{n} + (-1)^{n - 1}tr(A)t^{n - 1} \dots    
    \end{gather*}
    Ma per ipotesi io so che:
    \begin{gather*}
        P_A(t) = (\lambda_1 - t)\dots = (-t)^{n} 
    \end{gather*}
    E allora è dimostrata. 
\end{proof}

\begin{definition}
    Sia $V$ uno spazio vettoriale su di un campo $K$, allora
    se $f$ è l'applicazione lineare su quello spazio e $\lambda$ un
    suo autovalore allora definisco \textbf{autospazio di $f$ relativo a $\lambda$ }
    il seguente sottospazio vettoriale di dV:
    \begin{align}
        \ker(f - \lambda I_V)
    \end{align}
    che coincide con 
    \begin{gather*}
        \{v \in V : v \text{ autovettore di f con autovalore } \lambda\} \cup \{0_V\}
    \end{gather*}
\end{definition}
I due insiemi sono uguali perché
\begin{gather*}
    v \in Ker(f - \lambda I_V) \Leftrightarrow (f - \lambda I_V)(v) = 0 \Leftrightarrow f(v) - \lambda I_V(v) = 0 \Leftrightarrow f(v) = \lambda v 
\end{gather*}

\begin{definition}
    Si definisce \textbf{molteplicità geometrica di $\lambda$} come la dimensione
    dell'autospazio e si indica come $m_g(\lambda)$. Per cui $\lambda$ è autovalore
    per $f$ se e solo se $m_g(\lambda) \geq 1$. 
\end{definition}

\begin{definition}
    Supponiamo adesso che la definizione di $V$ sia finita, allora 
    posso definire \textbf{molteplicità algebrica di $\lambda$} il massimo
    $s \in N$ rale che:
    \begin{align}
        (t - \lambda)^{s}  
    \end{align}
    divide $p_f$ e si denota come $m_a(\lambda)$.Per cui $\lambda$ è autovalore
    per $f$ se e solo se $m_a(\lambda) \geq 1$. 
\end{definition}


\end{document}
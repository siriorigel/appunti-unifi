\documentclass[a4paper, oneside]{article}
\usepackage{graphicx}
\usepackage{amsthm}
\usepackage{amsmath}
\usepackage[a4paper,
            bindingoffset=0.2in,
            left=2cm,
            right=2cm,
            top=2cm,
            bottom=2cm,
            footskip=.25in]{geometry}
\usepackage[italian]{babel}
\usepackage{pgfplots}
\usepackage{tabularx}
\usepackage{tikz}
\usepackage{wrapfig}
\usepackage{color}
\pagecolor{black}
\color{white}
\graphicspath{ {./images/} }
\usetikzlibrary{shapes.geometric}
\usetikzlibrary{datavisualization}
\usetikzlibrary{datavisualization.formats.functions}
\pgfplotsset{width=10cm,compat=1.9}

\title{Fisica (BANCHI)}
\author{Tommaso Miliani}
\date{13-03-25}

\begin{document}
\theoremstyle{definition}
\theoremstyle{theorem}
\theoremstyle{lemma}

\newtheorem{definition}{Definizione}[section]
\newtheorem{theorem}{Teorema}[section]
\newtheorem{lemma}{Proposizione}[theorem]
\newtheorem{example}{Esempio}[section]

\maketitle

\section{Esercizi banchi}
\subsection{Esercizio su moodle (DPM)}
\begin{wrapfigure}{r}{0.4\textwidth}
    \centering
    \label{Figsnd0i}
    \caption{culo}
    \begin{tikzpicture}
        \draw(0, 0) -- (4, 0);
        \draw(2.5, 0) rectangle (3, 0.5) node[midway] {$B$};
        \draw(2.75, 0.5) -- (2.75, 2) node[midway, right] {$l$};
        \draw(2.75, 2) -- (1.25, 1) node[midway, above] {$l$};
        \filldraw(1.25, 1) circle (1pt) node[anchor= east] {$A$};
        \draw[->](3.5, 0) -- (4, 0) node[at end, below] {$x$};
        \draw[->](3.5, 0) -- (3.5, 0.5) node[at end, left] {$y$};
        \draw[->, very thick, red](1.25, 1) -- (2, 1.5) node[midway, above] {$\vec{T}_A$};
        \draw[->, very thick, red](1.25, 1) -- (1.25, 0.25) node[at end, right] {$m_A\vec{g}$};
        \draw[->, very thick, red](2.75, 0) -- (2.75, -0.5) node[at end, right] {$m_B\vec{g}$};
        \draw[->, very thick, red](2.75, 0.5) -- (2.75, 1) node[at end, right] {$m_B\vec{g}$};
    \end{tikzpicture}    
\end{wrapfigure}
Presi due punti materiali e con l'approssimazione
che il filo è teso e che i due oggetti appesi sono due
punti materiali, si assume anche che B sia in quiete e che
A parta da fermo quando $\theta = \frac{\pi}{2}$.\\
Posto il sistema di riferimento, allora posso misurare l'accelerazione
con un sistema di riferimento ideale e il filo inizialmente
uguale e potrebbe scorrere (si assume quindi un solo grado di libertà 
dicendo che B è in quitee e che il filo non scorra).
Studiando B si ottiene la relazione:
\begin{gather*}
     \vec{T}_B  + \vec{N} + m_b \vec{g} = 0   
\end{gather*}
Studiano invece A, il quale si muove all'istante zero e quindi
posso dier:
\begin{gather*}
    m_A\vec{a}_A = m_A \vec{g} + \vec{T}_A   
\end{gather*}
Ora dato che A si muove, la sua traiettoria sarà una parabola 
e quindi posso dire dall'equazione vettoriale devo capire quando si alza:
con la schematizzazione del problema posso dire che B si alza quando A inizia
a muoversi, definiti allora i versori di A, posso dire che la proiezione
lungo $\hat{n}$ è dato da: 
\begin{gather*}
    -m_A l \dot{\theta}^{2} = m_A g \cos\theta - T_A 
\end{gather*}
ALlora posso dire che:
\begin{gather*}
    T_A = m_A g \cos\theta + m_A l \dot{\theta}^{2} 
\end{gather*}
E quindi
\begin{gather*}
    N = m_B g - T_B \geq 0
\end{gather*}
E allora posso riscrovere, dato che la velocitià di A posso esprimerla
come $l \dot{\theta}$,
\begin{gather*}
    m_A(g \cos\theta + \frac{v_A^{2} }{l}) \leq m_B g 
\end{gather*}
DAto che tutte le forze sono conservative, allora posso utilizzare
la conservazione dell'energia meccanica anche perché c'è solo un vincolo:
\begin{gather*}
    E = \frac{1}{2}m_A v_A^{2} + m_A gy_A 
\end{gather*} 
Allora essendo che tutta questa per deginizione è una costatne,
allora devo eguagliarla all'energia al tempo zero:
\begin{gather*}
    E_i  = E(v_A = 0, y_A = l) \Rightarrow = m_A g l \\
\end{gather*}
Imponendo le due uguaglianze ottengo:
\begin{gather*}
    v_A^{2} = 2g(l -y_A) 
\end{gather*}
Sostituendo la velocità nella formula trovata prima posso ottenere,
posto
\begin{gather*}
    \Delta y = l - y_A = l\cos\theta \geq 0
\end{gather*}
Da qui trovo il coseno di teta e quindi Sostituendo tutto 
nell'espressione sopra
\begin{gather*}
    m_A(\frac{\Delta y}{l} + 2\frac{\Delta y}{l}) \leq m_B
\end{gather*}
Allora ottengo l'espressione per il delta y come:
\begin{gather*}
    \Delta y_{max} = \frac{m_B l}{3m_A}
\end{gather*}
Ossia il valore massimo oltre il quale la masssa $B$ inizia
ad alzarsi e quindi fa scorrere il filo. 

\subsection{Esercizio di esame febbraio 2021}
\begin{wrapfigure}{r}{0.4\textwidth}
    \centering
    \label{asd}
    \caption{IMMAGINE IDS DS MDOOODLE DE D}
    \begin{tikzpicture}
        
    \end{tikzpicture}    
\end{wrapfigure}
DAto il testo (che si spera l'utente abbia letto) si fanno le seguenti
assunzioni: essendo la guida fissa, essa è descritta dall'equazione
$y = kx^{2}$, avendo a che fare con un punto materiale non essendoci
attrito, ho anche a che fare con un vincolo bilatero, e ho un SDR inerziale
poiché la guida non si sposta ed è fissa. Quello che posso fare è
scrivere le forze in gioco e la reazione vincolare della guida. Per
ipotesi di mancanza di attrito so anche che $N$ è ortogonale alla
traiettoria. Tutte le forze sono conservative e quindi l'energia meccanic
si conserva ed il vincolo è bilatero 
\begin{gather*}
    E = \frac{1}{2}mv^{2} - mgy =  const = 0. 
\end{gather*} 
Allora posso ricavare la velocitià:
\begin{gather*}
    v = \sqrt{2gy} 
\end{gather*}
Essendo $y = kx^{2}$, la $y$ finale, ossia quella priam che tocchi terra
è qurlla della x massima e quindi la $x_{MAX}$ è semplicemente L e allora 
\begin{gather*}
    v_F = \sqrt{2gkL^{2} } 
\end{gather*} 
PUNTO B: \\
\begin{wrapfigure}{r}{0.4\textwidth}
    \centering
    \label{dfs}
    \caption{fdas}
    \begin{tikzpicture}
        
    \end{tikzpicture}    
\end{wrapfigure}
Mentre $\vec{N}_x$ è concorde con l'asse di riferimento, allora
il segno di $\vec{N}_y$ non lo è e per come abbiamo scelto il sistema
di riferimento non è possibile che sia concorde.
DAto che la guida può solo spingere ma non tirare, allora
\begin{gather*}
    \vec{N}_x, \vec{N}_y \geq 0  
\end{gather*}
Ora
\begin{gather*}
    \vec{N} + m\vec{g} = m\vec{a}   
\end{gather*}
Che posso proiettare lungo la direzione $x$ ed $y$ ed ottengo
che 
\begin{gather*}
    N_x = m\dot{x} \\
    mg - m\dot{y} = N_y
\end{gather*}
DAto che si hanno due gradi di libertà, dobbiamo utilizzare delle
approssimazioni: fin tanto che il corpo è attaccato alla guida allora
questo soddisfa le equazioni della guida:
\begin{gather*}
    \vec{r} = P - O = x\hat{i} + kx^{2}\hat{j}    
\end{gather*}
Derivando, si ottiene l'eqauzione della velocitià:
\begin{gather*}
    \vec{v} = \dot{x}\hat{i} + 2kx\dot{x} \hat{j}   
\end{gather*}
Derivando ancora:
\begin{gather*}
    \vec{a} = \ddot{x}\hat{i} + 2k(\dot{x}^{2} + x\ddot{x})\hat{j}  
\end{gather*}
Il secondo metodo è utilizzare l'energia meccanica:
\begin{gather*}
    v^{2}  =2gy  
\end{gather*}
Sostituendo la relaszione ottenuta preimac di v:
\begin{gather*}
    v^{2} = \dot{x}^{2}(1 + 4k^{2}x^{2}) = 2gkx^{2}   
\end{gather*}
Allora:
\begin{gather*}
    \dot{x}^{2} = \frac{2gkx^{2} }{1 + 4k^{2}x^{2}  } 
\end{gather*}
METODO ALTERNATIVO: \\
COnsiderato che $\vec{N}_y$ e $\vec{N}_x$ non sono più generici
e io so che $N$ deve essere ortogonale alla traiettoria:
allora se vado a disegnare la traiettoria, avrò un versore tangente
ed un versore invece ortogonale ad $\hat{u}_t$:
\begin{gather*}
    \vec{N} \cdot  \hat{u}_C = 0 \\
    \vec{N} \cdot \vec{v} = 0 \\     
\end{gather*}   
Sostituendo quelle equzioni della velocitià allora posso ottenere:
\begin{gather*}
    N_x \dot{x} - N_y2kx\dot{x} = \dot{x}(N_x - 2kxN_y)
\end{gather*}
Ci sono allora due soluzioni per questa equazione:
\begin{gather*}
    N_x = 2kxN_y \quad e \quad x = 0
\end{gather*}
Da l'equazione:
\begin{gather*}
    \left\{\begin{array}{l}
        N_x = m\ddot{x} \\
        N_Y = m(g - \ddot{y})
    \end{array}\right. \\
    \ddot{x} = \frac{N_x}{m} = \frac{2kxN_y}{m}
\end{gather*}
Sostituendo all'interno della equzione per $N_y$ allora:
\begin{gather*}
    mg - N_y = m2k(\dot{x} + \frac{2kxN_y}{m})
\end{gather*}
SO anche che:
\begin{gather*}
    \dot{x}^{2} = \frac{2gkx^{2} }{1 + 4k^{2}x^{2}  } 
\end{gather*}
ALlora sostituendo e raccogliendo $N_y$:
\begin{gather*}
    N_y(1  + 4k^{2}x^{2} ) = mg(1 - \frac{4k^{2} x^{2} }{1 + 4k^{2} x^{2}  })
\end{gather*}
E allora:
\begin{gather*}
    N_y = \frac{mg}{(1 + 4k^{2}x^{2} )^{2} }
\end{gather*}
E allora non si stacca mai dalla guida. 

\end{document}
\documentclass[a4paper, oneside]{article}
\usepackage{graphicx}
\usepackage{amsthm}
\usepackage{amsmath}
\usepackage[a4paper,
            bindingoffset=0.2in,
            left=2cm,
            right=2cm,
            top=2cm,
            bottom=2cm,
            footskip=.25in]{geometry}
\usepackage[italian]{babel}
\usepackage{pgfplots}
\usepackage{tabularx}
\usepackage{wrapfig}
\graphicspath{ {./images/} }
\usetikzlibrary{datavisualization}
\usetikzlibrary{datavisualization.formats.functions}
\pgfplotsset{width=10cm,compat=1.9}

\title{Fisica 1}
\author{Tommaso Miliani}
\date{28-02-25}

\begin{document}
\theoremstyle{definition}
\theoremstyle{theorem}
\theoremstyle{lemma}

\newtheorem{definition}{Definizione}[section]
\newtheorem{theorem}{Teorema}[section]
\newtheorem{lemma}{Proposizione}[theorem]

\maketitle

\subsection{Deviazione verso oriente dei gravi in caduta}
\begin{wrapfigure}{r}{0.4\textwidth}
    \centering
    \label{gjds}
    \caption{gd}
    \begin{tikzpicture}
        \draw[->](0, 0) -- (4, 0) node[at end, below] {$x$} node[midway, below] {$R_T$};
        \draw[->](0, 0) -- (0, 4) node[at end, left] {$z$};
        \draw(0, 2) arc (90 :0:2);
        \draw(2, 0) -- (3, 0) node[at end, above] {$m$};
        \filldraw (3, 0) circle(1pt);
    \end{tikzpicture}    
\end{wrapfigure}
Nel sistema di riferimento della Terra accade un fenomeno molto peculiare quando si fa cadere un grave 
da una certa altezza all'equatore: l'oggetto appare deviato sempre verso oriente. Al momento $t= 0$ posso
dire che, dal momento che non c'è vincolo esplicito ottenendo:
\begin{gather*}
    \left\{\begin{array}{c}
        x = R_T m h \\
        y = 0 \\
        z = 0
    \end{array}\right. \qquad \left\{\begin{array}{c}
        \dot{x} = 0 \\
        \dot{y} = 0 \\
        \dot{z} = 0   
    \end{array}\right.
\end{gather*}
Faccio un'approssimazione lecita (ma pur sempre un approssimazione) secondo la quale io
considero che non ci sia movimento rispetto all'asse y trascurando il fatto che
possa variare e quindi 
\begin{gather*}
    m\vec{g}' = -mg'\hat{i}   
\end{gather*}
Potremmo ora avere la forza di Coriolis, intanto esplicitiamo la velocità
relativa:
\begin{gather*}
    \vec{\Omega} = \Omega \hat{k}, \qquad \Omega = 7 \cdot 10^{-5} \\   
    \vec{v}_R = \dot{x}\hat{i} + \dot{y}\hat{j} + \dot{z}\hat{k} \\
    \vec{F}_{co} = -2m\Omega\hat{k} \times (\dot{x} \hat{i} + \dot{y} \hat{j} +\dot{z}\hat{k}) = \\
    =2m\Omega(\dot{y}\hat{i} - \dot{x}\hat{j})   
\end{gather*}
Scrivendo il secondo principio della dinamica, allora trovo l'accelerazione
che è data proprio da quella relativa e quindi posso scrivere in componenti:
\begin{gather*}
    \vec{F} = m\vec{g} + \vec{F}_{co}   
\end{gather*}
Mentre la forza di Coriolis lungo g ha questo:
\begin{align*}
    x:& \quad -mg' + 2 m \Omega \dot{y} = m\ddot{x}   \\
    y:& \quad -2m\Omega \dot{x} = m\ddot{y} \\
    z:& \quad 0 = m\ddot{z} 
\end{align*}
    Questo vuol dire che z è sempre zero così come la derivata prima e seconda e quindi
non agisce alcuna forza lungo l'asse z. E quindi non ho alcun moto
nel piano $x y$. Riscrivendo la formula nuovamente si può ottenere:
\begin{gather*}
    \ddot{x} = -2\Omega \dot{y} = -g' \\
    \ddot{y} = -2\Omega \dot{x}    
\end{gather*}
Prendendo ora l'integrale generico delle funzioni differenziali (che sono in questo caso
accoppiate poiché dipendono luna dall'altra):
\begin{gather*}
    \int_{0}^{t}\ddot{y} \ dt = -2\Omega \int_{0}^{t} \dot{x}  \ dt = \\
    = \dot{y}(t) - \dot{y}( 0) = -2\Omega (x(t) - x(0))  
\end{gather*}
Dato che le condizioni iniziali mi impongono che $y = 0$, allora si ottiene 
la seguente relazione una volta svolto l'integrale:
\begin{gather*}
    \dot{y} = -2\Omega (x - (R_T + h)) 
\end{gather*}
Il valore massimo in modulo è il valore minimo di $R_T$ e quindi il modulo massimo è:
\begin{gather*}
    \left| \dot{y}  \right|_{max} = 2\Omega h 
\end{gather*}
Ossia il valore massimo lungo $\dot{y}$ della velocità, che in questo caso è 
un termine molto piccolo e quindi posso trascurarlo come da ipotesi iniziale,
inoltre questo ci dice che 
\begin{gather*}
    |\dot{y}| < 2\Omega h = 14 \cdot  10^{3} m/s  
\end{gather*} 
Ma quindi per gli altri valori di y:
\begin{gather*}
    |2\Omega \dot{y}| < 4\Omega^{2}h = 2 \cdot  10^{-6}m/s^{2}    
\end{gather*}
che devo confrontare con $g$, la quale è 7 ordini di grandezza più grande 
facendo sì che domini il moto lungo $x$ e poiché la forza di Coriolis
 ha un effetto minimo si può ragionevolmente trascurare,e quindi si ottiene che:
\begin{gather*}
    |2\Omega\dot{y}| << g' 
\end{gather*}
Le equazioni iniziali diventano dunque:
\begin{gather*}
    \ddot{x} = g' \\
    \ddot{y} = -2\Omega \dot{x} 
\end{gather*}
L'ho integrata solo per trovare l'ordine di grandezza del termine e quindi mi è servita solo
per una pura stima e non ho ancora trovato il valore effettivo di y. Allora posso mettere 
i valori iniziali e quindi:
\begin{gather*}
    \dot{x} = -g't \\
    x = -\frac{1}{2}g' t^{2} + (R_T + h) 
\end{gather*}
Avendo ora trovato $\dot{x}$ posso sostituire in $y$ e ottenere che:
\begin{gather*}
    \ddot{y} = 2\Omega g't
\end{gather*} 
Con le approssimazioni che ho fatto mi sono trovato ora le 
formule effettive e quindi si ottiene che:
\begin{gather*}
    \dot{y} = \Omega g' t^{2}  
\end{gather*}
Per $t = 0$ non ho termini aggiuntivi e quindi la velocità è diversa da zero ed è positiva
quindi è diretta verso  est(entrante nella lavagna) e quindi:
\begin{gather*}
    y = \frac{1}{3} \Omega g' t^{3} 
\end{gather*} 
Ma il tempo di quanto tocca terra posso ricavarmelo da $x$ ottenendo allora:
\begin{gather*}
    \frac{1}{2}g't^{2}_f = h \\
    t_f = \sqrt{\frac{2h}{g'}}  
\end{gather*}
Sostituendo nella y, si ottiene la y finale e quindi 
(trascurando la curvatura della Terra, che è del tutto ragionevole):
\begin{gather*}
    y_f = \frac{1}{3} \Omega g' \left( \frac{2h}{g'} \right)^{\frac{3}{2}}
\end{gather*}
In linea di principio con una grande altezza c'è uno spostamento apprezzabile
e quindi tutte le approssimazioni che abbiamo fatto sono ragionevoli vista la bassa quota dell'esperimento
e quindi la deviazione è sempre verso oriente come avevamo ipotizzato. 

\section{Il lavoro}
\begin{wrapfigure}{r}{0.4\textwidth}
    \centering
    \label{FUif}
    \caption{ff}
    \begin{tikzpicture}
        \draw(0, 0) -- (0, 1) node[midway, left] {$h$};
        \draw(0, 0) -- (1, 0);
        \draw(0, 1) -- (1, 0);
        \draw(1,0) rectangle (1.5, 0.5);
        \draw[dashed](2, 0) -- (5, 0);
        \draw(2, 0) -- (2, 1) node[midway, left] {$h$};
        \draw(2, 1) -- (5, 0) node[midway, below] {$L$};
    \end{tikzpicture}    
\end{wrapfigure}
Se avessi un grande oggetto da spostare molto massiccio,
dovrei avere una forza tale che possa equilibrare e vincere la forza peso 
dell'oggetto per poterlo sollevare, oppure vincere la resistenza della forza di
attrito statica per poterlo muovere orizzontalmente.
Come hanno quindi fatto gli egiziani a portare i blocchi delle piramidi sopra le piramidi stesse
per la loro costruzione? Hanno creato delle impalcature
che permettessero di far strisciare i blocchi su di un piano inclinato con
angolo molto piccolo. Come misuro quindi lo "sforzo" necessario per compiere
questo spostamento? La grandezza che abbiamo bisogno per
poter spingere il blocco fino alla fine è proprio il lavoro. Si definisce
il lavoro come la forza per lo spostamento; il lavoro è sempre di una forza
e si indica con il simbolo $\delta L$, ossia il lavoro infinitesimo definito
come il prodotto scalare
\begin{align}
    \delta L = \vec{F} \cdot  \vec{\delta S} 
\end{align}
E' quindi lo spostamento della forza lungo un certo cammino, ma il suo
valore non cambia poiché dipende solo dagli istanti iniziali e finali,
si definisce quindi il lavoro lungo un certo cammino. Per trovare il
valore devo fare l'integrale di linea, ossia l'integrale della traiettoria
con il limite che mi fa tendero lo spostamento a $d x$. Per ognuno di questi
divido il mio percorso in tanti pezzi per cui si ottiene.
\begin{align}
    L_{AB, \Gamma} = \int_{A}^{B} \vec{F} \cdot  \vec{\delta S} \ ds  
\end{align}

\end{document}
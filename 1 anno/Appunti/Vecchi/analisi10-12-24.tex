\documentclass[a4paper, oneside]{article}
\usepackage{graphicx}
\usepackage{amsthm}
\usepackage{amsmath}
\usepackage[a4paper,
            bindingoffset=0.2in,
            left=2cm,
            right=2cm,
            top=2cm,
            bottom=2cm,
            footskip=.25in]{geometry}
\usepackage[italian]{babel}
\usepackage{pgfplots}
\usepackage{tabularx}
\usepackage{wrapfig}
\graphicspath{ {./images/} }
\usetikzlibrary{datavisualization}
\usetikzlibrary{datavisualization.formats.functions}
\pgfplotsset{width=10cm,compat=1.9}

\title{Analisi}
\author{Tommaso Miliani}
\date{10-12-24}

\begin{document}
\theoremstyle{definition}
\theoremstyle{theorem}
\theoremstyle{lemma}

\newtheorem{definition}{Definizione}[section]
\newtheorem{theorem}{Teorema}[section]
\newtheorem{lemma}{Proposizione}[theorem]

\maketitle

\section{Fine degli integrali impropri}
Se voglio definire l'integrale sull'intervallo illimitato
allora prima faccio il limite dell'integrale quando un suo limite
tende a infinito e poi definisco l'integrale improprio.
Esistono vari criteri di convergenza per l'integrale improprio:
l'unico criterio che abbraccia le funzioni con segno variabile è proprio
quello della: 
\begin{theorem}[Criterio della convergenza assoluta]
    Prendendo un numero reale $a \in R$ e sia
    \begin{gather*}
        \int_{a}^{c} f(t) \ dt, c > a;
    \end{gather*}
    Se converge
    \begin{gather*}
        \int_{a}^{\infty } |f(t) | \ dt.  
    \end{gather*} allora è convergente anche
    \begin{gather*}
        \int_{a}^{+\infty} f(t) \ dt. 
    \end{gather*}
\end{theorem}
\begin{definition}
    LA parte positiva e negativa di f è definito nel seguente modo:
    \begin{align}
        f_+ &: [a, \infty) \to R . \\
        f_x(x) = &\left\{\begin{array}{c}
            f(x), \quad x > 0\\
            0, \qquad x \leq 0
        \end{array}\right. \\
        f_- &: (-\infty , a] \to R . \\
        f_x(x) = &\left\{\begin{array}{c}
            -f(x), \quad x < 0 \\
            0, \qquad x \geq 0
        \end{array}\right.
    \end{align}
    Inoltre si hanno le seguenti relazioni con la funzione originaria:
    \begin{align}
        f(x) = f_+(x) - f_-(x). \\
        |f(x)| = f_+(x) + f_-(x).
    \end{align}
\end{definition}

\begin{proof}
    Inoltre per il teorema del confronto si ha che $|f(x)|$ ha come integrale
impoprio e per le disuguaglianze $f_+(x) \leq |f(x) \leq f_-(x)$. Allora siccome l'integrale del valore
assoluto è convergente anche $f_+(x)$ e $f_-(x)$ sono convergenti. Per definizione
esitono finiti:
\begin{gather*}
    \lim_{c \to \infty }  \int_{a}^{c} f_+(x) \ dx \\
    \lim_{c \to \infty } \int_{a}^{c} f_-(x) \ dx.
\end{gather*}
E quindi è finito pure:
\begin{gather*}
    \lim_{c \to \infty }  \int_{a}^{c} f_+(x) \ dx - \int_{a}^{c} f_-(x) \ dx.
\end{gather*}
COn la proprietà della linearità degli integrali allora si dimostra che:
esiste finito
    \begin{gather*}
        \lim_{c \to \infty } \int_{a}^{c} f(x) \  dx 
    \end{gather*}
\end{proof}

Un esempio è 
\begin{gather*}
    \int_{1}^{+\infty } \frac{sin(x)}{x^{2} } \ dx. 
\end{gather*}
Passando al limite indefinito si ottiene:
\begin{gather*}
    \int_{1}^{+\infty } \left| \frac{sin(x)}{x^{2} } \right| \ dx
\end{gather*}
So che questo tipo di integrali convergono se l'esponente del denominatore
è $> 1$. Allora posto:
\begin{gather*}
    \left| \frac{sin(x)}{x^{2} } \right| \leq \frac{1}{x^{2} } 
\end{gather*}
PEr il criterio del confronto converge questo integrale per il criterio
del confronto converge quello di partenza. 
Un'altro esempio:
\begin{gather*}
    \int_{1}^{+\infty } \frac{sin(x)}{x} \ dx 
\end{gather*}
Passo al valore assoluto ed ottenere la convergenza assoluta:
\begin{gather*}
    \int_{1}^{+\infty } \frac{sin(x)}{x} \ dx \\
    \left| \frac{sin(x)}{x} \right| \leq \frac{1}{x}  
\end{gather*}
Dal momento che la seconda diverge allora anche la prima deve divergere
in quanto il criterio del confronto non riesce a darmi informazioni
sulla convergenza del primo. Allora intervengo integrando per parti:
Posto quindi 
\begin{gather*}
    f'(x) = sin(x), \quad g(x) = \frac{1}{x} \\
    f(x)  = -cos(x), \quad g'(x) = -\frac{1}{x^{2} } \\
    f(x)g(x) - \int f(x)g'(x) \ dx: \\
    \frac{-cos(x)}{x} |^{c}_{1} + \int_{1}^{c} -\frac{cos(x)}{x^{2}} \ dx. \\
    \Rightarrow  \ \exists \ finito \ \int_{1}^{+\infty } \frac{sin(x)}{x} \ dx
\end{gather*}

\begin{lemma}
    La convergenza non implica la convergenza assoluta.
\end{lemma}

Cosa accade se una funzione non è limitata all'interno di $(a, b]$?
Il suo integrale non è definibile poiché c'è un asintoto in a.
\begin{theorem}[Integrali impropri in intervalli finiti]
    Supponiamo che sia integrabile in senso standard in $[c, b], \forall c > a$:
    allora si applica un'operazione di limite:
    \begin{gather*}
        \lim_{c \to a} \int_{c}^{b} f(x) \ dx 
    \end{gather*} 
    Se esiste finito allora f è integrabile in senso improprio in a, b
    con integrale convergente. Se invece esiste e fa $+ \infty$ allora f
    è integrabile in $[a, b]$ come integrale divergente. Se questo
    integrale non dovesse esistere allora f non è integrabile
    nemmeno in senso improprio. Le funzioni nella forma $\frac{1}{x^{p} }$ hanno
    come integrale tra 0 e 1 che è opposto rispetto al loro comportamento che
    hanno quando si fa l'integrale improprio fino a $+ \infty$. DIverge
    se $p \geq 1$ e converge altrimenti.
\end{theorem}

Si esistendono i teoremi degli integrali impropri, la convergenza
assoluta e gli altri anche a questo tipo di integrali impropri. Tuttavia
gli integrali devono essere fatti chiusi da una parte e aperti dall'altra.

\section{Serie numeriche}
A cosa servono? SI estende l'operazione di sommatoria a infiniti
addendi ma sono numerabili e ciascuno dei quali ha un indice ad infiniti addendi.
GLi insiemi infiniti sono molto diversi tra loro come
$N, Z, Q, R$ e sono tutti infiniti ma solo N, Z, Q sono numerabili poiché
hanno corrispondenza biunivoca.
\begin{gather*}
    \sum_{i = 1}^{n} a_n = ? 
\end{gather*}
Posto $a_n = 1$ ossia la successione costante di 1 che $= +\infty $.
Cosa succede se faccio
\begin{gather*}
    \sum_{n = 1}^{+\infty } \frac{1}{n} 
\end{gather*}
QUesta somma fa $+ \infty $anche se sono tutti termini minori di uno
e far pensare che tutte le volte che si sommano quantità piccole allora
il risultato è infinito, ma 
\begin{gather*}
    \sum_{n = 0}^{+\infty } \frac{1}{10^{n} } = 1,\bar{1}     
\end{gather*}
\begin{gather*}
    \sum_{n = 1}^{+\infty } \frac{1}{n^{2} } < +\infty  
\end{gather*}
Poiché si ha un'analogia con il suo integrale improprio.

\begin{definition}
    Data una successione $a_n$ di numeri reali si indica con
    \begin{align}
        \sum_{n = 1}^{+ \infty } 
    \end{align}
    La serie numerica generata da $a_n$.
\end{definition}
Fissato un numero naturale grande definisco:
\begin{gather*}
    S_n = \sum_{n = 1}^{N}a_n. 
\end{gather*}
Questo numero si chiama somma parziale o ridotta ennesima della serie.
Se esiste finito
\begin{gather*}
    \lim_{N \to +\infty } S_N  = l, \Rightarrow \sum_{n = 1}^{+\infty } a_n   \\
    \lim_{N \to +\infty } S_N = \pm \infty \Rightarrow  \sum_{n = 1}^{+\infty } a_n = \pm \infty \\
    \not \exists \lim_{N \to +\infty } S_N  \Rightarrow \ la \ serie \ e' \ indefinita. 
\end{gather*}



\end{document}
\documentclass[a4paper, oneside]{article}
\usepackage{graphicx}
\usepackage{amsthm}
\usepackage{amsmath}
\usepackage[a4paper,
            bindingoffset=0.2in,
            left=2cm,
            right=2cm,
            top=2cm,
            bottom=2cm,
            footskip=.25in]{geometry}
\usepackage[italian]{babel}
\usepackage{pgfplots}
\usepackage{tabularx}
\usepackage{tikz}
\usepackage{wrapfig}
\usepackage{color}
\definecolor{page}{rgb}{0.129,0.157,0.212}
\pagecolor{page}
\color{white}
\graphicspath{ {./images/} }
\usetikzlibrary{shapes.geometric}
\usetikzlibrary{datavisualization}
\usetikzlibrary{datavisualization.formats.functions}
\pgfplotsset{width=10cm,compat=1.9}

\title{Fisica}
\author{Tommaso Miliani}
\date{25-03-25}

\begin{document}
\theoremstyle{definition}
\theoremstyle{theorem}
\theoremstyle{lemma}

\newtheorem{definition}{Definizione}[section]
\newtheorem{theorem}{Teorema}[section]
\newtheorem{lemma}{Proposizione}[theorem]
\newtheorem{example}{Esempio}[section]

\maketitle

\section{Altri esempi corpo rigido}

\subsection{Asta in rotazione}
\begin{wrapfigure}{r}{0.4\textwidth}
    \centering
    \caption{}
    \begin{tikzpicture}
        
    \end{tikzpicture}    
\end{wrapfigure}



Adesso si verifica che l'equilibrio sia stabile o instabile:
con la derivata seconda si ha:
\begin{gather*}
    Mg \frac{L}{2}\cos\alpha - \frac{2}{3}M\omega^{2}L^{2}\cos^{2}\alpha + \frac{1}{3}M\omega^{2}L^{2}     
\end{gather*}
Calcolata nelle soluzioni, allora si ha che:
\begin{gather*}
    \alpha = 0:\left\{\begin{array}{l}
        > 0 \quad \omega^{2} < \frac{3}{2}\frac{g}{L} \\
        < 0 \quad \omega^{2} > \frac{3}{2}\frac{g}{L}
    \end{array}\right. \\
    \alpha = \pi \text{ sempre instabile}
\end{gather*}
Nel caso limite in cui $\cos\alpha = \frac{3g}{2\omega^{2}L}$ si ha che
la derivata seconda vale esattamente:
\begin{gather*}
    Mg\frac{L}{2}\frac{3g}{2\omega^{2}L} - \frac{2}{3}M\omega^{2}L^{2}\frac{9g^{2}}{4\omega^{4}L^{2} } + \frac{1}{3}M\omega^{2}L^{2}   
\end{gather*}
E allora diventa:
\begin{gather*}
    \frac{3Mg^{2} }{4\omega^{2}} - \frac{2}{3}\frac{Mg^{2} }{\omega^{2} } + \frac{1}{3}M\omega^{2}L^{2} = \frac{1}{3}M\omega^{2}L^{2}\left(1 - \frac{9}{4}\frac{g^{2} }{\omega^{4} L^{2} }\right)  
\end{gather*}
Diventa stabile se e solo se:
\begin{gather*}
    \omega^{2} > \frac{3}{2}\frac{g}{L} 
\end{gather*}

\subsection{La formula fondamentale per i punti del corpo rigido}
Nel corpo rigido ogni corpo si muove con una certa velocità che ha 3 gradi
di libertà per la velocità del centro e 3 gradi di libertà per 
la velocità angolare:
\begin{gather*}
    \vec{v}_P = \vec{v}_O + \omega \times(P - O)  
\end{gather*}
Nel caso di moti traslatori 
\begin{gather*}
    \vec{\omega} = 0, \qquad \vec{v}_P = \vec{v}_O   
\end{gather*}
Il momento angolare allora è dato da:
\begin{gather*}
    \vec{L}_{\Omega} = \sum_{i = 1}^{n}(P_i - \Omega) \times m_i \vec{v}_i = \sum_{i = 1}^{n}(P_i - \Omega) \times m_i \vec{v}_O = (C - \Omega)\times M \vec{v}_O      
\end{gather*}

Nel caso del moto ROTOtraslatorio,
\begin{gather*}
    \vec{v}_O \neq 0, \quad \vec{\omega} \neq 0  
\end{gather*}
Allora il momento angolare sarà:
\begin{gather*}
    \vec{L}_{\Omega} = \sum_{i = 1}^{n}(P_i - \Omega)\times m_i\vec{v}_i = \sum_{i = 1}^{n}m_i(P_i - \Omega)\times \vec{v}_O + \sum_{i = 1}^{n}m_i(P_i - \Omega) \times (\vec{\omega} \times (P_i - O) )     
\end{gather*}
Nel caso in cui il momento angolare sia applicato sull'origine del 
sistema di riferimento e che questo coincida esattamente con il centro
del corpo rigido, allora si semplifica :
\begin{gather*}
    \vec{L}_O = \sum_{i = 1}^{n}m_i(P_I - O)\times (\vec{\omega} \times (P_i - O))  
\end{gather*}
Nel caso in cui $\vec{\omega}$ sia costante, allora essendo il suo versore
costante, il momento si semplifica ulteriormente.

\subsection{IL momento assiale}
\begin{align}
    L_{\omega, O} = \vec{L}_O \cdot  \hat{u}_{\omega} = \sum_{i = 1}^{n}m_i \cdot \hat{u}_{\omega} (P_i - O) \times (\hat{u}_{\omega} \times (P_i - O)) \omega    
\end{align}
Con le regole del prodotto misto si ottiene:
\begin{gather*}
    L_{\omega, O} =  \left(\sum_{i = 1}^{n} m_i \left| \hat{u}_{\omega} \times (P_i - O)  \right|^{2}  \right) \omega
\end{gather*}
Posso allora esprimere la distanza dall'asse di rotazione
di ogni punto come
\begin{align}
    |\hat{u}_{\omega} \times (P_i - O)| = d_i 
\end{align}
Posso allora esprimere il momento di inerzia:
\begin{align}
    I = \sum_{i = 1}^{n}m_id_i^{2}  = \int d^{2} dm 
\end{align}
E quindi il momento di inerzia:
\begin{gather*}
    L_{\vec{\omega} } = I \omega
\end{gather*}
L'equazione cardinale si semplifica allora come:
\begin{align}
    \vec{M}_{\Omega} \cdot  \hat{u}_{\omega} = I_{\hat{\omega} } \dot{\omega}  
\end{align}
diventando il momento assiale delle forze.2

\end{document}
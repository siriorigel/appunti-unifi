\documentclass[a4paper, oneside]{article}
\usepackage{graphicx}
\usepackage{amsthm}
\usepackage{amsmath}
\usepackage[a4paper,
            bindingoffset=0.2in,
            left=2cm,
            right=2cm,
            top=2cm,
            bottom=2cm,
            footskip=.25in]{geometry}
\usepackage[italian]{babel}
\usepackage{pgfplots}
\usepackage{tabularx}
\usepackage{wrapfig}
\graphicspath{ {./images/} }
\usetikzlibrary{datavisualization}
\usetikzlibrary{datavisualization.formats.functions}
\pgfplotsset{width=10cm,compat=1.9}

\title{Fisica}
\author{Tommaso Miliani}
\date{09-12-24}

\begin{document}
\theoremstyle{definition}
\theoremstyle{theorem}
\theoremstyle{lemma}

\newtheorem{definition}{Definizione}[section]
\newtheorem{theorem}{Teorema}[section]
\newtheorem{lemma}{Proposizione}[theorem]

\maketitle

\section{Massa e sua definizione operativa}
\begin{wrapfigure}{r}{0.4\textwidth}
    \centering
    \label{Fig 1.1}
    \caption{Carrellino delle masse}
    \begin{tikzpicture}
        \draw ( 0, 0) -- (4, 0);
        \draw (0.5, 0) rectangle (1, 1);
        \draw (3.5, 0) rectangle (4, 1);
        \draw (1.75, 0.25) rectangle (2.75, 0.75);
        \draw (2, 0.13) circle (0.13);
        \draw (2.5, 0.13) circle (0.13);
        \draw[decoration={aspect=0.3, segment length=1mm, amplitude=1mm,coil},decorate] (1, 0.5) --  (1.75, 0.5);
        \draw[decoration={aspect=0.3, segment length=1mm, amplitude=1mm,coil},decorate] (2.75,0.5) --  (3.5, 0.5);
        \draw (2, 0.75) rectangle (2.5, 1.25) node[midway] {$M$};
    \end{tikzpicture}    
\end{wrapfigure}

Il carrellino inizia ad oscillare se applico una certa forza su di esso e le ruote
servono a smorzare l'attrito con il terreno. Adesso arriviamo ai principi della
dinamica attraverso questo. In cinematica sostanzialmente si sceglieva aribtrariamente
il sistema di riferimento: mentre in dinamica il sistema di riferimento può cambiare
la descrizione del sistema fisico: si definiscono quindi
solo una classe di sistemi di riferimento ossia i sistemi di riferimento
inzerziali. 
\begin{enumerate}
    \item Un oggetto si muove di moto rettilineo uniforme se e solo se non ci sono forze applicate su di esso;
    \item 
\end{enumerate}
Sperimentalmente le forze applicate su di un oggetto sono dovute ad altri corpi:
selezioniamo idelmente degli oggetti lontani da altri: un oggetto molto lontano
da tutti gli altri corpi per evitare proprio gli effetti di tutte le altre forte:
adesso però avendo cambiato il sistema di riferimento può cambiare tutto.
Definisco sistema inzeriale un sistema in cui un oggetto è in quiete o si muove di moto 
rettilineo uniforme. Il primo principio diventa una affermazione dell'esistenza
dei sistemi di riferimento inzerziali:
\begin{gather*}
    \vec{v} = \vec{v} + \vec{v}_o
\end{gather*}
Se adesso un corpo è sottoposto a delle forze, 
il secondo principio ci dice che la somma delle forze  (e quindi la risultante) è
\begin{align}
    \sum_{i = 1}^{n} k\vec{F}_i = M\vec{a}   \\
    \vec{a} \propto \frac{1}{k}\frac{\sum_{i = 1}^{n} \vec{F}_i}{M} 
\end{align}
\begin{wrapfigure}{r}{0.4\textwidth}
    \centering
    \label{Fig 1.2}
    \caption{Piano inclinato}
    \begin{tikzpicture}
        \draw (0, 3) -- (4, 0);
        \draw (0, 3) -- (0.2, 3.3);
        \draw[decoration={aspect=0.3, segment length=1mm, amplitude=1mm,coil},decorate] (0.155, 3.2) --  (0.8, 2.7);
        \draw (0.95, 2.6) circle (0.15) node[right] {M}; 
        \draw (0, 0) -- (4, 0); 
        \draw (3.5, 0.35) arc (140: 180: 0.55) node[midway, left] {$\alpha$};
        \draw[->] (0.95, 2.7) -- (0.95, 3.7) node[at end, right] {$\vec{N} $};
        \draw[->] (1.1, 2.55) -- (2.05, 1.8) node[at end, above] {$\vec{P}_{//}$};
        \draw[->] (0.85, 2.5) -- (0, 1.5) node[at end, below] {$\vec{P}_{\perp}$}; 
    \end{tikzpicture}    
\end{wrapfigure}
Se la massa è definita come il periodo del carrellino sopra oppure
definita come l'allungamento della molla, allora ho bisogno di un k, una
costante che mi permetta di specificare il rapporto diretto tra la massa e
la sua relativa accelerazione. Arrivo a dire però che $k = 1$ se definisco
individualmente la forza, la massa e l'accelerazione allora si possono scegliere
le unità di misura in modo più semplice possibile ed esprimerlo come uno. 
\begin{align}
    \sum_{i = 1}^{n}\vec{F}_i = M \vec{a}   
\end{align}
Su di un piano inclinato per esempio io lascio cadere un oggetto (il quale
può cadere in modo diverso date forme diverse o che l'aria ha effetti differenti
dal vuoto).
\begin{wrapfigure}{r}{0.4\textwidth}
    \centering
    \label{Fig 1.3}
    \caption{Sistema di riferimento con carrucola}
    \begin{tikzpicture}
        \draw (0, 0) -- (3, 0);
        \draw (3, 0) -- (3, -2);
        \draw (1, 0) rectangle (2, 1) node[midway] {M};
        \draw[->] (1.5, 0) -- (1.5, -1) node[at end, right] {$\vec{P}$};
        \draw (3, 0) -- (3.25, 0.25);
        \draw (3.37, 0.37) circle (0.13);
        \draw (2, 0.5) -- (3.4, 0.5);
        \draw (3.5, 0.4) -- (3.5, -1);
        \draw[->,  very thick] (3.5, -1) -- (3.5, -2) node[at end, right] {$\vec{F}$};
    \end{tikzpicture}     
\end{wrapfigure}


\end{document}
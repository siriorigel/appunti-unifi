\documentclass[a4paper, oneside]{article}
\usepackage{graphicx}
\usepackage{amsthm}
\usepackage{amsmath}
\usepackage[a4paper,
            bindingoffset=0.2in,
            left=2cm,
            right=2cm,
            top=2cm,
            bottom=2cm,
            footskip=.25in]{geometry}
\usepackage[italian]{babel}
\usepackage{pgfplots}
\usepackage{tabularx}
\usepackage{wrapfig}
\graphicspath{ {./images/} }
\usetikzlibrary{datavisualization}
\usetikzlibrary{datavisualization.formats.functions}
\pgfplotsset{width=10cm,compat=1.9}

\title{FIsica}
\author{Tommaso Miliani}
\date{16-12-24}

\begin{document}
\theoremstyle{definition}
\theoremstyle{theorem}
\theoremstyle{lemma}

\newtheorem{definition}{Definizione}[section]
\newtheorem{theorem}{Teorema}[section]
\newtheorem{lemma}{Proposizione}[theorem]

\maketitle

\section{Dinamica di Fletcher}
\begin{wrapfigure}{r}{0.4\textwidth}
    \centering
    \label{Fig 1.1}
    \caption{Dinamica di due corpi ed una carrucola.}
    \begin{tikzpicture}
        \draw (-1, 0) --(2, 0);
        \draw (2, 0) -- (2, -2);
        \draw (2, 0) -- (2.2, 0.2);
        \draw (2.35, 0.35) circle (0.15);
        \draw (0, 0) rectangle (1, 1) node[midway] {$M_1$};
        \draw (1, 0.5) -- (2.4, 0.5);
        \draw (2.5, 0.4) -- (2.5, -1.5);
        \draw (2.2, -1.5) rectangle (2.8, -2.1) node[midway] {$M_2$};
        \draw[->] (-0.5, 0) -- (-0.5, 0.5) node[at end, left] {$y$};
        \draw[->] (-0.5, 0) -- (0, 0) node[at end, below] {$x$};
        \draw[->] (2.2, -0.2) -- (2.7, -0.2) node[at end, right] {$y$}; 
        \draw[->] (2.2, -0.2) -- (2.2, -0.7) node[at end, below] {$x$};
    \end{tikzpicture}    
\end{wrapfigure}
Per lo studio di questo sistema prima di tutto si individua un sistema di riferimento 
inerziale: in questo caso si centra le assi del nostro sistema di riferimento
nel primo oggetto. Così si può approssimare il sistema di riferimento come se fosse
inerziale e solidale con la Terra entro certe approssimazioni. Disegnamo ora lo schema delle forze che agiscono sui corpi.
\begin{tikzpicture}
    \draw (0, 0) -- (2, 0);
    \draw (0.5, 0) rectangle (1.5, 1) node[midway] {$M_1$};
    \draw[->] (1, 1) -- (1, 2) node[at end, left] {$\vec{N} $};
    \draw[->] (1.5, 0.5) -- (2.5, 0.5) node[at end, above] {$\vec{T} $};
    \draw[->] (1, 0) -- (1, -1) node[midway, right] {$m_2 \vec{g}$};
    \draw (5, 0) rectangle (6, 1) node[midway] {$M_2$};
    \draw[->] (5.5, 1) -- (5.5, 2) node[at end, right] {$\vec{T}$};
    \draw[->] (5.5, 0) -- (5.5,  -1) node[midway, right] {$m_2 \vec{g} $};
\end{tikzpicture} \\
A questo punto possiamo impostare il sistema con tutte le forze e si ottiene per il primo
corpo (considerato che non si ha attrito ed il suo sistema di riferimento considerato
prima del blocco stesso):
\begin{gather*}
    \left\{ \begin{array}{c}
        \vec{N} = N\vec{j} \\
        m_1 \vec{g} = -m_1 \vec{j} \\
        \vec{T} = \vec{T} \vec{i}       
    \end{array}\right. \\
    \vec{N} + m_1 \vec{g} + \vec{T} =m_1 \vec{a}_1 \\
    \left\{\begin{array}{c}
        (y) \quad N - m_1 g = 0 \\
        (x) \quad T = m_1 \ddot{x_1}
    \end{array}\right.    \\
    N = m_1 g \\
    T =  \ddot{x_1}
\end{gather*}
Adesso si scrive lo schema delle forze per quanto riguarda il secondo
corpo con il sistema di riferimento sotto la carrucola:
\begin{gather*}
    \left\{\begin{array}{c}
        \vec{T}' = -T'\hat{i} \\
        m_2 \vec{g} = m_2 g \hat{i}    
    \end{array}\right.\\
    \vec{T}' + m_2 \vec{g} = m_2 \vec{a}_2 \\
    -T' + m_2g = m_2 \ddot{x}_2    
\end{gather*}
A questo punto si fanno delle assunzioni per risolvere il problema: prima di tutto
il filo è ideale (non si estende né si deforma); inoltre ho considerato
le tensioni come se fossero uguali e da questo si ottiene che:
\begin{gather*}
    \ddot{x_1} = \ddot{x_2} \\
    |\vec{T}| = |\vec{T}'|\\
    T = T'  
\end{gather*}
Imposto ora il sistema tra le tensioni ottenendo:
\begin{gather*}
    \left\{ \begin{array}{c}
        T = m_2 \ddot{x_1} \\
        -T + m_2 g = m_2 \ddot{x_1}
    \end{array}\right.
\end{gather*}
Sommando si ottiene:
\begin{gather*}
    m_2 g = (m_1 + m_2) \ddot{x_1} \\
\end{gather*}
\begin{align}
    \ddot{x_1} = \frac{m_2 g}{m_1 + m_2}
\end{align}
Questo ci torna poiché l'inerzia è la somma delle masse e la forza
che agisce su questo oggetto è solo la forza peso del secondo oggetto.
Se ora tolgo $m_1$ allora $m_2$ è in caduta libera. Altrimenti se non
ci fosse $m_2$ allora $\ddot{x_1} = 0$ Si ottiene allora la tensione
moltiplicando per $m_1$:
\begin{align}
    \boxed{T = m_1 \ddot{x_1} = \frac{m_1 m_2 g}{m_1 + m_2}}
\end{align}
Adesso integriamo quella di prima:
\begin{gather*}
    \dot{x}_1 = \frac{m_2 g}{m_1 + m_2} t + a \\
    x_1 = \frac{m_2g}{2(m_1 + m_2)}t^{2} + at + b  
\end{gather*}
La posizione fisicamente è descritta con queste formule ma sono io che
decidon dove mettere l'origine degli assi cartesiani e se io decido che
l'origine di essi è dove si trovava $m_1$ al tempo $ t = 0$ allora $b = 0$
altrimenti b diventa la distanza dall'origine della massa al tempo $t = 0$.

\section{Il filo ideale (NON COMPLETO)}
Il concetto di filo ideale in fisica presuppone che: applicata una forza
da entrambe le parti del filo di pari intensità allora non si ha alcuna deformazione.
\begin{wrapfigure}{r}{0.4\textwidth}
    \centering
    \label{Fig 2.1}
    \caption{Filo ideale}
    \begin{tikzpicture}
        \draw[<-] (0, 0) -- (1, 0) node[midway, above] {$\vec{T}$};
        \draw[-, ultra thick] (1, 0) -- (3, 0);
        \draw[->] (3, 0) -- (4, 0) node[midway, above] {$\vec{T}'$};
    \end{tikzpicture}    
\end{wrapfigure}
Nel caso del filo da solo allora, assumendo:
\begin{gather*}
    m_{filo} \sim 0 \qquad \vec{T}  + \vec{T}' = m_{filo} \vec{a} = 0  
\end{gather*}
Nel caso della dinamica di Fletcher invece si ha che lo stesso filo ideale e
le stesse tensioni hanno però somma diversa da zero in quanto la carrucola riesce
in qualche modo ad "agire" sul filo con una forza diversa da zero
\begin{tikzpicture}
    \draw (0, 4) -- (0, 0);
    \draw (0, 0) -- (4, 0);
    \draw (0, 4) arc (90: 0: 4);
    \draw (0, 0) -- (3.4, 2);
    \draw (3.4, 2.2) arc (45: 0: 0.5);
    \draw[->] (3.4, 2) -- (4, 2.45) node[at end, right] {$d\vec{N} $};
    \draw[->] (3.4, 2.2) -- (3.1, 2.6) node[at end, right] {$\vec{T}_{\phi} $};
\end{tikzpicture}
DAl punto di vista fisico il sistema si esprime quindi con le seguenti:
\begin{gather*}
    d\vec{N} + \vec{T}_{\phi} + \vec{T}'_{\phi} = m_{filo}\vec{ a} \sim 0    
    \left\{\begin{array}{c}
        dN - T_{\phi} \sin\left(\frac{d\phi}{2}\right) - T_{\phi}
    \end{array}\right. 
\end{gather*}


\section{Macchina di Atwood}
La macchina di Atwood è forse una delle macchine più importanti della fisica
classica poiché è riuscita a dare una spiegazione al secondo principio della
dinamica. Se siamo nel piano si hanno 2 gradi di libertà e complessivamente se ne
hanno quattro in quanto si ha anche la possibilità di oscillare. Il filo
in questo caso si appoggia per metà giro della carrucola. Attraverso un'accorta
preparazione dell'esperimento si possono ridurre a due eliminando i gradi di libertà.
Utilizziamo inoltre un filo idelale e quindi le posizioni delle masse sono legate in 
modo tale che il filo possa scendere dalla parte della massa più grande riducendo
il grado di libertà a 1.
\begin{wrapfigure}{r}{0.4\textwidth}
    \centering
    \label{Fig 3.1}
    \caption{Macchina di Atwood}
    \begin{tikzpicture}
        \draw (0, 0) circle (1);
        \draw (-1, 0) -- (-1, -1.5);
        \draw (1, 0) -- (1, -2.5);
        \draw[->] (0, 0) -- (0, 0.5) node[at end, left] {$y$};
        \draw[->] (0, 0) -- (0.5, 0) node[at end, below] {$x$};
        \draw (-1.25, -1.5) rectangle (-0.75, -2) node[midway] {$M_1$};
        \draw (0.75, -2.5) rectangle (1.25, -3) node[midway] {$M_2$};
        \draw[->] (-1, -1.5) -- (-1, - 1) node[at end, left] {$\vec{T}$};
        \draw[->] (-1, -2) -- (-1, -2.5) node[at end, right] {$m_1\vec{g}$};
        \draw[->] (1, -2.5) -- (1, -2) node[at end, left] {$\vec{T}'$};
        \draw[->] (1, -3) -- (1, -3.5) node[at end, right] {$m_2\vec{g}$};
    \end{tikzpicture}    
\end{wrapfigure}
\begin{gather*}
    1: \quad \vec{T} + m_1 \vec{g} = m_2 \vec{a}_1 \\
    2: \quad \vec{T}' + m_2 \vec{g} = m_1 \vec{a}_2      \\
    \left\{\begin{array}{c}
        T - m_2 g = m_2 \ddot{y}_1 \\
        T' - m_1g = m_1 \ddot{y}_2 
    \end{array}\right. \\
    Posto \ T = T' \ \wedge \ \ddot{y}_2 = -\ddot{y}_1 
\end{gather*}
\begin{align}
    \boxed{\ddot{y_1} = \frac{(m_2 - m_1)g}{m_1 + m_2}}
\end{align}
Posto $m_2 >> m_1$ allora $\ddot{y}_1 = g$. Se invece $m_1 >> m_2$ allora
$\ddot{y}_1 = -g$. Troviamo ora la tensione:
\begin{gather*}
    T = m_1 g + m_1 \ddot{y}_1 \Rightarrow 
\end{gather*}
\begin{align}
    \boxed{T =\frac{2m_1m_2}{m_1 + m_2}g}
\end{align}

\section{La molla ideale}
\begin{wrapfigure}{r}{0.4\textwidth}
    \centering
    \label{Fig 4.1}
    \caption{La molla ideale}
    \begin{tikzpicture}
        \draw (0, 0) -- (0.5, 0);
        \draw (0.5, 0) -- (0.5, -1);
        \draw[decoration={aspect=0.3, segment length=1.2mm, amplitude=1mm,coil},decorate,opacity=0.9] (0.5, -0.5) -- (1.5,-0.5);
        \draw (0.5, -1) -- (2.5, -1);
        \draw (1.5, -1) rectangle (2.5, 0) node[midway] {$M$};
    \end{tikzpicture}    
\end{wrapfigure}
Una molla ha una posizione di riposo e qundo essa si trova in quella posizione allora
non esercita forze. Dallo schema la forza della molla è come segue:
\begin{align}
    \vec{F} = -k(x - l_r)\hat{i}  
\end{align}
La Molla tende ad esercitare una forza che è la fa tornare sempre
alla posizione di riposo. Lo schema delle forze del sistema si rappresenta
come segue, ponendo il sistema di riferimento inerziale proprio nel punto in cui la molla
è a riposo. Dal punto di vista delle forze quindi:
\begin{gather*}
    \vec{F}_d = -kx\hat{i} \\
    -kx = m \ddot{x} \\
    \boxed{m\ddot{x} + kx = 0}   
\end{gather*}
\begin{center}
    \begin{tikzpicture}
    \draw (0, 0) -- (0.5, 0);
    \draw (0.5, 0) -- (0.5, -1);
    \draw[decoration={aspect=0.3, segment length=4.2mm, amplitude=3mm,coil},decorate,opacity=0.9] (0.5, -0.5) -- (2.5,-0.5);
    \draw (2.5, -1) rectangle (3.25, -0.25) node[midway] {$M$};
    \draw (0.5, -1) -- (4, -1);
    \draw[<->] (0.75, -1.25) -- (1.75, -1.25) node[midway, below] {$l_r$};
    \draw[->] (2.5, -0.6) -- (2, -0.6) node[at end, below] {$\vec{F}_d$};
    \draw[->] (2.85, -0.25) -- (2.85, 0.25) node[at end, right] {$\vec{N} $};
    \draw[->] (2.85, -1) -- (2.85, -1.5) node[at end, right] {$m \vec{g}$};
\end{tikzpicture} 
\end{center}
Ossia l'equazione del moto armonico. Questa equazione differenziale del secondo
ordine omogenea non è integreabile facilmente, comunque si ottiene: 

\begin{gather*}
    \ddot{x} + \frac{k}{m}x = 0, \ posto \ \omega^{2} = \frac{k}{m} \\
    \ddot{x} + w^{2}x = 0   
\end{gather*}
Quindi diventa:
\begin{gather*}
    x = A \cos(\omega t + \phi) \\
    \dot{x} = -A\sin(\omega t + \phi)\omega \\
    \ddot{x} = -A \cos(wt + \phi)\omega^{2} \\
    -A \cos(\omega t + \phi)\omega^{2} + \omega^{2} A \cos(\omega t + \phi) = 0
\end{gather*}

\end{document}
\documentclass[a4paper, oneside]{article}
\usepackage{graphicx}
\usepackage{amsthm}
\usepackage{amsmath}
\usepackage[a4paper,
            bindingoffset=0.2in,
            left=2cm,
            right=2cm,
            top=2cm,
            bottom=2cm,
            footskip=.25in]{geometry}
\usepackage[italian]{babel}
\usepackage{pgfplots}
\usepackage{tabularx}
\usepackage{wrapfig}
\graphicspath{ {./images/} }
\usetikzlibrary{datavisualization}
\usetikzlibrary{datavisualization.formats.functions}
\pgfplotsset{width=10cm,compat=1.9}

\title{Fisica (Cuccoli)}
\author{Tommaso Miliani}
\date{18-02-25}

\begin{document}
\theoremstyle{definition}
\theoremstyle{theorem}
\theoremstyle{lemma}

\newtheorem{definition}{Definizione}[section]
\newtheorem{theorem}{Teorema}[section]
\newtheorem{lemma}{Proposizione}[theorem]

\maketitle

\section{Moto di un oggetto con forza di tipo elastico su un piano inclinato}
\begin{wrapfigure}{r}{0.4\textwidth}
    \centering
    \label{Fig 1.1}
    \caption{Binario con molla ed oggeto}
    \begin{tikzpicture}
        \draw (-1, 0) -- (4, 0);
        \draw[decoration={aspect=0.3, segment length=4mm, amplitude=1mm,coil},decorate] (-1, 0.5) -- (1, 0.5);
        \draw (-1, 0) -- (-1, 1);
        \draw (1, 0) rectangle (2, 1); 
        \draw[->] (1.5, 0.5) -- (1.5, -0.5) node[at end, right] {$m\vec{g} $}; 
        \draw[->] (1.5, 0.5) -- (1.5, 1.5) node[at end, right] {$\vec{N} $};
        \draw[->] (1, 0.5) -- (0.5, 0.5) node[midway, above] {$\vec{F}_e$};
    \end{tikzpicture}    
\end{wrapfigure}
Nel caso di una molla ideale, questa obbedisce alla legge di Hooke: 
alla sua estremità essa applica una forza elastica data da:
\begin{align}
    \vec{F}_e = K\Delta l \hat{u}  
\end{align}
PErché una molla possa applicare una forza ad un oggetto allora occorre
che sia fissata. La molla considerata è inoltre perfettamente simmetrica:
essa applica la stessa forza sia se vengono tirate che compresse. La legge di 
Hooke vale se e solo se $\Delta l \to 0$, altrimenti una molla non ideale 
perde le sue caratteristiche e si deforma.  Un'altra caratteristica di una molla
ideale è la sua assenza di massa. \\
La situazione nel disegno si possono scrivere le seguenti:
\begin{align}
    \vec{N} + m\vec{g} + \vec{F}_e = m\vec{a}    
\end{align}
LA sitauzione vettoriale:
\begin{gather*}
    \vec{F}_e = -K\Delta l \hat{i} \\
    \vec{a} = \ddot{x} \hat{i} \\
    \vec{N} = N\hat{j} \\
    \vec{g} = -g\hat{j}      
\end{gather*}
SI ottiene allora combinando tutto insieme:
\begin{gather*}
    -K\Delta l = m\ddot{x} \\
    N  = mg
\end{gather*}
Fissato l'origine del sistema di riferimento nel punto in cui la
molla è a riposo (a noi non interessa chi ha spostato la massa nella posizione
in cui la molla non è più a riposo), allora si ottiene:
\begin{align}
    m\ddot{x} &= -K x \\
    m \ddot{x} + K x &= 0 
\end{align}
Questa equazione differenziale è risolvibile facilmente ed ha le seguenti caratteristiche:
è lineare, omogenea, del secondo ordine e a coefficienti costanti. \\
L'ordine di un equazione differenziale è dato dall'ordine massimo della derivata che
compare nell'equazione, la linearità è data dalla potenza dell'incognita x. E' omogenea
poiché non compare il termine noto, inoltre è a coefficienti costanti in quanto
i coefficienti non sono variabili rispetto al tempo. \\
Dato che e entrambi i coefficienti sono positivi, allora possiamo dividere tutto per $m$ e
ottenere:
\begin{gather*}
    \ddot{x} + \frac{K}{m} x = 0
\end{gather*}
Posto allora il nuovo coefficiente uguale al quadrato di un numero reale
(per ricordarci che è un numero positivo), allora si ottiene:
\begin{gather*}
    \ddot{x} + \Omega^{2} x = 0 \\
    \ddot{x} = -\Omega^{2}x 
\end{gather*}
Le possibili soluzioni sono:
\begin{align}
    x_1(t) = \cos\Omega t \\
    x_2(t) = \sin\Omega t
\end{align}


\section{Il moto armonico }
Poiché derivandole due volte si ottiene esattamente il secondo membro
della differenziale. La soluzione generale si ottiene combinando le due ottenendo
una combinazione lineare di seno e coseno:
\begin{align}
    x(t) =& a\cos\Omega t + b \sin\Omega t
\end{align}
L'equazione differenziale è proprio l'equazione che descrive un moto
armonico: il requisito essenziale è la derivata seconda e la funzione x con
lo stesso segno nello stesso membro.  Due modi equivalenti per la soluzione
generale sono anche:
\begin{align}
    x(t) &= A \cos(\Omega t + \phi) \\
    x(t) &= A \sin(\Omega t + \phi)
\end{align}
Essendo tutte e tre equivalenti, le tre equazioni possono essere utilizzate 
tutte al fine del risolvimento degli esercizi senza distinzione.
Se applichiamo la regola della somma e sottrazione degli argomenti 
di seno e coseno, allora si riottiene la forma estesa dell'equazione 
se si ottiene che:
\begin{align}
    a =& A \cos\phi \\
    b =& -A \sin\phi
\end{align}
E quindi:
\begin{align}
    \tan\phi =& -\frac{b}{a} \\
    A =& \sqrt{a^{2} + b^{2} } 
\end{align}
Le costanti che compaiono prendono quindi i seguenti nomi:
$\Omega$ è chiamata \textbf{pulsazione} del moto armonico che
ha le dimensioni di $t^{-1}$ ed è univocamente determinata dalla fisica
del problema e quindi anche $a, b$, dipendendo dalle condizioni iniziali, cambiano. 
Il periodo del moto armonico è dato proprio da:
\begin{align}
    T =& \frac{2\pi}{\Omega}
\end{align}
E quindi la sua frequenza per definizione non è nient'altro che:
\begin{align}
    \nu = \frac{\Omega}{2\pi}
\end{align}
Derivando l'equzione del moto armonico si ottiene:
\begin{gather*}
    \dot{x}(t) = -a\Omega \sin\Omega t + b\Omega \cos\Omega t \\
    \Rightarrow  \dot{x}(t = 0) = b \Omega = v_0   
\end{gather*}
La velocità in $t = 0$ è proprio data dalla derivata e quindi si può ricavare
\begin{gather*}
    b = \frac{v_0}{\Omega}
\end{gather*}
Tutte queste considerazioni sono possibili solo grazie al fatto che ho posto
il sistema di riferimento della sistema proprio nel punto in cui la
molla è a riposo.


\section{Il caso specifico: una molla attaccata al soffitto}
\begin{wrapfigure}{r}{0.4\textwidth}
    \centering
    \label{FIg 3.1}
    \caption{Sistema con molla attaccata al soffitto}
    \begin{tikzpicture}
        \draw(0, 0) -- (3, 0);
        \draw[decoration={aspect=0.3, segment length=4mm, amplitude=1mm,coil},decorate] (1.5, 0) -- (1.5, -2);
        \draw (1, -2) rectangle (2, -3);
        \draw[->, very thick] (1.5, -2) -- (1.5, -1) node[midway, right] {$\vec{F}_e$};
        \draw[->] (1.5, -3) -- (1.5, -4) node[at end, right] {$m\vec{g} $};
        \draw[decoration={aspect=0.3, segment length=1mm, amplitude=1mm,coil},decorate] (1, 0) -- (1, -1);
        \draw[|-|] (0.5, -1) -- (0.5, -2) node[midway, right] {$\Delta l$};
    \end{tikzpicture}    
\end{wrapfigure}
Nella seguente immagine non ho ancora definito l'origine del sistema di riferimento 
ma posso già definire i vettori con gli opportuni versori ottenendo la seguente situazione:
\begin{gather*}
    \vec{F}_e = K \Delta l \hat{j} \\
    \vec{g} = -g\hat{j} \\
    \vec{a} = \ddot{y} \hat{j} \\
    K\Delta l = - mg = m \ddot{y}      
\end{gather*}
Adesso si pone l'origine del sistema di riferimento nel punto in cui
la molla è a riposo. Se chiamiamo allora $\Delta l = -y$ si ottiene
l'equzione del moto armonico non omogeneo:
\begin{gather*}
    -K y -mg = m\ddot{y} \\
    my + Ky = -mg
\end{gather*}
Per risolvere questa equazione differenziale posso operare attraverso una
sostituzione di variabile nella seguente maniera:
\begin{gather*}
    y(t) = \zeta (t) + \hat{y} \Rightarrow \ddot{y} (t) = \ddot{\zeta}(t)  \\
    m\ddot{\zeta} + K\zeta + K \hat{y} = -mg \\
    \text{Posto} \ \hat{z} = -\frac{mg}{K} \\
    m \ddot{\zeta} + K \zeta = 0   
\end{gather*}
Quando $\zeta$ è zero, allora si ritorna alla situazione in cui l'origine del sistema
di riferimento è proprio la posizione di equilibrio. 
\begin{align}
    y(t) = A\cos(\Omega t + \phi)- \frac{mg}{K}
\end{align}
Derivando si ottiene :
\begin{align}
    \dot{y} (t) = -\Omega A \sin(\Omega t + \phi) 
\end{align}
E quindi si hanno le impostazioni delle soluzioni:
\begin{align}
    \left\{\begin{array}{c}
        A \cos\phi = \frac{mg}{K} \\
        A \sin\phi = 0
    \end{array}\right. \Rightarrow \left\{\begin{array}{c}
        A = \frac{mg}{K} \\
        \phi = 0
    \end{array}\right.
\end{align}
Poiché $A = 0$ non ci da alcuna informazione sul moto .
\begin{gather*}
    y(t) = \frac{mg}{K} \left( \cos(\Omega t)  - 1\right)
\end{gather*}
\begin{align}
    y_{max} &= -2\frac{mg}{K} \\
    \dot{y}_{max} &= \frac{mg}{K} \Omega = g \sqrt{\frac{m}{g}}  
\end{align}
Dopo quanto  tempo si raggiunge la quota massima:
\begin{gather*}
    \Omega t_{min} = \pi \\
    t_{min} = \frac{\pi}{\Omega} = \frac{T}{2}
\end{gather*}
Quando si raggiunge il massimo modulo per la prima volta:
\begin{gather*}
    t_{max} = \frac{\pi}{2\Omega} = \frac{\text{Periodo}}{4} 
\end{gather*}


\section{Moto del pendolo semplice (o pendolo matematico)}
\begin{wrapfigure}{r}{0.4\textwidth}
    \centering
    \label{FIg 2.2}
    \caption{Un pendolo semplice}
    \begin{tikzpicture}
        \draw(0, 0) -- (1, -2);
        \filldraw[gray] (1, -2) circle (3pt);
        \draw[dashed] (0, 0) -- (0, -2.5);
        \draw[dashed](1, -2) arc (-60: -90: 2);
        \draw[->](1, -2) -- (1, -3) node[at end, below] {$m\vec{g}$ };
        \draw[->] (1,-2) -- (1.5, -1.75) node[at end, above] {$\hat{u}_t$};
        \draw[dashed] (0.5, -1) arc (-60:-90:1) node[midway, below] {$\theta$};
        \draw[->, thick](1, -2) -- (0.5, -1) node[at end, above] {$\vec{N} $};
        \draw[->] (1, -2) -- (1.5, -3) node[at end, right] {$\vec{g}_r$};
        \draw[->] (1, -2) -- (0.5, -2.25) node[at end, below] {$\vec{g}_t$}; 
        \filldraw (0, - 2.25) circle (1pt) node[anchor = north] {$A$};
    \end{tikzpicture}    
\end{wrapfigure}
Il pendolo semplice è in opposizione al pendolo fisico o pendolo composto. Cos'è un pendolo?
Un pendolo è costruito con un filo ed una massa sufficientemnete piccola
da poter essere approssimata as un punto materiale. Dal punto di vista matamatico
si comporta qualunque punto materiale su di una traiettoria circolare su di un piano
verticale come un pendolo.  Questo accadrà se e solo se il filo è completamente teso e
se il punto materiale è vincolato in qualche modo in moto tale da poter solo
ruotare e che non cada verticalmente (senza attrito).\\
La risultante in quessto caso sarà: 
\begin{gather*}
     \vec{ N} + m\vec{g } = m\vec{a}   
\end{gather*}
Chiamiamo i versori $\hat{u}_n$ il versore di $\vec{N}$ e $\hat{u_t}$ il versore 
della forza centripeta. Se il punto è attaccato ad un filo rigido allora si ha che $N \geq 0$,
inoltre se è una guida o un asticella allora $N \subset R$. \\
Si decompone ora la forza peso nella componente radiale e tangenziale
ottenendo la seguente:
\begin{align}
    m \vec{g} = -mg\cos\theta \hat{u}_n -mg\sin\theta \hat{u}_t  
\end{align}

L'accelerazione ora sarà data dalla derivata seconda:
\begin{align}
    \vec{a} = \ddot{s}\hat{u}_t + \frac{\dot{s}^{2} }{l}\hat{u}_n   
\end{align}

Dividendo le componenti tangenziali e normali si ottiene:
\begin{align}
    \left\{\begin{array}{c}
        -mg \sin\theta = m \ddot{s} \\
        -mg\cos\theta + N = \frac{\dot{s}^{2}  }{l} 
    \end{array}\right.
\end{align}
Noi però sappiamo che dalla definizione degli archi con gli angoli in radianti:
\begin{align}
    S &= \theta l \\
    \dot{S} &= \dot{\theta}l \\
    \ddot{S} &= \ddot{\theta} l  
\end{align}
Quindi possiamo riscrivere le due equazioni con le nuove sostituzioni:
\begin{align}
    \left\{\begin{array}{c}
        ml \ddot{\theta} = -mg\sin\theta \\
        N =ml\dot{\theta}^{2} + mg\cos\theta    
    \end{array}\right.
\end{align}
IL primo membro della seconda è sicuramente positiva e quindi sarà
soddisfatta se e solo se $-\frac{\pi}{2} < \theta < \frac{\pi}{2}$ altrimenti
non sarà negativo. Se l'oggetto nel movimento si ferma, allora non arriverà mai sopra il corpo
del pendolo poiché si azzererà $\dot{\theta}$ che non mi fa tornare l'Euazione 
e quindi si affloscia il filo.\\
Risolvendo la prima allora ottengo :
\begin{gather*}
    \ddot{\theta} + \frac{g}{2}\sin\theta = 0
\end{gather*} 
IPOTESI: Piccole oscillazioni, ossia $\theta(t) \leq \frac{\pi}{20}$ si ottiene attraverso lo sviluppo di Taylor
che $\sin\theta \approx \theta$ e quindi si ottiene la differenziale di un moto
armonico:
\begin{align}
    \ddot{\theta} + \frac{g}{l}\theta = 0.
\end{align}
IL moto è armonico solo per questa approssimazione, altrimenti non è
armonico. 
\begin{align}
    \Omega = \sqrt{\frac{g}{l}} 
\end{align}
E quindi il periodo è dato:
\begin{align}
    T = \frac{2\pi}{\Omega} = 2\pi \sqrt{\frac{l}{g}} 
\end{align}
L'isocronismo del pendolo funziona se e solo se si compiono piccole oscillazioni,
altrimenti il periodo è un'altra cosa. Se l'oscillazion è più grande si trova una formula
in sviluppo in serie (senza dimostrazione poihcè si fa a lab) à popriroe:
\begin{align}
    2\pi \sqrt{\frac{l}{g}}\left( 1 + \left( \frac{1}{2} \right)^{2} \sin^{2} \left( \frac{\theta}{2} \right) 
    + \left( \frac{13}{24} \right)^{2} \sin^{4} \left( \frac{\theta}{2} + ..\right) \right) 
\end{align}
Le caratteristiche di un pendolo sono la massa , la lunghezza del filo, l'accelerazione di
gravità e l'ampiezza del moto (che è adimensionale) quindi il pedolo non diopedne
dimensionalmente dall'ampiezza del moto. Facendo l'analisi dimensionale per il periodo:
\begin{gather*}
    \left[ T \right] = \left[ m^{\alpha} l^{\beta }  g^{\gamma} \right] \\
    \left[ T \right] = \left[ m^{\alpha} l^{\beta}  (l t^{-2} )^{\gamma}  \right]
\end{gather*}
Si ottiene che:
\begin{gather*}
    0 = \alpha \\
    0 = \beta + \gamma \Rightarrow \beta = \frac{1}{2} \\
    1 = -2\gamma \Rightarrow \gamma = -\frac{1}{2}
\end{gather*}
Quindi il periodo sarà:
\begin{align}
    T = \text{const} \sqrt{\frac{l}{g}} 
\end{align}
E la costante sarà $2\pi$ per piccole oscillazioni, altrimenti si utlizza
la formulona generale. E se il sistema di riferimento è inerziale, come si può determinare
tutto.

\section{Sistema di riferimento non inerziale}
SR inzeriale: 
\begin{gather*}
    \vec{F} = m \vec{a}  
\end{gather*}
Se è noto il moto di S' rispetto ad S, allora  noi sappiamo che
\begin{gather*}
    \vec{a} = \vec{a}' + \vec{a}_t + \vec{a}_c    
\end{gather*}
E quindi la risultante per il primo principio saraà:
\begin{gather*}
    \vec{F} = m\vec{a}' + m\vec{a}_t + m\vec{a}_c  
\end{gather*}
Quindi esprimendo tutte le forze con le loro accelerazioni si ottien infine 
che 
\begin{gather*}
    \vec{F}' 0 m\vec{a}'  
\end{gather*}
Posso quindi lavorare in un sistema di riferimento non inerziale se e solo se
sono disposto a lavorare con due pseudo forze aggiuntive.


\end{document}
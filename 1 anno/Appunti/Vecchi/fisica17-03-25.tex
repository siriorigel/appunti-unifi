\documentclass[a4paper, oneside]{article}
\usepackage{graphicx}
\usepackage{amsthm}
\usepackage{amsmath}
\usepackage[a4paper,
            bindingoffset=0.2in,
            left=2cm,
            right=2cm,
            top=2cm,
            bottom=2cm,
            footskip=.25in]{geometry}
\usepackage[italian]{babel}
\usepackage{pgfplots}
\usepackage{tabularx}
\usepackage{tikz}
\usepackage{wrapfig}
\usepackage{color}
\definecolor{page}{rgb}{0.129,0.157,0.212}
\pagecolor{page}
\color{white}
\graphicspath{ {./images/} }
\usetikzlibrary{shapes.geometric}
\usetikzlibrary{datavisualization}
\usetikzlibrary{datavisualization.formats.functions}
\pgfplotsset{width=10cm,compat=1.9}

\title{Fisica}
\author{Tommaso Miliani}
\date{17-03-25}

\begin{document}
\theoremstyle{definition}
\theoremstyle{theorem}
\theoremstyle{lemma}

\newtheorem{definition}{Definizione}[section]
\newtheorem{theorem}{Teorema}[section]
\newtheorem{lemma}{Proposizione}[theorem]
\newtheorem{example}{Esempio}[section]

\maketitle

\section{Il Momento angolare}
Dato che un sistema di $n$ punti ha $n \cdot 3$ gradi di libertà
ed il corpo rigido ne ha $6$, allora dobbiamo trovare un'analoga
per le forze esterne in modo tale che possa descrivere tutti i gradi di libertà.
Si definisce allora il momento angolare per un sistema di punti come
\begin{align}
    \vec{L}_{\Omega} =  \sum_{i = 1}^{n} (P_i - \Omega) \times m_i \vec{v}_i  = \sum_{i = 1}^{n} (P_i - \Omega) \times \vec{q}_i  = \sum_{i = 1}^{n}\vec{l}_i  
\end{align}
Scegliendo un punto qualsiasi $\Omega$ che non deve per forza
coincidere con l'origine degli assi cartesiani (è totalmente
arbitrario) e per ciascun punto del sistema posso definire il vettore
posizione e spostamento. Tutte le definizioni del momento angolare sono 
ugualmente valide, così come la somma dei singoli momenti angolari ed il momento
angolare del sistema o del singolo punto è una quantità vettoriale rispetto ad
un punto qualsiasi $\Omega$. Scegliendo un sistema di riferimento
qualsiasi, posso allora definire la distanza rispetto all'origine delle coordinate come::
\begin{gather*}
    \vec{r}_{i, \Omega} = \vec{r}_i - \vec{r}_{\Omega}  
\end{gather*}

\section{Seconda equazione cardinale della dinamica}
Derivando rispetto al tempo la somma del momento angolare 
allora, essendo che ci sono molte cose che dipendono dal tempo
ottengo:
\begin{gather*}
    \dot{\vec{L} }_{\Omega} = \sum_{i = 1}^{n}(P_i - \Omega) \times m_i \vec{a}_i + \sum_{i = 1}^{n} (\vec{v}_i - \vec{v}_{\Omega}) \times m_i \vec{v}_i     
\end{gather*}
Noi sappiamo che per la seconda di Newton la risultante delle forze è proprio
la somma delle masse per accelerazione, posso allora spezzare la somma ottenendo:
\begin{gather*}
    \vec{v}_i \times \vec{v}_i = 0  
\end{gather*}
Faccio allora la somma solo sul secondo prodotto vettoriale della seconda somma  ottenendo allora
una nuova espressione del momento angolare:
\begin{gather*}
    \dot{\vec{L} } = \sum_{i = 1}^{n} (P_i - \Omega) \times \vec{F}_i - \vec{v}_{\Omega}\times \sum_{i = 1}^{n}m_i\vec{v}_i      
\end{gather*}
Data la definizione di momento di una forza rispetto ad un polo (punto)
$\Omega$ qualsiasi, come:
\begin{align}
    \vec{M} = (P - \Omega) \times \vec{F}  
\end{align}
Dato che le forze sono la somma delle forze interne ed esterne, allora
posso dire che le forze sono la somma delle forze dovute a tutti gli altri
punti 
\begin{gather*}
    \vec{F}_i = \sum_{j \neq  i}^{n}\vec{F}_{i, j}^{(INT)} + \vec{F}_i^{(EXT)}      
\end{gather*}
Dato questo la somma dei momenti angolari diventa:
\begin{gather*}
    \sum_{i = 1}^{n}(P - \Omega) \times \vec{F}_i = \sum_{i = 1}^{n}\sum_{j \neq i}^{n}  (P_i - \Omega) \times F_{i, j}^{(INT)} + \sum_{i = 1}^{n}(P_i - \Omega) \times \vec{F}_i^{(EXT)}   
\end{gather*}
Allora posso esprimere il momento angolare come la seconda cardinale:
\begin{align}
\boxed{    \dot{L}_{\Omega} = \vec{M}_{\Omega}^{(EXT)} - \vec{v}_{\Omega}\times \vec{Q}     
}\end{align}

Opzioni per calcolare il momento quando il secondo termine è nullo
\begin{enumerate}
    \item $\vec{v}_{\Omega} = 0$;
    \item $\Omega = C_M$;
    \item $\vec{v}_{\Omega} \times \vec{v}_{CM} = 0$.   
\end{enumerate} 
Si nota ora che con questa equazione (vettoriale) con tre gradi di libertà
ottengo 6 equazioni scalari con le due cardinali. Con le tre equazioni
cardinali della dinamica si può descrivere qualsiasi moto. E' sempre
vero che le informazioni della seconda sono indipendenti dalla prima? NO, in generale
non è vero poiché in moti come quello traslatorio le informazioni che
ottengo da questa 

\subsection{Il cambio di polo(rinuncia agli studi)} 
Se invece di avere il polo $\Omega$ utilizzassi il polo
$\Omega'$ e quindi imponessi il momento angolare con questo 
nuovo polo $\Omega'$  non cambierebbe assolutamente nulla:
\begin{gather*}
    \vec{L}_{\Omega} = \sum_{i = 1}^{n}(P_i - \Omega) \times m_i \vec{v}_i = \sum_{i = 1}^{n} ((P_i - \Omega') - (\Omega' - \Omega)) \times m_i \vec{v}_i     
\end{gather*}
Tornerei a quella di partenza con $\Omega'$. Se volessi spezzare
invece la sommatoria si otterrebbe:
\begin{gather*}
    \vec{L}_{\Omega} = \vec{L}_{\Omega'} + (\Omega' - \Omega) \times \vec{Q}   
\end{gather*}
Se la quantità di moto è zero allora il momento angolare non dipende dal polo scelto .
Per il modulo delle forze con un polo diverso si ha che
\begin{gather*}
    \vec{M}_{TOT}^{EXT} = \sum_{i = 1}^{n}(P_i - \Omega)\times \vec{F}_i = \sum_{i = 1}^{n}(P_i - \Omega') \times \vec{F} + \sum_{i = 1}^{n} (\Omega' - \Omega) \times \vec{F}_i  = \\
    \vec{M}_{TOT, \Omega'}^{EXT} + (\Omega' - \Omega) \times \vec{F}_i^{EXT}        
\end{gather*}
Se cambio quindi il polo il momento cambia, se però la risultante delle forze esterne è uguale
a zero, allora il momento risultante delle forze esterne è indipendente dal polo.
Se $\Omega' = C_M$ allora il momento angolare rispetto ad un polo generico 
è il momento angolare rispetto al centro di massa :
\begin{gather*}
    \vec{L}_{\Omega} = \vec{L}_C + (C - \Omega) \times M\vec{v}_c   
\end{gather*}
Che succede se $\Omega = C_M$ e $\vec{v}_{\Omega} = 0$?
si ottiene il seguente sistema:
\begin{gather*}
    \left\{\begin{array}{l}
       \vec{F}^{TOT} = \dot{\vec{Q} } \Rightarrow \vec{F}^{EXT} = 0 \Rightarrow \vec{Q} const      \\
       \vec{M}_{\Omega}^{EXT} = \dot{\vec{L}}_{\Omega}  \Rightarrow \vec{M}_{\Omega}^{EXT} = 0 \Rightarrow \vec{L}_{\Omega} const   
    \end{array}\right.
\end{gather*} 
Che è proprio la terza cardinale. Dalla seconda e terza di Newton si ottengono
le equazioni con cardinali con $\vec{L}, \vec{Q}$ isolati. SI ottiene, per un sistema di due punti:
\begin{gather*}
    \vec{Q} = \vec{q}_1 + \vec{q}_2 = const \\
    \dot{\vec{Q}} = 0 = \dot{\vec{q}}_1 + \dot{\vec{q}}_2 \\
    \vec{F}_{1, 2} = -\vec{F}_{2, 1}        
\end{gather*}  
\begin{wrapfigure}{r}{0.4\textwidth}
    \centering
    \caption{Francobollo 1}
    \begin{tikzpicture}
        \draw[dashed](0, 0) -- (3, 0);
        \filldraw(0, 0) circle (1pt) node[anchor = east] {$m_1$} ;
        \filldraw(3, 0) circle (1pt) node[anchor = west] {$m_2$} ;
        \draw[->](0, 0) -- (0, -1) node[at end, left] {$\vec{v}_1$};
        \draw[->](3, 0) -- (3, 1) node[at end, right] {$\vec{v}_2$};
        \draw[->](2, 1) arc(0:180:0.5) node[midway, above] {$\vec{\omega}$};
        \filldraw(1.5, 0) circle (1pt) node[anchor = south] {$C$};
    \end{tikzpicture}    
\end{wrapfigure}
Ottenendo metà del secondo principio della dinamica. 
Posso esprimere il momento angolare come:
\begin{gather*}
    \vec{L}_C = (P_1 - C)\times m_1 \vec{v}_1 + (P_2 - C)\times m_2 \vec{v}_2 \\
    m_1 = m_2 = m \\
    |P_1 - C| = |P_2 - C| = d   \\
    |\vec{v}_1| = \omega d, |\vec{v}_2| = \omega d  
\end{gather*}
Si può ora esprimere tutto in coordinate polari attraverso l'utilizzo
di un angolo $\phi$ e quindi:
\begin{gather*}
    \left\{\begin{array}{l}
        (P_1 - C) = d\hat{u}_r  \Rightarrow  \vec{v}_1 = d\dot{\hat{u}}_r = d \dot{\phi} \hat{u}_{\phi}   \\
        (P_2 - C) = d\hat{u}_r   \Rightarrow  \vec{v}_2 = d\dot{\hat{u}}_r = d \dot{\phi} \hat{u}_{\phi}
    \end{array}\right.
\end{gather*}
In un moto di sola rotazione(non c'è traslazione del sistema che sta ruotando)
e quindi la scelta del polo non è obbligatoria. In generale il momento angolare
aiuta quando si hanno fenomeni di rotazione.

\subsection{Il momento angolare non è sempre parallelo alla velocità angolare}
\begin{wrapfigure}{r}{0.4\textwidth}
    \centering
    \caption{Francobollo 2}
    \begin{tikzpicture}
        \draw(0, 0) -- (0, 4);
        \draw(-1, 1) -- (1, 3);
        \draw[dashed](-2, 2) -- (2, 2);
        \filldraw(-1, 1) circle (1pt) node[anchor = east] {$m_2$};
        \filldraw(1, 3) circle (1pt) node[anchor = west] {$m_2$};
        \draw(0.5, 2.5) arc(45:0:0.7) node[midway, right] {$\alpha$};
        \draw[->](-0.25, 3.5) arc(180: 360: 0.25) node[at end, above] {$\vec{\omega}$};
    \end{tikzpicture}    
\end{wrapfigure}
Le coordinate polari non posso più utilizzarle in questo esercizio
poiché siamo nelle tre dimensioni ma posso utilizzare le coordinate 
cilindriche. 

\section{Caso delle forze parallele}
Date le equazioni cardinali si possono ottenere:
\begin{gather*}
    \left\{\begin{array}{l}
        \vec{F}^{(INT)} = \dot{\vec{P}}_{TOT} \\
        \vec{M}_{\Omega}  = \dot{\vec{L}}_{\Omega} + (V r \times \vec{P}_{TOT} )  
    \end{array}\right.
\end{gather*}
Dato allora il caso della somma delle forze come:
\begin{gather*}
    \vec{F}^{INT} = \sum_{i = 1}^{n} \vec{F}_i = \left(\sum_{i = 1}^{n} f_i\right) \hat{ u}    
\end{gather*}
Allora si ha che il vettore distanza delle forze è proprio 
il vettore applicazione delle forze:
\begin{align}
    \vec{r}_f = \frac{\sum_{i = 1}^{n} f_i \vec{r}_i }{\sum_{i = 1}^{n} f_i} 
\end{align}
Ossia la media pesata delle forze totali.
Per la formula della somma delle forze i due
sistemi hanno lo stesso risultante: basta dimostrare che
abbiano lo stesso momento risultante:
\begin{gather*}
    \vec{M} = \sum_{i} (\vec{r}_i \times f_i \hat{u}) = \left(\sum_{i} f_i \vec{r}_i \right) \times \hat{u} = \vec{r}_f \times \vec{F}   
\end{gather*}

\subsection{Il caso della forza peso}
Nel caso della forza peso, considerata la forza peso come:
\begin{gather*}
    \vec{g} = -g\hat{k}  
\end{gather*}
si ha allora che $f_i = -m_ig$ e dalle relazioni precedenti si ottiene che
applicata nel punto $\vec{r}_G$ quel polo prende il nome
di \textbf{baricentro}, che nei corpi non troppo estesi coincide con il centro 
di massa e si esprime come:
\begin{align}
    \vec{r}_G = \frac{1}{M}\sum_{i = 1}^{n}m_i \vec{r}_i = \vec{r}_C    
\end{align} 

\subsection{L'utilizzo del baricentro: la forza di trascinamento nei sistemi complessi}
Dato un sistema di riferimento assiale, le forze parallele
si applicano tutte nel centro di massa poiché sono tutte parallele.
\begin{gather*}
    \vec{F}_{T, i} = -m_i \vec{a}_{O'}  
\end{gather*}
E quindi la loro somma è proprio la 
risultante di tutte le forze applicate.

\section{Lavoro ed energia nei sistemi}
Per un sistema di punti è ragionevole pensare
che l'energia cinetica sia la somma dell'energia cinetica
di tutti i punti e quindi dato un sistema di riferimento
(con terna di assi sempre parallele a quella del sistema fisso)
e dalle equazioni cardinali e scelto $\vec{r} = (O' - O)$ si ottiene
allora: 
\begin{gather*}
    \vec{r}_i = \vec{r}_i' + \vec{r}_C \\
    \vec{v}_i = \vec{v}_i' + \vec{v}_C      
\end{gather*} 
Data allora la definizione del vettore posizione
come media pesata delle forze, si può ottenere:
\begin{gather*}
    \vec{r}_C = \frac{\sum_{i = 1}^{n} m_i (\vec{r}_i' + \vec{r}_C)}{\sum_{i = 1}^{n}m_i} 
\end{gather*}


\subsection{Teorema di Konig per l'energia cinetica}
L'energia cinetica di un sistema di punti è esattamente la somma
delle energie cinetiche di tutti i punti del sistema:
\begin{align}
    K = &\sum_{i = 1}^{n}\frac{1}{2}m_i v_i^{2} = \sum_{i = 1}^{n}\frac{1}{2}m_i \vec{v}_i \cdot  \vec{v}_i = \\
    &\sum_{i = 1}^{n}\frac{1}{2}m_i (\vec{v}_i' + \vec{v}_C) \cdot  (\vec{v}_i' + \vec{v}_C )  = \\
    &\sum_{i = 1}^{n} \frac{1}{2}m_i v_i^{2} + \sum_{i = 1}^{n}m_i \vec{v}_i' \vec{v}_C  + \sum_{i = 1}^{n}\frac{1}{2}m_i \vec{v}_C^{2}   
\end{align}
I singoli termini allora indicano grandezze diverse: il primo termine p proprio
l'energia cinetica rispetto al centro di massa, il secondo termine è sempre zero mentre
la terza grandezza è l'energia cinetica del centro di massa.
Il teorema di Konig per l'energia cinetica ci dice proprio che l'energia cinetica
di un sistema è sempre esprimibile come la somma dell'energia cinetica del centro di massa
come se avesse la massa in un punto e velocità del centro di massa
ed un termine che è l'energia cinetica rispetto al centro di massa per
ogni punto. 
\begin{gather*}
    \delta L_i = dK_i \\
    \sum_{i = 1}^{n}\delta L_i  = \sum_{i = 1}^{n}dK_i = dK
\end{gather*}
Generalmente il lavoro delle forze interne è diverso da zero mentre
posso ignorare spesso il lavoro delle forze esterne (dimostrato mooolto più avanti)
per il corpo rigido però posso dire che il lavoro delle forze interne sia zero.

\subsection{Teorema di Konig per il momento angolare}
Rispetto ad un polo $\Omega$ scriviamo:
\begin{gather*}
    \vec{L}_{\Omega} = \sum_{i = 1}^{n}(P_i - \Omega)\times m_i \vec{v}_i \\
    \sum_{i = 1}^{n}((P_i - C) + (C - \Omega))\times m_i (\vec{v}_i + \vec{v}_C)    
\end{gather*} 
Spezzando tutto allora:
\begin{gather*}
    \sum_{i = 1}^{n}(P_i - C) \times m_i \vec{v}_i' + \sum_{i = 1}^{n}(C - \Omega) \times m_i \vec{v}_i ' + \sum_{i = 1}^{n} (P_i - C) \times m_i \vec{v}_C + \sum_{i = 1}^{n}(C - \Omega) \times m_i \vec{v}_C    
\end{gather*}
Il secondo ed il terzo termine sono zero poiché è esattamente la distanza dal centro
meno la distanza dal centro per la definizione di centro di massa:
$P_i - C$ è esattamente (quando sommo tutte le masse) la distanza dal centro di massa dal centro di massa
. Gli altri due pezzi sono combinati ottenendo:
\begin{gather*}
    \vec{L}_{\Omega} = \vec{L}_C' + (C - \Omega) \times M\vec{v}_C   
\end{gather*} 
Il primo termine è proprio il momento angolare rispetto al centro di massa:
\begin{gather*}
    \vec{L}_C' = \sum_{i = 1}^{n}(P_i - C) \times m_i \vec{v}_i   
\end{gather*}
mentre il secondo termine è il momento angolare del centro di massa
\begin{gather*}
    (C - \Omega) \times M\vec{v}_C 
\end{gather*}
IN meccanica quantistica il primo termine è proprio lo spin
di una particella (in altri testi si usa anche per gli oggetti macroscopici)
che è intrinseco della particella. Per il corpo rigido sarà molto utile
per definire il momento angolare.

\subsection{Il momento angolare del moto circolare uniforme}
\begin{wrapfigure}{r}{0.4\textwidth}
    \centering
    \caption{Mom. angolare cerchi}
    \begin{tikzpicture}
        \draw(0, 0) circle(2) node[anchor = east] {$O$};
        \draw(0, 0) -- (1.41, 1.41) node[midway, below] {$\vec{r}$};
        \draw[->](1.41, 1.41) -- (1, 1.82) node[at end, right] {$\vec{v}$};
        \draw[->](0, 0) -- (0.5, 0) node[at end, below] {$x$};
        \draw[->](0, 0) --(0, 0.5) node[at end, left] {$y$};
        \draw[->] (0.4, 0) arc (0:45:0.4) node[midway, right] {$\phi$};
    \end{tikzpicture}    
\end{wrapfigure}
Nel moto circolare uniforme è accelerato ed il suo momento angolare
è dato dalla relazione:
\begin{gather*}
    \vec{L}_O  = \vec{r}\times m\vec{v} = (mr^{2})\dot{\phi}\hat{k}   
\end{gather*}
Con le seguenti sostituzioni:
\begin{gather*}
    v = \omega r = \dot{\phi}r \\
    w = \dot{\phi}
\end{gather*}
Il teorema delle forze vive mi dice che è costante l'energia cinetica
e la forza centripeta non compie lavoro sulla traiettoria (quindi è accelerato)
e cambia la direzione della velocità ma non compie nessun lavoro
sul corpo e quindi l'energia cinetica è costante.

\end{document}
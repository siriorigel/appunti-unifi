\documentclass[a4paper, oneside]{book}
\usepackage{graphicx}
\usepackage{amsthm}
\usepackage{amsmath}
\usepackage[a4paper,
            bindingoffset=0.2in,
            left=2cm,
            right=2cm,
            top=2cm,
            bottom=2cm,
            footskip=.25in]{geometry}
\usepackage[italian]{babel}
\usepackage{pgfplots}
\usepackage{tabularx}
\usepackage{wrapfig} %wrapping images in text
\usepackage{color}
\pagecolor{black}
\color{white}
\graphicspath{ {./images/} }
\usetikzlibrary{datavisualization}
\usetikzlibrary{datavisualization.formats.functions}
\pgfplotsset{width=10cm,compat=1.9}

\title{FIsica}
\author{Tommaso Miliani}
\date{Appunti 21-02-25}

\begin{document}
\theoremstyle{definition}
\theoremstyle{theorem}
\theoremstyle{lemma}

\newtheorem{definition}{Definizione}[chapter]
\newtheorem{theorem}{Teorema}[chapter]
\newtheorem{lemma}{Proposizione}[theorem]

\maketitle

\section{SIstemi con rotazione}
NElla situazione in cui $\vec{\omega}$ è costante, noi sappiamo 
che su di una giostra il pavimento è orizzontale e gira intorno ad un
asse vertical . COme si scelgono i sistemi di riferimento? Se ne sceglie
uno solidale con la giostra $S'$ in modo tale che 
\begin{wrapfig}
    \begin{tikzpicture}
        \draw(0, 0) -- (3, 0);
        \draw(0, 0) -- (1, 3);
        \draw(1, 3) -- (4, 3);
        \draw(4, 3) -- (3, 0);
        \draw(1.5, 1.5) circle (1);
        \draw[->](1.5, 1.5) -- (1.5, 2.5) node[at end, right] {$\vec{\omega}$};
        \draw (1.5, 1.5) -- (2, 2.5);
        \filldraw (2, 2.5) circle(1pt);
        \draw
    \end{tikzpicture}
\end{wrapfig}
\begin{gather*}
    
\end{gather*}

\end{document}
\documentclass[a4paper, oneside]{article}
\usepackage{graphicx}
\usepackage{amsthm}
\usepackage{amsmath}
\usepackage[a4paper,
            bindingoffset=0.2in,
            left=2cm,
            right=2cm,
            top=2cm,
            bottom=2cm,
            footskip=.25in]{geometry}
\usepackage[italian]{babel}
\usepackage{pgfplots}
\usepackage{tabularx}
\usepackage{tikz}
\usepackage{wrapfig}
\usepackage{color}
\definecolor{page}{rgb}{0.129,0.157,0.212}
\pagecolor{page}
\color{white}
\graphicspath{ {./images/} }
\usetikzlibrary{shapes.geometric}
\usetikzlibrary{datavisualization}
\usetikzlibrary{datavisualization.formats.functions}
\pgfplotsset{width=10cm,compat=1.9}

\title{Banchi esercizi}
\author{Tommaso Miliani}
\date{20-03-25}

\begin{document}
\theoremstyle{definition}
\theoremstyle{theorem}
\theoremstyle{lemma}

\newtheorem{definition}{Definizione}[section]
\newtheorem{theorem}{Teorema}[section]
\newtheorem{lemma}{Proposizione}[theorem]
\newtheorem{example}{Esempio}[section]

\maketitle

\section{Esercizio esame novembre 24}
\begin{wrapfigure}{r}{0.4\textwidth}
    \centering
    \caption{Schematizzazione del problema}
    \begin{tikzpicture}
        \draw(0, 0) circle(3);
        \draw[dashed, very thin](-4, 0) -- (4, 0) node[at end, below] {$x$};
        \draw[dashed, very thin](0, -4) -- (0, 4) node[at end, left] {$y$};
        \filldraw (0, 2) circle(1pt) node[anchor = east] {$P$};
    \end{tikzpicture}    
\end{wrapfigure}
Sul filo agiscono diverse forze ma non essendoci la gravità,
la forza che la rimpiazza è quella centrifuga e tende a spostarlo
verso una direzione (è radiale). Le ipotesi sono:
\begin{enumerate}
    \item La piattaforma ruota con velocità $\omega_0$ costante (c'è un motore esterno che tiene in rotazione il disco).
    \item Il filo del pendolo è ideale;
    \item Si lavora con la schematizzazione del punto materiale;
    \item SIstema non inerziale $O_{xy}$.
\end{enumerate}
Punto a: \\
Per calcolare la distanza si può scrivere per le coordinate:
\begin{gather*}
    \left\{\begin{array}{l}
        x_P = l\sin\phi \\
        y_P = a + l\sin\phi
    \end{array}\right.
\end{gather*}
Adesso per calcolare la distanza dal centro $O$ basterà calcolare la distanza
dal filo come:
\begin{gather*}
    |P - O|^{2} = x_P^{2} + y_P^{2} = l^{2}\sin^{2}\phi + a^{2} + l^{2}\cos^{2}\phi + 2al\cos\phi        
\end{gather*}
E quindi
\begin{gather*}
    |P - O| = \sqrt{l^{2} + a^{2} +2al\cos\phi} 
\end{gather*}
PUNTO B: \\
Sul punto P ci sono diverse forze: prima di tutto la forza di trascinamento verso l'esterno,
la forza di Coriolis (che non so dove agisce) e la forza centrifuga. Di queste forze quella che compie
lavoro è quella di trascinamento mentre quelle che non compiono lavoro sono la forza di Coriolis e la tensione (poiché
è sempre ortogonale alla direzione tangente). \\
Il lavoro della forza di trascinamento è data da:
\begin{gather*}
    V = -\frac{1}{2}m\omega^{2}_0 |P - O|^{2}   = -\frac{1}{2}m\omega_0^{2}(l^{2} + a^{2} + 2al\cos\phi) 
\end{gather*}
I termini sono quasi tutti costanti, possiamo allora considerare solo
il coseno e riassumere quindi l'energia potenziale come una costante meno
il pezzo non costante:
\begin{gather*}
    V = cost - m\omega_0^{2}al\cos\phi 
\end{gather*}
Derivando si ottiene proprio l'espressione:
\begin{gather*}
    V' = -m\omega_0^{2}al(-sin\phi) = 0 
\end{gather*}
Le cui soluzioni per ottenere l'equilibrio sono proprio $0, \pi$. Verifichiamo
allora la stabilità di questo equilibrio, per cui la derivata seconda sarà proprio:
\begin{gather*}
    V''= m\omega_0^{2}al\cos\phi  \Rightarrow  \left\{\begin{array}{l}
        > 0, \quad \phi = 0 \text{ stabile}\\
        < 0, \quad \phi = \pi \text{ instabile}
    \end{array}\right.
\end{gather*}

PUNTO C: \\
L'energia non è conservata in un sistema di riferimento esterno a causa
del motore del disco che lo tiene in rotazione mentre si conserva nel sistema
di riferimento ruotante $O_{x, y}$. Il ruolo del motore allora è descritto
con la forza di trascinamento. Possiamo allora scrivere l'energia nel sistema di
riferimento ruotante con un termine cinetico e uno potenziale:
\begin{gather*}
    \left\{\begin{array}{l}
        \dot{x}_P = l\cos\phi \cdot  \dot{phi} \\
        \dot{y}_P = -l\sin\phi \cdot  \dot{phi}
    \end{array}\right.
\end{gather*} 
Da qui allora si ha che 
\begin{gather*}
    \dot{V}_P^{2} = \dot{x}_P^{2} + \dot{y}_P^{2}  = l^{2}\dot{\phi}^{2}  
\end{gather*}
Troviamo allora il classico risultato del moto circolare e l'energia allora è ora:
\begin{gather*}
    E = \frac{1}{2}ml^{2}\dot{\phi}^{2} - m\omega_0^{2}al\cos\phi   
\end{gather*}
Con la costante dell'energia potenziale che posso liberamente mettere uguale 
a zero poiché l'energia meccanica è definita a meno di una costante. Con il metodo
dell'energia allora posso derivare l'energia ottenendo:
\begin{gather*}
    \dot{E} = 0 = \frac{1}{2}ml^{2}2\ddot{\phi}\dot{\phi} - m\omega^{2}_0al(-\sin\phi)\dot{\phi} = 0   
\end{gather*}
Semplificando si ottiene la seguente equazione di moto:
\begin{gather*}
    l\ddot{\phi} + \omega_0^{2} a\sin\phi = 0
\end{gather*}
Diventa allora l'equazione del pendolo senza la forza peso
ma con la forza centrifuga. Nel limite allora delle piccole oscillazioni 
sappiamo allora che $\phi$ è vicino a quella di equilibrio. 
\begin{gather*}
    \ddot{\phi} + \Omega^{2}\phi = 0 , \quad \Omega^{2} = \frac{\omega_0^{2} a}{l} 
\end{gather*}
Allora il periodo delle oscillazioni si ha che è:
\begin{gather*}
    T = \frac{2\pi}{\Omega}, \Omega = \sqrt{\frac{l}{a}} 
\end{gather*}
PUNTO D: \\
Senza le approssimazioni delle piccole oscillazioni dobbiamo 
prima studiare il moto e trovare la tensione del filo. Si sa da Newton che:
\begin{gather*}
    m\vec{a} = \vec{T} + \vec{F}_{co} + \vec{F}_T    
\end{gather*}
Allora queste si esprimono come:
\begin{align*}
    \vec{F}_{co} &= -2m\omega_0 \times \vec{v}  \\
    &=-2m\omega_0l\dot{\phi}\hat{u}_n  
\end{align*}
Dato che la tensione è opposta a $\hat{u}_n$ si ha:
\begin{gather*}
    \vec{T} = -T\hat{u}_n   
\end{gather*} 
E la trascinamento:
\begin{gather*}
    \vec{F}_T = m\omega_0^{2}(P - O)  
\end{gather*} 
Proietto allora lungo la direzione N tutte le forze ottenendo che:
posso scrivere intanto la tensione:
\begin{gather*}
    \vec{T} = m\vec{a} - \vec{F}_{co} - \vec{F}_T    \\
    \vec{T} = - \hat{u}_n \vec{T} = -(m\vec{a}-\vec{F}_{co} - \vec{F}_T) \hat{u}_n      
\end{gather*} 
Allora scriviamo i vettori che compaiono come proiezioni sulla direzione n:
\begin{gather*}
    \vec{a}\cdot \hat{u}_n = -l\dot{\phi}^{2} \text{ centripeta} \\
    \vec{F}_T \cdot  \hat{u}_n = m\omega_0^{2} (P - O)\cdot \hat{u}_n      
\end{gather*}
Possiamo esprimere il versore lungo n come:
\begin{gather*}
    \hat{u}_n = \sin\phi\hat{i} + \cos\phi\hat{j}   
\end{gather*}
E si ottiene allora la forza di trascinamento come:
\begin{gather*}
    \vec{F}_T = m\omega_0^{2}(x_P\sin\phi + y_P\cos\phi)  
\end{gather*}
Allora date le sostituzioni delle coordinate di P, possiamo
allora dire che:
\begin{gather*}
    m\omega_0^{2}(l + a\cos\phi) 
\end{gather*}
Adesso la proiezione della forza di coriolis è proprio:
\begin{gather*}
    \vec{F}_{co} -2m\omega_0l\dot{\phi}
\end{gather*}
Sostituendo tutte le espressioni trovate ora nell'espressione
per la tensione si ottiene allora:
\begin{gather*}
    T = ml\dot{\phi}^{2}-2m\omega_0l\dot{\phi} - 2m\omega_0 l \dot{\phi} 
\end{gather*}
Dobbiamo allora trovare $\dot{\phi}$ per poter trovare l'espressione per la tensione.
L'energia la abbiamo scritta prima come:
\begin{gather*}
    E = \frac{1}{2}ml^{2}\dot{\phi}^{2} - m\omega_0^{2}al\cos\phi   
\end{gather*}
Parte da fermo $(\dot{\phi} = 0)$ e quindi:
\begin{gather*}
    E_0 = -m\omega _0^{2}al\cos\phi_0 
\end{gather*}
E quindi portando dall'altra parte possiamo ottenere:
\begin{gather*}
    \frac{1}{2}ml^{2} \dot{\phi}^{2} = -m\omega_0^{2}al(\cos\phi_= - \cos\phi)  \\
    \dot{\phi}^{2} = \frac{2\omega_0^{2}a}{l} (\cos\phi - \cos\phi_0) \geq 0 
\end{gather*} 
Dato che l'espressione ha senso se e solo se è positiva, allora devo (dato $\phi_0 = \frac{\pi}{6}$) che
\begin{gather*}
    -\phi_0 \leq \phi \leq \phi_0
\end{gather*}
Anche senza le piccole oscillazioni allora so che il pendolo può oscillare (in 
questa situazione) solo tra quei due valori. ALlora $\dot{\phi}$ diventa:
\begin{gather*}
    \dot{\phi} = -\sqrt{\frac{2\omega_0^{2} a}{l}(\cos\phi - \cos\phi_0)} 
\end{gather*}
Il segno meno è dovuto poiché nel primo periodo $\phi$ diminuisce 
e quindi la tensione diventa:
\begin{gather*}
    T = ml\left(\frac{2\omega_0^{2} a}{l}(\cos\phi - \cos\phi_0) - 2\omega_0 \cdot  \left( -\sqrt{\frac{2\omega_0^{2} a}{l}(\cos\phi - \cos\phi_0)} \right) + \frac{\omega_0^{2}}{l}(l + a \cos\phi)\right)
\end{gather*}
Raccogliendo allora:
\begin{gather*}
    T = ml\omega_0^{2}\left(\frac{3a}{l}\cos\phi - \frac{2a}{l}\cos\phi_0 + 1 + 2\sqrt{\frac{2a}{l}(\cos\phi - \cos\phi_0)} \right) 
\end{gather*}


\end{document}
\documentclass[a4paper, oneside]{article}
\usepackage{graphicx}
\usepackage{amsthm}
\usepackage{amsmath}
\usepackage[a4paper,
            bindingoffset=0.2in,
            left=2cm,
            right=2cm,
            top=2cm,
            bottom=2cm,
            footskip=.25in]{geometry}
\usepackage[italian]{babel}
\usepackage{pgfplots}
\usepackage{tabularx}
\usepackage{wrapfig}
\graphicspath{ {./images/} }
\usetikzlibrary{datavisualization}
\usetikzlibrary{datavisualization.formats.functions}
\pgfplotsset{width=10cm,compat=1.9}

\title{Lab merda}
\author{Tommaso Miliani}
\date{26-02-25}

\begin{document}
\theoremstyle{definition}
\theoremstyle{theorem}
\theoremstyle{lemma}

\newtheorem{definition}{Definizione}[section]
\newtheorem{theorem}{Teorema}[section]
\newtheorem{lemma}{Proposizione}[theorem]

\maketitle

\section{ALtra esperienza (grande borrata)}
\begin{wrapfigure}{r}{0.4\textwidth}
    \centering
    \label{Fig 12.2}
    \caption{Grafico del fit lineare delle misure}
    \begin{tikzpicture}
        \draw(0, 0) -- (3, 0) node[at end, below] {$p$};
        \draw(0, 0) -- (0, 3) node[at end, left] {$h$};
        \draw[|-|](0.5, 2.75) -- (0.5, 2.5);
        \draw[|-|](1, 2.25) -- (1, 2);
        \draw[|-|](1.5, 1.75) -- (1.5, 1.5);
        \draw[|-|](2, 1.25) -- (2, 1);
        \draw[|-|](2.5, 0.75) -- (2.5, 0.5);
    \end{tikzpicture}    
\end{wrapfigure}
COn questo grafico (un fit lineare che va compilato mentre si prendono le misure),
allora se esce fuori un punto che sta fuori dalla retta ideale, questo va o scartato oppure
è un campanello di allarme per informarci che qualcosa non va. SI ottiene
allora $E$:
\begin{align}
    E = \frac{L^{2} }{4ab^{3}c}
\end{align}
Si può fare una stima a priori propagando l'errore in modo tale da vedere
come si propaga l'errore:
\begin{gather*}
    \frac{\Delta E}{E} = 3\frac{\Delta L}{L} + \frac{\Delta a}{a} + 3\frac{\Delta b}{b} + \frac{\Delta c}{c}
\end{gather*}
DOve $c$ è la differenza di quota tra i pesi:
\begin{gather*}
    c = \frac{h_0 - h}{p}
\end{gather*}
E quindi si ottiene che:
\begin{gather*}
    \frac{\Delta c}{c} = \frac{\Delta h_0 + \Delta h}{h_0 - h} + \frac{\Delta p}{p}
\end{gather*}
Facendo una stima dei contributi a propri si può utilizzare i valori di
$a, b$ calcolati a priori con le relative incertezze e prenderne il prim valore per
poter poi determinare a termine dell'esperienza le nuove grandezze in modo più preciso:
per la lunghezza $L$, questa è calcolata tra i coltelli. Se si fanno tutte le misurte
perfettamente , la misura più critica sarà quella di b. 


\end{document}
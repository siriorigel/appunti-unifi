\documentclass[a4paper, oneside]{article}
\usepackage{graphicx}
\usepackage{amsthm}
\usepackage{amsmath}
\usepackage[a4paper,
            bindingoffset=0.2in,
            left=2cm,
            right=2cm,
            top=2cm,
            bottom=2cm,
            footskip=.25in]{geometry}
\usepackage[italian]{babel}
\usepackage{pgfplots}
\usepackage{tabularx}
\usepackage{wrapfig}
\graphicspath{ {./images/} }
\usetikzlibrary{shapes.geometric}
\usetikzlibrary{datavisualization}
\usetikzlibrary{datavisualization.formats.functions}
\pgfplotsset{width=10cm,compat=1.9}

\title{FIsica}
\author{Tommaso Miliani}
\date{07-03-25}

\begin{document}
\theoremstyle{definition}
\theoremstyle{theorem}
\theoremstyle{lemma}

\newtheorem{definition}{Definizione}[section]
\newtheorem{theorem}{Teorema}[section]
\newtheorem{lemma}{Proposizione}[theorem]

\maketitle

\section{Le forze centrali a simmetria sferica}
\begin{wrapfigure}{r}{0.4\textwidth}
    \centering
    \label{FIgn}
    \caption{}
    \begin{tikzpicture}
        \filldraw(0, 0) circle (1pt) node[anchor = east] {$C$};
        \filldraw(1, 1) circle (1pt) node[anchor = west] {$P$};
        \draw[dashed, thin] (0, 0) -- (1, 1) node[midway, below] {$r$};
        \draw[->] (1, 1) -- (0.8, 0.8) node[midway, above] {$\hat{u}_r$};
    \end{tikzpicture}    
\end{wrapfigure}
Le forze centrali sono delle forze sempre dirette verso un punto
come la gravità nella gravitazione universale, o la forza elettrostatica
nell'elettromagnetismo; inoltre sono a simmetria sferica poiché il loro modulo 
è in funzione solamente del raggio vettore e nient'altro. Questo tipo di forze sono conservative?
il lavoro di questo tipo di forze non è altro che
\begin{gather*}
    \delta L = \vec{F} \cdot \vec{dr}    
\end{gather*}
E quindi il vettore
\begin{gather*}
    \vec{r}= r\hat{u}_r  
\end{gather*}
Per cui:
\begin{gather*}
    \vec{dr} = dr\hat{u}_r + r\vec{du}_r   
\end{gather*}
DAto che il prodotto scalare tra lo stesso vettore è uguale
ad 1 ma la somma dei contributi $d\vec{u}m\cdot \vec{u} + \vec{u}d\vec{u}  = 0$
allora \begin{gather*}
    \vec{du} \perp \hat{u}  
\end{gather*}    
Allora il lavoro diventa:
\begin{gather*}
    \delta L = f(\hat{u}_r ) \cdot  (d\vec{u}\vec{u} + \vec{u}d\vec{u})
\end{gather*}
PEr cui preso l'integrale per un lavoro continuo e finito, allora si ottiene:
\begin{gather*}
    L_{AB} = \int_{A}^{B}f(r) dr = G(r_B) - G(r_A) 
\end{gather*}
Posto allora 
\begin{gather*}
    V(r) = -G(r)
\end{gather*}
Posso risalire alla definizione di energia potenziale e quindi questo tipo di forze
sono conservative.

\subsection{Forze a simmetria sferica non centrali}
\begin{wrapfigure}{r}{0.2\textwidth}
    \centering
    \caption{Forza non centrale a simmetria sferica}
    \begin{tikzpicture}
        \draw(0, 0) -- (2, 0);
        \draw(0, 0) -- (0, 2);
        \draw[dashed, thin](0, 0) -- (1.41, 1.41) node[midway, below] {$r$};
        \draw (0.5, 0) arc (0:45:0.5) node[midway, right] {$\theta$};
        \draw[->] (1.41, 1.41) -- (1, 1.82) node[midway, above] {$\vec{F}$};
        \draw[->] (1.41, 1.41) -- (1.82, 1.82) node[at end, right] {$\hat{u}_r$};
    \end{tikzpicture}    
\end{wrapfigure}
PRendendo una circuitazione allora l'integrale di linea:
\begin{gather*}
    \oint \vec{F}\vec{dr} = f(r) 2\pi R   
\end{gather*}
Essendo questo integrale non uguale a zero, allora la forza non è conservativa.

\subsection{Forze centrali non a simmetria sferica}
\begin{wrapfigure}{r}{0.2\textwidth}
    \centering
    \caption{Forza centrale non a simmetria sferica}
    \begin{tikzpicture}
        \draw(0, 0) -- (2, 0) node[at end, below] {$x$};
        \draw(0, 0) -- (0, 2) node[at end, left] {$y$};
        \draw[->] (0, 0) -- (1, 0) node[at start, left] {$C$} node[midway, below] {$R$} node[at end, below] {$A$};
        \draw[->] (0, 1) -- (0, 0) node[at start, left] {$B$};
        \draw[dashed] (0, 0) -- (0.7, 0.7);
        \draw[->](0.7, 0.7) -- (1, 1) node[at end, right] {$\vec{F}$};
        \draw[->] (1, 0) arc (0: 90: 1); 
    \end{tikzpicture}    
\end{wrapfigure}
Queste forze sono centrali ma non hanno una simmetria sferica per cui il loro modulo cambia
se cambia la loro posizione sulla sfera allora in questo caso, 
la forza è:
\begin{gather*}
    \vec{F} = f(r)\cos\theta\hat{u}_r  
\end{gather*}
Allora l'integrale di linea:
\begin{gather*}
    \oint \vec{F}\vec{dr} = \int_{C}^{A} \vec{F}\vec{dr} + \int_{A}^{B} \vec{F}\vec{dr} + \int_{B}^{C}\vec{F}\vec{dr}           
\end{gather*}
Quindi l'integrale di linea, risolvendo tutti gli integrali spezzati diventa:
\begin{gather*}
    \oint \vec{F}\vec{dr} = \int_{r = 0}^{r_A = R} f(r)dr + 0 + 0   
\end{gather*}
Così come nel disegno, la forza non è simmetria rispetto ad una sfera poiché 
non dipende solo ed esclusivamente dal modulo del raggio vettore
che congiunge la forza al centro della sfera ideale ma anche dall'orientazione
del vettore forza. 


\section{Il caso della molla tridimensionale}
\begin{wrapfigure}{r}{0.3\textwidth}
    \centering
    \label{Fig}
    \caption{ads}
    \begin{tikzpicture}
        \draw[decoration={aspect=0.3, segment length=2mm, amplitude=1.3mm,coil},decorate,opacity=0.9] (0, 0) -- (2, 1);
        \filldraw(0, 0) circle (1pt) node[anchor = east] {$C$};
        \filldraw(2, 1) circle (1pt) node[anchor = south] {$m$};
        \draw[->](2, 1) -- (2.5, 1.25) node[at end, above] {$\vec{F}_d$};
        \draw[|-|](0, 0.5) -- (1, 1) node[midway, above] {$l_0$};
    \end{tikzpicture}    
\end{wrapfigure}
Se io attacco la molla in un certo punto allora quando tiro la molla
si ha una forza di richiamo verso il centro della molla e quindi questa
forza è proprio centrale a simmetria sferica e avrà quindi una certa 
energia potenziale e sarà anche conservativa .
\begin{gather*}
    \vec{F}_d = -K(r - l_0)\hat{u}_r \\
    V = \frac{1}{2}K(r - l_0)^{2}   
\end{gather*}
Allora l'energia meccanica diventa nel caso di un certo angolo rispetto da uno degli
assi nel piano di giacenza della molla:
\begin{gather*}
    E = \frac{1}{2}mv^{2} + \frac{1}{2}kr^{2} \\
    \vec{r} = r\hat{u}_r \\
    \vec{v} = \dot{r}\hat{u}_r + r\dot{\theta } \hat{u}_{\theta}        \\
    v^{2} = \dot{r}^{2} + r^{2}\dot{\theta}^{2}      
\end{gather*}
La sola energia meccanica non mi basta per risolvere il problema
poiché ho due gradi di libertà e dunque un equazione sola non mi basta per
poter risolvere il problema.

\section{Piattaforma ruotante con $\vec{\omega} = 0$  }
\begin{wrapfigure}{r}{0.4\textwidth}
    \centering
    \caption{La piattaforma ruotante}
    \begin{tikzpicture}
        \node[ellipse,
        draw,
	    minimum width = 4cm, 
	    minimum height = 2cm] (e) at (0,0) {};
        \draw[dashed, very thin, ->](0, -1.5) -- (0, 1.5) node[at end, above] {$z$};
        \draw[->](0.5, 1.3) -- (0.5, 1.6) node[midway, right] {$\vec{\omega}$};
        \draw[|-|](0, 0) -- (1, 0) node[midway, below] {$\rho$};
        \filldraw(1, 0) circle(1pt) node[anchor = west] {$m$};
    \end{tikzpicture}    
\end{wrapfigure}
Poniamo l'osservatore in un SDR non inerziale proprio sopra la Piattaforma
osservando un punto P anch'esso sulla piattaforma. Sul punto p
agiscono una forza di trascinamento mentre la forza di gravità
è bilanciata dalla normale del piano. 
\begin{gather*}
    \vec{F}_t = -m\vec{\omega}\times(\vec{\omega} \times (\rho\hat{u}_P  + z\hat{k})  )  = +m\omega^{2}\rho\hat{u}_P  
\end{gather*}
Vale solo il contributo di $rho$ poiché l'oggetto non si muove lungo la verticale.
Nel caso di un piccolo spostamento $\vec{dr}$ allora si ottiene:
\begin{gather*}
    \vec{dr} = d\rho \hat{u}_P + \rho d\hat{u}_P + dz\hat{k}      
\end{gather*} 

Allora si ottiene proprio il lavoro come:
\begin{gather*}
    \delta L = \vec{F}_t \vec{dr} = (m\omega^{2}\rho\hat{u}_P  )\cdot (d\rho\hat{u}_P + \rho \vec{dr} + dz\hat{k}   )  = m\omega^{2}\rho d\rho  
\end{gather*}
L'energia centrifuga 
\begin{gather*}
    V = -\frac{1}{2}m\omega^{2}\rho^{2}  
\end{gather*}
La forza di coriolis
\begin{gather*}
    \vec{F}_{co} -2\vec{\omega}\times \vec{v}   
\end{gather*}
La forza di Coriolis non fa lavoro poiché è sempre ortogonale
allo spostamento infatti:
\begin{gather*}
    \delta L = \vec{F}\cdot \vec{dr} = (-2m\vec{\omega} \times \vec{v}  ) \cdot  \vec{v} \cdot dt = 0   
\end{gather*}
FA sempre lavoro zero e quindi non altera il bilancio energetico
mentre se si muove allora si cambiano le condizioni e quindi potrebbero
non valere più le condizioni di vincolo liscio. \\
Nel SDR non inerziale in assenza di attriti e altre forze, si conserva
l'energia meccanica:
\begin{gather*}
    E = \frac{1}{2}mv^{2} - \frac{1}{2}m\omega^{2}\rho^{2}   
\end{gather*} 
Per un osservatore esterno alla piattaforma ruotante l'energia meccanica
non si conserva. Infatti un'osservatore esterno vede che l'energia cinetica
cambia e non vede altre forze conservative(la centrifuga non c'è) ma l'energia
meccanica non si conserva poiché c'è $\omega$ costante data dal motore
della piattaforma ruotante che compie lavoro per tenere un certo regime. 
Nel caso della pallina che è su questa piattaforma:
\begin{gather*}
    t = 0, \rho = \rho_0, \dot{\rho} = 0, E = -\frac{1}{2}m\omega^{2}\rho_0^{2}   \\
    t = t_f, \rho = R, \dot{\rho} = ?, E = \frac{1}{2}m\dot{\rho}^{2}- \frac{1}{2}m\omega^{2}R^{2}      
\end{gather*} 
Dato che l'energia si conserva allora posso uguagliarle e ottenere:
\begin{gather*}
    \dot{\rho} = \pm \omega\sqrt{R^{2} - \rho_0^{2}}  
\end{gather*}
Ha senso? Si perché R è il bordo della giostra e quindi quella radice
è ben definita e quindi è vero sia per la conservazione dell'energia
ed in tutti i casi del movimento da $\rho_0 $ a $R$ e viceversa. 
Derivando l'espressione dell'energia posso ottenere il caso $F = ma$ nel caso
della potenziale della centrifuga:
\begin{gather*}
    \dot{\rho}(m\ddot{\rho} - m\omega^{2}\rho) = 0 
\end{gather*}

\subsection{Il sistema dell'antimolla}
Nel caso di una molla che è fissata al centro di una piattaforma
ruotante con una massa, se vince il contributo della rotazione 
allora si ha una antimolla poiché la massa attaccata alla molla tende ad uscire
dalla piattaforma ruotante vincendo la costante della molla; altrimenti
si ha un moto armonico semplice se $k >> \omega$.
\begin{gather*}
    E = \frac{1}{2}m\dot{\rho^{2} } - \frac{1}{2}m\omega\rho^{2} + \frac{1}{2}k \rho^{2}   
\end{gather*}

\section{Studio dell'equilibrio del sistema}
Nello studio di un sistema con vincoli ideali e forze conservative, allora
lo studio della posizione di equilibrio è relativamente facile.
In un qualsiasi SdR, tenendo conto delle forze apparenti e riconducendoci
dunque al caso ideale, allora un corpo rimane fermo (equilibrio più semplice) quando
la risultante è zero . Se le forze sono conservative allora se la risultante è nulla
vuol dire che in un solo grado di libertà
\begin{gather*}
    \vec{F} = -\vec{\nabla} V = 0  
\end{gather*}
vuol dire che:
\begin{gather*}
    \frac{dV}{dx} = 0
\end{gather*}
\begin{wrapfigure}{r}{0.4\textwidth}
    \centering
    \caption{Pendolo}
    \begin{tikzpicture}
        \draw(0, 0) -- (2, 0);
        \draw(1, 0) -- (1, -2) node[midway, left] {$L$};
        \filldraw(1, -2) circle (1pt)node[anchor = east] {$m$};
    \end{tikzpicture}    
\end{wrapfigure}
Posso impostare il sistema di riferimento di un pendolo dove voglio 
in funzione dell'unico grado di libertà, ossia dell'angolo $\theta$ e quindi
posso identificare col SdR centrato nella massa 
\begin{gather*}
    V = mgz
\end{gather*}
Le energie potenziali $V1$ e $V2$ differiscono di una costante e quindi vanno
bene entrambe e le loro derivate sono proprio:
\begin{gather*}
    V_1 = (L - L\cos\theta)mg \\
    V_2 = -mgL \cos\theta
\end{gather*}
Derivando:
\begin{gather*}
    \frac{dV}{d\theta}-mgL(-\sin\theta) = mgL \sin\theta
\end{gather*}
Allora quando $\theta = 0$ la derivata è zero.


\end{document}
\documentclass[a4paper, oneside]{article}
\usepackage{graphicx}
\usepackage{amsthm}
\usepackage{amsmath}
\usepackage[a4paper,
            bindingoffset=0.2in,
            left=2cm,
            right=2cm,
            top=2cm,
            bottom=2cm,
            footskip=.25in]{geometry}
\usepackage[italian]{babel}
\usepackage{pgfplots}
\usepackage{tabularx}
\usepackage{tikz}
\usepackage{wrapfig}
\usepackage{color}
\definecolor{page}{rgb}{0.129,0.157,0.212}
\pagecolor{page}
\color{white}
\graphicspath{ {./images/} }
\usetikzlibrary{shapes.geometric}
\usetikzlibrary{datavisualization}
\usetikzlibrary{datavisualization.formats.functions}
\pgfplotsset{width=10cm,compat=1.9}

\title{Fisica}
\author{Tommaso Miliani}
\date{01-04-25}

\begin{document}
\theoremstyle{definition}
\theoremstyle{theorem}
\theoremstyle{lemma}

\newtheorem{definition}{Definizione}[section]
\newtheorem{theorem}{Teorema}[section]
\newtheorem{lemma}{Proposizione}[theorem]
\newtheorem{example}{Esempio}[section]

\maketitle


\section{Il teorema di Hoygen Steiner}
\begin{wrapfigure}{r}{0.4\textwidth}
    \centering
    \caption{}
    \begin{tikzpicture}
        \draw(0, 0) circle (2);
        \draw(0, 0) node[anchor = east] {$O$};
        \draw[->](0, 0) -- (3, 0) node[at end, below] {$x$};
        \draw[dashed, ->](0, 0) -- (0, 3) node[at end, right] {$z$};
        \draw[dashed, ->](0, 0) -- (1, 1) node[at end, below] {$y$};
        \filldraw(0, -1) circle (1pt) node[anchor =east] {$C$};
        \filldraw(1, 0) circle (1pt) node[anchor = north] {$A$};
    \end{tikzpicture}    
\end{wrapfigure}
IL teorema collega momenti di inerzia rispetto ad assi paralleli.
Preso un corpo rigido con un dato centro di massa e scelto un asse lungo
tale centro di massa farà in modo che il momento di inerzia si possa
calcolare come:
\begin{gather*}
    I_C = \sum_{i = 1}^{n}m_id_i^{2}  
\end{gather*}
Con la seguente notazione posso esprimere un asse che passa lungo il centro
di massa:
\begin{gather*}
    \hat{C} 
\end{gather*}
Se prendessi un altro punto $A$ che si trova a distanza $d$ dall'asse
passante per C voglio allora calcolarmi il momento di inerzia e ottengo che per
un punto qualsiasi rispetto all'asse è proprio:
\begin{gather*}
    I_{\hat{C} } = \sum_{i = 1}^{n}m_i (x_i^{2} + y_i^{2} ) 
\end{gather*}
Poiché posso considerare che stia su un piano perpendicolare all'asse
e quindi il momento di inerzia del punto A sarà proprio:
\begin{gather*}
    I_{\hat{A} } = \sum m_i((x_i - D)^{2} + y_i^{2} ) = \sum m_i(x_i^{2} + y_i^{2} ) + \sum m_iD^{2} - 2\sum m_ix_iD
\end{gather*}
Il primo termine è proprio il momento di inerzia rispetto al centro di massa, mentre il secondo è
il momento di inerzia totale della massa ed il terzo è il momento di inerzia uguale a zero.
Allora il momento di inerzia tra due assi paralleli è proprio il momento di inerzia del primo
asse più la distanza per la massa del corpo:
\begin{gather*}
    I_{\hat{A} } = I_{\hat{C} } + MD^{2} 
\end{gather*}
E quindi il momento di inerzia minimo è quello che passa per il centro di massa.

\section{Il momento di inerzia rispetto ad un asse}
Si può calcolare il momento di inerzia rispetto ad un dato asse in un corpo
rigido attraverso la seguente (per una figura piana):
\begin{align}
    I_x = \sum_{i = 1}^{n}m_iy_i^{2}  
\end{align}

\subsection{La sbarra omogenea sottile}
\begin{wrapfigure}{r}{0.4\textwidth}
    \centering
    \caption{Sbarra omogenea}
    \begin{tikzpicture}
        \draw(0, 0) rectangle (3, 0.5);
        \draw[|-|](0, -0.5) -- (3, -0.5) node[midway, below] {$x$};
        \draw[->](1.5, 0.25) -- (4, 0.25) node[at end, below] {$x$};
        \draw[->](1.5, 0.25) -- (1.5, 1) node[at end, left] {$y$};
        \draw[|-|](1.5, 0.75) -- (2, 0.75) node[midway, above] {$x$};
        \filldraw(2, 0.75) circle (1pt) node[anchor = west] {$dm$};
    \end{tikzpicture}    
\end{wrapfigure}
Per calcolare il suo momento di inerzia possiamo dividere la sbarra in tanti piccoli pezzi
e dunque il mio momento di inerzia non è più una somma ma un integrale, scelto allora 
un sistema di coordinate ed un asse ortogonale per il centro di massa
si può dire che:
\begin{gather*}
    I = \int dm d^{2} 
\end{gather*}
La massa infinitesima è data da:
\begin{gather*}
    dm = dx \lambda
\end{gather*}
SOstituendo nell'integrale allora:
\begin{gather*}
    I = \int_{-\frac{L}{2}}^{\frac{L}{2}}  dx \lambda x^{2} 
\end{gather*}
E allora il centro di massa di questo oggetto è dato da:
\begin{align}
    I_{\hat{C} } = \frac{1}{12}ML^{2}, \qquad M = \lambda L 
\end{align}
Rispetto ad un asse passante per il bordo il momento di inerzia diventa: 
\begin{gather*}
    I_{\hat{A} } = \lambda \int_{0}^{L} x^{2}dx = \frac{1}{3}ML^{2}  
\end{gather*}

\subsection{Il rettangolo omogeneo}
\begin{wrapfigure}{r}{0.4\textwidth}
    \centering
    \caption{}
    \begin{tikzpicture}
        \draw(0, 0) rectangle (3, 2);
        \draw[|-|](-0.5, 0) -- (-0.5, 2) node[midway, left] {$b$};
        \draw[|-|](0, -0.5) -- (3, -0.5) node[midway, below] {$a$};
        \draw[dashed, ->] (0, 1) -- (4, 1);
        \filldraw(2, 0) rectangle (2.1, 2);
        \draw[dashed, ->](1.5, -1) --(1.5, 3) node[at end, right] {$y$};
        \filldraw (1.5, 1) circle (1pt) node[anchor = east] {$C$};
    \end{tikzpicture}    
\end{wrapfigure}

In questo rettangolo essendo che è omogeneo, la sbarretta infinitesima di cui voglio
calcolare il momento di inerzia posso schiacciarle tutte in un unica sbarrettina e quindi 
è proprio equivalente al caso della sbarretta omogenea di lunghezza $b$ e
massa $M$. Lo stesso vale per il momento di inerzia rispetto all'asse $Y$ considerando le piccole
sbarrette di lunghezza $a$. Dato che il momento di inerzia è additivo, il momento
di inerzia somma rispetto all'asse $z$ è proprio:
\begin{align}
    I_{\hat{z} } = \frac{1}{12}M(a^{2} +b^{2} )
\end{align}

\section{Il parallelepipedo omogeneo}
Dato che il parallelepipedo è omogeneo, posso calcolare il momento di inerzia
come se fosse tutto schiacciato nel caso del rettangolo omogeneo e quindi il suo momento di
inerzia è proprio:
\begin{align}
    I_{\hat{z} } = \frac{1}{12}M(a^{2} + b^{2} )
\end{align}

\subsection{Il cerchio omogeneo}
\begin{wrapfigure}{r}{0.4\textwidth}
    \centering
    \caption{Il momento del cerchio}
    \begin{tikzpicture}
        \draw(0, 0) circle (2);
        \draw(-3, 0) -- (3, 0) node[at end, below] {$x$};
        \draw(0, -3) -- (0, 3) node[at end, right] {$y$}; 
        \draw[|-|](-2, -2) -- (0, -2) node[midway, below] {$R$};
        \draw(0, 0) circle (1.5);
        \draw(0, 0) circle (1.4);
        \filldraw (0.7, 0) circle (0pt) node[anchor = north] {$r$};
    \end{tikzpicture}    
\end{wrapfigure}
Posso considerare per calcolare il suo momento di inerzia una corona
circolare infinitesima e quindi calcolare il momento di inerzia per una
corona spessa $dr$. La superficie infinitesima della corona sarà data da:
\begin{gather*}
    dS = 2\pi r dr. \\
    dm  = \sigma dS
\end{gather*}
Dato che la corona ha un certo spessore $dr$, allora  posso calcolare il momento 
di inerzia per una corona e fare l'integrale unidimensionale per considerare
tutte le corone e ricavare il momento totale rispetto all'asse $z$
\begin{gather*}
    I_z = \int_{0}^{R}\sigma 2 \pi r^{3}dr = \sigma 2\pi \frac{R^{4} }{4}    
\end{gather*}
E dato che la massa totale è proprio $M = \pi R^{2}\sigma$, allora il momento di inerzia
del cerchio omogeneo sarà:
\begin{align}
    I_z = \frac{1}{2}MR^{2} 
\end{align} 
Allora dato che il momento di inerzia è la somma dei momenti di inerzia
rispetto agli altri assi, posso dire che:
\begin{gather*}
    I_z = I_x + I_y
\end{gather*}
E quindi 
\begin{align}
    I_x &= \frac{1}{4}MR^{2}\\
    I_y &= \frac{1}{4}MR^{2}  
\end{align}

\subsection{Il momento rispetto al cilindro}
\begin{wrapfigure}{r}{0.4\textwidth}
    \centering
    \caption{Il cilindro}
    \begin{tikzpicture}
        \draw(0, 1) -- (3, 1);
        \draw(0, -1) -- (3, -1);
        \draw(0, 1) arc(125:235:1.2);
        \draw(3, 1) arc(55:-55:1.2);
        \draw(3, 1) arc(125:235:1.2);
        \draw[->, dashed](-1, 0) -- (4, 0) node[at end, below] {$x$};
        \draw[->, dashed](1.5, -1.5) -- (1.5, 1.5) node[at end, right] {$y$};
    \end{tikzpicture}    
\end{wrapfigure}
Posso considerarlo equivalente al momento del cerchio tutto schiacciato (dato che è 
omogeneo) e quindi la ruota fisica è proprio un cerchio e questo è del tutto
lecito in quanto non cambia considerarlo come un cerchio della massa del cilindro.
Rispetto all'asse $y$ invece la cosa si complica: il momento di inerzia rispetto
a quell'asse è dato da quello del cerchio più il termine di Hoygen Steiner:
\begin{gather*}
    dI = \frac{1}{4}dmR^{2}+ dmx^{2}  
\end{gather*}
ALlora dato che 
\begin{gather*}
    dm = \rho dV = \rho \pi R^{2} dx  
\end{gather*}
E allora il momento di inerzia è esprimibile come integrale unidimensionale
\begin{gather*}
    I = \int_{-\frac{h}{2}}^{\frac{h}{2}}\left(\rho \pi R^{2} dx\left(\frac{R^{2} }{4} + x^{2} \right)\right) 
\end{gather*}
\begin{align}
    I_y = M(\frac{R^{2} }{4} + \frac{h^{2} }{12})
\end{align}

\subsection{La sfera omogenea}
\begin{wrapfigure}{r}{0.4\textwidth}
    \centering
    \caption{}
    \begin{tikzpicture}
        \draw(0, 0) circle (2);
        \draw[->](-3, 0) -- (3, 0);
        \draw[->](0, -3) -- (0, 3);
        \node[ellipse,
        draw,
	    minimum width = 0.1cm, 
	    minimum height = 2.6cm] (e) at (1.5,0) {};
        \draw(0, 0) -- (1.5, 1.25) node[midway, above] {$R$};
        \filldraw(1, 0) circle(0pt) node[anchor = north] {$x$};
        \draw[dashed] (1.5, 0) -- (1.5, 1.25) node[midway, left] {$r$};
    \end{tikzpicture}    
\end{wrapfigure}
\begin{gather*}
    r^{2} = R^{2} - x^{2} \\
    dS = \pi r^{2} = \pi(R^{2} - x^{2}  )    \\
    dV = dx dS = \pi (R^{2} - x^{2}  ) dx \\
    dm = \rho d V
\end{gather*}
Il momento di inerzia rispetto ad un asse (per gli altri assi è uguale):
\begin{gather*}
    I_x = \int \frac{1}{2}dmr^{2} 
\end{gather*}
Per cui sostituendo e risolvendo (Lazy ahh)
\begin{align}
    I_x = \frac{2}{5}MR^{2} 
\end{align}

\end{document}
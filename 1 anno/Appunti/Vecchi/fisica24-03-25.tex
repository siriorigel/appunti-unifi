\documentclass[a4paper, oneside]{article}
\usepackage{graphicx}
\usepackage{amsthm}
\usepackage{amsmath}
\usepackage[a4paper,
            bindingoffset=0.2in,
            left=2cm,
            right=2cm,
            top=2cm,
            bottom=2cm,
            footskip=.25in]{geometry}
\usepackage[italian]{babel}
\usepackage{pgfplots}
\usepackage{tabularx}
\usepackage{tikz}
\usepackage{wrapfig}
\usepackage{color}
\definecolor{page}{rgb}{0.129,0.157,0.212}
\pagecolor{page}
\color{white}
\graphicspath{ {./images/} }
\usetikzlibrary{shapes.geometric}
\usetikzlibrary{datavisualization}
\usetikzlibrary{datavisualization.formats.functions}
\pgfplotsset{width=10cm,compat=1.9}

\title{FIsica}
\author{Tommaso Miliani}
\date{24-03-25}

\begin{document}
\theoremstyle{definition}
\theoremstyle{theorem}
\theoremstyle{lemma}

\newtheorem{definition}{Definizione}[section]
\newtheorem{theorem}{Teorema}[section]
\newtheorem{lemma}{Proposizione}[theorem]
\newtheorem{example}{Esempio}[section]

\maketitle

\section{Il corpo rigido}
E' un oggetto esteso in cui ciascun punto rispetto a qualsiasi sistema
di coordinate è tale per il quale la distanza tra i punti stessi
rimane costane. Il corpo rigido è quindi per definizione:
\begin{gather*}
    |\vec{r}_i - \vec{r}_j| = const  
\end{gather*}
\subsection{La statica del corpo rigido}
Il corpo rigido rimane in equilibrio quando non cambia
la quantità di moto del sistema e quindi:
\begin{gather*}
    \vec{F}^{(EXT)} = 0 \\
    \vec{M}^{(EXT)} = 0    
\end{gather*}
\begin{wrapfigure}{r}{0.4\textwidth}
    \centering
    \caption{parallelepipedo}
    \begin{tikzpicture}
        \draw(0, 0) -- (3, 0);
        \draw(0.5, 0) -- (2, 2);
        \draw(2, 2) -- (3, 2);
        \draw(3, 2) -- (1.5, 0);
        \draw[->](0, 0) -- (0.5, 0) node[at end, below] {$x$};
        \draw[->](0, 0) -- (0, 0.5) node[at end, left] {$y$};
        \draw[->](1, 0) -- (1, 0.5) node[at end, above] {$\vec{N}$};
        \draw[->](1.75, 1) -- (1.75, 0.5) node[at end, right] {$M\vec{g}$};
    \end{tikzpicture}    
\end{wrapfigure}
Immaginando di avere un piano di appoggio e un parallelepipedo storto:
questo oggetto è in equilibrio oppure no? Devo vedere con le cardinali
se esiste un momento torcente per il quale il corpo cade.
Se le dimensioni del corpo sono piccole rispetto alle dimensioni della Terra
allora posso dire con buona approssimazione che l'accelerazione locale
di gravità è costante e parallela per tutti i corpi. Posso allora
considerare la forza peso come concentrata nel centro di massa (baricentro)
e quindi applicata la prima cardinale:
\begin{align*}
    M\vec{g} + \vec{N} &= 0 \\
    -Mg + N &= 0 \\
    N &= Mg  
\end{align*}
Posso allora applicare la seconda cardinale e quindi preso un polo
opportuno, posso allora dire che:
\begin{gather*}
    M_{z}^{(EST)} = Nx - Mgx' 
\end{gather*}
E allora per eguagliarla i bracci devono essere uguali: $x = x'$.
La risultante deve essere sempre applicata sulla base di appoggio. Perché
la differenza dei bracci tra la forza peso e quella normale si annullino,
il centro di massa deve trovarsi lungo la verticale passante per la base di appoggio,
altrimenti per la seconda cardinale non c'è stabilità.

\subsection{La carrucola fissa}
\begin{wrapfigure}{r}{0.4\textwidth}
    \centering
    \caption{Carrucola fissa}
    \begin{tikzpicture}
        \draw(0, 0) circle (1);
        \draw(-1, -2) -- (-1, 0);
        \draw(1, -2) -- (1, 0);
        \draw[->](1, -2) -- (1, -2.5) node[at end, left] {$\vec{F}_2$};
        \draw[->](-1, -2) -- (-1, -2.5) node[at end, left] {$\vec{F}_1$};
        \draw[->](0, 0) -- (0, -0.5) node[at end, right] {$M\vec{g}$};
        \draw[->](0, 0) -- (0, 0.5) node[at end, right] {$\vec{N}$};
    \end{tikzpicture}    
\end{wrapfigure}
NEl caso ideale la carrucola segue la fune, ma nel caso 
reale il movimento della fune non sempre è seguito da quello
della carrucola a meno che non ci sia un attrito molto forte.
Perché siano in equilibrio (trascurando la massa della fune)
le due forze si dovrebbero uguagliare:
\begin{gather*}
    \vec{N} + M\vec{g} + \vec{F}_1 + \vec{F}_2 = 0   
\end{gather*}
Lungo x allora si ha:
\begin{gather*}
    Nx - F_1 \cos\alpha + F_2 \cos\beta = 0 \\
    N_y + Mg - F_1 \sin\alpha - F_2 \sin\beta = 0
\end{gather*}
Dove $\alpha$ e $\beta$ sono gli angoli rispetto all'orizzontale
delle forze (nel disegno sono a novanta). Se scegliessi il polo di applicazione
del momento nel centro della carrucola allora avrei che i momenti sono rappresentati
dalle forze e la corda si stacca lungo la tangente per via di come è fatta e quindi
non c'è bisogno di conoscere gli angoli in quanto basta conoscere il modulo della
forza ed il braccio (la massa è infatti trascurabile).
\begin{gather*}
    \vec{M}_C^{EXT}  = 0 
\end{gather*}
E quindi si ottiene:
\begin{gather*}
    F_1 = F_2 
\end{gather*}
Per stare in equilibrio, senza sapere nessun modulo. 
\subsection{Il problema della scala}
\begin{wrapfigure}{r}{0.4\textwidth}
    \centering
    \caption{Il problema della scala}
    \begin{tikzpicture}
        \draw(0, 0) -- (3, 0);
        \draw(0, 0) -- (0, 3);
        \draw(0, 2) -- (1, 0);
        \draw[->](0, 2) -- (0.75, 3) node[at end, above] {$\vec{N}_1$};
        \draw[->](1, 0) -- (1, 0.75) node[at end, right] {$\vec{N}_2$};
        \draw[->](1, 0) -- (0.25, 0) node[at end, below] {$\vec{F}_d$};
        \draw[->] (0.5, 1) -- (0.5, 0.25) node[at end, left] {$M\vec{g}$};
    \end{tikzpicture}    
\end{wrapfigure}
DAto che si ha attrito statico, io posso dire che i vettori sono
bilanciati così (data la forza di attrito):
\begin{gather*}
    |\vec{F}_a| \leq \mu_s|\vec{N}_2|  
\end{gather*}
\begin{gather*}
    \vec{N}_1 + \vec{N}_2 + M\vec{g} + \vec{F}_a  = 0 \Rightarrow  N_1 = F_a\\
    N_1 - \vec{F}_a = 0  \Rightarrow N_2 = Mg \\
    N_2 - Mg = 0 \Rightarrow N_2 = Mg
\end{gather*}
Con la prima cardinale non sono in grado di determinare se la scala
rimane in equilibrio oppure no, devo scegliere allora un polo di riduzione
per calcolare il momento e quindi dirmi a che angolo $\alpha$ è in equilibrio.
Scegliendo il polo nel punto in cui la scala si appoggia per terra,
posso allora esprimere il momento come:
\begin{gather*}
    (C - B) = \frac{L}{2}(-\sin\alpha \hat{i} + \cos\alpha \hat{j}) \\
    (A - B) = 2(C - B)
\end{gather*}
\begin{gather*}
    \vec{M}_B^{EXT} = (C - B)\times M\vec{g} + (A - B)\times \vec{N}_1  \\
    =  \frac{L}{2}(-\sin\alpha  \hat{i} + \cos\alpha \hat{j} ) \times (-Mg\hat{j} )  + L (-\sin\alpha \hat{i} + \cos\alpha \hat{j}) \times N_1 \hat{i} = \\
    = + \frac{L}{2}\sin\alpha Mg\hat{k} - L \cos\alpha N_1 \hat{k} = 0   
\end{gather*}
Si ottiene allora che:
\begin{gather*}
    N_1 = \frac{1}{2} \tan\alpha Mg
\end{gather*}
Quindi la forza di attrito:
\begin{gather*}
    N_1 = F_a \leq \mu_s N_2 \\
    \tan\alpha \leq 2\mu_s
\end{gather*}
\subsection{La cerniera sferica}
\begin{wrapfigure}{r}{0.4\textwidth}
    \centering
    \caption{}
    \begin{tikzpicture}
        
    \end{tikzpicture}    
\end{wrapfigure}
Voglio sapere le condizioni di equilibrio; l'unico vincolo imposto è
che la forza sia sempre orizzontale e sapere se questo equilibrio è stabile
oppure instabile.
\begin{gather*}
    \vec{F} = F\hat{i} \\
    M\vec{g} = -Mg\hat{j} \\
    \vec{R} = R_x\hat{i} + R_y \hat{j}        
\end{gather*}
Imponendo ora la prima cardinale si ottiene:
\begin{gather*}
    \vec{F} + M\vec{g} + \vec{R} = 0 \\
    F + R_x = 0 \\
    -Mg + R_y = 0 \\
    R_x = F \\
    R_y = Mg   
\end{gather*}
Per la seconda cardinale invece:
\begin{gather*}
    (C - A) \times \vec{g} + (B - A) \times \vec{F} = \frac{L}{2}(\sin\alpha \hat{i} - \cos\alpha \hat{j} ) \times (-Mg\hat{j} ) + L(\sin\alpha \hat{i} + \cos\alpha \hat{j} ) \times F\hat{i}  = \\
    -\frac{L}{2} \sin\alpha Mg\hat{k} + L\cos\alpha F = 0  
\end{gather*}
Allora per essere in equilibrio si deve verificare:
\begin{gather*}
    F = \frac{1}{2}\tan\alpha M g
\end{gather*}
Posso associare una certa energia potenziale a questo sistema e quindi
studiarne non solo la dinamica ma anche se l'equilibrio è stabile oppure
instabile? Potrei allora indicare l'energia potenziale della forza peso:
\begin{gather*}
    V_{peso} = -\frac{L}{2}\cos\alpha Mg
\end{gather*}
Posso trattare la forza $F$ come se fosse la forza di gravità e quindi
ottenere la sua energia potenziale proprio come per la forza peso (dato che è sempre
orizzontale ad ogni istante); indicato allora:
\begin{gather*}
    V_F = - L\sin\alpha Mg
\end{gather*}
L'energia potenziale totale allora è la somma: 
\begin{gather*}
    V = V(\alpha) = -\frac{L}{2}\cos\alpha Mg- L\sin\alpha Mg
\end{gather*}
FAtta allora la derivata e posta uguale a zero:
\begin{gather*}
    \tan\alpha = \frac{2F}{Mg}
\end{gather*}
Facendo allora la derivata seconda ottengo se è stabile o instabile:
\begin{gather*}
    \frac{d^{2} V}{d\alpha^{2} } = \frac{L}{2}\cos\alpha Mg + \sin\alpha L F
\end{gather*}
Calcolata con quello che abbiamo trovato prima e considerato
che $0 < \alpha < \frac{\pi}{2}$ allora si ottiene che la derivata
è sempre maggiore di zero e quindi è un equilibrio stabile. 

\subsection{}
\begin{wrapfigure}{r}{0.4\textwidth}
    \centering
    \caption{}
    \begin{tikzpicture}
        \draw(0, 0) -- (3, 0);



        \draw(5, 0) -- (5.5, 1);
    \end{tikzpicture}    
\end{wrapfigure}
Evitando il movimento ulteriore per il quale si spostano da un lato
le sbarrette oppure quando l'angolo tra i due è superiore di $180$°,
per ogni sbarretta ci sono delle forze che agiscono su di loro, tra cui 
la forza peso, la stessa forza elastica.
Applicando  

\end{document}
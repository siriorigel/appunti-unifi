\documentclass[a4paper, oneside]{article}
\usepackage{graphicx}
\usepackage{amsthm}
\usepackage{amsmath}
\usepackage[a4paper,
            bindingoffset=0.2in,
            left=2cm,
            right=2cm,
            top=2cm,
            bottom=2cm,
            footskip=.25in]{geometry}
\usepackage[italian]{babel}
\usepackage{pgfplots}
\usepackage{tabularx}
\usepackage{wrapfig}
\graphicspath{ {./images/} }
\usetikzlibrary{datavisualization}
\usetikzlibrary{datavisualization.formats.functions}
\pgfplotsset{width=10cm,compat=1.9}

\title{Geometria(RUBEI)}
\author{Tommaso Miliani}
\date{28-02-25}

\begin{document}
\theoremstyle{definition}
\theoremstyle{theorem}
\theoremstyle{lemma}

\newtheorem{definition}{Definizione}[section]
\newtheorem{theorem}{Teorema}[section]
\newtheorem{lemma}{Proposizione}[theorem]

\maketitle

\section{Sottospazi affini}
\begin{definition}[Sottospazi affini]
    Sia $K^{n}$ un campo,allora sia $n \in N - \{0\}$. SIa
    $S$ un sottospazio di $K^{n}$. Si dice che $S$ è un sottospazio
    affine se $\exists v \in K^{n} \wedge \exists Z \subset K^{n}$ tale che:
    \begin{align}
        S = v + Z.
    \end{align}   
    Dove 
    \begin{gather*}
        v + Z = \{v + z : z \in Z\}
    \end{gather*} 
    Un sottospazio affine è un sottospazio traslato
    rispetto al sottospazio di uno spazio qualsiasi. 
\end{definition}
\begin{definition}[Direzione del sottospazio]
    Il sottospazio dal quale si ricava il sottospazio affine
    $S$ si chiama \textbf{direzione} di $S$, o si scrive anche:
    $Z = dir(S)$ ed è univocamente determinata.
\end{definition}
\begin{lemma}
    La direzione è univocamente determinata
\end{lemma}
\begin{proof}
    Sia $S$ un sottospazio affine con direzione $Z$  quindi:
    \begin{gather*}
        S = v + Z e S = v' + Z'
    \end{gather*}
    Con $v, v' \in K^{n}$ e quindi essendo $z, z' \in K^{n}$, si ha che:
    \begin{gather*}
        \{v + z : z \in Z\} = \{v' + z' : z' \in Z\}
    \end{gather*}  
    In particolare in questo insieme ci sta proprio $v$ e quindi se sta in questo 
    insieme allora sta anche nell'altro e quindi $\exists z' \in Z' : v = v' = z' \in Z'$.
    e allora: $v - v' = z' \in Z'$. Allora $\exists z \in Z : v' = v + z$ e quindi
    $v - v' = z \in Z$. Ho dimostrato che $v - v'$ ed il suo opposto $\in Z, Z'$
    allora devo dimostrare che $Z \subseteq Z' \wedge Z' \subseteq Z$. \\
    PReso $t \in Z$, io so che $ v + t \in v + Z = v' + Z'$, esiste
    allora $v + t \in v' + Z'$ e quindi $\exists t' \in Z'$ tale che $v + t = v' + t'$
    ALlora posso scrivere:
    \begin{gather*}
        t = v' - v + t'
    \end{gather*} 
    ALlora entrambi i pezzi del secondo membro sono in $Z'$ ed essendo chiuso
    per la somma, $t \in Z'$. Allora è dimostrato che $Z \subseteq Z' \wedge Z' \subseteq Z$ e
    allora $Z = Z'$. 
\end{proof}
\begin{definition}[DImensione del sottospazio affine]
    La dimensione di un sottospazio affine è proprio la dimensione
    della direzione del sottospazio affine.
\end{definition}
Esempio:
\begin{gather*}
    r = \left\{ \begin{pmatrix} 1 \\
    2 \end{pmatrix} + t \begin{pmatrix} 1 \\
    1\end{pmatrix} : t \in R \right\}
\end{gather*}
Allora questo sottospazio sarà proprio la retta che passa dal punto
$(1, 2)$ e tutti i punti della retta sono ottenuti sommando come multiplo di t
che diventa allora la sua direzione. 
\begin{definition}
    UN sottospazio affine di dimensione $1$ è una retta, di dimensione $2$ 
    è piano, di dimensione $n - 1$ si dice iperpiano. 
\end{definition}

\begin{lemma}
    SIa $K$ un campo, e sia $n \in N$, e sia $Z$ un sottospazio vettoriale
    di $K^{n}$ di dimensione $r$, allora $\exists A \in M((n - r) \times n, K) : Z
    = \{x \in K^{m} : Ax = 0 \}$. \\
    Sia S un sottospazio affine di $K^{n}$ di dimensione $r$
    allora $\exists A \in M((n - r) \times n, K)$ di rango $n -r$
    e $\exists b \in K^{m - n}$ tale che:
    \begin{gather*}
        S = \{x \in K^{n} : Ax = b \}
    \end{gather*}  
\end{lemma}
\begin{proof}
    Dato un sottospazio vettoriale $Z$ di $K^{n}$, allora questo
    è lo span di $r$ vettori indipendenti  e sia $\left<Z_1, \dots, Z_r\right>$ una base
    di $Z$. Creando la trasposta delle righe si ottiene una matrice che ha quindi 
    un formato $r \times n$ di rango $r$. \\
    SI considera $L = \{x \in K^{n} \} : (z_i, x) = 0$  ossia il prodotto
    scalare diventa l'insieme delle soluzioni di $Ax = 0$ ed in pratica la trasposta 
    di $z_1 \cdot  x$ e così via e quindi il suo prodotto scalare è zero se e solo
    $x = 0$. \\
    Quindi essendo $\dim(L) = n - r$:
    \begin{gather*}
        L = \left< l_1 , \dots, l_{n - r}\right>
    \end{gather*}  
    Posso chiamare $A'$ la matrice ottenuta trasponendo tutte queste soluzioni
    e diventa quindi una matrice di formato $n - r \times n$ e di rango $n - r$. 
    COnsiderato l'insieme 
    \begin{gather*}
        U = \{x \in K^{n} : A' x = 0 \}
    \end{gather*}
    Ossia l'insieme dei prodotti scalari con gli elementi di $L$ e quindi fare
    il prodotto e quindi:
    \begin{gather*}
        \dim(U) = rk col(A') - rk(A') = r
    \end{gather*} 
    SI osserva inoltre che $Z \subset  U$ poiché se prendessi un
    elemento qualsiasi di $Z$ come combinazione lineare degli elementi di Z
    e questo è uguale a qualsiasi elemento di $L$, essendo che $\dim(Z) = r$
    allora si ha proprio:
    \begin{gather*}
        U = Z 
    \end{gather*}
    Si è provato allora che $Z$ è l'insieme delle soluzioni di un sistema
    omogeneo. Proviamo la seconda parte del lemma:
    Sia $S$ un sottospazio affine di $Z$, allora voglio dimostrare che sia 
    proprio l'insieme delle soluzioni di un sistema lineare:
    \begin{gather*}
        S = \{x \in K^{n} : \exists z \in Z : x = v + z\} \\
         = \{x \in K^{n} : x - v \in Z\}
    \end{gather*} 
    Per la prima parte della proposizione si sa che $Z$ è proprio l'insieme
    delle soluzioni e quindi :
    \begin{gather*}
        S = \{x \in K^{n} : A(x - v) = 0\}
    \end{gather*}
    Quindi posso vedere che $Ax = Av$ e quindi, posto $b = Av$ si 
    vede che S è l'insieme delle soluzioni di un sistema lineare. 
\end{proof}
Posso vedere quindi i sottospazi affini come le soluzioni
dei sistemi lineari con le matrici
Si ha allora un corollario:
\begin{lemma}
    Sia $s$ una retta in $R^{3}$ e quindi un sottospazio affine
    di dimensione $1$ allora c'è una matrice tale che $s$ è l'insieme
    delle soluzione dell'insieme $Ax = b$  e allora $\exists A \in M(2 \times3, R)$
    ed $\exists b \in R^{2}$.
    \begin{gather*}
        \exists \alpha, \beta, \gamma, \alpha', \beta', \gamma', \delta, \delta' : rk\left(\begin{tabular}{c c c}
            $\alpha$ & $\beta$ & $\gamma$ \\
            $\alpha'$ & $\beta'$ & $\gamma'$
        \end{tabular}\right) = 2 \Rightarrow  \\
        s = \{x \in R^{3} : \begin{array}{c}
            \alpha x_1 + \beta x_2 + \gamma x_3 = \delta \\
            \alpha' x_1 + \beta' x_2 + \gamma' x_3 = \delta' \\
        \end{array}\}
    \end{gather*} 

    Sia $\Pi$ un piano din $R^{3}$, quindi $\exists A \in M(2 \times 3, R)$
    di rango 1 e $\exists b \in R^{1} :$
    \begin{gather*}
        \Pi = \{x \in R^{3} : Ax = b\}
    \end{gather*}  
    In pratica quest mi da un equazione sola:
    \begin{gather*}
        \Pi = \{x \in R^{3} : \alpha x_1  \beta x_2 + \gamma x_3\}
    \end{gather*}
    Almeno uno dei tre deve essere diverso da zero. 
\end{lemma}

\end{document}
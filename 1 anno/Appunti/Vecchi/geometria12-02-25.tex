\documentclass[a4paper, oneside]{article}
\usepackage{graphicx}
\usepackage{amsthm}
\usepackage{amsmath}
\usepackage[a4paper,
            bindingoffset=0.2in,
            left=2cm,
            right=2cm,
            top=2cm,
            bottom=2cm,
            footskip=.25in]{geometry}
\usepackage[italian]{babel}
\usepackage{pgfplots}
\usepackage{tabularx}
\usepackage{wrapfig}
\graphicspath{ {./images/} }
\usetikzlibrary{datavisualization}
\usetikzlibrary{datavisualization.formats.functions}
\pgfplotsset{width=10cm,compat=1.9}

\title{Geometria}
\author{Tommaso Miliani}
\date{12-03-25}

\begin{document}
\theoremstyle{definition}
\theoremstyle{theorem}
\theoremstyle{lemma}

\newtheorem{definition}{Definizione}[section]
\newtheorem{theorem}{Teorema}[section]
\newtheorem{lemma}{Proposizione}[theorem]

\maketitle

\section{AUtovettori e eautovalroti}
\begin{definition}
    IL polinomio \textbf{caratteristico} di una matrice $A \in M(n \times n, K)$ è il polinomio
    definito come 
    \begin{align}
        P_A(t) = \det(A - tI_n)
    \end{align}
\end{definition}

\begin{example}
    \begin{gather*}
        A = \left( \begin{array}{c c c}
            2 & 0 & 0 \\
            0 & 7 & 0 \\
            0 & 0 & 7
        \end{array} \right)
    \end{gather*}
    ALlora il polinomio caratteristico:
    \begin{gather*}
        P_A(t) = \det\left( \begin{array}{c c c}
            2 - t & 0 & 0 \\
            0 & 7 - t& 0 \\
            0 & 0 & 7 - t
        \end{array} \right) = (2 - t)(7 - t)^{2} 
    \end{gather*}
\end{example}
\begin{example}
    \begin{gather*}
        B = \left( \begin{array}{c c c c}
            1 & 2 & 0 & 0 \\
            0 & 3 & 0 & 0 \\
            0 & 0 & 4 & 5 \\
            0 & 0 & 5 & 0            
        \end{array} \right)
    \end{gather*}
    Il suo polinomio caratteristico è proprio:
    \begin{gather*}
        P_B(t) = \det\left( \begin{array}{c c c c}
            1 - t & 2 & 0 & 0 \\
            0 & 3 - t & 0 & 0 \\
            0 & 0 & 4 - t & 5 \\
            0 & 0 & 5 & 0            
        \end{array} \right) = ((4 - t)(-t) - (25))(3- t)(1 - t)
    \end{gather*}
\end{example}

\begin{definition}
    Sia $V$ uno spazio vettoriale di dimensioni finite e sia
    la funzione $f : V \to V$ un'applicazione lineare. Definisco
    allora il polinomio caratteristico dell'applicazione lineare come:
    \begin{align}
        P_f(t) = \det(M_{B, B} (f) - tI_n)  
    \end{align}
\end{definition}

\begin{definition}[Matrici simili]
    Siano $A$ e $B$ due matrici quadrate si dicono \textbf{simili}
    se esiste una matrice invertibile $C \in GL(n, K)$ tale che:
    \begin{align}
        A = C^{-1} BC 
    \end{align}
\end{definition}

\begin{lemma}
    LA definizione del polinomio caratteristico della funzione
    non dipende dalla scelta di B di V.
\end{lemma}
\begin{proof}
    PRese $B$ e $B'$ due basi di V allora io so che:
    \begin{gather*}
        P_{M_{B, B}(f)} = P_{M_{B', B'}(f)}
    \end{gather*}
    Dimostriamo allora che matrici simili hanno lo stesso polinomio
    caratteristico
\end{proof}

\end{document}
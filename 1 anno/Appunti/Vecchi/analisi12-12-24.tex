\documentclass[a4paper, oneside]{article}
\usepackage{graphicx}
\usepackage{amsthm}
\usepackage{amsmath}
\usepackage[a4paper,
            bindingoffset=0.2in,
            left=2cm,
            right=2cm,
            top=2cm,
            bottom=2cm,
            footskip=.25in]{geometry}
\usepackage[italian]{babel}
\usepackage{pgfplots}
\usepackage{tabularx}
\usepackage{wrapfig}
\graphicspath{ {./images/} }
\usetikzlibrary{datavisualization}
\usetikzlibrary{datavisualization.formats.functions}
\pgfplotsset{width=10cm,compat=1.9}

\title{Analisi}
\author{Tommaso Miliani}
\date{12-12-24}

\begin{document}
\theoremstyle{definition}
\theoremstyle{theorem}
\theoremstyle{lemma}

\newtheorem{definition}{Definizione}[section]
\newtheorem{theorem}{Teorema}[section]
\newtheorem{lemma}{Proposizione}[theorem]

\maketitle

\section{Proseguo serie}
Esempi di serie convergenti e divergenti:
Divergenti:
\begin{gather*}
    a_n = 1 \ \forall n \Rightarrow \lim_{N \to +\infty } \sum_{n = 1}^{N } 1 = +\infty  
\end{gather*}
\begin{gather*}
    a_n = n \Rightarrow \sum_{n = 1}^{+\infty } n \\
    S_n = 1 + 2 + \dots + n \Rightarrow \frac{n(n+1)}{2} \\
    \lim_{N \to +\infty} S_N = \lim_{N \to +\infty } \frac{n(n + 1)}{2} = +\infty.   
\end{gather*}
Indeterminate (devo scegliere una successione che  cambi segno,
altrimenti o converge o diverge):
\begin{gather*}
    a_n = (-1)^{n} = \left\{\begin{array}{c}
        1, \ con \ n \ pari \\
        -1, \ con \ n \ dispari
    \end{array}\right. \\
    \Rightarrow  \sum_{n = 1}^{+\infty } a_n = indeterminata  
\end{gather*}
Se ne sommo un numero pari ottengo 0, se sommo un numero dispari di
numeri allora si ha -1, e quindi $\not\exists \lim_{N \to +\infty } S_N$. 
Convergente:
\begin{gather*}
    a_n = 0 \ \forall n \in N \Rightarrow  \sum_{n = 1}^{\infty } a_n \\
    S_0 = S_1 = \dots = 0 \\
    S_N = 0 \ \forall N. 
\end{gather*}
Un'altro esempio convergente non banale (che parte da zero poiché è definita pure in zero):
\begin{gather*}
    \sum_{n = 0}^{\infty }\frac{1}{2^{n} } = 2. 
\end{gather*}
PEr risolvere gli esercizi ci sono due serie di riferimento: quelle armoniche e
quelle geometriche che si possono utilizzare:

\section{Serie geometriche}
QUeste sono delle serie della forma:
\begin{align}
    \sum_{n = 0}^{\infty } q^{n}, \qquad q \in R  
\end{align}
Il termine $q$ è anche chiamato ragione della serie: sto sommando un numero
reale che io ho scelto all'inizio e lo sommo infinite volte:
Fissato un numero $N \in N$ molto grande, allora considero:
\begin{gather*}
    (1 + q + \dots + q^{N})(1 - q) \\
    \Rightarrow  1 - q^{N + 1} 
\end{gather*}
Allora si ha che: 
\begin{gather*}
    1 + q + \dots + q^{N} = \frac{1 - q^{N +1 } }{1 - q} 
\end{gather*}
HA quindi come somme parziali:
\begin{gather*}
    S_N =  \frac{1 - q^{N +1 } }{1 - q} , \qquad \ \forall N \in N, q \ne 1.
\end{gather*}
Adesso passando al limite per $N \to \infty $ si ottiene:
\begin{gather*}
    \lim_{N \to +\infty } q^{N+ 1} 
\end{gather*}
Adesso bisogna discutere il limite al variare del parametro q:
\begin{gather*}
   \\lim_{N \to +\infty } q^{N+ 1} = \left\{ \begin{array}{c}
        + \infty , se q > 1 \\
        1 \ se \ q = 1 \\
        0 \ se \ q < |1| \\
        \not \exists \ se \ q \leq -1
    \end{array} \right.
\end{gather*}
Adesso torando alle somme parziali si ottiene il seguente risultato:
\begin{gather*}
    \lim_{N \to +\infty } S_N = \lim_{N \to +\infty } \frac{1 - q^{N + 1} }{1 - q} =
    \left\{ \begin{array}{c}
        + \infty , \ se \ q > 1 \\
        \frac{1}{1 - q} \ se \ q < |1| \\
        \not \exists \ se \ q \leq -1
    \end{array} \right.  
\end{gather*}
ALlora tornando alla serie di partenza:
\begin{gather*}
    \sum_{n = 0}^{\infty } q^{n}  = \left\{ \begin{array}{c}
        diverge , \ se \ q > 1 \\
        diverge \ se \ q = 1 \\
        converge \ a \ \frac{1}{1 - q} \ se \ q < |1| \\
        \not \exists \ se \ q \leq -1
    \end{array} \right.
\end{gather*}
Altra osservazione: noi sappiamo che $1,\bar{1} = \sum_{n = 0}^{+\infty } \left(\frac{1}{10}\right)^{n}$ posto
allora $ q =  1/10$ si ha proprio la sommatoria geometrica generata prima. E ci permette di ottenenere
la frazione generatrice di tutti i numeri periodici: 
\begin{gather*}
    \sum_{n = 0}^{+\infty } q^{n} = \frac{10}{9}  
\end{gather*}   

\section{Serie armoniche}
Una serie armonica è una serie che somma infiniti numeri molto piccoli che
però divergono.
\begin{align}
    \sum_{n = 1}^{+\infty } \frac{1}{n} \ diverge 
\end{align}
\begin{proof}
    Posta la somma parziale:
    \begin{gather*}
        S_N = 1 + \frac{1}{2} + \dots + \frac{1}{n}
    \end{gather*}
    Osservo che questa cresce al crescere di N ed il motivo è che sto sommando
    termini positivi e quindi si ha che $S_{N+1} \geq S_N$. Supponiamo ora per assurdo
    che $\lim_{N \to +\infty } S_N = S \in R$ e che quindi converga: allora 
    prendo una somma che ha come indice il doppio di un numero naturale: $S_{2N}$
    allora ho $2N$ addendi e se guardo alla somma" dei primi N addendi si ha esattamente $S_N$
    allora $S_{2N} = S_N + \frac{1}{N+1} + \dots +\frac{1}{2N}$. Voglio dimostrare ora
    che questa differenza è costante: allora la disuguaglianza sopra è dimostrata e quindi 
    $S_{2N} \geq S_N + \frac{1}{2}, \forall N$, allora siccome ho assunto che $S_N$ tenda ad
    S, passado al limite ottengo $S \geq S + \frac{1}{2}$ allora si ha un assurdo.
    La serie è dunque divergente.
\end{proof}

\section{I criteri delle serie}
\begin{theorem}[Criterio del confronto]
    Nelle serie generate da due successioni tali che: $0 \leq a_n \leq b_n$
    Se allora diverge $a_n$ allora anche la serie di $b_n$ diverge. Se invece
    converge la serei grande ($b_n$) allora converge anche la piccola.
\end{theorem}


\begin{theorem}[Criterio del confronto asintotico]
    Siano $a_n, b_n \geq 0$, $b_n > 0$ allora
    \begin{align}
        \exists \lim_{n \to +\infty } \frac{a_n}{b_n} = L \in (0, +\infty ), L \neq 0. 
    \end{align}
    E quindi 
    \begin{gather*}
        \sum_{n = 1}^{+\infty } a_n \quad e \quad \sum_{n = 1}^{+\infty } b_n  
    \end{gather*}
    hanno lo stesso carattere.
\end{theorem}

Esempio: detereminare il carattere della serie
\begin{gather*}
    \sum_{n = 0}^{+\infty } \log(1 + \frac{1}{2^{n} }) \\
    a_n =  \log(1 + \frac{1}{2^{n} }), b_n = \frac{1}{2^{n} } \Rightarrow  \\
    \lim_{n \to +\infty } \frac{a_n}{b_n} = 1  
\end{gather*}
Allora hanno lo stesso carattere ed esiste finito il limite della
serie di partenza.


\begin{theorem}[Criterio della radice]
    Posta la serie $a_n$ con $n \in N$ allora se:
    \begin{align}
        \exists \lim_{n \to +\infty }  \sqrt{n}{a_n} = L 
    \end{align}
    Se $L > 1$ la serie diverge, altrimenti converge.
\end{theorem}


\begin{theorem}[Criterio del rapporto]
    Se
    \begin{align}
        \exists \lim_{n \to +\infty } \frac{a_{n + 1}}{a_n} = L 
    \end{align}
    Se $L > 1$ la serie diverge, altrimenti converge.
\end{theorem}

Esempio: 
\begin{gather*}
    \sum_{n = 0}^{+\infty } \frac{1}{n!}. 
\end{gather*}
Scelta allora la successione
\begin{gather*}
    a_n = \frac{1}{n!} > 0
\end{gather*}
Applicando il criterio del rapporto si ottiene che:
\begin{gather*}
    \frac{a_{n + 1}}{a_n} = \frac{n!}{(n + 1)!}  = \frac{1}{n + 1} \\
    \lim_{n \to +\infty } \frac{1}{n + 1} = 0, L < 1. 
\end{gather*}
La serie quindi converge (Potevo utilizzare la serie armonica
$\frac{1}{n^{2}}$ e grazie al criterio del confronto si otteneva
la convergenza)


\begin{theorem}[Criterio dell'integrale]
    Sia una funzione definita in $f [1, +\infty ] $ continua e decrescente.
    Ponendo $a_n = f(x) \ \forall n \in N$. Quindi:
    \begin{gather*}
        \int_{1}^{+\infty } f(x)  
    \end{gather*}
    E la serie: 
    \begin{gather*}
        \sum_{n = 1}^{+\infty } a_n  \quad \left(\sum_{n = 1}^{+\infty } f(x )\right)
    \end{gather*}
    Hanno lo stesso carattere.
\end{theorem}

\end{document}
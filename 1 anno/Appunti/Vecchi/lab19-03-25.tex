\documentclass[a4paper, oneside]{article}
\usepackage{graphicx}
\usepackage{amsthm}
\usepackage{amsmath}
\usepackage[a4paper,
            bindingoffset=0.2in,
            left=2cm,
            right=2cm,
            top=2cm,
            bottom=2cm,
            footskip=.25in]{geometry}
\usepackage[italian]{babel}
\usepackage{pgfplots}
\usepackage{tabularx}
\usepackage{tikz}
\usepackage{wrapfig}
\usepackage{color}
\definecolor{page}{rgb}{0.129,0.157,0.212}
\pagecolor{page}
\color{white}
\graphicspath{ {./images/} }
\usetikzlibrary{shapes.geometric}
\usetikzlibrary{datavisualization}
\usetikzlibrary{datavisualization.formats.functions}
\pgfplotsset{width=10cm,compat=1.9}

\title{Lab (zoccols)}
\author{Tommaso Miliani}
\date{19-03-25}

\begin{document}
\theoremstyle{definition}
\theoremstyle{theorem}
\theoremstyle{lemma}

\newtheorem{definition}{Definizione}[section]
\newtheorem{theorem}{Teorema}[section]
\newtheorem{lemma}{Proposizione}[theorem]
\newtheorem{example}{Esempio}[section]

\maketitle

\section{Le approssimazioni nell'esperimento del pendolo}
\begin{wrapfigure}{r}{0.4\textwidth}
    \centering
    \caption{Il pendolo ideale}
    \begin{tikzpicture}
        \draw(0, 0) -- (0, -2) node[at end, below] {$\phi_0 = 0$};
        \draw(0, 0) -- (0.75, -1.8);
        \draw[dashed](0.75, -1.75) arc (-60:-120:1.5) node[at start, below] {$C$};
        \draw(0, -1) arc (-90:-60:0.75) node[midway, below] {$\phi_0$}; 
    \end{tikzpicture}    
\end{wrapfigure}
Massa puntiforme $R << l = \bar{OC}$ con il filo ideale:
\begin{enumerate}
    \item $m_f =  0$, $m_f << m$;
    \item Inestensibilità: $l$ costante;
    \item Perfetta flessibilità;
    \item Seguendo le leggi di un pendolo semplice ideale
\end{enumerate}
La legge del pendolo semplice che segue è proprio:
\begin{align}
    \ddot{\phi} + \frac{g}{l}\sin\phi = 0
\end{align}
E quindi:
\begin{gather*}
    \dot{\phi}^{2} = 2\frac{g}{l}(\cos\phi - \cos\phi_0) 
\end{gather*}
Per cui il suo periodo è proprio:
\begin{gather*}
    T_0 = 2\pi\sqrt{\frac{l}{g}} 
\end{gather*}
Per ottenere l'angolo bisogna integrare
l'espressione di $\dot{\phi}$. Si separa allora $dt$ e quindi si ottiene
come primo passaggio:
\begin{gather*}
    \frac{d\phi}{\sqrt{2(\cos\phi - \cos\phi_0)}} = \left(\frac{g}{l}\right)^{\frac{1}{2}}dt 
\end{gather*}
\begin{gather*}
    \left(\frac{g}{l}\right)^{\frac{1}{2}} \int_{\frac{3T}{4}}^{T} dt = \int_{0}^{\phi_0} \frac{d\phi}{\sqrt{2(\cos\phi - \cos\phi_0)}} 
    = \left(\frac{g}{l}\right)^{\frac{1}{2}}\frac{T}{4} 
\end{gather*}
Quando $\phi \to \theta$ allora si può esprimere il dimezzamenteo del seno di $\phi$:
\begin{gather*}
    \sin\frac{\phi}{2} = \sin\frac{\phi_0}{2} + \sin\theta
\end{gather*}
Allora possiamo esprimere l'integrale come:
\begin{gather*}
    \int_{0}^{\frac{\pi}{2}}\frac{d\theta}{(1 - \sin^{2}\frac{\phi}{2} \sin^{2}\theta)^{\frac{1}{2}} } = K(\sin^{2}\frac{\phi_0}{2} ) 
\end{gather*}
Dove $K$ è una costante che compare nella risoluzione dell'integrale
e dato che $\sin^{2}\frac{\phi_0}{2} < 1$ per piccole oscillazioni.  Allora 
E quindi il suo periodo è una funzione per cui è esprimibile come:
\begin{gather*}
    T = T(l, g, \phi_0) = \left(2\pi\sqrt{\frac{l}{g}} \right) \cdot  f(\phi_0) = T_0(l, g) \cdot  f(\phi_0) \\
    10\text{°} \leq \phi_0 \leq 25\text{°}
\end{gather*} 
\begin{gather*}
    \begin{tabular}{c | c | c | c}
        $\phi_0$ & $\phi_0(rad)$ & $\frac{1}{4}\sin^{2}\frac{\phi_0}{2}$ & $\frac{\phi}{64}\sin^{4}\frac{\phi_0}{2}$ \\
        \hline 
        $10\text{°}$ & $0.125$ & $1.9 \cdot 10^{-3} $ & $8 \cdot 10^{-6}  $ \\
        $20\text{°}$ & $0.349$ & $7.5 \cdot 10^{-3}$ & $1.3 \cdot 10^{-4}$ \\
        $25\text{°}$ & $0.436$ & $1.17 \cdot 10^{-2}$ & $3.4 \cdot 10^{-4} $ 
    \end{tabular}
\end{gather*}
AI fini della misura data l'incertezza con cui misuro il periodo
con le piccole oscillazioni o con le oscillazioni generali è indifferente
i due metodi:
\begin{align}
    \left( \frac{\delta T}{T} \right) &\approx \frac{1}{4}\sin^{2}\frac{\phi}{2} << \left(\frac{\Delta T}{T}\right) \\
    \left( \frac{\delta T}{T} \right) &\approx \frac{1}{4}\sin^{2}\frac{\phi_0}{2} \geq \left(\frac{\Delta T}{T}\right) 
\end{align}
\begin{wrapfigure}{r}{0.4\textwidth}
    \centering
    \caption{Il pendolo}
    \begin{tikzpicture}
        \draw[dashed](0, 0) -- (0, -2) node[at start, above] {$O$};
        \draw(-1, 0) -- (1, 0);
        \draw(0, 0) -- (1, -2) node[midway, right] {$l$};
        \draw(0, -1) arc (-90:-60:1) node[midway, below] {$\phi_0$};
        \draw(0, -2) -- (1, -2) node[midway, above] {$x_0$};
        \draw()
    \end{tikzpicture}    
\end{wrapfigure} 

\end{document}
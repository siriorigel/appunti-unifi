\documentclass[a4paper, oneside]{article}
\usepackage{wrapfig}
\usepackage{graphicx}
\usepackage{amsthm}
\usepackage{amsmath}
\usepackage{amssymb}
\usepackage[a4paper,
            bindingoffset=0.2in,
            left=2cm,
            right=2cm,
            top=2cm,
            bottom=2cm,
            footskip=.25in]{geometry}
\usepackage[italian]{babel}
\usepackage{pgfplots}
\usepackage{tabularx}
\usepackage{tikz-3dplot}
\usepackage{wrapfig}
\usepackage{color}
\usepackage{multicol}
\usepackage{arydshln}
\usepackage{mathtools}
\usepackage{enumerate}
\usepackage{graphicx}
\usepackage{svg}
\usepackage{cancel}
\usepackage[d]{esvect}
\usepackage[dvipsnames]{xcolor}
\usepackage{pgfplots}
\usepackage{pifont}
\usetikzlibrary{patterns}
\makeindex
%\usepackage{animate}
%\usepackage{xfp} % utile se vuoi fare calcoli aggiuntivi
\pgfplotsset{compat=1.18}
\usetikzlibrary{tikzmark}
\newcommand{\TikzNCbar}[4][10pt]{
\tikz[overlay,remember picture]{\draw[#2] (#3) --++(0,-#1) -| (#4);}}

\graphicspath{ {images/} }

\definecolor{redish}{rgb}{255, 0, 30}
\definecolor{page}{rgb}{0.129,0.157,0.212}
\pagecolor{page}
\color{white}   
\graphicspath{ {./images/} }
\usetikzlibrary{shapes.geometric}
\usetikzlibrary{datavisualization}
\usetikzlibrary{datavisualization.formats.functions}
\pgfplotsset{width=10cm,compat=1.9}

\setlength\dashlinedash{0.2pt}
\setlength\dashlinegap{1.5pt}
\setlength\arrayrulewidth{0.3pt}

\newcommand\eqq{\stackrel{\mathclap{\normalfont\mbox{?}}}{=}}
\newcommand\bulletout  {\labelitemfont \textbullet}
\newcommand{\tab}{\hspace*{2em}}
\newcommand{\xmark}{
\tikz[scale=0.23] {
    \draw[line width=0.7,line cap=round] (0,0) to [bend left=6] (1,1);
    \draw[line width=0.7,line cap=round] (0.2,0.95) to [bend right=3] (0.8,0.05);
}}
\newcommand{\cmark}{
\tikz[scale=0.23] {
    \draw[line width=0.7,line cap=round] (0.25,0) to [bend left=10] (1,1);
    \draw[line width=0.8,line cap=round] (0,0.35) to [bend right=1] (0.23,0);
}}
% Comando:
%   \potato[opzioni]{(x,y)}{scala}
%
% Opzioni = facoltative (es. fill=red!20, draw=black, thick)
% (x,y)   = centro della patata
% scala   = fattore di scala
%
\def\potatoshape{
  (1,0) (2,1.5) (1.6,3) (0.3,2.7) (-0.4,1.2)
}
\newcommand{\potato}[3][draw=white]{
  \begin{scope}[shift={#2}, scale=#3]
    \draw[#1]
      plot [smooth cycle, tension=1]
      coordinates {\potatoshape};
  \end{scope}
}
 \newcommand{\hookbox}[1]{
\begin{center}
\hfill\break
\begin{tikzpicture}
\node[inner sep=0pt,outer sep=0pt,anchor=base] (A) {
\begin{minipage}{\dimexpr\linewidth-5em}
\centering
#1
\end{minipage}
};
% Draw the left bracket
\draw ([xshift=0pt]A.north west) -- ++(0, 0.5) -- ++(0.4, 0);
% Draw the right bracket
\draw ([xshift=0pt]A.south east) -- ++(0, -0.5) -- ++(-0.4, 0);
\end{tikzpicture}
\end{center}} 
\title{Meccanica Analitica}
\author{Gariboldi Alessandro}
\date{ }


\begin{document}

\newtheoremstyle{theoremEnv}
                {}          % Space above
                {}          % Space below
                {\slshape}  % Body font
                {}          % Indent amount
                {\bfseries} % Head font
                {.}         % Punctuation after head
                {\newline}         % Space after theorem head
                {}          % Theorem head spec
\theoremstyle{theoremEnv}

\newtheorem{definition}{Definizione}[section]
\newtheorem{theorem}{Teorema}[section]
\newtheorem{lemma}{Lemma}[section]
\newtheorem{observation}{Oss.}[section]
\newtheorem{corollary}{Corollario}[theorem]
\newtheorem{example}{Esempio}[section]
\newtheorem{problem}{Problema}[section]
\newtheorem{solution}{Soluzione}[section]
\newtheorem{proposition}{Proposizione}[section]


\maketitle
\section{24/02/26}
\subsection{Introduzione al corso}
Organizzazione d'esame:
\begin{itemize}
    \item ci sono i parziali (uno a metà corso e uno alla fine) di cui uno recuperabile
    \item c'è un totale al posto dei compitini
    \item c'è un orale al quale si entra con la media aritmetica dei parziali (possono esserci esercizi ma chiaramente piu torico)
    \item appunti liberi all'esame
\end{itemize}
Materiale del corso:
\begin{itemize}
    \item Libri consigliato: fasano, marmi meccanica analitica, sennò anche goldstein o barletti, frosali meccanica razionale.
    \item ovviamente moodle con dispense varie
    \item Per eserciziario barletto ricci mecc. raz. esercizi o anche lo spiegel.
\end{itemize}
Generale introduzione al corso:\\
Storicamente è un corso che nasce da matematici, in cui ci si interessa alla formalizzazione matematica del moto di sistemi meccanici complessi,
ad esempio un braccio meccanico, per questo si chiama meccanica razionale, ovvero abbiamo un sistema meccanico e lo studiamo con strumenti matematici.\\
Ora si preferisce il termine analitico per via dei modelli che si usano recentemente.\\
Il corso si struttura principalmente in due parti:
\begin{itemize}
    \item meccanica lagrangiana
    \item meccanica hamiltoniana
\end{itemize}
\subsection{Notazioni varie e richiami di fisica 1}
Prendiamo un vettore generico si può esprimere in diversi modi:\\
\begin{tikzpicture}
    \draw[->] (0,0) -- (3,0) node[below] {$x$};
    \draw[->] (0,0) -- (0,3) node[left] {$z$};
    \draw[->] (0,0) -- (-2,-2) node[below] {$y$};
    %versori
    \draw[->,red] (0,0) -- (1,0) node[below] {$\hat{i}$};
    \draw[->,red] (0,0) -- (0,1) node[left] {$\hat{k}$};
    \draw[->,red] (0,0) -- (-0.7,-0.7) node[below] {$\hat{j}$};
    \draw[->,thick] (0,0) -- (2,1) node[above, midway] {$\vec{r}$};
    \node at(2,1)[above] {$A$};
\end{tikzpicture}
\hfill\\
Il vettore che stiamo considerando può essere rappresentato in diversi modi:
\begin{gather*}
    A = \begin{bmatrix}
        x\\
        y\\
        z
    \end{bmatrix}\qquad
    \vv{r} = \begin{bmatrix}
        x\\
        y\\
        z
    \end{bmatrix}\qquad
    \vv{r} = B - A
\end{gather*}
Nell'ultimo modo $B$ è un punto di riferimento, ad esempio l'origine, e $A$ è il punto che stiamo considerando.\\
Ricordiamo la norma di un vettore che è:
\begin{gather*}
    |\vv{r}| = \sqrt{x^2 + y^2 + z^2}
\end{gather*}
Si ricorda anche il prodotto scalare e i versori in generale.\\
Una proprietà che ci interessa del prodotto scalare è la simmetria:
\begin{gather*}
    \left\langle \vv{u} , \vv{v} \right\rangle = \left\langle \vv{v} , \vv{u} \right\rangle
\end{gather*}
E la linearità:
\begin{gather*}
    \left\langle \alpha \vv{u} + \beta \vv{v} , \vv{w} \right\rangle = \alpha \left\langle \vv{u} , \vv{w} \right\rangle + \beta \left\langle \vv{v} , \vv{w} \right\rangle
    \alpha, \beta \in \mathbb{R}\\
    \vv{u} , \vv{v} , \vv{w} \in \mathbb{R}^3
\end{gather*}
Il prodotto vettoriale invece si ricorda essere:
\begin{gather*}
    \vv{w} = \vv{u} \land \vv{v}
\end{gather*}
Con $\vv{w}$ che è un vettore perpendicolare a $\vv{u}$ e $\vv{v}$, e con una norma che è:
\begin{gather*}
    |\vv{w}| = |\vv{u}| |\vv{v}| \sin(\vv{v} \angle \vv{u})
\end{gather*}
Questo invece gode della proprietà di antisimmetria:
\begin{gather*}
    \vv{u} \land \vv{v} = - \vv{v} \land \vv{u}\\
    (\alpha \vv{u} + \beta \vv{v}) \land \vv{w} = \alpha (\vv{u} \land \vv{w}) + \beta (\vv{v} \land \vv{w})
\end{gather*}
Definiamo una terna ortonormale:
\begin{gather*}
    {\widehat{i}, \widehat{j}, \widehat{k}} \text{è una terna ortonormale se:}
    \widehat{i} \land \widehat{j} = \widehat{k} = -\widehat{j} \land \widehat{i}\\
    \widehat{j} \land \widehat{k} = \widehat{i} = -\widehat{k} \land \widehat{j}\\
    \widehat{k} \land \widehat{i} = \widehat{j} = -\widehat{i} \land \widehat{k}
\end{gather*}
Vediamo il prodotto vettoriale tra due vettori generici:
\begin{gather*}
    \vv{u} = u_x \widehat{i} + u_y \widehat{j} + u_z \widehat{k}\\
    \vv{v} = v_x \widehat{i} + v_y \widehat{j} + v_z \widehat{k}\\
    \vv{u} \land \vv{v} = (u_x \widehat{i} + u_y \widehat{j} + u_z \widehat{k}) \land (v_x \widehat{i} + v_y \widehat{j} + v_z \widehat{k})\\
    = (u_x \widehat{i} \land v_x \widehat{i}) + (u_x \widehat{i} \land v_y \widehat{j}) + (u_x \widehat{i} \land v_z \widehat{k}) + (u_y \widehat{j} \land v_x \widehat{i}) + (u_y \widehat{j} \land v_y \widehat{j}) + (u_y \widehat{j} \land v_z \widehat{k}) + (u_z \widehat{k} \land v_x \widehat{i}) + \\(u_z \widehat{k} \land v_y \widehat{j}) + (u_z \widehat{k} \land v_z \widehat{k})\\
    = 0 + u_x v_y (\widehat{i} \land \widehat{j}) + u_x v_z (\widehat{i} \land \widehat{k}) + u_y v_x (\widehat{j} \land \widehat{i}) + 0 + u_y v_z (\widehat{j} \land k) + u_z v_x (\widehat{k} \land i) + u_z v_y (\widehat{k} \land j) + 0\\
    = u_x v_y (\widehat{k}) - u_x v_z (\widehat{j}) - u_y v_x (\widehat{k}) + u_y v_z (\widehat{i}) - u_z v_x (\widehat{j}) - u_z v_y (\widehat{i})
\end{gather*}
Ora definiamo la somma vettoriale:
\begin{gather*}
    \vv{u} + \vv{v} = (u_x + v_x) \widehat{i} + (u_y + v_y) \widehat{j} + (u_z + v_z) \widehat{k}
\end{gather*}
\begin{example}
    Vediamo ora un esercizio in cui abbiamo un'asta di lunghezza $l$ che ha i suoi estremi uno sull'asse $x$ e uno sull'asse $y$, e vogliamo calcolare il centro dell'asta.
    \begin{tikzpicture}
    \draw[->] (0,0) -- (3,0) node[below] {$x$};
    \draw[->] (0,0) -- (0,3) node[left] {$y$};
    \draw[very thick] (0,2) -- (2,0) node[above, midway] {$l$};
    \draw[->, red] (0,0) -- (1,1) node[at end,below] {$C$};
    \draw[->, red] (0,2) -- (1,1) node[at start, above] {$\textcolor{white}{A}$};
    \node at(2,0)[below] {$B$};
    \draw[->] (0,0) -- (0.5,0) node[below] {$\widehat{i}$};
    \draw[->] (0,0) -- (0,0.5) node[left] {$\widehat{j}$};
    \node at(0,0)[below] {$O$};
    \draw (0,1.6) arc (0:-10:2) node[above, midway] {$\alpha$};
    \end{tikzpicture}
    \begin{gather*}
        (C-O) = (A-O) + (C-A)\\
        (A-O) = \widehat{j} l \cos(\alpha)\\
        (C-A) = \widehat{i} l \sin(\alpha)\\
    \end{gather*}
\end{example}
\begin{theorem}[dei seni]
    Da un qualsiasi triangolo del tipo:
    \begin{tikzpicture}
        \draw(0,0) -- (2,0) node[midway, below] {$a$} -- (1,1.5) node[midway, above] {$b$} -- cycle node[midway, above] {$c$};
        %angoli
        \draw(0.3,0) arc (0:58:0.3) node[right] {$\beta$};
        \draw(1.7,0) arc (180:122:0.3) node[left] {$\gamma$};
        \draw(1.2,1.2) arc (-60:-136:0.3) node[below, midway] {$\alpha$};
    \end{tikzpicture}
    \hfil\\
    Si ha che:
    \begin{gather*}
        \frac{a}{\sin(\alpha)} = \frac{b}{\sin(\beta)} = \frac{c}{\sin(\gamma)}
    \end{gather*}
\end{theorem}
\begin{theorem}[di Carnot]
    \begin{gather*}
        |a|^2 = |b|^2 + |c|^2 - 2 |b| |c| \cos(\alpha)
    \end{gather*}
\end{theorem}
\begin{proposition}[Derivazione funzione composta]
    \begin{gather*}
        f(x) : U \subseteq \mathbb{R} \to \mathbb{R}^n\\
        x = x(y) : V \subseteq \mathbb{R} \to \mathbb{R}\\
        y \subseteq V \qquad h(y) \doteq f(x(y))\\
        \frac{dh}{dy} = \frac{df}{dx} \mid_{x(y)}  \cdot \frac{dx}{dy}
    \end{gather*}
    \begin{gather*}
        \int_{y_1}^{y_2} h(y) dy = \int_{y_1}^{y_2} f(x(y)) dy = \int_{x(y_1)}^{x(y_2)} f(x) (\frac{dy}{dx}) dx
    \end{gather*}
\end{proposition}
\begin{example}
    Prendiamo una semicirconferenza centrata nell'origine e esprimiamola come una curva:
    \begin{tikzpicture}
        \draw[->] (-1.5,0) -- (1.5,0) node[below] {$x$};
        \draw[->] (0,0) -- (0,2) node[left] {$y$};
        \draw[thick] (1,0) arc (0:180:1);
        \draw(0,0) -- (0.7,0.7) node[at end, right] {$A$};
        \draw[dashed] (0.7,0) -- (0.7,0.7) node[at start, below]{$1$};
        \draw (0.2,0) arc (0:45:0.2) node[above] {$\theta$};
        \node at(0,0)[below] {$O$};
        %versori
        \draw[->,red] (0,0) -- (0.5,0) node[below] {$\hat{i}$};
        \draw[->,red] (0,0) -- (0,0.5) node[left] {$\hat{j}$};
    \end{tikzpicture}
    \begin{gather*}
        \vv{r} = \cos(\theta) \widehat{i} + \sin(\theta) \widehat{j}\qquad 0 < \theta < \pi \\
        \vv{r} = t \widehat{i} + \sqrt{1-t^2} \widehat{j} \qquad -1 < t < 1
    \end{gather*}
\end{example}
    \noindent
    \begin{proposition}
        Due parametrizzazioni $r_1(t_1)$ e $r_2(t_2)$ sono equivalenti \underbar{se}:\\
        $\exists$ una relazione biunivoca: $t_2(t_1) \quad U_1 \to U_2$ \\
        t.c. $r_2(t_2(t_1)) = r_1(t_1)$
    \end{proposition}
    \begin{proposition}
        Una relazione generica è invertibile se è biunivoca:
        \begin{gather*}
            t_2 = t_2(t_1) \qquad U_1 \to U_2\\
            y = f(x) \to \frac{df}{dx} \overset{<}{>} 0
        \end{gather*}
        allora la mia rel. è invertibile
    \end{proposition}
\end{document}